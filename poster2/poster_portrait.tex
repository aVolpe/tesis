\documentclass[portrait,final,a0paper,fontscale=0.277]{baposter}

\usepackage{calc}
\usepackage{graphicx}
\usepackage{amsmath}
\usepackage{amssymb}
\usepackage{relsize}
\usepackage{multirow}
\usepackage{rotating}
\usepackage{bm}
\usepackage{url}

\usepackage{graphicx}
\usepackage{multicol}

%\usepackage{times}
%\usepackage{helvet}
%\usepackage{bookman}
\usepackage{palatino}
\usepackage[utf8]{inputenc}
\usepackage[spanish]{babel}
\usepackage{booktabs}
\usepackage{tikz}
\usepackage{tkz-kiviat}
\usetikzlibrary{shapes,arrows,shadows}
%\usepackage[style=numeric,sorting=none,backend=biber]{biblatex}
%\addbibresource{../bibliography.bib}


\newcommand{\captionfont}{\footnotesize}

\graphicspath{{images/}{../images/}}
\usetikzlibrary{calc}

\newcommand{\SET}[1]  {\ensuremath{\mathcal{#1}}}
\newcommand{\MAT}[1]  {\ensuremath{\boldsymbol{#1}}}
\newcommand{\VEC}[1]  {\ensuremath{\boldsymbol{#1}}}
\newcommand{\Video}{\SET{V}}
\newcommand{\video}{\VEC{f}}
\newcommand{\track}{x}
\newcommand{\Track}{\SET T}
\newcommand{\LMs}{\SET L}
\newcommand{\lm}{l}
\newcommand{\PosE}{\SET P}
\newcommand{\posE}{\VEC p}
\newcommand{\negE}{\VEC n}
\newcommand{\NegE}{\SET N}
\newcommand{\Occluded}{\SET O}
\newcommand{\occluded}{o}

%%%%%%%%%%%%%%%%%%%%%%%%%%%%%%%%%%%%%%%%%%%%%%%%%%%%%%%%%%%%%%%%%%%%%%%%%%%%%%%%
%%%% Some math symbols used in the text
%%%%%%%%%%%%%%%%%%%%%%%%%%%%%%%%%%%%%%%%%%%%%%%%%%%%%%%%%%%%%%%%%%%%%%%%%%%%%%%%

%%%%%%%%%%%%%%%%%%%%%%%%%%%%%%%%%%%%%%%%%%%%%%%%%%%%%%%%%%%%%%%%%%%%%%%%%%%%%%%%
% Multicol Settings
%%%%%%%%%%%%%%%%%%%%%%%%%%%%%%%%%%%%%%%%%%%%%%%%%%%%%%%%%%%%%%%%%%%%%%%%%%%%%%%%
\setlength{\columnsep}{1.5em}
\setlength{\columnseprule}{0mm}

%%%%%%%%%%%%%%%%%%%%%%%%%%%%%%%%%%%%%%%%%%%%%%%%%%%%%%%%%%%%%%%%%%%%%%%%%%%%%%%%
% Save space in lists. Use this after the opening of the list
%%%%%%%%%%%%%%%%%%%%%%%%%%%%%%%%%%%%%%%%%%%%%%%%%%%%%%%%%%%%%%%%%%%%%%%%%%%%%%%%
\newcommand{\compresslist}{%
\setlength{\itemsep}{1pt}%
\setlength{\parskip}{0pt}%
\setlength{\parsep}{0pt}%
}

%%%%%%%%%%%%%%%%%%%%%%%%%%%%%%%%%%%%%%%%%%%%%%%%%%%%%%%%%%%%%%%%%%%%%%%%%%%%%%
%%% Begin of Document
%%%%%%%%%%%%%%%%%%%%%%%%%%%%%%%%%%%%%%%%%%%%%%%%%%%%%%%%%%%%%%%%%%%%%%%%%%%%%%
\begin{document}

%%%%%%%%%%%%%%%%%%%%%%%%%%%%%%%%%%%%%%%%%%%%%%%%%%%%%%%%%%%%%%%%%%%%%%%%%%%%%%
%%% Here starts the poster
%%%---------------------------------------------------------------------------
%%% Format it to your taste with the options
%%%%%%%%%%%%%%%%%%%%%%%%%%%%%%%%%%%%%%%%%%%%%%%%%%%%%%%%%%%%%%%%%%%%%%%%%%%%%%
% Define some colors

%\definecolor{lightblue}{cmyk}{0.83,0.24,0,0.12}
\definecolor{lightblue}{rgb}{0.145,0.6666,1}

% Draw a video
\newlength{\FSZ}
\newcommand{\drawvideo}[3]{% [0 0.25 0.5 0.75 1 1.25 1.5]
   \noindent\pgfmathsetlength{\FSZ}{\linewidth/#2}
   \begin{tikzpicture}[outer sep=0pt,inner sep=0pt,x=\FSZ,y=\FSZ]
   \draw[color=lightblue!50!black] (0,0) node[outer sep=0pt,inner sep=0pt,text width=\linewidth,minimum height=0] (video) {\noindent#3};
   \path [fill=lightblue!50!black,line width=0pt] 
     (video.north west) rectangle ([yshift=\FSZ] video.north east) 
    \foreach \x in {1,2,...,#2} {
      {[rounded corners=0.6] ($(video.north west)+(-0.7,0.8)+(\x,0)$) rectangle +(0.4,-0.6)}
    }
;
   \path [fill=lightblue!50!black,line width=0pt] 
     ([yshift=-1\FSZ] video.south west) rectangle (video.south east) 
    \foreach \x in {1,2,...,#2} {
      {[rounded corners=0.6] ($(video.south west)+(-0.7,-0.2)+(\x,0)$) rectangle +(0.4,-0.6)}
    }
;
   \foreach \x in {1,...,#1} {
     \draw[color=lightblue!50!black] ([xshift=\x\linewidth/#1] video.north west) -- ([xshift=\x\linewidth/#1] video.south west);
   }
   \foreach \x in {0,#1} {
     \draw[color=lightblue!50!black] ([xshift=\x\linewidth/#1,yshift=1\FSZ] video.north west) -- ([xshift=\x\linewidth/#1,yshift=-1\FSZ] video.south west);
   }
   \end{tikzpicture}
}

\hyphenation{resolution occlusions}
%%
\begin{poster}%
  % Poster Options
  {
  % Show grid to help with alignment
  grid=false,
  % Column spacing
  colspacing=1em,
  % Color style
  bgColorOne=white,
  bgColorTwo=white,
  borderColor=lightblue,
  headerColorOne=black,
  headerColorTwo=lightblue,
  headerFontColor=white,
  boxColorOne=white,
  boxColorTwo=lightblue,
  % Format of textbox
  textborder=roundedleft,
  % Format of text header
  eyecatcher=true,
  headerborder=closed,
  headerheight=0.1\textheight,
%  textfont=\sc, An example of changing the text font
  headershape=roundedright,
  headershade=shadelr,
  headerfont=\Large\bf\textsc, %Sans Serif
  textfont={\setlength{\parindent}{1.5em}},
  boxshade=plain,
%  background=shade-tb,
  background=plain,
  linewidth=2pt
  }
  % Eye Catcher
  {\includegraphics[height=5em]{imagenes/logouna.png}} 
  % Title
  {\bf\textsc{Juegos serios como apoyo a la formación de profesionales: Una aplicación al área de enfermería}\vspace{0.5em}}
  % Authors
  {\textsc{ Mirta González y Arturo Volpe }}
  % University logo
  {% The makebox allows the title to flow into the logo, this is a hack because of the L shaped logo.
    \includegraphics[height=9.0em]{imagenes/logopol.png}
  }

%%%%%%%%%%%%%%%%%%%%%%%%%%%%%%%%%%%%%%%%%%%%%%%%%%%%%%%%%%%%%%%%%%%%%%%%%%%%%%
%%% Now define the boxes that make up the poster
%%%---------------------------------------------------------------------------
%%% Each box has a name and can be placed absolutely or relatively.
%%% The only inconvenience is that you can only specify a relative position 
%%% towards an already declared box. So if you have a box attached to the 
%%% bottom, one to the top and a third one which should be in between, you 
%%% have to specify the top and bottom boxes before you specify the middle 
%%% box.
%%%%%%%%%%%%%%%%%%%%%%%%%%%%%%%%%%%%%%%%%%%%%%%%%%%%%%%%%%%%%%%%%%%%%%%%%%%%%%
    %
    % A coloured circle useful as a bullet with an adjustably strong filling
    \newcommand{\colouredcircle}{%
      \tikz{\useasboundingbox (-0.2em,-0.32em) rectangle(0.2em,0.32em); \draw[draw=black,fill=lightblue,line width=0.03em] (0,0) circle(0.18em);}}

%%%%%%%%%%%%%%%%%%%%%%%%%%%%%%%%%%%%%%%%%%%%%%%%%%%%%%%%%%%%%%%%%%%%%%%%%%%%%%
  \headerbox{Introducción}{name=introduccion,column=0,row=0}{
%%%%%%%%%%%%%%%%%%%%%%%%%%%%%%%%%%%%%%%%%%%%%%%%%%%%%%%%%%%%%%%%%%%%%%%%%%%%%%
  	El rol de las Tecnologías de la Información y la Comunicación (TIC's) en la 		educación es, tradicionalmente, un mecanismo más para transmitir el 
  	conocimiento del maestro al alumno, reemplazando libros y presentaciones\cite{laptop:instructionism}. Actualmente las nuevas corrientes pedagógicas 
  	requieren que el rol de las tecnologías en la educación sea más activo dentro  
  	del proceso de aprendizaje, ayudando incluso al alumno en la construcción del  
  	conocimiento.

	Este trabajo se centra en los juegos serios, que se definen como videojuegos 
	que tienen un objetivo pedagógico específico, sin descuidar los aspectos  
	lúdicos ni la implicación del usuario\cite{sg:aoverview,ludus:sg,abt1987serious}.
   \vspace{0.3em}
 }

%%%%%%%%%%%%%%%%%%%%%%%%%%%%%%%%%%%%%%%%%%%%%%%%%%%%%%%%%%%%%%%%%%%%%%%%%%%%%%
  \headerbox{Objetivo General}{name=objetivo,column=0,below=introduccion}{
%%%%%%%%%%%%%%%%%%%%%%%%%%%%%%%%%%%%%%%%%%%%%%%%%%%%%%%%%%%%%%%%%%%%%%%%%%%%%%
   Identificar y valorar los factores pedagógicos, de diseño, de implementación y 
   de evaluación que influyen a la creación de herramientas educativas que utilizan 
   las corrientes pedagógicas actuales apoyadas en las TIC's. 

   \vspace{0.3em}
  }

%%%%%%%%%%%%%%%%%%%%%%%%%%%%%%%%%%%%%%%%%%%%%%%%%%%%%%%%%%%%%%%%%%%%%%%%%%%%%%
  \headerbox{Problema}{name=problema,column=0,below=objetivo}{
%%%%%%%%%%%%%%%%%%%%%%%%%%%%%%%%%%%%%%%%%%%%%%%%%%%%%%%%%%%%%%%%%%%%%%%%%%%%%%
      Este trabajo se centra en las problemáticas referentes al área de
enfermería proponiendo una solución basada en las TIC's. Se considera como
campo de estudio al Instituto Doctor Andrés Barbero, cuyos estudiantes del último año de la carrera de Licenciatura en Enfermería son tomados como población objetivo.

Uno de los principales inconvenientes de los estudiantes es la poca
disponibilidad de tiempo que poseen\cite{iab:tesis_alumnos}, la difícil 
personalización de la enseñanza en las prácticas\cite{iab:tesis_alumnos} y el nerviosismo ante las primeras prácticas.

Según~\cite{humphreys2013developing} los alumnos de enfermería son estudiantes
divergentes, es decir aprenden a través de experimentación activa, e
interiorizan el conocimiento reflexionando sobre la experiencia. Los juegos
serios son una herramienta ideal para este tipo de
estudiantes\cite{humphreys2013developing}. 
 \vspace{0.3em}
  }
  
%%%%%%%%%%%%%%%%%%%%%%%%%%%%%%%%%%%%%%%%%%%%%%%%%%%%%%%%%%%%%%%%%%%%%%%%%%%%%%
  \headerbox{Referencias}{name=referencias,column=0,below=problema}{
%%%%%%%%%%%%%%%%%%%%%%%%%%%%%%%%%%%%%%%%%%%%%%%%%%%%%%%%%%%%%%%%%%%%%%%%%%%%%%
    \smaller
    %\printbibliography{}
    \bibliographystyle{plain} % Plain referencing style
    \renewcommand{\refname}{}
    \bibliography{../bibliography} % Use the example bibliography file sample.bib
  }
 
%%%%%%%%%%%%%%%%%%%%%%%%%%%%%%%%%%%%%%%%%%%%%%%%%%%%%%%%%%%%%%%%%%%%%%%%%%%%%%
\headerbox{Solución propuesta}{name=solucion,column=1,span=2,row=0}{
  %%%%%%%%%%%%%%%%%%%%%%%%%%%%%%%%%%%%%%%%%%%%%%%%%%%%%%%%%%%%%%%%%%%%%%%%%%%%%%
Se diseña y desarrolla un Juego Serio para dispositivos móviles llamado
\textit{eTesai}, el cual ofrece a los estudiantes de enfermería un medio para
realizar procedimientos de enfermería y cuyo objetivo es servir como 		herramienta de apoyo al aprendizaje. En la figura se observa los componentes 
de la solución, y las herramientas utilizadas para su desarrollo.

\begin{center}
	\includegraphics[scale=0.32]{../resumen/images/full.png}
\end{center}

\begin{multicols}{2}
    \begin{itemize}
    \item \textbf{Front-end:} es una aplicación \textit{Android}, la cual es utilizada por los estudiantes de enfermería para realizar procedimientos 
    de enfermería. Las escenas presentadas corresponden a los siguientes 
    procedimientos:
 
   \textbf{Extracción de muestras de sangre}
  
  	El objetivo principal de este procedimiento es extraer muestras de la sangre 
  	del paciente para su análisis en un laboratorio.
    	\begin{center}
    		\includegraphics[width=4cm]{../solucion/images/hemocultivo_gui.jpg}
    	\end{center}
		
	\textbf{Valoración de la escala de Glasgow}
	
		El objetivo principal de esta escena es diagnosticar el estado de 
		conciencia de un paciente en estado crítico, utilizando la escala 
		de \textit{Glasgow}
		\begin{center}
		\includegraphics[width=4cm]{../solucion/images/glasgow_comandos_voz.jpg}
		\end{center}
		
    \item \textbf{Back-end:} los registros de uso del \textit{front-end} son almacenados bajo demanda en un servidor \textit{back-end}, el cual se encarga de asociar los registros con los alumnos y almacenarlos de manera persistente.
    \end{itemize}
\end{multicols}

 \vspace{-0.6em}
}

%%%%%%%%%%%%%%%%%%%%%%%%%%%%%%%%%%%%%%%%%%%%%%%%%%%%%%%%%%%%%%%%%%%%%%%%%%%%%%
  \headerbox{Resultados obtenidos}{name=resultados,column=1, span=2, row=0,below=solucion}{
%%%%%%%%%%%%%%%%%%%%%%%%%%%%%%%%%%%%%%%%%%%%%%%%%%%%%%%%%%%%%%%%%%%%%%%%%%%%%%
\begin{multicols}{2}

\begin{itemize}
\item \textbf{Aceptación de la solución}

Los niveles de aceptación de la muestra fueron obtenidos a través de 
una encuesta compuesta por 27 preguntas cerradas de tipo Likert y 
4 preguntas abiertas.

\begin{center}

\scriptsize 
\begin{tabular}{lc}
\toprule
Factores        & Promedio encuesta  \\
\midrule
Motivación               & De acuerdo    			\\
Facilidad de exploración & De acuerdo               \\
Sensación de Inmersión   & De acuerdo               \\
Pedagogía                & De acuerdo               \\
Representación           & Parcialmente de acuerdo  \\
Retroalimentación        & Parcialmente de acuerdo  \\
Utilidad                 & De acuerdo               \\
\bottomrule
\end{tabular}

%\tiny
%\begin{tikzpicture}[label distance=.15cm]
%    \tkzKiviatDiagram[scale=.35,%
%                            lattice=9,
%                            %step=10,
%                            ]
%                        {Motivación,
%                         Exploración,
%                        Inmersión,
%                         Pedagogía,
%                         Representación,
%                         Retroalimentación,
%                         Utilidad}
%     \tkzKiviatLine[thick,
%                        color=blue!25!white,
%                        mark=ball,
%                        ball color=blue,
%                        mark size=5pt,
%                        opacity=.2, 
%                        fill=blue!20](6.7,6.8,6.3,6.7,5.3,6.0,6.9)
%     \tkzKiviatGrad[prefix={$0,$},unity=1](1) 
% \end{tikzpicture}
 \end{center}
 
 \item \textbf{Comparación de conocimientos}
	
	Los datos fueron obtenidos en base a un examen realizado a los 
	que formaron parte de la población objetivo, estos exámenes 
	constaban de 5 preguntas sobre extracción de sangre y 5 preguntas 
	sobre la valoración con la escala de Glasgow.	
	
	\begin{center}
		\scriptsize 
		\begin{tabular}{lr}
			\toprule
			Promedio muestra & \textbf{3.82} \\
			Promedio grupo control  & \textbf{3.47} \\
			\midrule
			Promedio total    & \textbf{3.49} \\
			\bottomrule
		\end{tabular}
	\end{center}
 
 
 \item \textbf{Correlaciones}
 
 	Las correlaciones se realizaron con datos obtenidos en exámenes tomados a la muestra después de su interacción con la solución y de los registros de actividades de usuario.
	
	\begin{center}

	\scriptsize 
	\begin{itemize}
		\item \textbf{Puntaje máx juego \textit{Extracción de Sangre} y promedio 
		\textit{Extracción de Sangre} examen}, $0.74$, 
		correlación positiva muy fuerte.
		\item \textbf{Puntaje máx juego \textit{Glasgow} y promedio 
		\textit{Glasgow} examen}, $0.54$, correlación positiva 
		fuerte.
		\item \textbf{Promedio \textit{Glasgow} examen y tiempo jugado 
		\textit{Glasgow}}, $0.86$, correlación positiva muy fuerte.
		\item \textbf{Puntaje máx juego \textit{Extracción de Sangre} y tiempo 
		jugado \textit{Extracción de Sangre}}, $0.30$, correlación positiva 
		moderada.
		\item \textbf{Puntaje máx juego \textit{Glasgow} y tiempo jugado 
		\textit{Glasgow}}, $0.61$, correlación positiva fuerte. 
	\end{itemize}
		
	\end{center}
	
	
\end{itemize}


\end{multicols}
   \vspace{0.3em}
  }
%%%%%%%%%%%%%%%%%%%%%%%%%%%%%%%%%%%%%%%%%%%%%%%%%%%%%%%%%%%%%%%%%%%%%%%%%%%%%%
%\headerbox{Speed}{name=speed,column=2,row=0,below=solucion,bottomaligned=resultado}{
  %%%%%%%%%%%%%%%%%%%%%%%%%%%%%%%%%%%%%%%%%%%%%%%%%%%%%%%%%%%%%%%%%%%%%%%%%%%%%%
%hola
%   \vspace{0.0em}
%  }
%%%%%%%%%%%%%%%%%%%%%%%%%%%%%%%%%%%%%%%%%%%%%%%%%%%%%%%%%%%%%%%%%%%%%%%%%%%%%%
 % \headerbox{Source Code}{name=source,column=2,below=resultados,above=bottom}{
%%%%%%%%%%%%%%%%%%%%%%%%%%%%%%%%%%%%%%%%%%%%%%%%%%%%%%%%%%%%%%%%%%%%%%%%%%%%%%
  %	hola
  %}
%%%%%%%%%%%%%%%%%%%%%%%%%%%%%%%%%%%%%%%%%%%%%%%%%%%%%%%%%%%%%%%%%%%%%%%%%%%%%%
  \headerbox{Conclusiones}{name=questions,column=1,span=2,below=resultados,above=bottom}{
%%%%%%%%%%%%%%%%%%%%%%%%%%%%%%%%%%%%%%%%%%%%%%%%%%%%%%%%%%%%%%%%%%%%%%%%%%%%%%
   La solución es válida como herramienta educativa de apoyo a los 
   estudiantes de enfermería, dándoles la posibilidad de probar sus conocimientos en cualquier 
   lugar y momento. No se considera que se aprenda mejor con respecto otros métodos, sino que, 
   ya sea por motivación, interés, curiosidad o los aspectos lúdicos, puede ayudar en el
   proceso de enseñanza-aprendizaje dándole un rol más activo a las TIC's en este ámbito. 
   \vspace{0.3em}
  }


\end{poster}

\end{document}
