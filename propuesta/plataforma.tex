%! TEX root = ../main.tex
\section{Selección de plataforma}
\label{sec:seleccion_plataforma}

En esta sección se comparan los motores de videojuegos \textit{Unreal Engine, CryEngine, Blender Game Engine, ShiVa3D y Unity3D} según unos criterios de selección que serán citados previamente. Estos criterios están relacionados
con los requisitos que debe cumplir la solución.

Además, se justificara la elección del motor de videojuegos a utilizarse para el desarrollo de la solución propuesta.

\subsection{Criterios de selección}

En el momento de seleccionar la herramienta o motor de juego que se ajuste mejor
a la solución propuesta para facilitar su desarrollo y el cumplimiento de los
requisitos definidos en~\ref{sec:problema_requisitos}, se tuvieron en cuenta los
siguientes criterios:

\observacion{Tincho: Es necesario explicar por que se toma cada una de estas
hipótesis?}

\begin{itemize}
\item Licencia de uso
\item Plataformas móviles.
\item Curva de aprendizaje.
\item Lenguajes de programación para el desarrollo.
\item Tienda de librerías y paquetes.
\item Tamaño de la tienda de librerías y paquetes.
\item Representación en \textit{2D} y \textit{3D}.
\item Formatos de modelo \textit{3D} soportados.
\item Soporte de comunidades.
\item Entorno de desarrollo o \Gls{ide}.
\item Licencia del entorno de desarrollo.
\end{itemize}


\subsection{Comparación}

A continuación se compararan los criterios citados anteriormente para cada uno de los
motores teniendo en cuenta tanto aspectos técnicos como de aprendizaje y de uso.

\newgeometry{bottom=1cm}
%\begin{landscape}


\begin{sidewaystable}
%\begin{table}
\begin{tabulary}{\textwidth}{L|CCCCCCCC}
\toprule
Característica / Motor &
Unreal Engine          &
CryEngine              &
ShiVa3D                &
Unity3D                &
Blender Game Engine \\
\midrule
Distribución & iOS, soporta otros dispositivos en la versión comercial. &
iOs, Android & iOS, Android, Windows Phone, BlackBerry & Android, WindowsPhone,
iOs, BlackBerry & Soporte en desarrollo para Android \\ 

Tienda de librerias & Sí, en estado Alpha & No, existen tiendas de
terceros & Sí & Sí & Sí \\

Tamaño de tienda & Mediana & Mediana & Pequeña & Grande & Grande \\

Comunidad & Grande & Moderada & Moderada & Grande & Grande \\
Licencia del entorno de desarrollo & Gratuita solo para uso no comercial &
Propietaria & Propietaria & Gratuita para uso no comercial & GPL \\

Formatos soportados & fbx, dds, raw, ASE & Formatos propios & dae & FBX, OBJ,
Max, Blend, dae, 3ds, dxf, MB, MA, etc & 3ds, dae, fbx, dxf, etc \\

Licencia del motor & Versiones antiguas gratuitas para uso no comercial.
& Gratuita solo para uso no comercial & Propietaria, solamente la versión Web es
gratis & Versión limitada gratis, disponible para uso no comercial & GPL \\

Curva de aprendizaje & Compleja & Compleja & Compleja & Sencilla & Compleja \\

Lenguajes de desarrollo & Unreal Script, C++ y C++, Lua, FlowGraph
(propietario) y Lua, C++ y \cs{}, UnityScript y Boo \\

\Gls{ide} & Si, propio & Sí, propio & Sí, propio & Sí, propio & Sí, propio \\
\bottomrule

\end{tabulary}
\caption{Comparacion entre motores de videojuegos}
\label{tab:comparacion_motores_juegos}
\end{sidewaystable}
%\end{table}

%\end{landscape}
\restoregeometry

Por todo lo expuesto en~\ref{tab:comparacion_motores_juegos} y de acuerdo a los
requisitos que debe reunir la solución propuesta se elige a \textit{Unity3D} como el
motor de juego ideal para el desarrollo de este trabajo.

Las principales razones de esta elección son la cantidad de plataformas móviles
diferentes a las que se puede exportar y distribuir la aplicación ya que uno de
los ejes del trabajo es brindarle ubicuidad al usuario. Además, Unity posee una
gran comunidad de desarrolladores la cual es importante en los momentos en el
que se poseen dudas que se desean aclarar con rapidez.

Unity posee una gran tienda de activos en donde se pueden encontrar desde
modelos 3D, sonidos hasta librerías que ofrecen extender o incluir
funcionalidades. La versión gratuita de Unity es realmente completa y sobre todo
soporta gran cantidad de formatos de modelos, imágenes y sonidos.
