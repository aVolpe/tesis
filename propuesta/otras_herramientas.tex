\section{Entorno de desarrollo}

La creación de la solución es un proceso complejo, el cual se compone de varios
pasos necesarios, para llevar a cabo el desarrollo completo es necesario recurrir
a una gran cantidad de tecnologías, herramientas, \textit{frameworks}, y
recursos.

En esta sección se describen todos las partes involucradas en el desarrollo de
la solución propuesta, con excepción del motor de simulación, el cual es
descrito en~\ref{sec:seleccion_plataforma}.

La solución se compone de dos partes, la primera es la aplicación que los
usuarios utilizan para realizar las prácticas, denominada solución, y la segunda
parte, es un servidor que se encarga de almacenar la información sobre los
usuarios de la solución y como la utilizan, denominado \textit{backend}.

Primeramente se definen las herramientas de gestión de código, pues ellas son
utilizadas tanto por la solución como por el \textit{backend}, luego se definen
las herramientas específicas de la creación de la simulación y por último, se
citan y describen de manera breve las herramientas utilizadas por el
\textit{backend}.

\subsection{Herramientas de gestión de código}

La gestión del codigo fuente desarrollado como parte de esta tesis se realizo
mediante la utilización de la herramienta de control de código fuente
\textit{Git}\cite{git}, \textit{Git} es de código abierto bajo la licencia
\Gls{gnu}.

El servidor de \textit{Git} utilizado es \textit{BitBucket}\cite{bitbucket}, el
cual es un servidor que almacena repositorios \textit{GIT} de manera gratuita,
la principal carácteristica y motivación por la cual se utiliza este servicio es
que el mismo permite mantener varios repositorios privados de manera
gratuita\cite{bitbucket}, de esta manera, se utilizan distintos repositorios,
específicos para cada parte del presente trabajo, un repositorio para los
documentos de las reuniones y la propuesta de tesis, un repositorio que mantiene
el código fuente del libro de la tesis, un repositorio que mantiene el código
fuente, imágenes y diferentes archivos multimedia de la solución, y finalmente
un repositorio que mantiene el código fuente del servidor que almacena los datos
de los usuario.

\subsection{Desarrollo de la simulación}

El desarrollo de la solución requiere de una variedad importante de tecnologías
aparte del motor, cuya elección fue descrita en~\ref{sec:seleccion_plataforma},
las herramientas descritas en esta sección complementan al motor seleccionado o
facilitan la creación de contenido.

Se utilizo el \textit{Unity Editor} para la creación de las escenas,
\textit{MonoDevelop} y \cs para la parte de programación de la interacción entre
los componentes de la solución.

Adicionalmente se utilizaron varias herramientas de diseño para crear
componentes 3D y 2D, \textit{Make Human} es utilizado para la creación de los
pacientes, \textit{3D Max} permite la creación de objetos como gazas, y otros
elementos utilizados dentro de la solución. En cuanto a los gráficos 2D, se
utilizo \textit{Photoshop}, y diversas páginas web que proveían contenido
gratuíto.

\subsubsection{Unity Editor}

El editor de Unity totalmente integrado y expansible es la interfaz en la se
crean los juegos o simulaciones. Este editor importa todos los activos que se
hayan seleccionado y los organiza dentro de el. Permite compilar las escenas con
los terrenos, luces, audio, personajes, física, entre otros. Se puede agregar
interacción a través de scripting. En este editor se puede probar y editar en
forma simultanea los juegos o simulaciones y desplegarlos en las plataformas
elegidas. Existen cinco vistas en el editor: explorador de proyectos, inspector,
jerarquía, escena y juego. Cada una de estas escenas cuenta con las herramientas
necesarias para permitir un fácil desarrollo del juego\cite{unity3d}.


\subsubsection{MonoDevelop}

MonoDevelop es de código abierto, bajo la licencia \Gls{gnu} y siendo apoyado
principalmente por la comunidad \textit{Mono}. Es el \Gls{ide} utilizado por
defecto en el desarrollo de aplicaciones para \textit{Unity3D}, el mismo soporta
varios lenguajes de programación, como \cs, \textit{UnityScript}, y
\textit{Boo}. Es un \Gls{gnu} multiplataforma que soporta \textit{Windows} al
igual que \textit{Unity3D}.

\subsubsection{\cs}

\cs es un lenguaje de programación orientado a objetos diseñado por
\textit{Microsoft}, es uno de los tres lenguajes disponibles en la plataforma
\textit{Unity3D}, los otros dos lenguajes son \textit{UnityScript} y
\textit{Boo}\cite{unity:script}.

La selección de \cs como lenguaje de programación esta relacionada con la
familiaridad de los autores con lenguajes similares, por ser el lenguaje con más
ayuda en linea, y por que las librerias no diseñadas específicamente para
\textit{Unity3d} pueden ser utilizadas.

Cabe mencionar que la \textit{UnityScript} es normalmnete confundido con
\textit{JavaScript}, lenguaje del que esta inspirado, pero las librerias
disponibles para \textit{JavaScript} no funcionan con \textit{UnityScript}, y la
cantidad de librerias disponibles para \textit{Boo} es muy limitada.

\subsubsection{Herramientas de diseño}

Las herramientas utilizadas para crear modelos 3D son las siguientes:

\begin{itemize}
\item \textbf{MakeHuman}: es un software gratuito y de código abierto bajo la
    licencia \Gls{agnu} para crear personajes humanos 3D. Es una herramienta
    diseñada para simplificar la creación de seres humanos virtuales utilizando
    una interfaz gráfica de usuario\cite{makehuman}. 
\item \textbf{3DS MAX}: es un software privado de modelado 3D completo, que
    además posee herramientas para animación, simulación y renderizacion. Esta
    herramienta fue utilizada para crear objetos 3D que no fueran personajes
    humanos y para exportar modelos de un formato a otro que fuera compatible
    con \textit{Unity3d}\cite{3dsmax}.
\item \textbf{Photoshop}: es una herramienta de edición de gráficos 2D de
    \textit{Adobe}, permite la creación y edición de gráficos, es utilizada para
    la creación de iconos, botones y demás contenido 2D que forma parte de la
    solución.
\end{itemize}

\subsection{Desarrollo del \textit{backend}}

A fin de registrar las actividades del usuario, se necesita de un servidor que
almacene los datos centralizados de todos los usuarios y las actividades que
estos realizan.

Este servidor debe proveer:

\begin{itemize}
    \item \textbf{Alta disponibilidad}: el servidor debe estar disponible en
        todo momento, cualquier día de la semana y a cualquier hora. Los
        requisitos de accesibilidad son estrictos, pues se necesita que los
        usuarios envíen datos sin inconvenientes cuando crean necesario.
    \item \textbf{Accesibilidad}: el servidor debe poder ser accesible desde
        cualquier red móvil.
\end{itemize}

Para el desarrollo de la aplicación web que almacena los datos se utiliza
\Gls{javaee} en su versión 6, la misma se utiliza por la familiarización de los
autores con la tecnología, y la facilidad que provee la misma para la
realización de servicios web que permitan la interacción con la solución.

\Gls{javaee} es un estándar de software empresarial de código abierto cuyo
desarrollo es dirigido por la comunidad\cite{javaee}, la implementación
utilizada es de \textit{RedHat} llamada \textit{JBoss} en su versión 7.1,
\textit{JBoss} es un esfuerzo de la comunidad dirigido por \textit{RedHat}. 

Para los servicios se utiliza la arquitectura \Gls{rest}, la principal
motivación para utilizar \Gls{rest} es eficiencia en el uso de la
red\cite{pautasso2008restful}, el cual es también es la motivación para la
utilización \Gls{json}. 

La implementación del lado del servidor de la arquitectura \Gls{rest} es
\textit{RestEasy}, \textit{RestEasy} es un proyecto de código abierto dirigido
por \textit{RedHat} y que forma parte de la plataforma \textit{JBoss}, de parte
de la solución se utiliza la implementación por defecto de \textit{Unity3D}.

Otro factor determinante es la facilidad de interacción que existe entre Unity3D
y los servicios \Gls{rest}, debido a que \Gls{rest} cumple con el protocolo
HTTP, y en ambos ecosistemas (\textit{JavaEE} y \textit{Unity3d}) existen
implementaciones disponibles y de fácil utilización.

Para el desarrollo del servidor web se utilizo el \Gls{ide} \textit{Eclipse}, el
cual es un proyecto de código abierto dirigido por la \textit{Eclipse
    Fundation}\cite{eclipse}, el mismo provee facilidades para el desarrollo de
servicios web \Gls{rest}.

El almacenamiento permanente de los datos se logra con la utilización de
\textit{PostgreSQL}, el cual es un motor de bases de datos de código abierto
dirigido por una comunidad de desarrolladores llamada \textit{PostgreSQL Global
    Development Group}. La versión elegida es la 9.1.

\subsubsection{OpenShift}

A fin de obtener las características necesarias, de alta disponibilidad y
accesibilidad, se utiliza la herramienta de plataforma como servicio de
\textit{RedHat} llamada \textit{OpenShift}, el cual es un producto de código
abierto dirigido por \textit{RedHat}.

\textit{OpenShift} provee diferentes plataformas, entre las cuales se encuentran
las herramientas seleccionadas \textit{PostgreSQL} 9.1 y \textit{JBoss
    Application Server} 7.1.

Además de dirigir el proyecto, \textit{RedHat} provee un servicio limitado y
gratuito\cite{openshift:pricing}.\footnote{Existen versiones completas del
    producto mantenidas por \textit{RedHat}, las cuales tienen un costo mensual
    y de acuerdo a las funcionalidades utilizadas\cite{openshift:pricing}} Para
esta tesis se utilizo el servicio gratuito con la plataforma \textit{JBoss
    Application Server 7.1} y \textit{PostgreSQL 9.1}.

Otra característica importante de \textit{OpenShift} es la facilidad con la cual
se pueden desplegar nuevas versiones de la aplicación, utiliza un repositorio
\textit{GIT} para mantener el código de la última versión, y cada vez que se
actualiza este repositorio, la aplicación se despliega con la versión
actualizada.

\observacion{Ver si hay que agregar las librerías utilizadas}
% Facebook
% NGUI 3
% Diferentes paquetes de gráficos.
