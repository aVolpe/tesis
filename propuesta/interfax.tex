\section{Interfaz}


\subsection{Inicio}

La solución se inicia con un escenario que esta inspirado en el laboratorio de
enfermería del \Gls{iab}, es la primera experiencia que tiene un
usuario al utilizar la misma, sirve como un menú principal, desde
este punto todas las opciones son accesibles para el usuario, este escenario es
denominado~\emph{Inicio}.

\fixme{La primera vez que}{} se inicia la solución se muestra una pequeña ventana
solicitando el número de teléfono del usuario, se utiliza esta información como un
identificador del usuario para así poder asociar la información del mismo con un
alumno en particular.

\subsubsection{Descripción del entorno}
\label{sec:inicio_descripcion}

\observacion{NO sería más práctico poner imágenes y descripciones breves?}

La escena mostrada como pantalla de inicio de la aplicación muestra como fondo
la sala de un hospital con los elementos típicos de estos lugares, esta es la
que se utiliza como escenografía principal en las escenas de los procedimientos.
Además de este fondo, se muestran varias opciones en forma de botones que serán
descriptas a continuación y un mensaje en donde se recomienda al usuario el uso
de auriculares.

\begin{figure}[H] 
\centering 
\includegraphics[scale=0.2]{propuesta/images/sala.jpg}
\caption{Edificio y decoración inspirados en los laboratorios de enfermería del
    \Gls{iab}, este edificio se utiliza para los diferentes escenarios}
\label{fig:sala_perspectiva}
\end{figure}


\observacion{Resumir}

Se observa en la figura~\ref{fig:sala_perspectiva} la decoración utilizada en
el escenario \emph{Inicio}, mientras se muestran las opciones, se ejecuta una
animación que recorre el escenario mostrando los detalles importantes, como la
camilla, el lector de estadísticas vitales, y demás elementos del escenario.

\subsubsection{Opciones}

\observacion{Agregar imágenes con marcos y luego describirlos, en vez de las
    imágenes actuales con demasiado texto}

Las opciones disponibles en la pantalla de inicio son presentadas en forma de
botones los cuales tienen una breve descripción que identifica la función que
cumplen. 

\begin{itemize}
\item Botón \enquote{Enviar Progreso}: esta función envía toda la información
    acerca de la actividad que el usuario realizó en la aplicación a un servidor
    \emph{backend} que se encarga de almacenar estos datos.
\item Botón \enquote{Salir de la simulación}: esta función permite salir de la
    aplicación.
\item Botón \enquote{Facebook}: esta función permite al usuario ingresar a su
    cuenta de Facebook.
\item Botón \enquote{Extracción de sangre}: esta función permite ingresar a la
    escena correspondiente al procedimiento de extracción de muestras de sangre
    permitiendo al usuario jugar una nueva partida.
\item Botón \enquote{Explorar Glasgow}: esta función permite ingresar a la
    escena correspondiente al procedimiento para explorar las reacción de un
    paciente con un diagnóstico específico de la escala de Glasgow permitiendo
    al usuario jugar una nueva partida, el diagnóstico se selecciona una vez
    presionado este botón a través de la ventana de~\emph{Elegir estado del
        paciente}.
\item Botón \enquote{Evaluar Glasgow}: esta función permite ingresar a la escena
    correspondiente al procedimiento para la valoración y diagnóstico de la
    escala de Glasgow para un paciente con estado aleatorio permitiendo al
    usuario jugar una nueva partida.
\end{itemize}


\subsection{Extracción de muestras de sangre}
\observacion{Agregar pantalla de resultados}

A continuación se detallan cada una de las opciones y formas disponibles de
interactuar con la escena del procedimiento de extracción de muestras de sangre.

Los detalles a continuación toman en cuenta las hipótesis definidas
en~\ref{sec:hemocultivo_hipotesis}. 

\subsubsection{Descripción del entorno}

\observacion{Menos verboso y más imagenes}

Al seleccionar el procedimiento de extracción de sangre en la pantalla de inicio 
la aplicación inmediatamente muestra la escena del procedimiento, se muestra una 
sala de hospital igual a la de la pantalla de inicio pero con un paciente en una 
de las camas, a este paciente es a quien se le realizará el procedimiento.

La posición inicial de la cámara se ubica en un ángulo en donde se puedan ver 
bien los brazos del paciente para facilitar al usuario la realización del 
procedimiento.


\subsubsection{Entidades}

\observacion{Ver la posibilidad de hacer el decoupling entre la descripción
    general (requisitos) y la implementación final}

En la extracción de sangre existen dos entidades principales, el \emph{paciente}
y el \emph{usuario}, cada entidad mantiene un estado independiente de la otra
entidad.

El \emph{paciente} es una entidad con estado complejo, el cual es constantemente
modificado por las acciones del usuario, en resumen, la información que contiene
el paciente es:

\begin{itemize}
    \item \textbf{Jeringas}: un paciente puede tener cero o más jeringas en
        cualquier momento, no se limita la cantidad de jeringas que puede
        insertar el usuario.
    \item \textbf{Manos}: almacena el estado de las manos, el paciente reacciona
        ante peticiones del usuario, puede abrir o cerrar cualquier mano en
        cualquier momento.
    \item \textbf{Torniquetes}: es el conjunto de torniquetes que tiene
        actualmente el paciente, notar que los torniquetes pueden ser colocados
        en cualquier parte del brazo, pero existen lugares \enquote{correctos} y
        lugares \enquote{incorrectos}, la diferencia consiste en la distancia a
        los puntos de extracción, estos lugares están predefinidos.
    \item \textbf{Zonas esterilizadas}: son aquellas áreas del cuerpo que el
        usuario esterilizó, no existe un límite para las zonas esterilizadas.
        Una vez que una jeringa es extraída, una zona esterilizada pasa a estar
        contaminada y a la espera de que el usuario la presione.
    \item \textbf{Zonas presionadas}: son aquellas zonas que, una vez
        contaminadas por la extracción de una jeringa, han sido presionadas por
        el usuario.
    \item \fixme{Contaminado}{?}: define si alguna acción realizada por el
        usuario provocó que el paciente se contamine, existen varias cadenas de
        eventos que pueden hacer que esto ocurra:
        \begin{itemize}
            \item Inyección de una jeringa cuando existe otra inyectada.
            \item Inyección en un lugar inadecuado.
            \item Inyección en un lugar no esterilizado.
            \item Inyección en un brazo cuya mano este abierta.
            \item Inyección fuera del alcance de los torniquetes actuales.
            \item Interacción con el paciente sin que el enfermero tenga la mano
                estéril.
        \end{itemize}
        Es importante notar que este estado no es afectado directamente por una
        acción del usuario, sino por la consecuencia de una acción.
\end{itemize}

\observacion{Ahora mismo el capítulo es un information dump}

El \emph{usuario o enfermero} mantiene un estado en todo momento del cual
dependen sus acciones, por ejemplo, si la mano del enfermero no está
esterilizada, cualquier interacción con el paciente provocará que se
contamine.

\begin{itemize}
    \item \textbf{Manos}: almacena la información acerca de la esterilidad de
        las manos.
    \item \textbf{Guantes, gorro, bata y tapaboca}: almacenan la información
        acerca de los equipamientos que tiene el usuario en un momento dado.
    \item \textbf{Elemento actual}: es el elemento que está activo en
        cualquier momento, un elemento es una herramienta de la vida real,
        como por ejemplo un torniquete, una gaza.
\end{itemize}

\subsubsection{Acciones}

Las acciones que puede realizar el usuario se clasifican en tres, los
\emph{comandos de voz} que simulan una conversación entre el paciente y
enfermero, \emph{las opciones}, que engloban las acciones que puede realizar un
enfermero en cuanto a bioseguridad, y \emph{los elementos} que son las
herramientas que puede utilizar el enfermero durante el procedimiento.

\paragraph{Comando de voz}

Para representar la interacción del usuario con el paciente usando la voz se
implementó un menú que es activado y mostrado en pantalla cuando el usuario
habla, este menú muestra una serie de órdenes que el usuario le diría al
paciente normalmente. Las opciones de menú se detallan a continuación:

\begin{itemize}
    \item \textbf{Explicar procedimiento}: Permite que el usuario explique el
        procedimiento que se va a realizar al paciente. 
\item \textbf{Abrir la mano izquierda}: esta función le indica al paciente que
    abra su mano izquierda, como resultado el paciente realiza esta acción.
\item \textbf{Cerrar la mano izquierda}: esta función le indica al paciente que
    cierre su mano izquierda, como resultado el paciente realiza esta acción.
\item \textbf{Abrir la mano derecha}: esta función le indica al paciente que
    abra su mano derecha, como resultado el paciente realiza esta acción.
\item \textbf{Cerrar la mano derecha}: esta función le indica al paciente que
    cierre su mano derecha, como resultado el paciente realiza esta acción.
\end{itemize}

\begin{figure}[H]
\centering
\includegraphics[scale=0.5]{propuesta/images/hemocultivo_comando_voz.jpg}
\caption{Opciones mostradas al detectar sonido en la escena de extracción
    de sangre.}
\label{fig:hemocultivo_voz_gui}
\end{figure}

En la figura~\ref{fig:hemocultivo_voz_gui} se observan las opciones anteriormente
descritas, este menú aparece inmediatamente después de que el sistema detecte
que el usuario este hablando.

\paragraph{Opciones}

\observacion{Hay que resumir y reorganizar}

Las \emph{Opciones} son aquellas acciones que puede realizar el usuario y afectan
únicamente al paciente. Representan a los aspectos de bioseguridad, es decir,
acciones como lavarse las manos, calzarse guantes, gorro, bata y tapaboca.

Estas opciones afectan al estado de la entidad \emph{enfermero}.

\paragraph{Elementos}

Los \emph{Elementos} representan las herramientas que utiliza un enfermero
durante el procedimiento, un solo elemento puede ser utilizado en cualquier
momento.


\begin{figure}[H]
\centering
\includegraphics[scale=0.5]{propuesta/images/hemocultivo_elementos.jpg}
\caption{Interfaz con los elementos en el paciente, se observa el torniquete (es
    un toro negro), la jeringa, una zona esterilizada (el área cerca del
    torniquete) y un algodón para punzar (Cerca de la muñeca, es una figura
    blanca), todo en el brazo derecho.}
\label{fig:hemocultivo_elementos}
\observacion{Menos verboso}
\end{figure}


Los elementos, que podemos observar en la
figura~\ref{fig:hemocultivo_elementos}, son:

\begin{itemize}
\item \textbf{Torniquete}: es el primer elemento que se debe usar, para utilizarlo
    , el mismo se activa al presionar una parte del brazo del paciente,
    en ese momento, el torniquete aparece en el brazo del paciente, para
    extraerlo, se debe presionar el torniquete y elegir la opción extraer. En la
    figura~\ref{fig:hemocultivo_elementos} se observa al torniquete como un toro
    negro alrededor del brazo izquierdo.

\item \textbf{Esterilizador}: es un elemento que se utiliza para realizar la
    higienización del punto de punción, para utilizarlo se debe presionar
    cualquier parte del brazo del paciente, a continuación aparece una gaza, la
    cual debe ser agitada con un dedo durante un segundo para que se cree una
    zona estéril, la zona estéril creada, es visible a través de una cápsula
     transparente de color verde.

\item \textbf{Jeringa}: Es el elemento utilizado para realizar la extracción, su
    utilización es similar a la del \emph{Torniquete}, solo que además de la
    opción de extracción tiene dos opciones adicionales.

    A través de un menú contextual, se ofrece la posibilidad de realizar un
    acercamiento, como se observa en la
    figura~\ref{fig:hemocultivo_jeringa_zoom}, en la vista ampliada, se puede
    realizar la extracción de sangre utilizando dos dedos, con el primero se
    presiona el tambor y con el segundo dedo se extrae el émbolo\footnote{El
        tambor es la parte de la jeringa que almacena el fluido, mientras que el
        émbolo es la parte que se utiliza para presionar o succionar el fluido}.
    
\item \textbf{Algodón}: El algodón se utiliza para presionar una zona que
    recientemente fue punzada, para utilizar este elemento, basta con presionar
    el brazo del paciente durante un segundo.

\end{itemize}


\begin{figure}
\centering
\includegraphics[scale=0.5]{propuesta/images/hemocultivo_jeringa_ampliada.jpg}
\caption{Vista de la jeringa ampliada, facilitando la extracción de sangre. Se
    agregan flechas azules para facilitar la comprensión del cómo se extrae
    sangre.}
\label{fig:hemocultivo_jeringa_zoom}
\end{figure}



\subsubsection{Reglas para la evaluación durante la ejecución}

%La reglas del procedimiento de extracción de sangre fueron definidas de acuerdo
%a los pasos requeridos según el protocolo del procedimiento y al orden en el que
%son requeridos. Es decir, cada paso del protocolo tiene asociado una regla
%dentro del motor que lo representa y las condiciones asociadas a cada regla
%están determinadas por el orden en que deben realizarse dentro del protocolo.

%Cada regla tiene una o mas condiciones que deben ser cumplidas para que un paso
%del protocolo realizado se considere correcto.

Las reglas definidas dentro de la extracción de sangre definen las acciones
que se deben llevar a cabo para completar el procedimiento, es necesario que
todas las reglas sean cumplidas para obtener un puntaje perfecto.

Cada regla contiene información acerca del estado del progreso del alumno en un
paso en particular, estas reglas pueden tener como dependencias a otras reglas,
es decir, una regla sólo se puede cumplir si una regla anterior se cumple, este
es el caso de las reglas que definen la extracción de un torniquete, la cual
depende de la regla que define la colocación del torniquete.  

Las reglas definen el mecanismo que se utiliza para proveer una retroalimentación
al usuario una vez finalizada la partida, pues la regla almacena información 
acerca del progreso del usuario en cada paso.

A continuación se da una breve descripción de las reglas utilizadas y sus
diversos estados.

\observacion{Si hay orden tienen que enumerar}
\observacion{Quizás haya una manera de visualizar esto, para que se vea el orden
    y las dependencias}
\observacion{Definitivamente hay que visualizar esto y reducir la descripción
    total. En todo caso hacer comentarios sobre cosas demasiado importantes
    después, pero una lista como seta es demasiada sobreiformación y no se
    entiende}


\begin{itemize}
\item \textbf{Explicar procedimiento}: define cuando el usuario ha explicado el
    procedimiento, debe ser la primera regla que se cumple, existiendo dos
    estados que no cumplen la condición, cuando no es la primera regla que se
    cumple y cuando el usuario no realizó la acción.

\item \textbf{Higienización de manos}: define sí las manos del enfemero están
    limpias, existen dos estados para esta regla, cuando se lavó las manos y
    cuando no realizó esta acción. Es importante notar que de esta regla
    dependen varias reglas posteriores, es decir, si esta regla no se cumple,
    las reglas de bioseguridad no pueden ser cumplidas.

\item \textbf{Calzar guantes}: este paso es inmediatamente posterior a la
    higienización de las manos, y como este, varias reglas dependen del momento
    en el que se calcen los guantes. Si el paciente se calza los guantes, la
    única forma de equivocarse en este paso es que las manos estén sucias.

\item \textbf{Gorro, bata y tapaboca}: estos tres pasos son similares, pues
    tienen las mismas dependencias y el orden en el que se cumplan no está
    definido, las posibilidades con este paso son:
    \begin{itemize}
    \item Incompleto: si el usuario no se puso el gorro, bata o tapaboca en
        ningún momento de la práctica.
    \item Manos sucias: si la regla \emph{Higienización de manos} no se cumple
        cuando el usuario se vista con el gorro, bata o tapaboca.
    \item Sin guantes: si la regla \emph{Calzar guantes} no se cumple al momento
        de ponerse el gorro, tapaboca o bata. 
    \end{itemize}
    
\item \textbf{Colocar torniquete}: este paso debe ser realizado una vez que el
    usuario tenga el guante calzado, el torniquete puede ser colocado en
    cualquier parte del brazo, pero depende del lugar de punción de la jeringa.
    Así, existen dos casos donde esta regla falla, cuando no se coloca el
    torniquete en ningún lugar, y cuando el torniquete se coloca en un lugar
    inadecuado. Es importante notar que esta regla se activa cuando se realiza
    la punción de la jeringa.

\item \textbf{Cerrar manos}: debe ocurrir después de que se explique el
    procedimiento y antes de que se inserte la jeringa, además deberá ser la
    mano correspondiente al lugar donde se inyectó la jeringa. Así, existen dos
    formas de no realizar correctamente este paso, que la punción haya ocurrido en otro brazo, o
    que la punción no se realizó. Es importante notar que esta regla no se activa
    cuando el usuario solicita al paciente que cierre su mano, sino cuando la
    jeringa es insertada, esto es así pues es importante el estado de la mano
    cuando el usuario realiza la punción.

\item \textbf{Esterilizar zona}: El usuario puede esterilizar varias zonas, pero
    la única zona que activa esta regla, es aquella donde se realizó la punción.
    Así esta regla tiene dos posibles formas de proveer retroalimentación en
    caso de que no se cumpla, que no se esterilizó ningún lugar del cuerpo, o
    que el lugar esterilizado no sea el lugar de punción.

\item \textbf{Realizar punción}: Existen dos casos de error, si el usuario
    inserta la jeringa en un lugar incorrecto (los lugares correctos se definen
    en~\ref{sec:hemocultivo_hipotesis}), y si el usuario no realiza la punción.

\item \textbf{Retirar Torniquete}: esta regla es dependiente de la regla
    \emph{Colocar Torniquete}, y de qué torniquete se retira. Existiendo dos
    casos de error, el usuario retira un torniquete que no activó la regla
    \emph{Colocar Torniquete} (indicando que este es el torniquete correcto), y
    que no se retire ningún torniquete.

\item \textbf{Abrir mano}: una vez realizada la punción, y retirado el torniquete,
    se debe solicitar al paciente que abra sus manos, así, es dependiente de la
    regla \emph{Realizar punción}, y tiene tres casos de error, que la punción
    no se haya realizado, que la mano no se haya cerrado antes, y que la mano
    que se abra no corresponda con la mano que se cerró.

\item \textbf{Extraer Sangre}: depende únicamente de si la jeringa con la cual
    se extrae sangre es la correcta, así hay dos posibles casos de error, que
    la sangre no se haya extraído, y que la regla \emph{Realizar punción} no se
    cumpla.

\item \textbf{Retirar Jeringa}: se controla que la jeringa que se extraiga sea
    la jeringa que cumplió el paso \emph{Realizar punción}, así, existen dos
    casos de error, que la jeringa no hay sido extraída y que se extrajo sangre
    de una jeringa incorrecta.

\item \textbf{Presionar zona de punción}: debe realizarse inmediatamente después
    de \emph{Retirar Jeringa}, y tiene dos casos de error, que la jeringa no
    haya sido extraída, y que la zona presionada no corresponda con la zona de
    punción.

\item \textbf{Quitar bata, gorro y tapaboca}: depende de la regla \emph{Bata,
        gorro y tapaboca}, y de que la jeringa haya sido extraída, si el
    enfermero se extrae la bata, gorro o tapaboca antes de retirar la jeringa, se
    produce un error. Otro caso para que esta regla no se cumpla es que la regla
    \emph{Bata, gorro y tapaboca} no se cumpla.

\item \textbf{Descalzar guantes}: depende de que las reglas \emph{Calzar
        Guantes} y \emph{Retirar jeringa}, así existen dos posibles casos de
    error, que el usuario no se calce los guantes, o que la jeringa no haya sido
    extraída.

\item \textbf{Limpiar manos}: Como paso final, se debe proceder a realizar una
    higienización de manos, este debe ser el paso final, y tiene como
    prerrequisito que todas las reglas anteriores hayan sido lanzadas.

\end{itemize}

\paragraph{Retroalimentación y puntuación final}
\label{sec:puntuacion_hemocultivo}
\observacion{Más imágenes, menos descripción}

Cada regla tiene asociado un peso, de acuerdo a la dificultad de realizar el
paso, este peso es utilizado al final de la partida para darle una puntuación al
usuario acerca de su rendimiento en la partida.

Además, un regla puede quedar en uno de diferentes estados al final de la
partida como se mostró anteriormente, cada uno de esos estados posee un
significado en el contexto del procedimiento y por lo tanto tienen información
asociada para que al final de la partida se muestre una retroalimentación
correcta al usuario por paso realizado o no.

\subsubsection{Registro de actividad}

\observacion{Resumir explicando por que y para que con una tabla de ejemplo}

Cada acción que realiza el usuario dentro de la simulación provoca un evento, y
estos eventos son registrados de manera transparente para el usuario.

Existen otros tipos de eventos que no son generados por acciones, por ejemplo,
cuando la simulación termina, el motor de reglas lanza un evento por regla,
indicando su estado.

Un subconjunto de todos los eventos, son registrados en un archivo de texto en
formato \Gls{json}, el mismo es posteriormente enviado a un servidor que
almacena la información de todos los usuarios.

Los eventos registrados, son aquellos que involucran a las opciones, elementos,
utilización de la jeringa, torniquete, higienización, y como un caso especial,
todas las reglas también son registradas (independientemente de si son
satisfactorias o no).

La información que se almacena permite reproducir exactamente todo el desarrollo
de la simulación, con excepción de los movimientos de la cámara. De esta manera,
los datos recabados permiten saber en que tareas los usuarios se encontraron con
un mayor número de inconvenientes.

\subsubsection{Interfaz del usuario}

La interfaz principal de este escenario posee dos menús, uno a cada lado de la
pantalla, las opciones son representadas como botones que poseen una imagen
intuitiva que representa la función que realizan. 

\begin{figure}
\centering
\includegraphics[scale=0.5]{propuesta/images/hemocultivo_gui.jpg}
\caption{Vista de la interfaz principal del escenario \emph{Extracción de
        sangre}, con todas las opciones desplegadas.}
\label{fig:hemocultivo_gui}
\end{figure}


En la figura~\ref{fig:hemocultivo_gui} se observa la interfaz, se muestra en el
centro al paciente, y las diferentes opciones que tiene el usuario para
interactuar con el paciente y consigo mismo como enfermero.

Existen dos opciones principales dentro de la interfaz, en la parte superior
derecha se encuentra el menú \emph{Opciones} y en la parte superior izquierda, el menú
\emph{Elementos}.

Al presionar el menú \emph{Opciones} aparecen las distintas
acciones que puede realizar el usuario en cuanto a aspectos de bioseguridad, el
lavado de manos es idempotente, \fixme{es decir no importa cuantas veces se
    presione, el resultado será el mismo, en cambio los demás botones (bata,
    guante, tapaboca y gorro) representan la acción de calzar}{Hace falta tanto
    detalle?}/descalzar, es decir, la primera vez que se presiona la opción
bata, el estado del enfermero cambia a \emph{Con Bata}, si se presiona una
segunda vez, el estado cambia a \emph{Sin Bata}.

En el menú \emph{Elementos} se despliegan opciones que representan a lo elementos que se
utilizan para realizar el procedimiento, una vez presionado un elemento queda
seleccionado (simulando que es la herramienta que el enfermero tiene en la mano
en ese momento), sólo un elemento puede ser seleccionado a la vez. Si el mismo
botón se vuelve a presionar inmediatamente después de haber sido presionado, el
elemento se des-selecciona (simulando que el enfermero dejó la herramienta).

Adicionalmente, existen dos indicadores del estado del paciente, en la parte
inferior derecha, denominados \emph{Indicadores de bioseguridad}, se muestran
iconos representando los elementos que tiene actualmente el enfermero, estos
incluyen los gorros, bata y tapaboca. El elemento que actualmente esta
seleccionado se muestra en la parte superior izquierda de la pantalla, a
diferencia de los indicadores de bioseguridad, se muestra un único icono a la vez.

\observacion{Mucha sobre-información, resumir más y con grafos, imágenes,
    tablas.}

Para la utilización de los elementos, existe un menú contextual\footnote{Un menú
    que se despliega al presionar un elemento, es contextual pues varía de
    acuerdo al elemento seleccionado.}, que lista las opciones disponibles por
elemento, como se observa en la figura~\ref{fig:hemocultivo_torniquete_cm}.

\begin{figure}
\centering
\includegraphics[scale=0.5]{propuesta/images/hemocultivo_contextual.jpg}
\caption{Menú contextual del elemento torniquete.}
\label{fig:hemocultivo_torniquete_cm}
\end{figure}

\subsection{Valoración de la escala de Glasgow}
\observacion{Agregar pantalla de resultados}

La escena  \emph{Valoración de la escala de Glasgow} busca simular el
comportamiento de un paciente con daño cerebral, por ello se presenta en dos
modos distintos, en la primera, el usuario no conoce el estado del paciente,
este modo se conoce como \emph{Evaluar Glasgow}, y en su segundo modo, el
usuario elige el estado del paciente antes de iniciar la escena, esto modo
se conoce como \emph{Exploración Glasgow}. 

La reducida cantidad de diferencias entre ambos modos de la práctica permiten
que ambas sean descritas en esta sección, las características explicadas son
comunes para ambas, salvo que se especifique lo contrario.

En el modo \emph{Exploración Glasgow}, antes de iniciar la práctica, se le
permite al usuario seleccionar el estado del paciente mediante una interfaz, en
cambio, en el modo \emph{Evaluar Glasgow}, el estado del paciente no se conoce
de antemano y será responsabilidad del usuario determinarlo.

La escena se inicia mostrando un paciente acostado en una camilla, de manera
similar a la escena \emph{Extracción de sangre}, la principal diferencia es que
el paciente no se encuentra en ninguna posición en particular, simplemente esta
acostado en la camilla.

Existe una pantalla particular dentro de esta escena, conocida como
\emph{Pantalla de Diagnóstico}, esta pantalla, permite al usuario ingresar su
diagnóstico del paciente, contiene cuatro preguntas básicas, la valoración
motora, verbal, ocular y general de acuerdo a lo definido en~\ref{sec:glasgow} y
a las hipótesis definidas en~\ref{sec:glasgow_hipotesis}.

\emph{Exploración Glasgow}.

\subsubsection{Entidades}

\observacion{Acaso la única entidad no es el paciente?}
\observacion{Problema de formato?}
\observacion{Hay que separar la descripción de la simulación de la
    implementación}

Al igual que en \emph{Extracción de sangre}, existen dos entidades principales,
pero en esta escena, el enfermero no almacena información alguna, y el
paciente sólo almacena su estado, que se define al inicio. De esta forma, las
entidades no se modifican en ningún momento.

La información almacenada por la entidad paciente es su estado motor, verbal y
ocular, el cual es un conjunto de números, cuyos posibles valores se definen en
en~\ref{sec:glasgow_protocolo}, la definición de estos números varían de acuerdo
al tipo de la escena:

\begin{itemize}
    \item \textbf{Exploración}: la escena de exploración se inicia con la
        \emph{Pantalla de diagnóstico}, donde el usuario selecciona el estado
        del paciente que desea, este estado se mantendrá constante durante toda
        la escena.
    \item \textbf{Evaluación}: al inicio se crean tres números de manera
        aleatoria, el algoritmo que crea estos valores, lo hace de tal manera
        que el estado del paciente es consistente, por ejemplo, el paciente
        nunca tendrá un estado verbal \enquote{Orientado} (valor 5 en la escala)
        y un estado ocular \enquote{Ausente} (valor 1 en la escala), pues esto
        no tendría sentido, si no puede abrir los ojos (estado
        \enquote{ausente}), no puede saber donde esta (estado
        \enquote{orientado}).
\end{itemize}

Aunque el estado de las entidades no se modifique, esto no significa que no
puedan realizar acciones entre ellas, sino que estas acciones y los eventos
generados no alteran el estado de las entidades.

\subsubsection{Acciones} 

Las acciones se clasifican en dos tipos, los \enquote{comandos de voz} y las
\enquote{opciones a través de menú contextual}. 

\paragraph{Menú Contextual}

Las opciones a través del menú contextual se relacionan a acciones que puede
realizar el enfermero sobre una parte particular del cuerpo del paciente, en las
extremidades, el menú despliega una sola opción, la cual es \emph{Pinchar}, que
provoca que el enfermero realice un estímulo doloroso al paciente, el
paciente reacciona ante este estímulo dependiendo de su valoración motora y
ocular. 

\begin{itemize}
    \item Si el estado ocular del paciente es \enquote{Al dolor}, el paciente
        abrirá los ojos inmediatamente después de que se presione la opción. 
    \item  La respuesta motora varía de acuerdo a su estado, si el mismo es
        \enquote{Localiza}, el paciente mueve sus manos hasta el origen del
        dolor, si el estado es \enquote{Retira}, moverá la extremidad que
        sufrió el estímulo lejos de su posición inicial, si es
        \enquote{Flexión anormal}, el paciente reaccionará comprimiendo el
        cuerpo, indistintamente de la ubicación del estímulo doloroso, y si el
        estado es \enquote{Extiende}, el paciente extenderá el cuerpo.
\end{itemize}

\paragraph{Comandos de voz}

Las opciones disponibles a través de los comandos de voz, simulan una
interacción verbal entre el enfermero y el paciente, y se agrupan en tres
tipos, verbales, oculares y motoras.

Las preguntas y posibles respuestas de tipo \emph{verbal}, se pueden ver en la
tabla~\ref{tab:glasgow_opciones_respuesta}. 

\begin{table}[H]
\centering
\begin{tabulary}{1.2\textwidth}{LCCCCC}
\toprule
\textbf{Pregunta} & \textbf{Orientado} & \textbf{Confusa} & \textbf{Palabras
    inapropiadas} & \textbf{Palabras incomprensibles} & \textbf{Ausente} \\
\midrule
¿Qué día es? & El día de la semana actual & Cualquier día de la semana menos el
correcto & La respuesta a otra pregunta en estado orientado & Gritos, gruñidos y
quejidos & No emite sonido \\
¿Cuál es su nombre? & \emph{Carlos Benitez} & Respuesta coherente sin mencionar
su nombre & Respuesta a otra pregunta en estado orientado & Gritos, gruñidos y
quejidos & No emite sonido \\
¿Donde se encuentra? & \emph{En una cama de hospital} & \emph{En mi dormitorio} &
Respuesta a otra pregunta en estado orientado & Gritos, gruñidos y quejidos & No
emite sonido \\
\bottomrule
\end{tabulary}
\caption{Posibles respuestas de acuerdo al estado verbal del paciente.}
\label{tab:glasgow_opciones_respuesta}
\end{table}

Las opciones del menú por comandos de voz del tipo \emph{motor}, son cuatro: 

\begin{itemize*}
    \item Mueva el brazo
    \item Mueva la pierna
    \item Mueva la mano
    \item Mueva la cabeza
\end{itemize*}

A diferencia de las opciones verbales, estas no tienen una respuesta sonora, en
cambio, si el estado motor es \enquote{Obedece},el paciente reacciona moviendo
una extremidad, en caso contrario, el paciente no realiza acción alguna.

Finalmente, existe una opción \emph{ocular}, la cual es \enquote{Abra sus ojos por
favor}, la cual no tiene una respuesta sonora, y sólo si el paciente tiene un
estado ocular \enquote{Al hablar} abre los ojos, en caso contrario, no realiza
acción alguna.

\subsubsection{Pantalla de diagnóstico}

Una vez que el usuario decida que está listo para dar un diagnóstico, procede a
finalizar la escena, en ese momento se presenta una pantalla donde el mismo
puede diagnosticar al paciente, como se observa en la
figura~\ref{fig:glasgow_gui_resultados}.

Las opciones presentadas al usuario son cuatro, puntuación verbal, ocular,
motora y un diagnóstico del estado de conciencia del paciente. Estos valores
se describen en~\ref{sec:glasgow_protocolo}.

\begin{figure}[H]
\centering
\includegraphics[scale=0.5]{propuesta/images/glasgow_diagnostico.jpg}
\caption{Vista de la \emph{Pantalla de diagnóstico}, donde el usuario puede
    asignar una puntuación a cada aspecto analizado del paciente.}
\label{fig:glasgow_gui_resultados}
\end{figure}

En el modo \emph{Evaluación Glasgow}, esta misma pantalla se utiliza la inicio
de la escena para que el usuario pueda seleccionar el estado deseado del
paciente.

La única diferencia que existe entre la \emph{Pantalla de diagnóstico} entre la
exploración y la evaluación, es que en la evaluación, los posibles valores para
cada aspecto a diagnosticar van de $0$ a $8$, y en la exploración, son los
valores mínimos y máximos válidos\footnote{Los valores máximos se definen
    en~\ref{sec:glasgow_protocolo} y son $4$ (ocular), $5$ (verbal) y $6$
    (motora)}.

\subsubsection{Valoración}
\label{sec:puntuacion_glasgow}

Para la evaluación del rendimiento del usuario al momento de llevar a cabo el
procedimiento de valoración de la escala de Glasgow se tuvo un enfoque
\fixme{completamente diferente al del procedimiento de extracción de sangre debido a la
    naturaleza propia del procedimiento}{Concéntrense en lo que tienen que
    explicar sin comparar con otros casos por que sino el párrafo se vuelve
    largo y pierde foco}.

Como se explicó anteriormente, el paciente puede estar en ciertos estados
específicos dentro de la escala, y además dentro de cada estado reacciona de una
forma en particular, por lo tanto, al inicio de la partida un componente interno
de la aplicación selecciona de forma aleatoria un estado para el paciente, de
forma tal que cada vez que una partida sea jugada no se repitan los estados de
forma seguida.

El estado aleatorio del paciente es guardado en una variable que no es
modificada hasta que se reinicie la partida. Al final de la partida, la
aplicación pide al usuario que valore el estado del paciente que le fue
presentado, una vez que el usuario confirme su respuesta la aplicación la
compara con el estado guardado y de esta forma puede informar al usuario acerca
de su rendimiento en el diagnóstico.

Además, cada posible respuesta dada por el usuario contiene información
relacionada al contexto del procedimiento y a la situación actual presentada, la
cual, es utilizada como retroalimentación al final de la partida. La puntuación
final depende de la cantidad de valoraciones correctas dadas por el usuario
para la respuesta verbal, motora, ocular y nivel de gravedad del paciente.

\subsubsection{Registro de actividad}

\fixme{Al igual que en la escena de \emph{Extracción de sangre}, las acciones
    del usuario son registradas en un archivo con formato \Gls{json} y enviadas
    a un servidor que almacena la información acerca de todos los usuarios.
}{Concéntrense en lo que tienen que explicar sin comparar con otros casos por
    que sino el párrafo se vuelve largo y pierde foco}

Se almacena además, el diagnóstico del usuario, la puntuación del mismo y el
estado inicial del paciente, así como el modo de la escena (evaluación o
exploración).

\subsubsection{Interfaz de usuario}
\observacion{Queda medio flotando esto}

La interfaz de usuario, como se observa en la figura~\ref{fig:glasgow_gui}, es
muy sencilla, se compone de solo una opción permanente, la cual permite al
usuario finalizar la escena y mostrar la \emph{Pantalla de diagnóstico}. Se
observan además, las opciones que se despliegan cuando la solución detecta que el
usuario emite palabras, conocidas como \emph{Comandos de Voz}, que permiten al
mismo interactuar con el paciente.

\begin{figure}[H]
\centering
\includegraphics[scale=0.5]{propuesta/images/glasgow_comandos_voz.jpg}
\caption{Interfaz de la escena \emph{Evaluación de Glasgow}, se observan los
    \emph{comandos de voz}, así como la opción que permite finalizar la escena
    (esquina inferior derecha).}
\label{fig:glasgow_gui}
\end{figure}

Además se ve en la interfaz, al paciente, que es el foco principal de la cámara.

\subsection{Pantalla de resultados}

Al finalizar ambas escenas, se presenta una pantalla de resultados, la cual es
la encargada de mostrar toda la información que fue recabada durante la escena,
esta información incluye los pasos correctos e incorrectos que realizó el
usuario.

\begin{figure}[H]
\centering
\includegraphics[scale=0.5]{propuesta/images/resultado_glasgow.jpg}
\caption{Pantalla de resultados mostrando los pasos correctos e incorrectos, en
    la escena \emph{Glasgow}.}
\label{fig:resultados_glasgow}
\end{figure}

Se observa en la figura~\ref{fig:resultados_glasgow} el diseño de la
pantalla, el título es la escena actual, en el pie se observa un resumen de la
puntuación, y el tiempo que duró la escena.

Se lista de manera ordenada los pasos en la parte izquierda de la pantalla, y en
la parte derecha se muestra información relevante acerca del motivo por el cual
no se cumplió cada paso o es caso contrario, se indica que el pao fue realizado 
correctamente.

Adicionalmente a la información de la sesión, se permite al usuario reiniciar la
escena, ir al menú, y compartir sus logros en las redes sociales.

\subsubsection{Retroalimentación}
\observacion{Cada retroalimentación debería ir con cada escena}
\observacion{Mover junto a su escena}

La retroalimentación se logra a través de la pantalla de resultados, en ella se
presenta información detalla de lo que realizó el usuario. 

Si el usuario realizó de manera incorrecta un paso del procedimiento, esta
información está contenida en una regla (\emph{Extracción de Sangre}), o en el
estado inicial del paciente (\emph{Evaluación de Glasgow}).

\begin{figure}[H]
\centering
\includegraphics[scale=0.5]{propuesta/images/resultado_hemocultivo.jpg}
\caption{Pantalla de resultados mostrando los pasos correctos e incorrectos en
    la escena \emph{Extracción de sangre}.}
\label{fig:resultados_hemocultivo}
\end{figure}

Como se observa en la figura~\ref{fig:resultados_hemocultivo}, el paso
\enquote{Poner Jeringa}, contiene la siguiente información \enquote{Puesta en un
    lugar incorrecto}, esto le indica al usuario, que el lugar donde la jeringa
fue insertada era incorrecto, otro posible valor de retroalimentación es
\emph{Jeringa no insertada}.

\subsubsection{Gamificación}
\observacion{Esto se podría expandir con laguna imagen}

En esta pantalla además se observan opciones que son estudiadas por la
\emph{Gamificación}, entre ellas observamos la puntuación total, el tiempo
utilizado y la utilización de redes sociales.

Si el usuario presiona el botón \enquote{Facebook}, se despliega el menú de
dicha red social, permitiendo que el mismo pueda agregar un mensaje
personalizado y el resultado de la sesión se comparta con el texto \enquote{Conseguí 15 puntos
    en 1475 segundos, en la \emph{Escena Extracción de sangre} jugando con
    YAVE}.

