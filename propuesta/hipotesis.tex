\section{Hipótesis de la simulación}
\label{sec:hipotesis}

Las escenas seleccionadas y definidas en~\ref{sec:seleccion_escenas} representan
las acciones que deben realizar los profesionales de enfermería a la hora de
realizar los procedimientos seleccionados, por limitaciones técnicas,
tecnológicas y de tiempo, no es posible realizar una simulación de todos los
pasos requeridos.

Los factores que influyen en que partes se simularán, que partes estarán
presentes solamente a través de opciones y que partes se omitirán son:

\begin{itemize}

    \item \textbf{Limitaciones técnicas}: acciones como la simulación del agua
        (necesarios para el lavado de manos), requieren de requisitos de
        hardware avanzados y un tiempo considerable de desarrollo. Las acciones
        que escapan al alcance del hardware y tiempo de los desarrolladores no
        son simuladas.

    \item \textbf{Importancia}: no todos los pasos definidos en el procedimiento
        oficial son necesarios, por ejemplo, la colocación de los elementos
        cerca del lugar de trabajo, es un paso necesario, pero es considerado un
        paso necesario y trivial.

        La importancia es evaluada por profesionales del \Gls{iab}, los cuales
        dieron su opinión acerca de cada aspecto simulado, el mismo es tenido en
        cuenta para determinar la importancia de cada acción.

    \item \textbf{Facilidad de realización en la vida real o en el laboratorio}:
        ciertos pasos son triviales en la vida real pero requieren un esfuerzo
        significativo para ser simuladas, como por ejemplo el lavado de manos es
        un procedimiento al que los alumnos están acostumbrados.

        La facilidad que tienen los alumnos con las acciones fue determinada por
        profesores del \Gls{iab}, determinaron que acciones son triviales para
        los alumnos y cuales presentan mayores dificultades en su vida
        profesional.

    \item \textbf{Familiaridad de los alumnos con la acción}: existen
        actividades que son recurrentes en la vida de un estudiante de
        enfermería, actividades que son realizadas en la mayoría de las
        materias prácticas, si los alumnos están familiarizados.

\end{itemize}

Estas hipótesis sirven para acotar el alcance de la simulación, definen qué se
simulará y cual es del detalle necesario para alcanzar las competencias básicas.

Existen hipótesis que son globales para toda la simulación, las mismas son:

\begin{itemize}

    \item \textbf{Comandos de voz con interfaz}: para enviar una petición al
        paciente (por ejemplo, preguntarle su nombre), no es necesario
        identificar las palabras del usuario, sino más bien detectar que ha
        hablado y listar las posibles preguntas que puede realizar.

    \item \textbf{Utilización de la interfaz}: para realizar una acción con los
        elementos, es suficiente con presionar el mismo y seleccionar una acción
        de una lista de opciones, no hace falta emular todas las posibles.

    \item \textbf{Acciones de bioseguridad}:\todox{Definir bioseguridad} Las
        acciones de bioseguridad, se realizan a través de una opción en la
        interfaz gráfica.

\end{itemize}

Otras hipótesis, son tomadas por escena, las dos escenas simuladas son muy
diferentes en el modo de interacción del usuario con su entorno, por ejemplo, en
la escena de extracción de sangre, el usuario interactúa con el paciente a
través de objetos, en la evaluación de Glasgow, la interacción con el paciente
es directa.

\subsection{Extracción de sangre}

Se presentan los pasos mostrados en la sección~\ref{sec:hemocultivo_protocolo},
y adicionalmente se establecen las hipótesis punto por punto y las
consideraciones que deben ser tomadas.

\begin{itemize}

\item \textbf{Preparar el equipo}: la preparación del equipo es un aspecto muy
    importante del procedimiento, pero no es un punto único de la extracción de
    sangre, además las prácticas de los alumnos cubren completamente este paso
    según comentarios de los profesores. \emph{Este paso no se simula}.

\item \textbf{Explicación al paciente del procedimiento a realizar}: es un
    aspecto importante del procedimiento, pero la simulación de una conversación
    alumno-paciente es compleja, según comentarios de los profesores, es
    suficiente con que los alumnos sepan que lo deben realizar y en que moento,
    no es necesario simular la conversación en sí. \emph{Este paso se simula a
        través de un comando de voz con la interfaz}.

\item \textbf{Asepsia de las manos}: este paso forma parte de un área más amplia
    conocida como bioseguridad, la cual es un aspecto transversal a todos los
    procedimientos realizados por los enfermeros. 
    La implementación de una simulación del lavado de mano es compleja, y es un
    aspecto que, al igual que la preparación del equipo, está cubierta por los
    laboratorios, aún así, es necesario que los alumnos sepan en que momento
    deben realizar la asepsia de sus manos. \emph{Se simula este paso a través de una
        opción en la interfaz}, no se simulan los pormenores del lavado de manos.

\item \textbf{Llevar el equipo a la unidad en donde se encuentra el paciente}:
    este es un paso trivial que deben realizar los profesionales, la simulación
    de este proceso no es importante según comentarios de los profesores. Este
    proceso no tiene importancia según los profesionales del \Gls{iab}, \emph{este
        paso no se simula}.

\item \textbf{Vestirse con bata estéril, tapaboca y gorro}: al igual que la
    asepsia de las manos, es importante que los alumnos sepan que lo deben
    hacer, pero no es importante que se simule como lo hacen. 

    Los estudiantes están familiarizados con estas acciones, \emph{se simula el
        momento y el orden en el que el jugador lo hace} a través de una opción
    en la interfaz, no se simula el proceso en sí.

\item \textbf{Calzarse los guantes}: es un paso relacionado a la bioseguridad,
    es importante que se sepa en que momento debe realizarse, pero no es
    necesario simular el proceso. 
    \emph{Se simula el momento y el orden en el que se realiza}, no se simulan
    los pormenores de la acción.

\item \textbf{Ubicar al paciente en posición adecuada}: la ubicación del
    paciente durante la extracción de sangre es un factor determinante para que
    la extracción pueda ser realizada correctamente.

    Los alumnos están familiarizados con este proceso según opinion de los
    profesionales, \emph{este paso no se simula}, el paciente está en la
    posición adecuada al inicio de la simulación.

\item \textbf{Elegir la zona a puncionar}: existen varias partes del antebrazo
    donde se puede proceder a realizar una inyección, el conocimiento de las
    mismas, y el procedimiento para detectarlas, es un factor importante del
    proceso.
    
    Las venas del cuerpo humano se detectan palpando los antebrazos, y sintiendo
    el pulso del paciente, existen dos áreas donde el pulso es suficientemente
    fuerte como para sel palpado, estos puntos y el pulso del paciente deben ser
    detectables por el jugador.

    \emph{Los puntos donde se debe punzar deben son identificables en la
        simulación}, a través de una palpación a los brazos del paciente. 

\item \textbf{Colocación del torniquete}: La ubicación y el momento de la
    colocación del torniquete es de vital importancia para el procedimiento, el
    mecanismo utilizado para colocarlo no es relevante, pues el mismo es
    trivial.

    \emph{El hecho de colocar el torniquete es simulado}, el mecanismo para
    hacerlo no es importante.

\item \textbf{Solicitar al paciente que cierre el puño}: El momento en el cual
    se solicita al paciente que cierre la mano es vital para que el
    procedimiento de extracción sea satisfactorio.

    \emph{Este paso es simulado} a través de un comando de voz.

\item \textbf{Esterilizar la zona de punción}: la esterilización de la zona de
    punción es un factor de suma importancia para el procedimiento, así como el
    momento en el que se realiza, el jugador debería poder esterilizar la zona
    antes de insertar la jeringa.
    
\item \textbf{Extraer el protector de la aguja}: La extracción del protector de
    la aguja es un paso necesario, pero trivial, el hecho de retirar el
    protector de la jeringa no es un paso necesario para el logro de las
    competencias básicas necesarias.

\item \textbf{Puncionar la piel con la aguja}: este es un paso central en el
    procedimiento, en el se deben tener en cuenta aspectos como la posición
    donde se realiza la punción, y el angulo con el que ingresa la aguja.

    La posición donde se realiza la punción es importante por que depende de la
    ubicación donde se colocó el torniquete, y debe ser en uno de los puntos del
    brazo donde existen venas capaces de soportar el procedimiento.

    En cuanto al ángulo de punción, es un conocimiento importante que deben
    tener los alumnos, el conocimiento es teórico y según comentarios de los
    profesores, es un tema en el cual los alumnos tienen suficiente práctica en
    el laboratorio.

\item \textbf{Tensar la zona de punción}: el proceso de tensar la zona de
    punción se realiza momentos durante la inserción de la jeringa, el mismo es
    trivial, y para simularlo se requiere que el usuario utilice tres dedos al
    mismo tiempo (dos para tensar y otro para realizar la punción), lo cual
    dificulta la utilización de la solución.

    Este paso no es simulado por la dificultad técnica que implica utilizar tres
    dedos para realizar una tarea que no es central para el procedimiento.

\item \textbf{Remover el torniquete}: Al igual que en la colocación del
    torniquete, el momento de la extracción es importante, pero no la forma de
    realizarlo.

\item \textbf{Solicitar la apertura del puño}: el momento exacto donde se debe
    solicitar al paciente que abra la mano es fundamental para la realización
    correcta de la simulación.

\item \textbf{Extraer la muestra se sangre necesaria}: este es el paso central
    de la práctica, tanto el momento, como la forma es importante simular.

\item \textbf{Presionar el brazo y extraer la aguja}: la presión del brazo para
    extraer la jeringa es un paso trivial, en cambio el momento en el que se
    extrae la aguja es conocimiento necesario para el procedimiento.

\item \textbf{Colocar algodón con alcohol en el punto de punción}: este paso es
    importante, tanto el momento en el cual se debe realizar, como la forma de
    realizarlo.

\item \textbf{Sellar la muestra y enviarlo a su destinatario}: es necesario que
    los alumnos sepan que este paso debe ser realizado, pero los detalles del
    mismo, no son necesarios para el logro de las competencias básicas, este
    paso no se simula.

\item \textbf{Retirar la bata, tapaboca, gorro y guantes}: es necesario que los
    alumnos sepan que deben desechar todos los elementos que fueron utilizados
    durante el proceso, la forma de hacerlo no es necesaria.

\item \textbf{Asepsia de las manos}: la asepsia final de las manos es un paso
    necesario para el procedimiento, así como la asepsia inicial, es importante
    que los alumnos sepan el momento en cual deben realizarlo.

\end{itemize}


\subsection{Evaluación de glasgow}
