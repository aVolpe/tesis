%! TEX root = ../main.tex
\section{Tecnologías disponibles}

Para el desarrollo de videojuegos se utilizan programas o herramientas especializadas en ello llamadas motores de videojuegos. A continuación se da una breve introducción de lo que es un motor de videojuego o motor gráfico.

El término “motor gráfico” o “motor de videojuegos” hace referencia a una serie de rutinas de programación que permiten el diseño, la creación, el desarrollo y la representación gráfica de un videojuego\cite{videojuego:telechea}.

Además, la gran mayoría de estos motores ofrecen a su vez características y funciones que facilitan la construcción del videojuego, como el motor físico (software capaz de realizar simulaciones de ciertos sistemas físicos como la dinámica de un cuerpo rígido, el movimiento de un fluido o la elasticidad) o detector de colisiones, sonidos, scripting, animaciones, inteligencia artificial, redes, streaming, administración de memoria, etc\cite{videojuego:telechea}.

El motor de juego utilizado depende de las características que posea el videojuego que se quiere desarrollar. Por lo mismo, a continuación se da una breve descripción de los motores mas utilizados actualmente para que posteriormente se puede seleccionar el mas adecuado.

\subsection{UDK (Unreal Development Kit)}

Unreal Engine\cite{unrealengine} es el motor de juegos desarrollado por Epic Games, se ofrece bajo un plan de suscripción de 19 dolares por mes. El servicio de suscripción permite a los desarrolladores unirse a una comunidad dedicada a la construcción de grandes juegos y evolución de Unreal Engine. La suscripción se puede cancelar en cualquier momento.

UnReal Engine 4 permite desarrollar juegos para plataformas como Windows PC, Mac, iOS y Android. También es compatible con Xbox One y PlayStation 4. Existen además trabajos recientes sobre otras plataformas como HTLM5 y Linux.

Sin embargo existe una versión gratuita de UnrealEngine, el Unreal Development Kit o UDK. El UDK es la edición gratuita de Unreal Engine 3 que proporciona acceso al galardonado motor de juegos 3D y herramienta profesional que se utiliza en el desarrollo de videojuegos blockbuster, visualización arquitectónica, el desarrollo de juegos para móviles, modelos 3D, películas digitales y más. Utilizando UDK se pueden implementar juegos y aplicaciones en Windows PC, iOS y Mac.


\subsection{Blender Game Engine}

Blender Game Engine\cite{blender} es el motor de juego de Blender Foundation que permite crear aplicaciones 3D interactivas o simulaciones. La principal diferencia entre el Blender Game Engine y el Blender convencional está en el proceso de renderizado. En el motor Blender normal las imágenes y animaciones se construyen fuera de linea es decir, una vez generadas no pueden ser modificadas. Por el contrario Blender Game Engine genera las escenas de forma continua en tiempo real e incorpora facilidades para la interacción del usuario durante el proceso de renderización.

Procesa la lógica de sonido, de la física y la representación de simulaciones en orden secuencial. El motor esta
escrito en C++. Posee un editor que proporciona una profunda interacción con la simulación, su funcionalidad se
puede ampliar con scripts Python y esta diseñado para abstraer las características complejas del motor en una interfaz de usuario simple, que no requiere programación.

El motor de juego puede simular contenido dentro de Blender, sin embargo, también incluye la posibilidad de exportar en plataformas como Windows, Linux y Mac OS. También hay soporte básico para plataformas móviles con el proyecto Android Blender Player GSOC 2012.

\subsection{CryEngine 3}

CryEngine 3\cite{cryengine} es el motor de juegos desarrollado por Crytek. CryEngine es un motor avanzado para el desarrollo de juegos, películas, simulaciones de alta calidad y aplicaciones interactivas. Es uno de los procesadores de alta gama mas rápido en el mundo, con características diseñadas específicamente para Windows PC, PlayStation 3 y Xbox 360.

Existe una versión gratuita de CryEngine con todas las funcionalidades. CryEngine es  WYSIWYG (What You See Is What You Get). 

CryEngine posee un conjunto de herramientas utilizadas para el análisis de rendimiento. La ultima versión
de CryEngine, CryEngine 3 es el único motor con la última tecnología en iluminación, física e inteligencia
artificial si se está desarrollando en Windows PC, PlayStation 3 y Xbox 360.


\subsection{ShiVa3D}

ShiVa3D\cite{shiva} es un paquete para el desarrollo de juegos y aplicaciones 3D, viene en un formato fácil de usar, pero con un editor WYSIWYG (What You See Is What You Get) muy potente.

ShiVa puede exportar juegos y aplicaciones para más de 20 plataformas de destino, incluyendo móviles como iOS, Android, BlackBerry y Windows Phone, de escritorio como Windows, Mac OS X y Linux, los navegadores web con soporte Flash y HTML5, así como Consolas como la Xbox 360, PlayStation3 y Nintendo Wii. La herramienta de creación se ejecuta en Windows y Mac OS X. Software y hardware Apple son requeridos con el fin de construir juegos para OS X y iOS.

ShiVa también ofrece el ShiVa Server, el cual es la solución perfecta para aplicaciones multi-jugador y multiusuario. Dependiendo de su ancho de banda, puede manejar miles de jugadores al mismo tiempo, sincronizar sus clientes, e incluso dejarlos hablar el uno al otro a través de VoIP. ShiVa Server está disponible como una licencia independiente. ShiVa Server está disponible en Windows y Linux.


\subsection{Unity3D}

Unity\cite{unity3d} es un poderoso motor para el desarrollo de juegos, un poderoso motor derenderizado totalmente integrado con un conjunto completo de herramientas intuitivas y flujos de trabajo rápidos para crear contenido 3D interactivo, desarrollo multi-plataforma sencillo, miles de activos de calidad listos para usar en la tienda
de activos y una gran comunidad donde se intercambian conocimientos.

En Unity se pueden mezclar de forma muy sencilla elementos 2D con 3D. Posee un editor intuitivo y flexible, 
el nivel visual y de audio son de gran calidad, un sistema de animación poderoso y flexible. Mantiene el 
rendimiento y la calidad de la escena manteniéndola fluida cualquiera sea el tamaño de pantalla. 

Permite desarrollar juegos para múltiples plataformas. Para plataformas moviles iOS, Andriod, Windows Phone 8, BlackBerry 10; plataformas de escritorio como Windows, aplicaciones de la Windows Store, Mac y Linux; plataformas web como Internet Explorer, Mozzilla Firefox, Google Chrome; y plataformas de consolas como Xbox 360, Xbox One, Wii, Wii U, Nintendo 3DS.

La versión paga de Unity, Unity Pro permite además plataformas como PlayStation 4, PlayStation 3 y PlayStation VITA.

Debido a la alta popularidad de Unity, un paquete fue desarrollado por Facebook para colocar la API de
Facebook en un SDK escrito en C\#, el cual es atractivo y fácil de usar en la tienda de activos de Unity.
