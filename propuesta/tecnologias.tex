%! TEX root = ../main.tex
\section{Tecnologías disponibles}

\observacion{Falta una descripción de más alto nivel del tipo de tecnología que se
    necesita}
\observacion{Traten de mantener la descripción del engine con la misma
    profundad, osino parece copipaste}

\observacion{Busquen una estructura Basia para la descripción, tipo:
\begin{itemize}
    \item Quien hace
    \item Tipo de licencia, precio
    \item características principales, lenguajes, etc.
    \item Que tan usado es
\end{itemize}
}

Para el desarrollo de videojuegos se utilizan programas o herramientas
especializadas en ello llamadas \enquote{Motores de videojuegos}. A continuación
se da una breve introducción de lo que es un motor de videojuego o motor
gráfico.

El término \enquote{motor de videojuegos} hace referencia a una serie de rutinas
de programación que permiten el diseño, la creación, el desarrollo y la
representación gráfica de un videojuego\cite{videojuego:telechea}.

Además, la gran mayoría de estos motores ofrecen a su vez características y
funciones que facilitan la construcción del videojuego, como el motor físico
(software capaz de realizar \enquote{simulaciones} de ciertos sistemas físicos
como la dinámica de un cuerpo rígido, el movimiento de un fluido o la
elasticidad) o detector de colisiones, sonidos, \textit{scripting}, animaciones,
inteligencia artificial, comunicación a través de redes, \textit{streaming},
administración de memoria, etc\cite{videojuego:telechea}.

El motor de juego a utilizar depende de las características que posea el
videojuego que se quiere desarrollar. \fixme{Por lo mismo}{NO}, a continuación
se da una breve descripción de los motores más utilizados actualmente, se
definen los criterios para de selección y se realiza una comparación entre los
mismos, y se define el motor que más se adecue a las necesidades.

\observacion{Comentar que puntos hay en comparación}

\subsection{Unreal Development Kit}

\textit{Unreal Engine} es el motor de juegos desarrollado por \textit{Epic
    Games}, se ofrece bajo un plan de suscripción mensual. El servicio de
suscripción permite a los desarrolladores unirse a una comunidad dedicada a la
construcción de grandes juegos y evolución del \textit{Unreal
    Engine}\cite{unrealengine}.

\textit{Unreal Engine} permite desarrollar juegos para plataformas como
\textit{Windows PC, Mac, iOS y Android}. También es compatible con \textit{Xbox
    One} y \textit{PlayStation 4}. \fixme{Existen}{} además trabajos recientes
sobre otras plataformas como \textit{HTLM5} y \textit{Linux}\cite{unrealengine}.

Sin embargo \fixme{existe}{} una versión gratuita del \textit{Unreal Engine}, el
\Gls{udk}. El \Gls{udk} es la edición gratuita de \textit{Unreal Engine
    3}\footnote{Versión previa a la actual versión comercial.} que proporciona
acceso al motor de juegos 3D y a la herramienta profesional que se utiliza en el
desarrollo de videojuegos \textit{blockbuster}, visualización arquitectónica, el
desarrollo de juegos para móviles, modelos 3D, películas digitales y más.
Utilizando \Gls{udk} se pueden implementar juegos y aplicaciones en
\textit{Windows PC}, \textit{iOS} y \textit{Mac}.


\subsection{Blender Game Engine}

\observacion{No hace falta hablar de blender}

\enquote{Blender Game Engine} es el motor de juego de \textit{Blender
    Foundation} que permite crear aplicaciones 3D interactivas o simulaciones,
desarrollabo bajo la licencia \Gls{gnu}\cite{blender}.

\textit{Blender Game Engine} genera las escenas de forma continua en tiempo real
e incorpora facilidades para la interacción del usuario durante el proceso de
\textit{renderización}\cite{blender}.

Procesa la lógica de sonido, de la física y la representación de simulaciones en
orden secuencial. El motor esta escrito en \textit{C++}. Posee un editor que
proporciona una profunda interacción con la simulación, su funcionalidad se
puede ampliar con \textit{scripts} \textit{Python} y esta diseñado para abstraer
las características complejas del motor en una interfaz de usuario simple, que
no requiere programación\cite{blender}.

El motor de juego puede simular contenido dentro de \textit{Blender}, sin
embargo, también incluye la posibilidad de exportar en plataformas como
\textit{Windows, Linux y Mac OS}. También hay soporte básico para plataformas
móviles con el proyecto \textit{Android Blender Player \fixme{GSOC}{?} 2012}.

\subsection{CryEngine 3}

\observacion{Las descripciones tienen que restar lo mejor y lo pero de cada
    engine, osino parece una propaganda}

\enquote{CryEngine 3} es el motor de juegos desarrollado por \textit{Crytek}.
\textit{CryEngine} es un motor avanzado para el desarrollo de juegos, películas,
simulaciones de alta calidad y aplicaciones interactivas. Tiene características
diseñadas específicamente para \textit{Windows PC, PlayStation 3 y Xbox
    360}\cite{cryengine}.

Existe una versión gratuita de \textit{CryEngine}, denominada \enquote{CryEngine
    Free SDK} con todas las funcionalidades, esta versión esta disponible para
su descarga, pero ya ha sido descontinuada\cite{cryengine:sdk}.

\textit{CryEngine} posee además un conjunto de herramientas utilizadas para el
análisis de rendimiento\cite{cryengine}, además de un \Gls{ide}, que permite
editar texturas, interacciones, vehículos, etc.

\subsection{ShiVa3D}

\textit{ShiVa3D} es un paquete para el desarrollo de juegos y aplicaciones 3D,
posee un \Gls{ide} \Gls{wysiwyg} potente\cite{shiva}.

\textit{ShiVa} puede exportar juegos y aplicaciones para más de $20$ plataformas
de destino, incluyendo móviles como \textit{iOS, Android, BlackBerry y Windows
    Phone}, de escritorio como \textit{Windows, Mac OS X y Linux}, los
navegadores web con soporte \textit{Flash y HTML5}, así como consolas como la
\textit{Xbox 360, PlayStation3 y Nintendo Wii}. El \Gls{ide} se ejecuta en
\textit{Windows} y \textit{Mac OS X}\cite{shiva}. 

Un producto relacionado es el \enquote{ShiVa Server}, el cual permite el
desarrollo de aplicaciones multijugador. Las carácteristicas de este servidor,
incluyen alto rendimiento, comunicación \textit{VoIp}, etc. \enquote{ShiVa
    Server} se distribuye con una licencia distinta a
\textit{Shiva3D}\cite{shiva}.

\subsection{Unity3D}

\textit{Unity}\cite{unity3d} es un motor para el desarrollo de juegos,
dearrollado por \textit{Unity Technologies}, incluye un \Gls{ide} con un un
motor de \textit{renderizado} y flujos de trabajo para la creación de contenido
3D interactivo, desarrollo multiplataforma. Además cuenta con una gran cantidad
de \textit{assets}\footnote{Un \textit{Asset} es un paquete \textit{Unity} que
    puede contener modelos, librerias, sonidos, etc.} disponibles en un
\enquote{Asset Store} y una gran comunidad donde se intercambian conocimientos.

En \textit{Unity} se pueden desarrollar de forma sencilla elementos 2D y 3D.
Posee un \Gls{ide} intuitivo y flexible, el nivel visual y de audio son de gran
calidad, un sistema de animación poderoso. 

Permite desarrollar juegos para múltiples plataformas. Entre las plataformas
móviles soportadas, encontramos a \textit{iOS, Andriod, Windows Phone 8,
    BlackBerry 10}, entre las plataformas de escritorio a \textit{Windows, Mac y
    Linux}, plataformas web como \textit{Internet Explorer, Mozzilla Firefox,
    Google Chrome}\footnote{Requiere un plugin para el navegador que está
    disponible para \textit{Windows y Mac}}, y entre plataformas de consolas a
\textit{Xbox 360, Xbox One, Wii, Wii U, Nintendo 3DS}.

La versión paga de \textit{Unity}, \textit{Unity Pro} permite además plataformas
como \textit{PlayStation 4, PlayStation 3 y PlayStation VITA}.

Debido a la alta popularidad de \textit{Unity}, un paquete fue desarrollado por
\textit{Facebook} que permite una interacción sencilla con la \Gls{api} de la
red social \textit{Facebook} en un \textit{Asset} escrito en \cs{}, este paquete
se encuentra en \textit{Asset Store} de \textit{Unity}.
