%! TEX root = ../main.tex
\chapter{Propuesta de solución}
\label{chap:solucion}


Descriptos los fundamentos teóricos del uso de videojuegos serios en el ambiente
educativo, sobre todo en aquellas áreas que requieren más practica para la
adquisición de la pericia necesaria, e identificando el área de enfermería como
uno de estas se definieron las problemáticas actuales de la educación de
enfermeros y seleccionamos algunos procedimientos que serán utilizados para el
contenido de la solución. A continuación se busca converger todos los aspectos
descriptos en capítulos anteriores.

La solución propuesta en este trabajo consiste en el desarrollo de una
aplicación para dispositivos móviles que se define como un juego serio llamado
\Gls{yave}, basado en el construccionismo, el cual incluye conceptos de la
gamificación, y de simulación. El juego consiste en ofrecer a los usuarios, en
este caso alumnos de enfermería, un medio en el cual puedan realizar
procedimientos de enfermería y cuyo objetivo es servir como herramienta de apoyo
en el aprendizaje.

\Gls{yave} permitirá al usuario poder seleccionar el procedimiento que quiera
realizar, en cada procedimiento se le dará la posibilidad de interactuar con un
paciente y con una conjunto de objetos que forman parte de las herramientas
requeridas para realizar el procedimiento seleccionado. Además, contempla la
posibilidad de realizar acciones relacionadas a la bioseguridad, y otros 
aspectos transversales a la educación de un enfermero.

La solución no solo le permitirá al usuario realizar los procedimientos para
poner en practica sus conocimientos sobre el mismo, sino también evaluará al
usuario, dándole al final de cada sesión una puntuación y describiendo los
pasos realizados correcta e incorrectamente, proporcionando información acerca
de los puntos incorrectos.


%

\section{Acciones condicionadas por eventos}

Las \gls{eca} son aquellas que son lanzadas una vez que se cumple un determinado
evento\cite{bailey2004event}. En las bases de datos relacionales, son conocidos



Las mismas pueden ser utilizadas para notificar que un determinado conjunto de
eventos ha ocurrido\cite{bailey2004event}, así como servir para almacenar
información acerca de la utilización de un determinado recurso.

\subsection{Motivación}

Las reglas del tipo \gls{eca} permiten reaccionar a determinados eventos, en
forma de una única regla, la cual facilita la declaración de las
mismas\cite{bailey2004event}.

Son principalmente útiles para analizar el comportamiento en tiempo real de un
sistema\cite{bailey2004event}.
%TODO agregar más motivaciones.



\subsection{Declaración}

Una \gls{eca}, se define como:

\begin{center}
	 Cuando ocurren una serie de \emph{eventos}, y se cumple una
	 \emph{condición}, entonces realizar una acción. \emph{Acción}
\end{center}

Los eventos determinan cuando una regla debe ser activada, los mismos se dividen
en dos categorías, primitivos y compuestos, los primeros son detectables, por
ejemplo, cuando se inserta una jeringa, y los compuestos, son la combinación de
uno o más primitivos\cite{bailey2004event}. Los eventos compuestos, se unen
mediante:
\begin{enumerate*}[label=\itshape\alph*\upshape)]
\item conjunción (\emph{y}),
\item disjunción (\emph{o}), y
\item secuencia (\emph{entonces}).
\end{enumerate*}
Sin embargo, no siempre son necesarios todas las posibles combinaciones, y las
combinaciones sencillas son más fáciles de optimizar y
probar\cite{bailey2004event}.

Una regla  puede tener argumentos, los cuales son el entorno en el cual se lanzo
el evento que lo lanza.

Las condiciones determinan si el entorno es el necesario para que la regla sea
activada.

La acción a ejecutar describe la lógica que debe ser ejecutada cuando se han
lanzado los eventos y la condición de la regla se ha cumplido.

\subsubsection{Dependencia entre reglas}

Las reglas pueden depender de otras reglas, lo cual se puede ver como que la
finalización de una regla es un evento que otra regla espera para poder ser
activada.

Las reglas pueden agregar información a un contexto compartido por todas las
reglas, de esta manera, se puede pasar parámetros entre distintas reglas, por
ejemplo, la regla \emph{Retirar Torniquete}, depende de la regla \emph{Insertar Torniquete}, pero debe responder solamente al torniquete
que ha activado la regla de inserción, es decir, el usuario puede extraer varios
torniquetes, y la regla no debe activarse, hasta que se extraiga el torniquete
que activo la primer regla.

Así, la regla \emph{Retirar Torniquete} depende de la regla \emph{Insertar
Torniquete}, y esta relación entre reglas, se da en dos formas:

\begin{itemize}
\item  \emph{Dependencia fuerte:} la regla \emph{Retirar Torniquete} solamente podrá
	ser elegida para ser lanzada cuando la regla \emph{Insertar Torniquete}
	haya sido cumplida.
\item  \emph{Dependencia de contexto}: la regla \emph{Retirar Torniquete} no se
	activará cuando los eventos a los que escucha se terminen, sino cuando
	los eventos a los que escucha sean lanzados con los parámetros adecuados
	(se extraiga el torniquete que lanzo la regla de inserción).
\end{itemize}




\subsection{Modelo de ejecución}

Para ejecutar un motor de reglas del tipo \gls{eca}, se debe tener en cuenta
principalmente dos factores, 
\begin{enumerate*}[label=\itshape\alph*\upshape)]
\item  Como se verifica el cumplimiento de cada regla, y, 
\item  Que ocurre cuando varias reglas son lanzadas al mismo tiempo.
\end{enumerate*}.

Para ambos casos se puede tomar un enfoque \emph{inmediato}, es decir que
inmediatamente cuando se lanza un evento, o se cumple una condición, se ejecuta
la regla. Además existen otros dos modos de ejecución, \emph{deferida}, y
\emph{desacoplada}, en la primera, se espera hasta que el lanzador del evento
culmine su trabajo, y luego se ejecute la regla, pero en la misma unidad de
trabajo, mientras que en la ejecución desacoplada, se encolan los trabajos y
otro hilo es el encargado de ejecutar las reglas.

La propuesta implementada, utiliza una ejecución inmediata, principalmente por
la sencillez de las reglas, es decir, las reglas no realizar un trabajo pesado,
solamente controlan el estado del entorno y lo validan.

Además, la ejecución inmediata es importante por que el entorno no sufre
modificaciones entre el evento lanzado y la ejecución de la regla, según
\cite{bailey2004event}, este es el factor más importante para determinar el tipo
de ejecución deseado.



\subsubsection{Estados de una regla}

Una regla puede estar en uno de los siguientes estados:

\begin{description}
\item[BEGIN] Es una regla que recién fue creada, no realiza ninguna
	acción.
\item[WAITING\_FOR\_RULE] Es un estado en el que esta esperando que otras reglas
	sean lanzadas.
\item[WAITING\_FOR\_EVENT] Es un estado en el que esta escuchando a que sean
	lanzados los eventos a los que escucha, este es el estado principal.
\item[WAITING\_FOR\_CONDITION] La regla ya no espera por ningún evento y las
	reglas de las que depende ya han sido lanzadas, se verifica cada cierto
	tiempo si el entorno cumple con una condición definida.
\item[FINISH] La regla ha sido lanzada, con un resultado no determinado, se pudo
	haber cumplido, como no, es el estado final de una regla. Cuando una
	regla llega a este estado, se lanza su evento de finalización.
\end{description}

Una regla puede estar en solo un estado, y solamente se permite que el estado
avance, desde \emph{BEGIN} hasta \emph{FINISH}.

\subsubsection{Ciclo de vida}

Cuando una regla es definida, y insertada al motor de reglas, inmediatamente
pasa al estado \emph{BEGIN}, luego se verifica si la misma depende de otras
reglas, sí este es el caso, pasa al estado \emph{WAITING\_FOR\_RULE} y escucha a
los eventos de finalización de las reglas anteriores.

Una vez que las reglas anteriores han sido finalizadas, la regla pasa al estado
\emph{WAITING\_FOR\_EVENT} sí deben escuchar por algún evento, en caso contrario
pasan al estado \emph{WAITING\_FOR\_CONDITION}.

Una vez que la regla está en estado \emph{WAITING\_FOR\_CONDITION}, pasa a un
motor que ejecuta su condición cada cierto tiempo, si la condición se cumple, la
regla se ejecuta, y la misma pasa a estado \emph{FINISH}, momento en el cual
notifica a las reglas que dependen de ella que ha sido lanzada.

Una vez que la regla esta en estado \emph{FINISH}, la misma sale del esquema de
ejecución, y solo esta disponible para obtener resultados.




%! TEX root = ../main.tex
\section{Tecnologías disponibles}

Para el desarrollo de videojuegos se utilizan programas o herramientas especializadas en ello llamadas motores de videojuegos. A continuación se da una breve introducción de lo que es un motor de videojuego o motor gráfico.

El término “motor gráfico” o “motor de videojuegos” hace referencia a una serie de rutinas de programación que permiten el diseño, la creación, el desarrollo y la representación gráfica de un videojuego\cite{videojuego:telechea}.

Además, la gran mayoría de estos motores ofrecen a su vez características y funciones que facilitan la construcción del videojuego, como el motor físico (software capaz de realizar simulaciones de ciertos sistemas físicos como la dinámica de un cuerpo rígido, el movimiento de un fluido o la elasticidad) o detector de colisiones, sonidos, scripting, animaciones, inteligencia artificial, redes, streaming, administración de memoria, etc\cite{videojuego:telechea}.

El motor de juego utilizado depende de las características que posea el videojuego que se quiere desarrollar. Por lo mismo, a continuación se da una breve descripción de los motores mas utilizados actualmente para que posteriormente se puede seleccionar el mas adecuado.

\subsection{UDK (Unreal Development Kit)}

Unreal Engine\cite{unrealengine} es el motor de juegos desarrollado por Epic Games, se ofrece bajo un plan de suscripción de 19 dolares por mes. El servicio de suscripción permite a los desarrolladores unirse a una comunidad dedicada a la construcción de grandes juegos y evolución de Unreal Engine. La suscripción se puede cancelar en cualquier momento.

UnReal Engine 4 permite desarrollar juegos para plataformas como Windows PC, Mac, iOS y Android. También es compatible con Xbox One y PlayStation 4. Existen además trabajos recientes sobre otras plataformas como HTLM5 y Linux.

Sin embargo existe una versión gratuita de UnrealEngine, el Unreal Development Kit o UDK. El UDK es la edición gratuita de Unreal Engine 3 que proporciona acceso al galardonado motor de juegos 3D y herramienta profesional que se utiliza en el desarrollo de videojuegos blockbuster, visualización arquitectónica, el desarrollo de juegos para móviles, modelos 3D, películas digitales y más. Utilizando UDK se pueden implementar juegos y aplicaciones en Windows PC, iOS y Mac.


\subsection{Blender Game Engine}

Blender Game Engine\cite{blender} es el motor de juego de Blender Foundation que permite crear aplicaciones 3D interactivas o simulaciones. La principal diferencia entre el Blender Game Engine y el Blender convencional está en el proceso de renderizado. En el motor Blender normal las imágenes y animaciones se construyen fuera de linea es decir, una vez generadas no pueden ser modificadas. Por el contrario Blender Game Engine genera las escenas de forma continua en tiempo real e incorpora facilidades para la interacción del usuario durante el proceso de renderización.

Procesa la lógica de sonido, de la física y la representación de simulaciones en orden secuencial. El motor esta
escrito en C++. Posee un editor que proporciona una profunda interacción con la simulación, su funcionalidad se
puede ampliar con scripts Python y esta diseñado para abstraer las características complejas del motor en una interfaz de usuario simple, que no requiere programación.

El motor de juego puede simular contenido dentro de Blender, sin embargo, también incluye la posibilidad de exportar en plataformas como Windows, Linux y Mac OS. También hay soporte básico para plataformas móviles con el proyecto Android Blender Player GSOC 2012.

\subsection{CryEngine 3}

CryEngine 3\cite{cryengine} es el motor de juegos desarrollado por Crytek. CryEngine es un motor avanzado para el desarrollo de juegos, películas, simulaciones de alta calidad y aplicaciones interactivas. Es uno de los procesadores de alta gama mas rápido en el mundo, con características diseñadas específicamente para Windows PC, PlayStation 3 y Xbox 360.

Existe una versión gratuita de CryEngine con todas las funcionalidades. CryEngine es  WYSIWYG (What You See Is What You Get). 

CryEngine posee un conjunto de herramientas utilizadas para el análisis de rendimiento. La ultima versión
de CryEngine, CryEngine 3 es el único motor con la última tecnología en iluminación, física e inteligencia
artificial si se está desarrollando en Windows PC, PlayStation 3 y Xbox 360.


\subsection{ShiVa3D}

ShiVa3D\cite{shiva} es un paquete para el desarrollo de juegos y aplicaciones 3D, viene en un formato fácil de usar, pero con un editor WYSIWYG (What You See Is What You Get) muy potente.

ShiVa puede exportar juegos y aplicaciones para más de 20 plataformas de destino, incluyendo móviles como iOS, Android, BlackBerry y Windows Phone, de escritorio como Windows, Mac OS X y Linux, los navegadores web con soporte Flash y HTML5, así como Consolas como la Xbox 360, PlayStation3 y Nintendo Wii. La herramienta de creación se ejecuta en Windows y Mac OS X. Software y hardware Apple son requeridos con el fin de construir juegos para OS X y iOS.

ShiVa también ofrece el ShiVa Server, el cual es la solución perfecta para aplicaciones multi-jugador y multiusuario. Dependiendo de su ancho de banda, puede manejar miles de jugadores al mismo tiempo, sincronizar sus clientes, e incluso dejarlos hablar el uno al otro a través de VoIP. ShiVa Server está disponible como una licencia independiente. ShiVa Server está disponible en Windows y Linux.


\subsection{Unity3D}

Unity\cite{unity3d} es un poderoso motor para el desarrollo de juegos, un poderoso motor derenderizado totalmente integrado con un conjunto completo de herramientas intuitivas y flujos de trabajo rápidos para crear contenido 3D interactivo, desarrollo multi-plataforma sencillo, miles de activos de calidad listos para usar en la tienda
de activos y una gran comunidad donde se intercambian conocimientos.

En Unity se pueden mezclar de forma muy sencilla elementos 2D con 3D. Posee un editor intuitivo y flexible, 
el nivel visual y de audio son de gran calidad, un sistema de animación poderoso y flexible. Mantiene el 
rendimiento y la calidad de la escena manteniéndola fluida cualquiera sea el tamaño de pantalla. 

Permite desarrollar juegos para múltiples plataformas. Para plataformas moviles iOS, Andriod, Windows Phone 8, BlackBerry 10; plataformas de escritorio como Windows, aplicaciones de la Windows Store, Mac y Linux; plataformas web como Internet Explorer, Mozzilla Firefox, Google Chrome; y plataformas de consolas como Xbox 360, Xbox One, Wii, Wii U, Nintendo 3DS.

La versión paga de Unity, Unity Pro permite además plataformas como PlayStation 4, PlayStation 3 y PlayStation VITA.

Debido a la alta popularidad de Unity, un paquete fue desarrollado por Facebook para colocar la API de
Facebook en un SDK escrito en C\#, el cual es atractivo y fácil de usar en la tienda de activos de Unity.

\section{Selección de plataforma}
\section{Entorno de desarrollo}

La creación de la solución es un proceso complejo, el cual se compone de varios
pasos necesarios, para llevar a cabo el desarrollo completo es necesario recurrir
a una gran cantidad de tecnologías, herramientas, \textit{frameworks}, y
recursos.

En esta sección se describen todos las partes involucradas en el desarrollo de
la solución propuesta, con excepción del motor de simulación, el cual es
descrito en~\ref{sec:seleccion_plataforma}.

La solución se compone de dos partes, la primera es la aplicación que los
usuarios utilizan para realizar las prácticas, denominada solución, y la segunda
parte, es un servidor que se encarga de almacenar la información sobre los
usuarios de la solución y como la utilizan, denominado \textit{backend}.

Primeramente se definen las herramientas de gestión de código, pues ellas son
utilizadas tanto por la solución como por el \textit{backend}, luego se definen
las herramientas específicas de la creación de la simulación y por último, se
citan y describen de manera breve las herramientas utilizadas por el
\textit{backend}.

\subsection{Herramientas de gestión de código}

La gestión del codigo fuente desarrollado como parte de esta tesis se realizo
mediante la utilización de la herramienta de control de código fuente
\textit{Git}\cite{git}, \textit{Git} es de código abierto bajo la licencia
\Gls{gnu}.

El servidor de \textit{Git} utilizado es \textit{BitBucket}\cite{bitbucket}, el
cual es un servidor que almacena repositorios \textit{GIT} de manera gratuita,
la principal carácteristica y motivación por la cual se utiliza este servicio es
que el mismo permite mantener varios repositorios privados de manera
gratuita\cite{bitbucket}, de esta manera, se utilizan distintos repositorios,
específicos para cada parte del presente trabajo, un repositorio para los
documentos de las reuniones y la propuesta de tesis, un repositorio que mantiene
el código fuente del libro de la tesis, un repositorio que mantiene el código
fuente, imágenes y diferentes archivos multimedia de la solución, y finalmente
un repositorio que mantiene el código fuente del servidor que almacena los datos
de los usuario.

\subsection{Desarrollo de la simulación}

El desarrollo de la solución requiere de una variedad importante de tecnologías
aparte del motor, cuya elección fue descrita en~\ref{sec:seleccion_plataforma},
las herramientas descritas en esta sección complementan al motor seleccionado o
facilitan la creación de contenido.

Se utilizo el \textit{Unity Editor} para la creación de las escenas,
\textit{MonoDevelop} y \cs para la parte de programación de la interacción entre
los componentes de la solución.

Adicionalmente se utilizaron varias herramientas de diseño para crear
componentes 3D y 2D, \textit{Make Human} es utilizado para la creación de los
pacientes, \textit{3D Max} permite la creación de objetos como gazas, y otros
elementos utilizados dentro de la solución. En cuanto a los gráficos 2D, se
utilizo \textit{Photoshop}, y diversas páginas web que proveían contenido
gratuíto.

\subsubsection{Unity Editor}

El editor de Unity totalmente integrado y expansible es la interfaz en la se
crean los juegos o simulaciones. Este editor importa todos los activos que se
hayan seleccionado y los organiza dentro de el. Permite compilar las escenas con
los terrenos, luces, audio, personajes, física, entre otros. Se puede agregar
interacción a través de scripting. En este editor se puede probar y editar en
forma simultanea los juegos o simulaciones y desplegarlos en las plataformas
elegidas. Existen cinco vistas en el editor: explorador de proyectos, inspector,
jerarquía, escena y juego. Cada una de estas escenas cuenta con las herramientas
necesarias para permitir un fácil desarrollo del juego\cite{unity3d}.


\subsubsection{MonoDevelop}

MonoDevelop es de código abierto, bajo la licencia \Gls{gnu} y siendo apoyado
principalmente por la comunidad \textit{Mono}. Es el \Gls{ide} utilizado por
defecto en el desarrollo de aplicaciones para \textit{Unity3D}, el mismo soporta
varios lenguajes de programación, como \cs, \textit{UnityScript}, y
\textit{Boo}. Es un \Gls{gnu} multiplataforma que soporta \textit{Windows} al
igual que \textit{Unity3D}.

\subsubsection{\cs}

\cs es un lenguaje de programación orientado a objetos diseñado por
\textit{Microsoft}, es uno de los tres lenguajes disponibles en la plataforma
\textit{Unity3D}, los otros dos lenguajes son \textit{UnityScript} y
\textit{Boo}\cite{unity:script}.

La selección de \cs como lenguaje de programación esta relacionada con la
familiaridad de los autores con lenguajes similares, por ser el lenguaje con más
ayuda en linea, y por que las librerias no diseñadas específicamente para
\textit{Unity3d} pueden ser utilizadas.

Cabe mencionar que la \textit{UnityScript} es normalmnete confundido con
\textit{JavaScript}, lenguaje del que esta inspirado, pero las librerias
disponibles para \textit{JavaScript} no funcionan con \textit{UnityScript}, y la
cantidad de librerias disponibles para \textit{Boo} es muy limitada.

\subsubsection{Herramientas de diseño}

Las herramientas utilizadas para crear modelos 3D son las siguientes:

\begin{itemize}
\item \textbf{MakeHuman}: es un software gratuito y de código abierto bajo la
    licencia \Gls{agnu} para crear personajes humanos 3D. Es una herramienta
    diseñada para simplificar la creación de seres humanos virtuales utilizando
    una interfaz gráfica de usuario\cite{makehuman}. 
\item \textbf{3DS MAX}: es un software privado de modelado 3D completo, que
    además posee herramientas para animación, simulación y renderizacion. Esta
    herramienta fue utilizada para crear objetos 3D que no fueran personajes
    humanos y para exportar modelos de un formato a otro que fuera compatible
    con \textit{Unity3d}\cite{3dsmax}.
\item \textbf{Photoshop}: es una herramienta de edición de gráficos 2D de
    \textit{Adobe}, permite la creación y edición de gráficos, es utilizada para
    la creación de iconos, botones y demás contenido 2D que forma parte de la
    solución.
\end{itemize}

\subsection{Desarrollo del \textit{backend}}

A fin de registrar las actividades del usuario, se necesita de un servidor que
almacene los datos centralizados de todos los usuarios y las actividades que
estos realizan.

Este servidor debe proveer:

\begin{itemize}
    \item \textbf{Alta disponibilidad}: el servidor debe estar disponible en
        todo momento, cualquier día de la semana y a cualquier hora. Los
        requisitos de accesibilidad son estrictos, pues se necesita que los
        usuarios envíen datos sin inconvenientes cuando crean necesario.
    \item \textbf{Accesibilidad}: el servidor debe poder ser accesible desde
        cualquier red móvil.
\end{itemize}

Para el desarrollo de la aplicación web que almacena los datos se utiliza
\Gls{javaee} en su versión 6, la misma se utiliza por la familiarización de los
autores con la tecnología, y la facilidad que provee la misma para la
realización de servicios web que permitan la interacción con la solución.

\Gls{javaee} es un estándar de software empresarial de código abierto cuyo
desarrollo es dirigido por la comunidad\cite{javaee}, la implementación
utilizada es de \textit{RedHat} llamada \textit{JBoss} en su versión 7.1,
\textit{JBoss} es un esfuerzo de la comunidad dirigido por \textit{RedHat}. 

Para los servicios se utiliza la arquitectura \Gls{rest}, la principal
motivación para utilizar \Gls{rest} es eficiencia en el uso de la
red\cite{pautasso2008restful}, el cual es también es la motivación para la
utilización \Gls{json}. 

La implementación del lado del servidor de la arquitectura \Gls{rest} es
\textit{RestEasy}, \textit{RestEasy} es un proyecto de código abierto dirigido
por \textit{RedHat} y que forma parte de la plataforma \textit{JBoss}, de parte
de la solución se utiliza la implementación por defecto de \textit{Unity3D}.

Otro factor determinante es la facilidad de interacción que existe entre Unity3D
y los servicios \Gls{rest}, debido a que \Gls{rest} cumple con el protocolo
HTTP, y en ambos ecosistemas (\textit{JavaEE} y \textit{Unity3d}) existen
implementaciones disponibles y de fácil utilización.

Para el desarrollo del servidor web se utilizo el \Gls{ide} \textit{Eclipse}, el
cual es un proyecto de código abierto dirigido por la \textit{Eclipse
    Fundation}\cite{eclipse}, el mismo provee facilidades para el desarrollo de
servicios web \Gls{rest}.

El almacenamiento permanente de los datos se logra con la utilización de
\textit{PostgreSQL}, el cual es un motor de bases de datos de código abierto
dirigido por una comunidad de desarrolladores llamada \textit{PostgreSQL Global
    Development Group}. La versión elegida es la 9.1.

\subsubsection{OpenShift}

A fin de obtener las características necesarias, de alta disponibilidad y
accesibilidad, se utiliza la herramienta de plataforma como servicio de
\textit{RedHat} llamada \textit{OpenShift}, el cual es un producto de código
abierto dirigido por \textit{RedHat}.

\textit{OpenShift} provee diferentes plataformas, entre las cuales se encuentran
las herramientas seleccionadas \textit{PostgreSQL} 9.1 y \textit{JBoss
    Application Server} 7.1.

Además de dirigir el proyecto, \textit{RedHat} provee un servicio limitado y
gratuito\cite{openshift:pricing}.\footnote{Existen versiones completas del
    producto mantenidas por \textit{RedHat}, las cuales tienen un costo mensual
    y de acuerdo a las funcionalidades utilizadas\cite{openshift:pricing}} Para
esta tesis se utilizo el servicio gratuito con la plataforma \textit{JBoss
    Application Server 7.1} y \textit{PostgreSQL 9.1}.

Otra característica importante de \textit{OpenShift} es la facilidad con la cual
se pueden desplegar nuevas versiones de la aplicación, utiliza un repositorio
\textit{GIT} para mantener el código de la última versión, y cada vez que se
actualiza este repositorio, la aplicación se despliega con la versión
actualizada.

\observacion{Ver si hay que agregar las librerías utilizadas}
% Facebook
% NGUI 3
% Diferentes paquetes de gráficos.

\section{Hipótesis de la simulación}
\label{sec:hipotesis}


\observacion{Ver en la tesis de Tardon: \emph{no es necesario simular todos los
        pasos}.}
\observacion{Ver si hay que cambiar la palabra trivial}

Las escenas seleccionadas y definidas en~\ref{sec:seleccion_escenas} representan
las acciones que deben realizar los profesionales de enfermería a la hora de
realizar los procedimientos seleccionados, por limitaciones técnicas,
tecnológicas y de tiempo, no es posible realizar una simulación de todos los
pasos requeridos.

Los factores que influyen en que partes se simularán, que partes estarán
presentes solamente a través de opciones y que partes se omitirán son:

\begin{itemize}

    \item \textbf{Limitaciones técnicas}: acciones como la simulación del agua
        (necesarios para el lavado de manos), requieren de requisitos de
        hardware avanzados y un tiempo considerable de desarrollo. Las acciones
        que escapan al alcance del hardware y tiempo de los desarrolladores no
        son simuladas.

    \item \textbf{Importancia}: no todos los pasos definidos en el procedimiento
        oficial son necesarios, por ejemplo, la colocación de los elementos
        cerca del lugar de trabajo, es un paso necesario, pero es considerado un
        paso poco importante y trivial.

        La importancia es evaluada por profesionales del \Gls{iab}, los cuales
        dieron su opinión acerca de cada aspecto simulado, el mismo es tenido en
        cuenta para determinar la importancia de cada acción.

    \item \textbf{Facilidad de realización en la vida real o en el laboratorio}:
        ciertos pasos son triviales en la vida real pero requieren un esfuerzo
        significativo para ser simuladas, como por ejemplo el lavado de manos es
        un procedimiento al que los alumnos están acostumbrados.

        La facilidad que tienen los alumnos con las acciones fue determinada por
        profesores del \Gls{iab}, determinaron que acciones son triviales para
        los alumnos y cuales presentan mayores dificultades en su vida
        profesional.

        Otro aspecto que influye en la facilidad de realización de los
        procedimientos es la familiarización, si los alumnos están
        familiarizados con los procedimientos, estos no son simulados.

\end{itemize}

Estas hipótesis sirven para acotar el alcance de la simulación, definen qué se
simulará y cual es del detalle necesario para alcanzar las competencias básicas.

Existen hipótesis que son globales para toda la simulación, las mismas son:

\begin{itemize}

    \item \textbf{Comandos de voz con interfaz}: para enviar una petición al
        paciente (por ejemplo, preguntarle su nombre), no es necesario
        identificar las palabras del usuario, sino más bien detectar que ha
        hablado y listar las posibles acciones que se pueden realizar.

    \item \textbf{Utilización de la interfaz}: para realizar una acción con los
        elementos, es suficiente con presionar el mismo y seleccionar una acción
        de una lista de opciones, no hace falta emular todas las posibles.

    \item \textbf{Acciones de bioseguridad}:\todox{Definir bioseguridad} Las
        acciones de bioseguridad, se realizan a través de una opción en la
        interfaz gráfica.

\end{itemize}

Otras hipótesis, son tomadas por escena, las dos escenas simuladas son
diferentes en el modo de interacción del usuario con su entorno, por ejemplo, en
la escena de extracción de sangre, el usuario interactúa con el paciente a
través de objetos, en la evaluación de Glasgow, la interacción con el paciente
es directa.

\subsection{Extracción de sangre}
\label{sec:hemocultivo_hipotesis}

Se presentan los pasos mostrados en la sección~\ref{sec:hemocultivo_protocolo},
y adicionalmente se establecen las hipótesis punto por punto y las
consideraciones que deben ser tomadas.

\begin{itemize}

\item \textbf{Preparar el equipo}: la preparación del equipo es un aspecto muy
    importante del procedimiento, pero no es un punto único de la extracción de
    sangre, además las prácticas de los alumnos cubren completamente este paso
    según comentarios de los profesores. \emph{Este paso no se simula}.

\item \textbf{Explicación al paciente del procedimiento a realizar}: es un
    aspecto importante del procedimiento, pero la simulación de una conversación
    alumno-paciente es compleja, según comentarios de los profesores, es
    suficiente con que los alumnos sepan que lo deben realizar y en que moento,
    no es necesario simular la conversación en sí. \emph{Este paso se simula a
        través de un comando de voz con la interfaz}.

\item \textbf{Asepsia de las manos}: este paso forma parte de un área más amplia
    conocida como bioseguridad, la cual es un aspecto transversal a todos los
    procedimientos realizados por los enfermeros. 
    La implementación de una simulación del lavado de mano es compleja, y es un
    aspecto que, al igual que la preparación del equipo, está cubierta por los
    laboratorios, aún así, es necesario que los alumnos sepan en que momento
    deben realizar la asepsia de sus manos. \emph{Se simula este paso a través de una
        opción en la interfaz}, no se simulan los pormenores del lavado de manos.

\item \textbf{Llevar el equipo a la unidad en donde se encuentra el paciente}:
    este es un paso trivial que deben realizar los profesionales, la simulación
    de este proceso no es importante según comentarios de los profesores. Este
    proceso no tiene importancia según los profesionales del \Gls{iab}, \emph{este
        paso no se simula}.

\item \textbf{Vestirse con bata estéril, tapaboca y gorro}: al igual que la
    asepsia de las manos, es importante que los alumnos sepan que lo deben
    hacer, pero no es importante que se simule como lo hacen. 

    Los estudiantes están familiarizados con estas acciones, \emph{se simula el
        momento y el orden en el que el jugador lo hace} a través de una opción
    en la interfaz, no se simula el proceso en sí.

\item \textbf{Calzarse los guantes}: es un paso relacionado a la bioseguridad,
    es importante que se sepa en que momento debe realizarse, pero no es
    necesario simular el proceso. 
    \emph{Se simula el momento y el orden en el que se realiza}, no se simulan
    los pormenores de la acción.

\item \textbf{Ubicar al paciente en posición adecuada}: la ubicación del
    paciente durante la extracción de sangre es un factor determinante para que
    la extracción pueda ser realizada correctamente.

    Los alumnos están familiarizados con este proceso según opinión de los
    profesionales, \emph{este paso no se simula}, el paciente está en la
    posición adecuada al inicio de la simulación.

\item \textbf{Elegir la zona a puncionar}: existen varias partes del antebrazo
    donde se puede proceder a realizar una inyección, el conocimiento de las
    mismas, y el procedimiento para detectarlas, es un factor importante del
    proceso.
    
    Las venas del cuerpo humano se detectan palpando los antebrazos, y sintiendo
    el pulso del paciente, existen dos áreas donde el pulso es suficientemente
    fuerte como para sel sentido, estos puntos y el pulso del paciente deben ser
    detectables por el jugador.

    \emph{Los puntos donde se debe punzar deben ser identificables en la
        simulación}. 

\item \textbf{Colocación del torniquete}: La ubicación y el momento de la
    colocación del torniquete es de vital importancia para el procedimiento, el
    mecanismo utilizado para colocarlo no es relevante, pues el mismo es
    trivial.

    \emph{El hecho de colocar el torniquete es simulado}, el mecanismo para
    hacerlo no es importante.

\item \textbf{Solicitar al paciente que cierre el puño}: El momento en el cual
    se solicita al paciente que cierre la mano es vital para que el
    procedimiento de extracción sea satisfactorio.

    \emph{Este paso es simulado} a través de un comando de voz.

\item \textbf{Esterilizar la zona de punción}: la esterilización de la zona de
    punción es un factor de suma importancia para el procedimiento, así como el
    momento en el que se realiza, \emph{el jugador debería poder esterilizar} la
    zona antes de insertar la jeringa \todox{ver si se debe poner mas detalle para explicar 
    que no se simula enteramente este paso}.
    
\item \textbf{Extraer el protector de la aguja}: La extracción del protector de
    la aguja es un paso necesario, pero trivial, el hecho de retirar el
    protector de la jeringa \emph{no es un paso necesario para el logro de las
        competencias básicas necesarias}, por ello, no se simula.

\item \textbf{Puncionar la piel con la aguja}: este es un paso central en el
    procedimiento, en el se deben tener en cuenta aspectos como la posición
    donde se realiza la punción, y el angulo con el que ingresa la aguja.

    La posición donde se realiza la punción es importante por que depende de la
    ubicación donde se colocó el torniquete, y debe ser en uno de los puntos del
    brazo donde existen venas capaces de soportar el procedimiento, \emph{en
        cada brazo existen dos puntos donde se puede inyectar}.

    En cuanto al ángulo de punción, es un conocimiento importante que deben
    tener los alumnos, el conocimiento es teórico y según comentarios de los
    profesores, es un tema en el cual los alumnos tienen suficiente práctica en
    el laboratorio, \emph{No se simula el ángulo en el cual se inserta la
        jeringa}, es decir, la jeringa siempre se inserta en el mismo angulo.

\item \textbf{Tensar la zona de punción}: el proceso de tensar la zona de
    punción se realiza momentos durante la inserción de la jeringa, el mismo es
    trivial, y para simularlo se requiere que el usuario utilice tres dedos al
    mismo tiempo (dos para tensar y otro para realizar la punción), lo cual
    dificulta la utilización de la solución.

    \emph{Este paso no se simula} por la dificultad técnica que implica utilizar
    tres dedos para realizar una tarea, conjuntamente con la facilidad con que
    se realiza acción. 

\item \textbf{Remover el torniquete}: Al igual que en la colocación del
    torniquete, \emph{se simula el momento} de la extracción por que es
    importante, pero no los detalles de la extracción.

\item \textbf{Solicitar la apertura del puño}: el momento exacto donde se debe
    solicitar al paciente que abra la mano es fundamental para la realización
    correcta de la simulación. \emph{Este paso se simula} a través de un comando
    de voz.

\item \textbf{Extraer la muestra se sangre necesaria}: este es el paso central
    de la práctica, tanto el momento, como la forma es importante simular.

    La extracción de la sangre \emph{se simula} la acción de la extracción, pero
    no detalles como la sangre extraída, la velocidad de extracción y la fuerza
    necesaria por limitaciones de la tecnología.

\item \textbf{Presionar el brazo y extraer la aguja}: la presión del brazo para
    extraer la jeringa es un paso trivial, en cambio el momento en el que se
    extrae la aguja es conocimiento necesario para el procedimiento.

    \emph{No se simula esta acción}, pues es un paso al que los alumnos están
    acostumbrados y de simularse, agrega una complejidad adicional a la
    extracción, que es un paso que se realiza al mismo tiempo.

\item \textbf{Colocar algodón con alcohol en el punto de punción}: este paso es
    importante, tanto el momento en el cual se debe realizar, como la forma de
    realizarlo.

    \emph{Se simula la colocación del algodón}, así como el tiempo que se debe
    presionar el mismo.

\item \textbf{Sellar la muestra y enviarlo a su destinatario}: es necesario que
    los alumnos sepan que este paso debe ser realizado, pero los detalles del
    mismo, no son necesarios para el logro de las competencias básicas,
    \emph{este paso no se simula}.


\item \textbf{Retirar la bata, tapaboca, gorro y guantes}: es necesario que los
    alumnos sepan que deben desechar todos los elementos que fueron utilizados
    durante el proceso, la forma de hacerlo no es necesaria.

    \emph{Se simula el momento en el que se extraen los elementos} a través de
    una opción en la interfaz.

\item \textbf{Asepsia de las manos}: la asepsia final de las manos es un paso
    necesario para el procedimiento, así como la asepsia inicial, es importante
    que los alumnos sepan el momento en cual deben realizarlo, \emph{este paso
        se simula} a través de una opción en la interfaz.

\end{itemize}

Estas hipótesis afectan directamente el desarrollo de la aplicación, dictando
que partes del procedimiento se simulan y como, pueden ser vistas como
requisitos funcionales de la solución.

\subsection{Evaluación de glasgow}
\label{sec:glasgow_hipotesis}

En la sección~\ref{sec:glasgow_protocolo} se definieron los pasos necesarios
para poder llevar a cabo el procedimiento, este procedimiento es un paso
rutinario que deben realizar los profesionales para poder determinar rápidamente
el estado de conciencia de un individuo. 

El paso central de la práctica es la valoración del paciente y la generación del
diagnostico, los demás pasos, no colaboran en el desarrollo de la competencia
básica y según opiniones de los profesores no son importantes.

Así, es suficiente con simular al paciente y las reacciones que tiene ante las
acciones del jugador, las siguientes hipótesis se basan en la interacción
paciente/jugador.

El último paso del procedimiento es el registro final de la puntuación, el mismo
es utilizado como mecanismo de evaluación y el registro en sí no se simula, es
decir, se solicita al jugador que realice el diagnostico (mediante un menú),
pero no en la condiciones que se utilizan en la vida real (anotando en el
registro médico del paciente).

Para simular la medición del estado del paciente, se toman las siguientes
hipótesis:

\begin{itemize}
    \item La provocación de un estimulo doloroso al paciente es una accion
        necesaria, pero no así los detalles de la misma, \emph{se simula el
            estimulo con una opción} al personar al paciente en alguna
        extremidad.
    \item El dialogo jugador/paciente se realiza a través de comandos de voz, el
        nombre del paciente es una información que se conoce de ante mano y
        \emph{Se simulan 7 posibles preguntas}, que incluyen solicitudes de
        apertura ocular, movimiento de extremidades, y preguntas generales como
        el nombre del paciente, el día y el lugar.
\end{itemize}



%! TEX root = ../main.tex

\section{Solución}
\label{sec:solucion}

\observacion{Ver donde pone la interacción con la cámara}

Se describe la arquitectura propuesta para la realización de una juego serio, se
utiliza la guía básica definida por~\cite{pereira2009design} y descrita
en~\ref{sec:desarrollo}.

Esta sección se enfoca en los aspectos técnicos de la creación del juego serio,
las competencias básicas relacionadas con la educación (segundo paso de la guía
descrita en~\ref{sec:desarrollo}) se define en las
secciones~\ref{sec:glasgow} y~\ref{sec:hemocultivo}.

\subsection{Partes de la simulación}

La simulación se compone de tres elementos principales, entidades (que son
objetos de la vida real), acciones (que son provocadas por las entidades) y
eventos (que son el resultado de una acción). 

Existen otros elementos dentro de la simulación, como la sala y la iluminación,
los mismos son importantes para crear un entorno similar a la realidad y son
estáticos, es decir no interactúan con el usuario más que para limitar la
exploración en el escenario y/o resaltar aspectos relevantes.

\subsubsection{Entidades}

Cualquier objeto o componente en el sistema que requiera la representación
explícita en el modelo\cite{banks2000dm}. Las entidades tienen atributos. Los
atributos son las características de una determinada entidad que son exclusivos
de esa entidad.

Una entidad tiene en todo momento, un estado y una lista de acciones que
puede realizar, esta lista de acciones esta definida por el estado del mismo,
las condiciones en la que se encuentra el entorno y la práctica actual.

La entidad \enquote{Enfermero} es la que es controlada por el usuario, a través
de la interacción con la interfaz gráfica.

\subsubsection{Acciones}

Las entidades se comunican a través de acciones, las cuales pueden tener
diversos orígenes, siempre una entidad inicia una acción. Las acciones provocan
cambios en el ambiente y provocan eventos. Las acciones no solo las
realiza el usuario, sino cualquier entidad.

Como ejemplo, una acción es esterilizar las manos, esta acción provoca un
cambio en el ambiente (las manos ahora son estériles) y fue realizada por la
interacción entre el usuario y la interfaz gráfica.

\subsubsection{Eventos}

Los eventos son ocurrencias instantáneas que cambia el estado de un
sistema\cite{banks2000dm}, cada acción que se realiza provoca una acción, y los
eventos son la mecanismo que tiene una entidad para ser notificada de las
acciones de otras entidades.

\subsubsection{Interacción con el entorno}

El usuario se desenvuelve en un entorno de tres dimensiones, en el cual realiza las
actividades relacionadas a la práctica, se distinguen dos tipos de movimientos
principales que el usuario puede realizar:

\begin{itemize}
    \item \textbf{Alejamiento o acercamiento}: es el acto de acercar o alejar la
        cámara, y por consiguiente al usuario del paciente. Se realiza
        utilizando dos dedos, para realizar un acercamiento, mientras se
        mantiene presionada la pantalla con ambos dedos, se procede a alejar un
        dedo del otro, para realizar un alejamiento, se debe acercar ambos
        dedos.
    \item \textbf{Rotación}: se refiere al movimiento de rotación al rededor de
        un foco, que en ambas escenas es el paciente, para realizara, se utiliza
        un dedo, y se mueve en dedo en cualquier dirección, la cámara, se moverá
        en la dirección contraria.
\end{itemize}

\subsection{Grafo de estados}

La solución tiene varias escenarios, y dentro de cada escenario, existen varias
pantallas que muestran información relevante de acuerdo a la situación de la
simulación, en~\ref{fig:grafo_estados} se observa la interacción entre las
diferentes pantallas y escenarios.

\begin{figure}[H] 
\centering 
\includegraphics[scale=0.5]{propuesta/grafo_escenas.png}
\caption{Navegación entre escenarios y pantallas. Los escenarios son los
    rectángulos con un borde dos rayas, y las pantallas tienen un borde con una
    sola raya.}
\label{fig:grafo_estados}
\end{figure}

La solución inicia con un escenario denominado \emph{Inicio}, en el cual se
permite al usuario observar los detalles del entorno simulado a la vez que
muestra las opciones que permiten iniciar las diferentes prácticas, compartir
su actividad, enviar los datos de utilización y finalmente salir de la
simulación.

Si el usuario selecciona en el \emph{inicio} la opción \emph{Extracción de
    sangre}, se inicia el escenario denominado \emph{Extracción de sangre}, en
el cual el usuario puede realizar el procedimiento de extracción de sangre, si
el usuario selecciona la opción \emph{Fin}, la simulación termina y se dirige a
el escenario \emph{Pantalla de resultados}.

Al seleccionar la opción \emph{Evaluación Glasgow}, se inicia el escenario
denominado \emph{Glasgow}, donde el usuario debe evaluar a un paciente en el
centro del escenario, si el usuario presiona la opción \emph{Fin} se inicia la
pantalla denominada \emph{Evaluar al paciente}, donde el usuario diagnostica el
estado del paciente, y finalmente al presionar el botón \emph{Fin}, la
simulación finaliza y se inicia el escenario \emph{Pantalla de resultados}.

La opción \emph{Exploración Glasgow} es similar, la diferencia es que antes de
iniciar el escenario \emph{Glasgow}, aparece la pantalla \emph{Elegir estado de
    paciente}, en el cual el usuario selecciona un estado para que el paciente
actué de acuerdo al mismo, luego se inicia la escena \emph{Glasgow} y si el
usuario presiona el botón \emph{Fin}, se inicia el escenario \emph{Pantalla de
    resultados}.

La pantalla de resultados muestra la información acerca de las acciones que
realizo el usuario, proveyendo información a modo de retroalimentación, en esta
pantalla el usuario puede compartir sus resultados por las redes sociales,
reiniciar el escenario y finalmente, poder volver a la \emph{Pantalla de
    inicio}.


\subsection{Inicio}

\subsubsection{Descripción del entorno}

La escena mostrada como pantalla de inicio de la aplicación muestra como fondo la sala de 
un hospital con los elementos típicos de estos lugares, esta es la que se utiliza como 
escenografía principal en los escenas de los procedimientos, haciendo que el usuario entre 
en ambiente. Además de este fondo, se muestras varias opciones en forma de botones que serán 
descriptas a continuación y un mensaje en donde se recomienda al usuario el uso de auriculares.

\subsubsection{Opciones}

Las opciones disponibles en la pantalla de inicio son presentadas en forma de
botones los cuales tienen una breve descripción que identifica la función que
cumplen. 

\todox{Agregar descripción del escenario y si es necesario pantalla donde se
    pone el número}

\begin{itemize}
\item Botón \enquote{Enviar Progreso}: esta función envía toda la información
    acerca de la actividad que el usuario realizo en la aplicación a un servidor
    backend que se encarga de almacenar estos datos.
\item Botón \enquote{Salir de la simulación}: esta función permite salir de la
    aplicación.
\item Botón \enquote{Facebook}: esta función permite al usuario ingresar a su
    cuenta de Facebook.
\item Botón \enquote{Extracción de sangre}: esta función permite ingresar a la
    escena correspondiente al procedimiento de extracción de muestras de sangre
    permitiendo al usuario jugar una nueva partida.
\item Botón \enquote{Explorar Glasgow}: esta función permite ingresar a la
    escena correspondiente al procedimiento para explorar las reacción de un
    paciente con un diagnostico especifico de la escala de Glasgow permitiendo
    al usuario jugar una nueva partida.
\item Botón \enquote{Evaluar Glasgow}: esta función permite ingresar a la escena
    correspondiente al procedimiento para la valoración y diagnostico de la
    escala de Glasgow para un paciente con estado aleatorio permitiendo al
    usuario jugar una nueva partida.
\end{itemize}


\subsection{Extracción de muestras de sangre}

A continuación se detallan cada una de las opciones y formas disponibles de
interactuar con la escena del procedimiento de extracción de muestras de sangre.

\subsubsection{Descripción del entorno}

Al seleccionar el procedimiento de extracción de sangre en la pantalla de inicio 
la aplicación inmediatamente muestra la escena del procedimiento, se muestra una 
sala de hospital igual a la de la pantalla de inicio pero con un paciente en una 
de las camas, a este paciente es a quien se le realizara el procedimiento.

La posición inicial de la cámara se ubica en un ángulo en donde se puedan ver 
bien los brazos del paciente para facilitar al usuario la realización del 
procedimiento.

\todox{Agregar descripción}

\subsubsection{Descripción de la interfaz}

La interfaz principal de este escenario posee dos menús, uno a cada lado de la
pantalla, las opciones son representadas como botones que poseen una imagen
intuitiva\todox{Ver si no hay que agregar esto como hipótesis} que representa la
función que realizan. 

\subsubsection{Entidades}

En la extracción de sangre existen dos entidades principales, el paciente y el
usuario, cada entidad mantiene un estado independiente de la otra entidad.

El paciente es una entidad con estado complejo, el cual es constantemente
modificado por las acciones del usuario, en resumen, la información que contiene
el estado del paciente es:

\begin{itemize}
    \item \textbf{Jeringas}: un paciente puede tener cero o más jeringas en
        cualquier momento, no se limita la cantidad de jeringas que puede
        insertar el usuario.
    \item \textbf{Manos}: almacena el estado de las manos, el paciente reacciona
        ante peticiones del usuario, puede abrir o cerrar cualquier mano en
        cualquier momento.
    \item \textbf{Torniquetes}: es el conjunto de torniquetes que tiene
        actualmente el paciente, notar que los torniquetes pueden ser colocados
        en cualquier parte del brazo, pero existen lugares \enquote{correctos} y
        lugares \enquote{incorrectos}, la diferencia consiste en la distancia a
        los puntos de extracción, estos lugares están predefinidos.
    \item \textbf{Zonas esterilizadas}: son aquellas áreas del cuerpo que el
        usuario esterilizó, no existe un límite para las zonas esterilizadas.
        Una vez que una jeringa es extraída, una zona esterilizada pasa a estar
        contaminada y a la espera de que el usuario la presione.
    \item \textbf{Zonas presionadas}: son aquellas zonas que, una vez
        contaminadas por la extracción de una jeringa, han sido presionadas por
        el usuario.
    \item \textbf{Contaminado}: define si alguna acción realizada por el usuario
        provoco que el paciente se contamine, existen varias cadenas de eventos
        que pueden provocar que esto ocurra:
        \begin{itemize}
            \item Inyección de una jeringa cuando existe otra inyectada.
            \item Inyección en un lugar en lugares inadecuados.
            \item Inyección en un lugar no esterilizado.
            \item Inyección en un brazo cuya mano este abierta.
            \item Inyección fuera del alcance de los torniquetes actuales.
            \item Interacción con el paciente sin que el mismo tenga la mano
                estéril.
        \end{itemize}
        Es importante notar que este estado no es afectado directamente por una
        acción del usuario, sino por la consecuencia de una acción.
\end{itemize}

El \emph{usuario o enfermero} mantiene un estado en todo momento del cual
dependen sus acciones, por ejemplo, si la mano del paciente no esta
esterilizada, cualquier interacción con el paciente provocara que el paciente se
contamine.

\begin{itemize}
    \item \textbf{Manos}: almacena la información acerca de la esterilidad de
        las manos.
    \item \textbf{Guantes, gorro, bata y tapaboca}: almacenan la información
        acerca de los equipamientos que tiene el usuario en un momento dado.
    \item \textbf{Elemento actual}: es el elemento que esta activo en
        cualquier momento, un elemento es una herramienta de la vida real,
        como por ejemplo un torniquete, una gaza.
\end{itemize}

\subsubsection{Acciones}


\paragraph{Comando de voz}

Para representar la interacción del usuario con el paciente usando la voz se
implemento un menú que es activado y mostrado en pantalla cuando el usuario
habla, este menú muestra una seria de ordenes que el usuario le haría al
paciente normalmente hablándole. Las opciones de menú se detalla a continuación:

\begin{itemize}
\item Explicar procedimiento: esta función sirve para detectar si el usuario
    realizo la acción de explicar al paciente acerca del procedimiento. 
\item Abrir la mano izquierda: esta función le indica al paciente que abra su
    mano izquierda, como resultado el paciente realiza esta acción.
\item Cerrar la mano izquierda: esta función le indica al paciente que cierre su
    mano izquierda, como resultado el paciente realiza esta acción.
\item Abrir la mano derecha: esta función le indica al paciente que abra su mano
    derecha, como resultado el paciente realiza esta acción.
\item Cerrar la mano derecha: esta función le indica al paciente que cierre su
    mano derecha, como resultado el paciente realiza esta acción.
\end{itemize}

\paragraph{Opciones}

En este menú se despliegan los botones que representan las opciones de
bioseguridad. Es decir, acciones como lavarse las manos, calzarse guantes,
ponerse gorro, ponerse bata y ponerse tapaboca.

Los elementos de bioseguridad que actualmente tiene puesto el usuario se
representan como se describió anteriormente y se muestran en la parte baja de la
pantalla. Desaparece quitarse que en ese caso se representa al volver a
seleccionar la misma opción.

\paragraph{Elementos}

En este menú se despliegan los botones que representan a lo elementos que se
utilizan para realizar el procedimiento, una vez presionado ese elemento queda
seleccionado. Solo un elemento puede ser seleccionado a la vez. Si el mismo
botón se vuelve a presionar inmediatamente después de haber sido presionado, el
elemento queda de-seleccionado.

%Estas opciones van cambiando el estado del jugador y pueden ser seleccionados
%mas de una opción a la vez además de permitir de-seleccionar una opción
%volviendo a tocar el botón correspondiente. También posee la opción de
%finalizar la partida la cual manda al usuario a la pantalla de resultados.
%% REVISAR ESTO , el comienzo es sobre opciones y el final sobre elementos %%

La herramienta seleccionada actualmente para realizar el procedimiento se se
muestra en la forma descripta anteriormente arriba de la pantalla principal del
procedimiento. Esta imagen representa lo que actualmente tiene en las manos el
jugador. Desaparece al de-seleccionar o terminar de usar la herramienta.

\todox{Agregar colocación}
\todox{Agregar utilización}

\subsubsection{Eventos}
\subsubsection{Motor de reglas}
\subsubsection{Registro de actividad}


\subsection{Valoración de la escala de Glasgow}

\subsubsection{Descripción del escenario}

La interfaz principal de este escenario posee un botón de finalización de
partida al costado con una imagen intuitiva que representa la función que
realiza. Este botón manda al usuario a la pantalla de resultados.

\subsubsection{Entidades}
\subsubsection{Acciones} 
\subsubsection{Eventos} 
\subsubsection{Pantalla de diagnostico}
\subsubsection{Registro de actividad}
\paragraph{Elementos y opciones}


\subsection{Pantalla de resultados}
\subsubsection{Descripción del escenario}
\subsubsection{Retroalimentacion}
\subsubsection{Gamificacion}
\subsubsection{Reinicio}
\subsubsection{Puntuación}
\subsubsection{Tiempo utilizado}
\subsubsection{Facebook 2}

\subsection{Partes de la simulación}
    \subsubsection{Entidades}
    \subsubsection{Eventos}
    \subsubsection{Acciones}
    \subsubsection{Interacción con la cámara}

\subsection{Grafo del desarrollo}
% podemos poner acá un gráfico mas o menos así (ver graphviz)
%           /---> Hemocultivo --\
%          /                     \              /-> Reiniciar
%  Inicio ------> Glasgow 1 -------> Resultados --> Inicio
%         \ \                    /              \-> Facebook 2
%          \ \--> Glasgow 2 ----/
%           \---> Salir 
%            \--> Facebook 1
%             \-> Enviar resultados

\subsection{Pantalla de inicio}
    \subsubsection{Descripción del escenario}
    \subsubsection{Enviar datos}
    \subsubsection{Glasgow}
    \subsubsection{Extracción de sangre}
    \subsubsection{Facebook 1}

\subsection{Extracción de sangre}
    \subsubsection{Descripción del escenario}
    \subsubsection{Descripción de la interfaz}
    \subsubsection{Entidades} % definimos cuales son las entidades
        %\subsubsubsection{Estado del enfermero}
        %\subsubsubsection{Objeto seleccionado}
    \subsubsection{Acciones} % definimos cuales son las acciones de esas entidades
        %\subsubsubsection{Comandos de voz}
        %\subsubsubsection{Opciones} %bata,mano,guante,y eso
        %\subsubsubsection{Elementos}
            %\subsubsubsubsection{Colocación}
            %\subsubsubsubsection{Utilización}
    \subsubsection{Eventos} % definimos cuales son los eventos que se lanzan en este proceso
    \subsubsection{Motor de reglas} % se define como funciona el motor de reglas acá
    \subsubsection{Registro de actividad} % se define como se registra las acciones del usuario (cuales)


\subsection{Glasgow 1 y 2}
    \subsubsection{Descripción del escenario}
    \subsubsection{Entidades} % definimos cuales son las entidades
        %\subsubsubsection{Reacciones del paciente}
    \subsubsection{Acciones} % definimos cuales son las acciones de esas entidades
        %\subsubsubsection{Acciones sobre el paciente}
        %\subsubsubsection{Comandos de voz}
    \subsubsection{Eventos} % definimos cuales son los eventos que se lanzan en este proceso
    \subsubsection{Pantalla de diagnostico}
    \subsubsection{Registro de actividad} % se define como se registra las acciones del usuario (cuales)
    
\subsection{Pantalla de resultados}
    \subsubsection{Descripción del escenario}
    \subsubsection{Retroalimentacion}
    \subsubsection{Gamificacion}
    \subsubsection{Reinicio}
    \subsubsection{Puntuación}
    \subsubsection{Tiempo utilizado}
    \subsubsection{Facebook 2}

%%! TEX root = ../main.tex
\section{Evaluación en tiempo de ejecución}

Las acciones realizadas por los usuarios dentro de la aplicación son evaluadas
para determinar si realizo o no el procedimiento de manera correcta y así
brindarle información al usuario sobre su rendimiento.

En esta sección se explica como son evaluados las acciones de los usuarios para
los diferentes procedimientos simulados.

\subsection{Extracción de muestras de sangre}

Para la evaluación de las acciones del usuario en este procedimiento se utilizo
un motor de reglas denominado \enquote{Acciones condicionadas por eventos}. A
continuación se explica en detalle cada aspecto relacionado tanto al motor como
a la forma de evaluación del rendimiento del usuario.

\subsubsection{Acciones condicionadas por eventos}

Un evento es la ocurrencia de un hecho en particular, y son identificados por un
nombre y un conjunto de parámetros, por ejemplo, cuando un evento es cuando el
enfermero inserta una Jeringa, el nombre de este evento es
\enquote{jeringa}.inserted, y sus parámetros podrían ser el lugar y el tiempo
de la inserción, así, la influencia del estudiante en la simulación es una
sucesión de eventos.

Por cada acción que realiza el usuario dentro de la simulación, existe un evento
relacionado, por consiguiente, es razonable estudiar algunos eventos para
determinar si los pasos realizados corresponden con los deseados. 

Para determinar si una sucesión de eventos es la correcta, se definen reglas,
una regla es una asociación de una condición y una acción, la condición define
si el entorno es el adecuado para realizar una acción, la cual es un
procedimiento que realiza la lógica deseada.

Las \gls{eca} son aquellas que son activadas una vez que se cumplen determinados
eventos\cite{bailey2004event}. En las bases de datos relacionales, son conocidos
como triggers, es decir, una base de datos relacional (u orientada a objetos) es
un motor de reglas \gls{eca}\cite{bailey2004event}\cite{behrends2006combining}.

Las mismas pueden ser utilizadas para notificar que un determinado conjunto de
eventos ha ocurrido\cite{bailey2004event}, así como servir para almacenar
información acerca de la utilización de un determinado recurso.


\paragraph{Motivación}

Las reglas del tipo \gls{eca} permiten reaccionar a determinados eventos, en
forma de una única regla, la cual facilita la declaración de las
mismas\cite{bailey2004event}.

Son principalmente útiles para analizar el comportamiento en tiempo real de un
sistema en una forma
reactiva\cite{bailey2004event}\cite{de2001eca}\cite{bailey2002analysis}, esta
característica esta impulsada principalmente por que son ejecutadas después de
la ocurrencia de un evento, y el entorno no es modificado, pudiendo así acceder
al mismo entorno que el qué lanzo el evento.

Definir si las acciones de un usuario son correctas utilizando un motor
\gls{eca} es sencillo desde el punto de vista que sólo se deben definir un
conjunto de acciones que se deben realizar, y agregar una acción que verifica si
los pasos realizados fueron los correctos.

\paragraph{Declaración}

Una \gls{eca}, se define como\cite{bailey2004event}\cite{behrends2006combining}:

\begin{center}
	 Cuando ocurren una serie de \emph{eventos}, y se cumple una
	 \emph{condición}, entonces realizar una \emph{Acción}.
\end{center}

Los \emph{eventos} determinan cuando una regla debe ser activada, los mismos se
dividen en dos categorías\cite{behrends2006combining}, primitivos y compuestos,
los primeros son detectables, por ejemplo, cuando se inserta una jeringa, y los
compuestos, son la combinación de uno o más
primitivos\cite{bailey2004event}\cite{behrends2006combining}. Los eventos
compuestos, se unen mediante:
\begin{enumerate*}[label=\itshape\alph*\upshape)]
\item conjunción (\emph{y}),
\item disyunción (\emph{o}), y
\item secuencia (\emph{entonces}).
\end{enumerate*}
Sin embargo, no siempre son necesarios todas las posibles combinaciones, y las
combinaciones sencillas son más fáciles de optimizar y
probar\cite{bailey2004event}.

La \emph{condición} de una regla determina si el entorno es el necesario para que la
regla sea activada, en esta condición el entorno que lanzó el evento esta
disponible.

La \emph{acción} a ejecutar describe la lógica que debe ser ejecutada cuando se han
lanzado los eventos y la condición de la regla se ha cumplido.

\paragraph{Dependencia entre reglas}

Las reglas pueden depender de otras reglas, lo cual se puede ver como que la
finalización de una regla es un evento que otra regla espera para poder ser
activada.

Las reglas pueden agregar información a un contexto compartido por todas las
reglas, de esta manera, se puede pasar parámetros entre distintas reglas, por
ejemplo, la regla \emph{Retirar Torniquete}, depende de la regla \emph{Insertar
Torniquete}, pero debe responder solamente al torniquete que ha activado
la regla de inserción, es decir, el usuario puede extraer varios torniquetes, y
la regla no debe activarse, hasta que se extraiga el torniquete que activo la
primer regla.

Así, la regla \emph{Retirar Torniquete} depende de la regla \emph{Insertar
Torniquete}, y esta relación entre reglas, se da en dos
formas\cite{bailey2004event}:

\begin{itemize}
\item  \emph{Dependencia fuerte:} la regla \emph{Retirar Torniquete} solamente podrá
	ser elegida para ser lanzada cuando la regla \emph{Insertar Torniquete}
	haya sido cumplida.
\item  \emph{Dependencia de contexto}: la regla \emph{Retirar Torniquete} no se
	activará cuando los eventos a los que escucha se terminen, sino cuando
	los eventos a los que escucha sean lanzados con los parámetros adecuados
	(se extraiga el torniquete que lanzo la regla de inserción).
\end{itemize}

\paragraph{Representación}

La definición de las reglas se realiza de la siguiente forma;
\begin{algorithm}[H]
\caption{Creación de regla de verificación de calzado de guantes}
\label{alg:rule:guante}
\lstset{style=sharpc}
\begin{lstlisting}
Rule.New("Regla de verificacion de calzado de guantes").
     When("enfermero.guantes.calzar").
     Then(e => e.Patient.ManosLimpias()).
\end{lstlisting}
\end{algorithm}
%TODO agregar indice de algoritmos

La regla anterior controla que el estudiante ha realizado la acción ``Calzarse
los guantes'', y en ese momento tenga las manos limpias, la variable \emph{e},
es el entorno, y a través de la propiedad \emph{Patient} obtiene el estado del
paciente en ese momento.

\paragraph{Modelo de ejecución}

Para ejecutar un motor de reglas del tipo \gls{eca}, se debe tener en cuenta
principalmente dos factores, 
\begin{enumerate*}[label=\itshape\alph*\upshape)]
\item  Como se verifica el cumplimiento de cada regla, y, 
\item  Que ocurre cuando varias reglas son lanzadas al mismo tiempo
\end{enumerate*}.

Para ambos casos se puede tomar un enfoque \emph{inmediato}, es decir que
inmediatamente cuando se lanza un evento, o se cumple una condición, se ejecuta
la regla. Además existen otros dos modos de ejecución, \emph{deferida}, y
\emph{desacoplada}, en la primera, se espera hasta que el lanzador del evento
culmine su trabajo, y luego se ejecuta la regla, pero en la misma unidad de
trabajo, mientras que en la ejecución desacoplada, se encolan los trabajos y
otro hilo es el encargado de ejecutar las reglas. Estos modos están inspirados
en las bases de datos relacionales, el deferido se ejecuta en la misma
transacción, y el desacoplado, inmediatamente después de que la transacción
termine\cite{bailey2004event}.

La propuesta implementada, utiliza una ejecución inmediata, principalmente por
la sencillez de las reglas, es decir, las reglas no realizar un proceso complejo,
solamente controlan el estado del entorno y lo validan.

Además, la ejecución inmediata es importante por que el entorno no sufre
modificaciones entre el evento lanzado y la ejecución de la regla, según
\cite{bailey2004event}, este es el factor más importante para determinar el tipo
de ejecución deseado.



\paragraph{Estados de una regla}

Una regla puede estar en uno de los siguientes estados:

\begin{description}
\item[BEGIN] Es una regla que recién fue creada, no realiza ninguna
	acción.
\item[WAITING\_FOR\_RULE] Es un estado en el que esta esperando que otras reglas
	sean lanzadas. En este estado, es un suscriptor de las reglas por la que
	espera, y no forma parte del ciclo de ejecución del motor de reglas.
\item[WAITING\_FOR\_EVENT] Es un estado en el que esta escuchando a que sean
	lanzados los eventos a los que escucha, este es el estado principal. En
	este estado, es un suscriptor de los eventos por los que espera, y no
	forma parte del ciclo de ejecución del motor de reglas. Se diferencia
	del estado anterior, en que los eventos escuchados pueden ser lanzados
	por cualquier objeto del entorno, no necesariamente una regla.
\item[WAITING\_FOR\_CONDITION] La regla ya no espera por ningún evento y las
	reglas de las que depende ya han sido lanzadas, se verifica cada cierto
	tiempo si el entorno cumple con una condición definida. 
\item[FINISH] La regla ha sido lanzada, con un resultado no determinado, se pudo
	haber cumplido, como no, es el estado final de una regla. Cuando una
	regla llega a este estado, se lanza su evento de finalización.
\end{description}

Una regla puede estar en solo un estado, y solamente se permite que el estado
avance, desde \emph{BEGIN} hasta \emph{FINISH}.


\paragraph{Ciclo de vida}

Cuando una regla es definida, y insertada al motor de reglas, inmediatamente
pasa al estado \emph{BEGIN}, luego se verifica si la misma depende de otras
reglas, sí este es el caso, pasa al estado \emph{WAITING\_FOR\_RULE} y escucha a
los eventos de finalización de las reglas anteriores.

Una vez que las reglas anteriores han sido finalizadas, la regla pasa al estado
\emph{WAITING\_FOR\_EVENT} sí deben escuchar por algún evento, en caso contrario
pasan al estado \emph{WAITING\_FOR\_CONDITION}.

Una vez que la regla está en estado \emph{WAITING\_FOR\_CONDITION}, pasa a un
motor que ejecuta su condición cada cierto tiempo, si la condición se cumple, la
regla se ejecuta, y la misma pasa a estado \emph{FINISH}, momento en el cual
notifica a las reglas que dependen de ella que ha sido lanzada.

Una vez que la regla esta en estado \emph{FINISH}, la misma sale del esquema de
ejecución, y solo esta disponible para obtener resultados.

Según el ejemplo de la regla definida en el código\ref{alg:rule:guante}, la
regla al terminar de ser construida pasa a estado \emph{BEGIN}, al no depender
de otras reglas, pasa inmediatamente al estado \emph{WAITING\_FOR\_EVENT},
cuando es lanzado el evento, la regla ejecuta la acción y pasa al estado
\emph{FINISH}.

\paragraph{Motor de ejecución}

Un motor de reglas \gls{eca}, requiere de un proceso que evalúe constantemente
las reglas para verificar si las mismas deben ser lanzadas o
no\cite{bailey2004event}\cite{galton2002two}, este motor puede utilizar el
algoritmo de RETE\cite{de2001eca} para realizar esta verificación, en la
propuesta presentada, la cantidad de reglas definidas, y la no dependencia
circular entre ellas, hace innecesario la implementación de tal
algoritmo\cite{de2001eca}. 

El motor de reglas actúa sobre aquellas reglas en estado
\emph{WAITING\_FOR\_CONDITION} e invoca al procedimiento que se encarga de
validar si la regla puede ser activada (el procedimiento es único por cada
regla), si el mismo determina que la regla puede ser lanzada, el motor ejecuta
la acción de la regla y modifica el estado de la regla a \emph{FINISH}.


\subsubsection{Definición de reglas}

La reglas del procedimiento de extracción de sangre fueron definidas de acuerdo
a los pasos requeridos según el protocolo del procedimiento y al orden en el que
son requeridos. Es decir, cada paso del protocolo tiene asociado una regla
dentro del motor que lo representa y las condiciones asociadas a cada regla
están determinadas por el orden en que deben realizarse dentro del protocolo.

Cada regla tiene una o mas condiciones que deben ser cumplidas para que un paso
del protocolo realizado se considere correcto.

\subsubsection{Retroalimentación y puntuación final}
\label{sec:puntuacion_hemocultivo}

Cada regla tiene asociado un peso, de acuerdo a la dificultad de realizar el
paso, este peso es utilizado al final de la partida para darle una puntuación al
usuario acerca de su rendimiento en la partida.

Además, un regla puede quedar en uno de diferentes estados al final de la
partida como se mostró anteriormente, cada uno de esos estados posee un
significado en el contexto del procedimiento y por lo tanto tiene información
asociada para que al final de la partida se muestre una retroalimentación
correcta al usuario por paso.

\subsection{Valoración de la escala de Glasgow}
\label{sec:puntuacion_glasgow}

Para la evaluación del rendimiento del usuario en el momento de llevar a cabo el
procedimiento de valoración de la escala de Glasgow se tuvo un enfoque
completamente diferente al del procedimiento de extracción de muestras de sangre
debido a la naturaleza propia del procedimiento. 

Como se explico anteriormente, el paciente puede estar en ciertos estados
específicos dentro de la escala, y además dentro de cada estado reacciona de un
forma en particular por lo tanto, al inicio de la partida un componente interno
de la aplicación selecciona de forma aleatoria un estado para el paciente, de
forma tal que cada vez que una partida sea jugada no se repitan los estados de
forma seguida.

El estado aleatorio del paciente es guardado en una variable que no es
modificada hasta que se reinicie la partida. Al final de la partida, la
aplicación pide al usuario que valore el estado del paciente que le fue
presentado, una vez que el usuario confirme su respuesta la aplicación la
compara con el estado guardado y de esta forma puede informar al usuario acerca
de su rendimiento en el diagnostico.

Además, cada posible respuesta dada por el usuario contiene información
relacionada al contexto del procedimiento y a la situación actual presentada la
cual es utilizada como retroalimentación al final de la partida. La puntuación
final dada depende de la cantidad de valoraciones correctas dadas por el usuario
para la respuesta verbal, motora, ocular y nivel de gravedad del paciente.










\section{Inconvenientes de diseño}

Los mayores inconvenientes de diseño de la aplicación se dieron en el momento de
validar tanto el contenido de la aplicación como la interfaz de usuario, para
sobrellevar estos inconvenientes fueron requeridos la intervención de terceros.

A continuación se explica en detalle cada uno.

\subsection{Interfaz de usuario}

Como parte del diseño y desarrollo de la aplicación propuesta como solución se
realizó una prueba de interfaz de usuario con alumnos de la carrera de
ingeniería en informática de la Facultad Politécnica de la Universidad Nacional
de Asunción, estas pruebas fueron realizadas con personas que están
acostumbradas al uso de interfaces similares y que de hecho pueden ser mas
criticas a la hora de evaluarlas. Esta prueba se explica en detalle en el
capítulo~\ref{chap:evaluacion} y los resultados en el capítulo~\ref{chap:analisis}.

Principalmente son dos las cualidades de una interfaz gráfica que se pueden
someter a prueba: su funcionalidad y su usabilidad. Con la primera se pretende
responder preguntas como ¿Se puede usar cierta función?, ¿Funciona como se
espera?, o ¿Es correcta?; y con la segunda, a ¿Puede el usuario utilizar
fácilmente la función?, o ¿Su uso es intuitivo y fácil de
aprender?\cite{fragaverificacion}.

Las pruebas de interfaces de usuario ayudan a que los usuarios puedan
concentrarse mas en el problema en vez de poner los esfuerzos en recordar todas
las opciones que ofrece la aplicación que se utiliza para resolver el
problema\cite{horowitz1993graphical}.

Luego de las pruebas con usuarios con experiencia en interfaces móviles, se
hicieron correcciones a los problemas encontrados en la interfaz, los mayores
inconvenientes fueron con respecto a la usabilidad y la interacción tanto con el
entorno como con los objetos dentro de la simulación Estas correcciones, como
paso posterior, fueron probadas por profesores de la carrera de enfermería del
Instituto Andrés Barbero los cuales dieron su visto bueno.

Otra de las razones por las cual la prueba fue realizada con alumnos que no
formaban parte de la población a la que iba dirigida la aplicación, es la poco
disponibilidad de tiempo con la que cuentan los alumnos de enfermería y mas aun
los profesionales que están encargados de su aprendizaje.

\subsection{Validaciones de contenido}

Llamamos validación de la simulación o la aplicación desarrollada al hecho de
que el contenido de la misma sea correcto y además que la forma de realizar o
representar dicho procedimiento este acorde al mismo. Este tipo de validaciones
fueron realizadas reiteradamente en reuniones con distintos profesores de la
carrera de enfermería del Instituto Andrés Barbero.

Cada corrección solicitada fue evaluada y aprobada posteriormente por los
mismos. Como validación final la aplicación fue presentada en totalidad frente a
un plenario de cuatro profesores del instituto.

El mayor inconveniente en cuanto a las validaciones fueron la forma de
representación tanto de la información como de la simulación de objetos.

\section{Backend de la solución}

\subsection{Registro de usuarios}
\subsection{Registro de eventos}

