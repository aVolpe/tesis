%! TEX root = ../main.tex

\section{Solución}
\label{sec:solucion}

\observacion{Ver donde pone la interacción con la cámara}

Se describe la arquitectura propuesta para la realización de una juego serio, se
utiliza la guía básica definida por~\cite{pereira2009design} y descrita
en~\ref{sec:desarrollo}.

Esta sección se enfoca en los aspectos técnicos de la creación del juego serio,
las competencias básicas relacionadas con la educación (segundo paso de la guía
descrita en~\ref{sec:desarrollo}) se define en las
secciones~\ref{sec:glasgow} y~\ref{sec:hemocultivo}.

\subsection{Partes de la simulación}

La simulación se compone de tres elementos principales, entidades (que son
objetos de la vida real), acciones (que son provocadas por las entidades) y
eventos (que son el resultado de una acción). 

Existen otros elementos dentro de la simulación, como la sala y la iluminación,
los mismos son importantes para crear un entorno similar a la realidad y son
estáticos, es decir no interactúan con el usuario más que para limitar la
exploración en el escenario y/o resaltar aspectos relevantes.

\subsubsection{Entidades}

Cualquier objeto o componente en el sistema que requiera la representación
explícita en el modelo\cite{banks2000dm}. Las entidades tienen atributos. Los
atributos son las características de una determinada entidad que son exclusivos
de esa entidad.

Una entidad tiene en todo momento, un estado y una lista de acciones que
puede realizar, esta lista de acciones esta definida por el estado del mismo,
las condiciones en la que se encuentra el entorno y la práctica actual.

La entidad \enquote{Enfermero} es la que es controlada por el usuario, a través
de la interacción con la interfaz gráfica.

\subsubsection{Acciones}

Las entidades se comunican a través de acciones, las cuales pueden tener
diversos orígenes, siempre una entidad inicia una acción. Las acciones provocan
cambios en el ambiente y provocan eventos. Las acciones no solo las
realiza el usuario, sino cualquier entidad.

Como ejemplo, una acción es esterilizar las manos, esta acción provoca un
cambio en el ambiente (las manos ahora son estériles) y fue realizada por la
interacción entre el usuario y la interfaz gráfica.

\subsubsection{Eventos}

Los eventos son ocurrencias instantáneas que cambia el estado de un
sistema\cite{banks2000dm}, cada acción que se realiza provoca una acción, y los
eventos son la mecanismo que tiene una entidad para ser notificada de las
acciones de otras entidades.

Un evento una consecuencia de una acción, por ejemplo, cuando el usuario 
realiza la acción \emph{Limpiar manos}, se crea el evento \emph{Manos Limpias}. 
Cualquier entidad de la simulación puede reaccionar ante diversos eventos, 
por ejemplo, la entidad \emph{Enfemero} reacciona ante el evento \emph{Manos
Limpias}, y cambia su estado interno para reflejar este evento (de ahora en 
más, se considera que las manos del enfemero están limpias).

Los Eventos son útiles para que varias entidades puedan reaccionar ante
una acción, cuando el enfermero extrae la jeringa del paciente, se crea un
evento para notificar este cambio, y el paciente reacciona de diversas maneras,
por ejemplo, desde ese momento, la zona de donde se extrajo la jeringa, pasa a 
estar contaminada, y la lista de jeringas del paciente se reduce. 

Existe otro componente aparte de las entidades que depende de los eventos 
para cambiar su estado, y es el \emph{Motor de Reglas}, este motor escucha
todos los eventos relacionados con las reglas, y se encarga de evaluar, así
es notificado de los cambios en el entorno.

\subsubsection{Interacción con el entorno}

El usuario se desenvuelve en un entorno de tres dimensiones, en el cual realiza las
actividades relacionadas a la práctica, se distinguen dos tipos de movimientos
principales que el usuario puede realizar:

\begin{itemize}
    \item \textbf{Alejamiento o acercamiento}: es el acto de acercar o alejar la
        cámara, y por consiguiente al usuario del paciente. Se realiza
        utilizando dos dedos, para realizar un acercamiento, mientras se
        mantiene presionada la pantalla con ambos dedos, se procede a alejar un
        dedo del otro, para realizar un alejamiento, se debe acercar ambos
        dedos.
    \item \textbf{Rotación}: se refiere al movimiento de rotación al rededor de
        un foco, que en ambas escenas es el paciente, para realizara, se utiliza
        un dedo, y se mueve en dedo en cualquier dirección, la cámara, se moverá
        en la dirección contraria.
\end{itemize}


\subsubsection{Acciones condicionadas por eventos}

Un evento es la ocurrencia de un hecho en particular, y son identificados por un
nombre y un conjunto de parámetros, por ejemplo, cuando un evento es cuando el
enfermero inserta una Jeringa, el nombre de este evento es
\enquote{jeringa.inserted}, y sus parámetros podrían ser el lugar y el tiempo de
la inserción, así, la influencia del usuario en la simulación es una sucesión de
eventos.

Por cada acción que realiza el usuario dentro de la simulación, existe un evento
relacionado, por consiguiente, es razonable estudiar algunos eventos para
determinar si los pasos realizados corresponden con los deseados. 

Para determinar si una sucesión de eventos es la correcta, se definen reglas,
una regla es una asociación de una condición y una acción, la condición define
si el entorno es el adecuado para realizar una acción, la cual es un
procedimiento que realiza la lógica deseada.

Las \gls{eca} son aquellas que son activadas una vez que se cumplen determinados
eventos\cite{bailey2004event}. En las bases de datos relacionales, son conocidos
como triggers, es decir, una base de datos relacional (u orientada a objetos) es
un motor de reglas \gls{eca}\cite{bailey2004event}\cite{behrends2006combining}.

Las mismas pueden ser utilizadas para notificar que un determinado conjunto de
eventos ha ocurrido\cite{bailey2004event}, así como servir para almacenar
información acerca de la utilización de un determinado recurso.


\paragraph{Motivación}

Las reglas del tipo \gls{eca} permiten reaccionar a determinados eventos, en
forma de una única regla, la cual facilita la declaración de las
mismas\cite{bailey2004event}.

Son principalmente útiles para analizar el comportamiento en tiempo real de un
sistema en una forma
reactiva\cite{bailey2004event}\cite{de2001eca}\cite{bailey2002analysis}, esta
característica esta impulsada principalmente por que son ejecutadas después de
la ocurrencia de un evento, y el entorno no es modificado, pudiendo así acceder
al mismo entorno que el qué lanzo el evento.

Definir si las acciones de un usuario son correctas utilizando un motor
\gls{eca} es sencillo desde el punto de vista que sólo se deben definir un
conjunto de acciones que se deben realizar, y agregar una acción que verifica si
los pasos realizados fueron los correctos.

\paragraph{Declaración}

Una \gls{eca}, se define como\cite{bailey2004event}\cite{behrends2006combining}:

\begin{center}
	 Cuando ocurren una serie de \emph{eventos}, y se cumple una
	 \emph{condición}, entonces realizar una \emph{Acción}.
\end{center}

Los \emph{eventos} determinan cuando una regla debe ser activada, los mismos se
dividen en dos categorías\cite{behrends2006combining}, primitivos y compuestos,
los primeros son detectables, por ejemplo, cuando se inserta una jeringa, y los
compuestos, son la combinación de uno o más
primitivos\cite{bailey2004event}\cite{behrends2006combining}. Los eventos
compuestos, se unen mediante:
\begin{enumerate*}[label=\itshape\alph*\upshape)]
\item conjunción (\emph{y}),
\item disyunción (\emph{o}), y
\item secuencia (\emph{entonces}).
\end{enumerate*}
Sin embargo, no siempre son necesarios todas las posibles combinaciones, y las
combinaciones sencillas son más fáciles de optimizar y
probar\cite{bailey2004event}.

La \emph{condición} de una regla determina si el entorno es el necesario para que la
regla sea activada, en esta condición el entorno que lanzó el evento esta
disponible.

La \emph{acción} a ejecutar describe la lógica que debe ser ejecutada cuando se han
lanzado los eventos y la condición de la regla se ha cumplido.

\paragraph{Dependencia entre reglas}

Las reglas pueden depender de otras reglas, lo cual se puede ver como que la
finalización de una regla es un evento que otra regla espera para poder ser
activada.

Las reglas pueden agregar información a un contexto compartido por todas las
reglas, de esta manera, se puede pasar parámetros entre distintas reglas, por
ejemplo, la regla \emph{Retirar Torniquete}, depende de la regla \emph{Insertar
Torniquete}, pero debe responder solamente al torniquete que ha activado
la regla de inserción, es decir, el usuario puede extraer varios torniquetes, y
la regla no debe activarse, hasta que se extraiga el torniquete que activo la
primer regla.

Así, la regla \emph{Retirar Torniquete} depende de la regla \emph{Insertar
Torniquete}, y esta relación entre reglas, se da en dos
formas\cite{bailey2004event}:

\begin{itemize}
\item  \emph{Dependencia fuerte:} la regla \emph{Retirar Torniquete} solamente podrá
	ser elegida para ser lanzada cuando la regla \emph{Insertar Torniquete}
	haya sido cumplida.
\item  \emph{Dependencia de contexto}: la regla \emph{Retirar Torniquete} no se
	activará cuando los eventos a los que escucha se terminen, sino cuando
	los eventos a los que escucha sean lanzados con los parámetros adecuados
	(se extraiga el torniquete que lanzo la regla de inserción).
\end{itemize}

\paragraph{Representación}

La definición de las reglas se realiza de la siguiente forma;
\begin{algorithm}[H]
\caption{Creación de regla de verificación de calzado de guantes}
\label{alg:rule:guante}
\lstset{style=sharpc}
\begin{lstlisting}
Rule.New(``Regla de verificacion de calzado de guantes'').
     When(``enfermero.guantes.calzar'').
     Then(e => e.Patient.ManosLimpias()).
\end{lstlisting}
\end{algorithm}
%TODO agregar indice de algoritmos

La regla anterior controla que el estudiante ha realizado la acción ``Calzarse
los guantes'', y en ese momento tenga las manos limpias, la variable \emph{e},
es el entorno, y a través de la propiedad \emph{Patient} obtiene el estado del
paciente en ese momento.

\paragraph{Modelo de ejecución}

Para ejecutar un motor de reglas del tipo \gls{eca}, se debe tener en cuenta
principalmente dos factores, 
\begin{enumerate*}[label=\itshape\alph*\upshape)]
\item  Como se verifica el cumplimiento de cada regla, y, 
\item  Que ocurre cuando varias reglas son lanzadas al mismo tiempo
\end{enumerate*}.

Para ambos casos se puede tomar un enfoque \emph{inmediato}, es decir que
inmediatamente cuando se lanza un evento, o se cumple una condición, se ejecuta
la regla. Además existen otros dos modos de ejecución, \emph{deferida}, y
\emph{desacoplada}, en la primera, se espera hasta que el lanzador del evento
culmine su trabajo, y luego se ejecuta la regla, pero en la misma unidad de
trabajo, mientras que en la ejecución desacoplada, se encolan los trabajos y
otro hilo es el encargado de ejecutar las reglas. Estos modos están inspirados
en las bases de datos relacionales, el deferido se ejecuta en la misma
transacción, y el desacoplado, inmediatamente después de que la transacción
termine\cite{bailey2004event}.

La propuesta implementada, utiliza una ejecución inmediata, principalmente por
la sencillez de las reglas, es decir, las reglas no realizar un proceso complejo,
solamente controlan el estado del entorno y lo validan.

Además, la ejecución inmediata es importante por que el entorno no sufre
modificaciones entre el evento lanzado y la ejecución de la regla, según
\cite{bailey2004event}, este es el factor más importante para determinar el tipo
de ejecución deseado.



\paragraph{Estados de una regla}

Una regla puede estar en uno de los siguientes estados:

\begin{description}
\item[BEGIN] Es una regla que recién fue creada, no realiza ninguna
	acción.
\item[WAITING\_FOR\_RULE] Es un estado en el que esta esperando que otras reglas
	sean lanzadas. En este estado, es un suscriptor de las reglas por la que
	espera, y no forma parte del ciclo de ejecución del motor de reglas.
\item[WAITING\_FOR\_EVENT] Es un estado en el que esta escuchando a que sean
	lanzados los eventos a los que escucha, este es el estado principal. En
	este estado, es un suscriptor de los eventos por los que espera, y no
	forma parte del ciclo de ejecución del motor de reglas. Se diferencia
	del estado anterior, en que los eventos escuchados pueden ser lanzados
	por cualquier objeto del entorno, no necesariamente una regla.
\item[WAITING\_FOR\_CONDITION] La regla ya no espera por ningún evento y las
	reglas de las que depende ya han sido lanzadas, se verifica cada cierto
	tiempo si el entorno cumple con una condición definida. 
\item[FINISH] La regla ha sido lanzada, con un resultado no determinado, se pudo
	haber cumplido, como no, es el estado final de una regla. Cuando una
	regla llega a este estado, se lanza su evento de finalización.
\end{description}

Una regla puede estar en solo un estado, y solamente se permite que el estado
avance, desde \emph{BEGIN} hasta \emph{FINISH}.


\paragraph{Ciclo de vida}

Cuando una regla es definida, y insertada al motor de reglas, inmediatamente
pasa al estado \emph{BEGIN}, luego se verifica si la misma depende de otras
reglas, sí este es el caso, pasa al estado \emph{WAITING\_FOR\_RULE} y escucha a
los eventos de finalización de las reglas anteriores.

Una vez que las reglas anteriores han sido finalizadas, la regla pasa al estado
\emph{WAITING\_FOR\_EVENT} sí deben escuchar por algún evento, en caso contrario
pasan al estado \emph{WAITING\_FOR\_CONDITION}.

Una vez que la regla está en estado \emph{WAITING\_FOR\_CONDITION}, pasa a un
motor que ejecuta su condición cada cierto tiempo, si la condición se cumple, la
regla se ejecuta, y la misma pasa a estado \emph{FINISH}, momento en el cual
notifica a las reglas que dependen de ella que ha sido lanzada.

Una vez que la regla esta en estado \emph{FINISH}, la misma sale del esquema de
ejecución, y solo esta disponible para obtener resultados.

Según el ejemplo de la regla definida en el código\ref{alg:rule:guante}, la
regla al terminar de ser construida pasa a estado \emph{BEGIN}, al no depender
de otras reglas, pasa inmediatamente al estado \emph{WAITING\_FOR\_EVENT},
cuando es lanzado el evento, la regla ejecuta la acción y pasa al estado
\emph{FINISH}.

\paragraph{Motor de ejecución}

Un motor de reglas \gls{eca}, requiere de un proceso que evalúe constantemente
las reglas para verificar si las mismas deben ser lanzadas o
no\cite{bailey2004event}\cite{galton2002two}, este motor puede utilizar el
algoritmo de RETE\cite{de2001eca} para realizar esta verificación, en la
propuesta presentada, la cantidad de reglas definidas, y la no dependencia
circular entre ellas, hace innecesario la implementación de tal
algoritmo\cite{de2001eca}. 

El motor de reglas actúa sobre aquellas reglas en estado
\emph{WAITING\_FOR\_CONDITION} e invoca al procedimiento que se encarga de
validar si la regla puede ser activada (el procedimiento es único por cada
regla), si el mismo determina que la regla puede ser lanzada, el motor ejecuta
la acción de la regla y modifica el estado de la regla a \emph{FINISH}.


\subsection{Grafo de estados}

La solución tiene varias escenarios, y dentro de cada escenario, existen varias
pantallas que muestran información relevante de acuerdo a la situación de la
simulación, en~\ref{fig:grafo_estados} se observa la interacción entre las
diferentes pantallas y escenarios.

\begin{figure}[H] 
\centering 
\includegraphics[scale=0.5]{propuesta/grafo_escenas.png}
\caption{Navegación entre escenarios y pantallas. Los escenarios son los
    rectángulos con un borde dos rayas, y las pantallas tienen un borde con una
    sola raya. Los estados inicial y final se muestran como círculos, notar que
    el estado inicial es además del punto de entrada, un escenario.}
\label{fig:grafo_estados}
\end{figure}

La solución inicia con un escenario denominado \emph{Inicio}, en el cual se
permite al usuario observar los detalles del entorno simulado a la vez que
muestra las opciones que permiten iniciar las diferentes prácticas, compartir
su actividad, enviar los datos de utilización y finalmente salir de la
simulación.

Si el usuario selecciona en el \emph{inicio} la opción \emph{Extracción de
    sangre}, se inicia el escenario denominado \emph{Extracción de sangre}, en
el cual el usuario puede realizar el procedimiento de extracción de sangre, si
el usuario selecciona la opción \emph{Fin}, la simulación termina y se dirige a
el escenario \emph{Pantalla de resultados}.

Al seleccionar la opción \emph{Evaluación Glasgow}, se inicia el escenario
denominado \emph{Glasgow}, donde el usuario debe evaluar a un paciente en el
centro del escenario, si el usuario presiona la opción \emph{Fin} se inicia la
pantalla denominada \emph{Evaluar al paciente}, donde el usuario diagnostica el
estado del paciente, y finalmente al presionar el botón \emph{Fin}, la
simulación finaliza y se inicia el escenario \emph{Pantalla de resultados}.

La opción \emph{Exploración Glasgow} es similar, la diferencia es que antes de
iniciar el escenario \emph{Glasgow}, aparece la pantalla \emph{Elegir estado de
    paciente}, en el cual el usuario selecciona un estado para que el paciente
actué de acuerdo al mismo, luego se inicia la escena \emph{Glasgow} y si el
usuario presiona el botón \emph{Fin}, se inicia el escenario \emph{Pantalla de
    resultados}.

La pantalla de resultados muestra la información acerca de las acciones que
realizo el usuario, proveyendo información a modo de retroalimentación, en esta
pantalla el usuario puede compartir sus resultados por las redes sociales,
reiniciar el escenario y finalmente, poder volver a la \emph{Pantalla de
    inicio}.

Se describen todos los escenarios, primeramente se da una descripción general de
los escenarios y se procede a explicar los detalles del mismo, incluyendo las
entidades, eventos y acciones que pueden ser realizados en el mismo.

\subsection{Inicio}

La solución se inicia con un escenario que esta inspirado en el laboratorio de
enfermería del \Gls{iab}, este escenario es la primera experiencia que tiene un
usuario al utilizar la misma, este escenario sirve como un menú principal, desde
este punto todas las opciones son accesibles para el usuario, este escenario es
denominado~\emph{Inicio}.

La primera vez que se inicia la solución, se muestra una pequeña ventana
solicitando el número del usuario, se utiliza esta información como un
identificador del usuario y así poder asociar la información del mismo con un
alumno en particular.

\subsubsection{Descripción del entorno}
\label{sec:inicio_descripcion}

La escena mostrada como pantalla de inicio de la aplicación muestra como fondo
la sala de un hospital con los elementos típicos de estos lugares, esta es la
que se utiliza como escenografía principal en las escenas de los procedimientos.
Además de este fondo, se muestras varias opciones en forma de botones que serán
descriptas a continuación y un mensaje en donde se recomienda al usuario el uso
de auriculares.

\begin{figure}[H] 
\centering 
\includegraphics[scale=0.2]{propuesta/sala.jpg}
\caption{Edificio y decoración inspirados en los laboratorios de enfermería del
    \Gls{iab}, este edificio se utiliza par los diferentes escenarios}
\label{fig:sala_perspectiva}
\end{figure}

Se observa en la figura~\ref{fig:sala_perspectiva} la decoración utilizada en
el escenario \emph{Inicio}, mientras se muestran las opciones, se ejecuta una
animación que recorre el escenario mostrando los detalles importantes, como la
camilla, el lector de estadísticas vitales, y demás elementos del escenario.

\subsubsection{Opciones}

Las opciones disponibles en la pantalla de inicio son presentadas en forma de
botones los cuales tienen una breve descripción que identifica la función que
cumplen. 

\begin{itemize}
\item Botón \enquote{Enviar Progreso}: esta función envía toda la información
    acerca de la actividad que el usuario realizo en la aplicación a un servidor
    \emph{backend} que se encarga de almacenar estos datos.
\item Botón \enquote{Salir de la simulación}: esta función permite salir de la
    aplicación.
\item Botón \enquote{Facebook}: esta función permite al usuario ingresar a su
    cuenta de Facebook.
\item Botón \enquote{Extracción de sangre}: esta función permite ingresar a la
    escena correspondiente al procedimiento de extracción de muestras de sangre
    permitiendo al usuario jugar una nueva partida.
\item Botón \enquote{Explorar Glasgow}: esta función permite ingresar a la
    escena correspondiente al procedimiento para explorar las reacción de un
    paciente con un diagnostico especifico de la escala de Glasgow permitiendo
    al usuario jugar una nueva partida, el diagnostico se selecciona una vez
    presionado este botón a través de la ventana de~\emph{Elegir estado del
        paciente}.
\item Botón \enquote{Evaluar Glasgow}: esta función permite ingresar a la escena
    correspondiente al procedimiento para la valoración y diagnostico de la
    escala de Glasgow para un paciente con estado aleatorio permitiendo al
    usuario jugar una nueva partida.
\end{itemize}


\subsection{Extracción de muestras de sangre}

A continuación se detallan cada una de las opciones y formas disponibles de
interactuar con la escena del procedimiento de extracción de muestras de sangre.

Los detalles a continuación toman en cuenta las hipótesis definidas
en~\ref{sec:hemocultivo_hipotesis}. 

\subsubsection{Descripción del entorno}

Al seleccionar el procedimiento de extracción de sangre en la pantalla de inicio 
la aplicación inmediatamente muestra la escena del procedimiento, se muestra una 
sala de hospital igual a la de la pantalla de inicio pero con un paciente en una 
de las camas, a este paciente es a quien se le realizara el procedimiento.

La posición inicial de la cámara se ubica en un ángulo en donde se puedan ver 
bien los brazos del paciente para facilitar al usuario la realización del 
procedimiento.


\subsubsection{Entidades}

En la extracción de sangre existen dos entidades principales, el \emph{paciente}
y el \emph{usuario}, cada entidad mantiene un estado independiente de la otra
entidad.

El \emph{paciente} es una entidad con estado complejo, el cual es constantemente
modificado por las acciones del usuario, en resumen, la información que contiene
el paciente es:

\begin{itemize}
    \item \textbf{Jeringas}: un paciente puede tener cero o más jeringas en
        cualquier momento, no se limita la cantidad de jeringas que puede
        insertar el usuario.
    \item \textbf{Manos}: almacena el estado de las manos, el paciente reacciona
        ante peticiones del usuario, puede abrir o cerrar cualquier mano en
        cualquier momento.
    \item \textbf{Torniquetes}: es el conjunto de torniquetes que tiene
        actualmente el paciente, notar que los torniquetes pueden ser colocados
        en cualquier parte del brazo, pero existen lugares \enquote{correctos} y
        lugares \enquote{incorrectos}, la diferencia consiste en la distancia a
        los puntos de extracción, estos lugares están predefinidos.
    \item \textbf{Zonas esterilizadas}: son aquellas áreas del cuerpo que el
        usuario esterilizó, no existe un límite para las zonas esterilizadas.
        Una vez que una jeringa es extraída, una zona esterilizada pasa a estar
        contaminada y a la espera de que el usuario la presione.
    \item \textbf{Zonas presionadas}: son aquellas zonas que, una vez
        contaminadas por la extracción de una jeringa, han sido presionadas por
        el usuario.
    \item \textbf{Contaminado}: define si alguna acción realizada por el usuario
        provoco que el paciente se contamine, existen varias cadenas de eventos
        que pueden provocar que esto ocurra:
        \begin{itemize}
            \item Inyección de una jeringa cuando existe otra inyectada.
            \item Inyección en un lugar inadecuado.
            \item Inyección en un lugar no esterilizado.
            \item Inyección en un brazo cuya mano este abierta.
            \item Inyección fuera del alcance de los torniquetes actuales.
            \item Interacción con el paciente sin que el enfermero tenga la mano
                estéril.
        \end{itemize}
        Es importante notar que este estado no es afectado directamente por una
        acción del usuario, sino por la consecuencia de una acción.
\end{itemize}

El \emph{usuario o enfermero} mantiene un estado en todo momento del cual
dependen sus acciones, por ejemplo, si la mano del enfermero no esta
esterilizada, cualquier interacción con el paciente provocara que se
contamine.

\begin{itemize}
    \item \textbf{Manos}: almacena la información acerca de la esterilidad de
        las manos.
    \item \textbf{Guantes, gorro, bata y tapaboca}: almacenan la información
        acerca de los equipamientos que tiene el usuario en un momento dado.
    \item \textbf{Elemento actual}: es el elemento que esta activo en
        cualquier momento, un elemento es una herramienta de la vida real,
        como por ejemplo un torniquete, una gaza.
\end{itemize}

\subsubsection{Acciones}

Las acciones que puede realizar el usuario se clasifican en tres, los
\emph{comandos de voz} que simulan una conversación entre el paciente y
enfermero, \emph{las opciones}, que engloban las acciones que puede realizar un
enfermero en cuanto a bioseguridad, y \emph{los elementos} que son las
herramientas que puede utilizar el enfermero durante el procedimiento.

\paragraph{Comando de voz}

Para representar la interacción del usuario con el paciente usando la voz se
implemento un menú que es activado y mostrado en pantalla cuando el usuario
habla, este menú muestra una seria de ordenes que el usuario le diría al
paciente normalmente. Las opciones de menú se detalla a continuación:

\begin{itemize}
    \item \textbf{Explicar procedimiento}: Permite que el usuario explique el
        procedimiento que se va a realizar al paciente. 
\item \textbf{Abrir la mano izquierda}: esta función le indica al paciente que
    abra su mano izquierda, como resultado el paciente realiza esta acción.
\item \textbf{Cerrar la mano izquierda}: esta función le indica al paciente que
    cierre su mano izquierda, como resultado el paciente realiza esta acción.
\item \textbf{Abrir la mano derecha}: esta función le indica al paciente que
    abra su mano derecha, como resultado el paciente realiza esta acción.
\item \textbf{Cerrar la mano derecha}: esta función le indica al paciente que
    cierre su mano derecha, como resultado el paciente realiza esta acción.
\end{itemize}

\begin{figure}[H]
\centering
\includegraphics[scale=0.5]{propuesta/hemocultivo_comando_voz.jpg}
\caption{Opciones mostradas al detectar sonido en la escena de extracción
    de sangre.}
\label{fig:hemocultivo_voz_gui}
\end{figure}

En la figura~\ref{fig:hemocultivo_voz_gui} se observa las opciones anteriormente
descritas, este menú aparece inmediatamente después de que el sistema detecte
que el usuario este hablando.

\paragraph{Opciones}

Las \emph{Opciones} son aquellas acciones que puede realizar y afectan
únicamente al paciente. Representan a los aspectos de bioseguridad, es decir,
acciones como lavarse las manos, calzarse guantes, gorro, bata y tapaboca.

Estas opciones afectan al estado de la entidad \emph{enfermero}.

\paragraph{Elementos}

Los \emph{Elementos} representan las herramientas que utiliza un enfermero
durante el procedimiento, un solo elemento puede ser utilizado en cualquier
momento.


\begin{figure}[H]
\centering
\includegraphics[scale=0.5]{propuesta/hemocultivo_elementos.jpg}
\caption{Interfaz con los elementos en el paciente, se observa el torniquete (es
    un toro negro), la jeringa, una zona esterilizada (el área cerca del
    torniquete) y un algodón para punzar (Cerca de la muñeca, es una figura
    blanca), todo en el brazo derecho.}
\label{fig:hemocultivo_elementos}
\end{figure}

Los elementos, que podemos observar en la
figura~\ref{fig:hemocultivo_elementos}, son:

\begin{itemize}
\item \textbf{Torniquete}: es el primer elemento que se debe usar, para utilizar
    el mismo, el mismo se activa al presionar una parte del brazo del paciente,
    en ese momento, el torniquete aparece en el brazo del paciente, para
    extraerlo, se debe presionar el torniquete y elegir la opción extraer. En la
    figura~\ref{fig:hemocultivo_elementos} se observa al torniquete como un toro
    negro alrededor del brazo izquierdo.

\item \textbf{Sanitizador}: es un elemento que se utiliza para realizar la
    higienización del punto de punción, para utilizarlo se debe presionar
    cualquier parte del brazo del paciente, a continuación aparece una gaza, la
    cual debe ser agitada con un dedo durante un segundo para que se cree una
    zona estéril, la zona estéril creada, es visible a través de una capsula
    verde transparente.

\item \textbf{Jeringa}: Es el elemento utilizado para realizar la extracción, su
    utilización es similar a la del \emph{Torniquete}, solo que aparte de la
    opción de extracción tiene dos opciones adicionales.

    A través de un menú contextual, se ofrece la posibilidad de realizar un
    acercamiento, como se observa en la
    figura~\ref{fig:hemocultivo_jeringa_zoom}, en la vista ampliada, se puede
    realizar la extracción de sangre utilizando dos dedos, con el primero se
    presiona el tambor y con el segundo dedo se extrae el émbolo\footnote{El
        tambor es la parte de la jeringa que almacena el fluido, mientras que el
        émbolo es la parte que se utiliza para presionar o succionar el fluido}.
    
\item \textbf{Algodón}: El algodón se utiliza para presionar una zona que
    recientemente fue punzada, para utilizar este elemento, basta con presionar
    el brazo del paciente durante un segundo.

\end{itemize}


\begin{figure}
\centering
\includegraphics[scale=0.5]{propuesta/hemocultivo_jeringa_ampliada.jpg}
\caption{Vista de la jeringa ampliada, facilitando la extracción de sangre. Se
    agregan flechas azules para facilitar la comprensión del cómo se extrae
    sangre.}
\label{fig:hemocultivo_jeringa_zoom}
\end{figure}



\subsubsection{Reglas para la evaluación durante la ejecución}

%La reglas del procedimiento de extracción de sangre fueron definidas de acuerdo
%a los pasos requeridos según el protocolo del procedimiento y al orden en el que
%son requeridos. Es decir, cada paso del protocolo tiene asociado una regla
%dentro del motor que lo representa y las condiciones asociadas a cada regla
%están determinadas por el orden en que deben realizarse dentro del protocolo.

%Cada regla tiene una o mas condiciones que deben ser cumplidas para que un paso
%del protocolo realizado se considere correcto.

Las reglas definidas dentro de la extracción de sangre definen las acciones
que se deben llevar a cabo para completar el procedimiento, es necesario que
todas las reglas sean cumplidas para obtener un puntaje perfecto.

Cada regla contiene información acerca del estado del progreso del alumno en un
paso en particular, estas reglas pueden tener como dependencias a otras reglas,
es decir, una regla solo se puede cumplir si una regla anterior se cumple, este
es el caso de las reglas que definen la extracción de un torniquete, la cual
depende de la regla que define la colocación del torniquete.  

Las reglas definen el mecanismo que se utiliza para proveer una retroalimentación
al usuario una vez finalizada la partida, pues la regla almacena información 
acerca del progreso del usuario en cada paso.

A continuación se da una breve descripción de las reglas utilizadas y sus
diversos estados.

\begin{itemize}
\item \textbf{Explicar procedimiento}: define cuando el usuario ha explicado el
    procedimiento, debe ser la primera regla que se cumple, existiendo dos
    estados que no cumplen la condición, cuando no es la primera regla que se
    cumple y cuando el usuario no realizo la acción.

\item \textbf{Higuienización de manos}: define sí las manos del enfemero están
    limpias, existen dos estados para esta regla, cuando se lavo las manos y
    cuando no realizo esta acción. Es importante notar que de esta regla
    dependen varias reglas posteriores, es decir, si esta regla no se cumple,
    las reglas de bioseeguridad no pueden ser cumplidas.

\item \textbf{Calzar guantes}: este paso es inmediatamente posterior a la
    higuienización de las manos, y como este, varias reglas dependen del momento
    en el que se calcen los guantes. Si el paciente se calza los guantes, la
    única forma de fallar este paso es que las manos estén sucias.

\item \textbf{Gorro, bata y tapaboca}: estos tres pasos son similares, pues
    tienen las mismas dependencias y el orden en el que se cumplan no está
    definido, las posiblidades con este paso son:
    \begin{itemize}
    \item Incompleto: si el usuario no se calzo el gorro, bata o tapaboca en
        ningún momento de la práctica.
    \item Manos sucias: si la regla \emph{Higuienización de manos} no se cumple
        cuando el usuario se vista con el gorro, bata o tapaboca.
    \item Sin guantes: si la regla \emph{Calzar guantes} no se cumple al momento
        de calzar el gorro, tapaboca o bata. 
    \end{itemize}
    
\item \textbf{Colocar torniquete}: este paso debe ser realizado una vez que el
    usuario tenga el guante calzado, el torniquete puede ser colocado en
    cualquier parte del brazo, pero depende del lugar de punción de la jeringa.
    Así, existen dos casos donde esta regla falla, cuando no se coloca el
    torniquete en ningún lugar, y cuando el torniquete se coloca en un lugar
    inadecuado. Es importante notar que esta regla se activa cuando se realiza
    la punción de la jeringa.

\item \textbf{Cerrar manos}: debe ocurrir después de que se explicará el
    procedimiento y antes de que se inserte la jeringa, además deberá ser la
    mano correspondiente al lugar donde se inyecto la jeringa. Así, existen dos
    formas de fallar este paso, que la punción haya ocurrido en otro brazo, o
    que la punción no se realizo. Es importante notar que esta regla no activa
    cuando el usuario solicita al paciente que se cierre su mano, sino cuando la
    jeringa es insertada, esto es así pues es importante el estado de la mano
    cuando el usuario realiza la punción.

\item \textbf{Esterilizar Zona}: El usuario puede esterilizar varias zonas, pero
    la única zona que activa esta regla, es aquella donde se realizo la punción.
    Así esta regla tiene dos posibles formas de proveer retroalimentación en
    caso de que no se cumpla, que no se esterilizó ningún lugar del cuerpo, o
    que el lugar esterilizado no sea el lugar de punción.

\item \textbf{Realizar punción}: Existen dos casos de error, si el usuario
    inserta la jeringa en un lugar incorrecto (los lugares correctos se definen
    en~\ref{sec:hemocultivo_hipotesis}), y sí el usuario no realiza la punción.

\item \textbf{Retirar Torniquete}: esta regla es dependiente de la regla
    \emph{Colocar Torniquete}, y de cual torniquete se retira. Existiendo dos
    casos de error, el usuario retira un torniquete que no activo la regla
    \emph{Colocar Torniquete} (indicando que este es el torniquete correcto), y
    que no se retire ningún torniquete.

\item \textbf{Abrir mano}: una vez realiza la punción, y retirado el torniquete,
    se debe solicitar al paciente que abra sus manos, así, es dependiente de la
    regla \emph{Realizar punción}, y tiene tres casos de error, que la punción
    no se haya realizado, que la mano no se haya cerrado antes, y que la mano
    que se abra no corresponda con la mano que se cerro.

\item \textbf{Extraer Sangre}: depende únicamente de si la jeringa con la cual
    se extrae sangre es la correcta, así hay dos posibles casos de errores, que
    la sangre no se haya extraído, y que la regla \emph{Realizar punción} no se
    cumpla.

\item \textbf{Retirar Jeringa}: se controla que la jeringa que se extraíga sea
    la jeringa que cumplió el paso \emph{Realizar punción}, así, existen dos
    casos de error, que la jeringa no hay sido extraída y que se extrajo sangre
    de una jeringa incorrecta.

\item \textbf{Presionar zona de punción}: debe realizarse inmediatamente después
    de \emph{Retirar Jeringa}, y tiene dos casos de error, que la jeringa no
    haya sido extraída, y que la zona punzada no corresponda con la zona de
    punción.

\item \textbf{Quitar bata, gorro y tapaboca}: depende de la regla \emph{Bata,
        gorro y tapaboca}, y de que la jeringa haya sido extraída, si el
    enfemero se extrae la bata, gorro o tapaboca antes de retirar la jeringa, se
    produce un error. Otro caso para que esta regla falle es que la regla
    \emph{Bata, gorro y tapaboca} no se cumpla.

\item \textbf{Descalzar guantes}: depende de que las reglas \emph{Calzar
        Guantes} y \emph{Retirar jeringa}, así existen dos posibles casos de
    error, que el usuario no se calce los guantes, o que la jeringa no haya sido
    extraída.

\item \textbf{Limpiar manos}: Como paso final, se debe proceder a realizar una
    higienizaron de manos, este debe ser el paso final, y tiene como
    prerrequisito que todas las reglas anteriores hayan sido lanzadas.

\end{itemize}

\paragraph{Retroalimentación y puntuación final}
\label{sec:puntuacion_hemocultivo}

Cada regla tiene asociado un peso, de acuerdo a la dificultad de realizar el
paso, este peso es utilizado al final de la partida para darle una puntuación al
usuario acerca de su rendimiento en la partida.

Además, un regla puede quedar en uno de diferentes estados al final de la
partida como se mostró anteriormente, cada uno de esos estados posee un
significado en el contexto del procedimiento y por lo tanto tiene información
asociada para que al final de la partida se muestre una retroalimentación
correcta al usuario por paso.

\subsubsection{Registro de actividad}

Cada acción que realiza el usuario dentro de la simulación provoca un evento, y
estos eventos son registrados de manera transparente para el usuario.

Existen otros tipos de eventos que no son generados por acciones, por ejemplo,
cuando la simulación termina, el motor de reglas lanza un evento por regla,
indicando su estado.

Un subconjunto de todos los eventos, son registrados en un archivo de texto en
formato \Gls{json}, el mismo es posteriormente enviado a un servidor que
almacena la información de todos los usuarios.

Los eventos registrados, son aquellos que involucran a las opciones, elementos,
utilización de la jeringa, torniquete, higienizador, y como un caso especial,
todas las reglas también son registradas (independientemente si son
satisfactorias o no).

La información que se almacena permite reproducir exactamente todo el desarrollo
de la simulación, con excepción de los movimientos de la cámara. De esta manera,
los datos recabados permiten saber en que tareas los usuarios se encontraron con
un mayor número de inconvenientes.

\subsubsection{Interfaz del usuario}

La interfaz principal de este escenario posee dos menús, uno a cada lado de la
pantalla, las opciones son representadas como botones que poseen una imagen
intuitiva\todox{Ver si no hay que agregar esto como hipótesis} que representa la
función que realizan. 

\begin{figure}
\centering
\includegraphics[scale=0.5]{propuesta/hemocultivo_gui.jpg}
\caption{Vista de la interfaz principal del escenario \emph{Extracción de
        sangre}, con todas las opciones desplegadas.}
\label{fig:hemocultivo_gui}
\end{figure}


En la figura~\ref{fig:hemocultivo_gui} se observa la interfaz, se observa en el
centro al paciente, y las diferentes opciones que tiene el usuario para
interactuar con el paciente y con el enfermero.

Existen dos opciones principales dentro de la interfaz, en la parte superior
derecha esta el menú \emph{Opciones} y en la parte inferior izquierda, el menú
\emph{Elementos}.

Al presionar el menú \emph{Opciones} aparecen las distintas
acciones que puede realizar el usuario en cuanto a aspectos de bioseguridad, el 
lavado de manos es idempotente, es decir no importa cuantas veces se presione, el 
resultado será el mismo, en cambio los demás botones (bata, guante, tapaboca y gorro)
representan la acción de calzar/descalzar, es decir, la primera vez que se presiona
la opción bata, el estado del enfermero cambia a \emph{Con Bata}, si se presiona una
segunda vez, el estado cambia a \emph{Sin Bata}.

El menú \emph{Elementos} se despliegan opciones que representan a lo elementos que se
utilizan para realizar el procedimiento, una vez presionado un elemento queda
seleccionado (simulando que es la herramienta que el enfemero tiene en la mano
en ese momento), solo un elemento puede ser seleccionado a la vez. Si el mismo
botón se vuelve a presionar inmediatamente después de haber sido presionado, el
elemento se des-selecciona (simulando que el enfemero dejo la herramienta).

Adicionalmente, existen dos indicadores del estado del paciente, en la parte
inferior derecha, denominados \emph{Indicadores de bioseguridad}, se muestran
iconos representando los elementos que tiene actualmente el paciente, estos
incluyen los gorros, bata y tapaboca. El elemento que actualmente esta
seleccionado se muestra en la parte superior izquierda de la pantalla, a
diferencia de los indicadores de bioseguridad, es un único icono.

Para la utilización de los elementos, existe un menú contextual\footnote{Un menú
    que se despliega al presionar un elemento, es contextual pues varia de
    acuerdo al elemento seleccionado.}, que lista las opciones disponibles por
elemento, como se observa en la figura~\ref{fig:hemocultivo_torniquete_cm}.

\begin{figure}
\centering
\includegraphics[scale=0.5]{propuesta/hemocultivo_contextual.jpg}
\caption{Menú contextual del elemento torniquete.}
\label{fig:hemocultivo_torniquete_cm}
\end{figure}

\subsection{Valoración de la escala de Glasgow}


La escena  \emph{Valoración de la escala de Glasgow} busca simular el
comportamiento de un paciente con daño cerebral, por ello se presenta en dos
modos distintos, en la primera, el usuario no conoce el estado del paciente,
este modo se conoce como \emph{Evaluar Glasgow}, y en su segundo modo, el
usuario elige el estado del paciente antes de iniciar la escena, esto modo
se conoce como \emph{Exploración Glasgow}. 

La reducida cantidad de diferencias entre ambos modos de la práctica permiten
que ambas sean descritas en esta sección, las características explicadas son
comunes para ambas, salvo se especifique lo contrario.

En el modo \emph{Exploración Glasgow}, antes de iniciar la práctica, se le
permite al usuario seleccionar el estado del paciente mediante una interfaz, en
cambio, en el modo \emph{Evaluar Glasgow}, el estado del paciente no se conoce
de antemano y será responsabilidad del usuario determinarlo.

La escena se inicia mostrando un paciente acostado en una camilla, de manera
similar a la escena \emph{Extracción de sangre}, la principal diferencia es que
el paciente no se encuentra en ninguna posición en particular, simplemente esta
acostado en la camilla.

Existe una pantalla particular dentro de esta escena, conocida como
\emph{Pantalla de Diagnostico}, esta pantalla, permite al usuario ingresar su
diagnostico del paciente, contiene cuatro preguntas básicas, la valoración
motora, verbal, ocular y general de acuerdo a lo definido en~\ref{sec:glasgow} y
a las hipótesis definidas en~\ref{sec:glasgow_hipotesis}.

\emph{Exploración Glasgow}.

\subsubsection{Entidades}

Al igual que en \emph{Extracción de sangre}, existen dos entidades principales,
solo que en esta escena, el enfermero no almacena información alguna, y el
paciente solo almacena su estado, que se define al inicio. De esta forma, las
entidades no se modifican en ningún momento.

La información almacenada por la entidad paciente es su estado motor, verbal y
ocular, el cual es un conjunto de números, cuyos posibles valores se definen en
en~\ref{sec:glasgow_protocolo}, la definición de estos números varían de acuerdo
al tipo de forma de la escena:

\begin{itemize}
    \item \textbf{Exploración}: la escena de exploración se inicia con la
        \emph{Pantalla de diagnostico}, donde el usuario selecciona el estado
        del paciente que desea, este estado se mantendrá constante durante toda
        la escena.
    \item \textbf{Evaluación}: al inicio se crean tres números de manera
        aleatoria, el algoritmo que crea estos valores, lo hace de tal manera
        que el estado del paciente es consistente, por ejemplo, el paciente
        nunca tendrá un estado verbal \enquote{Orientado} (valor 5 en la escala)
        y un estado ocular \enquote{Ausente} (valor 1 en la escala), pues esto
        no tendría sentido, si no puede abrir los ojos (estado
        \enquote{ausente}), no puede saber donde esta (estado
        \enquote{orientado}).
\end{itemize}

Aunque el estado de las entidades no se modifique, esto no significa que no
puedan realizar acciones entre ellas, solo que estas acciones y los eventos
generados no alteran el estado del las entidades.

\subsubsection{Acciones} 

Las acciones se clasifican en dos tipos, los \enquote{comandos de voz} y las
\enquote{opciones a través de menú contextual}. 

\paragraph{Menú Contextual}

Las opciones a través del menú contextual se relacionan a acciones que puede
realizar el paciente sobre una parte particular del cuerpo del paciente, en las
extremidades, el menú despliega una sola opción, la cual es \emph{Pinchar}, que
provoca que el enfermero realice un estimulo doloroso en el paciente, el
paciente reacciona ante este estímulo dependiendo de su valoración motora y
ocular. 

\begin{itemize}
    \item Si el estado ocular del paciente es \enquote{Al dolor}, el paciente
        abrirá los ojos inmediatamente después de que se presione la opción. 
    \item  La respuesta motora varia de acuerdo al su estado, si el mismo es
        \enquote{Localiza}, el paciente mueve sus manos hasta el origen del
        dolor, si el estado es \enquote{Retira}, moverá la extremidad que
        sufrió el estimulo lejos de su posición inicial, si es
        \enquote{Flexión anormal}, el paciente reaccionará comprimiendo el
        cuerpo, indistintamente de la ubicación del estimulo doloroso, y sí el
        estado es \enquote{Extiende}, el paciente extenderá el cuerpo.
\end{itemize}

\paragraph{Comandos de voz}

Las opciones disponibles a través de los comandos de voz, simulan una
interacción verbal entre el enfermero y el paciente, y se agrupan por tres
tipos, verbales, oculares y motoras.

Las preguntas y posibles respuestas de tipo \emph{verbal}, se pueden ver en la
tabla~\ref{tab:glasgow_opciones_respuesta}. 

\begin{table}[H]
\centering
\begin{tabulary}{1.2\textwidth}{LCCCCC}
\toprule
\textbf{Pregunta} & \textbf{Orientado} & \textbf{Confusa} & \textbf{Palabras
    inapropiadas} & \textbf{Palabras incomprensibles} & \textbf{Ausente} \\
\midrule
¿Qué día es? & El día de la semana actual & Cualquier día de la semana menos el
correcto & La respuesta a otra pregunta en estado orientado & Gritos, gruñidos y
quejidos & No emite sonido \\
¿Cuál es su nombre? & \emph{Carlos Benitez} & Respuesta coherente sin mencionar
su nombre & Respuesta a otra pregunta en estado orientado & Gritos, gruñidos y
quejidos & No emite sonido \\
¿Donde se encuentra? & \emph{En una cama de hospital} & \emph{En mi dormitorio} &
Respuesta a otra pregunta en estado orientado & Gritos, gruñidos y quejidos & No
emite sonido \\
\bottomrule
\end{tabulary}
\caption{Posibles respuestas de acuerdo al estado verbal del paciente.}
\label{tab:glasgow_opciones_respuesta}
\end{table}

Las opciones del menú contextual del tipo \emph{motor}, son cuatro: 

\begin{itemize*}
    \item Mueva el brazo
    \item Mueva la pierna
    \item Mueva la mano
    \item Mueva la cabeza
\end{itemize*}

A diferencia de las opciones verbales, estas no tienen una respuesta sonora, en
cambio, sí el estado motor es \enquote{Obecede},el paciente reacciona moviendo
una extremidad, en el caso contrario, el paciente no realiza acción.

Finalmente, existe una opción \emph{ocular}, la cual es \enquote{Abra su ojo por
favor}, la cual no tiene una respuesta sonora, y solo sí el paciente tiene un
estado ocular \enquote{Al hablar} abre los ojos, en caso contrario, no realiza
acción alguna.

\subsubsection{Pantalla de diagnostico}

Una vez que el usuario decida que esta listo para dar un diagnostico, procede a
finalizar la escena, en ese momento se presenta una pantalla donde el mismo
puede diagnosticar al paciente, como se observa en la
figura~\ref{fig:glasgow_gui_resultados}.

Las opciones presentadas al usuario son cuatro, puntuación verbal, ocular,
motora y un diagnostico del estado de conciencia del paciente. Estos valores
se describen en~\ref{sec:glasgow_protocolo}.

\begin{figure}[H]
\centering
\includegraphics[scale=0.5]{propuesta/glasgow_diagnostico.jpg}
\caption{Vista de la \emph{Pantalla de diagnóstico}, donde el usuario puede
    asignar una puntuación a cada aspecto analizado del paciente.}
\label{fig:glasgow_gui_resultados}
\end{figure}

En el modo \emph{Evaluación Glasgow}, esta misma pantalla se utiliza la inicio
de la escena para que el usuario pueda seleccionar el estado deseado del
paciente.

La única diferencia que existe entre la \emph{Pantalla de diagnostico} entre la
exploración y la evaluación, es que en la evaluación, los posibles valores para
cada aspecto a diagnosticar es de $0$ a $8$, y en la exploración, son los
valores mínimos y máximos validos\footnote{Los valores máximos se definen
    en~\ref{sec:glasgow_protocolo} y son $4$ (ocular), $5$ (verbal) y $6$
    (motora)}.

\subsubsection{Valoración}
\label{sec:puntuacion_glasgow}

Para la evaluación del rendimiento del usuario en el momento de llevar a cabo el
procedimiento de valoración de la escala de Glasgow se tuvo un enfoque
completamente diferente al del procedimiento de extracción de sangre debido a la
naturaleza propia del procedimiento. 

Como se explico anteriormente, el paciente puede estar en ciertos estados
específicos dentro de la escala, y además dentro de cada estado reacciona de un
forma en particular por lo tanto, al inicio de la partida un componente interno
de la aplicación selecciona de forma aleatoria un estado para el paciente, de
forma tal que cada vez que una partida sea jugada no se repitan los estados de
forma seguida.

El estado aleatorio del paciente es guardado en una variable que no es
modificada hasta que se reinicie la partida. Al final de la partida, la
aplicación pide al usuario que valore el estado del paciente que le fue
presentado, una vez que el usuario confirme su respuesta la aplicación la
compara con el estado guardado y de esta forma puede informar al usuario acerca
de su rendimiento en el diagnostico.

Además, cada posible respuesta dada por el usuario contiene información
relacionada al contexto del procedimiento y a la situación actual presentada la
cual es utilizada como retroalimentación al final de la partida. La puntuación
final dada depende de la cantidad de valoraciones correctas dadas por el usuario
para la respuesta verbal, motora, ocular y nivel de gravedad del paciente.

\subsubsection{Registro de actividad}

Al igual que en la escena \emph{Extracción de sangre}, las acciones del usuario
son registradas en un archivo con formato \Gls{json} y enviadas a un servidor
que almacena la información acerca de todos los usuarios.

Se almacena además, el diagnostico del usuario, la puntuación del mismo y el
estado inicial del paciente, así como el modo de la escena (evaluación o
exploración).

\subsubsection{Interfaz de usuario}

La interfaz de usuario, como se observa en la figura~\ref{fig:glasgow_gui}, es
muy sencilla, se compone de solo una opción permanente, la cual permite al
usuario finalizar la escena y muestra la \emph{Pantalla de diagnostico}. Se
observa además, las opciones que se despliegan cuando la solución detecta que el
usuario emite palabras, conocidas como \emph{Comandos de Voz}, que permiten al
mismo interactuar con el usuario.

\begin{figure}[H]
\centering
\includegraphics[scale=0.5]{propuesta/glasgow_comandos_voz.jpg}
\caption{Interfaz de la escena \emph{Evaluación de Glasgow}, se observa los
    \emph{comandos de voz}, así como la opción que permite finalizar la escena
    (esquina inferior izquierda).}
\label{fig:glasgow_gui}
\end{figure}

Además se ve en la interfaz, al paciente, que es el foco principal de la cámara.

\subsection{Pantalla de resultados}

Al finalizar ambas escenas, se presenta una pantalla de resultados, la cual es
la encargada de mostrar toda la información que fue recabada durante la escena,
esta información incluye los pasos correctos e incorrectos que realizo el
usuario.

\begin{figure}[H]
\centering
\includegraphics[scale=0.5]{propuesta/resultado_glasgow.jpg}
\caption{Pantalla de resultados mostrando los pasos correctos e incorrectos, en
    la escena \emph{Glasgow}.}
\label{fig:resultados_glasgow}
\end{figure}

Se observa en la figura~\ref{fig:resultados_glasgow} el diseño de la
pantalla, el título es la escena actual, en el pié se observa un resumen de la
puntuación, y el tiempo que duro la escena.

Se lista de manera ordenada los pasos en la parte izquierda de la pantalla, y en
la parte derecha se muestra información relevante acerca del motivo por el cual
no se cumplió cada paso.

Adicionalmente a la información de la sesión, se permite al usuario reiniciar la
escena, ir al menú, y compartir en las redes sociales.

\subsubsection{Retroalimentación}

La retroalimentación se logra a través de la pantalla de resultados, en ella se
presenta información detalla de lo que se realizo. 

Si el usuario realizo de manera incorrecta un paso del procedimiento, esta
información esta contenida en una regla (\emph{Extracción de Sangre}), o en el
estado inicial del paciente (\emph{Evaluación de Glasgow}).

\begin{figure}[H]
\centering
\includegraphics[scale=0.5]{propuesta/resultado_hemocultivo.jpg}
\caption{Pantalla de resultados mostrando los pasos correctos e incorrectos en
    la escena \emph{Extracción de sangre}.}
\label{fig:resultados_hemocultivo}
\end{figure}

Como se observa en la figura~\ref{fig:resultados_hemocultivo}, el paso
\enquote{Poner Jeringa}, contiene la siguiente información \enquote{Puesta en un
    lugar incorrecto}, esto le indica al usuario, que el lugar donde la jeringa
fue insertada era incorrecto, otro posible valor de retroalimentación es
\emph{Jeringa no insertada}.

\subsubsection{Gamificación}

En esta pantalla además se observan opciones que son estudiadas por la
\emph{Gamificación}, entre ellas observamos la puntuación total, el tiempo
utilizado y la utilización de redes sociales.

Si el usuario presiona el botón \enquote{Facebook}, se despliega el menú de
dicha red social, permitiendo que el mismo pueda agregar un mensaje
personalizado y la sesión se comparta con el texto \enquote{Conseguí 15 puntos
    en 1475 segundos, en la \emph{Escena Extracción de sangre} jugando con
    YAVE}.

