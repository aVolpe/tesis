%! TEX root = ../main.tex

\section{Solución}
\label{sec:solucion}

Se describe la arquitectura propuesta para la realización de una juego serio, se
utiliza la guía básica definida por~\cite{pereira2009design} y descrita
en~\ref{sec:desarrollo}.

Esta sección se enfoca en los aspectos técnicos de la creación del juego serio,
las competencias básicas relacionadas con la educación (segundo paso de la guía
definida por~\cite{pereira2009design}) se define en las
secciones~\ref{sec:glasgow} y~\ref{sec:hemocultivo}.

\subsection{Arquitectura}

La arquitectura esta basada en eventos, los diferentes componentes del entorno
interactúan y notifican acciones del usuario a través de un sistema de eventos.

Se compone de tres principales actores que se encuentran dentro del
escenario, los cuales son, el jugador, el paciente, y los objetos, cada actor es
independiente e interactúa con los demás 

\subsection{Prototipo}
\subsubsection{Diseño de interfaz}
\subsubsection{Interacción con la simulación}
\subsubsection{Modelado}
\subsection{Gamificación}

\subsection{Modelo del protocolo}
\subsubsection{Motor de reglas}
\subsection{Modelo de dominio}
\subsubsection{Entidades}
\subsubsection{Acciones}
\subsubsection{Eventos}
\subsection{Modelo de interacción}


