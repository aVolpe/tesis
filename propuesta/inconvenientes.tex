\section{Inconvenientes de diseño}

Los mayores inconvenientes de diseño de la aplicación se dieron en el momento de
validar tanto el contenido de la aplicación como la interfaz de usuario, para
sobrellevar estos inconvenientes fueron requeridos la intervención de terceros.

A continuación se explica en detalle cada uno.

\subsection{Interfaz de usuario}

Como parte del diseño y desarrollo de la aplicación propuesta como solución se
realizó una prueba de interfaz de usuario con alumnos de la carrera de
ingeniería en informática de la Facultad Politécnica de la Universidad Nacional
de Asunción, estas pruebas fueron realizadas con personas que están
acostumbradas al uso de interfaces similares y que de hecho pueden ser mas
criticas a la hora de evaluarlas. Esta prueba se explica en detalle en el
capítulo~\ref{chap:evaluacion} y los resultados en el capítulo~\ref{chap:analisis}.

Principalmente son dos las cualidades de una interfaz gráfica que se pueden
someter a prueba: su funcionalidad y su usabilidad. Con la primera se pretende
responder preguntas como ¿Se puede usar cierta función?, ¿Funciona como se
espera?, o ¿Es correcta?; y con la segunda, a ¿Puede el usuario utilizar
fácilmente la función?, o ¿Su uso es intuitivo y fácil de
aprender?\cite{fragaverificacion}.

Las pruebas de interfaces de usuario ayudan a que los usuarios puedan
concentrarse mas en el problema en vez de poner los esfuerzos en recordar todas
las opciones que ofrece la aplicación que se utiliza para resolver el
problema\cite{horowitz1993graphical}.

Luego de las pruebas con usuarios con experiencia en interfaces móviles, se
hicieron correcciones a los problemas encontrados en la interfaz, los mayores
inconvenientes fueron con respecto a la usabilidad y la interacción tanto con el
entorno como con los objetos dentro de la simulación Estas correcciones, como
paso posterior, fueron probadas por profesores de la carrera de enfermería del
Instituto Andrés Barbero los cuales dieron su visto bueno.

Otra de las razones por las cual la prueba fue realizada con alumnos que no
formaban parte de la población a la que iba dirigida la aplicación, es la poco
disponibilidad de tiempo con la que cuentan los alumnos de enfermería y mas aun
los profesionales que están encargados de su aprendizaje.

\subsection{Validaciones de contenido}

Llamamos validación de la simulación o la aplicación desarrollada al hecho de
que el contenido de la misma sea correcto y además que la forma de realizar o
representar dicho procedimiento este acorde al mismo. Este tipo de validaciones
fueron realizadas reiteradamente en reuniones con distintos profesores de la
carrera de enfermería del Instituto Andrés Barbero.

Cada corrección solicitada fue evaluada y aprobada posteriormente por los
mismos. Como validación final la aplicación fue presentada en totalidad frente a
un plenario de cuatro profesores del instituto.

El mayor inconveniente en cuanto a las validaciones fueron la forma de
representación tanto de la información como de la simulación de objetos.
