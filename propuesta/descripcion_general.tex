\section{Descripción general de la solución}

Una vez descriptos los fundamentos teóricos del uso de videojuegos en el ambiente educativo sobre todo en aquellos que requieren de mucha practica para plasmar los conocimientos. Siendo el área de enfermería una de estas áreas definimos las problemáticas actuales de esta y seleccionamos algunos procedimientos que serán utilizados para el contenido de la solución. A continuación se busca converger todos los aspectos descriptos en
capítulos anteriores.

La solución propuesta en este trabajo consiste en el desarrollo de una aplicación para dispositivos móviles que se define como un juego serio llamado YAVE el cual incluye ciertos aspectos de gamification. El juego consiste en ofrecer a los usuarios, en este caso alumnos de enfermería, un medio en el cual puedan realizar procedimientos de enfermería y el cual les puede servir como herramienta de apoyo en su aprendizaje.

YAVE permitirá al usuario poder seleccionar el procedimiento que quiera realizar, en cada procedimiento se le dará la posibilidad de interactuar con un paciente y con una conjunto de objetos que forman parte de las herramientas requeridas para realizar el procedimiento en cuestión. Además, le permitirá realizar acciones de bioseguridad.

La aplicación no solo le permitirá al usuario realizar los procedimientos para poner en practica sus conocimientos sobre el mismo sino también evaluará al usuario, dándole al final de la partida una puntuación final y diciéndole cuales son los pasos que realizó de manera correcta y cuales de manera incorrecta, proporcionándole una pequeña información sobre su error.