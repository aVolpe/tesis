\section{Encuesta para evaluar el conocimiento}
%\renewcommand{\labelenumi}{\arabic{enumi}.} 
\renewcommand{\labelenumii}{\arabic{enumii})}

\subsection{Datos del alumno}
\tabitem{} \textbf{Nombres y Apellidos:}

\subsection{Extracción de sangre}

Todas las preguntas se limitan al contexto de la extracción de sangre.

\begin{enumerate}[label=\bfseries OE\arabic*.:]
\item El torniquete debe ser extraído
    \begin{enumerate}
    \item Antes de punzar la jeringa
    \item Después de extraer la sangre
    \item Después de retirar la jeringa
    \item Antes de extraer la sangre
    \item Ninguna de las anteriores
    \end{enumerate}
\item Los guantes deben calzarse
    \begin{enumerate}
    \item Después de lavarse las manos.
    \item Antes de ponerse la bata.
    \item Antes de ponerse gorro y tapaboca.
    \item Después de ponerse la bata.
    \item Después de ponerse gorro y tapaboca.
    \item 1, 2 y 3 son correctas.
    \item 1, 4 y 5 son correctas.
    \item 1, 2 y 5 son correctas.
    \end{enumerate}
\item Se debe solicitar al paciente que abra la mano/puño
    \begin{enumerate}
    \item Después de extraer la sangre.
    \item Antes de punzar con la jeringa.
    \item Después de punzar con la jeringa.
    \item Antes de extraer la sangre.
    \item Después de extraer la jeringa.
    \end{enumerate}
\item El equipo de protección personal~(EPP) se compone de
    \begin{enumerate}
    \item Guantes
    \item Guantes, bata, tapaboca, gafas
    \item Guantes, bata, tapaboca, gorro
    \item Guantes, bata, gafas, gorro
    \item Guantes, tapaboca, gafas, gorro
    \end{enumerate}
\item De las siguientes opciones, cual es la primera que se realiza en un
    procedimiento de extracción de sangre?
    \begin{enumerate}
    \item Calzar guantes
    \item Explicar procedimiento al paciente
    \item Ubicar zona de punción
    \item Equiparse con los elementos de protección personal
    \item Limpiarse las manos
    \end{enumerate}
\end{enumerate}

\subsection{Evaluación utilizando la escala de Glasgow}

Se considera que:
\begin{itemize}
\item RM=Respuesta Motora
\item RO=Respuesta Ocular
\item RV=Respuesta Verbal
\end{itemize}

\begin{enumerate}[label=\bfseries OG\arabic*.:]
\item Si un paciente responde de manera incorrecta las preguntas que se le
    formulan, abre los ojos después de un estímulo doloroso y localiza ese
    estímulo. Su evaluación es:
\begin{enumerate}
 \item RM=4, RO=2 y RV=3
 \item RM=3, RO=3 y RV=2
 \item RM=5, RO=2 y RV=4
 \item RM=5, RO=4 y RV=4
 \item Ninguna de las anteriores.
\end{enumerate}

\item Un paciente con RM=2, RO=2 y RV=3. Está en el siguiente estado:
\begin{enumerate}
 \item Responde con gruñidos, abre los ojos ante un pedido verbal, realiza una
     flexión anormal de los miembros.
 \item Realiza extensión anormal de los miembros, abre los ojos ante un estimulo
     doloroso y responde verbalmente de manera inapropiada.
 \item Responde verbalmente de manera orientada, abre los ojos espontáneamente y
     evita estímulos dolorosos.
 \item Abre los ojos y mueve los miembros ante un orden verbal, responde
     verbalmente de manera confusa.
 \item Ninguna de las anteriores.
\end{enumerate}
\item Un paciente que abre los ojos ante un estimulo verbal, tiene una
    valoración ocular de:
\begin{enumerate}
 \item 1
 \item 2
 \item 3
 \item 4
 \item 5
 \item 6
\end{enumerate}
\item Un paciente que evita estímulos dolorosos, tiene una valoración motora de:
\begin{enumerate}
 \item 1
 \item 2
 \item 3
 \item 4
 \item 5
 \item 6
\end{enumerate}

\item Un paciente que responde correctamente a la pregunta: ¿Qué día es hoy?,
    tiene una evaluación verbal de:
\begin{enumerate}
 \item 1
 \item 2
 \item 3
 \item 4
 \item 5
 \item 6
\end{enumerate}



\end{enumerate}
