\section{Encuesta sobre apreciación}

\subsection{Datos del alumno}
\tabitem{} \textbf{Nombres y Apellidos:}

\subsection{Apreciación}

Se utiliza la escala de Likert de 7 puntos para las respuestas de las preguntas
sobre la apreciación. La escala utilizada es:

\begin{itemize}
    \item Totalmente en desacuerdo
    \item En desacuerdo
    \item Parcialmente en desacuerdo
    \item Neutral
    \item Parcialmente de acuerdo
    \item De acuerdo
    \item Totalmente de acuerdo
\end{itemize}

\begin{enumerate}[label=\bfseries SC\arabic*.:]
\item Las opciones de bioseguridad y explicar procedimiento son suficientes para
recordar las acciones que se deben realizar en la extracción de sangre

\item El estado del enfermero es visible y comprensible en todo momento mediante las
imágenes de abajo a la derecha


\item La utilización de los elementos (torniquete, jeringa, etc) es fácil e intuitiva.


\item Los elementos (torniquete, jeringa, etc) representan correctamente las funciones
que realizan en la vida real

\item Las acciones que se realizan con los elementos (torniquete, jeringa, etc) son
suficientes para el procedimiento de extracción de sangre

\item Los resultados proveen suficiente información para comprender por qué un paso
se realizo incorrectamente


\item Es importante mostrar los detalles de los pasos incorrectos ya que no sería
suficiente sólo decir qué pasos se hicieron correctamente.

\item Glasgow. El estado aleatorio del paciente ayuda a poner a prueba los
conocimientos

\item Glasgow. El estado aleatorio del paciente ayuda a que el procedimiento se
acerque más a la realidad

\item Glasgow. La respuesta ocular se puede medir correctamente con el parpadeo
del paciente en el juego.

\item Glasgow. La respuesta verbal se puede medir correctamente con las respuestas
a las preguntas que da el paciente

\item Glasgow. La respuesta motora se puede medir correctamente con los
movimientos que realiza el paciente

\item Glasgow. Se distinguen correctamente los diferentes estados del paciente

\item La opción de hablar para que aparezca el menú de ordenes verbales hace que el
juego sea mas interactivo, siendo más similar a la realidad

\item La escenografía en los juegos permite que entremos en ambiente para realizar
los procedimientos.

\item Los gráficos en tres dimensiones nos ayudan a entender mejor el entorno y las
posibles acciones

\item La falta de pistas durante el desarrollo de una partida permite plasmar y medir el
conocimiento acerca del tema

\item Es importante dar una puntuación total para ver el rendimiento

\item Motiva compartir y ver el progreso con otras personas a través del Facebook

\item La cantidad de tiempo de cada procedimiento jugado, motiva a seguir jugando
para mejorarlo

\item El puntaje de cada procedimiento jugado, motiva a seguir jugando para
mejorarlo

\item Interactuar con un paciente que reacciona a mis acciones, es mejor que utilizar
un maniquí inmóvil

\item La utilización de la simulación en todo momento provee más facilidades para
poner en práctica los conocimientos con respecto a las demás alternativas (libro,
laboratorio, campo de prácticas).

\item La utilización de herramientas alternativas, como la simulación, es útil para
complementar el estudio en clase y laboratorio

\item La aplicación ayuda a entender el procedimiento y a memorizar los pasos.

\item La simulación permite sentirse parte del laboratorio

\item Juegos cortos permiten jugar varias veces de seguido

\end{enumerate}

\subsection{Preguntas abiertas}

\begin{enumerate}[label=\bfseries SL\arabic*.:]
    \item ¿En qué cree beneficia y en que perjudica el uso de estas herramientas
        como apoyo al aprendizaje?
    \item ¿Cree que son necesarios métodos para complementar al
        laboratorio/aula?, si es sí, ¿Que tipo de herramientas podrían ayudar?
    \item ¿Que factores le impidió utilizar más a menudo la aplicación?
    \item ¿Qué le hubiera gustado que incluyera la aplicación?, con respecto al
        contenido de las escenas, escenas simuladas y aspectos generales
\end{enumerate}
