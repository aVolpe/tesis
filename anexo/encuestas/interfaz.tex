\section{Encuesta preliminar de usabilidad de interfaz}

Se busca conocer la experiencia con la aplicación y si se considera a la misma
como una herramienta útil como apoyo a los estudiantes.

Se utiliza la escala de Likert de 7 puntos para las respuestas de las preguntas
sobre la apreciación. La escala utilizada es:

\begin{itemize}
    \item Totalmente en desacuerdo
    \item En desacuerdo
    \item Parcialmente en desacuerdo
    \item Neutral
    \item Parcialmente de acuerdo
    \item De acuerdo
    \item Totalmente de acuerdo
\end{itemize}

\begin{enumerate}
\item En cuanto a la aplicación \gls{nombre} ¿Le parece que la simulación puede
    ser útil para apoyar el entrenamiento de profesionales en Enfermería?
\item En cuanto a la aplicación \gls{nombre} ¿Le parece que la utilización de
    dispositivos móviles puede ser útil para apoyar el entrenamiento de
    profesionales en Enfermería?
\item En cuanto a la aplicación \gls{nombre}. ¿Le parece que es realmente útil
    implementarlo en el campo de enfermería?
\item En cuanto al uso de la aplicación \gls{nombre}. ¿El uso de este tipo de
    herramienta le parece que aumenta más el interés de los alumnos en los temas
    que implementa?
\item En cuanto al contenido de la aplicación \gls{nombre}. ¿Le parece
    suficiente la cantidad de procedimientos que implementa?
\item En cuanto al uso de la aplicación \gls{nombre}. ¿La interfaz dificultó el
    uso de la aplicación?
\item En cuanto al uso de la aplicación \gls{nombre}. ¿Le motivó la existencia
    de una puntuación total, compartir resultado y tiempo por sesión?
\item En cuanto al uso de la aplicación \gls{nombre}. ¿Las acciones se realizan
    de manera similar a la realidad?
\item En cuanto al aspecto pedagógico. ¿Le ayudo a entender el procedimiento y a
    memorizar los pasos?
\item En cuanto al aspecto pedagógico. ¿Le ayudo a entender el procedimiento y a
    memorizar los pasos?
\item ¿Qué prácticas de enfermería cree que son ideales para la simulación?
\item En general. ¿Qué le hubiera gustado que incluyera la aplicación?
\end{enumerate}

\subsection{Resultados}

\begin{table}[H]
\centering
\begin{tabulary}{\textwidth}{cccccccccccccc}
\toprule
\textbf{N}      & \textbf{1} & \textbf{2} & \textbf{3} & \textbf{4} & \textbf{5}
& \textbf{6} & \textbf{7} & \textbf{8} & \textbf{9} & \textbf{10} & \textbf{11}
& \textbf{12} & \textbf{13} \\
\midrule
\textbf{1}      & 6 & 6 & 6 & 6 & 5 &   & 5 & 6 & 5 & 5  & 6  & 6  & 6  \\
\textbf{2}      & 5 & 5 & 4 & 4 & 5 & 6 & 5 & 6 & 5 & 6  & 7  & 6  & 6  \\
\textbf{3}      & 5 & 5 & 5 & 6 & 7 & 5 & 6 & 6 & 7 & 6  & 7  & 6  & 7  \\
\textbf{4}      & 6 & 6 & 7 & 7 & 7 & 6 & 6 & 7 & 5 & 6  & 5  & 7  & 7  \\
\textbf{5}      & 5 & 6 & 2 & 5 & 6 & 6 & 6 & 6 & 6 & 2  & 7  & 7  & 7  \\
\textbf{6}      & 6 & 6 & 5 & 6 & 5 & 6 & 7 & 7 & 6 & 4  & 6  & 6  & 6  \\ 
\textbf{7}      & 6 & 7 & 5 & 7 & 6 & 6 & 6 & 6 & 6 & 5  & 7  & 5  & 6  \\ 
\textbf{8}      & 6 & 6 & 6 & 7 & 7 & 6 & 6 & 6 & 5 & 7  & 7  & 7  & 7  \\ 
\bottomrule
\end{tabulary}
\caption{Apreciación de los alumnos por pregunta} 
\end{table}

\subsubsection{Observaciones}

\begin{enumerate}
\item Me perdí en algún momento y costo volver a retomar la practica
\item La mano apareció varias veces al ejercer presión, algunos efectos de audio
    harían mas claro el proceso, y el zoom fue muy difícil de manejar
\item 
\item Solo mejorar el gráfico cuando se utiliza el torniquete
\item 
\item Se podría dar mas herramientas (parches, algodón, alcohol). Una mejor
    precisión. No entendí si la simulación entendía donde esta la vena como
    zonas, al termina la simulación si se podría dar recomendaciones o ver para
    que casos sería la extracción, si se necesita cuantos ml. Sería como mejoras
    a futuro para la simulación.
\item Cuesta saber la posición exacta donde poner la jeringa. Faltaba indicar
    como esterilizar la zona y una indicación para que el paciente abra las
    manos y cierre las manos, en esas tres partes el sistema no daba mucha
    facilidad. Lo de lavarse las manos, la bata y los guantes esta super bien
    intuitivamente.
\item La parte de hablar y solicitar al paciente que cierre el puño no me
    pareció intuitivo, tampoco el de desechar la jeringa.
\end{enumerate}

