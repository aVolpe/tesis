\section{Reunión \#4}\label{reuniuxf3n-4}

\begin{itemize}
\itemsep1pt\parskip0pt\parsep0pt
\item
  \textbf{Fecha} 08 de Mayo de 2014.
\item
  \textbf{Presentes} Prof.~Miguela Hermosilla, Arturo Volpe, Mirta
  González y Prof Matilde (laboratorio)
\item
  \textbf{Motivo} Presentación de laboratorios de prácticas
\item
  \textbf{Lugar} Secretaría general del departamento de enfermería y
  laboratorios de prácticas de estudiantes de enfermería y obstetricia.
\end{itemize}

\subsection{Objetivos}\label{objetivos}

\begin{itemize}
\itemsep1pt\parskip0pt\parsep0pt
\item
  Validación de la navegación actual de la simulación.
\item
  Exploración de los laboratorios y observación de prácticas actuales de
  los estudiantes de enfermería.
\end{itemize}

\subsection{Desarrollo}\label{desarrollo}

La reunión se llevo a cabo en dos partes:

\subsubsection{Primera reunión}\label{primera-reuniuxf3n}

La primera reunión se llevo a cabo en la secretaría de la carrera de
enfermería y cumplió con el primer objetivo, la validación de la
navegación actual.

\paragraph{Observaciones}\label{observaciones}

\begin{itemize}
\itemsep1pt\parskip0pt\parsep0pt
\item
  Es necesario un acceso directo a la parte específica del cuerpo donde
  se esta
\item
  Sala:

  \begin{itemize}
  \itemsep1pt\parskip0pt\parsep0pt
  \item
    En general, Es necesario más realismo en la escena de la simulación.
  \item
    Agregar ventanas con los colores típicos de un hospital (amarillos,
    ver fotos de la segunda reunión) realizando la práctica.
  \end{itemize}
\item
  Camilla:

  \begin{itemize}
  \itemsep1pt\parskip0pt\parsep0pt
  \item
    Eliminar las barandas de la camilla actual (mientras más sencilla la
    camilla mejor)
  \item
    La camilla debe estar más alta.
  \end{itemize}
\item
  Pacientesente:

  \begin{itemize}
  \itemsep1pt\parskip0pt\parsep0pt
  \item
    Remera mangas cortas
  \item
    Sin Anteojos
  \end{itemize}
\item
  Simulación:

  \begin{itemize}
  \itemsep1pt\parskip0pt\parsep0pt
  \item
    Tomar en cuenta la presión de la sangre
  \end{itemize}
\end{itemize}

\subsubsection{Segunda reunión}\label{segunda-reuniuxf3n}

La segunda reunión se llevo a cabo en los laboratorios de enfermaría con
la Prof Matilde, la cual es profesora de laboratorio del primer semestre
de la carrera de enfermería, se nos presentaron tres laboratorios, los
dos primeros de arquitectura similar pero diferente propósito, el
primero es un laboratorio/sala convertido en un aula donde los alumnos
observan maquetas muy detalladas del cuerpo humano y tienen sus primeras
prácticas bajo supervisión de un profesor.

La segunda sala es un laboratorio/sala que cuenta con numerosas camas
donde los alumnos práctican todo lo referente al mantenimiento adecuado
de las camas, cuenta además con maniquís que los estudiantes utilizan
para interactuar con un cuerpo, el mismo tiene una contextura similar al
de un cuerpo humano y varias partes marcadas con alertas visuales sobre
puntos de referencia, como por ejemplo donde se debe vacunar, donde se
debe realizar la reanimación, donde están las venas donde se pueden
ingresar vías, etc.

La tercera sala es un laboratorio de obstetricia, en el cual se pueden
ver varios maniquíes de bebes y partes sexuales de la mujer, todo lo
necesario para poder simular un parto.

\paragraph{Observaciones}\label{observaciones-1}

\begin{itemize}
\itemsep1pt\parskip0pt\parsep0pt
\item
  Los alumnos en el primer laboratorio observan, tocan y palpan los
  brazos falsos para poder saber donde están las venas importantes para
  la instalación de vías. Aquí además aprenden donde se deben ubicar
  cuando se acercan a un paciente, donde debe estar el lugar estéril
  para depositar los elementos y como preparar el equipo necesario.
\item
  El segundo laboratorio además cuenta con esterilizadores, donde los
  alumnos aprenden conceptos básicos sobre la esterilización (la teoría
  se da en otra materia), aquí más bien se manipula el esterilizador
\item
  Existen varios tipos de vías que pueden ser utilizados para una vena,
  y cada vena a su vez es capaz de aguantar ciertos tipos de vías,
  siendo las venas de las manos las que requieren vías más pequeñas. El
  único tipo de vía que no es colocado por el enfermero es la vía
  central (requiere de un anestesiologo).
\end{itemize}
