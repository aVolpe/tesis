\section{Reunión \#1}\label{reuniuxf3n-1}

\begin{itemize}
\itemsep1pt\parskip0pt\parsep0pt
\item
  \textbf{Fecha} 12 de Diciembre de 2013
\item
  \textbf{Presentes} Miguela Hermosilla, Arturo Volpe, Mirta González
\item
  \textbf{Motivo} Presentación de propuesta y búsqueda de apoyo.
\item
  \textbf{Lugar} Secretaría de la carrera de Enfermería, Instituto
  Andrés Barbero
\end{itemize}

\subsection{Objetivos}\label{objetivos}

\begin{itemize}
\itemsep1pt\parskip0pt\parsep0pt
\item
  Presentar idea de tésis
\item
  Presentar la idea como un complemento a la educación actual
\item
  Investigar posibles áreas de aplicación de la idea
\item
  Investigar mecanismos de medición
\item
  Validación de la investigación previa.
\end{itemize}

\subsection{Desarrollo}\label{desarrollo}

Los interesados se presentaron en la secretaría del Instituto Andes
Barbero (IAB), para poder hablar con la directora de la carrera, la que
se presento como Mgs. Miguela Hermosilla.

Se procedió a la explicación de los objetivos y deseos por parte de los
interesados, Miguela parecía muy interesada y dispuesta a ayudar, los
puntos claves que menciono son:

\begin{itemize}
\itemsep1pt\parskip0pt\parsep0pt
\item
  No se debe enfocar como la utilización de redes Sociales, pues el IAB
  no promueve la utilización de las mismas por problemas internos.
\item
  Los alumnos del 1ero y 2do año tienen un laboratorio especializado
  donde realizan pruebas empíricas con muñecos, incluso deben pasar
  exámenes prácticos antes de poder aprobar la materia con exámenes
  \textbf{teóricos}.
\item
  En el 3er año se realiza el primer contacto con los pacientes, en el
  cual tan solo dialogan con los pacientes.
\item
  En el 4to año los alumnos acceden a quirofano, ya teniendo experiencia
  previa con los muñecos.
\item
  Existen tesis de alumnos de enfermería que hablan de la utilización de
  la tecnología para facilitar la educación.
\item
  Existen dos tipos de examenes:

  \begin{itemize}
  \itemsep1pt\parskip0pt\parsep0pt
  \item
    Prácticos, donde los alumnos prueban con muñecos y un instructor se
    encarga de medir la pericia del mismo.
  \item
    Teóricos: a través de un examen \textbf{tradicional}
  \end{itemize}
\item
  Anualmente ingresan 150 nuevos estudiantes.
\item
  Los instructores utilizan elementos distractores en sus exámenes para
  poder medir la capacidad del alumno de detectar los temas que se
  consideran más importantes
\end{itemize}

\subsection{Recomendaciones sobre la
tesis}\label{recomendaciones-sobre-la-tesis}

\begin{itemize}
\itemsep1pt\parskip0pt\parsep0pt
\item
  La población recomendada para el estudio son los estudiante de 3er/4to
  año.
\item
  La forma de medición recomendada es dotar a un grupo la tecnología y
  utilizar otros dos como grupos de control.
\item
  Cada grupo tiene un conjunto de instructores que se encargan de
  enseñar y medir la capacidad del alumno.
\item
  Posibles áreas para la simulación:

  \begin{itemize}
  \itemsep1pt\parskip0pt\parsep0pt
  \item
    Cuidados críticos (pediatria, cuidado de adultos)
  \item
    Cirugía
  \item
    Traumatologia
  \end{itemize}
\item
  Además menciono (con ayuda de colegas), las áreas que más
  inconvenientes genera en alumnos:

  \begin{itemize}
  \itemsep1pt\parskip0pt\parsep0pt
  \item
    Bio-seguridad
  \item
    Preparación de drogas
  \item
    Difusión de drogas
  \item
    Ética
  \end{itemize}
\item
  Se puede empezar el estudio en abril, donde los grupos empiezan, se
  recomienda empezar con el 2do grupo , que empieza la segunda semana de
  abril y la medición se puede realizar en Junio. Los motivos de elegir
  el segundo grupo son:

  \begin{itemize}
  \itemsep1pt\parskip0pt\parsep0pt
  \item
    El primero no esta lo suficientemente preparado.
  \item
    El tiempo del tercer grupo se solapa con exámenes y los alumnos no
    están en peores condiciones..
  \end{itemize}
\end{itemize}

\subsection{Otras recomendaciones}\label{otras-recomendaciones}

\begin{itemize}
\itemsep1pt\parskip0pt\parsep0pt
\item
  Reunirnos con los instructores de cada grupo el día 20-12-2013 para
  presentar la idea y buscar apoyo
\item
  Investigar sobre las tesis relacionadas en el área para averiguar
  cuales son las áreas que más cuestan.
\item
  Enviar un mail para solicitar más información y confirmar la reunión
  con los instructores.
\end{itemize}

\subsection{Actividades}\label{actividades}

\begin{itemize}
\itemsep1pt\parskip0pt\parsep0pt
\item
  Enviar correo para verificar disponibilidad de instructores.
\item
  Decidir sobre cual tema se realizará el trabajo y notificar para que
  Miguela pueda ayudarnos.
\end{itemize}
