\section{Reunión \#2}\label{reuniuxf3n-2}

\begin{itemize}
\itemsep1pt\parskip0pt\parsep0pt
\item
  \textbf{Fecha} 27 de Diciembre de 2013.
\item
  \textbf{Presentes} Miguela Hermosilla, Arturo Volpe, Mirta González.
\item
  \textbf{Motivo} Presentación de ítems pre-seleccionados y evaluación
  de los mismos por la Profesora Miguela Hermosilla.
\item
  \textbf{Lugar} Secretaría de la carrera de Enfermería, Instituto
  Andrés Barbero.
\end{itemize}

\subsection{Objetivos}\label{objetivos}

\begin{itemize}
\itemsep1pt\parskip0pt\parsep0pt
\item
  Obtener la valoración de un profesional de los elementos a simular.
\item
  Investigar en que materia aprenden a mezclar medicamentos.
\item
  Obtener una reunión con los instructores, pues serán ellos quienes
  realizarán las pruebas.
\end{itemize}

\subsection{Desarrollo}\label{desarrollo}

Los interesados se presentaron en la secretaría del Instituto Andes
Barbero (IAB), para reunirse con la directora de la carrera, Mgs.
Miguela Hermosilla.

Se expusieron los items pre-seleccionados para la simulación con el fin
de que puedan ser valorados.

\subsubsection{Items presentados}\label{items-presentados}

\begin{enumerate}
\def\labelenumi{\arabic{enumi}.}
\itemsep1pt\parskip0pt\parsep0pt
\item
  Interpretación de la escala de \emph{glasgow} (Enfermería en cuidados
  intensivos, 4to año).
\item
  Reanimación cardiopulmonar básica y avanzada, distinción de los
  fármacos más utilizados (Enfermería en cuidados intensivos, 4to año).
\item
  Test de \emph{apgar} (Pediatría, 3er año).
\item
  Mezcla y administración de fármacos.
\item
  Conocimiento de las normas y medidas prácticas referentes a la
  bioseguridad.
\end{enumerate}

\textbf{Además se consultaron los siguientes puntos:}

\begin{enumerate}
\def\labelenumi{\arabic{enumi}.}
\itemsep1pt\parskip0pt\parsep0pt
\item
  Identificación de las características anatómicas y fisiológicas del
  recién nacido (Salud del niño y del adolescente, 3er año).
\item
  Demostrar destreza en la atención inmediata del recién nacido (Salud
  del niño y del adolescente, 3er año).
\end{enumerate}

\subsection{Recomendaciones del
profesional}\label{recomendaciones-del-profesional}

\begin{enumerate}
\def\labelenumi{\arabic{enumi}.}
\itemsep1pt\parskip0pt\parsep0pt
\item
  Este ítem puede ser aplicado para los alumnos del 3er o 4to año. Posee
  práctica. \textbf{Su orden de importancia es 3.}
\item
  Este ítem puede ser aplicado para los alumnos del 3er o 4to año.
  \textbf{Su orden de importancia es 1}, esto es debido a que este ítem
  tiene cero práctica en la actualidad, por que, la malla curricular no
  cuenta con temas relacionados a los primeros auxilios.
\item
  Este ítem posee prácticas, sin embargo, enfermería sólo se encargaría
  de la valoración del test. Puede ser aplicado para los alumnos del 3er
  o 4to año. \textbf{Su orden de importancia es 4}.
\item
  Este ítem no necesita pericia. Puede ser aplicado para los alumnos del
  3er año. \textbf{Su orden de importancia es 5.}
\item
  Este ítem se considera transversal, es decir, bioseguridad es
  importante en cada procedimiento. Incluye: protección, lavado de
  manos, asepsia, anti-sepsia. Es el que más práctica tiene. Puede ser
  aplicado para alumnos de 3er año. \textbf{Su orden de importancia es
  2}.
\end{enumerate}

\emph{Cabe mencionar que el orden de importancia del ítem va de 1 a 5,
siendo el 1 el indicador de mayor valor.}

\textbf{En cuanto a los demás puntos, mencionó:}

\begin{enumerate}
\def\labelenumi{\arabic{enumi}.}
\itemsep1pt\parskip0pt\parsep0pt
\item
  Este ítem está incluido en 2 (Reanimación cardiopulmonar). Puede ser
  aplicado para los alumnos del 4to año.
\item
  Este ítem depende de a que se refiera, puede referirse a:

  \begin{itemize}
  \itemsep1pt\parskip0pt\parsep0pt
  \item
    la recepción del recién nacido,
  \item
    instalación de un vía,
  \item
    instalación de sonda nasogástrica.
  \end{itemize}

  Puede ser aplicado para alumnos del 4to año. Su nivel de complejidad
  depende del punto que se tome en cuenta, así como el enfoque del
  mismo. Por ejemplo, la \texttt{recepción de un recién nacido} es un
  conocimiento importante pero con el cual se cuenta práctica, en cambio
  la \texttt{instalación de una vía} es un proceso sumamente complejo,
  que no cuenta con práctica.
\end{enumerate}

\subsection{Otras informaciones}\label{otras-informaciones}

\begin{itemize}
\itemsep1pt\parskip0pt\parsep0pt
\item
  Todas las pericias son evaluadas por un instructor.
\item
  Existen 4 estudiantes por instructor en las áreas críticas (cuidados
  intensivos, urgencias). Un área critica es aquella en la cual el
  paciente depende exclusivamente de un procedimiento externo.
\item
  Existen 10 estudiantes por instructor en las áreas no críticas.
\item
  Los instructores definen las cosas que evalúan. Cuentan con una
  planilla por alumnos para el seguimiento de las pericias (si son
  logradas o no, incluye los procedimientos).
\item
  Los instructores no están en enero.
\item
  La Sra. Miguela Hermosilla se encontrará en las instalaciones del
  Instituto Andrés Barbero del 6 al 10 de enero de 8:00 a 13:00 horas.
  Debemos reunirnos en alguna de estas fechas con ella para que nos
  pueda presentar a algún instructor que se encuentre, con el fin de que
  podamos reunirnos con él.
\end{itemize}

\subsection{Conclusiones}\label{conclusiones}

\begin{itemize}
\itemsep1pt\parskip0pt\parsep0pt
\item
  Programar una reunión con los instructores, para ello se puede
  utilizar los días que la profesora Miguela no este de vacaciones y
  pedir su ayuda como contacto directo (si existe algún instructor
  presente).
\item
  Obtener la validación de un instructor acerca de los elementos a
  simular.
\end{itemize}
