\section{Reunión \#3}

\begin{itemize}
\itemsep1pt\parskip0pt\parsep0pt
\item
  \textbf{Fecha} 08 de Enero de 2014.
\item
  \textbf{Presentes} Prof.~Gloria Mora, Arturo Volpe, Mirta González.
\item
  \textbf{Motivo} Primer encuentro con un instructor
\item
  \textbf{Lugar} MECIP, Instituto Andrés Barbero.
\end{itemize}

\subsection{Objetivos}

Reunión con instructores para obtener su valoración acerca de los ítems
pre-seleccionados para simular, la situación actual de los alumnos de
enfermería y otras informaciones relevantes que nos pueda brindar.

\subsection{Desarrollo}

Los interesados acudieron a la Secretaría de la carrera de Enfermería en
el Instituto Andrés Barbero, donde fueron guiados por una secretaría
hasta el MECIP, donde se llevo a cabo la reunión con la profesora Gloria
Mora.

Los alumnos presentaron la propuesta y los ítems pre-seleccionados hasta
el momento.

La profesora Gloria Mora, es instructora en la materia ``Enfermería de
Urgencias'', encargada de los alumnos que atienden adultos que van al
hospital de Emergencias Médicas.

\subsubsection{Comentarios de la
profesora}

\begin{itemize}
\itemsep1pt\parskip0pt\parsep0pt
\item
  Enfermería en urgencias es una materia de cuarto curso, y se basa en
  tres competencias básicas, entre las cuales se encuentra el control de
  los signos vitales (frecuencia cardíaca, frecuencia respiratoria,
  presión arterial y temperatura).
\item
  La profesora es encargada de los alumnos durante 4 semanas, las cuales
  organiza como sigue:

  \begin{itemize}
  \itemsep1pt\parskip0pt\parsep0pt
  \item
    1 semana de prácticas en el laboratorio, si bien, según otras
    profesoras, los alumnos deben estar completamente preparados para
    las prácticas, Gloria menciona que prefiere una semana más bajo su
    supervisión para que los alumnos sepan como actuar y entiendan el
    lenguaje en el que se comunicará durante las prácticas.
  \item
    1 Semana donde la profesora guía a los alumnos en sus actividades,
    considera a esta semana como de conocimiento, exploración y
    adaptación al ambiente.
  \item
    2 Semanas durante las cuales vigila el desenvolvimiento de los
    alumnos y corrige sus actividades, al mismo tiempo que evalúa la
    pericia.
  \end{itemize}
\item
  Existe software que se utiliza para la instalación del catéter de PIC,
  de monitoreo y de soporte.
\item
  Control de los signos vitales:

  \begin{itemize}
  \itemsep1pt\parskip0pt\parsep0pt
  \item
    \textbf{Primarias}, son el control de los signos vitales, lo que
    menciono se llama el ABCDE de los signos vitales.
  \item
    \textbf{Secundarias}, la piel y daños secundarios.
  \end{itemize}
\end{itemize}

\paragraph{Evaluación}

Los alumnos del 4to curso de la materia \emph{Enfermería en Urgencias},
distribuyen la práctica como sigue:

\begin{itemize}
\itemsep1pt\parskip0pt\parsep0pt
\item
  Centro de emergencias médicas, 4 semanas, aquí es donde la profesora
  es la instructora. En este lugar atienden pacientes con accidentes y/o
  agresiones.
\item
  Urgencias en el Hospital de Clínicas, 4 semanas. En este lugar
  atienden pacientes crónicos y agudos.
\end{itemize}

Las prácticas tienen una duración de 4 horas, y son llevadas a cabo de
las 13 hasta las 17 horas (4 horas por día).

Algunas actividades que se evalúan son por ejemplo:

\begin{itemize}
\itemsep1pt\parskip0pt\parsep0pt
\item
  Posicionamiento del paciente al a hora de hacer tratamientos
  (elevación de piernas, posicionamiento de la cabeza)
\item
  Valoración del paciente (ABCDE)
\item
  Toma de sangre
\end{itemize}

Además nos entrego una copia de la hoja de evaluación, la cual es
completada por la misma para cada alumno, en la hoja se constatan las
competencias básicas y la progresión de los procedimientos en los
alumnos.

\paragraph{Tipos de alumnos}

\begin{itemize}
\itemsep1pt\parskip0pt\parsep0pt
\item
  \textbf{Desinteresado}, no muestra interés y escapa conscientemente de
  las prácticas. (\textasciitilde{}25\%)
\item
  \textbf{Introvertido}, es difícil que ayude, pero una vez que empieza
  a ayudar siempre ayuda sin problemas (\textasciitilde{}25\%)
\item
  \textbf{Extrovertidos}, ayudan sin problemas y se muestran interesados
  ante las enseñanzas, sienten que aplican la teoría
  (\textasciitilde{}50\%).
\end{itemize}

\paragraph{Tecnología}

Los alumnos deben estar familiarizados con las máquinas que sirven para
medir diferentes aspectos del estado de un paciente, y los que se
utilizan para diferentes tratamientos (como ejemplo un desfribilador).

Además, tienen permitido utilizar el celular siempre y cuando no estén
en la sala (pueden ir al baño, en la sala de descansos o cuando tienen
tiempo para comer).

\subsubsection{Otros}

Ante consultas sobre los elementos seleccionados, mostró un especial
interés por la escala de glasgow.
