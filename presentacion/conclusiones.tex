\section{Conclusiones}
\setcounter{sectiontotal}{7}

% Estado del arte
\begin{frame}
\frametitle{\pagetitle}
\framesubtitle{Estado del arte}
\begin{itemize}[<+->]

% REVISAR
\item Beneficios no explotados

\item El construccionismo y el constructivismo son pedagogías útiles para el
desarrollo de habilidades profesionales

\item Los juegos serios permiten una experiencia sin riesgos

%\item Los juegos serios actuales ofrecen una retroalimentación muy guiada

\item La enseñanza de enfermeros es un área propicia para la
aplicación de los juegos serios

\end{itemize}
\end{frame}

\begin{frame}
\frametitle{\pagetitle}
\framesubtitle{Contexto de aplicación}
\begin{itemize}[<+->]
\item En el IAB, los juegos serios poseen potencial para resolver los problemas de la formación de los profesionales de enfermería

\item Se debe brindar herramientas alternativas de bajo costo
\end{itemize}
\end{frame}

% Diseño del juego serio
\begin{frame}
\frametitle{\pagetitle}
\framesubtitle{Diseño}
\begin{itemize}[<+->]


\item Aspectos pedagógicos con profesores de práctica y directores de
        carrera

\item El uso de un puntaje por procedimiento motiva a los usuarios

\item Los estados aleatorios y funciones simplificadas facilitan la exploración

\item Los gráficos en tres dimensiones y partidas cortas aumentan la inmersión del usuario

\item Se debe proveer retroalimentación sólo al finalizar la partida

\item Se debe ofrecer retroalimentación detallada

\item Se deben utilizar indicadores de realización de acción

\item Se debe limitar la manipulación del punto de vista al utilizar elementos

\end{itemize}
\end{frame}

% Implementación del juego serio
\begin{frame}
\frametitle{\pagetitle}
\framesubtitle{Implementación}
\begin{itemize}[<+->]

\item Diferencias principales con el desarrollo tradicional de software

\item El uso de un motor de videojuego facilita el desarrollo

\item Se recomienda tener en cuenta el costo, requisitos mínimos, familiaridad,
    librerías, tienda y comunidad al seleccionar un motor de videojuegos
    
\item El uso de un motor de reglas condicionado por eventos es suficiente

\item Es necesario evaluar al usuario en el front-end

\item Se deben diseñar personajes sólo cuando se requiere un alto nivel de
    detalle o interacción

\item Es necesario enviar automáticamente los registros de utilización 

\end{itemize}
\end{frame}

% Evaluación
\begin{frame}
\frametitle{\pagetitle}
\framesubtitle{Evaluación}
\begin{itemize}[<+->]

\item Se debe determinar la dificultad y la relevancia de los temas a tratar con los
    profesores de cátedra.

\item Es necesario registrar todas las acciones del usuario

\item Los juego serios ayudan a los estudiantes de enfermería a poner a
        prueba sus conocimientos

\end{itemize}
\end{frame}
