\section{Conclusiones}
\setcounter{sectiontotal}{3}

\begin{frame}{Aspectos de diseño}
\begin{itemize}[<+->]
    \item Validar constantemente el contenido con profesionales del área
    \item Motivar a través de la utilización de aspectos lúdicos, como el
        puntaje y la medición del tiempo
    \item Utilizar escenarios aleatorios para favorecer la exploración
    \item Proveer retroalimentación al finalizar cada escenario
    \item Proveer ubicuidad para aumentar la utilización
    \item Imágenes representativas son útiles para indicar el estado de
        entidades
\end{itemize}
\end{frame}
\begin{frame}{Desarrollo}
\begin{itemize}[<+->]
    \item Utilizar un motor de videojuegos
    \item Seleccionar herramientas de acuerdo a las características necesarias
    \item Probar el desarrollo de manera frecuente
    \item Crear entornos en tres dimensiones para aumentar la inmersión
    \item Proveer retroalimentación clara al realizar una acción
\end{itemize}
\end{frame}
\begin{frame}{Evaluación}
\begin{itemize}[<+->]
    \item Diseñar herramientas para evaluar el conocimiento de los usuarios con
        apoyo de profesionales
    \item Registrar actividades del usuario
    \item Evaluar la opinión de los usuarios
\end{itemize}
\end{frame}

