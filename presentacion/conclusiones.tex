\section{Conclusiones}
\setcounter{sectiontotal}{7}

% Estado del arte
\begin{frame}
\frametitle{\pagetitle}
\framesubtitle{Estado del arte}
\begin{itemize}[<+->]

% REVISAR
\item Beneficios no explotados

\item El construccionismo y el constructivismo son pedagogías útiles para el
desarrollo de habilidades profesionales

\item Los juegos serios permiten una experiencia sin riesgos

%\item Los juegos serios actuales ofrecen una retroalimentación muy guiada

\item La enseñanza de enfermeros es un área propicia para la
aplicación de los juegos serios

\end{itemize}
\end{frame}

\begin{frame}
\frametitle{\pagetitle}
\framesubtitle{Contexto de aplicación}
\begin{itemize}[<+->]
\item En el IAB, los juegos serios poseen potencial para resolver los problemas de la formación de los profesionales de enfermería

\item Se debe brindar herramientas alternativas de bajo costo
\end{itemize}
\end{frame}

% Diseño del juego serio
\begin{frame}
\frametitle{\pagetitle}
\framesubtitle{Diseño}
\begin{itemize}[<+->]


\item La definición del contenido debe ser realizada con profesores de 
práctica y directores de carrera

\item La motivación se incrementa al utilizar puntaje por procedimientos

\item La exploración es facilitada por los estados aleatorios

\item La inmersión es aumentada con la utilización de gráficos 3D
y partidas cortas

\item Se debe proveer retroalimentación sobre el desempeño del usuario sólo
al finalizar la partida

%\item La información sobre el rendimiento del usuario debe ser detallada

%\item Se deben utilizar indicadores de realización de acciones

\item Limitar la manipulación del punto de vista al utilizar elementos

\end{itemize}
\end{frame}

% Implementación del juego serio
\begin{frame}
\frametitle{\pagetitle}
\framesubtitle{Implementación}
\begin{itemize}[<+->]

\item El desarrollo de juegos serios difieren del desarrollo de software tradicional 
en cuanto a la interacción y el uso de gráficos en tres dimensiones

%\item El uso de un motor de videojuego facilita el desarrollo

%\item Se recomienda tener en cuenta el costo, requisitos mínimos, familiaridad,
%    librerías, tienda y comunidad al seleccionar un motor de videojuego
    
%\item El uso de un motor de reglas condicionado por eventos es suficiente

\item Es necesario evaluar al usuario sin conexión a internet

\item Es costoso diseñar personajes y entornos en tres dimensiones
%\item Se deben diseñar personajes sólo cuando se requiere un alto nivel de
%    detalle o interacción

\item Es necesario enviar automáticamente los registros de utilización 

\end{itemize}
\end{frame}

\begin{frame}[noframenumbering]
\frametitle{\pagetitle}
\framesubtitle{Selección de tecnología}
\begin{itemize}[<+->]

\item Se recomienda utilizar Unity3D 

\item Se recomienda tener en cuenta el costo, requisitos mínimos, familiaridad,
librerías, tienda y comunidad al seleccionar un motor de videojuego

\item El uso de un motor de reglas condicionado por eventos es suficiente para
evaluar al usuario

\end{itemize}
\end{frame}

% Evaluación
\begin{frame}
\frametitle{\pagetitle}
\framesubtitle{Evaluación}
\begin{itemize}[<+->]

\item Validar relevancia y dificultad de temas a tratar con los profesores de
cátedra

\item Es necesario poder reproducir las sesiones de juego del usuario con los
registros de uso

\end{itemize}
\end{frame}

% Ventajas y desventajas
\begin{frame}
\frametitle{\pagetitle}
\framesubtitle{Ventajas}
\begin{itemize}[<+->]

%\item La utilización de dispositivos móviles permite su uso en cualquier lugar y momento

\item Las soluciones basadas en dispositivos móviles son factibles de aplicación en el Paraguay

\item La solución es beneficiosa para el aprendizaje de procedimientos de enfermería según el $100\%$ de los alumnos que la evaluaron

%\item La solución agrega un nivel adicional de preparación entre las clases teóricas y la práctica con pacientes

\item Existe una recepción positiva ante la utilización de los juegos serios

%\item Los juegos serios ayudan a los estudiantes de enfermería a poner a prueba sus conocimientos

\end{itemize}
\end{frame}

\begin{frame}
\frametitle{\pagetitle}
\framesubtitle{Desventajas}
\begin{itemize}[<+->]

\item Alto costo de implementación

\item No existen generadores de contenido para juegos serios

%\item Los alumnos no cuentan con dispositivos móviles de altas prestaciones

%\item La principal dificultad para utilizar en mayor medida la solución es el factor tiempo según el $64\%$ de los alumnos que la evaluaron

\end{itemize}
\end{frame}
