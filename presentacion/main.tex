\documentclass{beamer}
\usepackage{pgfpages}
\usepackage[utf8]{inputenc}
\usepackage[spanish]{babel}

% Buscar más en http://deic.uab.es/~iblanes/beamer_gallery/individual/Malmoe-dolphin-default.html
\usetheme{Malmoe}
%\usecolortheme{dolphin}
\usecolortheme{lily}


\setbeameroption{show notes on second screen} %un-comment to see the notes


\author[Mirta González y Arturo Volpe]{Mirta González \quad Arturo Volpe}

\title[Juegos serios como apoyo a la enseñanza
tradicional\hspace{2em}\insertframenumber/\inserttotalframenumber]%
{Juegos serios como apoyo a la enseñanza tradicional:  una aplicación a la
    formación de profesionales del área de enfermería}

%\date{Mayo, 2014} %leave out for today's date to be insterted

\institute{Universidad Nacional de Asunción\\Facultad Politécnica}

\begin{document}

\frame{\titlepage}

\begin{frame}
\frametitle{There Is No Largest Prime Number}
\framesubtitle{The proof uses \textit{reductio ad absurdum}.}
\begin{theorem}
There is no largest prime number.
\end{theorem}
\begin{proof}
\begin{enumerate}
\item<1-| alert@1> Suppose $p$ were the largest prime number.
\item<2-> Let $q$ be the product of the first $p$ numbers.
\item<3-> Then $q+1$ is not divisible by any of them.
\item<1-> But $q + 1$ is greater than $1$, thus divisible by some prime
number not in the first $p$ numbers.\qedhere
\end{enumerate}
\end{proof}
\end{frame}

\begin{frame}
\frametitle{Table of Contents}
\tableofcontents[currentsection]
\end{frame}
\note{Hola como estas}

\section{Parte 1}
\begin{frame}
    \frametitle{Introducción}
\end{frame}

\section{Parte 2}
\begin{frame}
    \frametitle{TIC's en la educación}
\end{frame}

\section{Parte 3}
\begin{frame}
    \frametitle{Juegos Serios}
\end{frame}
\end{document}
