\documentclass[xcolor=table]{beamer}
\usepackage{etex}
\usepackage{pgfpages}
\usepackage[utf8]{inputenc}
\usepackage[spanish]{babel}
\usepackage{caption}
\usepackage{subcaption}
\usepackage{pgfplots}
\usepackage{tikz}
\usepackage{tkz-kiviat}
\usepackage{fixpauseincludegraphics}
\usepackage{booktabs}
\usepackage{filecontents}
\usepackage{rotating}


\usetikzlibrary{shapes,arrows}
%
%% Buscar más en http://deic.uab.es/~iblanes/beamer_gallery/individual/Malmoe-dolphin-default.html
\usetheme{Malmoe}
%%\usecolortheme{dolphin}
\usecolortheme{lily}
\setbeamercovered{transparent}
\setbeamercolor{block title}{use=structure,fg=white,bg=blue!75!white}
\setbeamercolor{block body}{use=structure,fg=black,bg=gray!20!white}
%
%
%\setbeameroption{show notes on second screen} % Descomentar para no ver comentarios

% Esto agrega un número romano a los siguientes, para romper un frame en dos
% utilizar
%       \framebreak{}
\setbeamertemplate{frametitle continuation}[from second][(\insertcontinuationcountroman)]

\AtBeginSection[]
{
  \begin{frame}
    \frametitle{Índice}
    \tableofcontents[currentsection]
  \end{frame}
}


%%%%%%%%%%%%%%%%%%%%%
% Datos de la tesis %
%%%%%%%%%%%%%%%%%%%%%

\author[Mirta González y Arturo Volpe]{Mirta González \quad Arturo Volpe}

\title[Juegos serios como apoyo a la enseñanza
tradicional\hspace{2em}\insertframenumber/\inserttotalframenumber]%
{Juegos serios como apoyo a la enseñanza tradicional:  una aplicación a la
    formación de profesionales del área de enfermería}

%\date{Mayo, 2014} %leave out for today's date to be insterted

\institute{Universidad Nacional de Asunción\\Facultad Politécnica}

\begin{document}


%\end{frame}

\frame{\titlepage}

\begin{frame}
\frametitle{Índice}
\tableofcontents
\end{frame}

%\section{Introducción}
\setcounter{sectiontotal}{3}

\begin{frame}
\frametitle{\pagetitle}
\framesubtitle{Descripción}
\begin{figure}
\includegraphics[scale=.3]{imagenes/nhanduti}
\end{figure}
\end{frame}

\begin{frame}
\frametitle{\pagetitle}
\framesubtitle{Objetivo general}
\begin{block}{Descripción}
\centering
Identificar y valorar los aspectos pedagógicos, de diseño, de implementación y
de evaluación que influyen a la creación de herramientas educativas que utilizan
las corrientes pedagógicas actuales apoyadas en las TIC, especialmente los
juegos serios.
\end{block}


\end{frame}

\begin{frame}
\frametitle{\pagetitle}
\framesubtitle{Objetivos específicos}

\small
\begin{itemize}[<+->]
\item Proveer una visión actualizada de las corrientes pedagógicas apoyadas con las TIC.
\item Proveer una visión actualizada de los juegos serios.
\item Identificar áreas de aplicación de los juegos serios.
\item Seleccionar las herramientas tecnológicas disponibles para el desarrollo de juegos serios.
\item Contrastar en la práctica los conocimientos teóricos adquiridos a través del diseño
e implementación de un juego serio.
\item Evaluar la solución propuesta para identificar los
aspectos de diseño, desarrollo y evaluación de los juegos serios.
\end{itemize}
\end{frame}


%\section{TIC's en la educación}

\begin{frame}{Instruccionismo}
\end{frame}
\begin{frame}{Conductismo}
\end{frame}
\begin{frame}{Constructivismo}
\end{frame}
\begin{frame}{Construccionismo}
\end{frame}
\begin{frame}{Ventajas}

    \begin{itemize}[<+->]
        \item Nuevos modelos pedagógicos
        \item Eliminación de distancias
        \item Colaboración distribuida
        \item Motivación para aprender
        \item Adquisición de habilidades básicas
    \end{itemize}
\end{frame}
\begin{frame}{Desafíos}
\end{frame}

\section{Juegos serios}
1 página

Un juego serio es un videojuego que posee un propósito educacional explícito y
bien elaborado, y cuya intención no es la de únicamente entretener al
usuario\cite{abt1987serious,sg:aoverview,damien:sg}.

Los \emph{Juegos Serios} proveen una oportunidad muy importante para ayudar en
la enseñanza y desarrollo de profesionales\cite{mariluz:seiousgames}, por que
ayudan a crear el tipo de educación que los adultos prefieren, proveen
mecanismos para que los estudiantes cometan errores y experimenten con sus
ideas, con su conocimiento y con la teoría en un ambiente protegido sin riesgos
para la vida o la identidad\cite{sg:aoverview}. 

El campo de los \emph{Juegos Serios} rechaza la idea de que los profesionales de
la educación pueden ser reemplazados, la labor de estos profesionales es
imprescindible para la reflexión y orientación del
aprendizaje\cite{elearning:seiousgames}.

\subsection{Ventajas y desafíos}

Los juegos serios tienen ventajas y desafíos que los hacen únicos en su enfoque
para con la utilización de las TIC's y posee las siguientes ventajas:


\begin{itemize}

\item \textbf{Motivación interna}: favorecen la autoestima y tienen un factor
    motivacional\cite{guenaga2013serious}, permitiendo una implicación mayor del
    usuario en la actividad\cite{sg:aoverview}. La implicación del usuario
    dentro de la actividad, es un tema central en el desarrollo de los juegos
    serios\cite{charsky:2010}.

\item \textbf{Apoyo al aprendizaje}: ayudan al aprendizaje es por que los mismos
    se desarrollan en un entorno significativo y relevante al
    contexto\cite{sg:aoverview}. Las teorías modernas de aprendizaje sugieren
    que el aprendizaje es más efectivo cuando es activo, experiencial y basado
    en problemas\cite{guenaga2013serious}.

\item \textbf{Menos limitaciones}: juegos serios permite a sus usuarios
    experimentar en entornos y sistemas que no son posibles en la vida real, por
    cuestiones de costo, tiempo y aspectos relacionados a la
    seguridad\cite{sg:aoverview}.

\item \textbf{Similitud a la realidad}: constituyen un escenario privilegiado
    para el desarrollo de todos los componentes de las competencias, ya que
    permiten desarrollar vivencias en las que ponerlos en practica, permitiendo
    el entrenamiento en situaciones que en muchas ocasiones son similares a las
    que se encuentran en entornos reales\cite{guenaga2013serious,sg:aoverview}.
    
\item \textbf{Estimulación sensorial}: aumentan la capacidad de coordinación,
    percepción espacial y ampliación del campo visual, lo que tiene una
    incidencia en la lectura y el manejo eficiente en ambientes
    3D\cite{guenaga2013serious}. 

\end{itemize}

El potencial de un juego serio no es ilimitado, existen múltiples desafíos que
deben ser superados para poder desarrollar un juego serio que obtenga las ventajas
citadas previamente y pueda ser de utilidad en la educación formal.


\begin{itemize}

\item \textbf{Falta de investigación}: aunque en los últimos años los estudios
    del impacto de los juegos serios han aumentado considerablemente, son
    necesarios más estudios para probar su eficiencia\cite{sg:aoverview}.

\item \textbf{Expectativas muy altas}: un juego serio, como el resto de la
    \textit{media}, no puede cambiar el comportamiento de una persona por sí
    solo, pero sí permiten al jugador explorar las opciones, tener en cuenta las
    consecuencias de sus actos y poner en práctica sus
    conocimientos\cite{education:games,stapleton2004serious,videojuegos:gonzaleztardon}. 

\item \textbf{Evaluación tradicional}: la forma tradicional de evaluación presenta
    dificultades a los juegos serios, por ejemplo, las pruebas tradicionales
    contienen un grupo de preguntas, las cuales son vistas de manera
    independiente, en cambio en un juego serio, las acciones son dependientes
    del contexto y las acciones previamente realizadas\cite{shute2009melding}.

\item \textbf{Utilización incorrecta}: en cuanto al objeto pedagógico, el área
    en la cual se utiliza un juego serio es un factor determinante para el éxito
    del mismo\cite{stapleton2004serious}.

\item \textbf{Falta de recursos}: uno de los factores más complicados a la hora
    del desarrollo de juegos serios es la limitación de recursos financieros,
    esto no quiere decir que no existan recursos para su desarrollo, sino que,
    comparados con los recursos invertidos en otras \textit{media}, el
    presupuesto es insignificante\cite{stapleton2004serious,sg:aoverview}. Como
    consecuencia de las limitaciones financieras, los desarrolladores no siempre
    pueden acceder a tecnología de última generación\cite{stapleton2004serious}

\end{itemize}

\subsection{Actualidad}

Esta sección muestra las áreas donde más frecuentemente se utilizan los juegos
serios.

\begin{itemize}

\item \textbf{Militar}: durante más de $30$ años los videojuegos han sido
    reconocidos como herramientas factibles en el entrenamiento de militares. En
    $1996$ fue lanzado un videojuego llamado \emph{Marine Doom} en donde la
    tarea de los jugadores era el aprendizaje de formas de ataque, conservación
    de municiones, comunicarse con eficacia, dar órdenes al equipo de trabajo
    entre otros. De esta manera tuvo lugar una forma de entrenamiento más
    atractivo, sin el costo, dificultad, riesgos e inconvenientes que implicaría
    el mismo entrenamiento en un entorno real. Además se podían crear
    situaciones que en el mundo real serían muy difíciles de replicar y donde
    los errores pueden ser catastróficos\cite{education:games}.

\item \textbf{Salud}: los juegos de salud se utilizan para la formación de
    profesionales basada en la simulación. En $2008$ el Centro de Simulación
    \emph{Hollier} en \emph{Birmingham}, Reino Unido, realizó una prueba que
    permitió a médicos jóvenes experimentar y entrenar para diversos escenarios
    médicos a través de maniquíes virtuales como pacientes, de este modo el
    aprendizaje se da por la experiencia. En su disertación, \emph{Roger D.
        Smith}, realizó una comparación entre la enseñanza tradicional y la
    formación mediante realidad virtual y el uso de herramientas basadas en la
    tecnología de videojuegos en cuanto a la cirugía laparoscópica. Como
    conclusión afirmó que lo último era más barato, requería menos tiempo y que
    permitió menos errores médicos cuando los médicos se presentaban en una
    cirugía real debido a, entre otras cosas, la posibilidad de repetición de la
    experiencia sin riesgo alguno\cite{education:games}. 

\item \textbf{Juegos corporativos}: Este tipo de videojuegos se han utilizado
    para la selección de personal, la mejora de comunicación entre los
    directivos y su personal de confianza, y la formación de nuevos empleados.
    Los juegos serios pueden ser utilizados incluso para elaborar planes de
    negocios\cite{education:games}. 

\end{itemize}


\section{Definición del problema}

%1/2 página
%\begin{itemize}
%\item Estado actual
%\item Prácticas en laboratorio
%\item Problemas actuales
%\item Propuesta de solución
%\end{itemize}

Según una investigación los alumnos de enfermería, son estudiantes divergentes,
es decir aprenden a través de experimentación activa, e interiorizan el
conocimiento reflexionando sobre la experiencia. La simulación es una
herramienta ideal para este tipo de estudiantes\cite{humphreys2013developing}.

Es por ello que este trabajo de grado se centra en las problemáticas referentes
al área de enfermería proponiendo una solución tecnológica. Se considera como
campo de estudio al \gls{iab}, cuyos estudiantes del último año de la carrera de
Licenciatura en Enfermería son tomados como población objetivo.

A continuación se describe el estado actual de las prácticas realizadas por los
alumnos como parte de plan curricular, además de los problemas actuales y una
propuesta de solución a esos problemas.

\subsection{Prácticas en laboratorio}

El \gls{iab} cuenta con un laboratorio de prácticas especializado para los
estudiantes de enfermería. Es utilizado para desarrollar las materias prácticas
de manera a tener una formación previa a las prácticas de campo, es decir,
prácticas en hospitales con pacientes reales.

El número de alumnos dificulta la enseñanza individual por lo que las clases de
laboratorio se dividen en dos, la primera parte de desarrolla en un aula
convencional en donde el profesor enseñanza a los estudiantes usando a uno de
ellos como voluntarios o usando modelos de partes del cuerpo para realizar
explicaciones generales y responder dudas. La segunda parte se lleva a cabo en
un laboratorio que cuenta con herramientas como maniquíes, camas, utensilios,
entre otros. En esta segunda parte los alumnos pueden explorar y experimentar
por sí mismos bajo la tutela del profesor.

\subsection{Prácticas de campo}

Las prácticas de campo son aquellas prácticas profesionales que son realizadas
por los alumnos con pacientes humanos y en hospitales, bajo supervisión de un
profesional denominado instructor y bajo una continua evaluación de sus
acciones, las mismas son llevadas a cabo una vez que los alumnos finalizan las
prácticas de laboratorio.

Cada instructor posee un planilla por alumno donde se realiza el seguimiento de
sus actividades. La creación de esta planilla de actividades es responsabilidad
del instructor, el instructor debe basarse en las competencias básicas de la
asignatura y la misma es validada por la dirección de la carrera, y se considera
que un alumno ha adquirido la pericia\footnote{Sabiduría, práctica, experiencia
    y habilidad en una ciencia o arte.} necesaria para una asignatura solo si
pudo completar la planilla del instructor. Si el alumno no aprueba todas las
prácticas no tiene derecho a rendir el examen teórico de la materia y debe
volver a cursar la materia.

\subsection{Problemas actuales}

Si bien el nivel de actual de los egresados de la carrera de Licenciatura en
enfermería del \gls{iab} es considerando satisfactorio por las autoridades de
esta casa de estudios, existen inconvenientes según apreciaciones de profesores
y alumnos, además de algunas tesis de grado de la institución. A continuación se
citan estos inconvenientes.

Algunos profesores manifestaron en reuniones que antes de ir a las prácticas de
campo con los alumnos prefieren dar las primeras clases en laboratorio por lo
siguiente:

\begin{itemize}
\item Falta de preparación de los alumnos debido a que en ocasiones 
ciertos detalles no son cubiertos completamente por las prácticas en 
laboratorio.
\item Definición de un protocolo de comunicación entre alumnos y el 
profesor para su uso en las prácticas de campo.
\item Nerviosismo ante la primera práctica por parte de los alumnos.
\end{itemize}

En tanto, los mayores inconvenientes detectados por los alumnos son:
\begin{itemize}
\item Alta carga horario de los trabajos prácticos.
\item Reducida carga horaria para el estudio de las materias teóricas 
debido a las prácticas.
\item Poca flexibilidad de los profesores.
\item Falta de materiales actualizados para los profesores.
\item Problemas de transporte para llegar a hasta los hospitales o al 
    \gls{iab}.
\item Falta de preparación para las prácticas.
\item Alta cantidad de alumnos, debido a esto las prácticas de campo 
rara vez se realizan en un sólo hospital y son realizadas por grupo de
alumnos.
\end{itemize}

\subsection{Propuesta de solución}

Teniendo en cuenta que los mismos forman parte de un grupo de profesionales que requieren de un alto grado de prácticas y que en la actualidad se 
presentan varios factores que a un estudiante de esta carrera le impide poner 
a prueba todo el tiempo sus conocimientos y por todo lo expuesto anteriormente, en este trabajo se propone el desarrollo de un juego serio, que involucra la simulación de laboratorios virtuales, como una herramienta para 
el proceso de aprendizaje de los alumnos de la carrera de enfermería.

Se considera que los principales problemas que pueden ser abordados con esta 
solución son los siguientes: evaluación de los alumnos, seguimiento del 
progreso del alumno, tiempo de práctica, ubicuidad, realismo, enfoque 
individual. Ofreciendo una herramienta que no puede sustituir a un práctica 
de campo pero si puede servir de apoyo en el aprendizaje relacionado a las 
prácticas.

%\section{Componentes principales}
\setcounter{sectiontotal}{4}

\begin{frame}
    \frametitle{\pagetitle}
    \framesubtitle{Descripción general}
    %\pause{}
	\begin{figure}
		%\centering
        \includegraphicsx[scale=0.41]{../tecnologias/images/full.pdf}
		\caption{Esquema general de componentes de la solución}
	\end{figure}
\end{frame}

\begin{frame}
\frametitle{\pagetitle}
\framesubtitle{Motor de videojuego: Unity 3D}
\tiny

\begin{tabulary}{\textwidth}{rCCCCCCCC}
\toprule
& \multicolumn{3}{c}{\textbf{Motor}\footnote{\fullcite{videojuego:telechea}}} \\

\cmidrule(lr){2-4}  
\textbf{Característica}         &
\textbf{Unreal Engine}          &
\textbf{CryEngine}              &
\textbf{Unity3D}                \\
\midrule
\textbf{Distribución} & iOs. Otros en la versión comercial. & iOs, Android & {\color{blue!90!black}
Android, Windows Phone, iOs, BlackBerry} \\ 

\midrule
\textbf{Tienda} & Mediana & Mediana & \color{blue!90!black} Grande \\

\textbf{Comunidad} & Grande & Mediana & {\color{blue!90!black} Grande} \\

\midrule
\textbf{Formatos} & fbx, dds, raw, ASE & Formatos propios & {\color{blue!90!black} FBX, OBJ,
Max, Blend, dae, 3ds, dxf, MB, MA, etc} \\

\midrule
\textbf{Curva de aprendizaje} & Compleja & Compleja & {\color{blue!90!black} Sencilla} \\

\textbf{Lenguajes} & Unreal Script y C++ & C++, Lua & {\color{blue!90!black} \cs{}, UnityScript y Boo} \\

\midrule
\textbf{Licencia del motor} & Versiones antiguas gratuitas para uso no comercial. &
Gratuita solo para uso no comercial & {\color{blue!90!black} Versión limitada
gratuita, disponible para uso no comercial} \\

\textbf{IDE} & Si, gratuito no comercial & Sí, propietario & {\color{blue!90!black} Sí,
gratuito no comercial} \\
\bottomrule

\end{tabulary}

\end{frame}

\begin{frame}
\frametitle{\pagetitle}
\framesubtitle{Tecnologías utilizadas \- Front-end}
\begin{table}
\centering
\begin{tabular}{lrr}
\toprule
\textbf{Utilización} & \textbf{Tecnología} \\
\midrule
Motor de videojuego      & Unity3d         \\
IDE                      & MonoDevelop     \\
                         & Unity Editor    \\
\midrule
Modelado 3D              & 3ds Max         \\
Modelado 2D              & Photoshop       \\
Modelado de personajes   & MakeHuman       \\

\midrule
Lenguaje de programación & \cs{} \\
Gestión de código fuente & Git \\
Otras librerías          & NGUI            \\
                         & Facebook \\
                         
\bottomrule
\end{tabular}
\end{table}
\end{frame}

\begin{frame}
\frametitle{\pagetitle}
\framesubtitle{Tecnologías utilizadas \- Back-end}
\begin{table}
\centering
\begin{tabular}{lrr}
\toprule
\textbf{Utilización} & \textbf{Tecnología}  \\
\midrule
IDE                         & Eclipse \\
Lenguaje de programación    & Java \\
\midrule
Servidor de aplicaciones    & Jboss \\
Servidor de base de datos   & PostgreSQL \\
Proveedor de plataforma     & OpenShift \\
\midrule
Gestión de código fuente    & Git \\
Otras librerias             & Java EE \\

\bottomrule
\end{tabular}
\end{table}
\end{frame}


%\section{Diseño e implementación}
\setcounter{sectiontotal}{3}

% TODO Agregar lo de los ECA!
% TODO mejorar esta imagen
\begin{frame}
\frametitle{\pagetitle}
\framesubtitle{Flujo de interacción}
\centering
\begin{overprint}
\onslide<1|handout:1> \includegraphics[width=\textwidth]{../solucion/images/pantalla_inicio.jpg} 
\onslide<2|handout:0> \includegraphics[width=\textwidth]{imagenes/flujo/flujo2.png} 
\onslide<3|handout:0> \includegraphics[width=\textwidth]{imagenes/flujo/flujo3.png} 
\onslide<4|handout:0> \includegraphics[width=\textwidth]{imagenes/flujo/flujo4.png} 
\onslide<5|handout:0> \includegraphics[width=\textwidth]{../solucion/images/hemocultivo_jeringa_ampliada.jpg} 
\onslide<6|handout:0> \includegraphics[width=\textwidth]{imagenes/flujo/flujo6.png} 
\onslide<7|handout:0> \includegraphics[width=\textwidth]{../solucion/images/resultado_hemocultivo.jpg}
\onslide<8|handout:0> \includegraphics[width=\textwidth]{imagenes/flujo/flujo18.png} 
\onslide<9|handout:0> \includegraphics[width=\textwidth]{imagenes/flujo/flujo8.png} 
% Glasgow
\onslide<10|handout:0> \includegraphics[width=\textwidth]{imagenes/flujo/flujo9.png}
\onslide<11|handout:0> \includegraphics[width=\textwidth]{imagenes/flujo/flujo10.png}
%\onslide<12|handout:0> \includegraphics[width=\textwidth]{imagenes/flujo/flujo11.png}
\onslide<12|handout:0> \includegraphics[width=\textwidth]{imagenes/flujo/flujo12.png}
\onslide<13|handout:0> \includegraphics[width=\textwidth]{imagenes/flujo/flujo13.png}
\onslide<14|handout:0> \includegraphics[width=\textwidth]{imagenes/flujo/flujo14.png}
\onslide<15|handout:0> \includegraphics[width=\textwidth]{../solucion/images/glasgow_comandos_voz.jpg}
\onslide<16|handout:0> \includegraphics[width=\textwidth]{imagenes/flujo/flujo16.png}
\onslide<17|handout:0> \includegraphics[width=\textwidth]{imagenes/flujo/flujo17.png}





\end{overprint}
\end{frame}

\section{Evaluación}
\setcounter{sectiontotal}{11}


\begin{frame}
\frametitle{\pagetitle}
\framesubtitle{Metodología}

\pause{} 

\begin{table}
    \tiny
\centering
\begin{tabulary}{\textwidth}{rLLLL}
\toprule
\textbf{Objetivo}      & Evaluar la usabilidad de la interfaz      & Determinar la muestra                                              & Evaluar el uso y la aceptación & Evaluar el conocimiento\\
\midrule
\textbf{Participantes} & Alumnos de la carrera de II de la FP-UNA. & \multicolumn{3}{c}{Alumnos de la carrera de enfermería del IAB} \\
\midrule
\textbf{Muestra}       & 8 Alumnos                                 & 124 Alumnos                                                        & 11 Alumnos                     & 124 alumnos \\
\midrule
\textbf{Lugar}         & FP-UNA                                    & IAB                                                                &                                & IAB\\
\midrule
\textbf{Duración}      & 15 minutos                                & 10 minutos                                                         & 20 días                        & 20 minutos\\
\midrule
\textbf{Fases}           & \tabitem Explicación de los procedimientos. & \tabitem Presentación.          & \tabitem Explicación de la interfaz. & \tabitem Explicación de la prueba.\\
                & \tabitem Tareas.                            & \tabitem Encuesta.              & \tabitem Utilización de la solución. & \tabitem Encuesta sobre conocimiento\\
                & \tabitem Encuesta de usabilidad.            & \tabitem Recopilación de datos. & \tabitem Encuesta de apreciación.    & \tabitem Recopilación de datos.\\
                & \tabitem Recopilación de datos.             &                                 & \tabitem Recopilación de datos.\\
\bottomrule
\end{tabulary}
\end{table}

\end{frame}

\begin{frame}[fragile]
    \frametitle{\pagetitle}
    \framesubtitle{Pruebas preliminares}

\begin{filecontents}{interfazuso.dat}
n   p       s
1	9.38	6.48
2   5.88	4.64
3   7.75	3.13
\end{filecontents}
\pgfplotstableread{interfazuso.dat}{\InterfazUso}

\begin{figure}
    \begin{subfigure}[b]{.5\linewidth}
        \centering
        \begin{tikzpicture}[scale=.45]
           \begin{axis}[ybar,%
              legend pos=outer north east,
              xmin=0,
              xmax=4,
              xtick=data,
              symbolic x coords={0,1,2,3,4},
              ymin=0,ymax=10,
              %ytick={0,2,4,6,8,10},
              xticklabels={Contextual,Interfaz,Herramienta},
              ylabel= Tiempo (s),
              xlabel= Tipo de acción,
              bar width=10pt,
              %enlarge x limits={abs=2},
                ]   
        \addplot[color=blue,ybar,fill=blue!75,area legend] table [x = {n}, y = {p}] {\InterfazUso};
        \addlegendentry{Primer}
        \addplot[color=red,ybar,fill=red!75,area legend] table [x = {n}, y = {s}]
        {\InterfazUso};
        \addlegendentry[align=left]{Promedio \\ siguientes}
        \end{axis}
        \end{tikzpicture}
        \caption{Tiempo por tipo de actividad}
    \end{subfigure}\hfill
    \pause
    \begin{subtable}[b]{.5\linewidth}
        \tiny
        \begin{tabular}{lr}
        \toprule
        Variable & Apreciación \\
        \midrule
        Gráficos                & \pin{ForestGreen}{De acuerdo}              \\
        Interacción Entorno     & \pin{BurntOrange}{Neutral}                 \\
        Interacción Objetos     & \pin{BurntOrange}{Neutral}                 \\
        Características Entorno & \pin{ForestGreen}{Parcialmente de acuerdo} \\
        Usabilidad              & \pin{BurntOrange}{Neutral}                 \\
        Integración             & \pin{ForestGreen}{Parcialmente de acuerdo} \\
        \bottomrule
    \end{tabular}
        \caption{Disconformidad por variable}
    \end{subtable}
\end{figure}

\end{frame}

\begin{frame}
    \frametitle{\pagetitle}
    \framesubtitle{Encuesta para determinar muestra}

\begin{figure}
    %\begin{subfigure}[b]{.3\linewidth}
        \centering
        \begin{tikzpicture}[thick,scale=0.8, every node/.style={transform shape}]
            \pie[
                %explode=.2,
                text=legend,
                %style=drop shadow,
                %radius=3,
                %scale font,
                explode={0.1,0.1,0.3,0.3}
                ]%
            {%
                31.2 / Plan pos-pago,
                57   / Paquetes pre-pago,
                5.4  / Sin acceso,
                6.4  / Acceso ocasional}
        \end{tikzpicture}
        \caption{Acceso a internet}
    %\end{subfigure}\hfill
    %\pause{}
    \end{figure}
    
\end{frame}    

\begin{frame}
	 \frametitle{\pagetitle}
    \framesubtitle{Encuesta para determinar muestra}
    %\begin{subfigure}[b]{.3\linewidth}
    \begin{figure}
    
   
        \centering
        \begin{tikzpicture}[thick,scale=0.8, every node/.style={transform shape}]
            \pie[
                text=legend,
                rotate=61.3,
                explode={.1,.2,.2,.2}
                ]%
            {%
            61.3 / Android,
             8.6 / Symbian,
            12.9 / Windows Phone,
            17.2 / Otros}
        \end{tikzpicture}
        \caption{Sistema operativo}
   % \end{subfigure}\\
    %\pause{}
     \end{figure}
     
 \end{frame}
 
 \begin{frame}
	 \frametitle{\pagetitle}
    \framesubtitle{Encuesta para determinar muestra}
    %\begin{subfigure}[b]{.5\linewidth}
    \begin{figure}
        \centering
        \begin{tikzpicture}[thick,scale=0.8, every node/.style={transform shape}]
            \pie[
                text=legend,
                explode=.1
                ]%
            {%
                81.7 /  No Cumple,
                18.3 /  Cumple
            }
        \end{tikzpicture}
        \caption{Cumplimiento de requisitos}
    %\end{subfigure}
\end{figure}

\end{frame}

% Utilización
\begin{frame}
\frametitle{\pagetitle}
\framesubtitle{Utilización de la solución}

\begin{table}
\centering
\small
\begin{tabular}{lr}
\toprule
Partidas                         & 99 \\
Primera partida                  & 4 de Noviembre de 2014 \\
Última partida                   & 23 de Noviembre de 2014 \\
\midrule
\onslide+<2->{Tiempo total       & 11.134 s \\
Promedio de tiempo por partida   & 112 s}
\\\midrule
\onslide+<3->{Acciones           & 2.944 \\
Promedio de acciones por partida & 30}
\\\midrule
\onslide+<4->{Usuarios           & \textcolor{red}{8} \\
Promedio de partidas por usuario & 12}
\\\bottomrule
\end{tabular}
\end{table}

\end{frame}
% Utilización por hora
\begin{frame}[t,fragile]
    \frametitle{\pagetitle}
    \framesubtitle{Utilización de la solución}
\centering

% Datos
\begin{filecontents}{utilizaciontiempo.dat}
Hora	Cantidad
 0	 0
 1	 9
 2	 4
 3	 2
 4	 0
 5	11
 6	 8
 7	 3
 8	 5
 9	 7
10	 7
11	12
12	15
13	 9
14	 6
15	 0
16	 0
17	 0
18	 0
19	 0
20	 0
21	 0
22	 0
23	 1
24	 0
\end{filecontents}
\pgfplotstableread{utilizaciontiempo.dat}{\UtilizacionTiempo}

\begin{figure}
    \begin{tikzpicture}[thick,scale=0.85, every node/.style={transform shape}]
        \begin{axis}[
            title={},
            xlabel={Hora},
            ylabel={Partidas},
            xmin=0, xmax=24,
            ymin=0, ymax=16,
            xtick       = {0  , 3  , 6  , 9  , 12 , 15 , 18 , 21 , 24},
            xticklabels = {12 , 15 , 18 , 21 ,  0 , 3  ,  6 ,  9 , 12},
            %ytick={0,.25,.50,.75,1},
            legend pos=north east,
            ymajorgrids=true,
            xmajorgrids=true,
            grid style=dashed,
        ]
         
        \addplot[color=blue,fill=blue!5] table [x = {Hora}, y = {Cantidad}] {\UtilizacionTiempo};
        \end{axis}
    \end{tikzpicture}
    \caption{Utilización por hora}
\end{figure}
 

\end{frame}

% Utilización progreso
\begin{frame}
\frametitle{\pagetitle}
\framesubtitle{Utilización de la solución}

\begin{table}
\centering
\scriptsize
\begin{tabulary}{\linewidth}{lrrrr}
\toprule
& \multicolumn{2}{c}{Venopunción}    & \multicolumn{2}{c}{Glasgow}  \\
\cmidrule(lr){2-3}
\cmidrule(lr){4-5}
\textbf{Promedio} & \textbf{Puntaje (17)} & \textbf{Desviación} &
                  \textbf{Puntaje (4)} & \textbf{Desviación} \\
\midrule
Primera vez       & $6.1$        & $6.3$      & $1.4$       & $0.9$ \\
Siguientes veces  & $8.0$        & $3.8$      & $1.6$       & $1.5$ \\
Más altos          & $11.0$       & $4.9$      & $2.6$       & $0.9$ \\
\bottomrule
\end{tabulary}
\caption{Mejora por escenario}
\end{table}

\end{frame}

% Apreciación de la solución
\begin{frame}
\frametitle{\pagetitle}
\framesubtitle{Apreciación de la solución}

\begin{table}
\tiny
\begin{tabulary}{\textwidth}{lrll}
\toprule
\textbf{Factor}       & \textbf{Promedio encuesta} & \textbf{Puntos fuertes}             & \textbf{Puntos débiles} \\
\midrule
Motivación            & De acuerdo                 & \tabitem Puntaje                    & \tabitem Tiempo  \\
                      &                            & \tabitem Socialización              & \\
\midrule
Exploración           & De acuerdo                 & \tabitem Aleatoriedad               & \tabitem Utilización    \\
                      &                            & \tabitem Funciones                  & \\
\midrule
Inmersión             & De acuerdo                 & \tabitem Escenografía               & \tabitem Pertenencia\\
                      &                            & \tabitem Gráficos                   & \\
                      &                            & \tabitem Ordenes verbales           & \\
\midrule
Pedagogía             & De acuerdo                 & \tabitem Comprensión                & \\
                      &                            & \tabitem Retroalimentación limitada & \\
\midrule
Representación        & Parcialmente               & \tabitem Movimientos motrices       & \tabitem Reacción verbal\\
                      & de acuerdo                 &                                     & \tabitem Estados del paciente\\
\midrule
Retroalimentación     & Parcialmente               & \tabitem Detalles de los pasos      & \tabitem Iconos \\
                      & de acuerdo                 &                                     & \\
\midrule
Utilidad              & De acuerdo                 & \tabitem Complemento                & \\
                      &                            & \tabitem Facilitador                & \\
\bottomrule
\end{tabulary}
\caption{Aceptación por aspecto de la solución}
\end{table}

\end{frame}

% Conocimiento
\begin{frame}[t,fragile]
\frametitle{\pagetitle}
\framesubtitle{Encuesta para evaluar conocimiento}
%\centering

% Datos
\begin{filecontents}{objetiva.dat}
n	total        muestra	    control
1	0.2020159639 0.3636363636	0.1818181818
2	0.6027491017 0.6363636364	0.6
3	0.1332966457 0.09090909091	0.1363636364
4	0.2580191298 0.2727272727	0.2545454545
5	0.5869865976 0.8181818182	0.5636363636
6	0.1640933841 0	            0.1834862385
7	0.5256475617 0.6363636364	0.5137614679
8	0.2947317706 0.4545454545	0.2752293578
9	0.3081905611 0.1818181818	0.3211009174
10	0.4498120076 0.3636363636	0.4587155963
\end{filecontents}
\pgfplotstableread{objetiva.dat}{\Objetiva}

\begin{figure}
%\tiny
\begin{subfigure}[b]{.5\linewidth}
    \begin{tikzpicture}[thick,scale=0.5, every node/.style={transform shape}]
        \begin{axis}[
            ybar,
            width=\linewidth,
            bar width=9pt,
            x=25pt,
            height=10cm,
            axis on top,
            enlarge x limits=true,
            tick align=inside,
            tickwidth=0pt,
            title={},
            xlabel={Pregunta},
            ylabel={Puntos},
            xmin=1, xmax=10,
            ymin=0, ymax=1,
            xtick={0,1,2,3,4,5,6,7,8,9,10},
            ytick={0,.25,.50,.75,1},
            legend pos=north east,
            ymajorgrids=true,
            xmajorgrids=true,
            grid style=dashed,
        ]
         
        \only<1->{%
        \addplot[color=blue,fill=blue!30] table [x = {n}, y = {muestra}] {\Objetiva};
        \addlegendentry{Muestra}}
        \only<2->{%
        \addplot[color=red,fill=red!30] table [x = {n}, y = {control}] {\Objetiva};
        \addlegendentry{Control}}
        %\only<3->{\addplot[color=black,dashed] table [x = {n}, y = {total}] {\Objetiva};
        %\addlegendentry{Total}}
        \end{axis}
    \end{tikzpicture}
    \caption{Comparación de puntaje*}
\end{subfigure}\hfill
\pause
\pause
\pause
\begin{subtable}[b]{.5\linewidth}
    \centering
    \scriptsize
    \begin{tabular}{lr}
    \toprule
    \textbf{Grupo}         & \textbf{Promedio (10)} \\
    \midrule
    \onslide+<4->{Usuarios & \textbf{3.82}  \\}
    \onslide+<5->{Control  & \textbf{3.47} \\\midrule}
    \onslide+<6->{Total    & \textbf{3.49}}
    \\\bottomrule
    \end{tabular}
    \caption{Puntaje promedio por grupo*}
\end{subtable}
\end{figure}
 
\tiny 

\textbf{*} Los usuarios participaron del experimento luego del examen
final de la materia \textit{Enfemería en Urgencias}, la cual requiere el dominio
de ambos temas simulados como parte de sus competencias básicas.

\end{frame}

% Correlación
\begin{frame}
\frametitle{\pagetitle}
\framesubtitle{Correlación}

\begin{table}
\tiny
\centering
\begin{tabulary}{\textwidth}{Lrrrrrr}
\toprule
                                  & \multicolumn{2}{c}{Puntaje Max.} & \multicolumn{2}{c}{Tiempo de juego} & \multicolumn{2}{c}{Puntaje examen} \\
\cmidrule(lr){2-3} 
\cmidrule(lr){4-5} 
\cmidrule(lr){6-7}
                            & Venopunción             & Glasgow                & Venopunción          & Glasgow                 & Venopunción             & Glasgow \\
\midrule
P.M. Venopunción            & 1                       & 0.12                   & \pin{NavyBlue}{0.30} & \pin{RoyalPurple}{0.35} & \pin{ForestGreen}{0.74} & 0.55 \\
P.M. Glasgow                & 0.12                    & 1                      & 0.32                 & \pin{Aquamarine}{0.61}  & 0                       & \pin{BrickRed}{0.54} \\
\midrule
T.J. Venopunción            & \pin{NavyBlue}{0.30}    & 0.32                   & 1                    & 0.29                    & 0.04                    & 0.05\\
T.J. Glasgow                & \pin{RoyalPurple}{0.35} & \pin{Aquamarine}{0.61} & 0.29                 & 1                       & 0.69                    & \pin{Orange}{0.86}\\
\midrule
P.E. Venopunción            & \pin{ForestGreen}{0.74} & 0                      & 0.04                 & 0.69                    & 1                       & \pin{VioletRed}{0.78} \\
P.E. Glasgow                & 0.55                    & \pin{BrickRed}{0.54}   & 0.05                 & \pin{Orange}{0.86}      & \pin{VioletRed}{0.78}   & 1 \\
\bottomrule               
\end{tabulary}
\caption{Correlación entre factores estudiados} 
\end{table}

\tiny
\begin{itemize}
\item \pin{NavyBlue}{Venopunción. Rendimiento en la solución y tiempo de juego.}
\item \pin{ForestGreen}{Venopunción. Rendimiento en la solucion y en el exámen.}
\item \pin{Aquamarine}{Glasgow. Rendimiento en la solución y tiempo de juego.}
\item \pin{BrickRed}{Glasgow. Rendimiento en la solución y en el exámen.}
\item \pin{Orange}{Glasgow. Tiempo de juego y rendimiento en el exámen.}
\item \pin{RoyalPurple}{Rendimiento Venopunción y tiempo de juego Glasgow.}
\item \pin{VioletRed}{Puntaje examen Glasgow y Puntaje examen Venopunción}
\end{itemize}
\note{%

\begin{tabular}{|r|r|r|r|r|r|}
\hline  & M.F.  & F.     & M.     & W.     & N \\
\hline
Mínimo  & $0.7$ & $0.69$ & $0.39$ & $0.29$ & $0.19$ \\
\hline
\end{tabular}}
\end{frame}


%\input{./conclusion.tex}
%\section{Trabajos futuros}
\setcounter{sectiontotal}{1}

\begin{frame}
\frametitle{\pagetitle}
\framesubtitle{Trabajos Futuros}
\begin{itemize}[<+->]
    \item Otros procedimientos.
    \item Control del progreso
    \item Multijugador
    \item Escenarios dinámicos
    \item Plataformas de realidad virtual
    \item Dificultad de acuerdo al alumno
\end{itemize}
\end{frame}



\end{document}
