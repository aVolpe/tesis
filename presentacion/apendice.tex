\section{Apéndices}
\setcounter{sectiontotal}{10}
\newcounter{pagenumber}
\setcounter{pagenumber}{0}

\newcommand\apendixframetitle{%
\stepcounter{pagenumber}
Apéndices (%
\arabic{pagenumber}%
~de~%
\arabic{sectiontotal}%
)%
}

\newcommand\frameimage[2][Apendice]{%
\begin{frame}[noframenumbering]
\frametitle{\apendixframetitle}
\framesubtitle{#1}
\centering
\includegraphics[width=\textwidth,height=0.8\textheight,keepaspectratio]{#2}
\end{frame}
}

\begin{frame}[noframenumbering]
\frametitle{\apendixframetitle}
\framesubtitle{Tecnologías utilizadas \- Front-end}
\begin{table}
\centering
\begin{tabular}{lrr}
\toprule
\textbf{Utilización} & \textbf{Tecnología} & \textbf{Versión} \\
\midrule
Motor de videojuego      & Unity3d         & 4.5 \\
IDE                      & MonoDevelop     & 4 \\
                         & Unity Editor    & 4.5 \\
\midrule
Modelado 3D              & 3ds Max         & 2013 \\
Modelado 2D              & Photoshop       & 14 \\
Modelado de personajes   & MakeHuman       & 1.0 \\

\midrule
Lenguaje de programación & \cs{} \\
Gestión de código fuente & Git & 1.8 \\
Otras librerías          & NGUI            & 2.7\\
                         & Facebook \\
                         
\bottomrule
\end{tabular}
\end{table}
\end{frame}

\begin{frame}[noframenumbering]
\frametitle{\apendixframetitle}
\framesubtitle{Tecnologías utilizadas \- Back-end}
\begin{table}
\centering
\begin{tabular}{lrr}
\toprule
\textbf{Utilización} & \textbf{Tecnología}  & \textbf{Versión} \\
\midrule
IDE                         & Eclipse & Luna\\
Lenguaje de programación    & Java & 7\\
\midrule
Servidor de aplicaciones    & Jboss & 7 \\
Servidor de base de datos   & PostgreSQL & 9.2 \\
Proveedor de plataforma     & OpenShift \\
\midrule
Gestión de código fuente    & Git & 1.8\\
Otras librerias             & Java EE & 6\\

\bottomrule
\end{tabular}
\end{table}
\end{frame}

\begin{frame}[noframenumbering]
\frametitle{\apendixframetitle}
\framesubtitle{Justificación de la muestra}

\small

\begin{itemize}

\item El número de muestras tomadas fue 8, ya que según Nielsen son necesarios
    al menos $5$ participantes para poder obtener resultados significativos en
    una prueba de usabilidad. Además, Ritch asegura que la teoría de Nielsen es
    verdadera especialmente para pruebas simples. 

\item La utilización de $11$ alumnos es suficiente, ya que según estudios
    presentados en Nielsen, mientras menos experiencia tengan los sujetos de
    estudio con la solución planteada, serán necesarios menos para detectar un
    gran porcentaje de errores y fortalezas, y según Ritch, una base de $10$ a
    $12$ es suficiente para obtener resultados estadísticamente válidos.

\item Cabe destacar que por Norman se demuestra que, aunque el tamaño de la
    muestra sea pequeña y los datos no puedan ser distribuidos normalmente o los
    datos sean de escalas de tipo \textit{Likert}, los métodos paramétricos como
    el análisis de varianza, la regresión y la correlación pueden ser
    utilizados.

\end{itemize}
\end{frame}


\frameimage[Juegos Serios. Corrientes relacionadas]{../juegos_serios/corrientes_paralelas.png}
\frameimage[Implementación, Flujo del motor]{../solucion/images/rules_flow.png}

\begin{frame}[noframenumbering]
\frametitle{\apendixframetitle}
\framesubtitle{Implementación. Flujo}
\centering
\begin{tikzpicture}[thick,scale=0.6, every node/.style={transform shape}]
    % Place nodes
    \node [block] (1) {1. Objetivos de diseño};
    \node [block, right of=1, node distance=5cm] (2) {2. Competencias básicas relacionadas con la educación};
    \node [block, right of=2, node distance=5cm] (3) {3. Investigación del dominio};
    \node [block, below of=3, node distance=3cm] (4) {4. Diseño del juego};
    \node [block, left of=4, node distance=5cm] (5) {5. Tiempo en el juego};
    \node [block, left of=5, node distance=5cm] (6) {6. Acciones de jugabilidad};
    \node [block, below of=6, node distance=3cm] (7) {7. Indicadores};
    \node [block, right of=7, node distance=5cm] (8) {8. Representación e interacción};
    \node [block, right of=8, node distance=5cm] (9) {9. Implementación};
    \node [block, below of=9, node distance=3cm] (10) {10. Evaluación};
    % Draw edges
    \path [line] (1) -- (2);
    \path [line] (2) -- (3);
    \path [line] (3) -- (4);
    \path [line] (4) -- (5);
    \path [line] (5) -- (6);
    \path [line] (6) -- (7);
    \path [line] (7) -- (8);
    \path [line] (8) -- (9);
    \path [line] (9) -- (10);
\end{tikzpicture}
\end{frame}

\begin{frame}[noframenumbering]
\frametitle{\apendixframetitle}
\framesubtitle{Implementación. Descripción}

\begin{itemize}
    \item Orientado a eventos (utilizando el API de \cs{})
    \item Registro de LOG's en archivos del teléfono en formato Json.
\end{itemize}
\end{frame}

\frameimage[Juegos serios, Dimensiones]{../juegos_serios/desarrollo_dimensiones.png}
