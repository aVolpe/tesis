\section{Evaluación}
\setcounter{sectiontotal}{10}


\begin{frame}
\frametitle{\pagetitle}
\framesubtitle{Metodología}

\pause{} 

\begin{table}
    \tiny
\centering
\begin{tabulary}{\textwidth}{rLLLL}
\toprule
\textbf{Objetivo}      & Evaluar la usabilidad de la interfaz      & Determinar la muestra                                              & Evaluar el uso y la aceptación & Evaluar el conocimiento\\
\midrule
\textbf{Participantes} & Alumnos de la carrera de II de la FP-UNA. & \multicolumn{3}{c}{Alumnos de la carrera de enfermería del IAB} \\
\midrule
\textbf{Muestra}       & 8 Alumnos                                 & 124 Alumnos                                                        & 11 Alumnos                     & 124 alumnos \\
\midrule
\textbf{Lugar}         & FP-UNA                                    & IAB                                                                &                                & IAB\\
\midrule
\textbf{Duración}      & 15 minutos                                & 10 minutos                                                         & 20 días                        & 20 minutos\\
\midrule
\textbf{Fases}           & \tabitem Explicación de los procedimientos. & \tabitem Presentación.          & \tabitem Explicación de la interfaz. & \tabitem Explicación de la prueba.\\
                & \tabitem Tareas.                            & \tabitem Encuesta.              & \tabitem Utilización de la solución. & \tabitem Encuesta sobre conocimiento\\
                & \tabitem Encuesta de usabilidad.            & \tabitem Recopilación de datos. & \tabitem Encuesta de apreciación.    & \tabitem Recopilación de datos.\\
                & \tabitem Recopilación de datos.             &                                 & \tabitem Recopilación de datos.\\
\bottomrule
\end{tabulary}
\end{table}

\end{frame}

\begin{frame}[fragile]
    \frametitle{\pagetitle}
    \framesubtitle{Pruebas preliminares}

\begin{filecontents}{interfazuso.dat}
n   p       s
1	9.38	6.48
2   5.88	4.64
3   7.75	3.13
\end{filecontents}
\pgfplotstableread{interfazuso.dat}{\InterfazUso}

\begin{figure}
    \begin{subfigure}[b]{.5\linewidth}
        \centering
        \begin{tikzpicture}[scale=.45]
           \begin{axis}[ybar,%
              legend pos=outer north east,
              xmin=0,
              xmax=4,
              xtick=data,
              symbolic x coords={0,1,2,3,4},
              ymin=0,ymax=10,
              %ytick={0,2,4,6,8,10},
              xticklabels={Contextual,Interfaz,Herramienta},
              ylabel= Tiempo (s),
              xlabel= Tipo de acción,
              bar width=10pt,
              %enlarge x limits={abs=2},
                ]   
        \addplot[color=blue,ybar,fill=blue!75,area legend] table [x = {n}, y = {p}] {\InterfazUso};
        \addlegendentry{Primer}
        \addplot[color=red,ybar,fill=red!75,area legend] table [x = {n}, y = {s}]
        {\InterfazUso};
        \addlegendentry[align=left]{Promedio \\ siguientes}
        \end{axis}
        \end{tikzpicture}
        \caption{Tiempo por tipo de actividad}
    \end{subfigure}\hfill
    \pause
    \begin{subtable}[b]{.5\linewidth}
        \tiny
        \begin{tabular}{lr}
        \toprule
        Variable & Apreaciación \\
        \midrule
        Gráficos                & \pin{ForestGreen}{De acuerdo}              \\
        Interacción Entorno     & \pin{BurntOrange}{Neutral}                 \\
        Interacción Objetos     & \pin{BurntOrange}{Neutral}                 \\
        Características Entorno & \pin{ForestGreen}{Parcialmente de acuerdo} \\
        Usabililidad            & \pin{BurntOrange}{Neutral}                 \\
        Integración             & \pin{ForestGreen}{Parcialmente de acuerdo} \\
        \bottomrule
    \end{tabular}
        \caption{Disconformidad por variable}
    \end{subtable}
\end{figure}

\end{frame}

\begin{frame}
    \frametitle{\pagetitle}
    \framesubtitle{Encuesta para determinar muestra}

\begin{figure}
    \begin{subfigure}[b]{.3\linewidth}
        \centering
        \begin{tikzpicture}[thick,scale=0.35, every node/.style={transform shape}]
            \pie[
                %explode=.2,
                text=legend,
                %style=drop shadow,
                %radius=3,
                %scale font,
                explode={0.1,0.1,0.3,0.3}
                ]%
            {%
                31.2 / Plan pos-pago,
                57   / Paquetes pre-pago,
                5.4  / Sin acceso,
                6.4  / Acceso ocasional}
        \end{tikzpicture}
        \caption{Acceso a internet}
    \end{subfigure}\hfill
    \pause{}
    \begin{subfigure}[b]{.3\linewidth}
        \centering
        \begin{tikzpicture}[thick,scale=0.35, every node/.style={transform shape}]
            \pie[
                text=legend,
                rotate=61.3,
                explode={.1,.2,.2,.2}
                ]%
            {%
            61.3 / Android,
             8.6 / Symbian,
            12.9 / Windows Phone,
            17.2 / Otros}
        \end{tikzpicture}
        \caption{Sistema operativo}
    \end{subfigure}\\
    \pause{}
    \begin{subfigure}[b]{.5\linewidth}
        \centering
        \begin{tikzpicture}[thick,scale=0.35, every node/.style={transform shape}]
            \pie[
                text=legend,
                explode=.1
                ]%
            {%
                81.7 /  No Cumple,
                18.3 /  Cumple
            }
        \end{tikzpicture}
        \caption{Cumplimiento de requisitos}
    \end{subfigure}
\end{figure}

\end{frame}

% Utilización
\begin{frame}
\frametitle{\pagetitle}
\framesubtitle{Utilización de la solución}

\begin{table}
\centering
\small
\begin{tabular}{lr}
\toprule
Partidas                         & 99 \\
Primera partida                  & 4 de Noviembre de 2014 \\
Última partida                   & 23 de Noviembre de 2014 \\
\midrule
\onslide+<2->{Tiempo total       & 11.134 s \\
Promedio de tiempo por partida   & 112 s}
\\\midrule
\onslide+<3->{Acciones           & 2.944 \\
Promedio de acciones por partida & 30}
\\\midrule
\onslide+<4->{Usuarios           & \textcolor{red}{8} \\
Promedio de partidas por usuario & 12}
\\\bottomrule
\end{tabular}
\end{table}

\end{frame}
% Utilización por hora
\begin{frame}[t,fragile]
    \frametitle{\pagetitle}
    \framesubtitle{Utilización de la solución}
\centering

% Datos
\begin{filecontents}{utilizaciontiempo.dat}
Hora	Cantidad
 0	 0
 1	 9
 2	 4
 3	 2
 4	 0
 5	11
 6	 8
 7	 3
 8	 5
 9	 7
10	 7
11	12
12	15
13	 9
14	 6
15	 0
16	 0
17	 0
18	 0
19	 0
20	 0
21	 0
22	 0
23	 1
24	 0
\end{filecontents}
\pgfplotstableread{utilizaciontiempo.dat}{\UtilizacionTiempo}

\begin{figure}
    \begin{tikzpicture}[thick,scale=0.85, every node/.style={transform shape}]
        \begin{axis}[
            title={},
            xlabel={Hora},
            ylabel={Partidas},
            xmin=0, xmax=24,
            ymin=0, ymax=16,
            xtick       = {0  , 3  , 6  , 9  , 12 , 15 , 18 , 21 , 24},
            xticklabels = {12 , 15 , 18 , 21 ,  0 , 3  ,  6 ,  9 , 12},
            %ytick={0,.25,.50,.75,1},
            legend pos=north east,
            ymajorgrids=true,
            xmajorgrids=true,
            grid style=dashed,
        ]
         
        \addplot[color=blue,fill=blue!5] table [x = {Hora}, y = {Cantidad}] {\UtilizacionTiempo};
        \end{axis}
    \end{tikzpicture}
    \caption{Utilización por hora}
\end{figure}
 

\end{frame}

% Utilización progreso
\begin{frame}
\frametitle{\pagetitle}
\framesubtitle{Utilización de la solución}

\begin{table}
\centering
\scriptsize
\begin{tabulary}{\linewidth}{lrrrr}
\toprule
& \multicolumn{2}{c}{Venopunción}    & \multicolumn{2}{c}{Glasgow}  \\
\cmidrule(lr){2-3}
\cmidrule(lr){4-5}
\textbf{Promedio} & \textbf{Puntaje (17)} & \textbf{Desviación} &
                  \textbf{Puntaje (4)} & \textbf{Desviación} \\
\midrule
Primera vez       & $6.1$        & $6.3$      & $1.4$       & $0.9$ \\
Siguientes veces  & $8.0$        & $3.8$      & $1.6$       & $1.5$ \\
Más altos          & $11.0$       & $4.9$      & $2.6$       & $0.9$ \\
\bottomrule
\end{tabulary}
\caption{Mejora por escenario}
\end{table}

\end{frame}

% Apreciación de la solución
\begin{frame}
\frametitle{\pagetitle}
\framesubtitle{Apreciación de la solución}

\begin{table}
\tiny
\begin{tabulary}{\textwidth}{lrll}
\toprule
\textbf{Factor}                   & \shortstack{\textbf{Promedio encuesta} \\ (Promedio estandarizado)} & \textbf{Puntos fuertes}             & \textbf{Puntos débiles} \\
\midrule
Motivación          & De acuerdo (0.67) & \tabitem Puntaje                    & \tabitem Tiempo  \\
                    &                   & \tabitem Socialización              & \\
\midrule
Exploración         & De acuerdo (0.68) & \tabitem Aleatoriedad               & \tabitem Utilización    \\
                    &                   & \tabitem Funciones                  & \\
\midrule
Inmersión           & De acuerdo (0.63) & \tabitem Escenografía               & \tabitem Pertenencia\\
                    &                   & \tabitem Gráficos                   & \\
                    &                   & \tabitem Ordenes verbales           & \\
\midrule
Pedagogía           & De acuerdo (0.67) & \tabitem Comprensión                & \\
                    &                   & \tabitem Retroalimentación limitada & \\
\midrule
Representación      & Parcialmente      & \tabitem Movimientos motrices       & \tabitem Reacción verbal\\
                    & de acuerdo (0.53) &                                     & \tabitem Estados del paciente\\
\midrule
Retroalimentación   & Parcialmente      & \tabitem Detalles de los pasos      & \tabitem Iconos \\
                    & de acuerdo (0.60) &                                     & \\
\midrule
Utilidad            & De acuerdo (0.69) & \tabitem Complemento                & \\
                    &                   & \tabitem Facilitador                & \\
\bottomrule
\end{tabulary}
\caption{Aceptación por aspecto de la solución}
\end{table}

\end{frame}

%% Hipótesis
\begin{frame}
\frametitle{\pagetitle}
\framesubtitle{Aceptación por consideración de diseño}

\begin{table}
\scriptsize
\centering
\begin{tabular}{lcr}
\toprule
& \multicolumn{2}{c}{Promedio} \\
%\cmidrule(c){2-3} 
\cmidrule(lr){2-3}
Consideraciones de diseño          & Encuesta                & Estandarizado \\
\midrule
C1. Interacción a través de la voz & De acuerdo              & $0,55$ \\
C2. Extracción de elementos        & Parcialmente de acuerdo & $0,65$ \\
C3. Bioseguridad                   & De acuerdo              & $0,58$ \\
C4. Representación Iconográfica    & Parcialmente de acuerdo & $0,53$ \\
C5. Motivación                     & De acuerdo              & $0,65$ \\
C6. Retroalimentación limitada     & De acuerdo              & $0,61$ \\
C7. Movilidad                      & De acuerdo              & $0,66$ \\
\bottomrule
\end{tabular}
\caption{Aceptación por consideración de diseño}
\end{table}

\end{frame}

% Conocimiento
\begin{frame}[t,fragile]
\frametitle{\pagetitle}
\framesubtitle{Encuesta para evaluar conocimiento}
%\centering

% Datos
\begin{filecontents}{objetiva.dat}
n	total        muestra	    control
1	0.2020159639 0.3636363636	0.1818181818
2	0.6027491017 0.6363636364	0.6
3	0.1332966457 0.09090909091	0.1363636364
4	0.2580191298 0.2727272727	0.2545454545
5	0.5869865976 0.8181818182	0.5636363636
6	0.1640933841 0	            0.1834862385
7	0.5256475617 0.6363636364	0.5137614679
8	0.2947317706 0.4545454545	0.2752293578
9	0.3081905611 0.1818181818	0.3211009174
10	0.4498120076 0.3636363636	0.4587155963
\end{filecontents}
\pgfplotstableread{objetiva.dat}{\Objetiva}

\begin{figure}
%\tiny
\begin{subfigure}[b]{.5\linewidth}
    \begin{tikzpicture}[thick,scale=0.5, every node/.style={transform shape}]
        \begin{axis}[
            ybar,
            width=\linewidth,
            bar width=9pt,
            x=25pt,
            height=10cm,
            axis on top,
            enlarge x limits=true,
            tick align=inside,
            tickwidth=0pt,
            title={},
            xlabel={Pregunta},
            ylabel={Puntos},
            xmin=1, xmax=10,
            ymin=0, ymax=1,
            xtick={0,1,2,3,4,5,6,7,8,9,10},
            ytick={0,.25,.50,.75,1},
            legend pos=north east,
            ymajorgrids=true,
            xmajorgrids=true,
            grid style=dashed,
        ]
         
        \only<1->{%
        \addplot[color=blue,fill=blue!30] table [x = {n}, y = {muestra}] {\Objetiva};
        \addlegendentry{Muestra}}
        \only<2->{%
        \addplot[color=red,fill=red!30] table [x = {n}, y = {control}] {\Objetiva};
        \addlegendentry{Control}}
        %\only<3->{\addplot[color=black,dashed] table [x = {n}, y = {total}] {\Objetiva};
        %\addlegendentry{Total}}
        \end{axis}
    \end{tikzpicture}
    \caption{Comparación de puntaje*}
\end{subfigure}\hfill
\pause
\pause
\pause
\begin{subtable}[b]{.5\linewidth}
    \centering
    \scriptsize
    \begin{tabular}{lr}
    \toprule
    \textbf{Grupo}         & \textbf{Promedio (10)} \\
    \midrule
    \onslide+<4->{Usuarios & \textbf{3.82}  \\}
    \onslide+<5->{Control  & \textbf{3.47} \\\midrule}
    \onslide+<6->{Total    & \textbf{3.49}}
    \\\bottomrule
    \end{tabular}
    \caption{Puntaje promedio por grupo*}
\end{subtable}
\end{figure}
 
\tiny 

\textbf{*} Los usuarios participaron del experimento luego del examen
final de la materia \textit{Enfemería en Urgencias}, la cual requiere el dominio
de ambos temas simulados como parte de sus competencias básicas.

\end{frame}

% Correlación
\begin{frame}
\frametitle{\pagetitle}
\framesubtitle{Correlación}

\begin{table}
\tiny
\centering
\begin{tabulary}{\textwidth}{Lrrrrrr}
\toprule
                                  & \multicolumn{2}{c}{Puntaje Max.} & \multicolumn{2}{c}{Tiempo de juego} & \multicolumn{2}{c}{Puntaje examen} \\
\cmidrule(lr){2-3} 
\cmidrule(lr){4-5} 
\cmidrule(lr){6-7}
                            & Venopunción             & Glasgow                & Venopunción          & Glasgow                 & Venopunción             & Glasgow \\
\midrule
P.M. Venopunción            & 1                       & 0.12                   & \pin{NavyBlue}{0.30} & \pin{RoyalPurple}{0.35} & \pin{ForestGreen}{0.74} & 0.55 \\
P.M. Glasgow                & 0.12                    & 1                      & 0.32                 & \pin{Aquamarine}{0.61}  & 0                       & \pin{BrickRed}{0.54} \\
\midrule
T.J. Venopunción            & \pin{NavyBlue}{0.30}    & 0.32                   & 1                    & 0.29                    & 0.04                    & 0.05\\
T.J. Glasgow                & \pin{RoyalPurple}{0.35} & \pin{Aquamarine}{0.61} & 0.29                 & 1                       & 0.69                    & \pin{Orange}{0.86}\\
\midrule
P.E. Venopunción            & \pin{ForestGreen}{0.74} & 0                      & 0.04                 & 0.69                    & 1                       & \pin{VioletRed}{0.78} \\
P.E. Glasgow                & 0.55                    & \pin{BrickRed}{0.54}   & 0.05                 & \pin{Orange}{0.86}      & \pin{VioletRed}{0.78}   & 1 \\
\bottomrule               
\end{tabulary}
\caption{Correlación entre factores estudiados} 
\end{table}

\tiny
\begin{itemize}
\item \pin{NavyBlue}{Venopunción. Rendimiento en la solución y tiempo de juego.}
\item \pin{ForestGreen}{Venopunción. Rendimiento en la solucion y en el exámen.}
\item \pin{Aquamarine}{Glasgow. Rendimiento en la solución y tiempo de juego.}
\item \pin{BrickRed}{Glasgow. Rendimiento en la solución y en el exámen.}
\item \pin{Orange}{Glasgow. Tiempo de juego y rendimiento en el exámen.}
\item \pin{RoyalPurple}{Rendimiento Venopunción y tiempo de juego Glasgow.}
\item \pin{VioletRed}{Puntaje examen Glasgow y Puntaje examen Venopunción}
\end{itemize}
\note{%

\begin{tabular}{|r|r|r|r|r|r|}
\hline  & M.F.  & F.     & M.     & W.     & N \\
\hline
Mínimo  & $0.7$ & $0.69$ & $0.39$ & $0.29$ & $0.19$ \\
\hline
\end{tabular}}
\end{frame}

