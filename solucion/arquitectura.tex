%! TEX root = ../main.tex

\section{Aspecto generales}
\label{sec:solucion}

%\observacion{Esta sección 7 deberá empezar con una descripción general (gráfico)
%y debe ir explicando parte por parte}

A continuación se describen los aspectos generales de la solución, 
se toma como base de desarrollo el flujo de diseño definido
por~\cite{pereira2009design} y descrito en~\ref{sec:desarrollo}.

Esta sección se enfoca en los aspectos técnicos de la creación de la solución y en
las competencias básicas relacionadas con la educación (segundo paso de la guía
descrita en~\ref{sec:desarrollo}) que se definieron en las
secciones~\ref{sec:glasgow} y~\ref{sec:hemocultivo}.

La solución se compone de varios escenarios, un escenario representa a un procedimiento definido
en~\ref{sec:seleccion_escenas}, cada escenario tiene sus propias entidades, eventos
y requisitos específicos. %\fixme{Adicionalmente existe otra escena denominada
%Inicio que es la escena donde se inicia la solución.}{?}
%Inicio estaba en un \enquote

%\observacion{Grafo de la arquitectura? Esta un poco desordenado}
%
%Adicionalmente la solución cuenta con pantallas, que son vistas donde el usuario
%puede seleccionar varias opciones, a continuación se describe el funcionamiento
%global de la solución y la transición entre las escenas, pantallas y diversas
%opciones existentes. Posteriormente se define como se interactúa con el entorno,
%y a continuación se evalúa al alumno durante su experiencia con la solución.

\subsection{Flujo de la solución}

%\observacion{Mejorar siguiente parrafo}
La solución consiste en varios escenarios. En cada escenario se presentan varias 
pantallas que muestran información relevante de acuerdo a
la situación de la simulación. La figura~\ref{fig:grafo_estados} es una
representación abstracta de este flujo de acciones disponibles dentro de la
solución.

\begin{figure}[H] 
\centering 
\includegraphics[scale=0.5]{solucion/images/grafo_escenas.png}
\caption{Diagrama de navegación entre los distintos escenarios y pantallas
    disponibles en la solución.}
\label{fig:grafo_estados}
\end{figure}

 
La figura~\ref{fig:grafo_estados} muestra que la solución empieza con un escenario
denominado \emph{Inicio}. En la pantalla \emph{Inicio} se observa el entorno y las
opciones que conducen a los escenarios \emph{Venopunción} y
\emph{Glasgow}. Otras opciones incluidas en el escenario \emph{Inicio} son
\begin{enumerate*}[label=\itshape\alph*\upshape)]
\item enviar datos de uso al \emph{back-end},
\item iniciar sesión en el \emph{Facebook}, y,
\item salir de la simulación.
\end{enumerate*}


El escenario \emph{Venopunción} permite realizar el procedimiento de extracción 
de muestras de sangre y finaliza al presionar la opción
\emph{Finalizar Partida}, inmediatamente después de esto se muestra la pantalla 
\emph{Resultados} con información de retroalimentación y la opciones de compartir 
resultado y volver a la escenario \emph{Inicio}.

Para iniciar el escenario \emph{Glasgow} existen dos caminos:
\begin{enumerate}
\item \textbf{Explorar}: al seleccionar la opción \emph{Explorar
        Glasgow} se muestra la pantalla \emph{Elegir estado del
        paciente}, donde se personaliza el estado del paciente para luego iniciar
    la escena. El escenario finaliza con la opción \emph{Finalizar Partida}, inmediatamente después de 
    finalizar se muestra la pantalla \emph{Resultados}.
\item \textbf{Evaluar}: al seleccionar la opción \emph{Evaluar
        Glasgow} se inicia la escena \emph{Glasgow} con un paciente en estado 
    aleatorio.
    Al finalizar el escenario, se muestra la pantalla \emph{Evaluar al paciente}
    donde se permite diagnosticar el estado del paciente, luego se pasa a la
    pantalla \emph{Resultados}.
\end{enumerate}

Finalmente, en la pantalla \emph{Resultados} del escenario \emph{Glasgow} también se permite 
compartir el resultado de la partida y volver al escenario \emph{Inicio}.

Todos los escenarios son utilizados de manera similar, la interacción con la
escena es siempre la misma.

\subsection{Transformaciones del punto de vista del usuario}

El usuario se desenvuelve en un entorno de tres dimensiones, en el cual realiza
las actividades relacionadas al procedimiento, se distinguen dos tipos de acciones
principales que el usuario puede realizar:

\begin{itemize}
    \item \textbf{Alejamiento o acercamiento}: es el acto de acercar o alejar la
        cámara, y por consiguiente al usuario del paciente. Se realiza
        utilizando dos dedos. Para realizar un acercamiento, mientras se
        mantiene presionada la pantalla con ambos dedos, se procede a alejar un
        dedo del otro. En cambio para realizar un alejamiento, se debe acercar ambos
        dedos.
    \item \textbf{Desplazamiento}: se refiere al movimiento alrededor de
        un foco, que en ambas escenas es el paciente. Para realizarlo, se utiliza
        un dedo y se mueve el dedo en cualquier dirección, de esta manera la cámara se moverá
        en la dirección contraria.
\end{itemize}


%\subsection{Evaluación del usuario}
%\label{sec:eca_impl}
%
%La evaluación del alumno en el procedimiento de \emph{Glasgow} es sencilla, como el usuario debe dar 
%su diagnóstico al finalizar la partida sólo se requiere comparar este diagnóstico con el correcto. Para realizar la evaluación del alumno en el procedimiento de \emph{Venopunción} se utiliza un motor 
%\Gls{eca}, las características de estos motores están descritas en la sección~\ref{sec:eca}.
%
%Definir si las acciones de un usuario son correctas utilizando un motor
%\Gls{eca} es sencillo teniendo en cuenta que sólo se deben definir un
%conjunto de acciones que se deben realizar, y agregar una condición que verifica si
%los pasos realizados fueron los correctos.
%
%%Se describe como se crean las reglas, de manera a explicar como son utilizadas
%%para la evaluación de las acciones realizadas por el usuario.
%
%A continuación se muestra como se definen las reglas para la validación de las acciones realizadas 
%por el usuario.
%
%\begin{algorithm}[H]
%\caption{Creación de regla de verificación de calzado de guantes}
%\label{alg:rule_guante}
%\lstset{style=sharpc}
%\begin{lstlisting}
%Rule.New().
%     When(``enfermero.guantes.calzar'').
%     Then(enviroment => enviroment.
%            estadoPaciente.TieneManosLimpias()).
%\end{lstlisting}
%\end{algorithm}
%
%La regla del algoritmo~\ref{alg:rule_guante} controla que el estudiante ha
%realizado la acción \enquote{Calzarse los guantes}, y en ese momento tenga las
%manos limpias.
%
%
%
%\subsubsection{Estados de una regla}
%
%El ciclo de vida de una regla, como se observa en la figura~\ref{fig:rule_flow},
%se compone de los siguientes estados:
%
%\begin{figure}[H]
%\centering
%\includegraphics[width=12cm]{solucion/images/rules_flow.png}
%\caption{Ciclo de vida de una regla}
%\label{fig:rule_flow}
%\end{figure}
%
%\begin{enumerate}
%\item \textbf{BEGIN:} es una regla que recién fue creada, no se realiza ninguna
%	acción.
%\item \textbf{WAITING\_FOR\_RULE:} es un estado en el que esta esperando que otras reglas
%	sean lanzadas. En este estado, es un suscriptor de las reglas por la que
%	espera, y no forma parte del ciclo de ejecución del motor de reglas.
%\item \textbf{WAITING\_FOR\_EVENT:} es un estado en el que está esperando que sean
%	lanzados los eventos a los que está escuchando, este es el estado principal. En
%	este estado, es un suscriptor de los eventos por los que espera, y no
%	forma parte del ciclo de ejecución del motor de reglas.
%\item \textbf{WAITING\_FOR\_CONDITION:} la regla ya no espera por ningún evento y las
%	reglas de las que depende ya han sido lanzadas, se verifica cada cierto
%	tiempo si el entorno cumple con una condición definida. 
%\item \textbf{FINISH:} estado final de una regla.
%\end{enumerate}
%
%
%\subsubsection{Motor de ejecución}
%
%Un motor de reglas \gls{eca}, requiere de un proceso que evalúe constantemente
%las reglas para verificar si las mismas deben ser lanzadas o
%no\cite{bailey2004event,galton2002two}, un algoritmo comúnmente utilizado para
%realizar la verificación es el algoritmo de \enquote{RETE}\cite{de2001eca}. La cantidad de
%reglas definidas y la no dependencia circular entre ellas, hace innecesario la
%implementación de tal algoritmo\cite{de2001eca}. 
%
%De acuerdo a la descripción dada en~\ref{sec:eca_ejecucion}, la propuesta
%implementada utiliza una ejecución inmediata, principalmente por la sencillez
%de las reglas, es decir, las reglas no realizan un proceso complejo, solamente
%controlan el estado del entorno y lo validan.
%
%Además, la ejecución inmediata es importante por que el entorno no sufre
%modificaciones entre el evento lanzado y la ejecución de la regla, según
%\cite{bailey2004event}, este es el factor más importante para determinar el tipo
%de ejecución deseado.
%
%El motor de reglas actúa sobre aquellas reglas en estado
%\emph{WAITING\_FOR\_CONDITION} e invoca al procedimiento que se encarga de
%validar si la regla puede ser activada (el procedimiento es único por cada
%regla), si el mismo determina que la regla puede ser lanzada, el motor ejecuta
%la acción de la regla y modifica el estado de la regla a \emph{FINISH}.

%DEBE QUEDAR COMENTADO LO DE ABAJO
%A continuación, se describen todos los escenarios, primeramente se da una
%descripción general de los escenarios y se procede a explicar los detalles de
%los mismos, incluyendo las entidades, eventos y acciones que pueden ser
%realizados en el mismo.
