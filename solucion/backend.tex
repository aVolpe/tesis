\section{Backend de la solución}

La solución almacena información detallada acerca de las acciones del usuario,
las condiciones de estas acciones y el contexto en el cual fueron ejecutadas.

Esta información es almacenada en un servidor dedicado para su posterior
análisis.

\subsection{Registro de usuarios}

\observacion{Una vez que definen lo que son los registros pueden referirse como
registros}

Para poder determinar a qué usuario corresponde qué conjunto de datos, durante
la instalación de la solución en los dispositivos móviles de los usuarios, se
ingresa como un dato adicional el número de teléfono del mismo, estos datos se
registran en el \textit{BackEnd} al mismo tiempo en el que se muestra la
solución al usuario.

De esta forma, en el sistema se encuentran los datos proveídos por la
simulación, así como el nombre del usuario. Es necesario almacenar el nombre del
usuario para las diversas encuestas que se realizan (las encuestas para alumnos
que participaron en la prueba no son anónimas), de esta manera se puede saber,
dada una encuesta, a que alumno corresponde y el uso que le dio a la solución.
Estas encuestas son explicadas en detalle en~\ref{chap:evaluacion} y sus
resultados en~\ref{chap:analisis}.

\subsection{Registro de eventos}
\label{sec:backend_reg_eventos}

Las acciones que realiza el usuario son almacenadas en un archivo temporal,
dentro del dispositivo móvil del usuario, cuando este decide enviar la
información, estos datos se transmiten por la red a un servidor que almacena el
\textit{BackEnd}.

Una vez que el usuario envía sus datos de uso, el archivo de registros local se
limpia, permitiendo así que nuevos registros sean añadidos. Adicionalmente,
existe un archivo de respaldo, que contiene toda la información que se registró
del usuario, incluyendo aquellas que ya fueron enviadas al servidor
\textit{BackEnd}.

\subsection{Detalles de implementación}

El \textit{BackEnd} es un sistema web, desarrollado en \textit{Java}, y
\fixme{desplegado}{Language} en un servicio \textit{OpenShift}, el cual está
disponible $24$ horas al día, los $7$ días a la semana, asegurando así, que
cuando el usuario desee enviar datos lo pueda hacer sin problemas.

El sistema tiene dos servicios web que permiten el registro de usuarios y el
registro de eventos, ambos reciben una lista de elementos y los almacenan en una
base de datos \textit{PostgreSQL}.

Estos dos servicios implementan un mecanismo de seguridad sencillo, basado en
usuario y contraseña, el único objetivo de este mecanismo, es que los datos no
sean fácilmente accesibles, pues contienen datos sensibles.

La petición enviada desde la solución, contiene el número de teléfono del
usuario, un identificador único generado por \textit{Unity3D} y una lista de
eventos, en formato \Gls{json}.

\subsection{Utilización de información}

Las tablas~\ref{tab:glasgow_registro} y~\ref{tab:hemocultivo_registro} muestran
como la información almacenada puede ser utilizada para evaluar las acciones de
un usuario.

Por ejemplo, se puede analizar cuantas veces los usuarios activaron el menú
\enquote{Acciones de voz}, y de estas, cuantas veces lo utilizaron, para así
poder comprobar la hipótesis \enquote{Comandos de voz}
(ver~\ref{sec:hipotesis}), esta información es útil además para medir el
nivel de sensibilidad necesaria para activar el menú.

Además de las tablas~\ref{tab:glasgow_registro}
y~\ref{tab:hemocultivo_registro}, existe información que puede ser extraída del
registro de manera implícita, como:

\begin{itemize}
\item Cantidad de sesiones, de manera global y por usuario.
\item Tiempo utilizado, por sesión, por usuario, por día, etc.
\item Frecuencia de uso.
\item Rango de tiempo en el cual se utilizo la solución.
\item Evolución de los usuarios, comparando registros de varias sesiones.
\end{itemize}

Adicionalmente, al realizar una encuesta sobre los conocimientos de los alumnos
al finalizar el periodo de prueba, se pueden utilizar los registros para
verificar si las respuestas dadas en la encuesta corresponden a la utilización
de la solución.

El registro de actividad es la herramienta utilizada para realizar la evaluación
del uso de la solución.
