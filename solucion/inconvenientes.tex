\section{Inconvenientes de diseño del front-end}
%\observacion{mejorar conexión}
Una vez finalizada la descripción sobre el front-end de la solución, en esta sección 
detallaremos los inconvenientes que se presentaron en su diseño.

Los mayores inconvenientes de diseño del front-end se dieron en el momento de
validar tanto el contenido de la aplicación como la interfaz de usuario, para
sobrellevar estos inconvenientes fue requerida la intervención de terceros. A continuación 
se explica en detalle cada uno de los casos.

\subsection{Interfaz gráfica de usuario}

%\observacion{Resumir todo esto}

Como parte del diseño y desarrollo de la solución se realizó una prueba de
interfaz gráfica de usuario con alumnos de la carrera de Ingeniería en Informática de la
\Gls{fpuna}, estas pruebas fueron realizadas con personas que están
acostumbradas al uso de interfaces similares a la de la solución y que, de hecho pueden ser más
criticas a la hora de evaluarlas. Esta prueba y sus resultados se detallan  en el
capítulo~\ref{chap:evaluacion}.

Principalmente son dos las cualidades de una interfaz gráfica que se pueden
someter a prueba: la funcionalidad y la usabilidad. Con la primera se pretende
responder preguntas como \textit{¿Se puede usar cierta función?},
\textit{¿Funciona como se espera?}, o \textit{¿Es correcta?}; y con respecto a
al usabilidad, se espera poder responder a \textit{¿Puede el usuario
    utilizar fácilmente la función?}, o \textit{¿Su uso es intuitivo y fácil de
    aprender?}\cite{fragaverificacion}.

Las pruebas de interfaces de usuario ayudan a que los usuarios puedan
concentrarse más en el problema en vez de poner los esfuerzos en recordar todas
las opciones que ofrece la solución que se utiliza para resolver el
problema\cite{horowitz1993graphical}.

Luego de las pruebas de interfaz de usuario, se hicieron correcciones a los
problemas encontrados. Estas correcciones fueron
probadas por profesores de la carrera de enfermería del \Gls{iab} los cuales
dieron su visto bueno.

%Otra de las razones por la cual la prueba fue realizada con alumnos que no
%formaban parte de la población a la que iba dirigida la aplicación, es la poca
%disponibilidad de tiempo con la que cuentan los alumnos de enfermería y más aún
%los profesionales que están encargados de su aprendizaje.

\subsection{Validaciones de contenido}

Llamamos validación de la simulación o de la aplicación desarrollada al hecho de
que el contenido de la misma sea correcto y además que la forma de realizar o
representar dicho procedimiento este acorde al mismo. Este tipo de validaciones
fueron realizadas reiteradamente en reuniones con distintos profesores de la
carrera de enfermería del \Gls{iab}.

Cada corrección solicitada fue evaluada y aprobada posteriormente por los
mismos. Como validación final la aplicación fue presentada en totalidad frente a
un plenario de cuatro profesores del instituto.

El mayor inconveniente en cuanto a las validaciones fueron la forma de
representación tanto de la información como de la simulación de objetos.
