\section{\Gls{tic} en la educación}
%1/2 página

Las \Gls{tic} son un conjunto de herramientas tecnológicas y recursos utilizados para
comunicar, crear, diseminar, almacenar y manejar la
información\cite{unesco:ict}. Estas tecnologías abarcan computadoras
personales,internet, radio, televisión y telefonía\cite{tinio:ict} entre otros.

La utilización actual de las \Gls{tic} en la educación no es un fenómeno aislado,
responde a una evolución constante de la tecnología y metodología utilizada.
Existen cuatro pedagogías de entre varias en las que las tics han sido
utilizadas de manera activa, las cuales son el instruccionismo o educación
tradicional, el conductismo, el constructivismo y finalmente, el
construccionismo. Esto no implica que las tics no puedan ser aplicadas a otras
pedagogías, es más, existen otras corrientes que utilizan las \Gls{tic}, de diversas
maneras como el cognoscitivismo\cite{egenfeldt2007third} y el
conectivismo\cite{white:ict}. 

Este trabajo se basa en el construccionismo, pedagogía según la cual el
conocimiento es construido por el estudiante en lugar de ser trasmitido por el
profesor\cite{moses:2003} y esto sucede particularmente cuando el mismo se
involucre en la elaboración de un producto o artefacto que tenga un significado
y pueda ser compartido\cite{valdivia:sg}. El construccionismo y las tics siempre
han estado relacionados, ya que el mismo se originó con un lenguaje de
programación (LOGO)\cite{ict:ttc}. Posee un enfoque diferente en cuanto al uso
de las tics en la educación. Esta pedagogía se diferencia de la educación
tradicional en que el estudiante ya no es un receptor pasivo de información, en
cambio, el mismo participa activamente del proceso de aprendizaje construyendo
su propio conocimiento. Se diferencia del instruccionismo en que el
construccionismo utiliza la tecnología como medio cognitivo y no para la entrega
de contenido.

%\begin{itemize}
%\item Que son
%\item Educacion tradicional y mencionar las otras corrientes
%\item Construccionismo
%\end{itemize}