\section{Juegos serios}
1 página

Un juego serio es un videojuego que posee un propósito educacional explícito y
bien elaborado, y cuya intención no es la de únicamente entretener al
usuario\cite{abt1987serious,sg:aoverview,damien:sg}.

Los \emph{Juegos Serios} proveen una oportunidad muy importante para ayudar en
la enseñanza y desarrollo de profesionales\cite{mariluz:seiousgames}, por que
ayudan a crear el tipo de educación que los adultos prefieren, proveen
mecanismos para que los estudiantes cometan errores y experimenten con sus
ideas, con su conocimiento y con la teoría en un ambiente protegido sin riesgos
para la vida o la identidad\cite{sg:aoverview}. 

El campo de los \emph{Juegos Serios} rechaza la idea de que los profesionales de
la educación pueden ser reemplazados, la labor de estos profesionales es
imprescindible para la reflexión y orientación del
aprendizaje\cite{elearning:seiousgames}.

\subsection{Ventajas y desafíos}

Los juegos serios tienen ventajas y desafíos que los hacen únicos en su enfoque
para con la utilización de las TIC's y posee las siguientes ventajas:


\begin{itemize}

\item \textbf{Motivación interna}: favorecen la autoestima y tienen un factor
    motivacional\cite{guenaga2013serious}, permitiendo una implicación mayor del
    usuario en la actividad\cite{sg:aoverview}. La implicación del usuario
    dentro de la actividad, es un tema central en el desarrollo de los juegos
    serios\cite{charsky:2010}.

\item \textbf{Apoyo al aprendizaje}: ayudan al aprendizaje es por que los mismos
    se desarrollan en un entorno significativo y relevante al
    contexto\cite{sg:aoverview}. Las teorías modernas de aprendizaje sugieren
    que el aprendizaje es más efectivo cuando es activo, experiencial y basado
    en problemas\cite{guenaga2013serious}.

\item \textbf{Menos limitaciones}: juegos serios permite a sus usuarios
    experimentar en entornos y sistemas que no son posibles en la vida real, por
    cuestiones de costo, tiempo y aspectos relacionados a la
    seguridad\cite{sg:aoverview}.

\item \textbf{Similitud a la realidad}: constituyen un escenario privilegiado
    para el desarrollo de todos los componentes de las competencias, ya que
    permiten desarrollar vivencias en las que ponerlos en practica, permitiendo
    el entrenamiento en situaciones que en muchas ocasiones son similares a las
    que se encuentran en entornos reales\cite{guenaga2013serious,sg:aoverview}.
    
\item \textbf{Estimulación sensorial}: aumentan la capacidad de coordinación,
    percepción espacial y ampliación del campo visual, lo que tiene una
    incidencia en la lectura y el manejo eficiente en ambientes
    3D\cite{guenaga2013serious}. 

\end{itemize}

El potencial de un juego serio no es ilimitado, existen múltiples desafíos que
deben ser superados para poder desarrollar un juego serio que obtenga las ventajas
citadas previamente y pueda ser de utilidad en la educación formal.


\begin{itemize}

\item \textbf{Falta de investigación}: aunque en los últimos años los estudios
    del impacto de los juegos serios han aumentado considerablemente, son
    necesarios más estudios para probar su eficiencia\cite{sg:aoverview}.

\item \textbf{Expectativas muy altas}: un juego serio, como el resto de la
    \textit{media}, no puede cambiar el comportamiento de una persona por sí
    solo, pero sí permiten al jugador explorar las opciones, tener en cuenta las
    consecuencias de sus actos y poner en práctica sus
    conocimientos\cite{education:games,stapleton2004serious,videojuegos:gonzaleztardon}. 

\item \textbf{Evaluación tradicional}: la forma tradicional de evaluación presenta
    dificultades a los juegos serios, por ejemplo, las pruebas tradicionales
    contienen un grupo de preguntas, las cuales son vistas de manera
    independiente, en cambio en un juego serio, las acciones son dependientes
    del contexto y las acciones previamente realizadas\cite{shute2009melding}.

\item \textbf{Utilización incorrecta}: en cuanto al objeto pedagógico, el área
    en la cual se utiliza un juego serio es un factor determinante para el éxito
    del mismo\cite{stapleton2004serious}.

\item \textbf{Falta de recursos}: uno de los factores más complicados a la hora
    del desarrollo de juegos serios es la limitación de recursos financieros,
    esto no quiere decir que no existan recursos para su desarrollo, sino que,
    comparados con los recursos invertidos en otras \textit{media}, el
    presupuesto es insignificante\cite{stapleton2004serious,sg:aoverview}. Como
    consecuencia de las limitaciones financieras, los desarrolladores no siempre
    pueden acceder a tecnología de última generación\cite{stapleton2004serious}

\end{itemize}

\subsection{Actualidad}

Esta sección muestra las áreas donde más frecuentemente se utilizan los juegos
serios.

\begin{itemize}

\item \textbf{Militar}: durante más de $30$ años los videojuegos han sido
    reconocidos como herramientas factibles en el entrenamiento de militares. En
    $1996$ fue lanzado un videojuego llamado \emph{Marine Doom} en donde la
    tarea de los jugadores era el aprendizaje de formas de ataque, conservación
    de municiones, comunicarse con eficacia, dar órdenes al equipo de trabajo
    entre otros. De esta manera tuvo lugar una forma de entrenamiento más
    atractivo, sin el costo, dificultad, riesgos e inconvenientes que implicaría
    el mismo entrenamiento en un entorno real. Además se podían crear
    situaciones que en el mundo real serían muy difíciles de replicar y donde
    los errores pueden ser catastróficos\cite{education:games}.

\item \textbf{Salud}: los juegos de salud se utilizan para la formación de
    profesionales basada en la simulación. En $2008$ el Centro de Simulación
    \emph{Hollier} en \emph{Birmingham}, Reino Unido, realizó una prueba que
    permitió a médicos jóvenes experimentar y entrenar para diversos escenarios
    médicos a través de maniquíes virtuales como pacientes, de este modo el
    aprendizaje se da por la experiencia. En su disertación, \emph{Roger D.
        Smith}, realizó una comparación entre la enseñanza tradicional y la
    formación mediante realidad virtual y el uso de herramientas basadas en la
    tecnología de videojuegos en cuanto a la cirugía laparoscópica. Como
    conclusión afirmó que lo último era más barato, requería menos tiempo y que
    permitió menos errores médicos cuando los médicos se presentaban en una
    cirugía real debido a, entre otras cosas, la posibilidad de repetición de la
    experiencia sin riesgo alguno\cite{education:games}. 

\item \textbf{Juegos corporativos}: Este tipo de videojuegos se han utilizado
    para la selección de personal, la mejora de comunicación entre los
    directivos y su personal de confianza, y la formación de nuevos empleados.
    Los juegos serios pueden ser utilizados incluso para elaborar planes de
    negocios\cite{education:games}. 

\end{itemize}

