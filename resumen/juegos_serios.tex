%! TEX root = main.tex

\section{Juegos serios}

Un juego serio se define como un videojuego que posee un propósito educacional explícito y
bien elaborado, y cuya intención no es la de únicamente entretener al
usuario\cite{abt1987serious,sg:aoverview,damien:sg}.

Los \emph{Juegos Serios} proveen una oportunidad importante para ayudar en
la enseñanza y desarrollo de profesionales\cite{mariluz:seiousgames}, permiten crear el tipo de educación que los adultos prefieren, proveen
mecanismos para que los estudiantes cometan errores y experimenten con sus
ideas, con su conocimiento y con la teoría en un ambiente protegido sin riesgos
para la vida o la identidad\cite{sg:aoverview}. El campo de los \emph{Juegos
    Serios} rechaza la idea de que los profesionales de la educación pueden ser
reemplazados, la labor de estos profesionales es imprescindible para la
reflexión y orientación del aprendizaje\cite{elearning:seiousgames}.

\subsection{Ventajas y desafíos}

Los juegos serios presentan ventajas y desafíos que los hacen únicos en su enfoque
para con la utilización de las \gls{tic}. Entre las ventajas podemos encontrar:


\begin{itemize}

\item \textbf{Motivación interna}: favorecen la autoestima y poseen un factor
    motivacional intrínseco\cite{guenaga2013serious,martin2008model}, permitiendo una implicación mayor del
    usuario en la actividad\cite{sg:aoverview}. La implicación del usuario
    dentro de la actividad, es un tema central en el desarrollo de los juegos
    serios\cite{charsky:2010}.

\item \textbf{Apoyo al aprendizaje}: se desarrollan en un entorno significativo y relevante al
    contexto\cite{sg:aoverview}. Las teorías modernas sugieren
    que el aprendizaje es más efectivo cuando es activo, experiencial y basado
    en problemas\cite{guenaga2013serious}.

\item \textbf{Menos limitaciones}: permite a los usuarios experimentar en entornos 
	y sistemas, los cuales 
    no están disponibles en la vida real, por motivos relacionados al 
    costo, tiempo y seguridad de los usuarios\cite{sg:aoverview}.
    
\item \textbf{Similitud a la realidad}: constituyen un escenario privilegiado
    para el desarrollo de todos los componentes de las competencias, ya que
    permiten desarrollar vivencias donde ponerlos en practica, permitiendo
    el entrenamiento en situaciones que en ocasiones son similares a las
    que se encuentran en entornos reales\cite{guenaga2013serious,sg:aoverview}.
    
\item \textbf{Estimulación sensorial}: aumentan la capacidad de coordinación,
    percepción espacial y ampliación del campo visual\cite{guenaga2013serious}. 

\end{itemize}

El potencial de los \emph{Juegos Serios} no es ilimitado, múltiples desafíos deben ser
superados para poder desarrollar un juego serio que obtenga las ventajas citadas
previamente y pueda ser de utilidad en la educación formal:


\begin{itemize}

\item \textbf{Falta de investigación}: aunque en los últimos años los estudios
    del impacto de los juegos serios han aumentado considerablemente, son
    necesarios más estudios para probar su eficiencia\cite{sg:aoverview}.

\item \textbf{Expectativas muy altas}: un juego serio, como otras herramientas
    multimedia, no puede cambiar el comportamiento de una persona por sí solo,
    pero sí permiten al jugador explorar las opciones, tener en cuenta las
    consecuencias de sus actos y poner en práctica sus
    conocimientos\cite{education:games,stapleton2004serious,videojuegos:gonzaleztardon}. 

\item \textbf{Evaluación tradicional}: la forma tradicional de evaluación presenta
    dificultades a los juegos serios, por ejemplo, las pruebas tradicionales
    contienen un grupo de preguntas, las cuales son vistas de manera
    independiente, en cambio en un juego serio, las acciones son dependientes
    del contexto y las acciones previamente realizadas\cite{shute2009melding}.

\item \textbf{Utilización incorrecta}: en cuanto al objetivo pedagógico, el área
    en la cual se utiliza un juego serio es un factor determinante para el éxito
    del mismo\cite{stapleton2004serious}.

\item \textbf{Falta de recursos}: los recursos financieros destinados al
    desarrollo de juego serios es insignificante al ser comparado con los
    recursos destinados a otras herramientas
    multimedia\cite{stapleton2004serious,sg:aoverview}. Como consecuencia de las
    limitaciones financieras, los desarrolladores no siempre pueden acceder a
    tecnología de última generación\cite{stapleton2004serious}

\end{itemize}

\subsection{Actualidad}

Las áreas donde más frecuentemente se utilizan los juegos serios son:

\begin{itemize}

\item \textbf{Militar}: durante más de $30$ años los videojuegos han sido
    reconocidos como herramientas factibles en el entrenamiento de militares. Se utilizan para el aprendizaje de formas de ataque, conservación
    de municiones, comunicación eficaz,
    entre otros.  De esta manera permiten entrenamientos más
    atractivos, sin el costo, dificultad, riesgos ni inconvenientes que implican
    los mismos entrenamientos en un entorno real\cite{education:games}.

\item \textbf{Salud}: los juegos de salud se utilizan para la formación de
    profesionales basada en la simulación. En comparación con la educación
    tradicional, los juegos serios son más baratos, requieren menos tiempo, son
    más seguro y ayudan a los profesionales a cometer menos errores en
    situaciones reales\cite{education:games}. 

\item \textbf{Juegos corporativos}: este tipo de videojuegos se han utilizado
    para la selección de personal, la mejora de comunicación entre los
    directivos y su personal de confianza, y la formación de nuevos empleados.
    Los juegos serios pueden ser utilizados incluso para elaborar planes de
    negocios\cite{education:games}. 

\end{itemize}

