\documentclass[conference]{IEEEtran}
\pagestyle{plain}

\let\proof\relax 
\let\endproof\relax
\usepackage{float}
\usepackage{amsmath,amsthm}

\usepackage[T1]{fontenc}
\usepackage[spanish, es-noshorthands]{babel}
\usepackage[utf8]{inputenc}
\usepackage{csquotes}
\usepackage[style=numeric,sorting=none,backend=biber]{biblatex}
\usepackage{tabularx}
\usepackage{pdflscape}
\usepackage{listings}
\usepackage{subfigure}
\usepackage{array}
\usepackage{perpage}
\usepackage{backnaur}
\usepackage{enumerate}
\usepackage{graphicx}

\MakePerPage{footnote}
\addbibresource{../bibliography.bib}

\newcommand{\foreign}[1]{{\it #1}}
\renewcommand{\thesubsection}{\Alph{subsection}}
\DeclareMathOperator*{\argmax}{arg\,max}
\renewcommand{\thetable}{\arabic{table}}


\makeatletter
\long\def\@makecaption#1#2{\ifx\@captype\@IEEEtablestring%
\footnotesize\begin{center}{\normalfont\footnotesize #1}\\
{\normalfont\footnotesize\scshape #2}\end{center}%
\@IEEEtablecaptionsepspace
\else
\@IEEEfigurecaptionsepspace
\setbox\@tempboxa\hbox{\normalfont\footnotesize {#1.}~~ #2}%
\ifdim \wd\@tempboxa >\hsize%
\setbox\@tempboxa\hbox{\normalfont\footnotesize {#1.}~~ }%
\parbox[t]{\hsize}{\normalfont\footnotesize \noindent\unhbox\@tempboxa#2}%
\else
\hbox to\hsize{\normalfont\footnotesize\hfil\box\@tempboxa\hfil}\fi\fi}
\makeatother

\begin{document}


\title{Construccionismo como apoyo a la enseñanza tradicional:  una
	aplicación a la formación de profesionales del área de enfermería}

\author{\IEEEauthorblockN{Mirta Gonzalez}
\IEEEauthorblockA{Facultad Politécnica\\
    Universidad Nacional de Asunción\\
    San Lorenzo, Paraguay\\
    Email: mirti.gonz@gmail.com}
\and
\IEEEauthorblockN{Arturo Volpe}
\IEEEauthorblockA{Facultad Politécnica\\
    Universidad Nacional de Asuncién\\
    San Lorenzo, Paraguay\\
    Email: arturovolpe@gmail.com}}

\maketitle
\thispagestyle{plain}


\begin{abstract}
Lo fundamental de todo proceso pedagógico es el aprendizaje y no la enseñanza.
Es el aprendizaje del estudiante y su participación el logro deseado.

\end{abstract}

\begin{IEEEkeywords}
    juego serio, simulación, e-Educación, open source, enfermería,
    construccionismo
\end{IEEEkeywords}


\printbibliography{}

\end{document}
