\documentclass[conference]{util/IEEEtran}
\pagestyle{plain}

\let\proof\relax 
\let\endproof\relax
\usepackage{float}
\usepackage{amsmath,amsthm}

\usepackage[T1]{fontenc}
\usepackage[spanish, es-noshorthands]{babel}
\usepackage[utf8]{inputenc}
\usepackage{csquotes}
\usepackage[style=numeric,sorting=none,backend=biber]{biblatex}
\usepackage{tabularx}
\usepackage{tabulary}
\usepackage{pdflscape}
\usepackage{listings}
\usepackage{subfigure}
\usepackage{array}
\usepackage{perpage}
\usepackage{backnaur}
\usepackage{enumerate}
%\let\labelindent\relax
%\usepackage[shortlabels]{enumitem}
\usepackage{graphicx}
\usepackage{pgfplots}
\usepackage{tikz}
\usepackage{tkz-kiviat}
\usepackage[bookmarks]{hyperref}
\usepackage{booktabs}
\usepackage{rotating}
\usepackage{glossaries}
\usepackage{../util/pgf-pie}
\usetikzlibrary{shapes,arrows,shadows}


\MakePerPage{footnote}
\addbibresource{../bibliography.bib}

\newcommand{\foreign}[1]{{\it #1}}
\renewcommand{\thesubsection}{\Alph{subsection}}
\DeclareMathOperator*{\argmax}{arg\,max}
\renewcommand{\thetable}{\arabic{table}}


\makeatletter
\long\def\@makecaption#1#2{\ifx\@captype\@IEEEtablestring%
\footnotesize\begin{center}{\normalfont\footnotesize #1}\\
{\normalfont\footnotesize\scshape #2}\end{center}%
\@IEEEtablecaptionsepspace
\else
\@IEEEfigurecaptionsepspace
\setbox\@tempboxa\hbox{\normalfont\footnotesize {#1.}~~ #2}%
\ifdim \wd\@tempboxa >\hsize%
\setbox\@tempboxa\hbox{\normalfont\footnotesize {#1.}~~ }%
\parbox[t]{\hsize}{\normalfont\footnotesize \noindent\unhbox\@tempboxa#2}%
\else
\hbox to\hsize{\normalfont\footnotesize\hfil\box\@tempboxa\hfil}\fi\fi}
\makeatother

\newcommand{\tabitem}{~~\llap{\textbullet}~~}

\newtoks\customtok


\renewcommand*{\newacronymhook}{%
 \edef\dosetkeys{\noexpand\setkeys{glossentry}{user1={},\the\glskeylisttok}}%
 \dosetkeys
 \ifcsempty{@glo@useri}%
 {%
   \expandafter\customtok\expandafter{\the\glsshorttok}%
 }%
 {%
   \edef\custom{\the\glsshorttok, \csexpandonce{@glo@useri}}%
   \expandafter\customtok\expandafter{\custom}%
 }%
}

\newcommand*{\custompostdesc}[1]{%
  \ifcsempty{glo@#1@useri}{}{(\glsentryuseri{#1})}%
}

\renewcommand*{\CustomAcronymFields}{%
  user1={},%
  name={\the\glsshorttok},%
  description={\the\glslongtok\noexpand\custompostdesc{\the\glslabeltok}},%
  first={\the\glslongtok\space(\the\customtok)},%
  firstplural={\the\glslongtok\noexpand\acrpluralsuffix\space(\the\customtok)}%
  text={\the\glsshorttok},%
  plural={\the\glsshorttok\noexpand\acrpluralsuffix}%
}

\makeglossaries

\begin{document}

\newacronym[user1=Conseil Européen pour la Recherche Nucléaire]{cern}{CERN}{Organización Europea para la Investigación Nuclear}
\newacronym[user1=One Laptop Per Child]{olpc}{OLPC}{Una computadora por niño}
\newacronym[user1=Massachusetts Institute of Technology]{mit}{MIT}{Instituto Tecnológico de Massachusetts}
\newacronym{tic}{TIC's}{Tecnologías de la información y la comunicación}
\newacronym[user1=Event-Condition-Action]{eca}{ECA}{acciones condicionadas por eventos}
\newacronym{iab}{IAB}{Instituto Andrés Barbero}
\newacronym[user1=Graphics Processing Unit]{gpu}{GPU}{Unidad de procesamiento de gráficos}
\newacronym[user1=Application Programming Interface]{api}{API}{Interfaz de programación de aplicaciones }




\title{Juegos serios como apoyo a la formación de profesionales: Una aplicación al área de enfermería}
% Juegos serios como herramienta educativa: una aplicación a la formación de 
%profesionales del área de enfermería.
%Juegos serios para evaluar la utilización de las tecnologías en la educación.
%Juegos serios como apoyo a la formación de profesionales. Una aplicación al 
% área de enfermería.
% Evaluación de los juegos serios como apoyo a la formación de profesionales: 
% Una aplicación al área de enfemería.

\author{\IEEEauthorblockN{Mirta González}
\IEEEauthorblockA{Facultad Politécnica\\
    Universidad Nacional de Asunción\\
    San Lorenzo, Paraguay\\
    Email: mirti.gonz@gmail.com}
\and
\IEEEauthorblockN{Arturo Volpe}
\IEEEauthorblockA{Facultad Politécnica\\
    Universidad Nacional de Asunción\\
    San Lorenzo, Paraguay\\
    Email: arturovolpe@gmail.com}}

\maketitle
\thispagestyle{plain}


\begin{abstract}

Este trabajo describe el uso de las \gls{tic} en el proceso de aprendizaje, especialmente en forma de juegos serios como área de investigación. Se da una breve reseña de la inclusión de las tecnologías en la educación, desde su uso tradicional como proveedor de información hasta su rol actual en la construcción de conocimiento. 

Se describen las principales características y áreas de aplicación de los juegos serios como herramienta educativa. Se introduce a los juegos serios como solución tecnológica a las problemáticas del área de enfermería, las problemáticas y los detalles de la solución en forma de juego
serio también son descritas.

Finalmente, se detallan las metodologías utilizadas para evaluar la solución y
los resultados obtenidos más importantes. Estos resultados junto con la
experiencia de desarrollar la solución permiten obtener conclusiones
relacionadas al diseño, implementación y evaluación  de juegos serios orientados
a la educación, así como también identificar posibles trabajos futuros en el
área.


\end{abstract}

\begin{IEEEkeywords}
    Juegos serios, simulación, enfermería, educación, ubicuidad, aprendizaje, tecnologías de la información y la educación. 
\end{IEEEkeywords}


% La evaluación
% Un párrafo de Unity
% Resumir las escenas (General, objetivo, gráfico)
% Media página de TICS
% Media página de juegos serios
% Mencionar lo que se usa en la evaluación
% Población (Los tres gráficos)
% Abstract

% Objetivos
% Media página tics
% 1/2 página problema
% 1 Párrafo Unity
% 1 Página Arquitectura
% Ejemplos escenas (que se hace y objetivos)
% 1/2 evaluación
%%  Población (3 gráficos)
%%  Correlación
%%  Factores
% Conclusiones

\section{Introducción}
\setcounter{sectiontotal}{3}

\begin{frame}
\frametitle{\pagetitle}
\framesubtitle{Descripción}
\begin{figure}
\includegraphics[scale=.3]{imagenes/nhanduti}
\end{figure}
\end{frame}

\begin{frame}
\frametitle{\pagetitle}
\framesubtitle{Objetivo general}
\begin{block}{Descripción}
\centering
Identificar y valorar los aspectos pedagógicos, de diseño, de implementación y
de evaluación que influyen a la creación de herramientas educativas que utilizan
las corrientes pedagógicas actuales apoyadas en las TIC, especialmente los
juegos serios.
\end{block}


\end{frame}

\begin{frame}
\frametitle{\pagetitle}
\framesubtitle{Objetivos específicos}

\small
\begin{itemize}[<+->]
\item Proveer una visión actualizada de las corrientes pedagógicas apoyadas con las TIC.
\item Proveer una visión actualizada de los juegos serios.
\item Identificar áreas de aplicación de los juegos serios.
\item Seleccionar las herramientas tecnológicas disponibles para el desarrollo de juegos serios.
\item Contrastar en la práctica los conocimientos teóricos adquiridos a través del diseño
e implementación de un juego serio.
\item Evaluar la solución propuesta para identificar los
aspectos de diseño, desarrollo y evaluación de los juegos serios.
\end{itemize}
\end{frame}


\section{TIC's en la educación}

\begin{frame}{Instruccionismo}
\end{frame}
\begin{frame}{Conductismo}
\end{frame}
\begin{frame}{Constructivismo}
\end{frame}
\begin{frame}{Construccionismo}
\end{frame}
\begin{frame}{Ventajas}

    \begin{itemize}[<+->]
        \item Nuevos modelos pedagógicos
        \item Eliminación de distancias
        \item Colaboración distribuida
        \item Motivación para aprender
        \item Adquisición de habilidades básicas
    \end{itemize}
\end{frame}
\begin{frame}{Desafíos}
\end{frame}

\section{Juegos serios}
1 página

Un juego serio es un videojuego que posee un propósito educacional explícito y
bien elaborado, y cuya intención no es la de únicamente entretener al
usuario\cite{abt1987serious,sg:aoverview,damien:sg}.

Los \emph{Juegos Serios} proveen una oportunidad muy importante para ayudar en
la enseñanza y desarrollo de profesionales\cite{mariluz:seiousgames}, por que
ayudan a crear el tipo de educación que los adultos prefieren, proveen
mecanismos para que los estudiantes cometan errores y experimenten con sus
ideas, con su conocimiento y con la teoría en un ambiente protegido sin riesgos
para la vida o la identidad\cite{sg:aoverview}. 

El campo de los \emph{Juegos Serios} rechaza la idea de que los profesionales de
la educación pueden ser reemplazados, la labor de estos profesionales es
imprescindible para la reflexión y orientación del
aprendizaje\cite{elearning:seiousgames}.

\subsection{Ventajas y desafíos}

Los juegos serios tienen ventajas y desafíos que los hacen únicos en su enfoque
para con la utilización de las TIC's y posee las siguientes ventajas:


\begin{itemize}

\item \textbf{Motivación interna}: favorecen la autoestima y tienen un factor
    motivacional\cite{guenaga2013serious}, permitiendo una implicación mayor del
    usuario en la actividad\cite{sg:aoverview}. La implicación del usuario
    dentro de la actividad, es un tema central en el desarrollo de los juegos
    serios\cite{charsky:2010}.

\item \textbf{Apoyo al aprendizaje}: ayudan al aprendizaje es por que los mismos
    se desarrollan en un entorno significativo y relevante al
    contexto\cite{sg:aoverview}. Las teorías modernas de aprendizaje sugieren
    que el aprendizaje es más efectivo cuando es activo, experiencial y basado
    en problemas\cite{guenaga2013serious}.

\item \textbf{Menos limitaciones}: juegos serios permite a sus usuarios
    experimentar en entornos y sistemas que no son posibles en la vida real, por
    cuestiones de costo, tiempo y aspectos relacionados a la
    seguridad\cite{sg:aoverview}.

\item \textbf{Similitud a la realidad}: constituyen un escenario privilegiado
    para el desarrollo de todos los componentes de las competencias, ya que
    permiten desarrollar vivencias en las que ponerlos en practica, permitiendo
    el entrenamiento en situaciones que en muchas ocasiones son similares a las
    que se encuentran en entornos reales\cite{guenaga2013serious,sg:aoverview}.
    
\item \textbf{Estimulación sensorial}: aumentan la capacidad de coordinación,
    percepción espacial y ampliación del campo visual, lo que tiene una
    incidencia en la lectura y el manejo eficiente en ambientes
    3D\cite{guenaga2013serious}. 

\end{itemize}

El potencial de un juego serio no es ilimitado, existen múltiples desafíos que
deben ser superados para poder desarrollar un juego serio que obtenga las ventajas
citadas previamente y pueda ser de utilidad en la educación formal.


\begin{itemize}

\item \textbf{Falta de investigación}: aunque en los últimos años los estudios
    del impacto de los juegos serios han aumentado considerablemente, son
    necesarios más estudios para probar su eficiencia\cite{sg:aoverview}.

\item \textbf{Expectativas muy altas}: un juego serio, como el resto de la
    \textit{media}, no puede cambiar el comportamiento de una persona por sí
    solo, pero sí permiten al jugador explorar las opciones, tener en cuenta las
    consecuencias de sus actos y poner en práctica sus
    conocimientos\cite{education:games,stapleton2004serious,videojuegos:gonzaleztardon}. 

\item \textbf{Evaluación tradicional}: la forma tradicional de evaluación presenta
    dificultades a los juegos serios, por ejemplo, las pruebas tradicionales
    contienen un grupo de preguntas, las cuales son vistas de manera
    independiente, en cambio en un juego serio, las acciones son dependientes
    del contexto y las acciones previamente realizadas\cite{shute2009melding}.

\item \textbf{Utilización incorrecta}: en cuanto al objeto pedagógico, el área
    en la cual se utiliza un juego serio es un factor determinante para el éxito
    del mismo\cite{stapleton2004serious}.

\item \textbf{Falta de recursos}: uno de los factores más complicados a la hora
    del desarrollo de juegos serios es la limitación de recursos financieros,
    esto no quiere decir que no existan recursos para su desarrollo, sino que,
    comparados con los recursos invertidos en otras \textit{media}, el
    presupuesto es insignificante\cite{stapleton2004serious,sg:aoverview}. Como
    consecuencia de las limitaciones financieras, los desarrolladores no siempre
    pueden acceder a tecnología de última generación\cite{stapleton2004serious}

\end{itemize}

\subsection{Actualidad}

Esta sección muestra las áreas donde más frecuentemente se utilizan los juegos
serios.

\begin{itemize}

\item \textbf{Militar}: durante más de $30$ años los videojuegos han sido
    reconocidos como herramientas factibles en el entrenamiento de militares. En
    $1996$ fue lanzado un videojuego llamado \emph{Marine Doom} en donde la
    tarea de los jugadores era el aprendizaje de formas de ataque, conservación
    de municiones, comunicarse con eficacia, dar órdenes al equipo de trabajo
    entre otros. De esta manera tuvo lugar una forma de entrenamiento más
    atractivo, sin el costo, dificultad, riesgos e inconvenientes que implicaría
    el mismo entrenamiento en un entorno real. Además se podían crear
    situaciones que en el mundo real serían muy difíciles de replicar y donde
    los errores pueden ser catastróficos\cite{education:games}.

\item \textbf{Salud}: los juegos de salud se utilizan para la formación de
    profesionales basada en la simulación. En $2008$ el Centro de Simulación
    \emph{Hollier} en \emph{Birmingham}, Reino Unido, realizó una prueba que
    permitió a médicos jóvenes experimentar y entrenar para diversos escenarios
    médicos a través de maniquíes virtuales como pacientes, de este modo el
    aprendizaje se da por la experiencia. En su disertación, \emph{Roger D.
        Smith}, realizó una comparación entre la enseñanza tradicional y la
    formación mediante realidad virtual y el uso de herramientas basadas en la
    tecnología de videojuegos en cuanto a la cirugía laparoscópica. Como
    conclusión afirmó que lo último era más barato, requería menos tiempo y que
    permitió menos errores médicos cuando los médicos se presentaban en una
    cirugía real debido a, entre otras cosas, la posibilidad de repetición de la
    experiencia sin riesgo alguno\cite{education:games}. 

\item \textbf{Juegos corporativos}: Este tipo de videojuegos se han utilizado
    para la selección de personal, la mejora de comunicación entre los
    directivos y su personal de confianza, y la formación de nuevos empleados.
    Los juegos serios pueden ser utilizados incluso para elaborar planes de
    negocios\cite{education:games}. 

\end{itemize}


\section{Definición del problema}

%1/2 página
%\begin{itemize}
%\item Estado actual
%\item Prácticas en laboratorio
%\item Problemas actuales
%\item Propuesta de solución
%\end{itemize}

Según una investigación los alumnos de enfermería, son estudiantes divergentes,
es decir aprenden a través de experimentación activa, e interiorizan el
conocimiento reflexionando sobre la experiencia. La simulación es una
herramienta ideal para este tipo de estudiantes\cite{humphreys2013developing}.

Es por ello que este trabajo de grado se centra en las problemáticas referentes
al área de enfermería proponiendo una solución tecnológica. Se considera como
campo de estudio al \gls{iab}, cuyos estudiantes del último año de la carrera de
Licenciatura en Enfermería son tomados como población objetivo.

A continuación se describe el estado actual de las prácticas realizadas por los
alumnos como parte de plan curricular, además de los problemas actuales y una
propuesta de solución a esos problemas.

\subsection{Prácticas en laboratorio}

El \gls{iab} cuenta con un laboratorio de prácticas especializado para los
estudiantes de enfermería. Es utilizado para desarrollar las materias prácticas
de manera a tener una formación previa a las prácticas de campo, es decir,
prácticas en hospitales con pacientes reales.

El número de alumnos dificulta la enseñanza individual por lo que las clases de
laboratorio se dividen en dos, la primera parte de desarrolla en un aula
convencional en donde el profesor enseñanza a los estudiantes usando a uno de
ellos como voluntarios o usando modelos de partes del cuerpo para realizar
explicaciones generales y responder dudas. La segunda parte se lleva a cabo en
un laboratorio que cuenta con herramientas como maniquíes, camas, utensilios,
entre otros. En esta segunda parte los alumnos pueden explorar y experimentar
por sí mismos bajo la tutela del profesor.

\subsection{Prácticas de campo}

Las prácticas de campo son aquellas prácticas profesionales que son realizadas
por los alumnos con pacientes humanos y en hospitales, bajo supervisión de un
profesional denominado instructor y bajo una continua evaluación de sus
acciones, las mismas son llevadas a cabo una vez que los alumnos finalizan las
prácticas de laboratorio.

Cada instructor posee un planilla por alumno donde se realiza el seguimiento de
sus actividades. La creación de esta planilla de actividades es responsabilidad
del instructor, el instructor debe basarse en las competencias básicas de la
asignatura y la misma es validada por la dirección de la carrera, y se considera
que un alumno ha adquirido la pericia\footnote{Sabiduría, práctica, experiencia
    y habilidad en una ciencia o arte.} necesaria para una asignatura solo si
pudo completar la planilla del instructor. Si el alumno no aprueba todas las
prácticas no tiene derecho a rendir el examen teórico de la materia y debe
volver a cursar la materia.

\subsection{Problemas actuales}

Si bien el nivel de actual de los egresados de la carrera de Licenciatura en
enfermería del \gls{iab} es considerando satisfactorio por las autoridades de
esta casa de estudios, existen inconvenientes según apreciaciones de profesores
y alumnos, además de algunas tesis de grado de la institución. A continuación se
citan estos inconvenientes.

Algunos profesores manifestaron en reuniones que antes de ir a las prácticas de
campo con los alumnos prefieren dar las primeras clases en laboratorio por lo
siguiente:

\begin{itemize}
\item Falta de preparación de los alumnos debido a que en ocasiones 
ciertos detalles no son cubiertos completamente por las prácticas en 
laboratorio.
\item Definición de un protocolo de comunicación entre alumnos y el 
profesor para su uso en las prácticas de campo.
\item Nerviosismo ante la primera práctica por parte de los alumnos.
\end{itemize}

En tanto, los mayores inconvenientes detectados por los alumnos son:
\begin{itemize}
\item Alta carga horario de los trabajos prácticos.
\item Reducida carga horaria para el estudio de las materias teóricas 
debido a las prácticas.
\item Poca flexibilidad de los profesores.
\item Falta de materiales actualizados para los profesores.
\item Problemas de transporte para llegar a hasta los hospitales o al 
    \gls{iab}.
\item Falta de preparación para las prácticas.
\item Alta cantidad de alumnos, debido a esto las prácticas de campo 
rara vez se realizan en un sólo hospital y son realizadas por grupo de
alumnos.
\end{itemize}

\subsection{Propuesta de solución}

Teniendo en cuenta que los mismos forman parte de un grupo de profesionales que requieren de un alto grado de prácticas y que en la actualidad se 
presentan varios factores que a un estudiante de esta carrera le impide poner 
a prueba todo el tiempo sus conocimientos y por todo lo expuesto anteriormente, en este trabajo se propone el desarrollo de un juego serio, que involucra la simulación de laboratorios virtuales, como una herramienta para 
el proceso de aprendizaje de los alumnos de la carrera de enfermería.

Se considera que los principales problemas que pueden ser abordados con esta 
solución son los siguientes: evaluación de los alumnos, seguimiento del 
progreso del alumno, tiempo de práctica, ubicuidad, realismo, enfoque 
individual. Ofreciendo una herramienta que no puede sustituir a un práctica 
de campo pero si puede servir de apoyo en el aprendizaje relacionado a las 
prácticas.

%! TEX root = ../main.tex

\section{Solución}
\label{sec:solucion}

\observacion{Ver donde pone la interacción con la cámara}

Se describe la arquitectura propuesta para la realización de una juego serio, se
utiliza la guía básica definida por~\cite{pereira2009design} y descrita
en~\ref{sec:desarrollo}.

Esta sección se enfoca en los aspectos técnicos de la creación del juego serio,
las competencias básicas relacionadas con la educación (segundo paso de la guía
descrita en~\ref{sec:desarrollo}) se define en las
secciones~\ref{sec:glasgow} y~\ref{sec:hemocultivo}.

\subsection{Partes de la simulación}

La simulación se compone de tres elementos principales, entidades (que son
objetos de la vida real), acciones (que son provocadas por las entidades) y
eventos (que son el resultado de una acción). 

Existen otros elementos dentro de la simulación, como la sala y la iluminación,
los mismos son importantes para crear un entorno similar a la realidad y son
estáticos, es decir no interactúan con el usuario más que para limitar la
exploración en el escenario y/o resaltar aspectos relevantes.

\subsubsection{Entidades}

Cualquier objeto o componente en el sistema que requiera la representación
explícita en el modelo\cite{banks2000dm}. Las entidades tienen atributos. Los
atributos son las características de una determinada entidad que son exclusivos
de esa entidad.

Una entidad tiene en todo momento, un estado y una lista de acciones que
puede realizar, esta lista de acciones esta definida por el estado del mismo,
las condiciones en la que se encuentra el entorno y la práctica actual.

La entidad \enquote{Enfermero} es la que es controlada por el usuario, a través
de la interacción con la interfaz gráfica.

\subsubsection{Acciones}

Las entidades se comunican a través de acciones, las cuales pueden tener
diversos orígenes, siempre una entidad inicia una acción. Las acciones provocan
cambios en el ambiente y provocan eventos. Las acciones no solo las
realiza el usuario, sino cualquier entidad.

Como ejemplo, una acción es esterilizar las manos, esta acción provoca un
cambio en el ambiente (las manos ahora son estériles) y fue realizada por la
interacción entre el usuario y la interfaz gráfica.

\subsubsection{Eventos}

Los eventos son ocurrencias instantáneas que cambia el estado de un
sistema\cite{banks2000dm}, cada acción que se realiza provoca una acción, y los
eventos son la mecanismo que tiene una entidad para ser notificada de las
acciones de otras entidades.

\subsubsection{Interacción con el entorno}

El usuario se desenvuelve en un entorno de tres dimensiones, en el cual realiza las
actividades relacionadas a la práctica, se distinguen dos tipos de movimientos
principales que el usuario puede realizar:

\begin{itemize}
    \item \textbf{Alejamiento o acercamiento}: es el acto de acercar o alejar la
        cámara, y por consiguiente al usuario del paciente. Se realiza
        utilizando dos dedos, para realizar un acercamiento, mientras se
        mantiene presionada la pantalla con ambos dedos, se procede a alejar un
        dedo del otro, para realizar un alejamiento, se debe acercar ambos
        dedos.
    \item \textbf{Rotación}: se refiere al movimiento de rotación al rededor de
        un foco, que en ambas escenas es el paciente, para realizara, se utiliza
        un dedo, y se mueve en dedo en cualquier dirección, la cámara, se moverá
        en la dirección contraria.
\end{itemize}

\subsection{Grafo de estados}

La solución tiene varias escenarios, y dentro de cada escenario, existen varias
pantallas que muestran información relevante de acuerdo a la situación de la
simulación, en~\ref{fig:grafo_estados} se observa la interacción entre las
diferentes pantallas y escenarios.

\begin{figure}[H] 
\centering 
\includegraphics[scale=0.5]{propuesta/grafo_escenas.png}
\caption{Navegación entre escenarios y pantallas. Los escenarios son los
    rectángulos con un borde dos rayas, y las pantallas tienen un borde con una
    sola raya.}
\label{fig:grafo_estados}
\end{figure}

La solución inicia con un escenario denominado \emph{Inicio}, en el cual se
permite al usuario observar los detalles del entorno simulado a la vez que
muestra las opciones que permiten iniciar las diferentes prácticas, compartir
su actividad, enviar los datos de utilización y finalmente salir de la
simulación.

Si el usuario selecciona en el \emph{inicio} la opción \emph{Extracción de
    sangre}, se inicia el escenario denominado \emph{Extracción de sangre}, en
el cual el usuario puede realizar el procedimiento de extracción de sangre, si
el usuario selecciona la opción \emph{Fin}, la simulación termina y se dirige a
el escenario \emph{Pantalla de resultados}.

Al seleccionar la opción \emph{Evaluación Glasgow}, se inicia el escenario
denominado \emph{Glasgow}, donde el usuario debe evaluar a un paciente en el
centro del escenario, si el usuario presiona la opción \emph{Fin} se inicia la
pantalla denominada \emph{Evaluar al paciente}, donde el usuario diagnostica el
estado del paciente, y finalmente al presionar el botón \emph{Fin}, la
simulación finaliza y se inicia el escenario \emph{Pantalla de resultados}.

La opción \emph{Exploración Glasgow} es similar, la diferencia es que antes de
iniciar el escenario \emph{Glasgow}, aparece la pantalla \emph{Elegir estado de
    paciente}, en el cual el usuario selecciona un estado para que el paciente
actué de acuerdo al mismo, luego se inicia la escena \emph{Glasgow} y si el
usuario presiona el botón \emph{Fin}, se inicia el escenario \emph{Pantalla de
    resultados}.

La pantalla de resultados muestra la información acerca de las acciones que
realizo el usuario, proveyendo información a modo de retroalimentación, en esta
pantalla el usuario puede compartir sus resultados por las redes sociales,
reiniciar el escenario y finalmente, poder volver a la \emph{Pantalla de
    inicio}.


\subsection{Inicio}

\subsubsection{Descripción del entorno}

La escena mostrada como pantalla de inicio de la aplicación muestra como fondo la sala de 
un hospital con los elementos típicos de estos lugares, esta es la que se utiliza como 
escenografía principal en los escenas de los procedimientos, haciendo que el usuario entre 
en ambiente. Además de este fondo, se muestras varias opciones en forma de botones que serán 
descriptas a continuación y un mensaje en donde se recomienda al usuario el uso de auriculares.

\subsubsection{Opciones}

Las opciones disponibles en la pantalla de inicio son presentadas en forma de
botones los cuales tienen una breve descripción que identifica la función que
cumplen. 

\todox{Agregar descripción del escenario y si es necesario pantalla donde se
    pone el número}

\begin{itemize}
\item Botón \enquote{Enviar Progreso}: esta función envía toda la información
    acerca de la actividad que el usuario realizo en la aplicación a un servidor
    backend que se encarga de almacenar estos datos.
\item Botón \enquote{Salir de la simulación}: esta función permite salir de la
    aplicación.
\item Botón \enquote{Facebook}: esta función permite al usuario ingresar a su
    cuenta de Facebook.
\item Botón \enquote{Extracción de sangre}: esta función permite ingresar a la
    escena correspondiente al procedimiento de extracción de muestras de sangre
    permitiendo al usuario jugar una nueva partida.
\item Botón \enquote{Explorar Glasgow}: esta función permite ingresar a la
    escena correspondiente al procedimiento para explorar las reacción de un
    paciente con un diagnostico especifico de la escala de Glasgow permitiendo
    al usuario jugar una nueva partida.
\item Botón \enquote{Evaluar Glasgow}: esta función permite ingresar a la escena
    correspondiente al procedimiento para la valoración y diagnostico de la
    escala de Glasgow para un paciente con estado aleatorio permitiendo al
    usuario jugar una nueva partida.
\end{itemize}


\subsection{Extracción de muestras de sangre}

A continuación se detallan cada una de las opciones y formas disponibles de
interactuar con la escena del procedimiento de extracción de muestras de sangre.

\subsubsection{Descripción del entorno}

Al seleccionar el procedimiento de extracción de sangre en la pantalla de inicio 
la aplicación inmediatamente muestra la escena del procedimiento, se muestra una 
sala de hospital igual a la de la pantalla de inicio pero con un paciente en una 
de las camas, a este paciente es a quien se le realizara el procedimiento.

La posición inicial de la cámara se ubica en un ángulo en donde se puedan ver 
bien los brazos del paciente para facilitar al usuario la realización del 
procedimiento.

\todox{Agregar descripción}

\subsubsection{Descripción de la interfaz}

La interfaz principal de este escenario posee dos menús, uno a cada lado de la
pantalla, las opciones son representadas como botones que poseen una imagen
intuitiva\todox{Ver si no hay que agregar esto como hipótesis} que representa la
función que realizan. 

\subsubsection{Entidades}

En la extracción de sangre existen dos entidades principales, el paciente y el
usuario, cada entidad mantiene un estado independiente de la otra entidad.

El paciente es una entidad con estado complejo, el cual es constantemente
modificado por las acciones del usuario, en resumen, la información que contiene
el estado del paciente es:

\begin{itemize}
    \item \textbf{Jeringas}: un paciente puede tener cero o más jeringas en
        cualquier momento, no se limita la cantidad de jeringas que puede
        insertar el usuario.
    \item \textbf{Manos}: almacena el estado de las manos, el paciente reacciona
        ante peticiones del usuario, puede abrir o cerrar cualquier mano en
        cualquier momento.
    \item \textbf{Torniquetes}: es el conjunto de torniquetes que tiene
        actualmente el paciente, notar que los torniquetes pueden ser colocados
        en cualquier parte del brazo, pero existen lugares \enquote{correctos} y
        lugares \enquote{incorrectos}, la diferencia consiste en la distancia a
        los puntos de extracción, estos lugares están predefinidos.
    \item \textbf{Zonas esterilizadas}: son aquellas áreas del cuerpo que el
        usuario esterilizó, no existe un límite para las zonas esterilizadas.
        Una vez que una jeringa es extraída, una zona esterilizada pasa a estar
        contaminada y a la espera de que el usuario la presione.
    \item \textbf{Zonas presionadas}: son aquellas zonas que, una vez
        contaminadas por la extracción de una jeringa, han sido presionadas por
        el usuario.
    \item \textbf{Contaminado}: define si alguna acción realizada por el usuario
        provoco que el paciente se contamine, existen varias cadenas de eventos
        que pueden provocar que esto ocurra:
        \begin{itemize}
            \item Inyección de una jeringa cuando existe otra inyectada.
            \item Inyección en un lugar en lugares inadecuados.
            \item Inyección en un lugar no esterilizado.
            \item Inyección en un brazo cuya mano este abierta.
            \item Inyección fuera del alcance de los torniquetes actuales.
            \item Interacción con el paciente sin que el mismo tenga la mano
                estéril.
        \end{itemize}
        Es importante notar que este estado no es afectado directamente por una
        acción del usuario, sino por la consecuencia de una acción.
\end{itemize}

El \emph{usuario o enfermero} mantiene un estado en todo momento del cual
dependen sus acciones, por ejemplo, si la mano del paciente no esta
esterilizada, cualquier interacción con el paciente provocara que el paciente se
contamine.

\begin{itemize}
    \item \textbf{Manos}: almacena la información acerca de la esterilidad de
        las manos.
    \item \textbf{Guantes, gorro, bata y tapaboca}: almacenan la información
        acerca de los equipamientos que tiene el usuario en un momento dado.
    \item \textbf{Elemento actual}: es el elemento que esta activo en
        cualquier momento, un elemento es una herramienta de la vida real,
        como por ejemplo un torniquete, una gaza.
\end{itemize}

\subsubsection{Acciones}


\paragraph{Comando de voz}

Para representar la interacción del usuario con el paciente usando la voz se
implemento un menú que es activado y mostrado en pantalla cuando el usuario
habla, este menú muestra una seria de ordenes que el usuario le haría al
paciente normalmente hablándole. Las opciones de menú se detalla a continuación:

\begin{itemize}
\item Explicar procedimiento: esta función sirve para detectar si el usuario
    realizo la acción de explicar al paciente acerca del procedimiento. 
\item Abrir la mano izquierda: esta función le indica al paciente que abra su
    mano izquierda, como resultado el paciente realiza esta acción.
\item Cerrar la mano izquierda: esta función le indica al paciente que cierre su
    mano izquierda, como resultado el paciente realiza esta acción.
\item Abrir la mano derecha: esta función le indica al paciente que abra su mano
    derecha, como resultado el paciente realiza esta acción.
\item Cerrar la mano derecha: esta función le indica al paciente que cierre su
    mano derecha, como resultado el paciente realiza esta acción.
\end{itemize}

\paragraph{Opciones}

En este menú se despliegan los botones que representan las opciones de
bioseguridad. Es decir, acciones como lavarse las manos, calzarse guantes,
ponerse gorro, ponerse bata y ponerse tapaboca.

Los elementos de bioseguridad que actualmente tiene puesto el usuario se
representan como se describió anteriormente y se muestran en la parte baja de la
pantalla. Desaparece quitarse que en ese caso se representa al volver a
seleccionar la misma opción.

\paragraph{Elementos}

En este menú se despliegan los botones que representan a lo elementos que se
utilizan para realizar el procedimiento, una vez presionado ese elemento queda
seleccionado. Solo un elemento puede ser seleccionado a la vez. Si el mismo
botón se vuelve a presionar inmediatamente después de haber sido presionado, el
elemento queda de-seleccionado.

%Estas opciones van cambiando el estado del jugador y pueden ser seleccionados
%mas de una opción a la vez además de permitir de-seleccionar una opción
%volviendo a tocar el botón correspondiente. También posee la opción de
%finalizar la partida la cual manda al usuario a la pantalla de resultados.
%% REVISAR ESTO , el comienzo es sobre opciones y el final sobre elementos %%

La herramienta seleccionada actualmente para realizar el procedimiento se se
muestra en la forma descripta anteriormente arriba de la pantalla principal del
procedimiento. Esta imagen representa lo que actualmente tiene en las manos el
jugador. Desaparece al de-seleccionar o terminar de usar la herramienta.

\todox{Agregar colocación}
\todox{Agregar utilización}

\subsubsection{Eventos}
\subsubsection{Motor de reglas}
\subsubsection{Registro de actividad}


\subsection{Valoración de la escala de Glasgow}

\subsubsection{Descripción del escenario}

La interfaz principal de este escenario posee un botón de finalización de
partida al costado con una imagen intuitiva que representa la función que
realiza. Este botón manda al usuario a la pantalla de resultados.

\subsubsection{Entidades}
\subsubsection{Acciones} 
\subsubsection{Eventos} 
\subsubsection{Pantalla de diagnostico}
\subsubsection{Registro de actividad}
\paragraph{Elementos y opciones}


\subsection{Pantalla de resultados}
\subsubsection{Descripción del escenario}
\subsubsection{Retroalimentacion}
\subsubsection{Gamificacion}
\subsubsection{Reinicio}
\subsubsection{Puntuación}
\subsubsection{Tiempo utilizado}
\subsubsection{Facebook 2}

\subsection{Partes de la simulación}
    \subsubsection{Entidades}
    \subsubsection{Eventos}
    \subsubsection{Acciones}
    \subsubsection{Interacción con la cámara}

\subsection{Grafo del desarrollo}
% podemos poner acá un gráfico mas o menos así (ver graphviz)
%           /---> Hemocultivo --\
%          /                     \              /-> Reiniciar
%  Inicio ------> Glasgow 1 -------> Resultados --> Inicio
%         \ \                    /              \-> Facebook 2
%          \ \--> Glasgow 2 ----/
%           \---> Salir 
%            \--> Facebook 1
%             \-> Enviar resultados

\subsection{Pantalla de inicio}
    \subsubsection{Descripción del escenario}
    \subsubsection{Enviar datos}
    \subsubsection{Glasgow}
    \subsubsection{Extracción de sangre}
    \subsubsection{Facebook 1}

\subsection{Extracción de sangre}
    \subsubsection{Descripción del escenario}
    \subsubsection{Descripción de la interfaz}
    \subsubsection{Entidades} % definimos cuales son las entidades
        %\subsubsubsection{Estado del enfermero}
        %\subsubsubsection{Objeto seleccionado}
    \subsubsection{Acciones} % definimos cuales son las acciones de esas entidades
        %\subsubsubsection{Comandos de voz}
        %\subsubsubsection{Opciones} %bata,mano,guante,y eso
        %\subsubsubsection{Elementos}
            %\subsubsubsubsection{Colocación}
            %\subsubsubsubsection{Utilización}
    \subsubsection{Eventos} % definimos cuales son los eventos que se lanzan en este proceso
    \subsubsection{Motor de reglas} % se define como funciona el motor de reglas acá
    \subsubsection{Registro de actividad} % se define como se registra las acciones del usuario (cuales)


\subsection{Glasgow 1 y 2}
    \subsubsection{Descripción del escenario}
    \subsubsection{Entidades} % definimos cuales son las entidades
        %\subsubsubsection{Reacciones del paciente}
    \subsubsection{Acciones} % definimos cuales son las acciones de esas entidades
        %\subsubsubsection{Acciones sobre el paciente}
        %\subsubsubsection{Comandos de voz}
    \subsubsection{Eventos} % definimos cuales son los eventos que se lanzan en este proceso
    \subsubsection{Pantalla de diagnostico}
    \subsubsection{Registro de actividad} % se define como se registra las acciones del usuario (cuales)
    
\subsection{Pantalla de resultados}
    \subsubsection{Descripción del escenario}
    \subsubsection{Retroalimentacion}
    \subsubsection{Gamificacion}
    \subsubsection{Reinicio}
    \subsubsection{Puntuación}
    \subsubsection{Tiempo utilizado}
    \subsubsection{Facebook 2}

%! TEX root = ../main.tex
\section{Evaluación en tiempo de ejecución}

Las acciones realizadas por los usuarios dentro de la aplicación son evaluadas
para determinar si realizo o no el procedimiento de manera correcta y así
brindarle información al usuario sobre su rendimiento.

En esta sección se explica como son evaluados las acciones de los usuarios para
los diferentes procedimientos simulados.

\subsection{Extracción de muestras de sangre}

Para la evaluación de las acciones del usuario en este procedimiento se utilizo
un motor de reglas denominado \enquote{Acciones condicionadas por eventos}. A
continuación se explica en detalle cada aspecto relacionado tanto al motor como
a la forma de evaluación del rendimiento del usuario.

\subsubsection{Acciones condicionadas por eventos}

Un evento es la ocurrencia de un hecho en particular, y son identificados por un
nombre y un conjunto de parámetros, por ejemplo, cuando un evento es cuando el
enfermero inserta una Jeringa, el nombre de este evento es
\enquote{jeringa}.inserted, y sus parámetros podrían ser el lugar y el tiempo
de la inserción, así, la influencia del estudiante en la simulación es una
sucesión de eventos.

Por cada acción que realiza el usuario dentro de la simulación, existe un evento
relacionado, por consiguiente, es razonable estudiar algunos eventos para
determinar si los pasos realizados corresponden con los deseados. 

Para determinar si una sucesión de eventos es la correcta, se definen reglas,
una regla es una asociación de una condición y una acción, la condición define
si el entorno es el adecuado para realizar una acción, la cual es un
procedimiento que realiza la lógica deseada.

Las \gls{eca} son aquellas que son activadas una vez que se cumplen determinados
eventos\cite{bailey2004event}. En las bases de datos relacionales, son conocidos
como triggers, es decir, una base de datos relacional (u orientada a objetos) es
un motor de reglas \gls{eca}\cite{bailey2004event}\cite{behrends2006combining}.

Las mismas pueden ser utilizadas para notificar que un determinado conjunto de
eventos ha ocurrido\cite{bailey2004event}, así como servir para almacenar
información acerca de la utilización de un determinado recurso.


\paragraph{Motivación}

Las reglas del tipo \gls{eca} permiten reaccionar a determinados eventos, en
forma de una única regla, la cual facilita la declaración de las
mismas\cite{bailey2004event}.

Son principalmente útiles para analizar el comportamiento en tiempo real de un
sistema en una forma
reactiva\cite{bailey2004event}\cite{de2001eca}\cite{bailey2002analysis}, esta
característica esta impulsada principalmente por que son ejecutadas después de
la ocurrencia de un evento, y el entorno no es modificado, pudiendo así acceder
al mismo entorno que el qué lanzo el evento.

Definir si las acciones de un usuario son correctas utilizando un motor
\gls{eca} es sencillo desde el punto de vista que sólo se deben definir un
conjunto de acciones que se deben realizar, y agregar una acción que verifica si
los pasos realizados fueron los correctos.

\paragraph{Declaración}

Una \gls{eca}, se define como\cite{bailey2004event}\cite{behrends2006combining}:

\begin{center}
	 Cuando ocurren una serie de \emph{eventos}, y se cumple una
	 \emph{condición}, entonces realizar una \emph{Acción}.
\end{center}

Los \emph{eventos} determinan cuando una regla debe ser activada, los mismos se
dividen en dos categorías\cite{behrends2006combining}, primitivos y compuestos,
los primeros son detectables, por ejemplo, cuando se inserta una jeringa, y los
compuestos, son la combinación de uno o más
primitivos\cite{bailey2004event}\cite{behrends2006combining}. Los eventos
compuestos, se unen mediante:
\begin{enumerate*}[label=\itshape\alph*\upshape)]
\item conjunción (\emph{y}),
\item disyunción (\emph{o}), y
\item secuencia (\emph{entonces}).
\end{enumerate*}
Sin embargo, no siempre son necesarios todas las posibles combinaciones, y las
combinaciones sencillas son más fáciles de optimizar y
probar\cite{bailey2004event}.

La \emph{condición} de una regla determina si el entorno es el necesario para que la
regla sea activada, en esta condición el entorno que lanzó el evento esta
disponible.

La \emph{acción} a ejecutar describe la lógica que debe ser ejecutada cuando se han
lanzado los eventos y la condición de la regla se ha cumplido.

\paragraph{Dependencia entre reglas}

Las reglas pueden depender de otras reglas, lo cual se puede ver como que la
finalización de una regla es un evento que otra regla espera para poder ser
activada.

Las reglas pueden agregar información a un contexto compartido por todas las
reglas, de esta manera, se puede pasar parámetros entre distintas reglas, por
ejemplo, la regla \emph{Retirar Torniquete}, depende de la regla \emph{Insertar
Torniquete}, pero debe responder solamente al torniquete que ha activado
la regla de inserción, es decir, el usuario puede extraer varios torniquetes, y
la regla no debe activarse, hasta que se extraiga el torniquete que activo la
primer regla.

Así, la regla \emph{Retirar Torniquete} depende de la regla \emph{Insertar
Torniquete}, y esta relación entre reglas, se da en dos
formas\cite{bailey2004event}:

\begin{itemize}
\item  \emph{Dependencia fuerte:} la regla \emph{Retirar Torniquete} solamente podrá
	ser elegida para ser lanzada cuando la regla \emph{Insertar Torniquete}
	haya sido cumplida.
\item  \emph{Dependencia de contexto}: la regla \emph{Retirar Torniquete} no se
	activará cuando los eventos a los que escucha se terminen, sino cuando
	los eventos a los que escucha sean lanzados con los parámetros adecuados
	(se extraiga el torniquete que lanzo la regla de inserción).
\end{itemize}

\paragraph{Representación}

La definición de las reglas se realiza de la siguiente forma;
\begin{algorithm}[H]
\caption{Creación de regla de verificación de calzado de guantes}
\label{alg:rule:guante}
\lstset{style=sharpc}
\begin{lstlisting}
Rule.New("Regla de verificacion de calzado de guantes").
     When("enfermero.guantes.calzar").
     Then(e => e.Patient.ManosLimpias()).
\end{lstlisting}
\end{algorithm}
%TODO agregar indice de algoritmos

La regla anterior controla que el estudiante ha realizado la acción ``Calzarse
los guantes'', y en ese momento tenga las manos limpias, la variable \emph{e},
es el entorno, y a través de la propiedad \emph{Patient} obtiene el estado del
paciente en ese momento.

\paragraph{Modelo de ejecución}

Para ejecutar un motor de reglas del tipo \gls{eca}, se debe tener en cuenta
principalmente dos factores, 
\begin{enumerate*}[label=\itshape\alph*\upshape)]
\item  Como se verifica el cumplimiento de cada regla, y, 
\item  Que ocurre cuando varias reglas son lanzadas al mismo tiempo
\end{enumerate*}.

Para ambos casos se puede tomar un enfoque \emph{inmediato}, es decir que
inmediatamente cuando se lanza un evento, o se cumple una condición, se ejecuta
la regla. Además existen otros dos modos de ejecución, \emph{deferida}, y
\emph{desacoplada}, en la primera, se espera hasta que el lanzador del evento
culmine su trabajo, y luego se ejecuta la regla, pero en la misma unidad de
trabajo, mientras que en la ejecución desacoplada, se encolan los trabajos y
otro hilo es el encargado de ejecutar las reglas. Estos modos están inspirados
en las bases de datos relacionales, el deferido se ejecuta en la misma
transacción, y el desacoplado, inmediatamente después de que la transacción
termine\cite{bailey2004event}.

La propuesta implementada, utiliza una ejecución inmediata, principalmente por
la sencillez de las reglas, es decir, las reglas no realizar un proceso complejo,
solamente controlan el estado del entorno y lo validan.

Además, la ejecución inmediata es importante por que el entorno no sufre
modificaciones entre el evento lanzado y la ejecución de la regla, según
\cite{bailey2004event}, este es el factor más importante para determinar el tipo
de ejecución deseado.



\paragraph{Estados de una regla}

Una regla puede estar en uno de los siguientes estados:

\begin{description}
\item[BEGIN] Es una regla que recién fue creada, no realiza ninguna
	acción.
\item[WAITING\_FOR\_RULE] Es un estado en el que esta esperando que otras reglas
	sean lanzadas. En este estado, es un suscriptor de las reglas por la que
	espera, y no forma parte del ciclo de ejecución del motor de reglas.
\item[WAITING\_FOR\_EVENT] Es un estado en el que esta escuchando a que sean
	lanzados los eventos a los que escucha, este es el estado principal. En
	este estado, es un suscriptor de los eventos por los que espera, y no
	forma parte del ciclo de ejecución del motor de reglas. Se diferencia
	del estado anterior, en que los eventos escuchados pueden ser lanzados
	por cualquier objeto del entorno, no necesariamente una regla.
\item[WAITING\_FOR\_CONDITION] La regla ya no espera por ningún evento y las
	reglas de las que depende ya han sido lanzadas, se verifica cada cierto
	tiempo si el entorno cumple con una condición definida. 
\item[FINISH] La regla ha sido lanzada, con un resultado no determinado, se pudo
	haber cumplido, como no, es el estado final de una regla. Cuando una
	regla llega a este estado, se lanza su evento de finalización.
\end{description}

Una regla puede estar en solo un estado, y solamente se permite que el estado
avance, desde \emph{BEGIN} hasta \emph{FINISH}.


\paragraph{Ciclo de vida}

Cuando una regla es definida, y insertada al motor de reglas, inmediatamente
pasa al estado \emph{BEGIN}, luego se verifica si la misma depende de otras
reglas, sí este es el caso, pasa al estado \emph{WAITING\_FOR\_RULE} y escucha a
los eventos de finalización de las reglas anteriores.

Una vez que las reglas anteriores han sido finalizadas, la regla pasa al estado
\emph{WAITING\_FOR\_EVENT} sí deben escuchar por algún evento, en caso contrario
pasan al estado \emph{WAITING\_FOR\_CONDITION}.

Una vez que la regla está en estado \emph{WAITING\_FOR\_CONDITION}, pasa a un
motor que ejecuta su condición cada cierto tiempo, si la condición se cumple, la
regla se ejecuta, y la misma pasa a estado \emph{FINISH}, momento en el cual
notifica a las reglas que dependen de ella que ha sido lanzada.

Una vez que la regla esta en estado \emph{FINISH}, la misma sale del esquema de
ejecución, y solo esta disponible para obtener resultados.

Según el ejemplo de la regla definida en el código\ref{alg:rule:guante}, la
regla al terminar de ser construida pasa a estado \emph{BEGIN}, al no depender
de otras reglas, pasa inmediatamente al estado \emph{WAITING\_FOR\_EVENT},
cuando es lanzado el evento, la regla ejecuta la acción y pasa al estado
\emph{FINISH}.

\paragraph{Motor de ejecución}

Un motor de reglas \gls{eca}, requiere de un proceso que evalúe constantemente
las reglas para verificar si las mismas deben ser lanzadas o
no\cite{bailey2004event}\cite{galton2002two}, este motor puede utilizar el
algoritmo de RETE\cite{de2001eca} para realizar esta verificación, en la
propuesta presentada, la cantidad de reglas definidas, y la no dependencia
circular entre ellas, hace innecesario la implementación de tal
algoritmo\cite{de2001eca}. 

El motor de reglas actúa sobre aquellas reglas en estado
\emph{WAITING\_FOR\_CONDITION} e invoca al procedimiento que se encarga de
validar si la regla puede ser activada (el procedimiento es único por cada
regla), si el mismo determina que la regla puede ser lanzada, el motor ejecuta
la acción de la regla y modifica el estado de la regla a \emph{FINISH}.


\subsubsection{Definición de reglas}

La reglas del procedimiento de extracción de sangre fueron definidas de acuerdo
a los pasos requeridos según el protocolo del procedimiento y al orden en el que
son requeridos. Es decir, cada paso del protocolo tiene asociado una regla
dentro del motor que lo representa y las condiciones asociadas a cada regla
están determinadas por el orden en que deben realizarse dentro del protocolo.

Cada regla tiene una o mas condiciones que deben ser cumplidas para que un paso
del protocolo realizado se considere correcto.

\subsubsection{Retroalimentación y puntuación final}
\label{sec:puntuacion_hemocultivo}

Cada regla tiene asociado un peso, de acuerdo a la dificultad de realizar el
paso, este peso es utilizado al final de la partida para darle una puntuación al
usuario acerca de su rendimiento en la partida.

Además, un regla puede quedar en uno de diferentes estados al final de la
partida como se mostró anteriormente, cada uno de esos estados posee un
significado en el contexto del procedimiento y por lo tanto tiene información
asociada para que al final de la partida se muestre una retroalimentación
correcta al usuario por paso.

\subsection{Valoración de la escala de Glasgow}
\label{sec:puntuacion_glasgow}

Para la evaluación del rendimiento del usuario en el momento de llevar a cabo el
procedimiento de valoración de la escala de Glasgow se tuvo un enfoque
completamente diferente al del procedimiento de extracción de muestras de sangre
debido a la naturaleza propia del procedimiento. 

Como se explico anteriormente, el paciente puede estar en ciertos estados
específicos dentro de la escala, y además dentro de cada estado reacciona de un
forma en particular por lo tanto, al inicio de la partida un componente interno
de la aplicación selecciona de forma aleatoria un estado para el paciente, de
forma tal que cada vez que una partida sea jugada no se repitan los estados de
forma seguida.

El estado aleatorio del paciente es guardado en una variable que no es
modificada hasta que se reinicie la partida. Al final de la partida, la
aplicación pide al usuario que valore el estado del paciente que le fue
presentado, una vez que el usuario confirme su respuesta la aplicación la
compara con el estado guardado y de esta forma puede informar al usuario acerca
de su rendimiento en el diagnostico.

Además, cada posible respuesta dada por el usuario contiene información
relacionada al contexto del procedimiento y a la situación actual presentada la
cual es utilizada como retroalimentación al final de la partida. La puntuación
final dada depende de la cantidad de valoraciones correctas dadas por el usuario
para la respuesta verbal, motora, ocular y nivel de gravedad del paciente.










%\section{Conclusiones}
\setcounter{sectiontotal}{3}

\begin{frame}{Aspectos de diseño}
\begin{itemize}[<+->]
    \item Validar constantemente el contenido con profesionales del área
    \item Motivar a través de la utilización de aspectos lúdicos, como el
        puntaje y la medición del tiempo
    \item Utilizar escenarios aleatorios para favorecer la exploración
    \item Proveer retroalimentación al finalizar cada escenario
    \item Proveer ubicuidad para aumentar la utilización
    \item Imágenes representativas son útiles para indicar el estado de
        entidades
\end{itemize}
\end{frame}
\begin{frame}{Desarrollo}
\begin{itemize}[<+->]
    \item Utilizar un motor de videojuegos
    \item Seleccionar herramientas de acuerdo a las características necesarias
    \item Probar el desarrollo de manera frecuente
    \item Crear entornos en tres dimensiones para aumentar la inmersión
    \item Proveer retroalimentación clara al realizar una acción
\end{itemize}
\end{frame}
\begin{frame}{Evaluación}
\begin{itemize}[<+->]
    \item Diseñar herramientas para evaluar el conocimiento de los usuarios con
        apoyo de profesionales
    \item Registrar actividades del usuario
    \item Evaluar la opinión de los usuarios
\end{itemize}
\end{frame}


\section{Trabajos futuros}

La utilización de las \Gls{tic} en la educación es un área de estudio
interesante, si a esto se le añade el nuevo rol asumido por las \Gls{tic} en las
corrientes pedagógicas contemporáneas, y la enseñanza de profesionales de la
salud, los temas para trabajos de investigación son prácticamente ilimitados. 

En esta sección se describen posibles temas para trabajos futuros, que utilicen
a los juegos serios como área de investigación.

\begin{itemize} 
   
\item \textbf{Nuevos escenarios de práctica}: En el área de enfermería existen
    innumerables procedimientos cuya simulación puede tener un impacto positivo
    según apreciaciones de los profesores y alumnos.

\item \textbf{Visión de progreso}: añadir a la retroalimentación el progreso del
    alumno, donde se pueda ver como fue mejorando a través de diversas sesiones,
    cuales son sus puntos débiles y otros aspectos que pueden ser extraídos
    cuando se estudian los datos de varias sesiones de manera conjunta. 

\item \textbf{Integración con sistemas de monitoreo}: ofrecer un mecanismo a los
    docentes, donde se pueda obtener información a cerca de las debilidades y
    fortalezas del grupo de alumnos.

\item \textbf{Multijugador}: crear simulaciones donde varios alumnos participen
    al mismo tiempo, interactuando entre si, y creando conocimiento, permitirá
    explotar áreas que no son posibles con un sólo jugador, como: comunicación
    en clave, sincronización de actividades, trabajos multidisciplinarios.

\item \textbf{Escenarios dinámicos}: simulaciones centradas en crear escenas con
    entornos complejos, donde se deban realizar diferentes procedimientos de
    acuerdo a la situación, permitirán entrenar el poder y la velocidad de
    reacción, el nerviosismo y otros aspectos intrínsecos a situaciones
    desconocidas. 


\item \textbf{Exploración de plataformas de realidad virtual}: Utilizar al
    \emph{Oculus Rift} o herramientas sitiares, para crear entornos virtuales
    donde el jugador se puede desplazar e incluso utilizar elementos de forma
    natural\cite{makerbot,unity:vr}.


\item \textbf{Dificultad de acuerdo al alumno}: utilizar sistemas de tutoría
    inteligente, que pueden ayudar a determinar contenido cognitivo preciso para
    los usuarios de acuerdo a su nivel de conocimiento y de aptitud.


\end{itemize}


\printbibliography{}

\end{document}
