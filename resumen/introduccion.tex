
\section{Introducción}
%1/2 pagina

El rol de las \gls{tic} en la educación es, tradicionalmente, un mecanismo más
para transmitir el conocimiento del maestro al alumno, reemplazando libros y
diapositivas\cite{laptop:instructionism}. Actualmente las nuevas corrientes
pedagógicas requieren que el rol de las tecnologías en la educación sea más
activo dentro del proceso de aprendizaje, ayudando incluso al alumno en la
construcción del conocimiento.

Este trabajo se centra en los juegos serios, que se definen como videojuegos que
tienen un objetivo pedagógico específico, sin descuidar los aspectos lúdicos ni
el involucramiento del usuario\cite{sg:aoverview,ludus:sg,abt1987serious}. A
continuación se definen los objetivos de este trabajo final de grado.

\subsection{Objetivo General}

Identificar y valorar los factores pedagógicos, de diseño, de implementación y
de evaluación que influyen a la creación de herramientas educativas que utilizan
las corrientes pedagógicas actuales apoyadas en las \gls{tic}, especialmente los
juegos serios.

\subsection{Objetivos Específicos}

\begin{itemize}

\item Proveer una visión actualizada de los fundamentos y estado del arte de las
    nuevas corrientes pedagógicas y su relación con las \gls{tic}.

\item Proveer una visión actualizada de los juegos serios, sus principales
    características y sus ventajas y desventajas como herramientas pedagógicas.

\item Identificar áreas de aplicación de los juegos serios, para determinar un
    contexto local factible para su aplicación.

\item Clasificar y seleccionar las herramientas tecnológicas disponibles para el
    desarrollo de soluciones que involucran a los juegos serios.

\item Contrastar en la práctica los conocimientos teóricos adquiridos a través
    del diseño e implementación de un juego serio.

\item Evaluar la solución propuesta para la obtención de datos que permitan
    identificar aspectos de diseño, desarrollo y evaluación a tener en cuenta
    para la creación de un juego serio.

\end{itemize}
