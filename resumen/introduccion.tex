%! TEX root = main.tex

\section{Introducción}
%1/2 pagina

Tradicionalmente el rol que ocupaban las \Gls{tic} en la educación era sólo como un mecanismo más para transmitir el
conocimiento del maestro al alumno, reemplazando libros y
presentaciones\cite{laptop:instructionism}. Actualmente existen corrientes
pedagógicas que requieren que el rol de las tecnologías en la educación sea 
más activo dentro del proceso de aprendizaje, ayudando incluso al alumno en la
construcción del conocimiento.

Este trabajo se centra en los juegos serios, que son videojuegos que tienen
un objetivo pedagógico específico, sin descuidar los aspectos lúdicos ni la
implicación del usuario\cite{sg:aoverview,ludus:sg}. A continuación se definen 
los objetivos de este trabajo final de grado.

\subsection{Objetivo General}

Realizar una investigación sobre el rol actual de las tecnologías de la
información y la comunicación en la educación, poniendo énfasis en las 
corrientes pedagógicas actuales, y evaluar las distintas tecnologías 
disponibles para introducirlo en un área especifica.

\subsection{Objetivos Específicos}

\begin{itemize}
    \item Proveer un resumen de los fundamentos y estado actual de la corriente
        \emph{Construccionismo}, como herramienta pedagógica y su relación con
        las \Gls{tic}.

    \item Proveer un resumen acerca de los juegos serios y corrientes afines como
        herramientas para la aplicación del construccionismo.
    
    \item Proveer un resumen de las áreas de aplicabilidad de los juegos serios, poniendo
        énfasis en las áreas que permiten un enfoque construccionista.
        
    \item Identificar las características del área de enfermería que hacen que
        la misma sea un contexto factible para la aplicación de los juegos
        serios basados en el construccionismo.
    
    \item Analizar, evaluar y seleccionar las herramientas que nos permitan la
        implementación de un juego serio que simule un laboratorio de
        enfermería.
        
    \item Diseñar e implementar un juego serio construccionista que permita
        exponer las ventajas y desventajas como modelo de apoyo a la enseñanza
        tradicional. 
        
    \item Evaluar la solución propuesta para la obtención de datos que permitan
        medir las fortalezas y debilidades desde el punto de vista del usuario,
        en cuanto a factores de: exploración, representación, motivación, inmersión,
        utilidad, retroalimentación, y, pedagogía.

    \item Identificar las fortalezas y debilidades de los métodos utilizados
        para definir su aplicabilidad como herramienta de apoyo. 

     \item Identificar desde el ámbito del diseño, desarrollo y evaluación los 
        puntos que deben tenerse en cuenta a la hora de implementar este tipo de 
        herramientas de apoyo.
\end{itemize}
