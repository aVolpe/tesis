%! TEX root = main.tex

\section{Introducción}
%1/2 pagina

El rol de las \gls{tic} en la educación es, tradicionalmente, un mecanismo más
para transmitir el conocimiento del maestro al alumno, reemplazando libros y
diapositivas\cite{laptop:instructionism}. Actualmente las nuevas corrientes
pedagógicas requieren que el rol de las tecnologías en la educación sea más
activo dentro del proceso de aprendizaje, ayudando incluso al alumno en la
construcción del conocimiento.

Este trabajo se centra en los juegos serios, que se definen como videojuegos que
tienen un objetivo pedagógico específico, sin descuidar los aspectos lúdicos ni
el involucramiento del usuario\cite{sg:aoverview,ludus:sg,abt1987serious}. A
continuación se definen los objetivos de este trabajo final de grado.

\subsection{Objetivo General}

Identificar y valorar los factores pedagógicos, de diseño, de implementación y
de evaluación que influyen a la creación de herramientas educativas que utilizan
las corrientes pedagógicas actuales apoyadas en las \gls{tic}, especialmente en
los juegos serios. 

\subsection{Objetivos Específicos}

\begin{itemize}

    \item Proveer una visión actualizada de los fundamentos y estado del arte de
        las nuevas corriente pedagógicas y su relación con las \Gls{tic}.

    \item Proveer una visión actualizada de los juegos serios, sus áreas de
        aplicación y corrientes afines como herramientas pedagógicas.
    
    \item Identificar las características del área de enfermería que hacen que
        la misma sea un contexto factible para la aplicación de los juegos
        serios.
    
    \item Analizar, evaluar y seleccionar las herramientas disponibles para la
        implementación de un juego serio.
        
    \item Diseñar e implementar un juego serio que permita exponer las ventajas
        y desventajas como modelo de apoyo a la enseñanza tradicional. 
        
    \item Evaluar la solución propuesta para la obtención de datos que permitan
        medir las fortalezas y debilidades desde el punto de vista del usuario,
        en cuanto a factores de: exploración, representación, motivación,
        inmersión, utilidad, retroalimentación, y, pedagogía.
         
     \item Identificar aspectos de diseño, desarrollo y evaluación a tener en
         cuenta para la creación de un juego serio.

\end{itemize}
