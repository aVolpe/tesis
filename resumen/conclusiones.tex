\section{Conclusiones}

Durante esta trabajo de grado se estudió la utilización de las \Gls{tic} en la
educación, en especial a los juegos serios, y la teoría construccionista del
aprendizaje. Como resultado se diseñó y desarrolló una aplicación para
dispositivos móviles cuyo fin es el de servir como herramienta de apoyo en el
proceso de aprendizaje de los estudiantes de la carrera de enfermería.

Consideramos interesante la investigación de herramientas como la solución
propuesta en este trabajo ya que la tendencia actual es que las tecnologías
tengan un papel más activo en el proceso de enseñanza-aprendizaje, y las nuevas
corrientes pedagógicas requieren que sea así. 

Como se refleja en varios artículos, los estudiantes del nuevo milenio están
acostumbrados a las nuevas tecnologías y tienen otros estilos de aprendizaje,
por lo que debe asumirse el desafío de incorporar a la tecnología con mayor
fuerza, aportando aún más dinamismo en los procesos de enseñanza-aprendizaje.
Creemos además, que el factor motivacional que puede brindar este tipo de
herramientas influirá positivamente en los estudiantes y en el profesores.

A continuación se detallan las conclusiones obtenidas en cada una de las fases 
del desarrollo de la aplicación.

\subsection{Diseño de la aplicación}

En este apartado se detallan aquellas conclusiones obtenidas durante la 
etapa de diseño de la aplicación.

\begin{itemize}
\item \textbf{Retroalimentación constante con profesores y alumnos}.

Es importante tener una buena relación con los profesores, y además mantener 
su interés en el proyecto, pues es necesario obtener la mayor cantidad de 
conocimientos y detalles en cuanto al contenido de la aplicación y, sobre 
todo, para que los profesores evalúen cada paso realizado. Los estudiantes 
pueden brindar información sobre el contenido y sus necesidades desde otro 
punto de vista totalmente válido pues son los usuarios finales de la 
aplicación.

Debido a la apretada agenda que poseen los profesionales de salud, poder
reunirse con ellos frecuentemente y en un largo periodo de tiempo para validar
ideas y contenido de la aplicación resulta ser muy difícil, por lo cual es
importante recabar la máxima cantidad de información en cada reunión.	
	
	
\item \textbf{Realismo y facilidad de uso de la \Gls{gui}}.

En cuanto a la interacción del usuario con la aplicación, siempre es 
preferible que sea lo más natural posible. La forma de uso de la aplicación no 
debería ser un obstáculo para utilizarla. La forma en que los elementos son 
representados y utilizados dentro de la aplicación debe ser realista, o en 
todo caso debe representar de la mejor manera la realidad de acuerdo a las 
limitaciones y sin distraer al alumno del objetivo pedagógico.

La interacción a través de dispositivos móviles debe ser cuidadosamente 
diseñada para proveer de una interacción fluida y significativa.


\item \textbf{Ubicuidad de los dispositivos móviles}.

Según datos obtenidos en la \emph{Encuesta para obtener la muestra} el $98\%$ 
de los estudiantes encuestados cuenta con al menos un dispositivo móvil 
inteligente por lo que la idea de brindarles ubicuidad es un objetivo 
factible, no sólo por que cuentan con este medio sino por que ello 
significa brindarles la oportunidad de utilizar la herramienta en cualquier 
lugar y momento.


\item \textbf{Aspectos lúdicos como motivación}.

Los juegos serios no sólo permiten al estudiante experimentar, poner a prueba 
y adquirir conocimientos sino que, debido a sus características lúdicas, ayuda 
en la motivación siendo además una opción diferente con respecto a las demás
opciones tecnológicas utilizadas en el ámbito educativo.
    
En este sentido, en los resultados obtenidos de la evaluación de la solución 
se muestra que en promedio los usuarios manifestaron estar \enquote{De 
acuerdo} con el efecto motivacional de la solución en cuanto al uso de 
puntaje, tiempo y socialización.

\end{itemize}


\subsection{Desarrollo de la aplicación}

En este apartado se detallan aquellas conclusiones obtenidas durante la 
etapa de desarrollo de la aplicación.

\begin{itemize}
\item \textbf{Importancia del uso de motores de videojuegos}.

La utilización de motores gráficos modernos facilita la creación de juegos
serios, relacionados a procedimientos del área de enfermería, permiten crear y
manipular entornos realistas sin demasiadas complicaciones. Una buena elección 
del motor gráfico de acuerdo a las características de la aplicación que se 
desea implementar es sumamente importante.

\item \textbf{Pruebas para mejoras durante el desarrollo}.

La realización de prueba de usuarios es interesante durante el desarrollo de 
una aplicación ya que permiten visualizar los errores en su funcionamiento y 
de esta manera se pueden corregir las debilidades encontradas de manera 
temprana, siempre es mejor detectar la mayor cantidad de errores en el menor 
tiempo posible.

En el caso de la solución, se realizaron pruebas de \Gls{gui} para mejorar la
interacción de los usuarios con ella, estas pruebas revelaron principalmente
problemas con la interacción con el entorno y con los objetos. Como resultado 
se aplicaron mejoras que permitieron mejorar el uso y la intuitividad de la
interfaz.

\item \textbf{Validaciones de contenido de la aplicación}.

Estas validaciones deben ser realizadas con los profesionales expertos en el 
área ya que no sólo permiten encontrar errores sino que permiten asegurar que 
la forma de implementación de ciertos aspectos es correcta. En el caso de la
solución, esto permitió corregir errores en las reglas definidas para la
evaluación del procedimiento de extracción de muestras de sangre, y la
presentación de la escenografía de manera tal que fuera más realista.

Más allá de lo expuesto, estas validaciones con profesionales permiten que se
sientan parte del proyecto, manteniendo el interés y predisposición de los
profesionales.

\end{itemize}


\subsection{Evaluación de la aplicación}

En este apartado se detallan aquellas conclusiones obtenidas durante la 
etapa de evaluación de la aplicación.

\begin{itemize}
\item \textbf{Evaluación de conocimiento con apoyo de profesionales}.

Cuando se diseñan encuestas relacionadas con la medición del conocimiento, en
nuestro caso de estudiantes de la carrera de Licenciatura en Enfermería, es
imprescindible elaborarlas con la ayuda y opinión de profesionales en el área,
ya que tienen experiencia en la medición del conocimiento, esto no sólo ayuda 
a realizar una mejor medición sino que permite controlar que el contenido se
encuentre dentro del grado de dificultad acorde a los estudiantes a los que va
dirigido. 

En el caso de este trabajo, las preguntas se basaban en respuestas de 
selección múltiple con una sola respuesta verdadera, por lo que los 
profesionales ayudaron a evitar que las formulaciones y la lista de respuestas sean ambiguas o confusas.

\item \textbf{Importancia de los registros de actividades}.

El registro de actividades del usuario es una herramienta importante, pues
permite contrastar los datos obtenidos con otras metodologías, obteniendo
correlaciones. Además permiten evaluar información interesante sobre la
utilización de una aplicación, como frecuencia de uso, tiempo de uso, etc.

Estas herramientas deben ser transparentes para el usuario y deben ser capaces
de funcionar aún sin acceso constante a internet.

\item \textbf{Evaluación de la solución por los usuarios}.

Se deben diseñar encuestas que puedan obtener información que no se puedan
apreciar con las pruebas de conocimiento y los registros de actividades, como 
es la apreciación subjetiva de los estudiantes en cuanto a la aplicación en
diferentes aspectos como utilidad, motivación, representación, pedagogía,
entre otros. En el caso de este trabajo, sirvió para validar también algunas 
hipótesis planteadas.

\end{itemize}

\subsection{Otras conclusiones}

En esta sección se detallan conclusiones sobre aspectos transversales 
a los demás puntos tratados.

\begin{itemize}
\item \textbf{Utilidad de la simulación como herramienta de apoyo}.

Los profesores encargados del proceso de aprendizaje de los estudiantes de
enfermería con los que se trabajó en el desarrollo de este trabajo en todo
momento estuvieron abiertos al uso de las \Gls{tic} en la forma de juegos
serios, incluso mencionaron la idea de poder utilizarlos en clase. 

Esto permite concluir que están abiertas las posibilidades de inclusión de la
tecnología en formas más innovadoras con respecto a la forma de utilización en
la actualidad en nuestro país.

\item \textbf{Mantener interés de los estudiantes}.

Los estudiantes que formarán parte de la prueba de la aplicación deben estar
bien informados sobre el objetivo de la aplicación y la validez de la misma, 
ya que deben estar interesados en probarla y ayudar para facilitar la 
evaluación. Esto es importante sobre todo cuando la naturaleza de la 
aplicación no permite una evaluación controlada, como es el caso de este 
trabajo.

\end{itemize}

