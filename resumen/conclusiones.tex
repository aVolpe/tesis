\section{Conclusiones}

Durante esta trabajo de grado se analizó y evalúo la utilización de las
\Gls{tic} en la educación, en especial a los juegos serios. Como resultado se
diseñó y desarrolló una aplicación para dispositivos móviles cuyo fin es el de
servir como herramienta de apoyo en el proceso de aprendizaje de los estudiantes
de la carrera de enfermería. Así como también para identificar y valorar
los factores pedagógicos, de diseño, de implementación y de evaluación que
influyen en la creación de juegos serios.

%Se considera interesante la investigación de herramientas como la solución
%propuesta en este trabajo, debido a la tendencia actual de promover tecnologías
%que tengan un papel más activo en el proceso de enseñanza-aprendizaje, abriendo
%camino a las nuevas corrientes pedagógicas. 

Los estudiantes del nuevo milenio están acostumbrados a las nuevas tecnologías y
tienen otros estilos de aprendizaje. Por lo tanto se debe asumir el desafío de
incorporar a la tecnología con mayor fuerza, aportando aún más dinamismo en los
procesos de enseñanza-aprendizaje. Se considera además, que el factor motivacional
que pueden brindar este tipo de herramientas influirá positivamente en los
estudiantes y en el profesores.

A continuación se detallan las conclusiones obtenidas en cada una de las fases
del desarrollo de la aplicación.

\subsection{Estado del arte}

\begin{itemize}

% REVISAR
\item Los beneficios de las \gls{tic} aún no hay sido explotados completamente
    en el área  educativa, el rol de las \gls{tic} en la educación sigue siendo
    mayormente como proveedor de conocimiento debido a que el instruccionismo es
    aún la corriente pedagógica con mayor vigencia.

\item Los juegos serios implementados tienden a ofrecer una retroalimentación muy guiada, orientando 
    al usuario a hacer lo correcto sin brindarle suficiente espacio para que 
    desarrolle su pensamiento crítico y toma de decisiones.

\item Los juegos serios permiten una experiencia sin riesgos reales, permiten
    al usuario experimentar, poner a prueba y construir conocimientos sin los
    riesgos económicos y de salud presentes en la vida real.

% Ver si vale la pena citar
\item La enseñanza de profesionales de enfermería es un área propicia para la
    aplicación de los juegos serios, los estudiantes de enfermería requieren un
    alto grado de prácticas y poseen poco tiempo disponible para actividades
    ajenas a las académicas impuestas por su centro de estudios. 
    
\item Se deben utilizar herramientas alternativas de bajo costo, el nivel de acceso a la
    tecnología de los estudiantes del \gls{iab} es bajo. Proveer soluciones que
    requieren una inversión importante no es una alternativa viable actualmente.

\end{itemize}

\subsection{Diseño del juego serio}

\begin{itemize}


\item La definición de los aspectos pedagógicos, nivel de detalle requerido por
    procedimiento y las características del entorno debe ser realizada con
    profesores de prácticas y directores de carrera.

\item La utilización del puntaje por procedimiento motiva a los usuarios, los
    datos mostrados en el cuadro~\ref{tab:resultado_resumen_aspectos_aceptacion}
    permiten concluir que el principal factor motivacional es la visualización
    de un puntaje que resuma el desempeño del usuario y, en menor medida, la
    medición del tiempo y la posibilidad de compartir su rendimiento en las
    redes sociales. 

\item Los estados aleatorios y funciones simplificadas facilitan la exploración,
    los datos mostrados en el
    cuadro~\ref{tab:resultado_resumen_aspectos_aceptacion} permiten concluir que
    los principales factores que favorecen la exploración son la aleatoriedad en
    el estado del paciente y la representación simplificada de las funciones de
    los elementos. Sin embargo, esta simplificación debe diseñarse con mucho
    cuidado para no perder intuitividad en la interfaz. 


\item Los gráficos en tres dimensiones y las partidas cortas aumentan la
    inmersión del usuario, los datos en el
    cuadro~\ref{tab:resultado_resumen_aspectos_aceptacion} muestran que la
    utilización de gráficos en tres dimensiones para la representación de
    elementos y lugares a los que está familiarizado el usuario, así como
    proveer partidas cortas para evitar que el usuario pierda el contexto de sus
    acciones, favorecen a la inmersión. 

\item Se debe proveer retroalimentación sobre el desempeño del usuario sólo al
    finalizar la partida, los datos mostrados en el
    cuadro~\ref{tab:resultado_resumen_aspectos_aceptacion} permiten concluir que
    ofrecerles a los usuarios una simulación de los procedimientos con una
    retroalimentación limitada les ayuda a poner en práctica sus conocimientos y
    a comprender el procedimiento. 

\item La información sobre el rendimiento del usuario debe ser detallada, los
    datos mostrados en el cuadro~\ref{tab:resultado_resumen_aspectos_aceptacion}
    permiten concluir que los usuarios están de acuerdo con el hecho de
    proporcionarles una retroalimentación indicándoles los pasos que realizó de
    manera correcta e incorrecta dentro del procedimiento. Sin embargo, una
    breve causa acerca de las equivocaciones no es suficiente, se requiere
    información detallada. 
    
\item Se deben utilizar indicadores de realización de acciones, en la simulación
    de entornos de enfermería, no todas las acciones son visibles ante el ojo
    humano, para estos casos se debe diseñar un esquema que notifique al usuario
    sobre la realización de una acción.
    
\item Se debe limitar la manipulación del punto de vista al utilizar elementos
    en dispositivos móviles. En las pruebas preliminares de la interfaz se
    detecto que los usuarios tienen problemas al manipular el punto de vista
    mientras utilizan los elementos.

\end{itemize}

\subsection{Implementación del juego serio}

\begin{itemize}
%volver a revisar este titulo
\item La interacción y la utilización de gráficos en tres dimensiones, son las
    principales diferencias entre el desarrollo tradicional de software y el desarrollo de
    juegos serios. Las diversas formas de interacción implican un desafío al 
    momento de realizar pruebas. Los gráficos en tres dimensiones implican un desafío
    en cuanto al aspecto estético como a las posibilidades que posee el
    usuario.

\item El uso de un motor de videojuego facilita el desarrollo, crear un juego
    serio es un proceso sumamente complejo y costoso, la utilización de un motor
    moderno, como \emph{Unity3D}, facilita el desarrollo. 
        
\item Se recomienda tener en cuenta el costo, requisitos mínimos, familiaridad,
    librerías, tienda y comunidad al seleccionar un motor de videojuego.

\item Se recomienda utilizar la guía para el desarrollo de un juego
        serio definida por Pereira\cite{pereira2009design}. 

\item El uso de un motor de reglas condicionado por eventos es suficiente para
    evaluar al usuario, permite la evaluación del usuario al momento de realizar
    las acciones, lo que a su vez, permite tener acceso al contexto de la
    acción. 

\item Es necesario evaluar al usuario en el front-end, esto  permite brindar una
    mayor movilidad y posibilita una mayor fluidez en la experiencia al no
    depender de una conexión a internet. 

\item Se deben diseñar personajes sólo cuando se requiere un alto nivel de
    detalle o interacción, las diferentes fuentes de personajes permiten acceder
    a una gran cantidad de modelos de seres humanos, solo se deben modificar o
    crear personajes desde cero si se requiere un alto nivel de detalles.

\item Es necesario enviar automáticamente los registros de utilización, en la
    solución el usuario debe seleccionar la opción de enviar datos, lo que
    provocó que los datos no sean siempre enviados.

\end{itemize}

\subsection{Evaluación del juego serio}

\begin{itemize}

\item Se deben validar las pruebas de conocimiento con los profesores de cátedra para
    determinar la dificultad y la relevancia de los temas a tratar.

\item Es necesario registrar todas las acciones del usuario, no sólo sirven para
    determinar el rendimiento del usuario, sino además permiten evaluar el uso
    de la interfaz, la frecuencia de utilización, entre otros aspectos.

\item Los juegos serios ayudan a los estudiantes de enfermería a poner a prueba
    sus conocimientos, como se observa en el
    cuadro~\ref{tab:resultado_resumen_aspectos_aceptacion}, los estudiantes de
    enfermería consideran que el uso de la solución apoya al estudio en clase y
    laboratorio con respecto a otros materiales utilizados que poseen
    limitaciones físicas y ofreciéndoles un paciente que reacciona a sus
    acciones. 

\end{itemize}

