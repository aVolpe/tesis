\section{Definición del problema}

%1/2 página
%\begin{itemize}
%\item Estado actual
%\item Prácticas en laboratorio
%\item Problemas actuales
%\item Propuesta de solución
%\end{itemize}

Este trabajo de grado se centra en las problemáticas referentes al área de
enfermería proponiendo una solución basada en las \gls{tic}. Se considera como
campo de estudio al \gls{iab}, cuyos estudiantes del último año de la carrera de
Licenciatura en Enfermería son tomados como población objetivo.

Uno de los principales inconvenientes de los estudiantes es la poca
disponibilidad de tiempo que poseen, el cual se ve reducido por el tiempo que
invierten en medios de transporte\cite{iab:tesis_alumnos}. En cuanto a las
prácticas que realizan en un laboratorio, por la cantidad de estudiantes es muy
difícil la personalización de la enseñanza\cite{iab:tesis_alumnos} y en cuanto a
las prácticas en hospitales, uno de los inconvenientes es el nerviosismo ante las
primeras prácticas.

Según~\cite{humphreys2013developing} los alumnos de enfermería son estudiantes
divergentes, es decir aprenden a través de experimentación activa, e
interiorizan el conocimiento reflexionando sobre la experiencia. Los juegos
serios son una herramienta ideal para este tipo de
estudiantes\cite{humphreys2013developing}. 

A continuación se describe el estado actual de las prácticas realizadas por los
alumnos como parte del plan curricular, además de los problemas actuales y una
propuesta de solución a esos problemas.

\subsection{Prácticas en laboratorio}

El \gls{iab} cuenta con un laboratorio de prácticas especializado para los
estudiantes de enfermería. Este laboratorio es utilizado para desarrollar las
materias prácticas para tener una formación previa a las prácticas de
campo, es decir, prácticas en hospitales con pacientes reales.

El número de alumnos dificulta la enseñanza individual por lo que las clases de
laboratorio se dividen en dos, la primera parte se desarrolla en un aula
convencional en donde el profesor enseña a los estudiantes usando a un
alumno como voluntario, o usando modelos de partes del cuerpo para realizar
explicaciones generales y responder dudas. La segunda parte se lleva a cabo en
un laboratorio que cuenta con herramientas como maniquíes, camas, utensilios,
entre otros. En esta segunda parte los alumnos pueden explorar y experimentar
por sí mismos bajo la tutela del profesor.

\subsection{Prácticas de campo}

Las prácticas de campo son aquellas prácticas profesionales que son realizadas
por los alumnos con pacientes humanos y en hospitales, bajo supervisión de un
profesional denominado instructor y bajo una continua evaluación de sus
acciones, las mismas son llevadas a cabo una vez que los alumnos finalizan las
prácticas de laboratorio.

Cada instructor posee un planilla por alumno donde se realiza el seguimiento de
sus actividades. La creación de esta planilla de actividades es responsabilidad
del instructor, el instructor debe basarse en las competencias básicas de la
asignatura y la misma es validada por la dirección de la carrera. Se considera
que un alumno ha adquirido la pericia\footnote{Sabiduría, práctica, experiencia
y habilidad en una ciencia o arte.} necesaria para una asignatura sí
pudo completar la planilla del instructor. Si el alumno no aprueba todas las
prácticas no tiene derecho a rendir el examen teórico de la materia y debe
volver a cursar la materia.

\subsection{Problemas actuales}

Si bien el nivel de actual de los egresados de la carrera de Licenciatura en
Enfermería del \gls{iab} es considerando satisfactorio por las autoridades de
esta casa de estudios, se encuentran inconvenientes en la formación.
Apreciaciones de profesores, alumnos y de tesis de grado de la
institución\cite{iab:tesis_atencion,iab:tesis_alumnos} ponen en evidencia estas
falencias. A continuación se citan estos inconvenientes.

Los problemas manifestados por los profesores de la institución incluyen
nerviosismo ante la primer práctica por parte de los alumnos y falta de un
protocolo de comunicación entre el alumno y el profesor.

En tanto, los mayores inconvenientes detectados por los alumnos son:
\begin{itemize}
\item Alta carga horario de los trabajos prácticos.
\item Reducida carga horaria para el estudio de las materias teóricas 
debido a las prácticas.
\item Poca flexibilidad de los profesores.
\item Falta de materiales actualizados para los profesores.
\item Problemas de transporte para llegar hasta los hospitales o al 
    \gls{iab}.
\item Falta de preparación para las prácticas.
\item Alta cantidad de alumnos, debido a esto las prácticas de campo 
rara vez se realizan en un sólo hospital y son realizadas por grupo de
alumnos.
\end{itemize}

\subsection{Propuesta de solución}

Los enfermeros son profesionales que requieren un alto grado de prácticas, y
en la actualidad se presentan los problemas citados anteriormente en su formación. En este
trabajo se propone el desarrollo de un juego serio, que incluye la simulación
de laboratorios virtuales, como una herramienta de apoyo al proceso de aprendizaje
de los alumnos de la carrera de Licenciatura en Enfermería.

Se considera que los principales problemas que pueden ser abordados con esta
solución son los siguientes: evaluación de los alumnos, tiempo de práctica,
ubicuidad, realismo y enfoque individual. Se ofrece una herramienta que no puede
sustituir a un práctica de campo pero si puede servir como apoyo al aprendizaje.

