\section{Definición del problema}

%1/2 página
%\begin{itemize}
%\item Estado actual
%\item Prácticas en laboratorio
%\item Problemas actuales
%\item Propuesta de solución
%\end{itemize}

Según una investigación los alumnos de enfermería, son estudiantes divergentes,
es decir aprenden a través de experimentación activa, e interiorizan el
conocimiento reflexionando sobre la experiencia. La simulación es una
herramienta ideal para este tipo de estudiantes\cite{humphreys2013developing}.

Es por ello que este trabajo de grado se centra en las problemáticas referentes
al área de enfermería proponiendo una solución tecnológica. Se considera como
campo de estudio al \gls{iab}, cuyos estudiantes del último año de la carrera de
Licenciatura en Enfermería son tomados como población objetivo.

A continuación se describe el estado actual de las prácticas realizadas por los
alumnos como parte de plan curricular, además de los problemas actuales y una
propuesta de solución a esos problemas.

\subsection{Prácticas en laboratorio}

El \gls{iab} cuenta con un laboratorio de prácticas especializado para los
estudiantes de enfermería. Es utilizado para desarrollar las materias prácticas
de manera a tener una formación previa a las prácticas de campo, es decir,
prácticas en hospitales con pacientes reales.

El número de alumnos dificulta la enseñanza individual por lo que las clases de
laboratorio se dividen en dos, la primera parte de desarrolla en un aula
convencional en donde el profesor enseñanza a los estudiantes usando a uno de
ellos como voluntarios o usando modelos de partes del cuerpo para realizar
explicaciones generales y responder dudas. La segunda parte se lleva a cabo en
un laboratorio que cuenta con herramientas como maniquíes, camas, utensilios,
entre otros. En esta segunda parte los alumnos pueden explorar y experimentar
por sí mismos bajo la tutela del profesor.

\subsection{Prácticas de campo}

Las prácticas de campo son aquellas prácticas profesionales que son realizadas
por los alumnos con pacientes humanos y en hospitales, bajo supervisión de un
profesional denominado instructor y bajo una continua evaluación de sus
acciones, las mismas son llevadas a cabo una vez que los alumnos finalizan las
prácticas de laboratorio.

Cada instructor posee un planilla por alumno donde se realiza el seguimiento de
sus actividades. La creación de esta planilla de actividades es responsabilidad
del instructor, el instructor debe basarse en las competencias básicas de la
asignatura y la misma es validada por la dirección de la carrera, y se considera
que un alumno ha adquirido la pericia\footnote{Sabiduría, práctica, experiencia
    y habilidad en una ciencia o arte.} necesaria para una asignatura solo si
pudo completar la planilla del instructor. Si el alumno no aprueba todas las
prácticas no tiene derecho a rendir el examen teórico de la materia y debe
volver a cursar la materia.

\subsection{Problemas actuales}

Si bien el nivel de actual de los egresados de la carrera de Licenciatura en
enfermería del \gls{iab} es considerando satisfactorio por las autoridades de
esta casa de estudios, existen inconvenientes según apreciaciones de profesores
y alumnos, además de algunas tesis de grado de la institución. A continuación se
citan estos inconvenientes.

Algunos profesores manifestaron en reuniones que antes de ir a las prácticas de
campo con los alumnos prefieren dar las primeras clases en laboratorio por lo
siguiente:

\begin{itemize}
\item Falta de preparación de los alumnos debido a que en ocasiones 
ciertos detalles no son cubiertos completamente por las prácticas en 
laboratorio.
\item Definición de un protocolo de comunicación entre alumnos y el 
profesor para su uso en las prácticas de campo.
\item Nerviosismo ante la primera práctica por parte de los alumnos.
\end{itemize}

En tanto, los mayores inconvenientes detectados por los alumnos son:
\begin{itemize}
\item Alta carga horario de los trabajos prácticos.
\item Reducida carga horaria para el estudio de las materias teóricas 
debido a las prácticas.
\item Poca flexibilidad de los profesores.
\item Falta de materiales actualizados para los profesores.
\item Problemas de transporte para llegar a hasta los hospitales o al 
    \gls{iab}.
\item Falta de preparación para las prácticas.
\item Alta cantidad de alumnos, debido a esto las prácticas de campo 
rara vez se realizan en un sólo hospital y son realizadas por grupo de
alumnos.
\end{itemize}

\subsection{Propuesta de solución}

Teniendo en cuenta que los mismos forman parte de un grupo de profesionales que requieren de un alto grado de prácticas y que en la actualidad se 
presentan varios factores que a un estudiante de esta carrera le impide poner 
a prueba todo el tiempo sus conocimientos y por todo lo expuesto anteriormente, en este trabajo se propone el desarrollo de un juego serio, que involucra la simulación de laboratorios virtuales, como una herramienta para 
el proceso de aprendizaje de los alumnos de la carrera de enfermería.

Se considera que los principales problemas que pueden ser abordados con esta 
solución son los siguientes: evaluación de los alumnos, seguimiento del 
progreso del alumno, tiempo de práctica, ubicuidad, realismo, enfoque 
individual. Ofreciendo una herramienta que no puede sustituir a un práctica 
de campo pero si puede servir de apoyo en el aprendizaje relacionado a las 
prácticas.
