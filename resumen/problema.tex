\section{Definición del problema}

Este trabajo se centra en contrastar los conocimientos teóricos adquiridos sobre 
el uso de las \Gls{tic} en la educación y de los juegos serios a través del diseño 
e implementación de un juego serio para un contexto local.

Uno de los principales campos de aplicación de los juegos serios en la educación
es el área de la salud, un ejemplo de esto es la formación de profesionales de
enfermería. Según~\cite{humphreys2013developing} los alumnos de enfermería son
estudiantes divergentes\footnote{Los alumnos divergentes son aquellos que
    aprenden a través de experimentación activa, e interiorizan el conocimiento
    reflexionando sobre la experiencia\cite{humphreys2013developing}}, los
juegos serios son una herramienta ideal para este tipo de
estudiantes\cite{humphreys2013developing}. 

Así, se selecciona como contexto de aplicación a la enseñanza de la carrera
Licenciatura en Enfermería en el \Gls{iab}, cuyos estudiantes del último año son 
tomados como población objetivo.

%Teniendo que el campo de la salud es una de las áreas de aplicación más importantes 
%de los juegos serios. En este trabajo se selecciona como contexto de aplicación 
% al \gls{iab}, cuyos estudiantes del último año de la carrera de
%Licenciatura en Enfermería son tomados como población objetivo.

%Uno de los principales inconvenientes de los estudiantes es la poca
%disponibilidad de tiempo que poseen, el cual se ve reducido por el tiempo que
%invierten en medios de transporte\cite{iab:tesis_alumnos}. En cuanto a las
%prácticas que realizan en un laboratorio, por la cantidad de estudiantes es 
%muy difícil la personalización de la enseñanza\cite{iab:tesis_alumnos} y en 
%cuanto a las prácticas en hospitales, uno de los inconvenientes es el 
%nerviosismo ante las primeras prácticas.

%Según~\cite{humphreys2013developing} los alumnos de enfermería son estudiantes
%divergentes, es decir aprenden a través de experimentación activa, e
%interiorizan el conocimiento reflexionando sobre la experiencia. Los juegos
%serios son una herramienta ideal para este tipo de
%estudiantes\cite{humphreys2013developing}. 

%\subsection{Estado actual}
% ESTO ES ALGO QUE AGREGUE RECIENTEMENTE
Los estudiantes de enfermería requieren una gran cantidad de prácticas, en el
\Gls{iab} estas prácticas se dividen en:
\begin{itemize}
\item \textbf{Prácticas de laboratorio:} se realizan en un laboratorio
    especializado del \Gls{iab}, este laboratorio cuenta con herramientas como
    camas de hospitales, maniquíes, representaciones del cuerpo humano,
    utensilios, entre otros. Las prácticas de laboratorio son una
    preparación previa a las prácticas de campo.
\item \textbf{Prácticas de campo:} se realizan con pacientes humanos y en
    hospitales bajo la supervisión de un profesional denominado instructor. 
\end{itemize}

Cabe destacar que, los estudiantes deben aprobar todas las prácticas para tener 
derecho al examen teórico de la asignatura, de lo contrario deben volver a cursar la 
asignatura.

Los principales problemas actuales relacionados con estas prácticas en el \Gls{iab}
son los siguientes:

\begin{itemize}
%\item Los estudiantes deben aprobar todas las prácticas para tener derecho a
    %examen teórico de la asignatura, de lo contrario deben volver a cursar la
    %asignatura.
\item Los estudiantes poseen poca disponibilidad de tiempo para actividades
    extras debido al tiempo que dedican a las prácticas, a las clases teóricas y
    al transporte\cite{iab:tesis_alumnos}.
\item La personalización de la enseñanza es difícil debido a la gran cantidad de
    alumnos\cite{iab:tesis_alumnos}.
\item Falta de materiales actualizados para los
    profesores\cite{iab:tesis_alumnos}.
\item Nerviosismo ante las primeras prácticas.
\item La determinación de si un estudiante posee la pericia \footnote{Sabiduría,
        práctica, experiencia y habilidad en una ciencia o arte.} necesaria para
    un procedimiento lo realiza el instructor basándose en su percepción y una
    planilla de logros basada en las competencias básicas de la asignatura.
\end{itemize}

Debido a la importancia de las prácticas en la vida académica de los estudiantes
de enfermería y, teniendo en cuenta los problemas citados anteriormente, en este
trabajo se propone el desarrollo de un juego serio que incluye la simulación de
laboratorios virtuales como una herramienta de apoyo al proceso de aprendizaje
de los alumnos de las carrera de Licenciatura en Enfermería. 

Se considera que los principales problemas que pueden ser abordados con esta
solución son los siguientes: evaluación de los alumnos, seguimiento del progreso, disponibilidad de
tiempo, factor psicológico y enfoque individual. Se ofrece una herramienta que no puede
sustituir a un práctica de campo pero si puede servir como apoyo al aprendizaje.

