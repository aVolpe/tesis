\section{Trabajos futuros}

La utilización de las  en la educación es un área de estudio excitante, si a
esto se le añade la corriente pedagógicas del construccionismo, y la enseñanza
de profesionales de la salud, los temas para trabajos de investigación son
prácticamente ilimitados. 

En esta sección se describen posibles temas para trabajos futuros, que utilicen
las mismas bases pedagógica que las utilizadas en este trabajo.

\begin{itemize} 
   
\item \textbf{Nuevos escenarios de práctica}.

Este trabajo presenta dos procedimientos relacionados a la enfermería, en el
área existen innumerables procedimientos cuya simulación puede tener un impacto
positivo según apreciaciones de los profesores y alumnos.

\item \textbf{Visión de progreso}.

Una de las características de esta solución es la retroalimentación que recibe
el usuario al terminar una práctica, lo que, según los mismos alumnos, favorece
a la motivación y a las ganas de superación.

Un añadido a la retroalimentación, sería el progreso del alumno, un lugar donde
el mismo pueda ver como fue mejorando a través de diversas sesiones, donde se
observe cuales son los puntos débiles recurrentes y otros aspectos que pueden
ser extraídos cuando se estudian los datos de varias sesiones de manera
conjunta. 

\item \textbf{Control del progreso por parte de los docentes}.

El presente trabajo no propone mecanismos de evaluación por parte de los
docentes, esta es un área interesante, pues la información recabada acerca del
desempeño de los alumnos podría servir como una alerta al profesor.

Si se estudia el comportamiento de todos los alumnos de manera simultánea, se
podría obtener información acerca de las debilidades y fortalezas del grupo de
alumnos, y así los profesores tendrían una herramienta adicional para el
desarrollo de sus actividades académicas.

\item \textbf{Multijugador}.

El ser humano es un ser social, el construccionismo indica que el conocimiento
es fruto de la interacción social, crear simulaciones donde varios alumnos
participen al mismo tiempo, interactuando entre si, y creando conocimiento,
permitirá explotar áreas que no son posibles con un sólo jugador, como:
comunicación en clave, sincronización de actividades, trabajos
multidisciplinarios.

\item \textbf{Énfasis en el entorno}.

Las simulaciones presentadas en este trabajo se centran en los procedimientos,
el entorno es una herramienta auxiliar que aumenta el realismo y la inmersión.
Simulaciones centradas en crear escenas con entornos complejos, donde se deban
realizar diferentes procedimientos de acuerdo a la situación, permitirán
entrenar el poder y la velocidad de reacción, el nerviosismo y otros aspectos
intrínsecos a situaciones desconocidas. 


\item \textbf{Exploración de plataformas de realidad virtual}.

Herramientas como el \emph{Oculus Rift}, permiten crear entornos virtuales donde
el jugador se puede desplazar e incluso utilizar elementos de forma
natural\cite{makerbot}, cabe mencionar que desde finales del $2014$, estas
herramientas pueden ser utilizadas de manera gratuita con \emph{Unity3d}
\cite{unity:vr}.


\item \textbf{Dificultad de acuerdo al alumno}.

El nivel de dificultad de los diferentes desafíos debe ser acorde al nivel de
preparación de los usuarios. En este trabajo la dificultad es siempre la misma,
pues los alumnos seleccionados provienen del mismo entorno y aprobaron la misma
cantidad de asignaturas en su carrera.

Un aspecto interesante a analizar en este punto, son los sistemas de tutoría
inteligente, que pueden ayudar a determinar contenido cognitivo preciso para los
usuarios de acuerdo a su nivel de conocimiento y de aptitud.


\end{itemize}
