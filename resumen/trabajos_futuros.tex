\section{Trabajos futuros}

La utilización de las \Gls{tic} en la educación es un área de estudio
interesante, si a esto se le añade el nuevo rol asumido por las \Gls{tic} en las
corrientes pedagógicas contemporáneas, y la enseñanza de profesionales de la
salud, los temas para trabajos de investigación son prácticamente ilimitados. 

En esta sección se describen posibles temas para trabajos futuros, que utilicen
a los juegos serios como área de investigación.

\begin{itemize} 
   
\item \textbf{Nuevos escenarios de práctica}: En el área de enfermería existen
    innumerables procedimientos cuya simulación puede tener un impacto positivo
    según apreciaciones de los profesores y alumnos.

\item \textbf{Visión de progreso}: añadir a la retroalimentación el progreso del
    alumno, donde se pueda ver como fue mejorando a través de diversas sesiones,
    cuales son sus puntos débiles y otros aspectos que pueden ser extraídos
    cuando se estudian los datos de varias sesiones de manera conjunta. 

\item \textbf{Integración con sistemas de monitoreo}: ofrecer un mecanismo a los
    docentes, donde se pueda obtener información a cerca de las debilidades y
    fortalezas del grupo de alumnos.

\item \textbf{Multijugador}: crear simulaciones donde varios alumnos participen
    al mismo tiempo, interactuando entre si, y creando conocimiento, permitirá
    explotar áreas que no son posibles con un sólo jugador, como: comunicación
    en clave, sincronización de actividades, trabajos multidisciplinarios.

\item \textbf{Escenarios dinámicos}: simulaciones centradas en crear escenas con
    entornos complejos, donde se deban realizar diferentes procedimientos de
    acuerdo a la situación, permitirán entrenar el poder y la velocidad de
    reacción, el nerviosismo y otros aspectos intrínsecos a situaciones
    desconocidas. 


\item \textbf{Exploración de plataformas de realidad virtual}: Utilizar al
    \emph{Oculus Rift} o herramientas sitiares, para crear entornos virtuales
    donde el jugador se puede desplazar e incluso utilizar elementos de forma
    natural\cite{makerbot,unity:vr}.


\item \textbf{Dificultad de acuerdo al alumno}: utilizar sistemas de tutoría
    inteligente, que pueden ayudar a determinar contenido cognitivo preciso para
    los usuarios de acuerdo a su nivel de conocimiento y de aptitud.


\end{itemize}
