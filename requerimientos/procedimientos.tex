%! TEX root = ../main.tex
\section{Selección de procedimientos}
\label{sec:seleccion_escenas}

\observacion{Titulo: procedimiento?. Aclarar como se pasa de un procedimiento a
    una escena}


Con los criterios definidos, y como se mencionó, se seleccionaron dos
procedimientos de enfermería, los cuales serán incluidos en la solución. A
continuación se fundamentan tales elecciones y se entra en detalle acerca de los
procedimientos seleccionados.

Estos procedimientos son incluidos en la solución creando escenarios dentro 
del juego que permitan realizarlos, es así que cada procedimiento es 
representado con escenarios y escenas diferentes. MBAE??

% Yo pensaba agregarle esto tambien, pero viste que no hablamos nada de la
% solución y eso... por eso tengo miedito

% no entiendo lo que queres decir o tenes en tu cabeza. Dejo que vos nomas hagas esto entonces

\subsection{Extracción de muestras de sangre}
\label{sec:hemocultivo}

El procedimiento denominado \emph{Punción venosa} es utilizado frecuentemente
para extraer muestras de sangre, es un procedimiento invasivo que ofrece un
medio directo de acceso al sistema vascular. 

La frecuencia con la que se le utiliza esta relacionado a su utilidad para
análisis de rutina, los enfermeros lo realizan a diario y es similar a otros
procedimientos como la puesta de una vía intravenosa.

El procedimiento se puede resumir en el proceso de punzar una jeringa en el
brazo al paciente, extraer sangre y retirarla. Si bien estos pasos pueden parecer
sencillos, existen una gran cantidad de factores que definen si el procedimiento
fue realizado correctamente, entre ellas podemos encontrar a factores de 
bioseguridad, como la higuienzación correcta de los materiales, sociales, como
la explicación correcta del procedimiento al paciente, etc.

Este procedimiento es considerado como uno de los apropiados de acuerdo a
la apreciación de los profesionales del \Gls{iab} en diversas reuniones, y son: 

\begin{itemize}
\item El mismo posee pasos bien definidos que deben ser seguidos por el
    profesional de enfermería.
\item La complejidad del procedimiento no es muy alta, sus pasos son
    susceptibles de equivocaciones, especialmente en lo que se refiere a la
    bioseguridad, aspecto que sera con más detalle al final de esta sección. 
\end{itemize}



\subsubsection{Evaluación}

Se explican las competencias básicas de este procedimiento con
respecto al plan de estudio de los estudiantes de enfermería del \Gls{iab}, los
criterios según la planilla de practica que poseen los profesores para evaluar
al alumno.

La competencia básica que engloba al procedimiento de extracción de sangres es:

\begin{displayquote}
Ayudar en procedimientos invasivos
\end{displayquote}

Los elementos que son almacenados en la planilla de los instructores, son:
\pregunta{Esto fue extraido de planilla del instructor, se puede poner una
referencia la anexo}

\begin{itemize}
\item Informe al paciente acerca del procedimiento que va a ser
    realizado.
\item Preparación de material para la técnica aséptica.
\item Lavado de manos.
\item Calzado de guantes, chaleco estéril, tapaboca y gorro.
\item Preparación de campo.
\item Punción del brazo, extracción de sangre, compresión de zona de punción.
\item Cambio de aguja.
\item Introducción de la muestra en un frasco preparado para tal efecto.
\item Retiro de materiales y equipo de protección personal.
\item Etiquetación y envío a laboratorio.
\end{itemize}

Es decir, estos son los criterios que debe cumplir cualquier estudiante
para poder aprobar la práctica.

\subsubsection{Protocolo del procedimiento}
\label{sec:hemocultivo_protocolo}

La descripción formal del procedimiento, con todos los pasos necesarios
para poder llevar a cabo los criterios requeridos según~\cite{oms:extraccion}
y los profesores del \Gls{iab} son:

\begin{enumerate}
\item Preparar el equipo, lo que incluye seleccionar la jeringa adecuada.
\item Identificar al paciente, presentarse y explicarle el procedimiento que va
    a ser realizado.
\item Asepsia de las manos.
\item Llevar el equipo a la unidad en donde se encuentra el paciente.
\item Vestirse con bata estéril, tapaboca y gorro.
\item Calzarse los guantes.
\item Ubicar al paciente en posición adecuada, esto es, el brazo debe estar
    extendido y lo mas relajado posible.
\item Elegir la zona a puncionar, para ello se debe palpar la vena para
    averiguar sus características.
\item Colocar el torniquete, 6 a 10 centímetros por encima de la zona de
    punción.
\item Solicitar al paciente que cierre el puño.
\item Esterilizar la zona de punción.
\item Extraer el protector de la aguja.
\item Tensar la zona de punción.
\item Puncionar la piel con la aguja hacia arriba. La aguja se introduce con un
    ángulo de 10 a 20 grados\todox{Ver si no debe ir el lugar de punción}.
\item Remover el torniquete.
\item Solicitar la apertura del puño.
\item Extraer la muestra se sangre necesaria.
\item Presionar y extraer la aguja.
\item Colocar algodón con alcohol en el punto de punción.
\item Sellar la muestra y enviarlo a su destinatario.
\item Retirar el equipo utilizado, incluyendo bata, tapaboca, gorro y guantes.
\item Asepsia de las manos.
\end{enumerate}


\subsection{Valoración de la escala de Glasgow}
\label{sec:glasgow}

\fixme{La escala de Glasgow es una escala utilizada como una herramienta de
    valoración objetiva del estado de conciencia para las víctimas de
    traumatismo}{Alguna fuente} craneoencefálico. La escala esta compuesta por
la exploración y cuantificación de tres parámetros: la apertura ocular, la
respuesta verbal y la respuesta motora. Dando un puntaje a la mejor respuesta
obtenida en cada ítem. El puntaje obtenido para cada uno de los tres se suma,
con lo que se obtiene el puntaje final.

\observacion{Enúmerar parrafo siguiente}
Este procedimiento fue considerado como uno de los apropiados para la solución
propuesta ya que el mismo posee pasos bien definidos que deben ser seguidos por
el profesional de enfermería, no se presentan muchos casos como estos en la
realidad durante las practicas de campo realizados por los estudiantes, el
factor de aleatoriedad para cada uno de las respuestas puede ser
\fixme{potenciado altamente}{Reformular} con la solución, además el
procedimiento no es complejo. A continuación se explican las competencias
básicas de este procedimiento con respecto al plan de estudios de los
estudiantes de enfermería del \Gls{iab}, los criterios según la planilla de
practica que poseen los profesores para evaluar al alumno y el protocolo para
llevar a cabo el procedimiento según~\cite{protocolo} y los profesores del
\Gls{iab}.
\observacion{Ver si necesario extender su lista de criterios para que cubra bien
    este procedimiento}

\subsubsection{Competencias básicas}
\begin{itemize}
\item Identificar actividades de cuidados según problemas urgentes principales.
\end{itemize}

\subsubsection{Criterios de evaluación de la practica}
\begin{itemize}
\item Cita las actividades que debe realizar según la necesidad de urgencia.
    \begin{enumerate*}
    \item Control de signos vitales.
    \item Inspección cefalocaudal, escala de Glasgow.
    \item Preparación del equipo según prioridad del problema
    \item Analizar su participación en las actividades.
    \item Fundamenta científicamente sus decisiones.
    \end{enumerate*}
\end{itemize}

\subsubsection{Protocolo de práctica}
\label{sec:glasgow_protocolo}

\begin{enumerate}
\item Preparación del material (Escala de Glasgow)
\item Preparación del paciente: comprobar su identidad, mantener una ambiente
    tranquilo evitando interrupciones, requerir la atención del paciente.
\item Colocar al paciente en posición cómoda.
\item Medir la apertura ocular, respuesta motora y respuesta verbal.
\item Registrar la puntuación final obtenida.
\end{enumerate}

\observacion{Lo que viene de ahora en más, no debería ir antes?}
Como se comento anteriormente, la escala de coma de Glasgow incluye tres
parámetros: la apertura ocular, la respuesta motora y la respuesta verbal. Cada
parámetro cuenta con items que poseen una puntuación como se muestra en la
tablas~\ref{tab:seleccion_glasgow_respuestas_ocular},~\ref{tab:seleccion_glasgow_respuestas_motor}
y~\ref{tab:seleccion_glasgow_respuestas_verbal}. 

\begin{table}[!hbt]
\centering
\begin{tabular}{lr}
\toprule
\textbf{Apertura ocular} & \textbf{Valor} \\
\midrule
Espontánea & 4 \\
Al hablar & 3 \\
Al dolor & 2 \\
Ausente & 1 \\
\bottomrule
\end{tabular}
\caption{Valoración de las distintas respuestas en la escala de Glasgow,
    respecto a la reacción ocular}
\label{tab:seleccion_glasgow_respuestas_ocular}
\end{table}

\begin{table}[!hbt]
\centering
\begin{tabular}{lr}
\toprule
\textbf{Respuesta motora} & \textbf{Valor} \\
\midrule
Obedece & 6 \\
Localiza & 5 \\
Retira & 4 \\
Flexión anormal & 3 \\
Extiende & 2 \\
Ausente & 1 \\
 & \\
\bottomrule
\end{tabular}
\caption{Valoración de las distintas respuestas en la escala de Glasgow,
    referentes a las respuestas motoras}
\label{tab:seleccion_glasgow_respuestas_motor}
\end{table}

\begin{table}[!hbt]
\centering
\begin{tabular}{lr}
\toprule
\textbf{Respuesta verbal} & \textbf{Valor} \\
\midrule
Orientada & 5 \\
Confusa & 4 \\
Palabras inapropiadas & 3 \\
Palabras incomprensibles & 2 \\
Ausente & 1 \\
\bottomrule
\end{tabular}
\caption{Valoración de las distintas respuestas en la escala de Glasgow
    referentes a la respuesta verbal}
\label{tab:seleccion_glasgow_respuestas_verbal}
\end{table}

Cada uno de los parámetros se evalúa mediante una sub-escala. Cada respuesta se
puntúa con un número, siendo cada una de las sub-escalas independientes. El
estado de conciencia del paciente se determina sumando la puntuaciones de los
tres parámetros. Una vez obtenida la puntuación final, éste se valora según lo
indica la tabla~\ref{tab:seleccion_glasgow_estado}\todox{Estos valores no
    tenemos nosotros, tenemos solo tres.}.

\begin{table}[!hbt]
\centering
\begin{tabular}{llr}
\toprule
\textbf{Nivel ECG} & 
\textbf{Condición clínica} & 
\textbf{Puntuación} \\ 
\midrule
 ECG 1 & Muerte & 3 \\
 ECG 2 & Estado vegetativo & 4 a 6 \\
 ECG 3 & Discapacidad severa & 7 a 9 \\
 ECG 4 & Discapacidad moderada & 10 a 11 \\
 ECG 5 & Recuperacion total & 12 a 15 \\
\bottomrule
\end{tabular}
\caption{Escala de valoración del estado del paciente}
\label{tab:seleccion_glasgow_estado}
\end{table}

\subsection{Bioseguridad}

Además de los procedimientos de enfermería seleccionados y descriptos, existe un
tema mas abordado implícitamente en el procedimiento de extracción de muestras
de sangre, la bioseguridad. La bioseguridad incluye procesos como el lavado de
manos, el uso de elementos de protección personal, entre otros.

\fixme{Los profesores del \Gls{iab}}{Mover esto a la sección donde menciona tema
    de las entrevista????} mencionaron que la bioseguridad es un tema
transversal a todos lo demás y que representa un factor vulnerable por ser
susceptibles de olvido o equivocaciones.

En la solución no se abordara el tema como un procedimiento principal, sino solo
se buscara que los jugadores puedan recordar sin ninguna ayuda cuando y en que
orden dentro del procedimiento que este realizando debe llevar a cabo
actividades de bioseguridad.
