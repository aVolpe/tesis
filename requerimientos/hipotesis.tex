\section{Alcance de la simulación}
\label{sec:alcance}

Para representar los procedimientos seleccionados y descritos en la
sección\ref{sec:seleccion_escenas}, estos deben ser presentados dentro de
escenarios distintos con los elementos necesarios para poder llevarlos a cabo.
Sin embargo, por limitaciones técnicas y de tiempo, no es posible realizar una
simulación de todos los pasos requeridos.

%\observacion{Más énfasis en el hecho que distrae a la parte pedagógica}
%Según~\cite{videojuegos:gonzaleztardon}, existe un compromiso visible entre
%realismo y credibilidad, tomando en cuenta que uno de los objetivos de la
%solución es la presentación de situaciones simplificadas que permitan transmitir
%conocimiento, mientras más realismo exista, más detalles existirán y por
%consiguiente, los usuarios tendrán que concentrarse en un mayor número de
%detalles, lo que resulta contraproducente con el objetivo de la
%solución\cite{videojuegos:gonzaleztardon}. 

Es importante notar que no es un objetivo de la solución realizar una simulación
detallada de los procedimientos, sino proveer una experiencia que permita al
usuario comprender el procedimiento y sumergirse en el entorno.
Según~\cite{videojuegos:gonzaleztardon} es necesario definir minuciosamente qué
factores son simulados y cuáles no, pues si se simulan demasiados factores, el
usuario podría perderse en los detalles y dejar de lado los objetivos
pedagógicos. 

Los criterios y consideraciones de diseño que se describirán a continuación
sirven para acotar el alcance de la simulación, definen qué se simulará y cual
es el detalle necesario para alcanzar las competencias básicas.


\subsection{Factores limitantes}

A continuación se describen los factores que determinan por qué ciertos pasos
de los procedimientos no serán simulados dentro de la solución. Los factores son:


\begin{itemize}
\item  \textbf{Limitaciones técnicas.} 
    
    Las acciones que escapan al alcance del hardware, del software o de tiempo
    de los desarrolladores no son simuladas. Un ejemplo de este tipo de acciones
    es la simulación del agua (necesarios para el lavado de manos) que requiere
    de requisitos de hardware avanzados y un tiempo considerable de desarrollo.
        
    Los pasos del procedimiento venopunción que no se simulan por
    limitaciones técnicas son:
    \begin{itemize}
        \item Tensar la zona de punción.
        \item Ángulo de punción.
        \item Presionar el brazo en el momento en el que se introduce la jeringa.
    \end{itemize}
    
    
\item \textbf{Importancia de representación.}

    No todos los pasos definidos en el procedimiento oficial requieren ser
    simulados para alcanzar los objetivos pedagógicos, por ejemplo, la colocación 
    de los elementos cerca del lugar de
    trabajo, es un paso necesario en el procedimiento, pero es considerado un
    paso de menor importancia. La importancia es evaluada por profesionales del \Gls{iab}.
    
    Los pasos del procedimiento venopunción que no se simulan por 
    no ser importantes de representar en la simulación son:
    \begin{itemize}
        \item Llevar el equipo en la unidad donde se encuentra el paciente.
        \item Extraer el protector de la aguja.
        \item Sellar la muestra y enviarlo a su destinatario.
    \end{itemize}
    
    
\item \textbf{Facilidad de realización.}

    Ciertos pasos son fáciles de realizar en la vida real pero requieren un
    esfuerzo significativo para ser simulados, por ejemplo, preparar el
    equipo necesario. La facilidad que tienen los alumnos con las acciones es 
    determinada por profesores del \Gls{iab}.
    
    Otro aspecto que influye en la facilidad de realización de los
    procedimientos es la familiarización, si los alumnos están familiarizados
    con los procedimientos, estos no se simulan.
        
    Los pasos del procedimiento venopunción que no se simulan por
    ser fáciles de realizar son:
        
    \begin{itemize}
        \item Preparar el equipo.
        \item Ubicar al paciente en posición adecuada.
    \end{itemize}
    
    Los pasos del procedimiento de valoración de la escala de Glasgow 
    que no se simulan por la misma razón son:
    \begin{itemize}
    \item Preparar el material.
    \item Preparar al paciente.
    \item Colocar al paciente en posición cómoda.
    \end{itemize}
        
\end{itemize}

\subsection{Consideraciones de diseño}
\label{sec:hipotesis}

En este apartado se detallan las consideraciones de diseño asumidas durante el diseño de 
la solución. Algunas de ellas envuelven la forma de representación de acciones 
dentro de la simulación, estas consideraciones están  basadas en apreciaciones de los 
profesores del \Gls{iab} y en pruebas de usabilidad de interfaz las cuales son 
detalladas en el capítulo~\ref{chap:evaluacion}. Otras consideraciones representan 
asunciones de los autores 
en cuanto a la utilidad de ciertos aspectos de la simulación.

Las consideraciones de diseño formuladas son las siguientes:

% Hipótesis
\begin{enumerate}[label=\bfseries C\arabic*:]

\item \textbf{Interacción a través de la voz}: para enviar una petición o
    informarle sobre algo al paciente (por ejemplo, darle detalles del
    procedimiento) no es necesario identificar las palabras del usuario, sino
    más bien detectar que ha hablado y listar las posibles acciones que se
    pueden realizar.
    
    Los pasos del procedimiento de venopunción que son representados por 
    comando de voz son:
    
    \begin{itemize}
        \item Explicar procedimiento.
        \item Solicitar al paciente que cierre la mano.
        \item Solicitar al paciente que abra la mano.
    \end{itemize}
    
    Los pasos del procedimiento de valoración utilizando de la escala de \emph{Glasgow} 
    que son representados por la misma razón son:
    
    \begin{itemize}
        \item Medir la respuesta ocular por medio de peticiones al paciente.
        \item Medir la respuesta verbal por medio de preguntas al paciente.
        \item Medir la respuesta motora por medio de peticiones al paciente.
    \end{itemize}

\item \textbf{Extracción de elementos}: para realizar la acción de
    extraer un elemento utilizado en el paciente, se considera que realizarlo de
    una sola manera para todos los elementos disponibles aportan a una interfaz más
    intuitiva.

    Los pasos del procedimiento de venopunción que cumplen con la extracción 
    uniforme son:
    
    \begin{itemize}
        \item Extraer torniquete.
        \item Extraer jeringa.
    \end{itemize}
    
\item \textbf{Bioseguridad}: la bioseguridad es un área amplia y transversal a
    todos los procedimientos de enfermería, por lo que se considera complejo
    simular cada acción. Se considera suficiente que el usuario sepa en que momento
    realizar cada una de las acciones de bioseguridad. 
    
    Los pasos del procedimiento de venopunción relacionados a la bioseguridad son:
    \begin{itemize}
        \item Asepsia de las manos.
        \item Vestir con bata estéril, tapaboca estéril y gorro estéril.
        \item Calzar guantes.
        \item Extraer guantes, bata, tapaboca y gorro.
    \end{itemize}
    
\item \textbf{Representación iconográfica}: para representar el estado de los
    objetos es suficiente mostrar una imagen representativa. Ciertos estados son
    complejos de simular (como la esterilización de las manos), realizar una
    simulación de los estados de los objetos es complejo y requiere un nivel de
    detalle que desviará al usuario de los objetivos pedagógicos.
    
\item \textbf{Motivación}: indicadores de rendimiento, como un puntaje al final
    de cada procedimiento, y el tiempo total dentro del procedimiento, impactan
    positivamente en el involucramiento del usuario. La interacción social es
    otro factor que incrementa el nivel de compromiso del usuario.

\item \textbf{Retroalimentación limitada}: la ausencia de signos visuales que
    indiquen las acciones que debe realizar el usuario durante la experiencia,
    permite al usuario probar sus conocimientos, y le impide avanzar en
    el procedimiento utilizando una técnica de \emph{prueba y error}.

\item \textbf{Movilidad}: el uso de la solución en los dispositivos móviles de
    los usuarios permite más oportunidades de poner a prueba los conocimientos
    con respecto a alternativas tradicionales.


\end{enumerate}
% Simulados

%\subsection{Pasos de exploración}
\subsection{Decisiones de diseño}

No todas las acciones que deben ser realizadas dentro de la simulación están
limitadas por factores o se encuentran envueltas dentro de una hipótesis. La
representación de estas acciones fueron definidas por decisiones de diseño de
los autores, estas decisiones junto con los pasos involucrados son detallados a
continuación.

\begin{itemize}

\item \textbf{Acciones por interfaz de usuario:} acciones como generar un
    estímulo doloroso al paciente tienen limitaciones técnicas para su
    simulación, pero no pueden ser omitidas debido a su gran importancia en el
    procedimiento.
 
    El paso del procedimiento Valoración utilizando la escala de Glasgow que es
    importante de simular y posee limitaciones técnicas es:
    \begin{itemize} 
    \item Realizar estímulos dolorosos en diferentes partes del cuerpo. 
    \end{itemize}
    
\item \textbf{Simulación integra de pasos:} 
	los pasos no mencionados en las hipótesis, factores limitantes y criterios descritos 
	anteriormente, deben ser simulados de manera íntegra.
	
	Una característica de estos pasos es que dependen de la forma en la que son 
	realizados, por ejemplo la elección de una zona de punción, es posible punzar en cualquier
	parte del brazo, pero solo algunos lugares son válidos. 

	Los pasos del procedimiento de venopunción que cumplen con esto son:
	\begin{itemize}
    \item Elegir zona a punzar.
    \item Colocar torniquete.
    \item Esterilizar zona de punción.
    \item Elegir la zona a puncionar.
    \item Punzar la zona con la aguja.
    \item Extraer muestra de sangre.
    \item Colocar algodón en zona de punción.
    \item Presionar zona punción con el algodón.
	\end{itemize}

	Los pasos de Valoración utilizando la escala de Glasgow que cumplen con esto son:
	\begin{itemize}
    \item Registrar la puntuación de la respuesta ocular.
    \item Registrar la puntuación de la respuesta motora.
    \item Registrar la puntuación de la  respuesta verbal.
    \item Registrar el diagnóstico final.
	\end{itemize}

\end{itemize}
