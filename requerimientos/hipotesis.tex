\section{Alcance de la simulación}
\label{sec:hipotesis}

\observacion{Ir más al grano y decir que se debe encontrar un punto de
    equilibrio entre simplicidad y detalle, en la simulación, pero esa es una
    tesis de por sí, entonces ustedes van a basarse en otros trabajos y evaluaciones 
    de los } 
    
Para representar los procedimientos seleccionados y descritos en la sección\ref{sec:seleccion_escenas}
en la solución, estos deben ser presentados dentro de escenarios distintos con 
los elementos necesarios para poder llevarlos a cabo. Sin embargo, 
por limitaciones técnicas, tecnológicas y de tiempo, no es posible realizar 
una simulación de todos los pasos requeridos.

%Las escenas seleccionadas y definidas en la sección~\ref{sec:seleccion_escenas}
%representan las acciones que deben realizar los profesionales de enfermería a la
%hora de realizar los procedimientos seleccionados, por limitaciones técnicas,
%tecnológicas y de tiempo, no es posible realizar una simulación de todos los
%pasos requeridos.

Según~\cite{videojuegos:gonzaleztardon}, existe un compromiso visible entre
realismo y credibilidad, tomando en cuenta que uno de los objetivos de la
solución es la presentación de situaciones simplificadas que permitan transmitir
conocimiento, mientras más realismo exista, más detalles existirán y por
consiguiente, los usuarios tendrán que concentrarse en un mayor número de
detalles, lo que resulta contraproducente con el objetivo de la
solución\cite{videojuegos:gonzaleztardon}. Estas hipótesis sirven para acotar el
alcance de la simulación, definen qué se simulará y cual es del detalle
necesario para alcanzar las competencias básicas.


\subsection{Partes simuladas}

A continuación se fundamentan por qué ciertos pasos serán simulados, representados 
o no simulados dentro de la solución basados en limitaciones e hipótesis asumidas.

Cada uno de los pasos están agrupados de acuerdo a cada uno de estos aspectos.

\subsubsection{Factores limitantes}

Se refieren a cada uno de los aspectos que determinan si un paso 
será simulado dentro de la solución. Se clasifican en tres: limitaciones técnicas, 
importancia y facilidad de realización. Estos tres aspectos que influyen en qué partes se simularán y qué partes se omitirán, se detallan a continuación.

\begin{itemize}
\item  \textbf{Limitaciones técnicas}: acciones como la simulación del agua        
    (necesarios para el lavado de manos), requieren de requisitos de
    hardware avanzados y un tiempo considerable de desarrollo. Las acciones
    que escapan al alcance del hardware, del software o de tiempo de los
    desarrolladores no son simuladas.
        
    Los pasos del procedimiento de extracción de sangre que no se simularán por 
    limitaciones técnicas son:
    \begin{itemize}
        \item Tensar la zona de punción.
        \item Ángulo de punción.
        \item Presionar el brazo en el momento en el que se introduce la jeringa.
    \end{itemize}
    
    
\item  \textbf{Importancia}: no todos los pasos definidos en el procedimiento        
    oficial son necesarios de simular, por ejemplo, la colocación de los
    elementos cerca del lugar de trabajo, es un paso necesario en el
    procedimiento, pero es considerado un paso poco importante y fácil de
    realizar.

    La importancia fue evaluada por profesionales del \Gls{iab}, los
    cuales dieron su opinión acerca de cada aspecto simulado, el mismo
    es tenido en cuenta para determinar la importancia de cada
    acción.
    
    Los pasos del procedimiento de extracción de sangre que no se simularán por 
    no ser importantes son:
    \begin{itemize}
        \item Llevar el equipo en la unidad donde se encuentra el paciente.
        \item Extraer el protector de la aguja.
        \item Sellar la muestra y enviarlo a su destinatario.
    \end{itemize}
    
    
\item \textbf{Facilidad de realización en la vida real o en el laboratorio}:
    ciertos pasos son fáciles de realizar en la vida real pero requieren un
    esfuerzo significativo para ser simulados, como por ejemplo extraer el 
    protector de la aguja.

    La facilidad que tienen los alumnos con las acciones fue determinada por
    profesores del \Gls{iab}, determinaron qué acciones son fáciles de
    realizar para los alumnos y cuáles presentan mayores dificultades en su
    vida profesional.

    Otro aspecto que influye en la facilidad de realización de los
    procedimientos es la familiarización, si los alumnos están
    familiarizados con los procedimientos, estos no son simulados.
        
    Los pasos del procedimiento de extracción de sangre que no se simularán por 
    ser fáciles de realizar por los alumnos son:
        
    \begin{itemize}
        \item Preparar el equipo.
        \item Ubicar al paciente en posición adecuada.
    \end{itemize}
    
    y, los pasos del procedimiento de valoración de la escala de Glasgow 
    que no se simularán por la misma razón son:
    \begin{itemize}
    \item Preparar el material.
    \item Preparar al paciente.
    \item Colocar al paciente en posición cómoda.
    \end{itemize}
        
\end{itemize}

\subsubsection{Hipótesis}

Además de los pasos mencionados anteriormente, existen otros que son necesarios 
representar pero no necesariamente simular de forma detallada, para estos 
pasos se formularon hipótesis basadas en apreciaciones de 
los profesores del \Gls{iab} y en pruebas de usabilidad de interfaz las cuales 
son detalladas en el capítulo\ref{chap:evaluacion} y cuyos resultados se muestran 
en el capítulo\ref{chap:analisis}. A continuación se detallan cada una de estas 
hipótesis.

% Hipótesis
\begin{itemize}
\item 
    \textbf{Comandos de voz con interfaz}:  para enviar una petición o informarle 
    sobre algo al paciente (por ejemplo, darle detalles del procedimiento), 
    no es necesario identificar las palabras del usuario, sino más bien detectar
    que ha hablado y listar las posibles acciones que se pueden realizar.
    
    Los pasos del procedimiento de extracción de sangre que son representados por 
    comando de voz son:
    
    \begin{itemize}
        \item Explicar procedimiento.
        \item Solicitar al paciente que cierre el puño.
        \item Solicitar al paciente que abra el puño.
    \end{itemize}
    
    y, los pasos del procedimiento de valoración de la escala de Glasgow 
    que no se simularán por la misma razón son:
    \begin{itemize}
        \item Explicar el procedimiento.
        \item Medir la respuesta ocular por medio de peticiones al paciente.
        \item Medir la respuesta verbal por medio de preguntas al paciente.
        \item Medir la respuesta motora por medio de peticiones al paciente.
    \end{itemize}

\item
    \textbf{Extracción uniforme de elementos}: para realizar la acción de extraer 
    un elemento utilizado en el paciente, se considera que realizarlo de una sola 
    manera para todos los elementos convierte a la interfaz más intuitiva.

    %\textbf{Utilización de menú contextual}: para realizar una acción con los
    %    elementos, es suficiente con presionar el mismo y seleccionar una acción
    %    de una lista de opciones, no hace falta emular todas las posibles.
    
    Los pasos del procedimiento de extracción de sangre que cumplen con la extracción 
    uniforme son:
    
    \begin{itemize}
        \item Extraer torniquete.
        \item Extraer Jeringa.
    \end{itemize}
    
\item 
    \textbf{Acciones por menú}: acciones como el de generar un estímulo doloroso 
    al paciente tiene limitaciones técnicas para su simulación pero no pueden ser 
    omitidas debido a su gran importancia en el procedimiento. Por lo tanto, 
    acciones como estas son realizadas a través de menús contextuales.
    
    Los pasos del procedimiento de valoración de la escala de Glasgow son importantes 
    de simular pero poseen limitaciones técnicas son:
    \begin{itemize}
        \item Realizar estímulos dolorosos en diferentes partes del cuerpo.
    \end{itemize}
    
    

\item 
    \textbf{Acciones de bioseguridad}: la bioseguridad es un área muy amplia y 
    transversal a todos los procedimientos de enfermería por lo que se considera 
    que simular cada acción es complejo y que sólo basta con que el estudiante sepa 
    el momento en el que debe realizarse cada una de estas acciones y por lo mismo,
    es suficiente representarlas a través de opciones en la interfaz gráfica.
    
    Los pasos del procedimiento de extracción de sangre que son representados por 
    opciones son:
    %Las acciones de bioseguridad, se
    %    realizan a través de una opción en la interfaz gráfica.
    \begin{itemize}
        \item Asepsia de las manos.
        \item Vestirse con bata estéril.
        \item Vestirse con tapaboca estéril.
        \item Vestirse con gorro estéril.
        \item Calzar guantes.
        \item Extraer guantes.
        \item Extraer bata.
        \item Extraer tapaboca.
        \item Extraer Gorro
    \end{itemize}
    

\end{itemize}
% Simulados

\subsubsection{Pasos simulados con mayor detalle}

Los demás pasos no mencionados previamente deben ser simulados de forma 
integra ya que no se ven afectados por ninguna de las hipótesis y criterios descritos 
y por que la evaluación del paso depende de la elecciones por parte del usuario. 

Los pasos del procedimiento de extracción de sangre que cumplen con esto son:
\begin{itemize}
    \item Elegir zona a punzar.
    \item Colocar torniquete.
    \item Esterilizar zona de punción.
    \item Elegir la zona a puncionar.
    \item Punzar la zona con la aguja.
    \item Extraer muestra de sangre.
    \item Colocar algodón en zona de punción.
    \item Presionar zona punción con el algodón.
\end{itemize}

y, los pasos de valoración de la escala de Glasgow que también cumplen con esto son:
\begin{itemize}
    \item Registrar la puntuación de la respuesta ocular.
    \item Registrar la puntuación de la respuesta motora.
    \item Registrar la puntuación de la  respuesta verbal.
    \item Registrar el diagnóstico final.
\end{itemize}


%\subsection{Evaluación de glasgow}
%\label{sec:glasgow_hipotesis}

%\observacion{Esta descripción tiene otra estructura que la de extracción.
%    Unificar!!}
%\observacion{Resumir \emph{Mucho} más lo anterior y destacar/concertar más a la
%    solución}


%En la sección~\ref{sec:glasgow_protocolo} se definieron los pasos necesarios
%para llevar a cabo el procedimiento, este procedimiento es un paso rutinario que
%deben realizar los profesionales para poder determinar rápidamente el estado de
%conciencia de un individuo. 

%El paso central de la práctica es la valoración del paciente y la generación del
%diagnóstico, los demás pasos, no colaboran en el desarrollo de la competencia
%básica.

%Así, es suficiente con simular al paciente y las reacciones que tiene ante las
%acciones del jugador, las siguientes hipótesis se basan en la interacción
%paciente/jugador.

%Para simular la medición del estado del paciente, se toman las siguientes
%hipótesis:

%\begin{itemize}
%    \item La provocación de un estímulo doloroso al paciente es una acción
%        necesaria, pero no así los detalles de la misma, \emph{se simula el
%            estímulo con una opción} al presionar al paciente en alguna
%        extremidad.
%    \item El diálogo jugador/paciente se realiza a través de comandos de voz, el
%        nombre del paciente es una información que se conoce de ante mano y
%        \emph{Se simulan 7 posibles preguntas}, que incluyen solicitudes de
%        apertura ocular, movimiento de extremidades, y preguntas generales como
%        el nombre del paciente, el día y el lugar.
%\end{itemize}

%El último paso del procedimiento es el registro final de la valoración y el
%diagnóstico, el mismo es utilizado como mecanismo de evaluación y el registro en
%sí no se simula, es decir, se solicita al jugador que realice la valoración y el
%diagnóstico (mediante un menú), pero no en la condiciones que se realizan en la
%vida real (anotando en el registro médico del paciente).

% Criterios
%\subsection{Alcance de la simulación}
%
%Respecto a criterios de que se simula y que no:
%
%\begin{itemize}
%\item  \textbf{Limitaciones técnicas}: acciones como la simulación del agua        
%    (necesarios para el lavado de manos), requieren de requisitos de
%    hardware avanzados y un tiempo considerable de desarrollo. Las acciones
%    que escapan al alcance del hardware, del software o de tiempo de los
%    desarrolladores no son simuladas.
%        
%    Los elementos que no se simulan por liimtaciones técnicas son:
%    \begin{itemize}
%    \item Preparar el equipo
%    \item Llevar el equipo en la unidad donde se encuentra el paciente
%    \item Ubicar al paciente en posición adecuada
%    \item Extraer el protector de la aguja
%    \item Sellar la muestra y enviarlo a su destinatario
%    \end{itemize}
%\item  \textbf{Importancia}: no todos los pasos definidos en el procedimiento        
%    oficial son necesarios de simular, por ejemplo, la colocación de los
%    elementos cerca del lugar de trabajo, es un paso necesario en el
%    procedimiento, pero es considerado un paso poco importante y fácil de
%    realizar.
%
%    La importancia es evaluada por profesionales del \Gls{iab}, los
%    cuales dieron su opinión acerca de cada aspecto simulado, el mismo
%    es tenido en cuenta para determinar la importancia de cada
%    acción.
%    
%    Los elementos que no se simulan por no ser importantes son:
%    \begin{itemize}
%    \item Tensar la zona de punción
%    \item Angulo de punción
%    \item Presionar el brazo
%    \end{itemize}
%\end{itemize}
%
%Respecto a aspectos que son necesarios simular, pero no en su totalidad:
%
%% Hipótesis
%\begin{itemize}
%\item Hipótesis 1
%    \textbf{Comandos de voz con interfaz}:  para enviar una petición al        
%        paciente (por ejemplo, preguntarle su nombre), no es necesario
%        identificar las palabras del usuario, sino más bien detectar que ha
%        hablado y listar las posibles acciones que se pueden realizar.
%    \begin{itemize}
%    \item Explicar procedimiento
%    \item Solicitar al paciente que cierre el puño
%    \item Solicitar al paciente que abra el puño
%    \item Preguntas estado verbal
%    \item Solicitar apertura ocular
%    \end{itemize}
%\item Hipótesis 2
%
%    \textbf{Utilización de la interfaz}: para realizar una acción con los
%        elementos, es suficiente con presionar el mismo y seleccionar una acción
%        de una lista de opciones, no hace falta emular todas las posibles.
%    \begin{itemize}
%    \item Remover torniquete
%    \item Extraer Jeringa
%    \item Estimular parte del cuerpo
%    \end{itemize}
%\item Hipótesis 3
%    \textbf{Acciones de bioseguridad}: Las acciones de bioseguridad, se
%        realizan a través de una opción en la interfaz gráfica.
%    \begin{itemize}
%    \item Lavar manos
%    \item Poner Bata
%    \item Poner Tapaboca
%    \item Poner Gorro
%    \item Calzar Guantes
%    \item Extraer Guante
%    \item Extraer Bata
%    \item Extraer Tapaboca
%    \item Extraer Gorro
%    \end{itemize}
%\end{itemize}
%% Simulados
%
%Aspectos que deben ser simulados de manera íntegra:
%
%\begin{itemize}
%\item Elegir Zona a punzar
%\item Colocar torniquete
%\item Esterilizar zona de punción
%\item Lugar de punción
%\item Punzar
%\item Extraer muestra de sangre
%\item Colocar algodón en zona de punción
%\item Presionar zona punción
%\end{itemize}
