\section{Alcance de la simulación}
\label{sec:alcance}

\observacion{Agregar hipótesis de motivación (puntaje, facebook, etc)}
\observacion{Agregar hipótesis de iconos}
\observacion{Agregar hipótesis de Falta de pistas}

Para representar los procedimientos seleccionados y descritos en la
sección\ref{sec:seleccion_escenas}, estos deben ser presentados dentro de
escenarios distintos con los elementos necesarios para poder llevarlos a cabo.
Sin embargo, por limitaciones técnicas, tecnológicas y de tiempo, no es posible
realizar una simulación de todos los pasos requeridos.

\observacion{Más énfasis en el hecho que distrae a la parte pedagógica}
Según~\cite{videojuegos:gonzaleztardon}, existe un compromiso visible entre
realismo y credibilidad, tomando en cuenta que uno de los objetivos de la
solución es la presentación de situaciones simplificadas que permitan transmitir
conocimiento, mientras más realismo exista, más detalles existirán y por
consiguiente, los usuarios tendrán que concentrarse en un mayor número de
detalles, lo que resulta contraproducente con el objetivo de la
solución\cite{videojuegos:gonzaleztardon}. Los criterios e hipótesis que se 
describirán a continuación sirven para acotar el
alcance de la simulación, definen qué se simulará y cual es del detalle
necesario para alcanzar las competencias básicas.

\fixme{A continuación se fundamentan por qué ciertos pasos serán simulados,
    representados o no simulados dentro de la solución basados en limitaciones e
    hipótesis asumidas.}{Mover esto abajo de factores limitantes}

Cada uno de los pasos están agrupados de acuerdo a cada uno de estos aspectos.

\subsection{Factores limitantes}

Se refieren a cada uno de los aspectos que determinan si un paso 
será simulado dentro de la solución. Se clasifican en tres: limitaciones técnicas, 
importancia y facilidad de realización. Estos tres aspectos que influyen en qué partes 
se simularán y qué partes se omitirán, se detallan a continuación.

\observacion{No empezar con un ejemplo.}
\begin{itemize}
\item  \textbf{Limitaciones técnicas}: 
    
    Acciones como la simulación del agua (necesarios para el lavado de manos),
    requieren de requisitos de hardware avanzados y un tiempo considerable de
    desarrollo. Las acciones que escapan al alcance del hardware, del software o
    de tiempo de los desarrolladores no son simuladas. 
    \observacion{Reformular capítulo anterior}
        
    Los pasos del procedimiento de extracción de sangre que no se simularán por 
    limitaciones técnicas son:
    \begin{itemize}
        \item Tensar la zona de punción.
        \item Ángulo de punción.
        \item Presionar el brazo en el momento en el que se introduce la jeringa.
    \end{itemize}
    
    
\item  \fixme{Importancia}{de qué?} 

    No todos los pasos definidos en el procedimiento        
    oficial son \fixme{necesarios}{mandatorios} de simular, por ejemplo, la
    colocación de los elementos cerca del lugar de trabajo, es un paso necesario
    en el procedimiento, pero es considerado un paso poco importante y fácil de
    realizar.

    La importancia es evaluada por profesionales del \Gls{iab}, los
    cuales dieron su apreciación acerca de cada aspecto simulado, el mismo
    es tenido en cuenta para determinar la importancia de cada
    acción.
    
    Los pasos del procedimiento de extracción de sangre que no se simularán por 
    no ser importantes para la simulación son:
    \begin{itemize}
        \item Llevar el equipo en la unidad donde se encuentra el paciente.
        \item Extraer el protector de la aguja.
        \item Sellar la muestra y enviarlo a su destinatario.
    \end{itemize}
    
    
\item \textbf{Facilidad de realización en la vida real o en el laboratorio}:
    \observacion{Reformular título}

    Ciertos pasos son fáciles de realizar en la vida real pero requieren un
    esfuerzo significativo para ser simulados con realismo, como preparar el
    equipo necesario.

    La facilidad que tienen los alumnos con las acciones fue determinada por
    profesores del \Gls{iab}, determinaron qué acciones son fáciles de
    realizar para los alumnos y cuáles presentan mayores dificultades en su
    vida profesional.

    Otro aspecto que influye en la facilidad de realización de los
    procedimientos es la familiarización, si los alumnos están
    familiarizados con los procedimientos, estos no son simulados.
        
    Los pasos del procedimiento de extracción de sangre que no se simularán por 
    ser fáciles de realizar por los alumnos son:
        
    \begin{itemize}
        \item Preparar el equipo.
        \item Ubicar al paciente en posición adecuada.
    \end{itemize}
    
    Los pasos del procedimiento de valoración de la escala de Glasgow 
    que no se simularán por la misma razón son:
    \begin{itemize}
    \item Preparar el material.
    \item Preparar al paciente.
    \item Colocar al paciente en posición cómoda.
    \end{itemize}
        
\end{itemize}

\subsection{Hipótesis}
\label{sec:hipotesis}
\observacion{No utilizar: existen, que debería, no son}

\fixme{Además de los pasos mencionados anteriormente, existen otros pasos que
    son necesarios representar pero no necesariamente simular de forma
    detallada,}{Mejorar} para estos pasos se formularon hipótesis basadas en
apreciaciones de los profesores del \Gls{iab} y en pruebas de usabilidad de
interfaz las cuales son detalladas en el capítulo\ref{chap:evaluacion} y cuyos
resultados se muestran en el capítulo\ref{chap:analisis}. A continuación se
detallan cada una de estas hipótesis.

% Hipótesis
\begin{itemize}

\item \textbf{Comandos de voz con interfaz}:  para enviar una petición o
    informarle sobre algo al paciente (por ejemplo, darle detalles del
    procedimiento), no es necesario identificar las palabras del usuario, sino
    más bien detectar que ha hablado y listar las posibles acciones que se
    pueden realizar.
    
    Los pasos del procedimiento de extracción de sangre que son representados por 
    comando de voz son:
    
    \begin{itemize}
        \item Explicar procedimiento.
        \item Solicitar al paciente que cierre el puño.
        \item Solicitar al paciente que abra el puño.
    \end{itemize}
    
    y, los pasos del procedimiento de valoración de la escala de Glasgow 
    que son representados por la misma razón son:
    \begin{itemize}
        \item Explicar el procedimiento.
        \item Medir la respuesta ocular por medio de peticiones al paciente.
        \item Medir la respuesta verbal por medio de preguntas al paciente.
        \item Medir la respuesta motora por medio de peticiones al paciente.
    \end{itemize}

\item \textbf{Extracción uniforme de elementos}: para realizar la acción de
    extraer un elemento utilizado en el paciente, se considera que realizarlo de
    una sola manera para todos los elementos convierte a la interfaz más
    intuitiva.

    Los pasos del procedimiento de extracción de sangre que cumplen con la extracción 
    uniforme son:
    
    \begin{itemize}
        \item Extraer torniquete.
        \item Extraer jeringa.
    \end{itemize}
    
\item \textbf{Acciones por interfaz de usuario}: acciones como generar un
    estímulo doloroso al paciente tienen limitaciones técnicas para su
    simulación pero no pueden ser omitidas debido a su gran importancia en el
    procedimiento.
    
    \observacion{Hacer que esto sea una decisión de diseño y no una hipótesis}
    \observacion{No se debería hablar de soluciones aún}
    Los pasos del procedimiento de valoración de la escala de Glasgow que son
    importantes de simular pero poseen limitaciones técnicas son:
    \begin{itemize} 
    \item Realizar estímulos dolorosos en diferentes partes del cuerpo. 
    \end{itemize}
    
\item 
    \textbf{Acciones de bioseguridad}: la bioseguridad, que es un aspecto
    fundamental y transversal a todo procedimiento de enfermería. Es un área muy
    amplia y transversal a todos los procedimientos de enfermería por lo que se
    considera que simular cada acción es complejo y que sólo basta con que el
    estudiante sepa el momento en el que debe realizarse cada una de estas
    acciones y por lo mismo, es suficiente representarlas a través de opciones
    en la interfaz gráfica.

    \revisar{Mirar resumen, esta mejor descrito ahí}
    
    Los pasos del procedimiento de extracción de sangre que son representados
    por opciones son:
    \begin{itemize}
        \item Asepsia de las manos.
        \item Vestirse con bata estéril, tapaboca estéril y gorro estéril.
        \item Calzar guantes.
        \item Extraer guantes, bata, tapaboca y gorro.
    \end{itemize}
    
\item
    \textbf{Representación iconográfica}: para representar el estado del
    enfermero es suficiente mostrar una imagen representativa. Es decir, que
    para mostrar que el enfermero tiene una gorra, es suficiente mostrar una
    imagen de una gorra.
    
    Esta hipótesis se aplica para las opciones de bioseguridad, las opciones
    para la utilización de elementos así como para la representación del
    elemento seleccionado.
\item 
    \textbf{Factores motivantes}: indicadores de rendimiento, como un puntaje al
    final de cada procedimiento, y el tiempo total dentro del procedimiento,
    impactan positivamente en el involucramiento del usuario.

    La interacción social es otro factor que incrementa el nivel de compromiso
    del usuario.

\item 
    \textbf{Falta de pistas}: la ausencia de signos visuales que indiquen las
    acciones que debe realizar el usuario durante la experiencia, permite al
    usuario probar sus conocimientos, e impide al usuario avanzar en el
    procedimiento utilizando una técnica de \emph{prueba y error}.

\end{itemize}
% Simulados

\subsection{Pasos de exploración}

Los demás pasos no mencionados previamente deben ser simulados de forma 
integra ya que no se ven afectados por ninguna de las hipótesis y criterios previamente
descriptos.

La evaluación de estos pasos depende de elecciones que haga el usuario, por ejemplo
la elección de una zona de punción depende del usuario, existen varios lugares posibles, 
de los cuales solo algunos son válidos. 

Los pasos del procedimiento de extracción de sangre que cumplen con esto son:
\begin{itemize}
    \item Elegir zona a punzar.
    \item Colocar torniquete.
    \item Esterilizar zona de punción.
    \item Elegir la zona a puncionar.
    \item Punzar la zona con la aguja.
    \item Extraer muestra de sangre.
    \item Colocar algodón en zona de punción.
    \item Presionar zona punción con el algodón.
\end{itemize}

Los pasos de valoración de la escala de Glasgow que también cumplen con esto son:
\begin{itemize}
    \item Registrar la puntuación de la respuesta ocular.
    \item Registrar la puntuación de la respuesta motora.
    \item Registrar la puntuación de la  respuesta verbal.
    \item Registrar el diagnóstico final.
\end{itemize}


