\section{Hipótesis de la simulación}
\label{sec:hipotesis}

\observacion{Ir más al grano y decir que se debe encontrar un punto de
    equilibrio entre simplicidad y detalle, en la simulación, pero esa es una
    tesis de por sí, entonces ustedes van a basarse en otros trabajos y sentido
    común}

Las escenas seleccionadas y definidas en la sección~\ref{sec:seleccion_escenas}
representan las acciones que deben realizar los profesionales de enfermería a la
hora de realizar los procedimientos seleccionados, por limitaciones técnicas,
tecnológicas y de tiempo, no es posible realizar una simulación de todos los
pasos requeridos.

\observacion{Busquen algún articulo sobre simulación ver simulación (NDE: hace
    regencia errores de referencias (las que apuntan a tardon))}

Según~\cite{videojuegos:gonzaleztardon}, existe un compromiso visible entre
realismo y credibilidad, tomando en cuenta que uno de los objetivos de la
solución es la presentación de situaciones simplificadas que permitan transmitir
conocimiento, mientras más realismo exista, más detalles existirán y por
consiguiente, los usuarios tendrán que concentrarse en un mayor número de
detalles, lo que resulta contraproducente con el objetivo de la
solución\cite{videojuegos:gonzaleztardon}. Estas hipótesis sirven para acotar el
alcance de la simulación, definen qué se simulará y cual es del detalle
necesario para alcanzar las competencias básicas.

Los factores que influyen en que partes se simularán, que partes estarán
presentes solamente a través de opciones y que partes se omitirán son:

\begin{itemize}

    \item \textbf{Limitaciones técnicas}: acciones como la simulación del agua
        (necesarios para el lavado de manos), requieren de requisitos de
        hardware avanzados y un tiempo considerable de desarrollo. Las acciones
        que escapan al alcance del hardware, del software o de tiempo de los
        desarrolladores no son simuladas.

    \item \textbf{Importancia}: no todos los pasos definidos en el procedimiento
        oficial son necesarios de simular, por ejemplo, la colocación de los
        elementos cerca del lugar de trabajo, es un paso necesario en el
        procedimiento, pero es considerado un paso poco importante y fácil de
        realizar.

        \fixme{La importancia es evaluada por profesionales del \Gls{iab}, los
            cuales dieron su opinión acerca de cada aspecto simulado, el mismo
            es tenido en cuenta para determinar la importancia de cada
            acción}{Usar esto en vez de sentido común. (Ver observación al
            inicio del archivo)}.

    \item \textbf{Facilidad de realización en la vida real o en el laboratorio}:
        ciertos pasos son fáciles de realizar en la vida real pero requieren un
        esfuerzo significativo para ser simulados, como por ejemplo el lavado de
        manos es un procedimiento al que los alumnos están acostumbrados.

        La facilidad que tienen los alumnos con las acciones fue determinada por
        profesores del \Gls{iab}, determinaron que acciones son fáciles de
        realizar para los alumnos y cuales presentan mayores dificultades en su
        vida profesional.

        Otro aspecto que influye en la facilidad de realización de los
        procedimientos es la familiarización, si los alumnos están
        familiarizados con los procedimientos, estos no son simulados.

\end{itemize}
\fixme{Existen}{No utilizar tanto} hipótesis que son globales para toda la simulación, las mismas son:

\begin{itemize}

    \item \textbf{Comandos de voz con interfaz}: para enviar una petición al
        paciente (por ejemplo, preguntarle su nombre), no es necesario
        identificar las palabras del usuario, sino más bien detectar que ha
        hablado y listar las posibles acciones que se pueden realizar.

    \item \textbf{Utilización de la interfaz}: para realizar una acción con los
        elementos, es suficiente con presionar el mismo y seleccionar una acción
        de una lista de opciones, no hace falta emular todas las posibles.

    \item \textbf{Acciones de bioseguridad}: Las acciones de bioseguridad, se
        realizan a través de una opción en la interfaz gráfica.

\end{itemize}

\observacion{Revisar el orden de las cosas, explicar los escenarios primero y
    después la arquitectura, las alternativas, y rel resultado final (Su
    solución)}

Otras hipótesis, son tomadas por escena, las dos escenas simuladas son
diferentes en el modo de interacción del usuario con su entorno, por ejemplo, en
la escena \enquote{Extracción de sangre}, el usuario interactúa con el paciente
a través de objetos, en la \enquote{Evaluación de Glasgow}, la interacción con
el paciente es directa.

\subsection{Extracción de sangre}
\label{sec:hemocultivo_hipotesis}
\observacion{Esta un poco desconectado}

Se presentan los pasos mostrados en la sección~\ref{sec:hemocultivo_protocolo},
y adicionalmente se establecen las hipótesis punto por punto y las
consideraciones que deben ser tomadas.

\observacion{Resumir más}

\observacion{No termina de conectar esto, uno llega a esta parte y no entiende
    por que se esta hablando de por que no. Hay que vestir con la bata; No se
    entiende como se llego a eso}
\observacion{Ver donde poner esto para que no quede desconectado}

% AGRUPAR POR SIMILITUD Y NO POR PASO
\begin{itemize}

\item \textbf{Preparar el equipo}: la preparación del equipo es un aspecto muy
    importante del procedimiento, pero no es un punto único de la extracción de
    sangre, además las prácticas de los alumnos cubren completamente este paso
    \fixme{según comentarios de los profesores}{}. \emph{Este paso no se simula}.

\item \textbf{Explicación al paciente del procedimiento a realizar}: es un
    aspecto importante del procedimiento, pero la simulación de una conversación
    alumno-paciente es compleja, según comentarios de los profesores, es
    suficiente con que los alumnos sepan que lo deben realizar y en que momento,
    no es necesario simular la conversación en sí. \fixme{Este paso se simula a
        través de un comando de voz con la interfaz}{Hace falta describir
        implementación ya?}.

\item \textbf{Asepsia de las manos}: este paso forma parte de un área más amplia
    conocida como bioseguridad, la cual es un aspecto transversal a todos los
    procedimientos realizados por los enfermeros. La implementación de una
    simulación del lavado de mano es compleja, y es un aspecto que, al igual que
    la preparación del equipo, está cubierta por los laboratorios, aún así, es
    necesario que los alumnos sepan en que momento deben realizar la asepsia de
    sus manos. \emph{Se simula este paso a través de una opción en la interfaz},
    no se simulan los pormenores del lavado de manos.

\item \textbf{Llevar el equipo a la unidad en donde se encuentra el paciente}:
    según opinión de los profesionales del \Gls{iab}, los alumnos y egresados no
    encuentran problemas al realizar este paso, además es un paso que no aporta
    a las competencias básicas deseadas. \emph{Este paso no se simula}.

\item \textbf{Vestirse con bata estéril, tapaboca y gorro}: al igual que la
    asepsia de las manos, es importante que los alumnos sepan que lo deben
    hacer, pero no es importante que se simule como lo hacen. 

    Los estudiantes están familiarizados con estas acciones, \emph{se simula el
        momento en el que el jugador lo hace} a través de una opción
    en la interfaz, no se simula el proceso en sí.

\item \textbf{Calzarse los guantes}: es un paso relacionado a la bioseguridad,
    es importante que se sepa en que momento debe realizarse, pero no es
    necesario simular el proceso. \emph{Se simula el momento en el
        que se realiza}, no se simulan los pormenores de la acción.

\item \textbf{Ubicar al paciente en posición adecuada}: la ubicación del
    paciente durante la extracción de sangre es un factor determinante para que
    la extracción pueda ser realizada correctamente.

    Los alumnos están familiarizados con este proceso según opinión de los
    profesionales, \emph{este paso no se simula}, el paciente está en la
    posición adecuada al inicio de la simulación.

\item \textbf{Elegir la zona a puncionar}: existen varias partes del antebrazo
    donde se puede proceder a realizar una punción, el conocimiento de las
    mismas, y el procedimiento para detectarlas, es un factor importante del
    procedimiento.
    
    Las venas del cuerpo humano se detectan palpando los antebrazos, y sintiendo
    el pulso del paciente, existen dos áreas donde el pulso es suficientemente
    fuerte como para ser sentido, estos puntos y el pulso del paciente deben ser
    detectables por el jugador.

    \emph{Los puntos donde se debe punzar deben ser identificables en la
        simulación}. 

\item \textbf{Colocación del torniquete}: La ubicación y el momento de la
    colocación del torniquete es de vital importancia para el procedimiento, el
    mecanismo utilizado para colocarlo no es relevante, pues el mismo es fácil
    de realizar.

    \emph{El hecho de colocar el torniquete es simulado}, el mecanismo para
    hacerlo no es importante.

\item \textbf{Solicitar al paciente que cierre el puño}: El momento en el cual
    se solicita al paciente que cierre la mano es vital para que el
    procedimiento de extracción sea satisfactorio.

    \emph{Este paso es simulado} a través de un comando de voz.

\item \textbf{Esterilizar la zona de punción}: la esterilización de la zona de
    punción es un factor de suma importancia para el procedimiento, así como el
    momento en el que se realiza, el jugador debería poder esterilizar la zona
    antes de insertar la jeringa, no se simula el hecho de preparar los
    elementos necesarios para la esterilización, pero sí \emph{se simula, la
        utilización de estos elementos}.
    
\item \textbf{Extraer el protector de la aguja}: La extracción del protector de
    la aguja es un paso necesario, pero fácil de realizar, el hecho de retirar
    el protector de la jeringa \emph{no es un paso necesario para el logro de
        las competencias básicas necesarias}, por ello, no se simula.

\item \textbf{Puncionar la piel con la aguja}: este es un paso central en el
    procedimiento, en el se deben tener en cuenta aspectos como la posición
    donde se realiza la punción, y el ángulo con el que ingresa la aguja.

    La posición donde se realiza la punción es importante por que depende de la
    ubicación donde se colocó el torniquete, y debe ser en uno de los puntos del
    brazo donde existen venas capaces de soportar el procedimiento, \emph{en
        cada brazo existen dos puntos donde se puede inyectar}.

    En cuanto al ángulo de punción, es un conocimiento importante que deben
    tener los alumnos, el conocimiento es teórico y según comentarios de los
    profesores, es un tema en el cual los alumnos tienen suficiente práctica en
    el laboratorio, \emph{No se simula el ángulo en el cual se inserta la
        jeringa}, es decir, la jeringa siempre se inserta en el mismo ángulo.

\item \textbf{Tensar la zona de punción}: el proceso de tensar la zona de
    punción se realiza durante la inserción de la jeringa, el mismo es fácil de
    realizar, y para simularlo se requiere que el usuario utilice tres dedos al
    mismo tiempo (dos para tensar y otro para realizar la punción), lo cual
    dificulta la utilización de la solución.

    \emph{Este paso no se simula} por la dificultad técnica que implica utilizar
    tres dedos para realizar una tarea, conjuntamente con la facilidad con que
    se realiza acción. 

\item \textbf{Remover el torniquete}: Al igual que en la colocación del
    torniquete, \emph{se simula el momento} de la extracción por que es
    importante, pero no los detalles de la extracción.

\item \textbf{Solicitar la apertura del puño}: el momento exacto donde se debe
    solicitar al paciente que abra la mano es fundamental para la realización
    correcta de la simulación. \emph{Este paso se simula} a través de un comando
    de voz.

\item \textbf{Extraer la muestra de sangre necesaria}: este es el paso central
    de la práctica, tanto el momento, como la forma es importante simular.

    En la extracción de la sangre \emph{se simula} la acción de la extracción,
    pero no detalles como la sangre extraída, la velocidad de extracción y la
    fuerza necesaria por limitaciones de la tecnología.

\item \textbf{Presionar el brazo y extraer la aguja}: la presión del brazo para
    extraer la jeringa es un paso fácil de realizar, en cambio el momento en el
    que se extrae la aguja es un conocimiento necesario para el procedimiento.

    \emph{No se simula esta acción}, pues es un paso al que los alumnos están
    acostumbrados y de simularse, agrega una complejidad adicional a la
    extracción, que es un paso que se realiza al mismo tiempo.

\item \textbf{Colocar algodón con alcohol en el punto de punción}: este paso es
    importante, tanto el momento en el cual se debe realizar, como la forma de
    realizarlo.

    \emph{Se simula la colocación del algodón}, así como el tiempo que se debe
    presionar el mismo.

\item \textbf{Sellar la muestra y enviarlo a su destinatario}: es necesario que
    los alumnos sepan que este paso debe ser realizado, pero los detalles del
    mismo, no son necesarios para el logro de las competencias básicas,
    \emph{este paso no se simula}.


\item \textbf{Retirar la bata, tapaboca, gorro y guantes}: es necesario que los
    alumnos sepan que deben desechar todos los elementos que fueron utilizados
    durante el proceso y en que orden, la forma de hacerlo no es necesaria.

    \emph{Se simula el momento en el que se extraen los elementos} a través de
    una opción en la interfaz.

\item \textbf{Asepsia de las manos}: la asepsia final de las manos es un paso
    necesario para el procedimiento, así como la asepsia inicial, es importante
    que los alumnos sepan el momento en cual deben realizarlo, \emph{este paso
        se simula} a través de una opción en la interfaz.

\end{itemize}

Estas hipótesis afectan directamente el desarrollo de la aplicación, dictando
que partes del procedimiento se simulan y como, pueden ser vistas como
requisitos funcionales de la solución.

\subsection{Evaluación de glasgow}
\label{sec:glasgow_hipotesis}

\observacion{Esta descripción tiene otra estructura que la de extracción.
    Unificar!!}
\observacion{Resumir \emph{Mucho} más lo anterior y destacar/concertar más a la
    solución}

En la sección~\ref{sec:glasgow_protocolo} se definieron los pasos necesarios
para llevar a cabo el procedimiento, este procedimiento es un paso rutinario que
deben realizar los profesionales para poder determinar rápidamente el estado de
conciencia de un individuo. 

El paso central de la práctica es la valoración del paciente y la generación del
diagnóstico, los demás pasos, no colaboran en el desarrollo de la competencia
básica.

Así, es suficiente con simular al paciente y las reacciones que tiene ante las
acciones del jugador, las siguientes hipótesis se basan en la interacción
paciente/jugador.

Para simular la medición del estado del paciente, se toman las siguientes
hipótesis:

\begin{itemize}
    \item La provocación de un estímulo doloroso al paciente es una acción
        necesaria, pero no así los detalles de la misma, \emph{se simula el
            estímulo con una opción} al presionar al paciente en alguna
        extremidad.
    \item El diálogo jugador/paciente se realiza a través de comandos de voz, el
        nombre del paciente es una información que se conoce de ante mano y
        \emph{Se simulan 7 posibles preguntas}, que incluyen solicitudes de
        apertura ocular, movimiento de extremidades, y preguntas generales como
        el nombre del paciente, el día y el lugar.
\end{itemize}

El último paso del procedimiento es el registro final de la valoración y el
diagnóstico, el mismo es utilizado como mecanismo de evaluación y el registro en
sí no se simula, es decir, se solicita al jugador que realice la valoración y el
diagnóstico (mediante un menú), pero no en la condiciones que se realizan en la
vida real (anotando en el registro médico del paciente).

