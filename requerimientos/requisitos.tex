%! TEX root = ../main.tex
\section{Requisitos de la solución}
\label{sec:requisitos}

A fin de que la solución propuesta pueda cumplir con las competencias básicas
definidas por el plan de estudio, se definen varios tipos de requisitos.

Las condiciones que afectan a la solución como una herramienta, son:

\observacion{El jugar es un termino que tiene sentido solo cuando se explica que
    la solución es un juego}
\begin{itemize}
    \item Se debe permitir al \fixme{jugador}{Usuario} poder utilizar el entorno virtual en el
    momento y lugar que desee es decir, es decir, se le debe proveer ubicuidad.
\item Las escenas presentadas, junto con lo elementos dentro de ellas deben ser
    representados en tres dimensiones y deben ser lo mas realista
    posible\revisar{Debería ser abreves, que el escenario sea realista, osea el
        realismo es el requisito y que sea en 3D es una consecuencia}.
\item Cada escena representará un procedimiento de enfermería que debe ser
    realizado por el jugador.
\item El entorno debe permitir al jugador decidir libremente las acciones que
    quiere realizar, favoreciendo la exploración del entorno.
\item El entorno no debe brindar pistas al jugador acerca de la forma en la que
    se deben realizar las procedimientos.
\item El entorno debe brindar al jugador información final acerca de su
    desempeño en la escena.
\item La vision del jugador deberá poder se manipulada en tres dimensiones,
    permitiendo acercar, alejar, mover y rotar la vision para poder observar el
    entorno sin limitaciones.
\item El final de una partida debe indicarse mediante un botón en un menu
    ubicado en la pantalla principal de la escena.
\end{itemize}

\observacion{Esto hay que separar entre requisitos generales (ver +)}
\observacion{No debería ir en otro capítulo?}

Con respecto a la interacción del jugador con los diferentes elementos que
pueden ser utilizados dentro de la simulación, denominados objetos, son:

\begin{itemize}
\item La selección de objetos debe ser homogénea, y los mismos pueden ser
    accedidos en cualquier momento, representando la mesa de elementos que
    poseen los enfermeros durante la práctica profesional.
\item Debe ser claro que objeto actualmente esta en uso.
\item Debe ser posible simular la selección y des-selección de objetos.
\item Cada objeto disponible durante la simulación debe ser utilizado de
    manera independiente de los demás, y solamente un objeto a la vez.
\end{itemize}

La interacción entre el jugador y el paciente se toman los siguientes
requisitos:

\begin{itemize}
\item El jugador puede manipular el estado del paciente a través de objetos, la
    utilización de los mismos debe ser uniforme e intuitiva.
\item Las acciones sobre los objetos no deben ser muy detalladas, no deben
    distraer al jugador del objetivo principal de la simulación.
\item Todas las acciones sobre objetos que no son relevantes para el objetivo de
    la simulación, no deben ser simuladas, pero sí deben ser realizadas (por
    ejemplo puede existir una opción que indique la realización de cierta
    acción, pero la acción en sí no se simula).
\end{itemize}

En cuanto a las acciones que realiza el jugador mientras utiliza la simulación,
se consideran los siguientes requisitos:

\begin{itemize}
\item Las acciones realizadas por los jugadores en el entorno virtual deben ser
    registradas.
\item Cada acción realizada por el usuario debe ser validada por el entorno, de
    forma a ofrecer información correcta acerca de sus logros al final de la
    partida.
\item La validez de una acción puede requerir en algunos casos que se hayan
    hecho algunas acciones previas, pueden requerir un orden definido, o pueden
    depender del entorno en el cual se realizan las mismas.
\item No se deben dar pistas o mensajes al jugador durante la partida cuando
    este realice incorrectamente una acción.
\end{itemize}
