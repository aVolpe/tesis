%! TEX root = ../main.tex
\section{Requisitos de la solución}
\label{sec:requisitos}

En esta sección se definen los requisitos necesarios para que la solución propuesta pueda 
cumplir con las competencias básicas definidas por el plan de estudio.

\subsection{Requisitos generales}

Las condiciones generales que afectan a la solución como una herramienta, son:

\begin{itemize}
\item Se debe permitir al usuario poder utilizar el entorno virtual en el
    momento y lugar que desee, es decir, se le debe proveer movilidad.

\item Los escenarios presentados, junto con los elementos que los componen,
    deben ser lo suficientemente realistas para provocar un sentido de
    inmersión al usuario.

\item Cada escena debe representar un procedimiento de enfermería que debe ser
    realizado por el usuario.

\item Se debe permitir al usuario decidir libremente las acciones que
    quiere realizar, favoreciendo la exploración del entorno.

\item No se debe brindar informaciones al usuario durante los procedimientos acerca de la 
	forma en la que deben realizarlos.

\item Se debe brindar al usuario información acerca de su desempeño en
    la escena al final de la misma.
    
\item Las acciones realizadas por los usuario en el entorno virtual deben ser
    registradas.

\end{itemize}

\subsection{Requisitos de interacción}

Los requisitos con respecto a la interacción del usuario con el entorno y los diferentes 
elementos que lo componen, son:

%Con respecto a la interacción del usuario con el entorno y los diferentes elementos que
%pueden ser utilizados dentro de la simulación, denominados objetos, son:

\begin{itemize}

\item La selección de elementos debe ser homogénea, y los mismos deben poder ser
    accedidos en cualquier momento, representando la mesa de elementos que
    poseen los enfermeros durante la práctica profesional.

\item Debe ser claro que elemento actualmente está en uso.

\item Debe ser posible simular la selección y des-selección de elementos.

\item Cada elemento disponible durante la simulación debe ser utilizado de
    manera independiente de los demás, y solamente un elemento a la vez.

\item El usuario debe poder manipular el estado del paciente a través de elementos.

\item Las acciones sobre los elementos no deben ser muy detalladas, no deben
    distraer al usuario del objetivo principal de la simulación.

\item La visión del usuario debe poder ser manipulada en tres dimensiones,
    permitiendo acercar, alejar y mover la visión para poder observar el
    entorno.

\end{itemize}

\subsection{Requisitos de evaluación al usuario}

En cuanto a las acciones que realiza el usuario mientras utiliza la simulación,
se consideran los siguientes requisitos para su validación:

\begin{itemize}


\item Cada acción realizada por el usuario debe ser validada por el entorno, de
    forma a ofrecer información precisa acerca de sus logros al final de la
    partida.

\item La validez de una acción puede requerir en algunos casos que se hayan
    hecho algunas acciones previas, pueden requerir un orden definido, o pueden
    depender del entorno en el cual se realizan las mismas.

\item No se deben dar indicios o mensajes al usuario durante la partida cuando
    realice incorrectamente una acción.

\end{itemize}

