%! TEX root = ../main.tex
\section{Requisitos de la solución}
\label{sec:requisitos}

A fin de que la solución propuesta pueda cumplir con las competencias básicas
definidas por el plan de estudio, se definen varios tipos de requisitos.

Las condiciones generales que afectan a la solución como una herramienta, son:

\begin{itemize}
\item Se debe permitir al usuario poder utilizar el entorno virtual en el
    momento y lugar que desee es decir, es decir, se le debe proveer ubicuidad.

\item Los escenarios presentados, junto con los elementos que los componen,
    deberían ser lo suficientemente realistas para provocar un sentido de
    inmersión al usuario.

\item Cada escena representará un procedimiento de enfermería que debe ser
    realizado por el usuario.

\item El entorno debe permitir al usuario decidir libremente las acciones que
    quiere realizar, favoreciendo la exploración del entorno.

\item El entorno no debe brindar pistas al usuario acerca de la forma en la que
    se deben realizar las procedimientos.

\item El entorno debe brindar al usuario información final acerca de su
    desempeño en la escena.


\end{itemize}

Con respecto a la interacción del usuario con los diferentes elementos que
pueden ser utilizados dentro de la simulación, denominados objetos, son:

\begin{itemize}

\item La selección de objetos debe ser homogénea, y los mismos pueden ser
    accedidos en cualquier momento, representando la mesa de elementos que
    poseen los enfermeros durante la práctica profesional.

\item Debe ser claro que objeto actualmente esta en uso.

\item Debe ser posible simular la selección y des-selección de objetos.

\item Cada objeto disponible durante la simulación debe ser utilizado de
    manera independiente de los demás, y solamente un objeto a la vez.

\end{itemize}

Para la interacción entre el usuario, el paciente y el entorno se toman los
siguientes requisitos:

\begin{itemize}
\item El usuario puede manipular el estado del paciente a través de objetos, la
    utilización de los mismos debe ser uniforme e intuitiva.

\item Las acciones sobre los objetos no deben ser muy detalladas, no deben
    distraer al usuario del objetivo principal de la simulación.

\item Todas las acciones sobre objetos que no son relevantes para el objetivo de
    la simulación, no deben ser simuladas, pero sí deben ser realizadas (por
    ejemplo puede existir una opción que indique la realización de cierta
    acción, pero la acción en sí no se simula).

\item La vision del usuario deberá poder se manipulada en tres dimensiones,
    permitiendo acercar, alejar, mover y rotar la vision para poder observar el
    entorno sin limitaciones.

\end{itemize}

En cuanto a las acciones que realiza el usuario mientras utiliza la simulación,
se consideran los siguientes requisitos:

\begin{itemize}
\item Las acciones realizadas por los usuario es en el entorno virtual deben ser
    registradas.

\item Cada acción realizada por el usuario debe ser validada por el entorno, de
    forma a ofrecer información correcta acerca de sus logros al final de la
    partida.

\item La validez de una acción puede requerir en algunos casos que se hayan
    hecho algunas acciones previas, pueden requerir un orden definido, o pueden
    depender del entorno en el cual se realizan las mismas.

\item No se deben dar pistas o mensajes al usuario durante la partida cuando
    este realice incorrectamente una acción.

\item El final de una partida debe indicarse mediante un botón en un menu
    ubicado en la pantalla principal de la escena.

\end{itemize}
