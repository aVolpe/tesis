\chapter{Requerimientos}
\label{chap:requerimientos}

\observacion{Unificar juego y videojuego}

Realizada la propuesta de aplicación en la que se centra este trabajo de grado,
se describe los requerimientos necesarios para que la misma cumpla su objetivo
pedagógico.

Esta capítulo además se centra en los objetivos pedagógicos que se desean
obtener, el cual es un paso necesario para poder realizar el flujo de desarrollo
descrito en~\ref{sec:desarrollo}.

Primeramente se definen los criterios para definir cuales procedimientos pueden
ser simulados, luego se describen los procedimientos seleccionados, con los
detalles de los mismos y las competencias básicas que están definidas en su plan
de estudio. Se describe el aspecto de la bioseguridad, que es un aspecto
fundamental de todo procedimiento de enfermería.

A fin de determinar el alcance de la solución, se describen hipótesis acerca de
lo que es necesario incluir, cuales detalles deben ser cubiertos por completo,
cuales son necesarios sin mucho detalle y cuales pueden ser dejados de lado. 

El último aspecto tratado en este capítulo es el de los requisitos transversales
que debe cumplir la solución, como aspectos de jugabilidad, como manipulación de
la cámara, interfaz, etc, así como aspectos pedagógicos, como la
retroalimentación, muestra de información y progreso.

%! TEX root = ../main.tex
\section{Definición de criterios}
\label{sec:criterios}

Se realizaron varias reuniones con la coordinadora de la carrera de Licenciatura
en Enfermería del \Gls{iab} y profesores encargados de las prácticas de
laboratorio y campo de diversas materias de la carrera. De estas reuniones, y
del plan de estudios de la carrera de enfermería, se extraen procedimientos que
pueden ser simulados, cuya aceptación y validación son realizados por  
profesores del área de enfermería del \Gls{iab}.

De los procedimientos evaluados y discutidos se eligen dos ya que la variedad y
cantidad de procedimientos de enfermería existentes es muy alta e incluirlos
todos a la solución está fuera del alcance de este trabajo. Los principales
criterios para la selección de los procedimientos son los siguientes:

\begin{itemize}

    \item \textbf{Deben adecuarse a las limitaciones de la tecnología} 

    Escenarios que requieren la simulación de órganos internos, psicología
    humana, etc, pueden requerir una complejidad adicional a la hora de
    implementar la simulación, complejidad que escapa al alcance del presente
    trabajo.

    \item \textbf{Deben de ser de utilidad para los alumnos} 
    
    Deben ser procedimientos comunes en la vida profesional de un enfermero, o
    que no sea posible realizar practicas frecuentes. 

    Al ser una profesión relacionada con la salud, existen múltiples factores
    que dificultan las prácticas de enfermería, como la salud del paciente, la
    salud del profesional, enfermedades o procedimientos poco frecuentes, etc.

\item \textbf{Los procedimientos seleccionados no deben ser complejos ni
        extensos}. 

    Las simulaciones cortas permiten poder ser utilizadas en cualquier momento,
    sin ser interrumpidas. En cuanto al nivel de detalle del entorno, los
    entornos complejos dificultan la atención del
    usuario\cite{videojuegos:gonzaleztardon}, por ello, los factores que son
    simulados deben ser los suficientes para permitir al usuario sentirse dentro
    de la misma, pero no deberían ser muy complejos, sino, el usuario desviaría su
    atención hacia los detalles.

\item \textbf{Deben tener pasos definidos}
    
    Si el procedimiento tiene un objetivo claro, y un conjunto de pasos
    previamente definidos y previsibles, se facilita la aceptación del mismo por
    los profesores, es decir, es más fácil realizar una validación de la
    simulación.

    Hay un compromiso entre la validación de la simulación y la libertad de
    exploración de los usuarios, el escenario elegido debe tener pasos
    definidos, pero a la vez debe permitir al alumno explorar el escenario y
    tomar caminos alternativos.

\item \textbf{Deben ser difíciles de representar con realismo en un aula o
        laboratorio}

    Los laboratorios del \Gls{iab} cuentan con diferentes herramientas que
    facilitan ciertos procedimientos, pero estos no pueden abarcar el amplio
    rango que cubren los procedimientos de enfermería, por ello se debería
    elegir un procedimiento que muestre las ventajas de la tecnología y técnicas
    propuestas, el procedimiento tiene que representar un desafío a las técnicas
    actuales, este desafío puede ser 
    \begin{enumerate*}[label=\itshape\alph*\upshape.]
    \item técnico, como falta de herramientas, como equipos médicos, maniquíes,
        etc, o,
    \item humano, como falta de pacientes con la patología deseada.
    \end{enumerate*}
    
\end{itemize}

%! TEX root = ../main.tex
\section{Selección de procedimientos}
\label{sec:seleccion_escenas}

\observacion{Titulo: procedimiento?. Aclarar como se pasa de un procedimiento a
    una escena}


Con los criterios definidos, y como se mencionó, se seleccionaron dos
procedimientos de enfermería, los cuales serán incluidos en la solución. A
continuación se fundamentan tales elecciones y se entra en detalle acerca de los
procedimientos seleccionados.

Estos procedimientos son incluidos en la solución creando escenarios dentro 
del juego que permitan realizarlos, es así que cada procedimiento es 
representado con escenarios y escenas diferentes.

% Yo pensaba agregarle esto tambien, pero viste que no hablamos nada de la
% solución y eso... por eso tengo miedito

% no entiendo lo que queres decir o tenes en tu cabeza. Dejo que vos nomas hagas esto entonces

\subsection{Extracción de muestras de sangre}
\label{sec:hemocultivo}

El procedimiento denominado \emph{Punción venosa} es utilizado frecuentemente
para extraer muestras de sangre, es un procedimiento invasivo que ofrece un
medio directo de acceso al sistema vascular. 

La frecuencia con la que se le utiliza esta relacionado a su utilidad para
análisis de rutina, los enfermeros lo realizan a diario y es similar a otros
procedimientos como la puesta de una vía intravenosa.

El procedimiento se puede resumir en el proceso de punzar una jeringa en el
brazo al paciente, extraer sangre y retirarla. Si bien estos pasos pueden parecer
sencillos, existen una gran cantidad de factores que definen si el procedimiento
fue realizado correctamente, entre ellas podemos encontrar a factores de 
bioseguridad, como la esterilización correcta de los materiales, sociales, como
la explicación correcta del procedimiento al paciente, etc.

Este procedimiento es considerado como uno de los apropiados de acuerdo a
la apreciación de los profesionales del \Gls{iab} en diversas reuniones, y son: 

\begin{itemize}
\item El mismo posee pasos bien definidos que deben ser seguidos por el
    profesional de enfermería.
\item La complejidad del procedimiento no es muy alta, sus pasos son
    susceptibles de equivocaciones, especialmente en lo que se refiere a la
    bioseguridad, aspecto que sera con más detalle al final de esta sección. 
\end{itemize}



\subsubsection{Evaluación}

% Ver si esta bien explicar esta sección, o ir nomas al grano
Se explican las competencias básicas de este procedimiento con
respecto al plan de estudio de los estudiantes de enfermería del \Gls{iab}, los
criterios según la planilla de practica que poseen los profesores para evaluar
al alumno.

La competencia básica que engloba al procedimiento de extracción de sangres es:

\begin{displayquote}
Ayudar en procedimientos invasivos
\end{displayquote}

Los elementos que son almacenados en la planilla de los instructores, son:
\pregunta{Esto fue extraído de planilla del instructor, se puede poner una
referencia la anexo}

\begin{itemize}
\item Informe al paciente acerca del procedimiento que va a ser
    realizado.
\item Preparación de material para la técnica aséptica.
\item Lavado de manos.
\item Calzado de guantes, chaleco estéril, tapaboca y gorro.
\item Preparación de campo.
\item Punción del brazo, extracción de sangre, compresión de zona de punción.
\item Cambio de aguja.
\item Introducción de la muestra en un frasco preparado para tal efecto.
\item Retiro de materiales y equipo de protección personal.
\item Etiquetado y envío a laboratorio.
\end{itemize}

Es decir, estos son los criterios que debe cumplir cualquier estudiante
para poder aprobar la práctica.

\subsubsection{Protocolo del procedimiento}
\label{sec:hemocultivo_protocolo}

La descripción formal del procedimiento, con todos los pasos necesarios
para poder llevar a cabo los criterios requeridos según~\cite{oms:extraccion}
y los profesores del \Gls{iab}, se observan en la~\ref{fig:proc_hemocultivo}, y
pueden son:

\begin{figure}
\centering
\includegraphics[scale=0.5]{requerimientos/images/hemocultivo.png}
\caption{Procedimiento de extracción de sangre.}
\label{fig:proc_hemocultivo}
\end{figure}

\begin{enumerate}
\item Preparar el equipo, lo que incluye seleccionar la jeringa adecuada.
\item Identificar al paciente, presentarse y explicarle el procedimiento que va
    a ser realizado.
\item Asepsia de las manos.
\item Llevar el equipo a la unidad en donde se encuentra el paciente.
\item Vestirse con bata estéril, tapaboca y gorro.
\item Calzarse los guantes.
\item Ubicar al paciente en posición adecuada, esto es, el brazo debe estar
    extendido y lo mas relajado posible.
\item Elegir la zona a puncionar, para ello se debe palpar la vena para
    averiguar sus características.
\item Colocar el torniquete, 6 a 10 centímetros por encima de la zona de
    punción.
\item Solicitar al paciente que cierre el puño.
\item Esterilizar la zona de punción.
\item Extraer el protector de la aguja.
\item Tensar la zona de punción.
\item Puncionar la piel con la aguja hacia arriba. La aguja se introduce con un
    ángulo de $10$ a $20$ grados.
\item Remover el torniquete.
\item Solicitar la apertura del puño.
\item Extraer la muestra de sangre necesaria.
\item Presionar y extraer la aguja.
\item Colocar algodón con alcohol en el punto de punción.
\item Sellar la muestra y enviarlo a su destinatario.
\item Retirar el equipo utilizado, incluyendo bata, tapaboca, gorro y guantes.
\item Asepsia de las manos.
\end{enumerate}

\subsection{Valoración de la escala de Glasgow}
\label{sec:glasgow}

La escala de Glasgow es utilizada como una herramienta de valoración objetiva
del estado de conciencia de pacientes en estado crítico\cite{protocolo}. La
escala consistente en la evaluación de tres criterios de observación clínica,
los cuales son: 
\begin{enumerate*}
\item la respuesta ocular
\item la respuesta verbal, y
\item la respuesta motora.
\end{enumerate*}

\begin{table}[!hbt]
\centering
\begin{tabular}{llr}
\toprule
\textbf{Severidad} & 
\textbf{Puntuación} \\ 
\midrule
 Leve & 13 a 15 \\
 Moderado & 9 a 12 \\
 Grave & 3 a 8 \\
\bottomrule
\end{tabular}
\caption{Escala de valoración del estado del paciente\cite{helmick2007mild}.}
\label{tab:seleccion_glasgow_estado}
\end{table}

El puntaje que determina el estado del paciente se obtiene sumando la valoración
de cada una de las respuestas, en la figura~\ref{tab:seleccion_glasgow_estado}
se observan los posibles diagnósticos. A la vez cada respuesta se evalúa
mediante una escala independiente una de otra, donde cada respuesta se puntúa
con un número\cite{glasgow:doc}, los valores de cada respuesta se observa en las
tablas~\ref{tab:seleccion_glasgow_respuestas_ocular},~\ref{tab:seleccion_glasgow_respuestas_motor}
y~\ref{tab:seleccion_glasgow_respuestas_verbal}.

\begin{table}[!hbt]
\centering
\begin{tabular}{lr}
\toprule
\textbf{Apertura ocular} & \textbf{Valor} \\
\midrule
Espontánea & 4 \\
Al hablar & 3 \\
Al dolor & 2 \\
Ausente & 1 \\
\bottomrule
\end{tabular}
\caption{Valoración de las distintas respuestas en la escala de Glasgow,
    respecto a la reacción ocular}
\label{tab:seleccion_glasgow_respuestas_ocular}
\end{table}

\begin{table}[!hbt]
\centering
\begin{tabular}{lr}
\toprule
\textbf{Respuesta motora} & \textbf{Valor} \\
\midrule
Obedece & 6 \\
Localiza & 5 \\
Retira & 4 \\
Flexión anormal & 3 \\
Extiende & 2 \\
Ausente & 1 \\
\bottomrule
\end{tabular}
\caption{Valoración de las distintas respuestas en la escala de Glasgow,
    referentes a las respuestas motoras}
\label{tab:seleccion_glasgow_respuestas_motor}
\end{table}

\begin{table}[!hbt]
\centering
\begin{tabular}{lr}
\toprule
\textbf{Respuesta verbal} & \textbf{Valor} \\
\midrule
Orientada & 5 \\
Confusa & 4 \\
Palabras inapropiadas & 3 \\
Palabras incomprensibles & 2 \\
Ausente & 1 \\
\bottomrule
\end{tabular}
\caption{Valoración de las distintas respuestas en la escala de Glasgow
    referentes a la respuesta verbal}
\label{tab:seleccion_glasgow_respuestas_verbal}
\end{table}

El procedimiento se utiliza cuando existe un paciente con un estado de
conciencia indefinido, normalmente después de un accidente donde el paciente
recibió un traumatismo severo. El profesional que se encarga de evaluar debe
verificar el estado del paciente mediante
\begin{enumerate*}
\item preguntas sencillas,
\item estímulos, y
\item inspecciones de partes del cuerpo.
\end{enumerate*}. 
Una vez que se obtiene una valoración individual para los aspectos
motor, ocular y visual del paciente, se obtiene la suma de las valoraciones y se
obtiene un puntaje para el estado del paciente.

El procedimiento de diagnostico utilizando la escala de Glasgow es considerado
como uno de los apropiados para la solución propuesta ya que el mismo posee:

% Recheckear esto
\begin{itemize}
\item Permite la exploración del entorno pues hay diferentes formas de evaluar cada
    estado del paciente.
\item Es rara vez utilizada, pues las condiciones necesarias son críticas, son
    escasas las oportunidades presentadas a los estudiantes durante sus
    prácticas de campo. 
\item La cantidad de reacciones que evalúa el procedimiento es muy alto,
    así, es difícil que un alumno pueda evaluar todos los posibles estados.
\item No es un procedimiento complejo, pues no requiere la manipulación de
    elementos, aún así requiere pericia y debe ser realizado en el menor tiempo
    posible.
\end{itemize}

A continuación se explican las competencias básicas de este procedimiento con
respecto al plan de estudios de los estudiantes de enfermería del \Gls{iab}, los
criterios según la planilla de la practica que poseen los profesores para
evaluar al alumno y el protocolo para llevar a cabo el procedimiento
según~\cite{protocolo} y los profesores del \Gls{iab}.

\subsubsection{Evaluación}

La competencia básica que incluye a la evaluación del paciente mediante la
escala de Glasgow es:

\begin{displayquote}
Identificar actividades de cuidados según problemas urgentes principales.
\end{displayquote}

En el gráfico~\ref{fig:proc_glasgow} se observan los pasos necesarios para
realizar la evaluación utilizando la escala de Glasgow. Dentro del
procedimiento definido en la práctica profesional los criterios son:

\begin{figure}
\centering
\includegraphics[scale=0.5]{requerimientos/images/glasgow.png}
\caption{Evaluación de Glasgow.}
\label{fig:proc_glasgow}
\end{figure}

\begin{itemize}
\item Control de signos vitales.
\item Inspección cefalocaudal, 
\item Evaluación utilizando escala de Glasgow.
\item Preparación del equipo según prioridad del problema
\item Analizar su participación en las actividades.
\item Fundamenta científicamente sus decisiones.
\end{itemize}

Se observa que el punto \enquote{Evaluación utilizando escala de Glasgow}, es un
paso necesario para la evaluación inicial de un paciente en estado crítico.

\subsubsection{Protocolo de práctica}
\label{sec:glasgow_protocolo}

\begin{enumerate}
\item Preparación del material
\item Preparación del paciente: comprobar su identidad, mantener una ambiente
    tranquilo evitando interrupciones, requerir la atención del paciente.
\item Colocar al paciente en posición cómoda.
\item
    Medir la apertura ocular.
\item Evaluar la respuesta motora.
\item Medir la respuesta verbal.
\item Registrar la puntuación final obtenida.
\end{enumerate}

\section{Alcance de la simulación}
\label{sec:alcance}

Para representar los procedimientos seleccionados y descritos en la
sección\ref{sec:seleccion_escenas}, estos deben ser presentados dentro de
escenarios distintos con los elementos necesarios para poder llevarlos a cabo.
Sin embargo, por limitaciones técnicas, tecnológicas y de tiempo, no es posible
realizar una simulación de todos los pasos requeridos.

\observacion{Más énfasis en el hecho que distrae a la parte pedagógica}
Según~\cite{videojuegos:gonzaleztardon}, existe un compromiso visible entre
realismo y credibilidad, tomando en cuenta que uno de los objetivos de la
solución es la presentación de situaciones simplificadas que permitan transmitir
conocimiento, mientras más realismo exista, más detalles existirán y por
consiguiente, los usuarios tendrán que concentrarse en un mayor número de
detalles, lo que resulta contraproducente con el objetivo de la
solución\cite{videojuegos:gonzaleztardon}. Los criterios e hipótesis que se 
describirán a continuación sirven para acotar el
alcance de la simulación, definen qué se simulará y cual es del detalle
necesario para alcanzar las competencias básicas.

%\fixme{A continuación se fundamentan por qué ciertos pasos serán simulados,
%    representados o no simulados dentro de la solución basados en limitaciones e
%    hipótesis asumidas.}{Mover esto abajo de factores limitantes}

%Cada uno de los pasos están agrupados de acuerdo a cada uno de estos aspectos.

\subsection{Factores limitantes}

A continuación se describen los factores que determinan por qué ciertos pasos relacionados 
a los procedimientos no serán simulados dentro de la solución. Estos factores se clasifican 
en tres: limitaciones técnicas, importancia y facilidad de realización. Estos tres aspectos 
que influyen en qué partes se simularán y qué partes se omitirán, se detallan a continuación.


\begin{itemize}
\item  \textbf{Limitaciones técnicas.} 
    
    Las acciones que escapan al alcance del hardware, del software o
    de tiempo de los desarrolladores no son simuladas. Un ejemplo de este 
    tipo de acciones es la simulación del agua (necesarios para el lavado de manos) 
    que requiere de requisitos de hardware avanzados y un tiempo considerable de
    desarrollo.
        
    Los pasos del procedimiento de extracción de sangre que no se simularán por 
    limitaciones técnicas son:
    \begin{itemize}
        \item Tensar la zona de punción.
        \item Ángulo de punción.
        \item Presionar el brazo en el momento en el que se introduce la jeringa.
    \end{itemize}
    
    
\item \textbf{Importancia de representación.}

    No todos los pasos definidos en el procedimiento        
    oficial requieren ser simulados, por ejemplo, la
    colocación de los elementos cerca del lugar de trabajo, es un paso necesario
    en el procedimiento, pero es considerado un paso de menor importancia y fácil de
    realizar.

    La importancia es evaluada por profesionales del \Gls{iab}, los
    cuales dieron su apreciación acerca de cada aspecto simulado, el mismo
    es tenido en cuenta para determinar la importancia de cada
    acción.
    
    Los pasos del procedimiento de extracción de sangre que no se simularán por 
    no ser importantes de representar para la simulación son:
    \begin{itemize}
        \item Llevar el equipo en la unidad donde se encuentra el paciente.
        \item Extraer el protector de la aguja.
        \item Sellar la muestra y enviarlo a su destinatario.
    \end{itemize}
    
    
\item \textbf{Facilidad de realización.}

    Ciertos pasos son fáciles de realizar en la vida real pero requieren un
    esfuerzo significativo para ser simulados con realismo, como preparar el
    equipo necesario.

    La facilidad que tienen los alumnos con las acciones fue determinada por
    profesores del \Gls{iab}, determinaron qué acciones son fáciles de
    realizar para los alumnos y cuáles presentan mayores dificultades en su
    vida profesional.

    Otro aspecto que influye en la facilidad de realización de los
    procedimientos es la familiarización, si los alumnos están
    familiarizados con los procedimientos, estos no son simulados.
        
    Los pasos del procedimiento de extracción de sangre que no se simularán por 
    ser fáciles de realizar por los alumnos son:
        
    \begin{itemize}
        \item Preparar el equipo.
        \item Ubicar al paciente en posición adecuada.
    \end{itemize}
    
    Los pasos del procedimiento de valoración de la escala de Glasgow 
    que no se simularán por la misma razón son:
    \begin{itemize}
    \item Preparar el material.
    \item Preparar al paciente.
    \item Colocar al paciente en posición cómoda.
    \end{itemize}
        
\end{itemize}

\subsection{Hipótesis}
\label{sec:hipotesis}

En este apartado se detallan las hipótesis asumidas durante el desarrollo de 
la solución. Algunas de ellas envuelven la forma de representación de acciones 
dentro de la simulación, estas hipótesis están  basadas en apreciaciones de los 
profesores del \Gls{iab} y en pruebas de usabilidad de interfaz las cuales son 
detalladas en el capítulo\ref{chap:evaluacion} y cuyos resultados se muestran en el capítulo\ref{chap:analisis}. Otras hipótesis representan asunciones de los autores 
en cuanto a la utilidad de ciertos aspectos de la simulación.

Las hipótesis formuladas son las siguientes:

% Hipótesis
\begin{itemize}

\item \textbf{Comandos de voz con interfaz}:  para enviar una petición o
    informarle sobre algo al paciente (por ejemplo, darle detalles del
    procedimiento), no es necesario identificar las palabras del usuario, sino
    más bien detectar que ha hablado y listar las posibles acciones que se
    pueden realizar.
    
    Los pasos del procedimiento de extracción de sangre que son representados por 
    comando de voz son:
    
    \begin{itemize}
        \item Explicar procedimiento.
        \item Solicitar al paciente que cierre el puño.
        \item Solicitar al paciente que abra el puño.
    \end{itemize}
    
    y, los pasos del procedimiento de valoración de la escala de Glasgow 
    que son representados por la misma razón son:
    \begin{itemize}
        \item Explicar el procedimiento.
        \item Medir la respuesta ocular por medio de peticiones al paciente.
        \item Medir la respuesta verbal por medio de preguntas al paciente.
        \item Medir la respuesta motora por medio de peticiones al paciente.
    \end{itemize}

\item \textbf{Extracción uniforme de elementos}: para realizar la acción de
    extraer un elemento utilizado en el paciente, se considera que realizarlo de
    una sola manera para todos los elementos convierte a la interfaz más
    intuitiva.

    Los pasos del procedimiento de extracción de sangre que cumplen con la extracción 
    uniforme son:
    
    \begin{itemize}
        \item Extraer torniquete.
        \item Extraer jeringa.
    \end{itemize}
    
\item 
    \textbf{Acciones de bioseguridad}: la bioseguridad, que es un aspecto
    fundamental y transversal a todo procedimiento de enfermería. Es un área muy
    amplia y transversal a todos los procedimientos de enfermería por lo que se
    considera que simular cada acción es complejo y que sólo basta con que el
    estudiante sepa el momento en el que debe realizarse cada una de estas
    acciones y por lo mismo, es suficiente representarlas a través de opciones
    en la interfaz gráfica.

    \revisar{Mirar resumen, esta mejor descrito ahí}
    
    Los pasos del procedimiento de extracción de sangre que son representados
    por opciones son:
    \begin{itemize}
        \item Asepsia de las manos.
        \item Vestirse con bata estéril, tapaboca estéril y gorro estéril.
        \item Calzar guantes.
        \item Extraer guantes, bata, tapaboca y gorro.
    \end{itemize}
    
\item
    \textbf{Representación iconográfica}: para representar el estado del
    enfermero es suficiente mostrar una imagen representativa. Es decir, que
    para mostrar que el enfermero tiene una gorra, es suficiente mostrar una
    imagen de una gorra.
    
    Esta hipótesis se aplica para las opciones de bioseguridad, las opciones
    para la utilización de elementos así como para la representación del
    elemento seleccionado.
\item 
    \textbf{Factores motivantes}: indicadores de rendimiento, como un puntaje al
    final de cada procedimiento, y el tiempo total dentro del procedimiento,
    impactan positivamente en el involucramiento del usuario.

    La interacción social es otro factor que incrementa el nivel de compromiso
    del usuario.

\item 
    \textbf{Falta de pistas}: la ausencia de signos visuales que indiquen las
    acciones que debe realizar el usuario durante la experiencia, permite al
    usuario probar sus conocimientos, e impide al usuario avanzar en el
    procedimiento utilizando una técnica de \emph{prueba y error}.
    
\item 
    \textbf{Ubicuidad}: darle a los usuarios la posibilidad de poder utilizar 
    la solución en cualquier lugar y momento le ayuda a tener más oportunidades 
    de poner a prueba sus conocimientos.

\end{itemize}
% Simulados

%\subsection{Pasos de exploración}
\subsection{Decisiones de diseño}

No todas las acciones que deben ser realizadas dentro de la simulación están 
limitadas por factores o se encuentran envueltas dentro de una hipótesis. La 
representación de estas acciones fueron definidas por decisiones de diseño 
de los autores, estas decisiones junto con los pasos involucrados son detallados 
a continuación.

\begin{itemize}

\item \textbf{Acciones por interfaz de usuario:} acciones como generar un
    estímulo doloroso al paciente tienen limitaciones técnicas para su
    simulación pero no pueden ser omitidas debido a su gran importancia en el
    procedimiento.
 
    Los pasos del procedimiento de valoración de la escala de Glasgow que son
    importantes de simular pero poseen limitaciones técnicas son:
    \begin{itemize} 
    \item Realizar estímulos dolorosos en diferentes partes del cuerpo. 
    \end{itemize}
    
\item \textbf{Simulación integra de pasos:} Los pasos no mencionados de los procedimientos  
deben ser simulados de forma integra ya que no se ven afectados por 
ninguna de las hipótesis, factores limitantes y criterios descritos con anterioridad.

La evaluación de estos pasos depende de elecciones que haga el usuario, por ejemplo
la elección de una zona de punción depende del usuario, existen varios lugares posibles, 
de los cuales solo algunos son válidos. 

Los pasos del procedimiento de extracción de sangre que cumplen con esto son:
\begin{itemize}
    \item Elegir zona a punzar.
    \item Colocar torniquete.
    \item Esterilizar zona de punción.
    \item Elegir la zona a puncionar.
    \item Punzar la zona con la aguja.
    \item Extraer muestra de sangre.
    \item Colocar algodón en zona de punción.
    \item Presionar zona punción con el algodón.
\end{itemize}

Los pasos de valoración de la escala de Glasgow que también cumplen con esto son:
\begin{itemize}
    \item Registrar la puntuación de la respuesta ocular.
    \item Registrar la puntuación de la respuesta motora.
    \item Registrar la puntuación de la  respuesta verbal.
    \item Registrar el diagnóstico final.
\end{itemize}


\end{itemize}

%! TEX root = ../main.tex
\section{Requisitos de la solución}
\label{sec:requisitos}

A fin de que la solución propuesta pueda cumplir con las competencias básicas
definidas por el plan de estudio, se definen varios tipos de requisitos.

\subsection{Requisitos generales}

Las condiciones generales que afectan a la solución como una herramienta, son:

\begin{itemize}
\item Se debe permitir al usuario poder utilizar el entorno virtual en el
    momento y lugar que desee es decir, es decir, se le debe proveer ubicuidad.

\item Los escenarios presentados, junto con los elementos que los componen,
    deberían ser lo suficientemente realistas para provocar un sentido de
    inmersión al usuario.

\item Cada escena representará un procedimiento de enfermería que debe ser
    realizado por el usuario.

\item El entorno debe permitir al usuario decidir libremente las acciones que
    quiere realizar, favoreciendo la exploración del entorno.

\item El entorno no debe brindar pistas al usuario acerca de la forma en la que
    se deben realizar las procedimientos.

\item El entorno debe brindar al usuario información acerca de su desempeño en
    la escena al final de la misma.

\end{itemize}

\subsection{Requisitos de interacción con objetos}

Con respecto a la interacción del usuario con los diferentes elementos que
pueden ser utilizados dentro de la simulación, denominados objetos, son:

\begin{itemize}

\item La selección de objetos debe ser homogénea, y los mismos pueden ser
    accedidos en cualquier momento, representando la mesa de elementos que
    poseen los enfermeros durante la práctica profesional.

\item Debe ser claro que objeto actualmente esta en uso.

\item Debe ser posible simular la selección y des-selección de objetos.

\item Cada objeto disponible durante la simulación debe ser utilizado de
    manera independiente de los demás, y solamente un objeto a la vez.

\end{itemize}

\subsection{Requisitos de interacción con el entorno}
\observacion{No se entiende la diferencia con el anterior}

Para la interacción entre el usuario, el paciente y el entorno se toman los
siguientes requisitos:

\begin{itemize}
\item El usuario puede manipular el estado del paciente a través de objetos, la
    utilización de los mismos debe ser uniforme e intuitiva.

\item Las acciones sobre los objetos no deben ser muy detalladas, no deben
    distraer al usuario del objetivo principal de la simulación.

\item Todas las acciones sobre objetos que no son relevantes para el objetivo de
    la simulación, no deben ser simuladas, pero sí deben ser realizadas (por
    ejemplo puede existir una opción que indique la realización de cierta
    acción, pero la acción en sí no se simula).

\item La visión del usuario deberá poder ser manipulada en tres dimensiones,
    permitiendo acercar, alejar, mover y rotar la visión para poder observar el
    entorno sin limitaciones.

\end{itemize}

\subsection{Requisitos de interfaz}

En cuanto a las acciones que realiza el usuario mientras utiliza la simulación,
se consideran los siguientes requisitos:

\begin{itemize}
\item Las acciones realizadas por los usuario es en el entorno virtual deben ser
    registradas.

\item Cada acción realizada por el usuario debe ser validada por el entorno, de
    forma a ofrecer información precisa acerca de sus logros al final de la
    partida.

\item La validez de una acción puede requerir en algunos casos que se hayan
    hecho algunas acciones previas, pueden requerir un orden definido, o pueden
    depender del entorno en el cual se realizan las mismas.

\item No se deben dar pistas o mensajes al usuario durante la partida cuando
    este realice incorrectamente una acción.

\item El final de una partida debe indicarse mediante un botón en un menu
    ubicado en la pantalla principal de la escena.

\end{itemize}

Luego de haber definido los procedimientos, el alcance de los mismos y los requisitos 
que debe cumplir la solución, vale la pena mencionar ciertas características 
necesarias que también deben tenerse en cuenta a la hora de implementar la 
solución.

Una de estas características esta enfocada al uso de la solución en cualquier 
lugar y momento, por lo que es un requisito que pueda ser utilizada desde 
un dispositivo móvil.

Otra característica importante, enfocada ya a la interacción del alumno con la 
solución, es que debe ser representada en tres dimensiones para proveer un entorno 
de inmersión y realismo, permitiendo al alumno explorar y examinar los detalles del 
entorno a medida que va realizando el procedimiento.

El siguiente capítulo se encarga de los conceptos, tecnologías y opciones 
disponibles para el desarrollo de una solución que cumpla con los requisitos y 
características descritas.

%Definidos los procedimientos, el alcance de los mismos y los requisitos que debe
%tener cada procedimiento, vale la pena mencionar ciertas características
%necesarias que debe tener la solución propuesta.

%Debe ser utilizada desde un dispositivo móvil, así se puede ejecutar en
%cualquier momento, siendo el requisito que los alumnos tengan tiempo.

%La solución debe ser en tres dimensiones para proveer un entorno de inmersión
%y realismo, permitiendo al alumno explorar y examinar los detalles del entorno
%a la medida que realizan el procedimiento.

%Definidos los requisitos, el siguiente capitulo se encarga de los conceptos,
%tecnologías y opciones disponibles para el desarrollo de una solución que cumpla
%con los requisitos.

\observacion{Creo que lo dst debería ser un poco más formal}

