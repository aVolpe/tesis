%! TEX root = ../main.tex
\section{Definición de criterios}
\label{sec:criterios}

Se realizaron varias reuniones con la coordinadora de la carrera de Licenciatura
en Enfermería del \Gls{iab} y profesores encargados de las prácticas de
laboratorio y campo de diversas materias de la carrera.

En estas reuniones se valoraron los procedimientos que pueden ser simulados,
estos procedimientos son extraídos del plan de estudio de la carrera, y 
posteriormente fueron validados y aceptados por los profesores.

De los procedimientos evaluados y discutidos eligen dos ya que la variedad y 
cantidad de procedimientos de enfermería existentes es muy alta e incluirlos 
a la solución está fuera del alcance de este trabajo. Los principales criterios 
para la selección de los procedimientos son los siguientes:

\begin{itemize}

\item \textbf{El entorno simulado debe adecuarse a las limitaciones de la
        tecnología} 

    Escenarios que requieren la simulación de órganos internos, psicología
    humana, etc, pueden requerir una complejidad adicional a la hora de
    implementar la simulación, complejidad que escapa al alcance del presente
    trabajo.

\item \textbf{Deben de ser de utilidad para los alumnos} 
    
    Deben ser procedimientos comunes en la vida profesional de un enfermero, o
    que no sea posible realizar practicas frecuentes. 

    Al ser una profesión relacionada con la salud, existen múltiples factores
    que dificultan las prácticas de enfermería, como la salud del paciente, la
    salud del profesional, enfermedades o procedimientos poco frecuentes, etc.

\item \textbf{Los procedimientos seleccionados no deben ser complejos ni
        extensos}. 

    Las simulaciones cortas permiten poder ser utilizadas en cualquier momento,
    sin ser interrumpidas. En cuanto al nivel de detalle del entorno, los
    entornos complicados dificultan la atención del
    usuario\cite{videojuegos:gonzaleztardon}, por ello, los factores que son
    simulados deben ser los suficientes para permitir al usuario sentirse dentro
    de la misma, pero no debería ser muy complejo, sino, el usuario desviaría su
    atención hacia los detalles.

\item \textbf{Debe tener pasos definidos}
    
    Si el procedimiento tiene un objetivo claro, y un conjunto de pasos
    previamente definidos y previsibles, se facilita la aceptación del mismo por
    los profesores, es decir, es más fácil realizar una validación de la
    simulación.

    Hay un compromiso entre la validación de la simulación y la libertad de
    exploración de los usuarios, el escenario elegido debería tener una cantidad
    de pasos definidos, pero a la vez debería permitir al alumno explorar el
    escenario y tomar caminos alternativos.

\item \textbf{Debe ser difíciles de representar con realismo en un aula o
        laboratorio}

    Los laboratorios del \Gls{iab} cuentan con diferentes herramientas que
    facilitan ciertos procedimientos, pero estos no pueden abarcar el amplio
    rango que cubren los procedimientos de enfermería, por ello se debería
    elegir un procedimiento que muestre las ventajas de la tecnología y técnicas
    propuestas, el procedimiento tiene que representar un desafío a las técnicas
    actuales, este desafío puede ser 

    \begin{enumerate*}[label=\itshape\alph*\upshape.]
    \item técnico, como falta de herramientas, como equipos médicos, maniquís,
        etc, o
    \item humano, como falta de pacientes con la patología deseada,
        procedimiento que pone en riesgo la vida del paciente
    \end{enumerate*}
    
\end{itemize}
