%! TEX root = ../main.tex
\section{Definición de criterios}
\label{sec:criterios}

De reuniones con la 
coordinación de la carrera de Licenciatura
en Enfermería del \Gls{iab} y profesores encargados de las prácticas de
laboratorio y campo, se extraen procedimientos que
pueden ser simulados teniendo en cuenta el plan de estudios. 
%La aceptación y validación de los procedimientos son realizados por  
%profesores del área de enfermería del \Gls{iab}.

De los procedimientos evaluados y discutidos se eligen dos ya que la variedad y
cantidad de procedimientos de enfermería existentes es muy alta lo que escapa
del alcance del presente trabajo. Los principales criterios para la selección de los 
procedimientos son los siguientes:

\begin{itemize}

\item \textbf{Deben adecuarse a las limitaciones de la tecnología} 

    Escenarios que requieren la simulación de órganos internos, psicología
    humana, etc, pueden requerir una complejidad adicional a la hora de
    implementar la simulación, complejidad que escapa al alcance del
    presente trabajo.

%\item \textbf{Deben ser de utilidad para los alumnos} 
%
%    \observacion{Menos Abstracto}
%    
%    Deben ser procedimientos comunes en la vida profesional de un enfermero, o
%    que no sea posible realizar practicas frecuentes. 
%
%    Al ser una profesión relacionada con la salud, existen múltiples factores
%    que dificultan las prácticas de enfermería, como la salud del paciente, la
%    salud del profesional, enfermedades o procedimientos poco frecuentes, etc.

\item \textbf{Los procedimientos seleccionados no deben ser complejos ni
        extensos}. 

    Las simulaciones cortas permiten poder ser utilizadas en cualquier momento,
    sin ser interrumpidas. En cuanto al nivel de detalle del entorno, los
    entornos complejos dificultan la atención del
    usuario\cite{videojuegos:gonzaleztardon}, por ello, los factores que son
    simulados deben ser los suficientes para permitir al usuario sentirse dentro
    de la misma, pero no deberían ser muy complejos, sino, el usuario desviaría su
    atención hacia los detalles.

\item \textbf{Deben tener pasos bien definidos}
    
    Si el procedimiento tiene un objetivo claro, y un conjunto de pasos
    previamente definidos y previsibles, es más fácil realizar una validación de la
    simulación.

    Existe un compromiso entre la validación de la simulación y la libertad de
    exploración de los usuarios, el escenario debe tener pasos
    definidos, pero a la vez debe permitir al alumno explorar el entorno y
    tomar caminos alternativos.

\item \textbf{Deben poseer limitaciones para ser realizados en un aula o
        laboratorio}

    Los laboratorios del \Gls{iab} cuentan con diferentes herramientas que
    facilitan ciertos procedimientos, pero estos no pueden abarcar el amplio
    rango que cubren los procedimientos de enfermería. Por ello, se deberían
    elegir procedimientos que muestren las ventajas de la tecnología y técnicas
    propuestas, estos procedimientos tienen que representar un desafío a las
    técnicas actuales, este desafío puede ser 
    \begin{enumerate*}[label=\itshape\alph*\upshape)]
    \item técnico, como falta de herramientas, como equipos médicos, maniquíes,
        etc, o,
    \item humano, como falta de pacientes con la patología deseada.
    \end{enumerate*}
    
\end{itemize}
