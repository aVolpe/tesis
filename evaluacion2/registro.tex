\section{Registro de actividades}
\label{sec:registro}

Las metodologías anteriormente descritas incluyen exámenes\revisar{?} que miden
el conocimiento del alumno y su opinión con respecto a la solución propuesta,
pero para poder formar una opinión válida deben experimentar con la misma, para
ello se instala la misma en los dispositivos móviles de los 11 alumnos que
aceptaron formar parte del experimento y poseían móviles con los requisitos
mínimos.

La instalación se lleva a cabo en el \Gls{iab}, y se procede a mostrar un video
de la simulación, explicar la interfaz y realizar una muestra de como
desenvolverse en el entorno.

El periodo de prueba se extiende por 20 días, el mismo no es controlado, es
decir que existen factores que no pueden ser controlados, como:

\begin{itemize}
    \item Tiempo dedicado a la simulación.
    \item Que todas las acciones provengan del alumno deseado\revisar{Borrar}.
    \item Solamente el conocimiento del alumno es puesto a prueba, es decir, no
        se puede controlar que no reciba ayuda externa.
\end{itemize}

\observacion{Reformular}

Por estos motivos, el uso de la solución propuesta no puede ser considerado
como único factor determinante de los resultados de la encuesta objetiva
descrita en~\ref{sec:objetiva}.

La solución propuesta almacena información relacionada a la actividad del
usuario, incluyendo cuando y como utiliza las opciones, los pasos que realiza,
el orden y las condiciones de la escena cuando realiza cada acción.

El registro como un todo es enviado cada vez que el usuario desee, este envío
requiere una conexión a internet por ello no es automático. Adicionalmente el
último día de la prueba, todos los registros fueron enviados para que sean
analizados.


\subsection{Variables}

La utilización de la simulación, y el registro de las actividades genera una
gran cantidad de información, los factores que se desean medir están
relacionados a aquellos que pueden ser contrastados con los resultados de la
encuesta objetiva.

La utilización de la simulación nos permite obtener información relevante acerca
de como se utilizo la misma, se definen los criterios a medir:

\begin{description}

\item[Cantidad de sesiones] Se define como la cantidad de veces que un alumno
    inicia una escena. 

\item[Tiempo total y tiempo por sesión] Es la cantidad de tiempo que dura cada
    sesión, y la suma de todas las sesiones por usuario.

\item[Puntuación por regla completada] Las variables definidas
    en~\ref{sec:objetiva}, pueden ser contrastados con la puntuación obtenida
    por los alumnos en la simulación.

\end{description}


\observacion{No se entiende}

