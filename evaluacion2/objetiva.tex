\section{Encuesta objetiva}
\label{sec:objetiva}

A fin de obtener información acerca del conocimiento de los alumnos que
utilizaron la solución propuesta y los que no la utilizaron, los cuales
constituyen el grupo de control, se realiza un examen que consta de diez
preguntas.

El examen mide el nivel de conocimiento del alumno sobre los dos temas
simulados, contiene preguntas de nivel básico, medio y avanzado. Las mismas son
formuladas utilizando la lista de competencias básicas que debe tener un alumno
para aprobar la materia \textbf{Enfermería en Urgencias II}, y además están
aprobadas por los profesores de la cátedra.

Cada pregunta tiene el mismo peso, así la puntuación más baja obtenible es $0$, y
la más alta es $10$.

\subsection{Muestra}

El universo cuenta con 124 alumnos, de los cuales 11 son la muestra seleccionada
para el experimento, y los 113 alumnos restantes son utilizados como grupo de
control.

La utilización de 11 alumnos es suficiente, ya que según estudios
de~\cite{nielsen2000}, mientras menos experiencia tengan los sujetos de estudio
con la solución planteada, serán necesarios menos para detectar un gran
porcentaje de errores y fortalezas, y según~\cite{ritch2009}, una base de 10 a
12 es suficiente para obtener resultados estadísticamente válidos.

\subsection{Variables}

Se mide el rendimiento de los alumnos, para ello se utiliza el promedio de las
notas, tanto del conjunto total de alumnos, como de los que participaron del
experimento, y del grupo de control.

Adicionalmente, se busca medir el rendimiento por tema, para así poder
contrastar con los resultados del registro de actividades, que se explica en la
siguiente sección.

Siendo:

\begin{itemize}
    \item $po_i{_k}$ la respuesta del usuario $i$ a la pregunta $k$
    \item $n$ total de preguntas, es igual a 10
    \item $tc$ total de alumnos en el grupo de control, que es igual a 113.
    \item $t$ total de alumnos, que es 124
    \item $ts$ total de sujetos de estudio, que es igual a 11.
\end{itemize}

Se define el puntaje total de de cada alumno $pto_i$ del alumno $i$ como, 

\begin{equation*}
    pto_i = \sum_{j=1}^n{po_i{_j}}
\end{equation*}

Se define el promedio total de los alumnos, $promtotal$ como:

\begin{equation*}
    promtotal = \frac{\sum_{i=1}^t{pto_i}}{t}
\end{equation*}

El promedio total de los alumnos que participaron en el experimento
$promsujetos$, es:

\begin{equation*}
    promsujetos = \frac{\sum_{i=1}^ts{pto_i}}{t}
\end{equation*}

El promedio del grupo de control, $promcontrol$, se define como:

\begin{equation*}
    promcontrol = \frac{\sum_{i=1}^ts{pto_i}}{t}
\end{equation*}

\martin{Como se hace para que las últimas dos sumatorias solo incluyan a los
    alumnos encuestados y de control respectivamente? Debemos definir varias
    variables $ptop$ (puntaje total de objetiva que participo)?}


