\section{Objetivos}
\label{sec:objetivos}

Las registros de actividad y encuesta objetiva buscan obtener información sobre
el aprendizaje y la utilización de la solución, mientras que la encuesta
subjetiva busca obtener información acerca de las fortalezas y debilidades de
una simulación para el entrenamiento de enfermeros y de la solución propuesta.

Se definen los objetivos principales de la evaluación como siguen:

\begin{itemize}

\item Validar las hipótesis asumidas durante el desarrollo de la
    solución\martin{Se debe hacer referencia a una sección de la propuesta donde
        se hacen las hipótesis?}.

\item Proponer criterios que puedan utilizarse para la evaluación de
    soluciones similares.

\item Evaluar los puntos fuertes y débiles de la solución en cuanto a los
    criterios definidos.

\item Determinar los factores que afectan al uso de herramientas similares para
    apoyo a profesionales de enfermería.

\item Determinar el nivel de aceptación de herramientas alternativas para el
    aprendizaje.

\item Determinar que factores de la solución propuesta motivan a la utilización
    de soluciones similares.

\end{itemize}






%Ideas tipo tiro al aire de La princesa de cocho por si sirvan
%Para la evaluacion acerca de la validez de las hipotesis planteadas en este trabajo se hacen uso de metodologias como registros de actividades de los usuarios cuando utilizan la aplicacion y encuestas para valorar la opinion de los mismos sobre las caracteristicas de la aplicacion.

%Los objetivos principales de la evaluacion son los siguientes:

%* Verificar la validez de las hipotesis planteadas en el desarrollo de la solucion.
%* Proponer criterios que puedan utilizarse para la evaluacion de aplicaciones de esta naturaleza.
%* Obtener conclusiones acerca de factores externos que afectan el uso de la aplicacion.
%* Identificar los puntos importantes en los que se debe poner enfasis en el desarrollo de las aplicaciones con esta naturaleza.
%* Obtener sugerencias de las correlaciones entre el registro de actividades y el examenen de conocimientos realizado a los usuarios de la aplicacion.
%* Obtener conclusiones generales acerca del uso de la aplicacion como herramienta de apoyo al aprendizaje de estudiantes de enfermeria.









