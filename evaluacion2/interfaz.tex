\section{Interfaz de usuario}

La primera fue realizada con alumnos de la carrera de Ingeniería en Informática
de la Facultad Politécnica que pertenece a la Universidad Nacional de Asunción,
sin experiencia previa tanto con la solución como con los procedimientos
simulados, pero si familizarizados con la utilización de dispositivos móviles.

\cite{nielsen2000} recomienda una muestra de 5 personas para pruebas de
usabilidad.~\cite{ritch2009} menciona que con 5 individuos, se encuentran 85\%
de los errores en promedio, y que un grupo de 5 a 10 personas es adecuado para
pruebas de usabilidad sencillas.

Esta prueba no es de gran complejidad, el procedimiento es sencillo y esta
bien definido, se busca determinar que problemas presenta la interfaz, que
impedimentos encuentran usuarios acostumbrados a la tecnología pero no al
procedimiento, por ello se elige una muestra de 8 alumnos.

\subsection{Muestra}

La primera fue realizada con alumnos de la carrera de Ingeniería en Informática
de la Facultad Politécnica que pertenece a la Universidad Nacional de Asunción,
sin experiencia previa tanto con la solución como con los procedimientos
simulados, pero si familizarizados con la utilización de dispositivos móviles.

\cite{nielsen2000} recomienda una muestra de 5 personas para pruebas de
usabilidad.~\cite{ritch2009} menciona que con 5 individuos, se encuentran 85\%
de los errores en promedio, y que un grupo de 5 a 10 personas es adecuado para
pruebas de usabilidad sencillas.

Esta prueba no es de gran complejidad, el procedimiento es sencillo y esta
bien definido, se busca determinar que problemas presenta la interfaz, que
impedimentos encuentran usuarios acostumbrados a la tecnología pero no al
procedimiento, por ello se elige una muestra de 8 alumnos.
