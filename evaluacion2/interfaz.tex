\section{Interfaz de usuario}
\label{sec:interfaz}

Adicionalmente a las pruebas y mencionadas, durante el desarrollo se realizó una
prueba de interfaz para comprobar hipótesis sobre la interfaz del usuario,
específicamente buscando la retroalimentación de usuarios acostumbrados a
tecnología similar a la utilizada en la solución.

La prueba de interfaz de usuario se realiza con alumnos de la carrera de
Ingeniería en Informática de la Facultad Politécnica que pertenece a la
Universidad Nacional de Asunción\todox{Agregar al glosario}, sin experiencia
previa tanto con la solución como con los procedimientos simulados, pero si
familiarizados con la utilización de dispositivos móviles.

El experimento consiste en dos partes importantes involucradas en la recolección
de datos para su posterior análisis. Estas partes son las siguientes:

\begin{description}

\item[Simulación] Luego de una explicación acerca de las funciones y manejos
    generales de la solución por parte de los experimentadores, cada usuario
    completa una tarea que se le es asignada, como ayuda, recibe un papel que
    lista todos los pasos necesarios para llevar a cabo el procedimiento
    Extracción de muestras para hemocultivo utilizando la solución.
    	
    Las simulaciones son grabadas mediante programas de captura de pantalla, así
    como por detectores de eventos táctiles.
    	
\item[Encuesta] Posteriormente se le provee a cada sujeto una encuesta a cada
    usuario la cuál es utilizada para obtener una idea general acerca de la
    calidad de la simulación. 

\end{description} 

\observacion{Incompleta?}
\subsection{Muestra}

El número de muestras tomadas fue 8, ya que según~\cite{nielsen2000} son necesarios
al menos 5 participantes para poder obtener resultados significativos en una prueba
de usabilidad. Además,~\cite{ritch2009} asegura que la teoría de~\cite{nielsen2000}
es verdadera especialmente para pruebas simples. 

Se fundamenta el número de participantes, y que es una prueba sencilla, por qué:

\begin{itemize}

\item La prueba que se esta ejecutando es sencilla (no debería tomar más de 10
    minutos) 

\item Se busca solamente obtener información acerca de la interfaz, y no el
    funcionamiento en sí de la simulación, pues los usuarios no son expertos en
    el área y no tienen conocimiento acerca de qué debería pasar. 

\item No se busca medir el aprendizaje del usuario en temas no relacionados a la
    interfaz, es decir, no se mide el aprendizaje del usuario en el tema
    simulado.

\item El procedimiento esta bien definido y los pasos necesarios están a
    disposición del usuario en todo momento.

\end{itemize}

\subsection{Variables}

Las variables medidas durante las simulaciones y realización de la tarea son las
siguientes:

\observacion{Quizás convenga poner esto más adelante}

\begin{itemize}

\item Cantidad de opciones elegidas (agrupadas por opción): número de veces en
    que el usuario selecciona una determinada opción.

\item Cantidad de acciones realizadas (agrupadas por tipo): número de veces  en
    que el usuario realiza una determinada acción.

\item Número de pasos realizados: cantidad de pasos requeridos que son
    realizados en la simulación. 

\item Cantidad de movimientos espaciales: número de veces en que se modifica el
    estado de la cámara para realizar las acciones deseadas.

\item Tiempo de elección de opción: cuanto tiempo tarda el usuario en elegir una
    determinada opción.

\item Tiempo de realización de acción: cuanto tiempo tarda el usuario en
    realizar una determinada acción.

\item Cantidad de intentos para realizar acciones: número de veces en que un
    usuario intenta realizar correctamente una acción. 

\end{itemize}

En cuanto a la encuesta, las siguientes son las variables que fueron consideradas y medidas:

\begin{itemize}

\item Calidad gráfica: realismo y calidad de los modelos utilizados.

\item Interacción: desenvolvimiento en el entorno y utilización del hardware.

\item Interacción con objetos: utilización errónea de objetos.

\item Características del entorno: realismo del escenario y de los objetos
    utilizados.

\item Usabilidad de la interfaz: facilidad de uso de las opciones proveídas por
    la interfaz.

\item Integración con el hardware: facilidad de uso de la solución con un dispositivo móvil. 

\end{itemize}

\subsection{Métricas}

Para el análisis de la opinión de los usuarios, se utiliza la escala de Likert
explicada en~\ref{sec:likert}, y en el análisis de la interacción del usuario se
utilizan las grabaciones registradas durante el experimento.

Haciendo uso de estas variables descriptas anteriormente, las métricas
utilizadas son las siguientes:

\begin{itemize}
    
\item Media de error en la elección de opciones: se obtiene dividiendo la suma
    de las opciones realizadas sobre la cantidad ideal. 
    
\item Media de pasos correctos: se obtiene dividiendo la cantidad de pasos
    requeridos realizados sobre la cantidad de pasos requeridos. 
    
\item Cantidad de movimientos por tipo: número de movimientos que fueron
    realizados agrupados por tipo (desplazamiento, acercamiento/
    desplazamiento).
    
\item Evolución del tiempo de elección de opciones: variación del tiempo
    requerido para elegir una opción entre la primera vez y las demás ocasiones.
    
\item Evolución del tiempo de realización de acciones: variación del tiempo
    requerido para realizar una acción entre la primera vez y las demás
    ocasiones.
    
\item Media de error en la realización de acciones: se obtiene dividiendo la
    cantidad de intentos para realizar una determinada acción sobre la cantidad
    ideal.

\end{itemize}

\observacion{Replantear el orden, no se entiende bien quequiere lograr cada una
    de las disintas pruebas en función de los objetivos. Agregar un parrafo a
    cada prueba explicando en que y como contribuye a los objetivos}
