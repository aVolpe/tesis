%! TEX root = ../main.tex

\chapter{Evaluación}


Este capitulo define los mecanismos utilizados para evaluar la solución
propuesta, los mismos están orientados a la validación de las hipótesis
planteadas durante el desarrollo de la solución, lo que incluye aspectos
pedagógicos, de utilidad y de la participación activa del usuario entre otros
descriptos más adelante. Como parte de la evaluación se miden ciertas variables
relacionadas a los aspectos mencionados.

La evaluación se divide en cuatro partes principales:

\begin{description}
    \item[Encuesta de ubicación] Es una encuesta acerca del nivel de acceso a la
        tecnología que poseen los alumnos del 4to año del \Gls{iab}, de ahora en
        más \textit{el Universo}, esta encuesta sirve para definir la muestra.

    \item[Encuesta Subjetiva] Es una encuesta realizada a cada sujeto que
        participa de la solución, donde se busca la opinión del mismo acerca de
        la solución y factores relacionados a la misma. 

    \item[Encuesta Objetiva] Es un cuestionario que es completado por el
        universo de alumnos, donde se mide el conocimiento de los mismos, se
        utilizan a los alumnos que no son la muestra, como grupo de control.

    \item[Registro] Es información almacenada por la solución automáticamente,
        que contiene datos acerca de su utilización y el desempeño del alumno.
\end{description}


Adicionalmente se realiza una evaluación inicial para medir la calidad de la
interfaz y la interacción con la misma, esta evaluación es realizada con
personas no relacionadas al área de enfermería.

El capitulo define los objetivos de la evaluación, describe brevemente conceptos
transversales a las técnicas utilizadas y luego define las metodologías,
métricas y variables utilizadas en cada experimento.

\section{Objetivos}
\label{sec:objetivos}

Las registros de actividad y encuesta objetiva buscan obtener información sobre
el aprendizaje y la utilización de la solución, mientras que la encuesta
subjetiva busca obtener información acerca de las fortalezas y debilidades de
una simulación para el entrenamiento de enfermeros y de la solución propuesta.

Se definen los objetivos principales de la evaluación como siguen:

\begin{itemize}

\item Validar las hipótesis asumidas durante el desarrollo de la
    solución\martin{Se debe hacer referencia a una sección de la propuesta donde
        se hacen las hipótesis?}.

\item Proponer criterios que puedan utilizarse para la evaluación de
    soluciones similares.

\item Evaluar los puntos fuertes y débiles de la solución en cuanto a los
    criterios definidos.

\item Determinar los factores que afectan al uso de herramientas similares para
    apoyo a profesionales de enfermería.

\item Determinar el nivel de aceptación de herramientas alternativas para el
    aprendizaje.

\item Determinar que factores de la solución propuesta motivan a la utilización
    de soluciones similares.

\end{itemize}






%Ideas tipo tiro al aire de La princesa de cocho por si sirvan
%Para la evaluacion acerca de la validez de las hipotesis planteadas en este trabajo se hacen uso de metodologias como registros de actividades de los usuarios cuando utilizan la aplicacion y encuestas para valorar la opinion de los mismos sobre las caracteristicas de la aplicacion.

%Los objetivos principales de la evaluacion son los siguientes:

%* Verificar la validez de las hipotesis planteadas en el desarrollo de la solucion.
%* Proponer criterios que puedan utilizarse para la evaluacion de aplicaciones de esta naturaleza.
%* Obtener conclusiones acerca de factores externos que afectan el uso de la aplicacion.
%* Identificar los puntos importantes en los que se debe poner enfasis en el desarrollo de las aplicaciones con esta naturaleza.
%* Obtener sugerencias de las correlaciones entre el registro de actividades y el examenen de conocimientos realizado a los usuarios de la aplicacion.
%* Obtener conclusiones generales acerca del uso de la aplicacion como herramienta de apoyo al aprendizaje de estudiantes de enfermeria.










%! TEX root = ../main.tex
%! TEX root = ../main.tex

\section{Métricas generales}

Existen métricas que son usadas por más de un experimento\revisar{Ver el termino
    correcto}, a continuación se describen estas métricas:

\subsection{Escala de Likert}
\label{sec:likert}

Para la valoración de las variables medidas se utiliza la escala de
Likert\cite{Allen:2007} de 7 valores posibles. La escala de Likert es utilizada
para permitir a las personas indicar cuánto están de acuerdo o en desacuerdo con
respecto a ciertos puntos. Los valores utilizados, son:

\begin{enumerate}
    \item Totalmente en desacuerdo
    \item En desacuerdo
    \item Parcialmente en desacuerdo
    \item Neutral
    \item Parcialmente de acuerdo
    \item De acuerdo
    \item Totalmente de acuerdo
\end{enumerate}

Una vez valoradas y registradas todas las respuestas y con el objetivo de
eliminar las tendencias en la forma en la que son completadas las
encuestas\cite{Fischer2010} se utiliza el método de Doble Estandarización
recomendado en~\cite{Pagolu2011}. Este método, consiste en dos
estandarizaciones, la primera por fila, que en este caso representa a los
individuos y la segunda por columna donde cada columna representa una de las
diferentes preguntas de la encuesta.

Siendo:
\begin{itemize}
	\item $\min_i$ la respuesta de menor valor del usuario $i$.
	\item $\max_i$ la respuesta de mayor valor del usuario $i$.
\end{itemize}

Para cada respuesta $s$ del usuario $i$, el valor ajustado, por la primera 
normalización, $s_1$ se define como:

\begin{equation*}
s_1{_i}=\frac{s-\min_i}{\max_i-\min_i}
\end{equation*}

Y luego siendo:
\begin{itemize}
	\item $groupmin_i$ la respuesta ajustada de menor valor en el grupo $i$.
	\item $groupmax_i$ la respuesta ajustada de mayor valor en el grupo $i$
\end{itemize}

Para cada respuesta ajustada $s_1{_i}$ del usuario $i$, el valor ajustado $sa_i$ se
define como:	

\begin{equation*}
sa_i=\frac{s_{1_i}-groupmin_i}{groupmax_i-groupmin_i}
\end{equation*}

Obteniendo así un valor normalizado, tanto por individuo, como por pregunta, en
el rango $0$ y $1$.

Para la valoración absoluta de cada  item se utiliza la media de cada columna o
respuesta a una pregunta de la encuesta.

Siendo:
\begin{itemize} 
\item $r_{k_i}$ la respuesta del usuario $i$ a la pregunta $k$.
\item $t_k$ la cantidad total de usuarios que respondieron la pregunta $k$.
\end{itemize}

El puntaje promedio de cada pregunta o item evaluado  $p_k$ en la encuesta se
define como:

\begin{equation*}
p_k = \frac{\sum_{i=1}^n{r_{k_i}}}{t_k}
\end{equation*}

\subsubsection{Manejo de información faltante}
\label{sec:informacion_faltante}

\observacion{Falta mejorar}
En toda encuesta pueden existir preguntas que no sean respondidas, y existen
tres posibles formas de categorizar el patrón de ocurrencia de la falta de
respuestas\cite{leite2010performance}
\cite{leite2010performance}\cite{tsikriktsis2005review}:

\begin{description}
    \item[Información faltante completamente aleatoria] Cuando la información
        faltante es independiente de la variable medida y de otras variables.
    \item[Información faltante aleatoria] Cuando la información faltante depende
        de otras variables, pero no de la variable en sí. 
    \item[Información faltante no aleatoria] Cuando hay una relación entre la
        información faltante y el valor de la variable.
\end{description}

Existen tres mecanismos\revisar{No repitan}\cite{tsikriktsis2005review}
principales para lidiar con información faltante, eliminación, reemplazo, y
procedimientos basados en modelo.~\cite{tsikriktsis2005review} recomienda
utilizar un mecanismo de reemplazo para escalas del tipo Likert.

Las técnicas de reemplazo se clasifican en tres grandes
grupos\cite{tsikriktsis2005review}:
\begin{enumerate*}[label=\itshape\alph*\upshape.]
\item basadas en el promedio,
\item basadas en regresión, y,
\item imputación \emph{hot deck}.
\end{enumerate*}

La sustitución basada por promedio, se divide nuevamente en tres
grupos\cite{tsikriktsis2005review}; promedio
\begin{enumerate*}[label=\itshape\alph*\upshape.]
\item total,
\item del subgrupo, y,
\item por caso.
\end{enumerate*}

La sustitución del promedio total se realiza obteniendo el promedio de todas las
respuestas de esta pregunta, la sustitución de subgrupo es similar, solo que se
limita a aquellos sujetos del mismo subgrupo del sujeto que no respondió, y
finalmente, la sustitución por caso, es el promedio de las respuestas válidas
del sujeto.

\subsection{Correlación de variables aleatorias}
\label{sec:correlacion}

\observacion{Falta un mini parrafo que explique en forma general que es la
    correlación y después mencionar a pearson}

La correlación de Pearson\cite{BoslaughStatistics2008} mide la relación que
existe entre dos variables, $X$ e $Y$, el mismo esta comprendido entre $-1$ y
$1$, en su punto más bajo ($-1$) indica una de las dos variables crece mientras
la otra decrece, y en su punto más alto ($1$), indica que ambas crecen o
decrecen conjuntamente, el valor $0$, indica que no existe una relación entre
ambas variables.

\replantear{\cite{norman2010likert} menciona que la misma puede ser utilizada
    para variables medidas con la escala de Likert, aún cuando la misma es
    utilizada normalmente para variables cuantitativas.}


%! TEX root = ../main.tex

\section{Encuesta de ubicación}
\label{sec:ubicacion}

\observacion{No repitan (se refiere al obtener información)}

A fin de obtener información acerca del nivel de acceso  de los alumnos a la
tecnología, se realiza una encuesta que cuenta con diez preguntas, las cuales
buscan obtener información acerca del modelo de dispositivo móvil, el acceso a
Internet, y la predisposición de cada alumno a ayudar en el experimento.

En el año $2014$, el \Gls{iab} cuenta con $124$ alumnos en el cuarto año distribuidos en
tres secciones, el cual es considerado el Universo. De los 124, 93 de
ellos estuvieron dispuestos a participar de la prueba y completaron la encuesta.

Con los resultados de la encuesta de ubicación tecnológica, se seleccionan
aquellos alumnos que posean dispositivos móviles que superan o igualan las
especificaciones.


\subsection{Variables}

Se definen $2$ factores necesarios para que un alumno pueda ser considerado como
sujeto de experimento\revisar{Utilizar termino prueba}, el primero es la
predisposición del mismo a participar del experimento y el segundo es que posea
un dispositivo móvil que supere los requisitos mínimos.

Los requisitos mínimos para que la solución tenga un desempeño que garantice una
experiencia fluida a la hora de utilizarla son:

\begin{itemize}
    \item Sistema Operativo Android $4.0$ o superior
    \item Memoria ram de $512$MB o superior.
    \item Velocidad de procesador de $800$ GHz o superior.
    \item \Gls{gpu} Mali 400 o superior.
    \item Conexión frecuente a internet.
\end{itemize}

El conjunto de funcionalidades utilizadas por la solución requiere un \Gls{api}
Android de nivel 14\cite{android:api} o superior, lo cual corresponde a un
sistema operativo Android 4.0.

Los requisitos de \textit{hardware} mencionados, son requeridos por las
características de la simulación, y la plataforma utilizada, es requerida una
\Gls{gpu} por los gráficos en tres dimensiones.

La conexión a Internet es requerida, pues los registros de actividad de cada
dispositivo deben ser enviados y almacenados para su posterior interpretación y
análisis.

\section{Encuesta objetiva}
\label{sec:objetiva}

A fin de obtener información comparativa acerca del conocimiento de los alumnos
que utilizaron la aplicación y los que no la utilizaban, utilizados como un
grupo de control, se realizo un examen, que consta de diez preguntas.

El examen busca medir el nivel de conocimiento del alumno sobre los dos temas
simulados, contiene preguntas de nivel básico, medio y avanzado.

Las preguntas fueron formuladas utilizando la lista de competencias básicas que
debe tener un alumno para aprobar la materia \textbf{Enfermería en Urgencias
    II}, y posteriormente fueron aprobadas por los profesores de la cátedra.

Cada pregunta tiene el mismo peso, así la puntuación más baja obtenible es 0, y
la más alta es 10.

\section{Muestra}

El universo cuenta con 124 alumnos, de los cuales 11 son la muestra seleccionada
para el experimento, entonces se utilizan a los 113 alumnos restantes como un 
grupo de control.

\section{Encuesta subjetiva}
\label{sec:subjetiva}

Al final del periodo de prueba, cada alumno de la muestra completa una encuesta
con 31 preguntas que se utilizan para validar las hipótesis. Las preguntas están
agrupadas en dos, el primer grupo cuenta con 27 preguntas cerradas, es decir de
una sola respuesta en una lista de opciones, el segundo grupo cuenta con 4
preguntas abiertas. 

Cada encuesta es entregada a los alumnos que acordaron participar en el
experimento, mientras completan la encuesta, un guía está presente para
responder cualquier duda.

La métrica utilizada es la escala de Likert, descrita en la
sección~\ref{sec:likert}.

\subsection{Variables}
\label{sec:variables}

De acuerdo a los objetivos planteados en la sección~\ref{sec:objetivos}, se
busca describir los factores analizados en las pruebas y las variables
relacionadas a los mismos, las cuales, tienen por objetivo demostrar la validez
de las hipótesis planteadas en este trabajo.

Las variables se presentaran agrupadas en factores, los mismos representan
aquellos aspectos de la solución propuesta que buscan ser evaluados.

\subsubsection{Exploración}
\label{sec:sub_exploracion}

Este factor esta relacionado con la característica que posee la solución en
cuanto a la oportunidad que brinda al usuario para explorar cada uno de los
elementos del entorno simulado (paciente, herramientas propias del
procedimiento). En este sentido, se busca proveer facilidad de uso, intuitividad
y realismo en cuanto a las acciones y situaciones que se presentan en la
solución para que de esta manera, los elementos que la componen no representen
para el jugador un obstáculo que impida su uso.

Las variables que miden este aspecto son las siguientes:

\begin{description}

\item[Funciones realizadas por los elementos del juego] se refiere a la
    correctitud con la que una herramienta o elemento del juego representa las
    funciones que el mismo puede realizar en la vida real, en este sentido, se
    evalúa el realismo con el que es representado tal elemento.

\item[Aleatoriedad para afianzar conocimientos] se refiere al beneficio que
    puede traer el hecho de que el estado del paciente en el juego sea aleatorio
    en cuanto a la posibilidad que esto brinda al jugador para poner a prueba
    sus conocimientos teóricos.

\item[Aleatoriedad para representar realismo] se refiere al uso de estados
    aleatorios en el paciente para que de esta forma el procedimiento se asemeje
    mas a una situación real.

\item[Facilidad de uso] se refiere a lo fácil e intuitivo  que puede ser la
    utilización de los elementos del juego.

\end{description}

\subsubsection{Representación}
\label{sec:sub_representacion}

Este factor esta relacionado con la calidad y suficiencia con la que se
representan los diferentes objetos que son simulados en la solución. La
representación abarca tanto funcionalidad como aspecto del objeto.

De esta manera, se busca permitir al jugador realizar con los objetos las
acciones que requiera para llevar a cabo el procedimiento que se le presente en
la solución, y además, representar estos elementos de la mejor manera posible,
de forma realista.

Las variables que miden estos aspectos son las siguientes:

\begin{description}

\item [Movimientos motrices del paciente] se refiere a la suficiencia de los
    movimientos motrices que realiza el paciente en el escena correspondiente a
    la valoración de la escala de Glasgow.

\item [Movimientos oculares del paciente] se refiere a la suficiencia de los
    movimientos oculares que realiza el paciente en la escena correspondiente a
    la valoración del escala de Glasgow.

\item [Reacción verbal del paciente] se refiere a la suficiencia de las
    reacciones o respuestas verbales que realiza el paciente en la escena
    correspondiente a la valoración de la escala de Glasgow.

\item[Distinción entre los estados del paciente] se refiere a si los diferentes
    estados del paciente son distinguidos correctamente en el procedimiento de
    valoración de la Escala de Glasgow ya que esto es importante para que el
    jugador pueda diagnosticar correctamente al paciente.

\item[Acciones las herramientas] se refiere a si las diferentes acciones que
    pueden realizarse con los elementos o herramientas del juego en un
    determinado procedimiento de enfermería son suficientes para ese
    procedimiento, ya que, debido a las limitaciones de la tecnología estas
    acciones son limitadas.

\end{description}

\subsubsection{Gamificación}
\label{sec:sub_gamificacion}

Este factor esta relacionado con la importancia de incluir en la solución
aquellas características que son propias de un juego de vídeo convencional. Se
busca conocer el valor de estas características en cuanto a la motivación que
puedan producir en los jugadores tanto para volver a utilizar la solución como
para superarse en cada juego.

Las variables que miden estos aspectos son las siguientes:

\begin{description}

\item[Motivación del puntaje] se refiere a que tanto motiva al jugador que la
    solución le proporcione un puntaje total al final de cada partida para poder
    mejorar constantemente siendo este puntaje como una evaluación final de todo
    lo que realizo dentro de la partida.

\item[Importancia del puntaje] se refiere a que tan importante es para un
    jugador que se le proporcione un puntaje total al final de cada partida para
    poder visualizar su rendimiento.

\item[Socialización de los puntajes] se refiere a si el hecho de que las
    personas del mismo entorno compartan sus puntajes, experiencias y logros en
    las partidas a través de redes sociales pueda ser motivador.

\item[Medición del tiempo como motivación] se refiere a que tanto motiva al
    jugador que la solución le proporcione el tiempo que duro su partida
    sirviendo este tiempo como una evaluación de su precisión a la hora de
    realizar el procedimiento que se le presente.

\end{description} 


\subsubsection{Inmersión}
\label{sec:sub_inmersion}

Este factor esta relacionado con el sentimiento de formar parte de la escena. Es
decir, se trata de evaluar que tanto un jugador puede sentir que realmente se
encuentra dentro del juego para que de este modo el pueda entrar en ambiente
para realizar los procedimientos que se le presenten en sus partidas de juego.

Las variables que miden este aspecto son las siguientes:

\begin{description}

\item[Escenografía para entrar en ambiente] se refiere a la importancia de la
    escenografía de la partida para que el jugador entre en ambiente para
    realizar el procedimiento que se le presente.

\item[Juegos cortos como ayuda para la repetición] se refiere a si el hecho de
    que los procedimientos presentados en las partidas sean cortos contribuye a
    repetir las partidas varias veces de seguido entrando en un estado de
    inmersión.

\item[Gráficos en tres dimensiones para entender el entorno] se refiere a la
    importancia que tiene el uso de gráficos en tres dimensiones para que el
    jugador pueda entender mejor el entorno y las posibles acciones que puede
    realizar.

\item[Realismo a través de ordenes verbales] se refiere a si el hecho de que la
    solución brinde la posibilidad de que aparezca un menú de ordenes verbales
    en el momento en que el jugador habla hace que la acción de dar ordenes
    verbales se asemeje mas a la realidad.

\item[Simulación como herramienta] se refiere a si la simulación ayuda al
    jugador a sentirse parte del laboratorio, dando cierto realismo a la escena
    que se le presenta.

\end{description}

\subsubsection{Utilidad}
\label{sec:sub_utilidad}

Este factor esta relacionado con lo útil que puede ser la solución como
herramienta de apoyo al proceso de aprendizaje de los estudiantes de enfermería.

Las variables que miden este aspecto son las siguientes:

\begin{description}

\item[Simulación para complementar el estudio en clase y laboratorio] se
    refiere a que tanto las herramientas alternativas como la simulación pueden
    complementar a los métodos de aprendizaje tradicionales que son el estudio
    en clase y en el laboratorio.

\item[Simulación provee más facilidades para el estudio] se refiere a si las
    herramientas alternativas como la solución proveen más facilidades para
    poner en practica los conocimientos con respecto a los demás métodos de
    aprendizaje que son los libros, laboratorios y el campo de practicas.

\item[Interacción con el paciente] se refiere a si el hecho de que el jugador
    pueda interactuar con un paciente que responde a las acciones del jugador es
    mejor que utilizar un maniquí inmóvil como el de los laboratorios de
    practica.

\end{description}

\subsubsection{Retroalimentación}
\label{sec:sub_retroalimentacion}

Este factor esta relacionado con la importancia de ofrecer al jugador
información acerca de sus logros y errores de manera tal que el pueda estar
consciente de sus puntos fuertes y sus puntos débiles en los diversos
procedimientos que realice en la solución.

Las variables que miden este aspecto son las siguientes:

\begin{description}

\item[Detalles de los pasos realizados incorrectamente] se refiere a que tan
    importante es para el jugador que la solución no solo le diga los pasos que
    hizo de manera incorrecta sino también las causas por las cuales no los
    realizo correctamente.

\item[Suficiencia de los detalles de los pasos realizados incorrectamente] se
    refiere a sí son suficientes las justificaciones breves acerca de las causas
    por las cuales que realizo incorrectamente un paso.

\item[Iconos para representar el estado del jugador] se refiere a la
    suficiencia de mostrar iconos en la interfaz de la solución para
    representar el estado actual del jugador.

\end{description}

\subsubsection{Pedagogía}
\label{sec:sub_pedagogia}

Este factor esta relacionado a la utilidad y al beneficio que puede traer la
solución para apoyar el aprendizaje del jugador. De esta manera, se busca
obtener la validez real de este tipo de herramientas como aporte al aprendizaje,
proveyendo mas interacción al jugador.

Las variables que miden este aspecto son las siguientes:

\begin{description}

\item[La solución para memorizar y comprender el procedimiento] se refiere a
    que tanto ayuda la solución al jugador para entender los procedimientos que se
    le presenten y para memorizar los pasos de cada uno de ellos.

\item[Falta de pistas como ayuda al aprendizaje] se refiere a que tan efectivo
    resulta no dar pistas al jugador en el momento de realizar un procedimiento
    para que pueda plasmar y medir sus conocimientos.

\item[Suficiencia de los botones que indican acciones] se refiere a que tan
    suficiente es representar determinadas acciones  con un botón debido a
    limitaciones en la tecnología.

\end{description}

\section{Registro de actividades}
\label{sec:registro}

Las metodologías anteriormente descritas incluyen exámenes que miden el
conocimiento del alumno y su opinión con respecto a la solución propuesta, pero
para poder formar una opinión válida deben experimentar con la misma, para ello
se instala la misma en los dispositivos móviles de los 11 alumnos que aceptaron
formar parte del experimento y poseían móviles con los requisitos mínimos.

La instalación se lleva a cabo en el \Gls{iab}, y se procede a mostrar un video
de la simulación, explicar la interfaz y realizar una muestra de como
desenvolverse en el entorno.

El periodo de prueba se extiende por 20 días, el mismo no es controlado, es
decir que existen factores que no pueden ser controlados, como:

\begin{itemize}
    \item Tiempo dedicado a la simulación.
    \item Que todas las acciones provengan del alumno deseado.\martin{Se puede
            utilizar la palabra Qué?}
    \item Solamente el conocimiento del alumno es puesto a prueba, es decir, no
        se puede controlar que no reciba ayuda externa.
\end{itemize}

Por estos motivos, el uso de la solución propuesta no puede ser considerado
como único factor determinante de los resultados de la encuesta objetiva
descrita en~\ref{sec:objetiva}.

La solución propuesta almacena información relacionada a la actividad del
usuario, incluyendo cuando y como utiliza las opciones, los pasos que realiza,
el orden y las condiciones de la escena cuando realiza cada acción.

El registro como un todo es enviado cada vez que el usuario desee, este envío
requiere una conexión a internet por ello no es automático. Adicionalmente el
último día de la prueba, todos los registros fueron enviados para que sean
analizados.


\subsection{Variables}

La utilización de la simulación, y el registro de las actividades genera una
gran cantidad de información, los factores que se desean medir están
relacionados a aquellos que pueden ser contrastados con los resultados de la
encuesta objetiva.

La utilización de la simulación nos permite obtener información relevante acerca
de como se utilizo la misma, se definen los criterios a medir:

\begin{description}

\item[Cantidad de sesiones] Se define como la cantidad de veces que un alumno
    inicia una escena. 

\item[Tiempo total y tiempo por sesión] Es la cantidad de tiempo que dura cada
    sesión, y la suma de todas las sesiones por usuario.

\item[Puntuación por regla completada] Las variables definidas
    en~\ref{sec:objetiva}, pueden ser contrastados con la puntuación obtenida
    por los alumnos en la simulación.

\end{description}


\section{Interfaz de usuario}

La primera fue realizada con alumnos de la carrera de Ingeniería en Informática
de la Facultad Politécnica que pertenece a la Universidad Nacional de Asunción,
sin experiencia previa tanto con la solución como con los procedimientos
simulados, pero si familizarizados con la utilización de dispositivos móviles.

\cite{nielsen2000} recomienda una muestra de 5 personas para pruebas de
usabilidad.~\cite{ritch2009} menciona que con 5 individuos, se encuentran 85\%
de los errores en promedio, y que un grupo de 5 a 10 personas es adecuado para
pruebas de usabilidad sencillas.

Esta prueba no es de gran complejidad, el procedimiento es sencillo y esta
bien definido, se busca determinar que problemas presenta la interfaz, que
impedimentos encuentran usuarios acostumbrados a la tecnología pero no al
procedimiento, por ello se elige una muestra de 8 alumnos.

\subsection{Muestra}

La primera fue realizada con alumnos de la carrera de Ingeniería en Informática
de la Facultad Politécnica que pertenece a la Universidad Nacional de Asunción,
sin experiencia previa tanto con la solución como con los procedimientos
simulados, pero si familizarizados con la utilización de dispositivos móviles.

\cite{nielsen2000} recomienda una muestra de 5 personas para pruebas de
usabilidad.~\cite{ritch2009} menciona que con 5 individuos, se encuentran 85\%
de los errores en promedio, y que un grupo de 5 a 10 personas es adecuado para
pruebas de usabilidad sencillas.

Esta prueba no es de gran complejidad, el procedimiento es sencillo y esta
bien definido, se busca determinar que problemas presenta la interfaz, que
impedimentos encuentran usuarios acostumbrados a la tecnología pero no al
procedimiento, por ello se elige una muestra de 8 alumnos.

