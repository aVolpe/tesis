%! TEX root = ../main.tex

\section{Factores comunes}

\subsection{Escala de Likert}
\label{sec:likert}

Para la valoración de las variables medidas se utiliza la escala de
Likert~\cite{Allen:2007} de 7 valores posibles. La escala de Likert es utilizada
para permitir a las personas indicar cuánto están de acuerdo o en desacuerdo con
respecto a ciertos puntos. Los valores utilizados, son:

\begin{enumerate*}
    \item Totalmente en desacuerdo
    \item En desacuerdo
    \item Parcialmente en desacuerdo
    \item Neutral
    \item Parcialmente de acuerdo
    \item De acuerdo
    \item Totalmente de acuerdo
\end{enumerate*}

Una vez valoradas y registradas todas las respuestas y con el objetivo de
eliminar las tendencias en la forma en la son completadas las encuestas
\cite{Fischer2010} se utiliza el método de Doble Estandarización recomendado por
\cite{Pagolu2011}. Este método, consiste en dos estandarizaciones, la primera
por fila que en este caso representa a los individuos y la segunda por columna
donde cada columna representa una de las diferentes preguntas de la encuesta.

Siendo:
\begin{itemize}
	\item $\min_i$ la respuesta de menor valor del usuario $i$.
	\item $\max_i$ la respuesta de mayor valor del usuario $i$.
\end{itemize}

Para cada respuesta $s$ del usuario $i$, el valor ajustado, por la primera 
normalización, $s_1$ se define como:

\begin{equation*}
s1_i=\frac{s-\min_i}{\max_i-\min_i}
\end{equation*}

Y luego siendo:
\begin{itemize}
	\item $groupmin_i$ la respuesta ajustada de menor valor en el grupo $i$.
	\item $groupmax_i$ la respuesta ajustada de mayor valor en el grupo $i$
\end{itemize}

Para cada respuesta ajustada $s1_i$ del usuario $i$, el valor ajustado $sa_i$ se
define como:	

\begin{equation*}
sa_i=\frac{s1_i-groupmin_i}{groupmax_i-groupmin_i}
\end{equation*}

Obteniendo así un valor normalizado, tanto por individuo, como por pregunta, en
el rango $0$ y $1$.

Con el resultado final de la estandarización diferenciamos cuales son los
puntos fuertes y cuales los puntos débiles de la solución propuesta con
respecto a las respuestas dadas por los usuarios. Estos valores son relativos a
las respuestas originales dadas en la encuesta.

Para la valoración absoluta de cada  item se utiliza la media de cada columna o
respuesta a una pregunta de la encuesta.

Siendo:
\begin{itemize} 
\item $r_{k_i}$ la respuesta del usuario $i$ a la pregunta $k$.
\item $t_k$ la cantidad total de usuarios que respondieron la pregunta $k$.
\end{itemize}

El puntaje promedio de cada pregunta o item evaluado  $p_k$ en la encuesta se
define como:

\begin{equation*}
p_k = \frac{\sum_{i=1}^n{r_{k_i}}}{t_k}
\end{equation*}

\subsubsection{Manejo de información faltante}
\label{sec:informacion_faltante}

En toda encuesta pueden existir preguntas que no sean respondidas, y existen
tres posibles formas de categorizar el patrón de ocurrencia de falta de
respuestas\cite{leite2010performance}
\cite{leite2010performance}\cite{tsikriktsis2005review}:

\begin{description}
    \item[Información faltante completamente aleatoria] Cuando la información
        faltante es independiente de la variable medida y de otras variables.
    \item[Información faltante aleatoria] Cuando la información faltante depende
        de otras variables, pero no de la variable en sí. 
    \item[Información faltante no aleatoria] Cuando hay una relación entre la
        información faltante y el valor de la variable.
\end{description}

Existen tres mecanismos\cite{tsikriktsis2005review} principales para lidiar con
información faltante, eliminación, reemplazo, y procedimientos basados en
modelo.~\cite{tsikriktsis2005review} recomienda utilizar un mecanismo de
reemplazo para escalas del tipo Likert.

Las técnicas de reemplazo se clasifican en tres grandes
grupos\cite{tsikriktsis2005review}:
\begin{enumerate*}[label=\itshape\alph*\upshape.]
\item basadas en el promedio,
\item basadas en regresión, y,
\item imputación \emph{hot deck}.
\end{enumerate*}

La sustitución basada por promedio, se divide nuevamente en tres
grupos\cite{tsikriktsis2005review}; promedio
\begin{enumerate*}[label=\itshape\alph*\upshape.]
\item total,
\item del subgrupo, y,
\item por caso.
\end{enumerate*}

La sustitución del promedio total se realiza obteniendo el promedio de todas las
respuestas de esta pregunta, la sustitución de subgrupo es similar, solo que se
limita a aquellos sujetos del mismo subgrupo del sujeto que no respondió, y
finalmente, la sustitución por caso, es el promedio de las respuestas válidas
del sujeto.

\subsection{Correlación de variables aleatorias}
\label{sec:correlacion}
Correlación de pearson


