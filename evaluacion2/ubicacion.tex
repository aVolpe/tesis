%! TEX root = ../main.tex

\section{Encuesta de ubicación}
\label{sec:ubicacion}

A fin de obtener información acerca del nivel de acceso  de los alumnos a la
tecnología, se realiza una encuesta que cuenta con diez preguntas, las cuales
buscan obtener información acerca del modelo de dispositivo móvil, el acceso a
Internet, y la predisposición de cada alumno a ayudar en el experimento.

El \Gls{iab} contó\martin{En que tiempo debe ir? El resto esta en presente.} en
el 2014 con 124 alumnos en el cuarto año distribuidos en tres secciones, el cual es considerado
el Universo. De los 124 alumnos, 93 de ellos estuvieron dispuestos a participar de la prueba
y completaron la encuesta.

Se agrupan a los alumnos en diferentes grupos para determinar si sus
dispositivos celulares son capaces de ejecutar la solución de manera fluida, los
requisitos mínimos para garantizar esta experiencia son:

\begin{itemize}
    \item Sistema Operativo Android 4.0 o superior\todox{Explicar por que
            android}
    \item Memoria ram de 512MB o superior.
    \item Velocidad de procesador de 1 GHz o superior.
    \item GPU \todox{No se como poner la GPU, hay demasiada variedad}
\end{itemize}

Con los resultados de la encuesta de ubicación tecnológica, se seleccionan
aquellos alumnos que posean dispositivos móviles que superan o igualan las
especificaciones, se seleccionan un total de 19 estudiantes.

\martin{Hace falta más detalles? Una sección de métricas y variables, o es
    suficiente con mencionar los criterios mínimos de selección?}
