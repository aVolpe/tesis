\section{Encuesta de ubicación}

A fin de obtener información acerca del nivel de acceso a la tecnología de los
alumnos, la misma cuenta con diez preguntas, las cuales buscan obtener
información acerca del modelo de dispositivo móvil que tiene cada estudiante, el
acceso del mismo a Internet, y su predisposición a ayudar en el experimento.

El \Gls{iab} contó en el 2014 con 124 alumnos distribuidos en tres secciones,
el cual es considerado el Universo.

Se agruparon los alumnos en diferentes grupos para determinar si sus dispositivos
celulares eran lo suficientemente potentes\todox{Buscar mejor expresión} para
ejecutar la solución, los requisitos mínimos para garantizar una experiencia
fluida son:

\begin{itemize}
\item Sistema Operativo Android 4.0 o superior.
\item Memoria ram de 512MB o superior.
\item Velocidad de procesador de 1 GHz o superior.
\item GPU \todox{No se como poner la GPU, hay demasiada variedad}
\end{itemize}

Con los resultados de la encuesta de ubicación tecnológica, se seleccionan
aquellos alumnos que posean dispositivos móviles que superen las especificaciones,
se seleccionan un total de 11 estudiantes\todox{Agregar citas a estudios donde
dice que es suficiente 11 personas}\todox{no se si esto debe ir acá o en la
parte de resultados}.

Estos 11 alumnos seleccionados, representan la muestra que se utiliza para los
demás experimentos.
