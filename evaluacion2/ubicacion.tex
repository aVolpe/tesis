%! TEX root = ../main.tex

\section{Encuesta de ubicación}
\label{sec:ubicacion}

A fin de obtener información acerca del nivel de acceso  de los alumnos a la
tecnología, se realiza una encuesta que cuenta con diez preguntas, las cuales
buscan obtener información acerca del modelo de dispositivo móvil, el acceso a
Internet, y la predisposición de cada alumno a ayudar en el experimento.

En el año $2014$, el \Gls{iab} cuenta con $124$ alumnos en el cuarto año distribuidos en
tres secciones, el cual es considerado el Universo. De los 124, 93 de
ellos estuvieron dispuestos a participar de la prueba y completaron la encuesta.

Con los resultados de la encuesta de ubicación tecnológica, se seleccionan
aquellos alumnos que posean dispositivos móviles que superan o igualan las
especificaciones.


\section{Variables}

Se definen $5$ factores necesarios para que un alumno pueda ser considerado como
sujeto de experimento, 

Los requisitos mínimos para que la solución tenga un desempeño que garantice una
experiencia fluida a la hora de utilizarla son:

\begin{itemize}
    \item Sistema Operativo Android 4.0 o superior
    \item Memoria ram de 512MB o superior.
    \item Velocidad de procesador de 1 GHz o superior.
    \item \Gls{gpu} Mali 400 o superior.
    \item Conexión frecuente a internet.
\end{itemize}

\subsection{Sistema Operativo}

El conjunto de funcionalidades utilizadas por la solución requiere un \Gls{api}
Android de nivel 14\cite{android:api} o superior, lo cual corresponde a un
sistema operativo Android 4.0.
