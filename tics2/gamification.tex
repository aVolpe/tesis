\section{Gamification}
\label{sec:tics_GAMIFICATION}

Es el uso de mecánicas tradicionalmente usadas en los videojuegos en contextos distintos
a los juegos. Según Shell \cite{hj:gamification} un juego es una actividad cuyo fin es resolver un problema de manera entretenida. 

La \emph{Gamification}, mejora la actividad del usuario, el \emph{engagement} (enganchamiento o compromiso con el juego), el aprendizaje, la puntualidad (capacidad de completar una tarea o asignación antes del tiempo designado), el retorno a la inversión, la calidad y la colaboración.

\subsection{Principios}

Los principios de la gamification moderna según \cite{hj:gamification} son los 
siguientes:

\begin{itemize}
\item Objetivos bien definidos.
\item Mejor registro de resultados y tablas de puntuación.
\item Retroalimentación frecuente.
\item Libertad de elección de método para realizar la tarea.
\item Enseñanza y retroalimentación constante.
\end{itemize}

Cuando se mide el desempeño, el rendimiento mejora, cuando el rendimiento se mide y además se informa sobre esto, la tasa de mejora acelera. Cuando la retroalimentación se
presenta en forma de tablas y gráficos el impacto es aun mayor.

\subsection{Propiedades gamification}

Esencialmente, gamification intenta aplicar la mecánica de los juegos en otros entornos, como el ambiente educativo. Este concepto no está directamente relacionado con el diseño del juego, sino que trata de involucrar al usuario a través de pequeñas dosis de desafíos y recompensas con el fin de conseguir que el usuario realice ciertas acciones en diferentes ambientes\cite{breaking:gamification}.

Gamification trabaja para satisfacer algunos de los deseos humanos más fundamentales: el reconocimiento y la recompensa, de estado, de logros, competencia y colaboración, la auto-expresión, y el altruismo.\cite{breaking:gamification}.

La mecánica del juego pueden ser de diferentes tipos\cite{breaking:gamification}, tales como:

\begin{itemize}
	\item Comportamiento (centrado en el comportamiento humano y la psiquis humana),
	\item Retroalimentación (en relación con el ciclo de retroalimentación en la mecánica de juego, y
	\item La progresión (utilizada para estructurar y extender la acumulación de habilidades significativas).
\end{itemize}


Existen otros mecanismos de juego que se pueden utilizar para los materiales gamification y actividades educativas\cite{breaking:gamification}, tales como:

\begin{itemize}
	\item El tiempo (los jugadores tienen un tiempo limitado para realizar una tarea).
	\item La exploración (los jugadores tienen que explorar y descubrir cosas que les sorprenderán).
	\item Los desafíos entre los usuarios (los jugadores pueden darse desafíos unos a otros y competir para el logro de los objetivos, los objetos, medallas, etc.).
\end{itemize}

Para que sea eficaz a largo plazo, gamification debe ser algo más que la adición de este tipo de elementos para un contexto no-juego, también debe actuar sobre la motivación intrínseca de los jugadores\cite{framework:gamification}. 

Con el fin de tener una motivación intrínseca para realizar una tarea, la persona debe mantenerse en un estado entre la ansiedad (si el desafío supera las capacidades de la persona) y el aburrimiento (si la persona siente que la tarea es demasiado fácil ). Este es un estado conocido como flujo. Objetivos claros, un sentido de control, retroalimentación inmediata y, sobre todo, un equilibrio entre habilidad y reto son algunos de los factores que contribuyen a fluir\cite{framework:gamification}.

La relación, el deseo de interactuar y conectarse con otras personas, es una de las necesidades humanas innatas que conducen a la motivación intrínseca\cite{framework:gamification}.

Por lo tanto, los sistemas con gamification no sólo deben abordar la motivación extrínseca de los jugadores, sino también considerar la forma de conducir a los jugadores la motivación intrínseca. Debería centrarse en cómo crear experiencias significativas, proporcionar un sentido de relación entre los jugadores, mejorar su reconocimiento social, y dar la autonomía y el propósito de sus acciones. También debe mantener a los jugadores en un estado de flujo y proporcionar una experiencia divertida conjunto\cite{framework:gamification}. 


\subsection{Elementos del juego}

Los elementos del juego son el conjunto de componentes y características de los juegos de vídeo que se pueden utilizar en contextos no-juego\cite{framework:gamification}.

El flujo y diversión deben ser considerados en un diseño como sección transversal del sistema, transversal a los otros componentes\cite{framework:gamification}.

A continuación mapeamos como se implementarían estos conceptos con los elementos del juego, según\cite{framework:gamification}:

\begin{itemize}
	\item Retroalimentación y recompensas: puntos, barras de progreso, insignias, trofeos, tabla de calificación.
	\item Amigos: compartir, invitar a amigos, dar/comercializar/vender bienes virtuales, tablas de clasificación (gráfico social).
	\item Jugabilidad: niveles, objetivos intermedios, objetivos claros, fracaso divertido, reglas, economía virtual, calendarios de recompensas.
\end{itemize}

Los componentes transversales de flujo y la diversión se logran a través de la forma en que las actividades se establecen en el sistema. El dominio y el progreso son los que hacen que las experiencias sean divertidas. La sensación de dominio y el progreso se puede implementar a través de los elementos de la jugabilidad, los amigos y conceptos de retroalimentación y recompensas. Lo mismo ocurre con el flujo. El jugador puede mantenerse en un canal de flujo cuando él o ella está óptimamente desafiado proporcionando tareas que no son ni demasiado fácil ni demasiado difícil. Esto podría lograrse proporcionando retroalimentación inmediata, objetivos intermedios y diferentes niveles de progresión. De esta manera, el reto es equilibrado con las habilidades del jugador\cite{framework:gamification}.
