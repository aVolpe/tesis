\section{Problemas actuales}

Durante la historia de las \Gls{tic} en la educación, se han encontrado
diferentes dificultades a la hora de aplicar los nuevos conceptos en la
educación, desde los primeros enfoques que carecían de bases pedagógicas válidas
hasta la actualidad, el principal problema es falta de motivación de los
profesionales de la educación para emplear las
\Gls{tic}\cite{punie:ict}\cite{ict:romeo}.

\fixme{El contenido proveído actualmente puede ser considerado como un conjunto
    de buenas prácticas\cite{punie:ict} y así, omiten completa o parcialmente el
    contexto donde esa buena práctica fue generado. }{No se entiende de donde
    sale esto, de que habla y para qué?}

Aún así, las \Gls{tic} han tenido un impacto positivo en la educación, pero el
mismo no es el esperado\cite{punie:ict}, por ejemplo, iniciativas como el
\emph{edutainment} que prometían ser la solución a los problemas
educacionales no cumplieron las expectativas. 

Sucesivos fracasos en los resultados obtenidos dotaron a los \emph{edutainment}
de una reputación negativa, y hoy en día son considerados \fixme{como el peor tipo de
    educación}{Ablandar lenguaje}, pues son un ejercicio de \emph{prueba-error}
ocultos bajo un juego poco entretenido\cite{resnick:2004}, además de su
incapacidad de enseñar como aplicar conceptos aprendidos a un entorno
real\cite{resnick:2004}.

Mientras que la utilización de las \Gls{tic} puede eliminar problemas actuales
como el aislamiento y la falta de pensamiento de alto nivel\cite{punie:ict}, la
brecha social existente implica otro riesgo para la utilización de las \Gls{tic}
en la educación, aquellos que no posean los recursos económicos necesarios para
acceder a la misma no se verán beneficiados por las \Gls{tic}\cite{punie:ict}.

\fixme{Empresas que están en el área de las \Gls{tic}}{Mejorar/pulir lengua} en
educación siguen en la época donde los juegos son prueba y error, esto no
significa que los mismos no funcionen, sino que pueden ser mejorados
considerablemente\cite{egenfeldt2007third}.

Otro de los desafíos actuales es la dificultad comercial impuesta por la
historia de los mismos, es muy difícil para los juegos actuales presentar
promesas realistas, principalmente por el antecedente sentado por los
\emph{edutainment}\cite{egenfeldt2007third}
