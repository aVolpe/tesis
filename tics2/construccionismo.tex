\section{Construccionismo y las TIC's}
\label{sec:tics_CONSTRUCCIONISMO}

\fixme{El construccionismo es una corriente}{decir primero que es, antes de
    comprara} pedagógica con un enfoque diferente en cuanto al uso de las
\Gls{tic} en la educación. Esta pedagogía se diferencia de la educación
tradicional en que el estudiante ya no es un receptor pasivo de información, en
cambio, el mismo participa activamente del proceso de aprendizaje construyendo
su propio conocimiento. 

\fixme{El construccionismo}{no repetir} utiliza la tecnología como medio
cognitivo  a \fixme{diferencia}{} de la educación tradicional que la utiliza para la
entrega de contenido. 

\fixme{El construccionismo}{no repetir} es un alternativa prometedora a la
educación tradicional. Desde el punto de vista tecnológico, el construccionismo
es ideal pues el mismo requiere un alto dinamismo en el traspaso del
conocimiento \cite{sasha:construtivism}. 

\fixme{El construccionismo}{no repetir} y las \Gls{tic} siempre han estado
relacionados, ya que el mismo se originó con un lenguaje de programación
(LOGO)\cite{ict:ttc}. Un característica importante de esta relación es que
tienen la capacidad de eliminar los problemas de
distancia\cite{mariluz:seiousgames}.


\subsection{Historia}

En la decada de $1980$, \emph{Seymour Papert} adoptó el término construccionismo
para representar una método pedagógico practicado por \fixme{John Dewey}{?} a
principios del siglo 20. Este método buscaba que la responsabilidad de aprender
recaiga en el estudiante. 

Papert trabajó directamente con el psicólogo evolutivo y filósofo suizo Jean
Piaget. \fixme{Este}{qué?} último, había elaborado con anterioridad sus teorías
de la educación y construcción del conocimiento al ver e interactuar con los
niños y a partir de esta observación dio origen al constructivismo, según el
cual, el conocimiento debe ser construido por el estudiante y los nuevos
significados deben ser obtenidos relacionándolos con significados anteriores por
los mismos estudiantes haciendo uso así de sus propios sistemas de relaciones.

El construccionismo se \fixme{diferencia}{} de lo anterior en que los estudiantes
construyen las ideas o partes del mundo utilizando herramientas. La elaboración
de representaciones mentales mediante la construcción y el intercambio es la
metáfora del marco construccionista. 

Durante $1980$, Seymor Papert, Wally Feurzeig, Marvin Minsky y John McCarthy y los
miembros del Departamento de Inteligencia Artificial del \Gls{mit} y una
compañía de tecnología en Cambridge, Massachusetts, desarrollaron un nuevo
lenguaje de programación llamado LOGO que tenía por objeto que los estudiantes
construyeran sus \fixme{modelos}{que modelos} en notación LOGO@. Este juego
introduciría de forma natural las ideas de los procedimientos, funciones,
variables, recursividad, la modularidad, simulación, verificación, entre otros.

La creación del lenguaje de programación LOGO dio inicio al construccionismo.

Los desarrolladores de LOGO no solo alentaron la promoción de formas
construccionistas de enseñanza y aprendizaje sino también alentaron otra forma
de aprendizaje nueva y no tradicional con las diferentes herramientas tecnológicas. 

De vuelta en la década de 1980, cuando se produjo LOGO y se acuñó el
construccionismo, la comunidad del construccionismo era en su mayoría  ingenieros
informáticos y matemáticos\cite{historia:2014}.

\observacion{Toda la sección hay que reordernar}

\subsection{Bases Pedagógicas}

Para el construccionismo, el conocimiento es construido por el estudiante en
lugar de ser trasmitido por el \fixme{profesor}{explicar el rol del profesor en
    el construccionismo}\cite{moses:2003} y esto sucede particularmente cuando
el mismo se compromete en la elaboración de un producto o artefacto que tenga un
significado y pueda ser compartido\cite{valdivia:sg}. De esta manera, se permite
a los estudiantes elaborar sus propias interpretaciones razonadas del mundo
mediante la interacción con el mismo.

Según Papert, los alumnos estarán mucho más involucrados en su aprendizaje si
construyen artefactos que los demás pueden ver, criticar y tal vez utilizar. Y
además, el alumno se enfrenta a problemas complejos con estas construcciones,
harán el esfuerzo por resolver problemas y aprender ya que la construcción les
motivará\cite{const:vs}.

El enfoque construccionista establece que los seres humanos conocen y aprenden
de formas diferentes por lo tanto, no se puede elaborar una jerarquía de estilos
de aprendizajes\cite{valdivia:sg}.

\subsection{Estado del Arte}
\observacion{Construccionismo en la presente}

El construccionismo pone énfasis en el \emph{Aprender haciendo}, esta idea
\fixme{mejora}{ref} la práctica educativa tradicional o instruccionismo. El instruccionismo
se basa en el concepto de que existe un profesor y un estudiante, el profesor
transfiere el conocimiento que ha adquirido a un alumno que es receptor pasivo
de información de esta manera, se enfoca más en la capacidad del profesor. 

Existen varios emprendimientos o \emph{\fixme{amigos del contruccionismo}{?}},
para la mayoría de ellos las computadoras son esenciales mientras que para otros
el mayor esfuerzo está en la incorporación de la tecnología en su práctica
educativa\cite{papertian:const}.

Algunos de estos emprendimientos son:

\observacion{Cambiar el ``description'' por ``itemize''}
\begin{description}

\item[Lenguaje de programación LOGO]  A mediados de la década de 1960 Seymour
	Papert, que había estado trabajando con Piaget en Ginebra, llegó a
	Estados Unidos donde co-fundó el Laboratorio de Inteligencia Artifical
	del MIT con Marvin Minsky. Papert trabajó con el equipo de Bolt, Beranek
	y Newman, liderado por Wallace Feurzeig, que creó la primera versión del
	logotipo en 1967. A lo largo de la década de 1970 Logo fuen incubado en
	el MIT y algunos otros sitios de investigación. El lenguaje de
	programación Logo, un dialecto de Lisp, fue diseñado como una
	herramienta para el aprendizaje. Sus características como la
	modularidad, extensibilidad, interactividad y flexibilidad se derivan de
	este objetivo. 
	%http://el.media.mit.edu/logo-foundation/logo/index.html

	El lenguaje Logo es la cuna del construccionismo, se basa en el
	principio de que se aprende mejor haciendo, pero se aprende todavía
	mejor si se combina la acción con la verbalización  y la reflexión
	acerca de lo que se ha hecho. Fundamentalmente consiste en presentar a
	los niños retos intelectuales que puedan ser resueltos mediante el
	desarrollo de programas en Logo. El proceso de revisión manual de los
	errores contribuye a que el niño desarrolle habilidades metacognitivas
	al poner en práctica procesos de auto-corrección\cite{logo:sg}.
	%http://es.wikipedia.org/wiki/Logo_(lenguaje_de_programaci%C3%B3n)

    \observacion{Ver que decir  qu eno sea repetitivo con todo lo que ya se
        dijo.}


%\item[Simulación] La simulación en el ámbito de la educación fue evolucionando
%desde simples motores de reglas hasta complejos entornos, la simulación
%demostró ser una herramienta muy útil el ámbito laboral
%\cite{mariluz:seiousgames}, pues enseña al alumno a encarar situaciones muy
%difíciles de representar en entornos completamente controlados y provee
%mecanismos para comprobar la efectividad de la herramienta. 

%Actualmente la simulación se utiliza más en el ámbito empresarial pues las
%empresas son las más necesitas de innovar en el ámbito de la enseñanza. Un
%ejemplo de esta necesidad se da, por ejemplo, en el entrenamiento de nuevos
%vendedores, es muy difícil enseñar a un vendedor como debe vender los productos
%con un pizzarón y/o una presentación, en cambio la simulación permite que el
%mismo pueda probar cosas nuevas y experiencias de sus compañeros (o
%instructor), convirtiendo así el aprendizaje en
%colectivo\cite{mariluz:seiousgames}. En el ámbito académico la simulación mas
%utilizada en campos físicos (como simulación de fluidos), meteorología
%(simulación de tormentas y fenómenos climáticos), etc. 

%\item[Serious Games] Diseñado con el propósito de aprender. Generalmente hace
%uso de la simulación para permitir un aprendizaje más realista.

%\item[Lego Serious Play] Es una iniciativa de Lego que busca fomentar el
%pensamiento creativo por medio de la construcción por parte de los estudiantes
%de su identidad y experiencias utilizando legos. 

\item[\Gls{olpc}]. El esfuerzo se centra en dotar a los niños de una computadora
	duradera, accesible y potente en los países en desarrollo, se dice que
	es un descendiente directo del construccionismo. Con esto se busca que
	la computadora personal sea utilizada como un laboratorio intelectual y
	un vehículo para la auto-expresión. OLPC no tiene que ver con la
	escolarización o la escuela, más bien las utiliza como medio de
	distribución de las computadoras a los niños, los cuales pueden
	utilizarlas para aprender en cualquier lugar y momento. Se busca
	fomentar el aprendizaje natural, es decir, aquel aprendizaje sin
	enseñanza.

    \fixme{Los problemas atribuidos al experimento}{experimento?} OLPC son
    predominantemente las críticas a la política, el liderazgo o de la
    intransigencia de la escuela en vez del construccionismo o computadora
    personal para los niños pobres. El experimento audaz de Nicholas Negroponte
    (co-fundador de \Gls{olpc}) y Sugata Mitra para dejar las computadoras desde
    un helicóptero sobre una aldea de África se basa en la creencia en el
    construccionismo\cite{papertian:const}.

    \observacion{Poner referencias}
    \observacion{Terrible descripción del proyecto OLPG}
    \observacion{No se de que portal sacaron esta descripción}

\item[Fabricación personal] Neil Gershenfeld, colega de Papert en el Media Lab
	del \Gls{mit} dictó un curso titulado \emph{Cómo hacer casi cualquier
		cosa}. La idea se centraba en la creación de  la tecnología que
	se necesita para resolver los problemas que se poseen. Esta
	auto-confianza, la autonomía personal y la agencia sobre la tecnología
	han estado en el centro de trabajo de Papert durante años. Papert no
	sólo defendió la idea de que los niños posean computadoras personales,
	sino también que a la larga ellos debían mantenerlas, repararlas e
	incluso construirlas.

	Junto con la capacidad para utilizar la tecnología para inventar
	soluciones a los problemas de significado personal, los estudiantes no
	sólo tienen acceso a la información, sino que tienen una mayor capacidad
	para darle forma a su mundo. La fabricación personal promueve la visión
	de Papert \emph{Si se puede utilizar la tecnología para hacer las cosas,
		usted puede hacer las cosas muchos más interesantes y usted
		puede aprender mucho más haciéndolo}\cite{papertian:const}.

\end{description}

%http://constructingmodernknowledge.com/cmk08/wp-content/uploads/2012/10/StagerConstructionism2012.pdf

