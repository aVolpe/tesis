%! TEX root = ../main.tex
\section{Desarrollo de Serious Game}
\label{sec:desarrollo}
\observacion{Esta marea mas considerando que 2.7 es serious games y 2.7 es otra
    cosa}


Pereira\cite{pereira2009design} en el diseño del juego \emph{Living Forest}
utiliza los pasos definidos a continuación como modelo de creación de un juego
serio a partir de la definición previa de las competencias básicas que se desean
enseñar.

Primero se definen las competencias básicas y luego se diseña y desarrolla el
juego. A continuación la figura~\ref{fig:tics_flujo_diseño_prop} muestra el
proceso de desarrollo y luego se explica cada ítem.

\begin{figure}[ht!]
\centering
\begin{tikzpicture}[auto]
    % Place nodes
    \node [block] (1) {Objetivos de diseño};
    \node [block, right of=1, node distance=5cm] (2) {Competencias básicas relacionadas con la educación};
    \node [block, right of=2, node distance=5cm] (3) {Investigación del dominio};
    \node [block, below of=3, node distance=3cm] (4) {Diseño del juego};
    \node [block, left of=4, node distance=5cm] (5) {Tiempo en el juego};
    \node [block, left of=5, node distance=5cm] (6) {Acciones de jugabilidad};
    \node [block, below of=6, node distance=3cm] (7) {Indicadores};
    \node [block, right of=7, node distance=5cm] (8) {Representación e interacción};
    \node [block, right of=8, node distance=5cm] (9) {Implementación};
    \node [block, below of=9, node distance=3cm] (10) {Evaluación};
    % Draw edges
    \path [line] (1) -- (2);
    \path [line] (2) -- (3);
    \path [line] (3) -- (4);
    \path [line] (4) -- (5);
    \path [line] (5) -- (6);
    \path [line] (6) -- (7);
    \path [line] (7) -- (8);
    \path [line] (8) -- (9);
    \path [line] (9) -- (10);
\end{tikzpicture}

\caption{Flujo de diseño propuesto de un Serious Game}
\label{fig:tics_flujo_diseño_prop}
\end{figure}


\subsection{Partes del flujo de diseño}

A continuación se describen cada una de las partes mencionadas
en~\ref{fig:tics_flujo_diseño_prop}.

\subsubsection{Objetivos de diseño}

Definen cuál es el propósito del juego, donde se toman en cuenta los objetivos
pedagógicos, así como también objetivos que garanticen que el mismo sea
agradable, intuitivo y motivador.

\subsubsection{Competencias básicas relacionadas con la educación} 

Se identifican aquellas que influyen en el diseño del juego, se definen los
conocimientos mínimos que se desea que tenga un usuario que lo utilice.

Las competencias básicas pueden tener diferentes orígenes, en el ámbito
académico se pude utilizar el plan de estudios, en una empresa se pueden
utilizar los objetivos y la visión de la misma.

\subsubsection{Investigación del dominio}

Esta fase se encarga de recabar información exacta acerca del dominio en el cual
se desenvuelve el juego serio, en esta fase es importante que participe un
experto en el dominio, por ejemplo, en el ámbito académico se puede contar con
un profesor experto en el dominio.

Es importante realizar la pregunta \emph{¿Que nivel de detalle es necesario?},
para así definir que contenido incluir, y que factores se deben analizar.

Además es necesario investigar las acciones que se podrían realizar dentro del
juego, como se desenvolverá el jugador, por cada acción definida, se deben
analizar los elementos y factores relacionados que se deben modelar.

\subsubsection{Diseño del juego}

A partir de la idea original y basado en la información recogida se determina
el papel desempeñado por el jugador (de acuerdo a la semántica y pragmática de
las acciones y decisiones que está llamado a hacer). 

Se define el nivel de aproximación a la realidad, el nivel de detalle del
entorno, del jugador y de las acciones.

Otro factor que se debe tener en cuenta en esta fase es la cantidad de tiempo
que pasará un jugador en el juego, se deben modelar todas las acciones del
jugador y el entorno de acuerdo a este tiempo.

\subsubsection{Tiempo de juego}

El primer factor que se debe estudiar es el periodo de adaptación del jugador,
lo que dependen de la intuitividad del juego, este tiempo debe ser analizado por
separado a la hora de realizar un análisis de los resultados.

Si el juego tiene una duración reducida, se tienen que analizar mecanismos para
mostrar los resultados de las decisiones a largo plazo, además de como mostrar
los resultados de las acciones que de corto plazo.

\subsubsection{Indicadores}

Es todo aquello que muestre información relevante al jugador acerca de su
estado, ejemplos de este tipo de indicadores son el puntaje, tiempo empleado,
objetivos cumplidos. 

La definición de como se juzgará la calidad de una partida del jugador debe ser
definida, normalmente mediante un puntaje general, el mismo debe mostrar
claramente los resultados de las acciones, si las mismas fueron positivas o
negativas para el logro final de los objetivos.

\subsubsection{Representación e interacción}

Representación se refiere a como se visualiza el entorno
\fixme{simulado}{serious games, es solo sobre simular?, cuidado con las
    terminologias por que tienen una sección de simulación}, e interacción como
se relaciona el jugador con su entorno.

Se inicia con un bosquejo de las representaciones de la escena del juego, para
así poder definir de los elementos que forman partes de la escena.

Otros bosquejos necesarios son los del concepto que se modela en la lógica del
juego, para así definir las animaciones del entorno y del jugador.

Se debe definir las alertas sonaras, que partes del entorno produce sonidos,
como el jugador recibe estas alertas (por ejemplo si el origen de las mismas es
siempre el mismo o importa la distancia a la cámara), se puede agregar música de
ambiente si el juego lo amerita.

Se define la interfaz del usuario, que información sera representada, las
acciones disponibles desde la misma, además se define si el mismo será en
primera persona (la cámara son los ojos del jugador) o en tercera persona (la
cámara se sitúa inmediatamente atrás y arriba de la cabeza), como será la
interacción con la cámara, acercamientos y movimientos para contemplar el
entorno.

\subsubsection{Implementación} 

En esta etapa se estudia el estado del arte de las plataformas tecnológicas
disponibles para el desarrollo del juego, se toman en cuenta los factores como
la disponibilidad de componentes, de documentación, lenguajes de programación y
herramientas de pruebas automáticas.

El proceso puede ser iterativo, entre sesiones de implementación y evaluación de
lo implementado, para así poder realizar optimizaciones enfocadas especialmente
en la estética, la retroalimentación y el estado del jugador.

\subsubsection{Evaluación} 

Durante el desarrollo del juego serio, se debe realizar varias sesiones de
evaluación durante el desarrollo, por ejemplo, con los responsables o expertos,
miembros de la audiencia objetivo. Así mismo, se debe realizar evaluaciones con
los grupos de interés se centra en la adaptación del juego (usabilidad). 

La primera se centra en la validación del modelo de la simulación
(refinamiento), mientras que la segunda evaluación sirve para probar el juego en
un escenario (parecido al final) y evaluar los aspectos relacionados con el
proceso de aprendizaje.

