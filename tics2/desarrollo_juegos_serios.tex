\section{Desarrollo de Serious Game}

\todox{Agregar mas contenido}

Pereira\cite{pereira2009design} en el diseño del juego \emph{Living Forest} utiliza los pasos definidos a continuación como modelo de creación de un juego serio a partir de la definición previa de las competencias básicas que se desean enseñar.

Primero se definen las competencias básicas y luego se diseña y desarrolla el juego. A
continuación la figura \ref{fig:tics_flujo_diseño_prop} muestra el proceso de desarrollo y luego se explica cada ítem.

\begin{figure}[ht!]
\centering
\begin{tikzpicture}[auto]
    % Place nodes
    \node [block] (1) {Objetivos de diseño};
    \node [block, right of=1, node distance=5cm] (2) {Competencias básicas relacionadas con la educación};
    \node [block, right of=2, node distance=5cm] (3) {Investigación del dominio};
    \node [block, below of=3, node distance=3cm] (4) {Diseño del juego};
    \node [block, left of=4, node distance=5cm] (5) {Tiempo en el juego};
    \node [block, left of=5, node distance=5cm] (6) {Acciones de jugabilidad};
    \node [block, below of=6, node distance=3cm] (7) {Indicadores};
    \node [block, right of=7, node distance=5cm] (8) {Representación e interacción};
    \node [block, right of=8, node distance=5cm] (9) {Implementación};
    \node [block, below of=9, node distance=3cm] (10) {Evaluación};
    % Draw edges
    \path [line] (1) -- (2);
    \path [line] (2) -- (3);
    \path [line] (3) -- (4);
    \path [line] (4) -- (5);
    \path [line] (5) -- (6);
    \path [line] (6) -- (7);
    \path [line] (7) -- (8);
    \path [line] (8) -- (9);
    \path [line] (9) -- (10);
\end{tikzpicture}

\caption{Flujo de diseño propuesto de un Serious Game}
\label{fig:tics_flujo_diseño_prop}
\end{figure}


\subsection{Partes del flujo de diseño}
\subsubsection{Objetivos de diseño}
Definen cuál es el propósito del juego

\subsubsection{Competencias básicas relacionadas con la educación} 

Se identifican aquellas que influyen en el diseño del juego

\subsubsection{Investigación del dominio}
\begin{itemize}
	\item Recabar información importante para el diseño.
	\item Participación de un experto en el dominio.
	\item Estadísticas sobre características.
	\item Una pregunta que surge es: que grado de detalle debe ser modelado?.
	\item Que incluir y que considerar.
	\item Identificar conjuntos de acciones hipotéticas a modelar (jugadores).
	\item Comprensión del dominio (funciones).
	\item Para cada función, analizar los elementos y servicios relacionados que se podrían modelar.
\end{itemize}

\subsubsection{Diseño del juego}
\begin{itemize}
	\item A partir de la idea original y basado en la información recogida. 
	\item Determinar el papel desempeñado por el jugador (de acuerdo a la semántica y pragmática de las acciones y decisiones que está llamado a hacer). 
	\item Aproximación a la realidad y exploración con tiempo limitado. 
\end{itemize}

\subsubsection{Tiempo de juego}
\begin{itemize}
	\item Permitir al jugador experimentar las consecuencias de la toma de decisiones a corto plazo.
	\item Tiempo para la primera experiencia (Adaptación).
\end{itemize}

\subsubsection{Indicadores}
Indicadores del progreso de un jugador, como por ejemplo: puntaje, cumplimiento de objetivos, etc.

\subsubsection{Representación e interacción}
\begin{itemize}
	\item Implementación de las representaciones de la escena del juego.
	\item Elementos que forman partes de la escena.
	\item Representar visualmente el concepto que se modela en la lógica del juego.
	\item Animaciones.
	\item Sonido.
	\item Punto de vista del jugador en la interfaz del juego.
	\item Movimientos de cámaras y zoom.
	\item Mantener coherencia en las representaciones.
	\item Diseño de componente de la interfaz. Por ejemplo panel de indicadores, barra de herramientas.
\end{itemize}

\subsubsection{Implementación} 
\begin{itemize}
	\item Distintas tareas del proceso de desarrollo. 
	\item Plataforma. La producción del juego involucra el estado del arte, el desarrollo de componentes, el estado del arte, la programación y el testeo. 
	\item Iteraciones (implementación y evaluación). 
	\item Realizar optimizaciones. 
	\item Poner énfasis en la estética, retro-alimentación y el estado. 
\end{itemize}

\subsubsection{Evaluación} 
\begin{itemize}
	\item Realizar varias sesiones de evaluación durante el desarrollo. Por ejemplo, con los responsables o expertos, miembros de la audiencia objetivo. 
	\item La evaluación con los grupos de interés se centra en la adaptación del juego (usabilidad). 
	\item La evaluación con los expertos se centra en la validación del modelo de la simulación (refinamiento). 
	\item Evaluación con la audiencia objetivo para probar el juego en un escenario(parecido al final) y evaluar los aspectos relacionados con el proceso de aprendizaje.
\end{itemize}

