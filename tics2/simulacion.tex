\section{Simulación}
\label{sec:tics_SIMULACION}

\obervacion{No se entiende como se llego a esto}

\observacion{Ver como organizar el contenido para que no sea una bolsa de
    conceptos}.

La simulación se define como el proceso de diseñar un modelo de un sistema real
y, llevar a cabo experimentos con este modelo, con el fin o bien de entender el
comportamiento del sistema o de la evaluación de distintas estrategias para la
operación del sistema\cite{ingalls2008introduction}. 
%[ingalls2008introduction]

Un juego y una simulación podrían llegar a ser muy parecidos, a veces los juegos
tienen motores de simulación\footnote{Un motor de simulación es un conjunto de
objetos y métodos que se utilizan para la construcción de modelos de
simulación que están dentro de las aplicaciones}, una de las diferencias
es que la simulación es muy dependiente del contexto. 

La simulación en el ámbito de la educación fue evolucionando desde simples
motores de reglas hasta complejos entornos, la simulación demostró ser una
herramienta muy útil en el ámbito laboral\cite{mariluz:seiousgames}, pues enseña
al alumno a encarar situaciones muy difíciles de representar en entornos
completamente controlados y provee mecanismos para comprobar la efectividad de
la herramienta. 

Actualmente la simulación se utiliza más en el ámbito empresarial pues las
empresas son las más necesitadas de innovar en el ámbito de la enseñanza. Un
ejemplo de esta necesidad se da, por ejemplo, en el entrenamiento de nuevos
vendedores, es muy difícil enseñar a un vendedor como debe vender los productos
con un pizarrón y/o una presentación, en cambio la simulación permite que el
mismo pueda probar cosas nuevas y experiencias de sus compañeros (o instructor),
convirtiendo así el aprendizaje en colectivo\cite{mariluz:seiousgames}. En el
ámbito académico la simulación es más utilizada en campos físicos (como simulación
de fluidos), meteorología (simulación de tormentas y fenómenos climáticos), etc. 

Existen dos tipos de simulaciones, en primer lugar están las experimentales que
ponen al estudiante en el lugar de un profesional y requieren que el mismo tome
decisiones para alcanzar los objetivos y en segundo lugar están las simbólicas
que buscan que el estudiante deduzca eventos, principios y mejores prácticas
\cite{charsky:2010}. 

\fixme{Una simulación}{sección?} esta conformada por:

\begin{description}

\item[Entidades] Cualquier objeto o componente en el sistema que requiera
	representación explícita en el modelo se define como
	entidad\cite{banks2000dm}. Las entidades poseen atributos. Los atributos
	son las características de una determinada entidad que son exclusivos de
	esa entidad. Por último, son aquellas que cambian el estado de una
	simulación. Ejemplo de entidades son: un médico o una jeringa en una
	simulación médica.

\item[Acciones] Las entidades interactúan entre sí a través de acciones. Estas
	acciones puede causar cambios en el estado de la simulación además de
	eventos. Ejemplo de una acción en una simulación médica es la
	esterilización de un instrumento.

\item[Eventos] Los eventos son hechos que ocurren de manera controlada pero no
	siempre predecible en el entorno simulado, los mismos afectan a las
	entidades y deben obligar a realizar alguna de las acciones disponibles
	para tal evento. Ejemplo de un evento en un simulación médica es un paro
	cardíaco del paciente.

\end{description}

La confianza en el modelo o la simulación según\cite{DoDSysEng2001} se establece
mediante:

\begin{description}

\item[La verificación] Es el proceso de determinar si la implementación
	representa con precisión las especificaciones del diseño. 

\item[La validación] Es el proceso de determinar el grado en el que el modelo
	representa de forma exacta la realidad de acuerdo al uso que se tiene
	previsto darle y el nivel de confianza que debe tenerse en la
	evaluación.

\item[La acreditación] Es el proceso de certificación de un modelo para su uso
	con un propósito específico.
%[DoDSysEng2001]

\end{description}



%En la actualidad, la utilización de la simulación como herramienta para el
%entrenamiento es cada vez mayor por partes de los profesores, quienes están
%cada vez más familiarizados con la tecnología. 

%Según\cite{humphreys2013developing} los tipos de estudiantes definidos por Kolb
%son:

%\begin{description}

%\item[Accommodating learners] Aprenden de la experiencia e interiorizan el
%aprendizaje a través de experimentación activa. 

%\item[Diverging learners] Aprenden a través de experimentación activa, e
%interiorizan el conocimiento reflexionando sobre la experiencia. 

%\item[Coverning learners] Aprenden a través del pensamiento abstracto e
%interiorizan el conocimiento a través de la experimentación activa.

%\item[Assimilating learners] Aprenden a través del pensamiento abstracto y las
%interiorizan reflexionando sobre las mismas. 
	
%\end{description}

%Teniendo en cuenta el caso de la enfermería, la misma es una ciencia que atrae
%a alumnos del tipo \emph{Diverging learners}, y la simulación es una
%herramienta ideal para este tipo de estudiantes.

La mayoría de la literatura encontrada acerca de la simulación y los cuidados de
la salud no proporcionan muchos detalles acerca de la implementación de modelos
de simulación en áreas amplias, se cree que esto se debe a la complejidad de
representar las actividades relacionas al cuidado de la salud dentro de un
modelo de simulación que debe, de hecho, ser una simplificación de las mismas.
Esta simplificación puede ser un proceso sumamente complejo, por lo cual la
mayoría se centra en una parte de las actividades hospitalarias pero no así en
todas. Cuanto mayor sea el detalle, la simulación \fixme{conducirá}{} a una representación
más realista lo cual aumenta la confianza en los grupos de interés, sin embargo,
más detalle requiere más datos validados y esto puede ser costoso de
obtener\cite{guna:simulation}.

Algunas aplicaciones específicas en el cuidado de la salud son:
\observacion{Incluir un párrafo porque el énfasis en simulación y salud}

\begin{description}

\item[Departamento de emergencia y accidentes] La mayoría de los trabajos
	realizados en esta área se refieren a la optimización de tiempo de
	espera de los pacientes y la organización del personal, de las
	habitaciones,de las ambulancias, para dar mejor atención a los
    pacientes. Un ejemplo de esto es \emph{Edsim} que se utiliza para aumentar
    el rendimiento en un departamento de emergencias en los Estados Unidos como
    parte de un sistema que permite el desvío de ambulancias en los períodos
    pico de demanda, el cual incluye la introducción de salones de descarga y la
    disminución del tiempo de estancia, sin pasar por el
    triaje\cite{guna:simulation}. 
	
\item[Instalaciones para pacientes hospitalizados] Los trabajos se centran en la
    mejora en la atención con respecto al flujo de pacientes así como la
    ocupación de camas. Muchos trabajos tratan de demostrar como se podrían
    utilizar modelos matemáticos para esto. \emph{Harper y Shahani} presentaron
    un modelo de simulación flexible relacionado a estás cuestiones de pacientes
    hospitalizados, el mismo utiliza \emph{TOCHSIM}, \fixme{flexible en el sentido de
        que aborda también}{pulir} problemas como la creación de una nueva
    unidad en el hospital\cite{guna:simulation}.

\item[Clínicas para pacientes ambulatorios] \fixme{En este sentido}{?} la
    simulación se utiliza para minimizar el tiempo de espera de los pacientes en
    clínicas externas, \fixme{es decir}{pulir}, aquellas en las que se sacan citas. El tiempo
    de espera no sólo implica la espera dentro de la clínica sino también el
    tiempo que pasa entre el momento en el que se solicita un cita y el día de
    la cita. Un ejemplo de esto es \emph{CLINSIM} que se utilizó en el Reino
    Unido para observar como la política de operación puede influir en los
    tiempos de espera de los pacientes\cite{guna:simulation}. 

\item[Formación médica y quirúrgica] Se centran en tareas específicas y en la
	formación de un conjunto limitado de habilidades referentes a estas
	tareas. Los ejemplos más recientes son entrenamiento para un intubación
	esofágica, capacitación y evaluación de capacidades laparoscópicas,
	entrenamiento para la palpación de tumores de mama\cite{mantovani:vr}. 

\item[Sistemas de formación de emergencias] Se refieren a aquellas simulaciones
	diseñadas para la rápida respuesta médica. Incluye desde pacientes
	virtuales dinámicos cuya acción por parte del estudiante produce un
	cambio clínico en el mismo y una respuesta al estudiante.  Otro ejemplo
	es el utilizado en la marina de EE.UU que intenta formar a los
	profesionales para su rápida acción frente a desastres civiles y donde
	la estabilización de pacientes se tenga que dar con recursos
	limitados~\cite{mantovani:vr}. 

\item[Entrenamiento para profesionales de salud mental] Janssen LP creó una
	simulación para educar a los psiquiatras y profesionales de la salud en
	lo que es tener esquizofrenia llamada \emph{el viaje en autobús} que
	trata de mostrar lo que pasa dentro de de la mente de una persona con
	esquizofrenia cuando viaja en autobús en base a experiencias relatadas
	por pacientes y médicos\cite{mantovani:vr}. 

\end{description}

\observacion{Buscar manera de poner todas estas secciones en un mismo plano,
    como para que tenga sentido el ir explicándolas (falta un conector), por
    ejemplo, donde se sitúan todos estos trabajos en la linea de tiempo que
    presentaron inicialmente? Esta demasiado desconectado e aislado.
    
    Se necesita complementar mas esa primera sección de manera tal a que se
    entienda de donde salio todo esto.}
