\section{Educación Tradicional}

La educación tradicional o instruccionismo se basa en el concepto de que existe
un profesor y un alumno. El profesor transfiere el conocimiento que ha adquirido
de diferentes métodos (educación, experiencia, etc) a un alumno que es un
receptor pasivo de información\cite{johnson2005instructionism}.

Se enfoca más en el profesor, y en la enseñanza, y en el producto final como
resultado de un proceso no interactivo y bien
documentado\cite{igi:instructionism}. Los mecanismos tradicionales para poder
probar la efectividad de este tipo de enseñanza son los exámenes.

La principal critica a este modelo es que se enfoca la enseñanza y no el
aprendizaje, mientras más se enseña, más se aprende. Esto contradice al sentido
común en el sentido de que, cosas básicas como caminar o hablar, aprendemos sin
la necesidad de un
profesor\cite{ackoff:education}\cite{johnson2005instructionism}.

Tradicionalmente el rol de las TIC's en la educación se vio relegada a la de
sustituto del libro y/o de presentaciones en clase, es decir, es un mecanismo
más para trasmitir el conocimiento del maestro al alumno.


