%! TEX root = ../main.tex

\section{Encuesta de ubicación}
\label{sec:ubicacion}

Para recabar información acerca del nivel de acceso  de los alumnos a la
tecnología, se realiza una encuesta que cuenta con diez preguntas, las cuales
buscan conocer sobre el modelo de dispositivo móvil, el acceso a
Internet, y la predisposición de cada alumno a ayudar en la prueba.

Con los resultados de la encuesta de ubicación tecnológica, se seleccionan
aquellos alumnos que posean dispositivos móviles que superan o igualan las
especificaciones descriptas más adelante. De esta encuesta se obtendrán los 
usuarios que formarán parte de la población que evaluará la versión final de 
la solución.

\subsection{Muestra}

En el año $2014$, el \Gls{iab} cuenta con $124$ alumnos en el cuarto año distribuidos en
tres secciones, el cual es considerado el \emph{Universo}. De los 124, 93 de
ellos estuvieron interesados en participar de la prueba y completaron la encuesta.

\subsection{Variables}

Se definen $2$ factores necesarios para que un alumno pueda ser considerado como
sujeto de prueba, el primero es la predisposición del mismo a participar de la
prueba y el segundo es que posea un dispositivo móvil que supere los requisitos
mínimos. Además de estos dos factores, la conexión a internet es requerida, pues 
los registros de actividad de cada dispositivo deben ser enviados y almacenados 
para su posterior interpretación y análisis. A continuación se describen las variables 
consideradas.


\begin{itemize}

\item \textbf{Requisitos mínimos:} son aquellos requerimientos técnicos con los que 
    debe cumplir completamente el dispositivo móvil del usuario para que la 
    solución tenga un desempeño que garantice una experiencia fluida a la hora de 
    utilizarla. Estos requisitos son:
    \begin{itemize}
        %%\item Sistema Operativo Android $4.0$ o superior
        \item Memoria ram de $512$MB o superior.
        \item Velocidad de procesador de $800$ GHz o superior.
        \item \Gls{gpu} Mali 400 o superior.
        %\item Conexión frecuente a internet.
    \end{itemize}
    Los requisitos de \textit{hardware} mencionados, son requeridos por las
    características de la simulación, una \Gls{gpu} es requerida por los gráficos en 
    tres dimensiones.

\item \textbf{Tipo de acceso a internet:} el tipo de acceso a internet que posee el 
    usuario en su dispositivo móvil. Puede ser una de las siguientes opciones:
    plan post-pago, paquetes pre-pago, acceso ocasional y sin acceso.
    
\item \textbf{Sistema Operativo:} se refiere al tipo de sistema operativo que posee 
    el dispositivo móvil del usuario.

    \observacion{Donde se menciona?}
    
\end{itemize}

\subsection{Métricas}

Las métricas utilizadas para estudiar los datos recogidos son sencillas ya que
sólo buscan determinar la población que evaluará la solución, estas métricas son
las siguientes:

\begin{itemize}
\item Porcentaje de encuestados con dispositivos móviles que cumplen y que no cumplen con 
los requisitos mínimos.
\item Porcentaje del tipo de acceso a internet de los usuarios.
\item Porcentaje del tipo de sistema operativo que poseen los dispositivos móviles de los 
encuestados.
\end{itemize}


