%! TEX root = ../main.tex
\section{Encuesta Subjetiva}

En el análisis de los resultados, existieron alumnos que no respondieron todas
las preguntas, para tratar este tipo de casos, es importante analizar la
naturaleza del patrón de datos faltantes\cite{carpita2011imputation}. Existen
tres posibles formas de categorizar el patrón de ocurrencia de falta de
respuestas\cite{leite2010performance}\cite{leite2010performance}\cite{tsikriktsis2005review}:


\begin{description}
    \item[Información faltante completamente aleatoria] Cuando la información
        faltante es independiente de la variable medida y de otras variables.
    \item[Información faltante aleatoria] Cuando la información faltante depende
        de otras variables, pero no de la variable en sí. 
    \item[Información faltante no aleatoria] Cuando hay una relación entre la
        información faltante y el valor de la variable.
\end{description}

Los datos muestran que la información faltante es completamente aleatoria en
relación a la variable medida y a las demás variables, de hecho, una sola
encuesta tiene información faltante, así, se establece que el tipo de
información faltante es \emph{Información faltante completamente aleatoria}.

Existen tres mecanismos\cite{tsikriktsis2005review} principales para lidiar con
información faltante, eliminación, reemplazo, y procedimientos basados en modelo
(?Model-based procedure XXX).\cite{tsikriktsis2005review} recomienda utilizar
un mecanismo de reemplazo para escalas del tipo Likert.

Las técnicas de reemplazo se clasifican en tres grandes
grupos\cite{tsikriktsis2005review}:
\begin{enumerate*}[label=\itshape\alph*\upshape.]
\item basadas en la promedio,
\item basadas en regresión, y,
\item imputación \emph{hot deck}.
\end{enumerate*}

La sustitución basada por promedio, se divide nuevamente en tres grupos;
promedio
\begin{enumerate*}[label=\itshape\alph*\upshape.]
\item total,
\item del subgrupo, y,
\item por caso.
\end{enumerate*}
La sustitución del promedio total se realiza obteniendo el promedio de todas las
respuestas de esta pregunta, la sustitución de subgrupo es similar, solo que se
limita a aquellos sujetos del mismo subgrupo del sujeto que no respondió, y
finalmente, la sustitución por caso, es el promedio de las respuestas válidas
del sujeto.

En su resumen de las diferentes técnicas y cuando se deben utilizar cada una,
\cite{tsikriktsis2005review}, recomienda la utilización de la sustitución basada
en promedio por caso. 

De esta forma se reemplazan los valores faltantes en la encuesta, con el
promedio del sujeto.
