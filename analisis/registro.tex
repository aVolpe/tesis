%! TEX root = ../main.tex

\section{Registro de actividades}
\label{sec:registro}

%Las metodologías anteriormente descritas incluyen encuestas que miden
%el conocimiento del alumno y su opinión con respecto a la solución propuesta,
El registro de actividades ayuda a identificar las  fortalezas y debilidades 
de la solución en cuanto al diseño y utilidad. Para que los alumnos puedan formar 
una opinión válida acerca de la solución primero 
deben experimentar con la misma, para ello se instala la solución en los dispositivos 
móviles de los alumnos.

La instalación de la solución se lleva a cabo en el \Gls{iab}, se procede 
a mostrar un vídeo de la simulación, explicar la interfaz de usuario y realizar una muestra 
de como desenvolverse en el entorno. El período de prueba se extiende por 20 días, el mismo no es
asistido, es decir, existen factores que no pueden ser controlados, como:

\begin{itemize}
    \item Tiempo dedicado a la simulación por parte del alumno.
    \item Que todas las acciones provengan del alumno.
    \item Que sólo el conocimiento del alumno es puesto a prueba, es decir, no
        se puede controlar que no reciba ayuda externa.
\end{itemize}

Por estos motivos, el uso de la solución propuesta no puede ser considerado
el único factor relacionado con los resultados obtenidos en la \emph{Encuesta para medir el 
conocimiento}, cuyos resultados son mostrados más adelante en la sección~\ref{sec:objetiva}.
%de la encuesta objetiva
%descrita en~\ref{sec:objetiva}.

La solución propuesta almacena información relacionada a la actividad del
usuario, incluyendo cuándo y cómo realiza las acciones, los pasos que realiza,
el orden y las condiciones de la escena cuando realiza cada acción.

El registro como un todo es enviado cada vez que el usuario desee, este envío
requiere una conexión a internet, por ello no es automático. Adicionalmente el
último día de la prueba, todos los registros fueron enviados para que sean
analizados.



%Sobre todo, ayuda a medir 
%el impacto pedagógico al permitir contrarrestar el uso y desempeño del usuario 
%con el puntaje obtenido por el mismo en la \emph{Encuesta para medir el conocimiento}.

\subsection{Muestra}

La muestra está conformada por los $11$ alumnos que aceptaron formar parte de 
la prueba y poseen dispositivos móviles que cumplen con los requisitos
mínimos descritos en la sección~\ref{sec:ubicacion}.

La utilización de $11$ alumnos es suficiente, ya que según estudios presentados
en~\cite{nielsen2000}, mientras menos experiencia tengan los sujetos de estudio
con la solución planteada, serán necesarios menos para detectar un gran
porcentaje de errores y fortalezas, y según~\cite{ritch2009}, una base de $10$ a
$12$ es suficiente para obtener resultados estadísticamente válidos.

\subsection{Variables}

%La utilización de la solución, y el registro de las actividades genera una
%gran cantidad de información, los factores que se desean medir están
%relacionados a aquellos que pueden ser contrastados con los resultados de la
%encuesta objetiva.

Los registros de actividades nos permiten obtener información relevante acerca
de cómo se utilizó la solución y cuál fue el desempeño de los usuarios, las variables a medir son 
las siguientes:


\begin{description}

\item[Cantidad de partidas:] se define como el número de veces que un usuario
    inicia una escena. 

\item[Tiempo total:] es la suma del tiempo empleado en todas las partidas.

\item[Tiempo total de partidas por usuario y por tipo:] es el tiempo total 
    empleado para jugar las partidas discriminadas por tipo y por usuario.

\item[Cantidad de acciones:] es la cantidad total de acciones realizadas por 
    los usuarios.
 
\item[Cantidad de partidas realizadas por usuario y por tipo:] es el número de 
    partidas jugadas por usuario discriminado por el procedimiento al que 
    corresponde.

\item[Cantidad de usuarios:] es el número de usuarios que utilizaron la solución.
    
\item[Puntuación de las partidas:] dado el registro de reglas cumplidas en una partida 
    del procedimiento de venopunción o el diagnóstico dado por el usuario 
    en una partida del procedimiento de valoración de la escala de Glasgow, se 
    puede obtener el desempeño del usuario en las partida. 
%    Esto puede ser contrastado 
%    con la puntuación obtenida por el usuario en la \emph{Encuesta Objetiva}.

%\item[Puntuación por regla cumplida] Las variables definidas
%    en~\ref{sec:objetiva}, pueden ser contrastadas con la puntuación obtenida
%    por los alumnos en la simulación.

\end{description}

\subsection{Métricas}

Las métricas utilizadas para el análisis de los registros de actividades son las siguientes:

\begin{description}
\item[Promedio de tiempo por partida:] se obtiene dividiendo el tiempo total empleado 
    en las partidas por el número de partidas.
\item[Promedio de acciones por partida:] se obtiene dividiendo la cantidad total de 
    acciones realizas por los usuarios por el número de partidas.
\item[Promedio de partidas por usuario:] se obtiene dividiendo el número total de partidas
    por el número de usuarios que utilizaron la solución.
\item[Total de sesiones jugadas por tipo:] es la suma del número de partidas jugadas por los 
    usuarios discriminadas por tipo.
\item[Total de tiempo jugado por tipo:] es la suma de la cantidad de tiempo empleado en una 
    partida por los usuarios discriminado por tipo.
\item[Promedio de siguientes puntajes por tipo y por usuario:] se obtiene dividiendo la suma 
    de los puntajes obtenidos en cada tipo de escenario por la cantidad de veces que jugó el 
    usuario, a excepción de la primera vez.
\end{description}

%Además de las métricas descritas también se utiliza la correlación de Pearson como se 
%explica en \ref{sec:correlacion} para identificar las relaciones entre los datos obtenidos en 
%la \emph{Encuesta Objetiva} y los obtenidos en el \emph{Registro de actividades}.

\subsection{Resultados obtenidos}

%Las actividades de los usuarios son registradas y almacenadas para su análisis, a
%continuación se presentan los resultados de ese análisis, el mismo fue descrito
%en~\ref{sec:registro}, 

En la tabla~\ref{tab:log_total} se observa un resumen del experimento, en cuanto a tiempo, partidas y acciones.


\begin{table}[H]
\centering
\begin{tabular}{lrrrrrrrr}
%\toprule
%\textbf{Variable}                         & \textbf{Valor} \\
\toprule
Partidas                         & 99 \\
Primera partida					 & 4 de noviembre de 2014 \\
Última partida					 & 23 de noviembre de 2014 \\
\midrule
Tiempo total                     & 11134 s \\
Promedio de tiempo por partida   & 112 s \\
\midrule
Acciones                         & 2944 \\
Promedio de acciones por partida & 30 \\
\midrule
Usuarios                         & 8 \\
Promedio de partidas por usuario & 12 \\
\bottomrule
\end{tabular}
\caption{Resumen de la información extraída del registro de actividades}
\label{tab:log_total}
\end{table}

La cantidad de partidas jugadas por usuario en el procedimiento de venopunción, se muestra en la
tabla~\ref{tab:log_hemocultivo_partida}, se observa que existen $3$ alumnos que no
participaron de la prueba o no se registraron sus actividades.

\begin{table}[H]
\centering
\begin{tabular}{lrrrrrrrr}
\toprule
& \multicolumn{2}{c}{Venopunción} \\
\cmidrule(lr){2-3} 
Alumno   & Sesiones jugadas & Tiempo jugado (s) \\
\midrule
 1       & 5                & 1202 \\
 2       & 19               & 2507 \\
 4       & 5                & 398  \\
 5       & 6                & 768  \\
 6       & 17               & 2371 \\
 7       & 7                & 707  \\
 9       & 1                & 126  \\
10       & 8                & 960  \\
\midrule
Total   & 68               & 9039 \\
\bottomrule
\end{tabular}
\caption{Número de partidas y tiempo total por alumno en segundos, en la escena
    de venopunción.}
\label{tab:log_hemocultivo_partida}
\end{table}



Los registros pueden no ser registrados sí
\begin{enumerate*}[label=\itshape\alph*\upshape)]
    \item el usuario utilizó la solución, no envió los datos y, luego
        desinstaló la solución o borró los datos de la misma, o,
    \item el usuario no utilizó la solución.
\end{enumerate*}

En la tabla~\ref{tab:log_glasgow_random_partida}, se observa la cantidad de
sesiones y tiempo total por alumno, en la escena de \textit{Glasgow}, en modo de
evaluación. Se observa que $5$ alumnos participaron en $22$ sesiones, en total
jugaron $1768$ segundos. En cambio, en la tabla~\ref{tab:log_glasgow_custom_partida} se 
observa que $4$ alumnos participaron en $9$ sesiones y en total jugaron $327$ segundos. 
Los alumnos prefirieron jugar en el modo que les permitía diagnosticar al paciente.

\begin{table}[H]
\centering
\begin{tabular}{lrrrrrrrr}
\toprule
& \multicolumn{2}{c}{Glasgow (Evaluación)} \\
                   \cmidrule(lr){2-3} 
Número de alumno   & Sesiones jugadas                            & Tiempo jugado (s) \\
\midrule
1     & 4  & 211 \\
2     & 8  & 738 \\
4     & 3  & 132 \\
6     & 1  & 97  \\
7     & 6  & 590 \\
\midrule
Total & 22 & 1768 \\
\bottomrule
\end{tabular}
\caption{Número de partidas y tiempo total por alumno en segundos, en la escena
    \textit{Glasgow}, en modo evaluación}
\label{tab:log_glasgow_random_partida}
\end{table}


\begin{table}[H]
\centering
\begin{tabular}{lrrrrrrrr}
\toprule
& \multicolumn{2}{c}{Glasgow (Exploración)} \\
                   \cmidrule(lr){2-3} 
Número de alumno   & Sesiones jugadas                            & Tiempo jugado (s) \\
\midrule
1        & 2 & 79 \\
2        & 3 & 80 \\
4        & 3 & 89 \\
6        & 1 & 79 \\
\midrule
Total   & 9 & 327 \\
\bottomrule
\end{tabular}
\caption{Número de partidas y tiempo total por alumno en segundos, en la escena
    \textit{Glasgow}, en modo exploración}
\label{tab:log_glasgow_custom_partida}
\end{table}


En las tablas~\ref{tab:log_hemocultivo_puntaje}
y~\ref{tab:log_glasgow_random_puntaje} se muestran los primeros puntajes y un
promedio de los puntajes siguientes obtenidos por cada alumno en los
procedimientos de venopunción y de la evaluación de la escala de
Glasgow. Se debe tener en cuenta el tiempo y la cantidad de veces que cada
alumno jugó cada uno de los procedimientos para valorar los resultados
mostrados. 

\begin{table}[H]
\centering
\begin{tabular}{lrrrrrrrr}
\toprule
& \multicolumn{2}{c}{Venopunción} \\
\cmidrule(lr){2-3} 
Número de alumno  & Primer Puntaje & Siguientes Puntajes \\
\midrule
 1                & 11             & 14.3 \\
 2                & 9              & 10.6 \\
 4                & 3              & 3.3  \\
 5                & 3              & 6.8  \\
 6                & 3              & 5.8  \\
 7                & 4              & 4    \\
 9                & 16             & \\
10                & 3              & 7.2  \\
\midrule
\textbf{Promedio} & 6.5            & 7.42 \\
\bottomrule
\end{tabular}
\caption{Puntaje obtenido la primera vez y el promedio de los puntajes de las siguientes veces
    por alumno, en la escena de venopunción}
\label{tab:log_hemocultivo_puntaje}
\end{table}


\begin{table}[H]
\centering
\begin{tabular}{lrrrrrrrr}
\toprule
& \multicolumn{2}{c}{Glasgow (Evaluación)} \\
                   \cmidrule(lr){2-3} 
Número de alumno   & Primer Puntaje & Siguientes Puntajes \\
\midrule
1     & 1 & 1.5 \\
2     & 2 & 2.3 \\
4     & 1 & 1.5 \\
6     & 2 & 2 \\
7     & 0 & 1 \\
\midrule
\textbf{Promedio} & 1.2 & 1.66 \\
\bottomrule
\end{tabular}
\caption{Puntaje obtenido la primera vez y el promedio de los puntajes de las siguientes veces
    por alumno, en la escena \textit{Glasgow}, en modo evaluación}
\label{tab:log_glasgow_random_puntaje}
\end{table}

En las tablas~\ref{tab:log_hemocultivo_puntaje}
y~\ref{tab:log_glasgow_random_puntaje} se observa que los alumnos que participaron de la prueba
mejoran su desempeño a medida que aumenta el número de partidas. 

%Es importante notar que la cantidad de partidas no es uniforme entre los
%alumnos, es decir hay alumnos con más de $10$ partidas y alumnos con menos de
%$5$, por ello, no es posible demostrar que existe un progreso a medida que
%aumenta el número de partidas.

Por último, en la figura~\ref{fig:utilizacion_hora} se puede observar la distribución de 
las partidas por hora del día. Los picos de uso de la solución se dan a las $13:00$ horas, 
horario de almuerzo, a las $17:00$ horas, fin de las actividades académicas y a las $01:00$ horas, 
horario libre. De esta manera los datos muestran que el mayor uso de la solución se da en los 
horarios libres de los alumnos, es decir, los alumnos deciden usar la solución en su tiempo libre.

\begin{filecontents}{utilizaciontiempo.dat}
Hora	Cantidad
 0	 0
 1	 9
 2	 4
 3	 2
 4	 0
 5	11
 6	 8
 7	 3
 8	 5
 9	 7
10	 7
11	12
12	15
13	 9
14	 6
15	 0
16	 0
17	 0
18	 0
19	 0
20	 0
21	 0
22	 0
23	 1
24	 0
\end{filecontents}
\pgfplotstableread{utilizaciontiempo.dat}{\UtilizacionTiempo}

\begin{figure}[H]
	\centering
    \begin{tikzpicture}[thick, scale=1.2]
        \begin{axis}[
            title={},
            xlabel={Hora},
            ylabel={Partidas},
            xmin=0, xmax=24,
            ymin=0, ymax=16,
            xtick       = {0  , 3  , 6  , 9  , 12 , 15 , 18 , 21 , 24},
            xticklabels = {12 , 15 , 18 , 21 ,  0 , 3  ,  6 ,  9 , 12},
            %ytick={0,.25,.50,.75,1},
            legend pos=north east,
            ymajorgrids=true,
            xmajorgrids=true,
            grid style=dashed,
        ]
         
        \addplot[color=blue,fill=blue!5] table [x = {Hora}, y = {Cantidad}] {\UtilizacionTiempo};
        \end{axis}
    \end{tikzpicture}
    \caption{Utilización de la solución por hora del día}
    \label{fig:utilizacion_hora}
\end{figure}
