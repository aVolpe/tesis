%! TEX root = ../main.tex

\section{Registro de actividades}
\label{sec:registro}

Las metodologías anteriormente descritas incluyen encuestas que miden
el conocimiento del alumno y su opinión con respecto a la solución propuesta,
para poder formar una opinión válida primero deben experimentar con la misma, para
ello se instala la solución en los dispositivos móviles de los alumnos que forman 
parte de la población objetivo.

La instalación de la solución se lleva a cabo en el \Gls{iab}, y se procede 
a mostrar un vídeo de la simulación, explicar la interfaz y realizar una muestra 
de como desenvolverse en el entorno.

El período de prueba se extiende por 20 días, el mismo no es
\fixme{controlado}{Asistido}, es decir que existen factores que no pueden ser
controlados, como:

\begin{itemize}
    \item Tiempo dedicado a la simulación.
    \item Que todas las acciones provengan del alumno.
    \item Solamente el conocimiento del alumno es puesto a prueba, es decir, no
        se puede controlar que no reciba ayuda externa.
\end{itemize}

Por estos motivos, el uso de la solución propuesta no puede ser considerado
el único factor relacionado con los resultados de la encuesta objetiva
descrita en~\ref{sec:objetiva}.

La solución propuesta almacena información relacionada a la actividad del
usuario, incluyendo cuando y como utiliza las acciones, los pasos que realiza,
el orden y las condiciones de la escena cuando realiza cada acción.

El registro como un todo es enviado cada vez que el usuario desee, este envío
requiere una conexión a internet por ello no es automático. Adicionalmente el
último día de la prueba, todos los registros fueron enviados para que sean
analizados.

El registro de actividades ayuda a identificar las  fortalezas y debilidades 
de la solución en cuanto al diseño y utilidad. Sobre todo, ayuda a medir 
el impacto pedagógico al permitir contrarrestar el uso y desempeño del usuario 
con el puntaje obtenido por el mismo en la \emph{Encuesta Objetiva}.

\subsection{Muestra}

La muestra esta conformada por los $11$ alumnos que aceptaron formar parte de 
la prueba y poseían dispositivos móviles que cumplen con los requisitos
mínimos.

\subsection{Variables}

La utilización de la simulación, y el registro de las actividades genera una
gran cantidad de información, los factores que se desean medir están
relacionados a aquellos que pueden ser contrastados con los resultados de la
encuesta objetiva.

La utilización de la simulación nos permite obtener información relevante acerca
de como se utilizo la misma, se definen los criterios a medir:

\begin{description}

\item[Cantidad de partidas:] se define como la cantidad de veces que un alumno
    inicia una escena. 

\item[Tiempo total:] es la suma del tiempo empleado en todas las partidas.

\item[Tiempo total de partidas por usuario y por tipo:] es el tiempo total 
    empleado para jugar las partidas discriminadas por tipo y por usuario.

\item[Cantidad de acciones:] es la cantidad total de acciones realizadas por 
    los usuarios.
 
\item[Cantidad de partidas realizadas por usuario y por tipo:] es el número de 
    partidas jugadas por usuario discriminado por el procedimiento al que 
    corresponde.

\item[Cantidad de usuarios:] es el número de usuarios que utilizaron la solución.
    
\item[Puntuación de las partidas:] Dado el registro de reglas cumplidas en una partida 
    del procedimiento de extracción de sangre o el diagnóstico dado por el usuario 
    en una partida del procedimiento de valoración de la escala de Glasgow, se 
    puede obtener el desempeño del usuario en las partida. Esto puede ser contrastado 
    con la puntuación obtenida por el usuario en la \emph{Encuesta Objetiva}.

%\item[Puntuación por regla cumplida] Las variables definidas
%    en~\ref{sec:objetiva}, pueden ser contrastadas con la puntuación obtenida
%    por los alumnos en la simulación.

\end{description}

\subsection{Métricas}

\begin{description}
\item[Promedio de tiempo por partida:] se obtiene dividiendo el tiempo total empleado 
    en las partidas por el número de partidas.
\item[Promedio de acciones por partida:] se obtiene dividiendo el cantidad total de 
    acciones realizas por el usuario por el número de partidas.
\item[Promedio de partidas por usuario:] se obtiene dividiendo el número total de partidas
    por el número de usuarios que utilizaron la solución.
\item[Total de sesiones jugadas por tipo:] es la suma de la cantidad de partidas jugadas por 
    usuario y tipo.
\item[Total de tiempo jugado por tipo:] es la suma de la cantidad de tiempo empleado en una 
    partida por usuario y tipo.
\item[Promedio de siguientes puntajes por tipo y por usuario:] se obtiene dividiendo la suma 
    de los puntajes obtenidos en cada tipo de escenario por la cantidad de veces que jugó el 
    usuario, a excepción de la primera vez.
\end{description}

Además de las métricas descriptas también se utiliza la correlación de Pearson como se 
explica en \ref{sec:correlacion} para identificar las relaciones entre los datos obtenidos en 
la \emph{Encuesta Objetiva} y los obtenidos en el \emph{Registro de actividades}.

\subsection{Resultados}

Las actividad de los usuarios es registrada y almacenada para su análisis, a
continuación se presentan los resultados de ese análisis, el mismo fue descrito
en~\ref{sec:registro}, en la tabla~\ref{tab:log_total} se observa un resumen del
experimento, en cuanto a tiempo, partidas y acciones.


\begin{table}[H]
\centering
\begin{tabular}{lrrrrrrrr}
\toprule
\textbf{Variable}                         & \textbf{Valor} \\
\midrule
Tiempo total                     & 11134\tabletodo{Seguro?} \\
Partidas                         & 99 \\
Acciones                         & 2944 \\
Promedio de tiempo por partida   & 112 \\
Promedio de acciones por partida & 30 \\
Promedio de partidas por usuario & 12,37 \\
Usuarios                         & 8 \\
\bottomrule
\end{tabular}
\caption{Resumen de la información extraída del registro de actividades.}
\label{tab:log_total}
\end{table}

\observacion{Falta unidad de tiempo}
\begin{table}[H]
\centering
\begin{tabular}{lrrrrrrrr}
\toprule
& \multicolumn{2}{c}{Extracción de sangre} \\
\cmidrule(lr){2-3} 
Alumno   & Sesiones jugadas & Tiempo jugado \\
\midrule
 1       & 5                & 1202 \\
 2       & 19               & 2507 \\
 4       & 5                & 398  \\
 5       & 6                & 768  \\
 6       & 17               & 2371 \\
 7       & 7                & 707  \\
 9       & 1                & 126  \\
10       & 8                & 960  \\
\midrule
Total   & 68               & 9039 \\
\bottomrule
\end{tabular}
\caption{Número de partidas y tiempo total por alumno en segundos, en la escena
    de extracción de sangre.}
\label{tab:log_hemocultivo_partida}
\end{table}

La cantidad de partidas jugadas por usuario, se ven en la
tabla~\ref{tab:log_hemocultivo_partida}, se observa que existen $3$ alumnos que no
participaron de la prueba o no se registro su actividad.

Los registros pueden no ser registrados sí
\begin{enumerate*}[label=\itshape\alph*\upshape)]
    \item el usuario utilizo la solución, no envió los datos y, luego
        desinstalo la solución o borro los datos de la misma, o,
    \item el usuario no utilizo la solución.
\end{enumerate*}

En la tabla~\ref{tab:log_glasgow_random_partida}, se observa la cantidad de
sesiones y tiempo total por alumno, en la escena de \textit{Glasgow}, en modo de
evaluación. Se observa que $5$ alumnos participaron en $22$ sesiones, en total
jugaron $1768$ segundos.

\begin{table}[H]
\centering
\begin{tabular}{lrrrrrrrr}
\toprule
& \multicolumn{2}{c}{Glasgow (Evaluación)} \\
                   \cmidrule(lr){2-3} 
Número de alumno   & Sesiones jugadas                            & Tiempo jugado \\
\midrule
1     & 4  & 211 \\
2     & 8  & 738 \\
4     & 3  & 132 \\
6     & 1  & 97  \\
7     & 6  & 590 \\
\midrule
Total & 22 & 1768 \\
\bottomrule
\end{tabular}
\caption{Número de partidas y tiempo total por alumno en segundos, en la escena
    \textit{Glasgow}, en modo evaluación}
\label{tab:log_glasgow_random_partida}
\end{table}


\begin{table}[H]
\centering
\begin{tabular}{lrrrrrrrr}
\toprule
& \multicolumn{2}{c}{Glasgow (Exploración)} \\
                   \cmidrule(lr){2-3} 
Número de alumno   & Sesiones jugadas                            & Tiempo jugado \\
\midrule
1        & 2 & 79 \\
2        & 3 & 80 \\
4        & 3 & 89 \\
6        & 1 & 79 \\
\midrule
Total   & 9 & 327 \\
\bottomrule
\end{tabular}
\caption{Número de partidas y tiempo total por alumno en segundos, en la escena
    \textit{Glasgow}, en modo exploración}
\label{tab:log_glasgow_custom_partida}
\end{table}


En las tablas~\ref{tab:log_hemocultivo_puntaje}
y~\ref{tab:log_glasgow_random_puntaje} se muestran los primeros puntajes y un
promedio de los puntajes siguientes obtenidos por cada usuario en los
procedimientos de extracción de sangre y de la evaluación de la escala de
Glasgow. Se debe tener en cuenta el tiempo y las cantidades de veces que cada
alumno jugó cada uno de los procedimientos para valorar los resultados
mostrados. 

\begin{table}[H]
\centering
\begin{tabular}{lrrrrrrrr}
\toprule
& \multicolumn{2}{c}{Extracción de sangre} \\
\cmidrule(lr){2-3} 
Número de alumno  & Primer Puntaje & Siguientes Puntajes \\
\midrule
 1                & 11             & 14.3 \\
 2                & 9              & 10.6 \\
 4                & 3              & 3.3  \\
 5                & 3              & 6.8  \\
 6                & 3              & 5.8  \\
 7                & 4              & 4    \\
 9                & 16             & \\
10                & 3              & 7.2  \\
\midrule
\textbf{Promedio} & 6.5            & 7.42 \\
\bottomrule
\end{tabular}
\caption{Puntaje obtenido la primera vez y el promedio de las siguientes veces
    por alumno, en la escena de extracción de sangre.}
\label{tab:log_hemocultivo_puntaje}
\end{table}


\begin{table}[H]
\centering
\begin{tabular}{lrrrrrrrr}
\toprule
& \multicolumn{2}{c}{Glasgow (Evaluación)} \\
                   \cmidrule(lr){2-3} 
Número de alumno   & Primer Puntaje & Siguientes Puntajes \\
\midrule
1     & 1 & 1.5 \\
2     & 2 & 2.3 \\
4     & 1 & 1.5 \\
6     & 2 & 2 \\
7     & 0 & 1 \\
\midrule
\textbf{Promedio} & 1.2 & 1.66 \\
\bottomrule
\end{tabular}
\caption{Puntaje obtenido la primera vez y el promedio de las siguientes veces
    por alumno, en la escena \textit{Glasgow}, en modo evaluación}
\label{tab:log_glasgow_random_puntaje}
\end{table}

Se observa en las tablas~\ref{tab:log_hemocultivo_puntaje}
y~\ref{tab:log_glasgow_random_puntaje} los alumnos que participaron de la prueba
mejoran su desempeño a medida que aumenta el número de partidas. 

Es importante notar que la cantidad de partidas no es uniforme entre los
alumnos, es decir hay alumnos con más de $10$ partidas y usuarios con menos de
$5$, por ello, no es posible demostrar que existe un progreso a medida que
aumenta el número de partidas.

