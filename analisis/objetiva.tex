%! TEX root = ../main.tex

\section{Encuesta objetiva}
\label{sec:objetiva}

A fin de obtener información acerca del conocimiento de los alumnos que utilizaron 
la solución propuesta y los que no la utilizaron, los cuales constituyen el grupo 
de control, se realiza una encuesta que consta de diez preguntas.

La encuesta mide el nivel de conocimiento del alumno sobre los dos temas
simulados, contiene preguntas de nivel básico, medio y avanzado. Las mismas son
formuladas utilizando la lista de competencias básicas que debe tener un alumno
para aprobar la materia \textbf{Enfermería en Urgencias II}. Las preguntas son
verificadas  por los profesores de la cátedra. Cada pregunta tiene el mismo
peso, así la puntuación más baja obtenible es $0$, y la más alta es $10$.

De esta manera se busca evaluar la influencia pedagógica y la utilidad de la 
solución como herramienta de apoyo al aprendizaje.


\subsection{Muestra}
\observacion{Se repite mucho lo de las muestras hay 2 universos nomas?}

El universo cuenta con $124$ alumnos, de los cuales $11$ son la muestra seleccionada
para la prueba, y los $113$ alumnos restantes son utilizados
como grupo de control.

\subsection{Variables}

Se busca medir el puntaje total de los alumnos en la \emph{Encuesta objetiva}. Esto 
se obtiene de la siguiente manera.

Siendo:

\begin{itemize}
    \item $po_i{_k}$ la respuesta del usuario $i$ a la pregunta $k$
    \item $n$ total de preguntas, es igual a 10
    \item $tc$ total de alumnos en el grupo de control, que es igual a 113.
    \item $t$ total de alumnos, que es 124
    \item $ts$ total de sujetos de estudio, que es igual a 11.
\end{itemize}

Se define el puntaje total de de cada alumno $pto_i$ del alumno $i$ como, 

\begin{equation*}
    pto_i = \sum_{j=1}^n{po_i{_j}}
\end{equation*}


\subsection{Métricas}

Como se mencionó, la \emph{Encuesta Objetiva} busca medir el rendimiento de los 
alumnos, para ello se utiliza como métrica principal el promedio de acierto, 
tanto del conjunto total de alumnos, como de los que participaron de la
prueba, y del grupo de control.

Se define el promedio total de los alumnos, $promtotal$ como:

\begin{equation*}
    promtotal = \frac{\sum_{i=1}^t{pto_i}}{t}
\end{equation*}

Se obtienen los promedios del grupo de control ($promcontrol$) y del grupo de alumno que
participaron en la prueba ($promsujetos$) de la misma manera.

\subsection{Resultados}
\label{sec:res_objetiva}

Como se detalló en la sección~\ref{sec:objetiva}, la encuesta realizada a cada
usuario, parte de la prueba, es utilizada para obtener una comparación en cuanto
al rendimiento de los usuarios que forman parte de la muestra y los que forman
parte del grupo de control.


\observacion{A esta altura ya no se entiende que es promcontrol}
\observacion{No estaría mal poner algún tipo de información que diga a que
aspectos se relacionad cada pregunta}
\begin{table}[!hbt]
\centering
\begin{tabular}{lrrr}
\toprule
\textbf{Pregunta} & 
\textbf{promsujetos} & 
\textbf{promcontrol} & 
\textbf{promtotal} \\ 
\midrule
1         & 0.36 & 0.18 & 0.20 \\
2         & 0.64 & 0.60 & 0.60 \\
3         & 0.09 & 0.14 & 0.13 \\
4         & 0.27 & 0.25 & 0.26 \\
5         & 0.82 & 0.56 & 0.59 \\
6         & 0.00 & 0.18 & 0.16 \\
7         & 0.64 & 0.51 & 0.53 \\
8         & 0.45 & 0.28 & 0.29 \\
9         & 0.18 & 0.32 & 0.31 \\
10        & 0.36 & 0.45 & 0.45 \\
\midrule
\textbf{Promedio}: & 3.82 & 3.47 & 3.49  \\
\bottomrule
\end{tabular}
\caption{Rendimiento promedio de usuarios por pregunta}
\label{tab:objetiva_rendimiento_por_pregunta}
\end{table}

La tabla~\ref{tab:objetiva_rendimiento_por_pregunta} muestra el nivel de acierto
en promedio por pregunta de los usuarios que forman parte de la muestra y de los
que forman parte del grupo de control, con sus respectivas desviaciones
estándar. Según estos datos, en el $60\%$ de los casos hay una leve mejoría en
cuanto al nivel de acierto para los usuarios que forman parte de la muestra.

Los datos sólo sugieren levemente una tendencia a la mejoría de los puntajes
para los usuarios que forman parte de la muestra, sin embargo, estos datos no
pueden ser tomados para realizar conclusiones ya que la cantidad de sesiones de
juego por usuario no se considera suficiente para que el uso de la solución
propuesta afecte realmente en el aprendizaje del mismo.
