\section{Correlación entre variables}
\label{sec:correlacion}

En esta sección se busca analizar las relaciones que puedan haber entre 
el uso de la solución y el rendimiento en la encuesta para evaluar el conocimiento, 
considerando sólo a los alumnos que participaron de la prueba de la solución.

En la tabla~\ref{tab:all_correlation} se observa la correlación entre seis
variables estudiadas, a fin de observar si existe alguna relación entre los
valores, se utiliza la correlación de \emph{Pearson}, descrita
en~\ref{sec:def_correlacion}. Las variables corresponden al \emph{Registro de
    actividades} y a los resultados de la \emph{Encuesta para medir el
    conocimiento}.


Las correlaciones  más significativas mostradas en la
tabla~\ref{tab:all_correlation}, son:

\begin{itemize}

\item Puntaje máximo obtenido en el procedimiento de venopunción en la solución
    y tiempo  jugado en el procedimiento de venopunción, $0.30$, relación
    positiva moderada. Así como una correlación positiva fuerte ($0,61$) entre
    puntaje máximo obtenido en el procedimiento de \textit{Glasgow} (evaluación)
    y tiempo jugado en el procedimiento de \textit{Glasgow} (evaluación).
    
    Esto podría sugerir que mientras más se utiliza la solución, mejor
    rendimiento se obtiene. Es un punto positivo pues muestra que los usuarios
    aprenden a utilizarla y mejoran con el tiempo.

\item Puntaje máximo obtenido en el procedimiento de venopunción en la solución
    y puntaje obtenido en el examen en lo referente a venopunción, $0.74$,
    relación positiva muy fuerte. Así como una correlación positiva fuerte
    ($0,54$) entre el puntaje máximo obtenido en el procedimiento de
    \textit{Glasgow} (evaluación) y puntaje obtenido en el examen en lo
    referente a \textit{Glasgow}.
    
    Esto podría sugerir que los alumnos con mejor rendimiento en la solución,
    obtuvieron el mejor rendimiento en la evaluación.

\item Tiempo jugado en el procedimiento de \textit{Glasgow} (evaluación) y
    puntaje obtenido en el examen en lo referente a \textit{Glasgow}, $0.86$,
    relación positiva muy fuerte. Lo que puede sugerir que los usuarios que más
    tiempo invirtieron en el procedimiento \textit{Glasgow}, también obtuvieron
    mejor puntaje en el examen.

\item Existe una correlación positiva moderada ($0,29$) entre el tiempo de
    utilización del procedimiento Venopunción, y la utilización del
    procedimiento \textit{Glasgow}, lo que sugiere que los usuarios dedicaron un
    tiempo similar en ambos procedimientos.

\item Existe una correlación positiva moderada ($0.35$) entre el puntaje mayor
    en el procedimiento Venopunción, y el tiempo de juego en el procedimiento
    \textit{Glasgow}, lo que parece indicar que los usuarios que completaron la
    mayor parte del procedimiento Venopunción, dedicaron más tiempo al
    procedimiento \textit{Glasgow}. 

\item Puntaje obtenido en el examen en lo referente a venopunción y puntaje
    obtenido en el examen en lo referente a \textit{Glasgow}, $0.78$, relación
    positiva muy fuerte. Esto podría sugerir que el nivel de conocimiento de los
    alumnos sobre ambos procedimientos está relacionado.

\end{itemize}

\begin{table}[H]
\centering

\begin{tabular}{lrrrrrr}
\toprule
        &
\begin{sideways}\textbf{Puntaje Máx Venopunción (juego)}\end{sideways}  &
\begin{sideways}\textbf{Puntaje Máx Glasgow (juego)}\end{sideways}        &
\begin{sideways}\textbf{Tiempo Jugado Venopunción}\end{sideways}         &
\begin{sideways}\textbf{Tiempo Jugado Glasgow}\end{sideways} &
\begin{sideways}\textbf{Puntaje Venopunción (examen)}\end{sideways}  &
\begin{sideways}\textbf{Puntaje Glasgow (examen)}\end{sideways}    \\
\midrule
Puntaje Máx Venopunción (juego)   & 1             & 0.12          & \textbf{0.30} & \textbf{0.35} & \textbf{0.74} & 0.55 \\
Puntaje Máx Glasgow (juego)       & 0.12          & 1             & 0.32          & \textbf{0.61} & 0             & \textbf{0.54}\\
Tiempo Jugado Venopunción         & \textbf{0.30} & 0.32          & 1             & 0.29          & 0.04          & 0.05\\
Tiempo Jugado Glasgow             & \textbf{0.35} & \textbf{0.61} & 0.29          & 1             & 0.69          & \textbf{0.86}\\
Puntaje Prom Venopunción (examen) & \textbf{0.74} & 0             & 0.04          & 0.69          & 1             & \textbf{0.78} \\
Puntaje Prom Glasgow (examen)     & 0.55          & \textbf{0.54} & 0.05          & \textbf{0.86} & \textbf{0.78} & 1 \\
\bottomrule               
\end{tabular}
\caption{Correlación entre factores estudiados} 
\label{tab:all_correlation}
\end{table}
