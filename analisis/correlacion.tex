\section{Correlación entre variables}
\label{sec:correlacion}

En esta sección se busca analizar las relaciones que puedan haber entre 
el uso de la solución y el rendimiento en la encuesta para evaluar el conocimiento, 
considerando sólo a los alumnos que participaron de la prueba de la solución.

En la tabla~\ref{tab:all_correlation} se observa la correlación entre seis 
variables estudiadas, a fin de observar si existe alguna relación entre los
valores, se utiliza la correlación de \emph{Pearson}, descrita
en~\ref{sec:correlacion}. Las variables corresponden al \emph{Registro de actividades} 
y a los resultados de la \emph{Encuesta para medir el conocimiento}.

%\begin{table}[H]
%\centering
%\begin{tabular}{lrrrrrr}
%\toprule
%        &
%\begin{sideways}\textbf{Tiempo de Uso}\end{sideways}             &
%\begin{sideways}\textbf{Encuesta solución}\end{sideways}        &
%\begin{sideways}\textbf{Encuesta conocimiento}\end{sideways}         &
%\begin{sideways}\textbf{Puntaje Máximo Extracción}\end{sideways} &
%\begin{sideways}\textbf{Puntaje Máximo Glasgow}\end{sideways}    \\
%\midrule
%Tiempo de Uso             & 1    & -0.2  & 0.15  & 0.62 & 0.41 & 0.78 \\
%Encuesta solución         & -0.2 & 1     & -0.07 & 0.04 & 0.11 & -0.28\\
%Encuesta conocimiento     & 0.15 & -0.07 & 1     & 0.44 & 0.44 & 0.02 \\
%Puntaje máximo Extracción & 0.62 & 0.04  & 0.44  & 1    & 0.96 & 0.44 \\
%Puntaje máximo Glasgow    & 0.41 & 0.11  & 0.44  & 0.96 & 1    & 0.27 \\
%\bottomrule               & 0.78 & -0.28 & 0.02  & 0.44 & 0.27 & 1    \\
%\end{tabular}
%\caption{Correlación entre factores estudiados} 
%\label{tab:all_correlation}
%\end{table}

%    \item Tiempo de uso y puntaje máximo extracción, $0,62$, correlación
%        positiva fuerte.
%    \item Tiempo de uso y puntaje máximo Glasgow, $0,78$, correlación positiva
%        muy fuerte.
%    \item Puntaje máximo extracción y encuesta objetiva, $0,44$, correlación
%        positiva fuerte.

Las correlaciones  más significativas mostradas en la 
tabla~\ref{tab:all_correlation}, son:

\begin{itemize}
	\item Puntaje máximo obtenido en el procedimiento de venopunción en la solución y tiempo  
	jugado en el procedimiento de venopunción, $0.30$, relación positiva moderada.
	\item Puntaje máximo obtenido en el procedimiento de venopunción en la solución y puntaje 
	obtenido en el examen en lo referente a venopunción, $0.74$, relación positiva muy fuerte.
	\item Puntaje máximo obtenido en el procedimiento de Glasgow (evaluación) y tiempo 
	jugado en el procedimiento de Glasgow (evaluación), $0.61$, relación positiva fuerte.
	\item Puntaje máximo obtenido en el procedimiento de Glasgow (evaluación) y puntaje obtenido en 
	el examen en lo referente a Glasgow, $0.54$, relación positiva fuerte.
	\item Tiempo jugado en el procedimiento de Glasgow (evaluación) y puntaje obtenido en el 
	examen en lo referente a Glasgow, $0.86$, relación positiva muy fuerte.
	\item Puntaje obtenido en el examen en lo referente a venopunción y tiempo jugado en el 
	procedimiento de Glasgow (evaluación), $0.35$, relación positiva moderada.
	\item Puntaje obtenido en el examen en lo referente a venopunción y puntaje obtenido en el examen 
	en lo referente a Glasgow, $0.78$, relación positiva muy fuerte.
\end{itemize}


%La tabla~\ref{tab:all_correlation} indica que existe una correlación positiva
%fuerte ($0,62$ y $0,78$) entre el tiempo de uso y el puntaje más alto obtenido,
%\fixme{lo que sugiere que mientras más se utiliza la solución, se obtienen mejores
%resultados}{Guardaaaa correlación no es igual que casualidad}. 
%
%Una correlación positiva fuerte entre el puntaje máximo obtenido en la
%Extracción y la encuesta objetiva ($0,44$), sugiere que existe una relación entre
%el nivel de conocimientos de los alumnos y su desempeño en la práctica.

\begin{table}[H]
\centering

\begin{tabular}{lrrrrrr}
\toprule
        &
\begin{sideways}\textbf{Puntaje Máx Venopunción (juego)}\end{sideways}  &
\begin{sideways}\textbf{Puntaje Máx Glasgow (juego)}\end{sideways}        &
\begin{sideways}\textbf{Tiempo Jugado Venopunción}\end{sideways}         &
\begin{sideways}\textbf{Tiempo Jugado Glasgow}\end{sideways} &
\begin{sideways}\textbf{Puntaje Venopunción (examen)}\end{sideways}  &
\begin{sideways}\textbf{Puntaje Glasgow (examen)}\end{sideways}    \\
\midrule
Puntaje Máx Venopunción (juego)    & 1    & 0.12  & \textbf{0.30}   & \textbf{0.35} & \textbf{0.74} & 0.55 \\
Puntaje Máx Glasgow (juego)       & 0.12 & 1     & 0.32 & \textbf{0.61} & 0 & \textbf{0.54}\\
Tiempo Jugado Venopunción     		 & \textbf{0.30}  & 0.32 & 1  & 0.29 & 0.04 & 0.05\\
Tiempo Jugado Glasgow 				 & \textbf{0.35} & \textbf{0.61}  & 0.29  & 1    & 0.69 & \textbf{0.86}\\
Puntaje Prom Venopunción (examen) & \textbf{0.74} & 0 	& 0.04  & 0.69 & 1 & \textbf{0.78} \\
Puntaje Prom Glasgow (examen)    		 & 0.55 & \textbf{0.54} & 0.05  & \textbf{0.86} & \textbf{0.78} & 1 \\
\bottomrule               
\end{tabular}
\caption{Correlación entre factores estudiados} 
\label{tab:all_correlation}
\end{table}
