\section{Prueba preliminar de usabilidad}
\label{sec:interfaz}

Durante el desarrollo de la solución se realizó una prueba para evaluar la 
interfaz de usuario, específicamente buscando la retroalimentación de usuarios 
acostumbrados a tecnología similar a la utilizada en la solución.

Esta prueba ayuda en el proceso de diseño e implementación de la solución con 
las características mencionadas en los objetivos del trabajo y acorde a los 
requerimientos. De esta manera se pueden identificar los aspectos que deben 
ser mejorados.

La prueba consta de dos partes importantes involucradas en la recolección
de datos para su posterior análisis. Estas partes son las siguientes:

\begin{description}

\item[Simulación:] luego de una explicación acerca de las funciones y manejos
    generales de la solución por parte de los encargados de la prueba, cada usuario
    completa una tarea que consiste en realizar el procedimiento de venopunción con la 
    solución, como ayuda, recibe una hoja con una lista de todos los pasos 
    necesarios para llevar a cabo el procedimiento.
    	
    Las simulaciones son grabadas con programas de captura de pantalla, así
    como por detectores de eventos táctiles.
    	
\item[Encuesta:] posteriormente se le provee una encuesta a cada
    usuario la cual es utilizada para obtener una idea general acerca de la
    calidad de la simulación según la percepción de los usuarios. Esta encuesta 
    contiene preguntas que son medidas mediante la escala de tipo Likert. 

\end{description} 

\subsection{Muestra}

La prueba de usabilidad de la interfaz de usuario se realiza con alumnos de la
carrera de Ingeniería en Informática de la \Gls{fpuna}, sin experiencia previa
tanto con la solución como con los procedimientos simulados, pero sí
familiarizados con la utilización de dispositivos móviles. La muestra no
requiere de sujetos que sean parte del \emph{población objetivo} ya que sólo
está orientada a mejorar aspectos de interfaz de usuario y no el contenido de la
solución, además se considera que la muestra puede brindar una evaluación más
crítica debido a su familiarización con interfaces similares a la de la
solución.

El número de muestras tomadas fue 8, ya que según~\cite{nielsen2000} son
necesarios al menos $5$ participantes para poder obtener resultados
significativos en una prueba de usabilidad. Además,~\cite{ritch2009} asegura que
la teoría de~\cite{nielsen2000} es verdadera especialmente para pruebas simples. 

Se fundamenta el número de participantes, y que es una prueba sencilla, ya que:

\begin{itemize}

\item La prueba no debería tomar más de $10$ minutos en ser realizada.

\item Se busca solamente obtener información acerca de la interfaz, y no el
    funcionamiento en sí de la simulación, pues los usuarios no son expertos en
    el área y no tienen conocimiento acerca las tareas.

\item No se busca evaluar el aspecto pedagógico de la solución sino sólo su interfaz gráfica.
%\item No se busca medir el aprendizaje del usuario en temas no relacionados a la
%    interfaz, es decir, no se mide el aprendizaje del usuario en el tema
%    simulado\revisar{No se entiende, no se mide el aspecto pedagógico solo la
%    interfaz gráfica de la simulación}.

\item El procedimiento de enfermería a realizarse con la solución está bien definido 
y los pasos necesarios están a disposición del usuario en todo momento.

\end{itemize}

\subsection{Variables}
\label{sec:evaluacion_interfaz_variables}

Antes de definir las variables, se deben primero definir los conceptos 
relacionados a los tipos de acciones que pueden realizarse sobre el paciente 
virtual en la solución, los mismos son:

%\observacion{Cual se encarga del diseño de la simulación?}
\begin{itemize}
\item \textbf{Acción por menú contextual:} se refiere a las acciones que el usuario 
    puede realizar utilizando el menú contextual que aparece sobre cada uno de los elementos 
    disponibles en la solución.
\item \textbf{Acción por menú de la \Gls{gui}:} se refiere a las 
    acciones que el usuario puede realizar seleccionando una opción en los menús 
    principales que presenta la interfaz de la solución.
\item \textbf{Acción con elemento:} se refiere a las actividades que el usuario 
    puede realizar cuando tiene seleccionado un elemento y que no involucre el 
    uso del menú contextual.
\end{itemize}


Las variables medidas durante la realización de la tarea con la solución son las
siguientes:

%\observacion{No repetir tanto la descripción en el título}

\begin{itemize}

\item \textbf{Tiempo de realización de la primera acción por tipo:} cuanto tiempo 
	le toma al usuario realizar la primera vez una acción agrupado por tipo (por menú 
	contextual, por menú de la \Gls{gui}, con elementos).

%\item \textbf{Tiempo de realización de la primera acción por menú contextual:} 
%    cuanto tiempo le toma al usuario realizar una acción por menú contextual la 
%    primera vez.
%
%\item \textbf{Tiempo de realización de la primera acción por \Gls{gui}:} cuanto 
%    tiempo le toma al usuario realizar una acción por menú de 
%    interfaz gráfica de usuario la primera vez.
%    
%\item \textbf{Tiempo de realización de la primera acción por herramienta:} cuanto 
%    tiempo le toma al usuario realizar una acción por herramienta la primera vez.

\item \textbf{Tiempo de realización de las siguientes acciones por tipo:} cuanto tiempo 
	le toma al usuario realizar las siguientes veces una acción agrupado por 
	tipo (por menú contextual, por menú de la \Gls{gui}, con elementos).
 
%\item \textbf{Tiempo de realización de las siguientes acciones por menú contextual:} 
%    cuanto tiempo le toma al usuario realizar una acción por menú 
%    contextual las siguientes veces.
%
%\item \textbf{Tiempo de realización de las siguientes acciones por \Gls{gui}:} 
%    cuanto tiempo le toma al usuario realizar una acción 
%    por interfaz gráfica de usuario las siguientes veces.
%
%\item \textbf{Tiempo de realización de las siguientes acciones por herramienta:} 
%    cuanto tiempo le toma al usuario realizar una acción por herramienta 
%    las siguientes veces.

\item \textbf{Tiempo total:} se refiere al tiempo empleado por el usuario para 
    completar la tarea asignada.

\item \textbf{Número de pasos realizados:} cantidad de pasos requeridos en la tarea 
    que son realizados por el usuario en la simulación. 

\item \textbf{Cantidad de movimientos espaciales por tipo:} número de veces en que se 
    modifica el estado de la cámara para realizar las acciones deseadas agrupados por 
    tipo (desplazamiento, acercamiento/alejamiento).

%    \observacion{Esto donde entra?}

\end{itemize}

En cuanto a la encuesta, las siguientes son las variables que fueron consideradas 
y medidas:

\begin{itemize}

\item \textbf{Calidad gráfica:} realismo y calidad de los modelos utilizados.

\item \textbf{Interacción:} desenvolvimiento en el entorno y utilización del 
    hardware.

\item \textbf{Interacción con objetos:} utilización errónea de objetos.

\item \textbf{Características del entorno:} realismo del escenario y de los 
    objetos utilizados.

\item \textbf{Usabilidad de la interfaz:} facilidad de uso de las opciones 
    proveídas por la interfaz.

\item \textbf{Integración con el hardware:} facilidad de uso de la solución con 
    un dispositivo móvil. 

\end{itemize}

\subsection{Métricas}

Para la medición de las variables relacionadas a la encuesta,  se utiliza la escala
de Likert con la \emph{Doble estandarización} explicada en la
sección~\ref{sec:likert}. 

En cambio, para la medición de las variables relacionadas a la interacción del usuario con 
la solución se utilizan las grabaciones registradas durante las pruebas y las
las siguientes métricas:

%Para el análisis de la encuesta realizada a los usuarios, se utiliza la escala
%de Likert con la \emph{Doble estandarización} explicada en la
%sección~\ref{sec:likert}, y en el análisis de la interacción del usuario con la
%solución se utilizan las grabaciones registradas durante la prueba.
%
%Haciendo uso de las variables descriptas anteriormente, las métricas
%utilizadas son las siguientes:

\begin{itemize}
    
\item \textbf{Tiempo promedio de realización de las siguientes acciones por menú contextual:} 
    se obtiene dividiendo la cantidad total de tiempo empleado en realizar acciones por menú 
    contextual por el número de veces que se realizaron esas acciones, sin considerar la primera 
    vez. 
    
\item \textbf{Tiempo promedio de realización de las siguientes acciones por \Gls{gui}:} 
    se obtiene dividiendo la cantidad total de tiempo empleado en realizar acciones por \Gls{gui} 
    por el número de veces que se realizaron esas acciones, sin considerar la primera 
    vez. 
    
\item \textbf{Tiempo promedio de realización de las siguientes acciones por herramienta:} 
    se obtiene dividiendo la cantidad total de tiempo empleado en realizar acciones por 
    herramienta por el número de veces que se realizaron esas acciones, sin considerar la 
    primera vez. 
    
\item \textbf{Promedio de pasos correctos:} se obtiene dividiendo la cantidad de 
    pasos requeridos realizados por los usuarios sobre la cantidad de pasos requeridos. 
    
\item \textbf{Promedio de movimientos por tipo:} se obtiene dividiendo el número de 
    movimientos que fueron realizados agrupados por tipo (desplazamiento, acercamiento/
    desplazamiento) por la cantidad de usuarios.
    
\item \textbf{Promedio del tiempo total:} se obtiene dividiendo el tiempo total empleado 
    por los usuarios para completar la tarea asignada por el número de usuarios.

\end{itemize}

\subsection{Resultados obtenidos}
\label{sec:res_interfaz}

A continuación de muestran y analizan los resultados obtenidos en la prueba. Los resultados 
se dividen en \emph{simulación} y \emph{encuesta} para una mejor comprensión.

\subsubsection{Simulación}

Las grabaciones realizadas a las sesiones de los usuarios se utilizan para medir
el grado de facilidad de aprendizaje de la interfaz de usuario.

Dados los tres tipos de acciones descritos en~\ref{sec:evaluacion_interfaz_variables}, la
tabla~\ref{tab:interfaz_tiempo_acciones} muestra el tiempo, en segundos,
que le tomo a cada usuario realizar una acción la primera vez y 
el tiempo que les tomo en promedio las demás veces, para cada uno de los tipos 
de acciones.

%\observacion{Hacer énfasis en la comparación entre el primer y los siguientes}

\begin{table}[!hbt]
\centering
\begin{tabular}{|c|c|c|c|c|c|c|}
\hline
& \multicolumn{2}{c|}{\textbf{Menú Contextual}} &
\multicolumn{2}{c|}{\textbf{Menú de la Interfaz}} & \multicolumn{2}{c|}{\textbf{Herramienta}}\\
\hline
\textbf{Usuario}  & \textbf{Primera} & \textbf{Siguientes} & \textbf{Primera} & \textbf{Siguientes} & \textbf{Primera} & \textbf{Siguientes} \\
\hline 1          & 8                & 2.25                & 3                & 9.14                & 11               & 3.0 \\
\hline 2          & 30               & 7.00                & 4                & 3.57                & 7                & 4.5 \\
\hline 3          & 5                & 2.25                & 5                & 1.86                & 1                & 1.0 \\
\hline 4          & 2                & 13.00               & 4                & 2.00                & 1                & 0.5 \\
\hline 5          & 18               & 2.75                & 6                & 4.43                & 6                & 3.0 \\
\hline 6          & 4                & 14.25               & 11               & 7.86                & 13               & 4.0 \\
\hline 7          & 5                & 8.00                & 4                & 4.71                & 20               & 2.5 \\
\hline 8          & 3                & 2.33                & 10               & 3.57                & 3                & 6.5 \\
\hline
\textbf{Promedio} & \textbf{9.38}    & \textbf{6.37}       & \textbf{5.88}    & \textbf{4.64}       & \textbf{7.75}    & \textbf{3.125} \\
\hline
\end{tabular}
\caption{Tiempo por acciones la primera vez y las siguientes veces que se realizo}
\label{tab:interfaz_tiempo_acciones}
\end{table}

En la tabla~\ref{tab:interfaz_tiempo_acciones} se observa consistentemente una 
mejora en el tiempo de realización de un tipo de acción con respecto a la primera vez 
que es realizada. 

\begin{filecontents}{interfazuso.dat}
n   p       s
1	9.38	6.48
2   5.88	4.64
3   7.75	3.13
\end{filecontents}
\pgfplotstableread{interfazuso.dat}{\InterfazUso}

\begin{figure}[H]
    
        \centering
        \begin{tikzpicture}[scale=.8]
           \begin{axis}[ybar,%
              legend pos=outer north east,
              xmin=1,
              xmax=3,
              x=2.5cm,
              enlarge x limits={abs=1cm},
              xtick=data,
              symbolic x coords={0,1,2,3,4},
              ymin=0,ymax=10,
              %ytick={0,2,4,6,8,10},
              xticklabels={Contextual,Interfaz,Herramienta},
              ylabel= Tiempo (s),
              xlabel= Tipo de acción,
              bar width=10pt,
              %enlarge x limits={abs=2},
                ]   
        \addplot[color=blue!90,ybar,fill=blue!55,area legend] table [x = {n}, y = {p}] {\InterfazUso};
        \addlegendentry{Primer}
        \addplot[color=red!90,ybar,fill=red!55,area legend] table [x = {n}, y = {s}]
        {\InterfazUso};
        \addlegendentry[align=left]{Promedio \\ siguientes}
        \end{axis}
        \end{tikzpicture}
        \caption{Tiempo por tipo de acción}
        \label{fig:interfaz_tiempo_acciones}
\end{figure}

En la figura~\ref{fig:interfaz_tiempo_acciones} se observa como en promedio el
usuario aprende, y en las siguientes acciones similares demora menos tiempo,
este es un factor importante y es el objetivo de esta prueba pues muestra que la
interfaz es fácil de usar, y con tres tipos de acciones, el usuario puede
utilizarla sin mayores inconvenientes. Se observa una mejoría del $30\%$ en las
\emph{Acciones por menú contextual}, $21\%$ en las \emph{Acciones por menú de la
    \Gls{gui}} y finalmente, una mejoría del $60\%$ en las \emph{Acciones con elementos}.


\begin{table}[hbt]
\centering
\small
\begin{tabular}{lrrr}
\toprule
\textbf{Jugador}  & \textbf{Desplazamiento} & \textbf{Acercamiento/alejamiento} & \textbf{Total} \\
\midrule
1        & 18         & 2    & 20 \\
2        & 7          & 8    & 15 \\
3        & 14         & 12   & 26 \\
4        & 9          & 14   & 23 \\
5        & 5          & 8    & 13 \\
6        & 14         & 4    & 18 \\
7        & 16         & 3    & 19 \\
8        & 4          & 3    &  7 \\
\midrule
\textbf{Promedio} & \textbf{10,88}      & \textbf{6,75} & \textbf{17,63} \\
\bottomrule
\end{tabular}
\caption{Cantidad de movimientos espaciales}
\label{tab:interfaz_cantidad_espaciales}
\end{table}

En la tabla~\ref{tab:interfaz_cantidad_espaciales} se observa la cantidad de
movimientos espaciales realizados por los usuarios, se observa que en promedio
se desplazaron $10,88$ veces por el escenario, y $6,75$ veces acercaron o
alejaron la cámara del paciente.

No existe una cantidad mínima o máxima de movimientos que el usuario debe realizar para acercar, 
alejar o desplazar la cámara. Los datos mostrados en la tabla~\ref{tab:interfaz_cantidad_espaciales} 
muestran que no son necesarias demasiados movimientos. Teniendo en cuenta esta información y la 
proveída en la tabla~\ref{tab:interfaz_tiempo_total}, se concluye 
que en promedio los usuarios realizan $1,7$ movimientos por minuto.

\begin{table}[!hbt]
\centering
\small
\begin{tabular}{lrrr}
\toprule
\textbf{Alumno} & \textbf{Tiempo (min)} \\
\midrule
1        & 8:32 \\
2        & 6:03 \\
3        & 8:33 \\
4        & 5:17 \\
5        & 6:55 \\
6        & 8:40 \\
7        & 7:03 \\
8        & 10:27 \\
\midrule
\textbf{Promedio} & \textbf{7:41} \\
\bottomrule
\end{tabular}
\caption{Tiempo de prueba por usuario}
\label{tab:interfaz_tiempo_total}
\end{table}

El tiempo total que se observa en la tabla~\ref{tab:interfaz_tiempo_total},
muestra que en promedio a cada alumno le tomo $7:41$ minutos realizar todos los
pasos especificados, es importante notar que este tiempo incluye el tiempo de
adaptación. 

La tabla~\ref{tab:interfaz_acciones} nos muestra la cantidad de pasos
realizados por los alumnos de un total de 19. Se observa que en promedio 
realizaron $16.75$ pasos.

\begin{table}[htb]
\centering
\small
\begin{tabular}{lrrr}
\toprule
\textbf{Alumno} & \textbf{Pasos realizados (19)} \\
\midrule
1 & 19 \\
2 & 15 \\
3 & 18 \\
4 & 15 \\
5 & 18 \\
6 & 16 \\
7 & 19 \\
8 & 14 \\
\midrule
\textbf{Promedio} & \textbf{16,75} \\
\bottomrule
\end{tabular}
\caption{Pasos realizados por alumno}
\label{tab:interfaz_acciones}
\end{table}



\subsubsection{Encuesta}


La encuesta es utilizada para obtener el grado de disconformidad de los usuarios
con respecto a la solución. Se utiliza la disconformidad para resaltar los
puntos débiles, así, aquellas variables que tengan el mayor porcentaje serán las
que deban ser mejoradas.


En la tabla~\ref{tab:interfaz_disconformidad_metrica} se observan que las
mayores disconformidades son la usabilidad de la interfaz de usuario que llega
al $51\%$, la interacción de los usuarios con el entorno que llega al $50\%$ y
la interacción con los objetos que llega al $49\%$. Otras disconformidades con
menor porcentaje son las características del entorno con un  $33\%$, la
integración con el hardware con un $27\%$ y por último la calidad gráfica con un
$17\%$.


\begin{table}[htb]
\centering
\begin{tabular}{lr}
\toprule
\textbf{Variable} & \textbf{Disconformidad (0-1)}\\
\midrule
Calidad Gráfica         & 0.17 \\
Interacción Entorno     & 0.50\\
Interacción Objetos     & 0.49\\
Características Entorno & 0.33\\
Usabililidad Interfaz   & 0.51\\
Integración Hardware    & 0.27\\
\bottomrule
\end{tabular}
\caption{Disconformidad por variable}
\label{tab:interfaz_disconformidad_metrica}
\end{table}

%La conclusión de esta prueba de interfaz, es que si bien, pudo ser utilizada sin
%mayores inconvenientes, existe un alto grado de disconformidad con la interfaz,
%además cabe resaltar, los sujetos de prueba son personas acostumbradas al uso de
%tecnologías similares. Otros puntos débiles encontrados en esta prueba son la
%interacción con el entorno y  con los objetos.

Como consecuencia de los resultados obtenidos, la usabilidad de interfaz y la interacción con objetos y 
con el entorno son mejoradas para obtener la versión final de la solución que es utilizada por 
los estudiantes de enfermería. Las demás pruebas mencionadas en este capítulo son realizadas con 
la versión final de la solución.
%elementos sufren modificaciones a fin de su utilización con usuarios no
%técnicos.

%Las demás pruebas mencionadas en este capítulo son realizadas con la versión
%final de la solución, la cual es obtenida luego de las mejoras realizadas a los
%puntos débiles detectados por esta prueba.
