%! TEX root = ../main.tex
\section{Objetivo General}

% Enfoque 3.89
El objetivo general de la presente tesis es investigar, estudiar y evaluar 
las diferentes tecnologías disponibles para apoyar el proceso de aprendizaje 
utilizando como base pedagógica el construccionismo para  diseñar un esquema 
de desarrollo que se pueda implementar en el área de enfermería. 


% Enqoue NEGATIVO
% \fixme{diseñar e implementar un
    %aplicación}{No puede ser un objetivo general!} 
%estudiar y evaluar tecnologías que permitan  apoyar el
%proceso de aprendizaje de los alumnos de la carrera de enfermería, utilizando
%como base pedagógica el construccionismo.


% Enfoque 1
%% Diseñar un esquema de desarrollo de tecnologías que permitan complementar el
%% proceso de aprendizaje de los alumnos de la carrera de enfermería, utilizando
%% como base pedagógica el construccionismo.

% Enfoque 2
%% Investigar, estudiar y evaluar las aplicaciones de juegos serios 
%% construccionistas como herramientas complementarias a la educación tradicional.

Con el fin de aproximarnos a nuestro propósito, se formulan los siguientes
objetivos específicos:

\begin{enumerate}
    \item Investigar los fundamentos y estado actual de la corriente
        \emph{Construccionismo}, como herramienta pedagógica y su relación con
        las TIC's.

    \item Investigar acerca de los juegos serios y corrientes afines como
        herramientas para la aplicación del construccionismo.
    
    \item Investigar las áreas de aplicabilidad de los juegos serios, poniendo
        énfasis en las áreas que permiten un enfoque construccionista.
        
    \item Identificar las características del área de enfermería que hacen que
        la misma sea un contexto factible para la aplicación de los juegos
        serios basados en el construccionismo.
    
    \item Analizar, evaluar y seleccionar las herramientas que nos permitan la
        implementación de un juego serio que simule un laboratorio de
        enfermería.
        
    \item Diseñar e implementar un juego serio construccionista que permita
        exponer las ventajas y desventajas como modelo de apoyo a la enseñanza
        tradicional. 

    
    \item Evaluar la solución propuesta para la obtención de datos que permitan
        medir las fortalezas y debilidades en cuanto al grado de aceptación,
        implicación y ventajas desde el punto de vista de los usuarios.

    \item Identificar las fortalezas y debilidades de los métodos utilizados
        para definir su aplicabilidad como herramienta de apoyo. 

    \item Definir y evaluar desde el ámbito del diseño los puntos que deben
        tenerse en cuenta a la hora de diseñar este tipo de herramientas de
        apoyo.

\end{enumerate}


