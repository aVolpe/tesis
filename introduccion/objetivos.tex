%! TEX root = ../main.tex
\section{Objetivo General}
\label{sec:objetivos_generales}

Realizar una investigación sobre el rol actual de las tecnologías de la
información y la comunicación en la educación, poniendo énfasis en las 
corrientes pedagógicas actuales, y evaluar las distintas tecnologías 
disponibles para introducirlo en un área especifica.

% Enfoque 3.89
%El objetivo general de la presente tesis es investigar, estudiar y evaluar 
%las diferentes tecnologías disponibles para apoyar el proceso de aprendizaje 
%utilizando como base pedagógica el construccionismo para  diseñar un esquema 
%de desarrollo que se pueda implementar en el área de enfermería. 
%\observacion{Creo que es la segunda vez que pregunto. Pero lo de enfermería es un
%    caso de prueba, no es el objetivo general (ver correcciones generales)}


% Enqoue NEGATIVO
% \fixme{diseñar e implementar un
    %aplicación}{No puede ser un objetivo general!} 
%estudiar y evaluar tecnologías que permitan  apoyar el
%proceso de aprendizaje de los alumnos de la carrera de enfermería, utilizando
%como base pedagógica el construccionismo.


% Enfoque 1
%% Diseñar un esquema de desarrollo de tecnologías que permitan complementar el
%% proceso de aprendizaje de los alumnos de la carrera de enfermería, utilizando
%% como base pedagógica el construccionismo.

% Enfoque 2
%% Investigar, estudiar y evaluar las aplicaciones de juegos serios 
%% construccionistas como herramientas complementarias a la educación tradicional.
\section{Objetivos Específicos}

Con el fin de aproximarnos a nuestro propósito, se formulan los siguientes
objetivos específicos:

\begin{enumerate}
    \item Proveer un resumen de los fundamentos y estado actual de la corriente
        \emph{Construccionismo}, como herramienta pedagógica y su relación con
        las \gls{tic}.
        % Proveer un resumen de los fundamentos y estado del arte de las
        % corrientes actuales relacionadas con las \gls{tic}.

    \item Proveer un resumen actualizado acerca de los juegos serios y
        corrientes afines como herramientas para la aplicación del
        construccionismo.
    
    \item Proveer un resumen actualizado de las áreas de aplicabilidad de los
        juegos serios, poniendo énfasis en las áreas que permiten un enfoque
        construccionista.
        
    \item Identificar las características del área de enfermería que hacen que
        la misma sea un contexto factible para la aplicación de los juegos
        serios basados en el construccionismo.
    
    \item Analizar, evaluar y seleccionar las herramientas que nos permitan la
        implementación de un juego serio que simule un laboratorio de
        enfermería.
        
    \item Diseñar e implementar un juego serio construccionista que permita
        exponer las ventajas y desventajas como modelo de apoyo a la enseñanza
        tradicional. 
        %\observacion{Recuerden tener/exponer las ventajas y desventajas en la
        %    conclusión}

    \item Evaluar la solución propuesta para la obtención de datos que permitan
        medir las fortalezas y debilidades desde el punto de vista del usuario,
        en cuanto a factores de:
        \begin{enumerate*}[label=\itshape\alph*\upshape)]
            \item exploración,
            \item representación,
            \item motivación,
            \item inmersión,
            \item utilidad,
            \item retroalimentación, y,
            \item pedagogía.
        \end{enumerate*}

    \item Identificar las fortalezas y debilidades de los métodos utilizados
        para definir su aplicabilidad como herramienta de apoyo. 

    \item Identificar desde el ámbito del diseño, desarrollo y evaluación los puntos que deben
        tenerse en cuenta a la hora de implementar este tipo de herramientas de
        apoyo.
\end{enumerate}
