%! TEX root = ../main.tex
\section{Objetivo General}

El objetivo general de la presente tesis es \fixme{diseñar e implementar un
    aplicación}{No puede ser un objetivo general!} que permita complementar el
proceso de aprendizaje de los alumnos de la carrera de enfermería, utilizando
como base pedagógica el construccionismo.

Con el fin de aproximarnos a nuestro propósito, se formulan los siguientes
objetivos específicos:

\begin{enumerate}
    \item \fixme{Realizar}{conjugación} una investigación acerca de los
        fundamentos y el estado del arte tanto del construccionismo como del uso
        de la simulación y los juegos serios como herramientas de aprendizaje.

    \item \fixme{Buscar y analizar un contexto concreto de aplicación}{Agregar
            esto}

    \item Diseñar e implementar un juego serio \fixme{que ayude a los estudiantes de la
            carrera de enfermería en su proceso de formación}{Que permita
            exponer los conceptos del contexto de aplicación de tal manera}.

    \item Analizar, evaluar y seleccionar las herramientas que nos permitan la
        implementación de un juego serio que permita simular un laboratorio de
        enfermería.

    \item Evaluar la solución propuesta para la obtención de datos que permitan
        medir el \fixme{grado de impacto y aceptación}{Más detalle}.

    \item Identificar las fortalezas y debilidades de los métodos utilizados
        para definir su aplicabilidad como herramienta complementaria. 

\end{enumerate}

\observacion{Lo importante de esta sección es que sea coherente. Lo que busca es
    probar la aplicabilidad de este tipo de herramientas a un contexto nuevo}

