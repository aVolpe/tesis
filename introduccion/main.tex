%! TEX root = ../main.tex
\chapter{Introducción}
\label{chap:introduccion}

%"Lo fundamental de todo proceso pedagógico es el aprendizaje y no 
%la enseñanza. Es el aprendizaje 
%del estudiante y su participación el logro deseado." (Unesco, 1995)


%\observacion{Mencionar la problemática. Falta un párrafo con XXX (compoter)
%    donde se menciona que estas corrientes no han sido explotadas y que tampoco
%    se han vuelto mainstream, (Ver que opina XXX (garcher))}

La educación tradicional, en la actualidad, se basa en el concepto de que el
profesor transfiere el conocimiento que ha adquirido de diferentes métodos
(educación, experiencia, etc.) a un alumno que es un receptor pasivo de
información, esta corriente es llamada
instruccionismo\cite{laptop:instructionism}. 

En el instruccionismo el uso de las \Gls{tic} cumple con un rol en el cual
sólo es un mecanismo más para transmitir el conocimiento del maestro al alumno,
reemplazando libros y presentaciones. 

Con la aparición de nuevas corrientes pedagógicas, el uso de las tecnologías
tiene un papel más activo dentro del proceso de aprendizaje, entre estas
corrientes podemos citar al conductismo, al constructivismo y al
construccionismo. 


Algunos ejemplos que incluyen a las \gls{tic} en la educación son:
\emph{e-Learning}, simulaciones educativas, \emph{edutainment}, el lenguaje de
programación LOGO, y los juegos serios. Los juegos serios son aquellos
videojuegos desarrollados con un propósito distinto al de puro entretenimiento.

El aprendizaje apoyado por las \Gls{tic}, utilizando a los juegos serios, tienen
la capacidad de eliminar los problemas de distancia, en el ámbito empresarial
son utilizados para enseñar a grupos de personas a trabajar en equipo, incluso
cuando estos se encuentran a distancias que le impiden reunirse en un aula
tradicional\cite{guenaga2013serious}. 

Los juegos serios permiten utilizar la exploración y el ensayo para desarrollar
habilidades y pericias en entornos
controlados\cite{humphreys2013developing,sg:aoverview}.
   
Tomando como base lo explicado anteriormente, se considera a los juegos serios
un campo interesante investigación que involucra el desarrollo de aplicaciones
tecnológicas que ayuden a estudiantes en el proceso de aprendizaje. 

Estas herramientas tienen especial importancia en ambientes donde las
limitaciones, de espacio, y tiempo dificultan la aplicación de técnicas
tradicionales\cite{education:games} como es el caso del entrenamiento de
profesionales de enfermería los cuales requieren de varias horas de práctica
durante su preparación. Las prácticas se realizan en instituciones como
hospitales escuela, donde los alumnos son supervisados por profesionales
mientras realizan las prácticas. Uno de los principales inconvenientes de estos
estudiantes es la poca disponibilidad de tiempo que poseen, en cuanto a las
prácticas que realizan en un laboratorio usualmente por la cantidad de
estudiantes es muy difícil la personalización de la enseñanza y en cuanto a las
prácticas en hospitales uno de los inconvenientes es el nerviosismo ante las
primeras prácticas.

%Según~\cite{humphreys2013developing} los alumnos de enfermería son estudiantes
%divergentes, es decir aprenden a través de experimentación activa, e
%interiorizan el conocimiento reflexionando sobre la experiencia. Los juegos
%serios son una herramienta ideal para este tipo de
%estudiantes\cite{humphreys2013developing}. 

Por todo lo expuesto anteriormente en este trabajo se propone  el desarrollo de
un juego serio, que involucra la simulación se laboratorios virtuales como una
herramienta de apoyo para el proceso de aprendizaje de los alumnos de la carrera
de enfermería.


%! TEX root = ../main.tex
\section{Objetivo General}
\label{sec:objetivos_generales}

Realizar una investigación sobre el rol actual de las tecnologías de la
información y la comunicación en la educación, poniendo énfasis en las 
corrientes pedagógicas actuales, y evaluar las distintas tecnologías 
disponibles para introducirlo en un área especifica.

% Enfoque 3.89
%El objetivo general de la presente tesis es investigar, estudiar y evaluar 
%las diferentes tecnologías disponibles para apoyar el proceso de aprendizaje 
%utilizando como base pedagógica el construccionismo para  diseñar un esquema 
%de desarrollo que se pueda implementar en el área de enfermería. 
%\observacion{Creo que es la segunda vez que pregunto. Pero lo de enfermería es un
%    caso de prueba, no es el objetivo general (ver correcciones generales)}


% Enqoue NEGATIVO
% \fixme{diseñar e implementar un
    %aplicación}{No puede ser un objetivo general!} 
%estudiar y evaluar tecnologías que permitan  apoyar el
%proceso de aprendizaje de los alumnos de la carrera de enfermería, utilizando
%como base pedagógica el construccionismo.


% Enfoque 1
%% Diseñar un esquema de desarrollo de tecnologías que permitan complementar el
%% proceso de aprendizaje de los alumnos de la carrera de enfermería, utilizando
%% como base pedagógica el construccionismo.

% Enfoque 2
%% Investigar, estudiar y evaluar las aplicaciones de juegos serios 
%% construccionistas como herramientas complementarias a la educación tradicional.
\section{Objetivos Específicos}

Con el fin de aproximarnos a nuestro propósito, se formulan los siguientes
objetivos específicos:

\begin{enumerate}
    \item Proveer un resumen de los fundamentos y estado actual de la corriente
        \emph{Construccionismo}, como herramienta pedagógica y su relación con
        las \gls{tic}.
        % Proveer un resumen de los fundamentos y estado del arte de las
        % corrientes actuales relacionadas con las \gls{tic}.

    \item Proveer un resumen actualizado acerca de los juegos serios y
        corrientes afines como herramientas para la aplicación del
        construccionismo.
    
    \item Proveer un resumen actualizado de las áreas de aplicabilidad de los
        juegos serios, poniendo énfasis en las áreas que permiten un enfoque
        construccionista.
        
    \item Identificar las características del área de enfermería que hacen que
        la misma sea un contexto factible para la aplicación de los juegos
        serios basados en el construccionismo.
    
    \item Analizar, evaluar y seleccionar las herramientas que nos permitan la
        implementación de un juego serio que simule un laboratorio de
        enfermería.
        
    \item Diseñar e implementar un juego serio construccionista que permita
        exponer las ventajas y desventajas como modelo de apoyo a la enseñanza
        tradicional. 
        %\observacion{Recuerden tener/exponer las ventajas y desventajas en la
        %    conclusión}

    \item Evaluar la solución propuesta para la obtención de datos que permitan
        medir las fortalezas y debilidades desde el punto de vista del usuario,
        en cuanto a factores de:
        \begin{enumerate*}[label=\itshape\alph*\upshape)]
            \item exploración,
            \item representación,
            \item motivación,
            \item inmersión,
            \item utilidad,
            \item retroalimentación, y,
            \item pedagogía.
        \end{enumerate*}

    \item Identificar las fortalezas y debilidades de los métodos utilizados
        para definir su aplicabilidad como herramienta de apoyo. 

    \item Identificar desde el ámbito del diseño, desarrollo y evaluación los puntos que deben
        tenerse en cuenta a la hora de implementar este tipo de herramientas de
        apoyo.
\end{enumerate}

\section{Estructura del libro}
    

%\fixme{A continuación se explica el proceso que fue realizado para alcanzar los objetivos citados 
%    anteriormente, las etapas de este proceso se presentan en diferentes capítulos los cuales 
%    tratan de diversos aspectos.}{Borrar}

% TIC's en la educación
%En el capitulo~\ref{chap:tics} se expande el estado actual de las \Gls{tic} en
%la educación, desde sus inicios, las expectativas creadas, los progresos
%realizados, y las experiencias, la evolución de la teoría hasta hoy en día.

En el capítulo~\ref{chap:tics} se presenta un resumen sobre el uso de las
\Gls{tic} en la educación, desde sus inicios apoyando a la educación
tradicional hasta su rol actual en las nuevas corrientes pedagógicas. Se
hace énfasis en la corriente pedagógica llamada construccionismo, explicando
su historia, base pedagógica, el rol y la forma de uso de las \Gls{tic} en
la misma.

% Juegos serios
%En el capitulo~\ref{chap:juegos_serios} se describen las características y áreas de aplicación de un 
%juego serio, incluyendo corrientes relacionadas y algunos casos de éxito. Se muestra 
%además un esbozo de como desarrollar un juego serio.

En el capítulo~\ref{chap:juegos_serios} se describen a los juegos serios, siendo 
estos una de las herramientas tecnológicas usadas dentro del proceso de aprendizaje, 
se detallan sus características, áreas de aplicación, las corrientes tecnológicas relacionadas, 
y el proceso de desarrollo además de describir algunos casos de éxito. También se describen 
tecnologías relacionadas a los juegos serios.


% Definición del problema

En el capítulo~\ref{chap:problema} se definen las características del problema
que se quiere abordar para darle una solución tecnológica. Enfocándonos en el
área de enfermería se describe la situación actual y una propuesta de solución tecnológica 
que puede brindar apoyo en el proceso de aprendizaje.



%nseñanza utilizado en el \Gls{iab}, enfocado específicamente a la carrera de
%Licenciatura en Enfermería, la distribución de los cursos, carga horaria,
%prácticas de laboratorio y prácticas en hospitales, así como los mecanismos de
%evaluación utilizados.

%Además se citan los problemas existentes, como las dificultades que tienen los
%alumnos en los distintos aspectos que influyen en la vida de un estudiante, y
%problemas inherentes a la enseñanza de profesiones técnicas con métodos de
%enseñanza tradicionales. Y por último se describen los procedimientos de enfermería 
%en los que se enfocará este trabajo.


% Hipotesis y requerimientos de la solucion

En el capítulo \ref{chap:requerimientos} se definen los criterios que deben tenerse en 
cuenta para seleccionar el contenido a abordar en la solución, se seleccionan y describen 
los procedimientos de enfermería que formarán parte del contenido, el alcance  
y los requisitos que debe cumplir la solución.

%En el capítulo~\ref{chap:requerimientos} se describen los requisitos que deben 
%tenerse en cuenta para el desarrollo de la solución propuesta, incluyendo las hipótesis asumidas
%con respecto a diferentes aspectos de los procedimientos de enfermería.

%Tecnologías utilizadas

En el capítulo~\ref{chap:tecnologias} se describen las diversas tecnologías utilizadas 
en el desarrollo del proyecto, haciendo especial énfasis en los motores de juego ya que 
los mismos proveen el entorno principal para el desarrollo de la solución, se realiza 
una comparación entre los motores de juegos actuales y se justifica por que se seleccionó 
uno en específico. 


% Propuesta de solución

%El capitulo~\ref{chap:solucion} describe una solución  para la aplicabilidad de juegos serios 
%en un contexto con las características mencionadas en el capitulo anterior utilizando los conceptos
%estudiados en el capítulo~\ref{chap:tics}, se proponen, además, consideraciones
%a tener en cuenta referentes a los componentes e interacción entre los mismos.

%Así mismo se definen mecanismos para tratar con los limites de una simulación, y
%como utilizar las herramientas descritas en el capítulo~\ref{chap:tics} en
%armonía con los objetivos pedagógicos.

En el capítulo~\ref{chap:solucion} se describe en detalle la solución propuesta basada en 
la aplicabilidad de los juegos serios en un contexto con las características mencionadas en 
el capítulo~\ref{chap:problema} y teniendo en cuenta las requisitos detallados en el 
capítulo~\ref{chap:requerimientos} haciendo uso de las tecnologías descritas en el capítulo 
anterior. 

% Definición de evaluación

Dada la solución propuesta, en el capítulo~\ref{chap:evaluacion} se propone una
forma de evaluarla, teniendo en cuenta las hipótesis planteadas durante su
desarrollo  y los objetivos del presente trabajo. Esta evaluación consiste en
una serie de pruebas, se define el universo, la muestra, y los criterios de
selección de la muestra.

% Análisis de resultados

Se presentan además los resultados de las diferentes
pruebas llevadas a cabo, obtenidos en forma tabular y con gráficos para
facilitar la comprensión. %Además se estudian factores de correlación entre las
%distintas variables medidas.

% Conclusiones

En el capítulo~\ref{chap:conclusion} se presentan las
conclusiones obtenidas a partir de los resultados obtenidos y la experiencia
adquirida durante el desarrollo de la solución.

% Trabajo furuto

Finalmente, en el capítulo~\ref{chap:futuro} se describen algunos trabajos posibles que pueden
ser desarrollados en el área teniendo en cuenta el presente trabajo. 

%\observacion{Volver a revisar más adelante}

