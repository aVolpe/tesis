%! TEX root = ../main.tex
\chapter{Introducción}


En la actualidad, la educación tradicional o instruccionismo se basa en el concepto de que el
profesor transfiere el conocimiento que ha adquirido de diferentes métodos
(educación, experiencia, etc) a un alumno que es un receptor pasivo de
información\cite{laptop:instructionism}. %Se enfoca más en el profesor, en la
%capacidad del mismo, y en el producto final como resultado de un proceso no
%interactivo y bien documentado\cite{igi:instructionism}, este proceso es
%conocido como Instruccionismo.

En la educación tradicional el uso de las \Gls{tic} cumplía con un rol en el 
cual solo eran un mecanismo mas para transmitir el conocimiento del maestro al alumno,
reemplazando libros y presentaciones. Actualmente, el uso de las tecnologías tiene un
papel mas activo dentro del proceso de aprendizaje.

Además del instruccionismo o educación tradicional existen otras corrientes pedagógicas, 
entre las que se pueden citar el conductismo, el constructivismo y el construccionismo. Este
trabajo se centra en el construccionismo.

El construccionismo tiene un enfoque diferente en cuanto a las \Gls{tic} ya que
utiliza la tecnología como un mecanismo de adquisición de conocimiento y no como
una herramienta proveedora del mismo\cite{sasha:construtivism}, es decir se
centra en como se generan los conocimientos y no en como proveer información
previamente establecida. 

Algunos de los proyectos que incluyen al construccionismo son: 
el lenguaje de programación LOGO, las simulaciones educativas de entornos virtuales, 
los juegos serios y Lego Serious Play.

El aprendizaje apoyado por las \Gls{tic} y el construccionismo utilizando los
juegos serios tienen la capacidad de eliminar los problemas de distancia, en el
ámbito empresarial se utilizan los juegos serios para enseñar a grupos de
personas a trabajar como un equipo, incluso cuando estos se encuentran a
distancias que le impiden reunirse en una clase tradicional\cite{mariluz:seiousgames}, 
es así que estudios anteriores han demostrado que el construccionismo es una alternativa 
muy prometedora al instruccionismo\cite{sasha:construtivism}.
   
De los proyectos citados anteriormente, la simulación es una opción tecnológica propicia 
para el construccionismo ya que permite utilizar la exploración y el ensayo para desarrollar 
habilidades y pericias en entornos controlados\cite{humphreys2013developing}. La simulación
puede ser complementada con la utilización de juegos serios ya que los mismos proveen una 
oportunidad muy importante para enseñar y desarrollar profesionales\cite{sg:aoverview}.

Basándonos en lo explicado anteriormente, se considera interesante el campo de
investigación que involucra el desarrollo de aplicaciones tecnológicas que
ayuden a estudiantes en el proceso de aprendizaje. 

Estas herramientas tienen especial importancia en ambientes donde las limitaciones,
de espacio, y tiempo dificultan la aplicación de técnicas tradicionales. Uno de
estos casos es el entrenamiento de profesionales de Enfermería los cuales
requieren de varias horas de práctica durante su preparación, y la misma se
realiza en instituciones como hospitales escuela, donde los alumnos son
supervisados por profesionales mientras realizan las practicas.

Además, según \cite{humphreys2013developing} los alumnos de enfermería, son
estudiantes divergentes, es decir aprenden a través de experimentación activa, e
interiorizan el conocimiento reflexionando sobre la experiencia. La
simulación es una herramienta ideal para este tipo de estudiantes\cite{humphreys2013developing}.

Teniendo en cuenta que los mismos forman parte de un grupo de
profesionales que requieren de un alto grado de practicas y que en la actualidad
se presentan varios factores que a un estudiante de esta carrera le impide poner
a prueba todo el tiempo sus conocimientos y por todo lo expuesto anteriormente, 
en este trabajo se propone el desarrollo de un juego serio, que involucra la simulación 
de laboratorios virtuales, como una herramienta para el proceso de aprendizaje de los 
alumnos de la carrera de enfermería.

%! TEX root = ../main.tex
\section{Objetivo General}
\label{sec:objetivos_generales}

% Enfoque 3.89
El objetivo general de la presente tesis es investigar, estudiar y evaluar 
las diferentes tecnologías disponibles para apoyar el proceso de aprendizaje 
utilizando como base pedagógica el construccionismo para  diseñar un esquema 
de desarrollo que se pueda implementar en el área de enfermería. 
\observacion{Creo que es la segunda vez que pregunto. Pero lo de enfermería es un
    caso de prueba, no es el objetivo general (ver correcciones generales)}


% Enqoue NEGATIVO
% \fixme{diseñar e implementar un
    %aplicación}{No puede ser un objetivo general!} 
%estudiar y evaluar tecnologías que permitan  apoyar el
%proceso de aprendizaje de los alumnos de la carrera de enfermería, utilizando
%como base pedagógica el construccionismo.


% Enfoque 1
%% Diseñar un esquema de desarrollo de tecnologías que permitan complementar el
%% proceso de aprendizaje de los alumnos de la carrera de enfermería, utilizando
%% como base pedagógica el construccionismo.

% Enfoque 2
%% Investigar, estudiar y evaluar las aplicaciones de juegos serios 
%% construccionistas como herramientas complementarias a la educación tradicional.
\section{Objetivos Específicos}

Con el fin de aproximarnos a nuestro propósito, se formulan los siguientes
objetivos específicos:

\observacion{Investigar no es un objetivo, es un medio!, un objetivo puede ser
    \enquote{Proveer un resumen actualizado del estado del arte}}

\begin{enumerate}
    \item Investigar los fundamentos y estado actual de la corriente
        \emph{Construccionismo}, como herramienta pedagógica y su relación con
        las TIC's.

    \item Investigar acerca de los juegos serios y corrientes afines como
        herramientas para la aplicación del construccionismo.
    
    \item Investigar las áreas de aplicabilidad de los juegos serios, poniendo
        énfasis en las áreas que permiten un enfoque construccionista.
        
    \item Identificar las características del área de enfermería que hacen que
        la misma sea un contexto factible para la aplicación de los juegos
        serios basados en el construccionismo.
    
    \item Analizar, evaluar y seleccionar las herramientas que nos permitan la
        implementación de un juego serio que simule un laboratorio de
        enfermería.
        
    \item Diseñar e implementar un juego serio construccionista que permita
        exponer las ventajas y desventajas como modelo de apoyo a la enseñanza
        tradicional. 
        \observacion{Recuerden tener/exponer las ventajas y desventajas en la
            conclusión}

    
    \item Evaluar la solución propuesta para la obtención de datos que permitan
        medir las fortalezas y debilidades en cuanto al grado de aceptación,
        implicación y ventajas desde el punto de vista de los usuarios.

    \item Identificar las fortalezas y debilidades de los métodos utilizados
        para definir su aplicabilidad como herramienta de apoyo. 

    \item Definir y evaluar desde el ámbito del diseño los puntos que deben
        tenerse en cuenta a la hora de diseñar este tipo de herramientas de
        apoyo.
        \observacion{En las conclusiones deben tener una lista de
            recomendaciones}

\end{enumerate}



\section{Estructura del libro}
    

A continuación se explica el proceso que fue realizado para alcanzar los objetivos citados 
anteriormente, las etapas de este proceso se presentan en diferentes capítulos los cuales 
tratan de diversos aspectos.

% TIC's en la educación
%En el capitulo~\ref{chap:tics} se expande el estado actual de las \Gls{tic} en
%la educación, desde sus inicios, las expectativas creadas, los progresos
%realizados, y las experiencias, la evolución de la teoría hasta hoy en día.

En el capítulo~\ref{chap:tics} se presenta un resumen sobre el uso de las 
\Gls{tic} en la educación, desde sus inicios apoyando a la educación tradicional 
hasta su rol actual en las nuevas corrientes pedagógicas. Se hace énfasis en la 
corriente pedagógica llamada construccionismo, explicando su historia, base 
pedagógica, el rol y la forma de uso de las \Gls{tic} en la misma.


% Juegos serios
%En el capitulo~\ref{chap:juegos_serios} se describen las características y áreas de aplicación de un 
%juego serio, incluyendo corrientes relacionadas y algunos casos de éxito. Se muestra 
%además un esbozo de como desarrollar un juego serio.

En el capítulo~\ref{chap:juegos_serios} se describen a los juegos serios, siendo 
estos una de las posibles herramientas tecnológicas usadas dentro del proceso de aprendizaje, 
se detallan sus características, áreas de aplicación, las corrientes tecnológicas relacionadas, 
y el proceso de desarrollo además de describir algunos casos de éxito.


% Definición del problema

En el capítulo~\ref{chap:problema} se definen las características del problema
que se quiere abordar para darle una solución tecnológica. Enfocándonos en el
área de enfermería se describe la situación actual y la alternativa tecnológica 
que puede brindar apoyo en el proceso de aprendizaje.



%nseñanza utilizado en el \Gls{iab}, enfocado específicamente a la carrera de
%Licenciatura en Enfermería, la distribución de los cursos, carga horaria,
%prácticas de laboratorio y prácticas en hospitales, así como los mecanismos de
%evaluación utilizados.

%Además se citan los problemas existentes, como las dificultades que tienen los
%alumnos en los distintos aspectos que influyen en la vida de un estudiante, y
%problemas inherentes a la enseñanza de profesiones técnicas con métodos de
%enseñanza tradicionales. Y por último se describen los procedimientos de enfermería 
%en los que se enfocará este trabajo.


% Hipotesis y requerimientos de la solucion

En el capítulo~\ref{chap:requerimientos} se describen los requisitos que deben 
tenerse en cuenta para el desarrollo de la solución propuesta, incluyendo las hipótesis asumidas
con respecto a diferentes aspectos a lo mencionado en el capítulo anterior.

%Tecnologías utilizadas

En el capítulo~\ref{chap:tecnologias} se describen las diversas tecnologías utilizadas 
en el desarrollo del proyecto, haciendo especial énfasis en los motores de juego ya que 
los mismos proveen el entorno principal para el desarrollo de la solución, se realiza 
una comparación entre los motores de juegos actuales y se justifica por que se seleccionó 
uno en específico.


% Propuesta de solución

%El capitulo~\ref{chap:solucion} describe una solución  para la aplicabilidad de juegos serios 
%en un contexto con las características mencionadas en el capitulo anterior utilizando los conceptos
%estudiados en el capítulo~\ref{chap:tics}, se proponen, además, consideraciones
%a tener en cuenta referentes a los componentes e interacción entre los mismos.

%Así mismo se definen mecanismos para tratar con los limites de una simulación, y
%como utilizar las herramientas descritas en el capítulo~\ref{chap:tics} en
%armonía con los objetivos pedagógicos.

En el capítulo~\ref{chap:solucion} se describe en detalle la solución propuesta basada en 
la aplicabilidad de los juegos serios en un contexto con las características mencionadas en 
el capítulo~\ref{chap:problema} y teniendo en cuenta las requisitos detallados en el 
capítulo~\ref{chap:requerimientos} haciendo uso de las tecnologías descritas en el capítulo 
anterior. 

% Definición de evaluación

Dada la solución propuesta, en el capítulo~\ref{chap:evaluacion} se propone una
forma de evaluarla, teniendo en cuenta las hipótesis planteadas durante su desarrollo  
y los objetivos del presente trabajo. Esta evaluación consiste en una serie de experimentos, 
se define el universo, la muestra, y los criterios de selección de la muestra.

% Análisis de resultados

En el capítulo~\ref{chap:analisis} se presentan los resultados de los diferentes
experimentos llevados a cabo, obtenidos en forma tabular y con gráficos para
facilitar la comprensión. Además se estudian factores de correlación entre las
distintas variables medidas.

% Conclusiones

En el capítulo~\ref{chap:conclusion} se presentan las
conclusiones obtenidas a partir de los resultados obtenidos y la experiencia
adquirida durante el desarrollo de la solución.

% Trabajo furuto

Finalmente, en el capítulo~\ref{chap:futuro} se describen algunos trabajos posibles que pueden
ser desarrollados en el área teniendo en cuenta el presente trabajo. 

%\observacion{Volver a revisar más adelante}

