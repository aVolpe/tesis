\section{Estructura del libro}

\observacion{El problema no es el IAB, el problema es la aplicación de este tipo
    de herramientas a un contexto con las características del curso del IAB}

% TIC's en la educación
El el capitulo~\ref{chap:tics} se expande el estado actual de las \Gls{tic} en
la educación, desde sus inicios, las expectativas creadas, los progresos
realizados, y las experiencias, la evolución de la teoría hasta hoy en día.

Se toma especial importancia a los juegos serios y a la simulación, además se
muestra como han evolucionado y afectado a la educación.

% Definición del problema

En el capitulo~\ref{chap:problema}(Problema) se desarrolla el método de
enseñanza utilizado en el \Gls{iab}, enfocado específicamente a la carrera de
Licenciatura en Enfermería, la distribución de los cursos, carga horaria,
prácticas de laboratorio y prácticas en hospitales, así como los mecanismos de
evaluación utilizados.

Además se citan los problemas existentes, como las dificultades que tienen los
alumnos en los distintos aspectos que influyen en la vida de un estudiante, y
problemas inherentes a la enseñanza de profesiones técnicas con métodos de
enseñanza tradicionales.

% Propuesta de solución

El capitulo~\ref{chap:solucion}~(Propuesta) \fixme{propone}{describe?} una
solución a los problemas del capitulo anterior utilizando los conceptos
estudiados en el capítulo~\ref{chap:tics}, se proponen, además, consideraciones
a tener en cuenta referentes a los componentes e interacción entre los mismos.

Así mismo se definen mecanismos para tratar con los limites de una simulación, y
como utilizar las herramientas descritas en el capítulo~\ref{chap:tics} en
armonía con los objetivos pedagógicos.

% Definición de evaluación

Dada la solución propuesta, en el capítulo~\ref{chap:evaluacion} se propone una
forma de evaluar \fixme{la misma}{No usar}, teniendo en cuenta las hipótesis
planteadas durante el desarrollo de \fixme{la misma}{No usar} y los objetivos
del presente trabajo. Esta evaluación consiste en una serie de experimentos, se
define el universo, la muestra, y los criterios de selección de la muestra.

% Análisis de resultados

En el capítulo~\ref{chap:analisis} se presentan los resultados de los diferentes
experimentos llevados a cabo, obtenidos en forma tabular y con gráficos para
facilitar la comprensión. Además se estudian factores de correlación entre las
distintas variables medidas.

% Conclusiones

Finalmente, en el capítulo~\ref{chap:conclusion}(Conclusión) se presentan las
conclusiones obtenidas a partir de los resultados obtenidos y la experiencia
adquirida durante el desarrollo de la solución.

\observacion{Volver a revisar más adelante}
