\section{Estructura del libro}
    
% TIC en la educación

En el capítulo~\ref{chap:tics} se presenta un resumen sobre el uso de las
\Gls{tic} en la educación, desde sus inicios apoyando a la educación tradicional
hasta su rol actual en las nuevas corrientes pedagógicas. 
%Se hace énfasis en la
%corriente pedagógica llamada construccionismo, explicando su historia, base
%pedagógica, el rol y la forma de uso de las \Gls{tic} en la misma.

% Juegos serios

En el capítulo~\ref{chap:juegos_serios} se describen a los juegos serios, siendo
estos una de las herramientas tecnológicas usadas dentro del proceso de
aprendizaje, se detallan sus características, áreas de aplicación, las
corrientes tecnológicas relacionadas, un ejemplo de proceso de desarrollo y además se 
describen algunos casos de éxito. 
%También se describen tecnologías relacionadas
%a los juegos serios.


% Definición del problema

En el capítulo~\ref{chap:problema} se definen las características del problema
que se quiere abordar. Enfocándonos en el
área de enfermería se describe la situación actual y una propuesta de solución
tecnológica que puede brindar apoyo en el proceso de aprendizaje de los 
estudiantes.

% Hipotesis y requerimientos de la solucion

En el capítulo \ref{chap:requerimientos} se definen los criterios que deben
tenerse en cuenta para seleccionar el contenido a abordar en la solución, se
seleccionan y describen los procedimientos de enfermería que formarán parte del
contenido, el alcance  y los requisitos que debe cumplir la solución.

%Tecnologías utilizadas

En el capítulo~\ref{chap:tecnologias} se describen las diversas tecnologías
utilizadas en el desarrollo de la solución, haciendo especial énfasis en los
motores de videojuegos ya que los mismos proveen el entorno principal para el
desarrollo, se realiza una comparación entre los motores de
videojuegos actuales y se justifica por qué se seleccionó uno en específico. 


% Propuesta de solución

En el capítulo~\ref{chap:solucion} se describe en detalle la implementación de la solución propuesta
basada en la aplicabilidad de los juegos serios en un contexto con las
características mencionadas en el capítulo~\ref{chap:problema} y teniendo en
cuenta las requisitos detallados en el capítulo~\ref{chap:requerimientos}
haciendo uso de las tecnologías descritas en el capítulo anterior. 

% Definición de evaluación

Dada la solución propuesta, en el capítulo~\ref{chap:evaluacion} se describen 
los diferentes métodos utilizados para su evaluación. Además, se presentan y analizan los 
resultados obtenidos, los mismos son mostrados en forma tabular y con gráficos para facilitar la comprensión.


%teniendo en cuenta las hipótesis planteadas durante su
%desarrollo  y los objetivos del presente trabajo. Esta evaluación consiste en
%una serie de pruebas, se define el universo, la muestra, y los criterios de
%selección de la muestra.

% Análisis de resultados

 

% Conclusiones

En el capítulo~\ref{chap:conclusion} se presentan las conclusiones obtenidas a
partir de los resultados obtenidos y la experiencia adquirida durante el
desarrollo de la solución.

% Trabajo furuto

Finalmente, en el capítulo~\ref{chap:futuro} se describen algunos posibles trabajos
que pueden ser desarrollados en el área teniendo en cuenta el presente
trabajo. 

