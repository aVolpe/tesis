\section{Estructura del libro}
    

A continuación se explica el proceso que fue realizado para alcanzar los objetivos citados 
anteriormente, las etapas de este proceso se presentan en diferentes capítulos los cuales 
tratan de diversos aspectos.

% TIC's en la educación
%En el capitulo~\ref{chap:tics} se expande el estado actual de las \Gls{tic} en
%la educación, desde sus inicios, las expectativas creadas, los progresos
%realizados, y las experiencias, la evolución de la teoría hasta hoy en día.

En el capítulo~\ref{chap:tics} se presenta un resumen sobre el uso de las 
\Gls{tic} en la educación, desde sus inicios apoyando a la educación tradicional 
hasta su rol actual en las nuevas corrientes pedagógicas. Se hace énfasis en la 
corriente pedagógica llamada construccionismo, explicando su historia, base 
pedagógica, el rol y la forma de uso de las \Gls{tic} en la misma.


% Juegos serios
%En el capitulo~\ref{chap:juegos_serios} se describen las características y áreas de aplicación de un 
%juego serio, incluyendo corrientes relacionadas y algunos casos de éxito. Se muestra 
%además un esbozo de como desarrollar un juego serio.

En el capítulo~\ref{chap:juegos_serios} se describen a los juegos serios, siendo 
estos una de las posibles herramientas tecnológicas usadas dentro del proceso de aprendizaje, 
se detallan sus características, áreas de aplicación, las corrientes tecnológicas relacionadas, 
y el proceso de desarrollo además de describir algunos casos de éxito.


% Definición del problema

En el capítulo~\ref{chap:problema} se definen las características del problema
que se quiere abordar para darle una solución tecnológica. Enfocándonos en el
área de enfermería se describe la situación actual y la alternativa tecnológica 
que puede brindar apoyo en el proceso de aprendizaje.



%nseñanza utilizado en el \Gls{iab}, enfocado específicamente a la carrera de
%Licenciatura en Enfermería, la distribución de los cursos, carga horaria,
%prácticas de laboratorio y prácticas en hospitales, así como los mecanismos de
%evaluación utilizados.

%Además se citan los problemas existentes, como las dificultades que tienen los
%alumnos en los distintos aspectos que influyen en la vida de un estudiante, y
%problemas inherentes a la enseñanza de profesiones técnicas con métodos de
%enseñanza tradicionales. Y por último se describen los procedimientos de enfermería 
%en los que se enfocará este trabajo.


% Hipotesis y requerimientos de la solucion

En el capítulo~\ref{chap:requerimientos} se describen los requisitos que deben 
tenerse en cuenta para el desarrollo de la solución propuesta, incluyendo las hipótesis asumidas
con respecto a diferentes aspectos a lo mencionado en el capítulo anterior.

%Tecnologías utilizadas

En el capítulo~\ref{chap:tecnologias} se describen las diversas tecnologías utilizadas 
en el desarrollo del proyecto, haciendo especial énfasis en los motores de juego ya que 
los mismos proveen el entorno principal para el desarrollo de la solución, se realiza 
una comparación entre los motores de juegos actuales y se justifica por que se seleccionó 
uno en específico.


% Propuesta de solución

%El capitulo~\ref{chap:solucion} describe una solución  para la aplicabilidad de juegos serios 
%en un contexto con las características mencionadas en el capitulo anterior utilizando los conceptos
%estudiados en el capítulo~\ref{chap:tics}, se proponen, además, consideraciones
%a tener en cuenta referentes a los componentes e interacción entre los mismos.

%Así mismo se definen mecanismos para tratar con los limites de una simulación, y
%como utilizar las herramientas descritas en el capítulo~\ref{chap:tics} en
%armonía con los objetivos pedagógicos.

En el capítulo~\ref{chap:solucion} se describe en detalle la solución propuesta basada en 
la aplicabilidad de los juegos serios en un contexto con las características mencionadas en 
el capítulo~\ref{chap:problema} y teniendo en cuenta las requisitos detallados en el 
capítulo~\ref{chap:requerimientos} haciendo uso de las tecnologías descritas en el capítulo 
anterior. 

% Definición de evaluación

Dada la solución propuesta, en el capítulo~\ref{chap:evaluacion} se propone una
forma de evaluarla, teniendo en cuenta las hipótesis planteadas durante su desarrollo  
y los objetivos del presente trabajo. Esta evaluación consiste en una serie de experimentos, 
se define el universo, la muestra, y los criterios de selección de la muestra.

% Análisis de resultados

En el capítulo~\ref{chap:analisis} se presentan los resultados de los diferentes
experimentos llevados a cabo, obtenidos en forma tabular y con gráficos para
facilitar la comprensión. Además se estudian factores de correlación entre las
distintas variables medidas.

% Conclusiones

En el capítulo~\ref{chap:conclusion} se presentan las
conclusiones obtenidas a partir de los resultados obtenidos y la experiencia
adquirida durante el desarrollo de la solución.

% Trabajo furuto

Finalmente, en el capítulo~\ref{chap:futuro} se describen algunos trabajos posibles que pueden
ser desarrollados en el área teniendo en cuenta el presente trabajo. 

%\observacion{Volver a revisar más adelante}
