%! TEX root = ../main.tex
\section{Problemas actuales}
\label{sec:problemas_actuales}

Sí bien el nivel actual de los egresados del \Gls{iab} en la carrera
Licenciatura en Enfermería es satisfactorio, existen ciertos inconvenientes, los
mismos son recabados de distintas fuentes, como tesis de
alumnos\cite{iab:tesis_alumnos}, comentarios de los profesores y opiniones de
alumnos egresados.

La información no bibliográfica presentada, es fruto de reuniones con
profesores, encargados y directores de la carrera de enfermería, las mismas
deben ser tomadas como experiencias de los mismos, \textbf{no representan un
    compendio de todos los problemas del \Gls{iab}}.

La población del \Gls{iab} esta compuesta en su gran mayoría por personas de
niveles económicos medio-bajos y un gran porcentaje de los alumnos son del
interior del país.

Desde el factor humano, los alumnos tienen varios inconvenientes para asistir a
clases, la mayoría reporto en~\cite{iab:tesis_alumnos} que los principales
inconvenientes son:

\begin{itemize}
    \item \textbf{Carga horaria de trabajos prácticos}, se refiere al tiempo necesario
        por los estudiantes para llevar a cabo un trabajo práctico. 
    \item \textbf{Carga horaria de materias teóricas}, se refiere al tiempo que consumen
        las materias teóricas, cuyo tiempo de estudio es reducido por la
        necesidad de acudir a prácticas de campo en horarios variados.
    \item \textbf{Poca flexibilidad de docentes}, los docentes no siempre toman en cuenta
        las dificultades que debe pasar un alumno para poder acudir a clase, o a
        una práctica de laboratorio.
    \item \textbf{Falta de materiales en los docentes}, la enfermería es un campo muy
        dinámico, los materias se vuelven obsoletos rápidamente, y los docentes
        no cuentan con una fuente actualizada de información.
    \item \textbf{Problemas de transporte}, la ubicación del \Gls{iab} facilita el acceso al
        mismo desde rutas internacionales, pero no se puede decir lo mismo de
        los hospitales donde se realizan prácticas profesionales. Este problema
        es acentuado por la gran cantidad de tiempo que deben pasar los alumnos
        en los medios de transporte para moverse desde sus respectivos hogares
        hasta el \Gls{iab} o a los campos de práctica.
\end{itemize}

Además de estos inconvenientes, existen otros problemas, relacionados al ámbito
familiar de los estudiantes\cite{iab:tesis_alumnos}:

\begin{itemize}
    \item \textbf{Ingreso familiar económico bajo}, la mayoría de las familias que
        soportan a alumnos son de escasos recursos económicos, lo que se acentúa
        cuando deben mantener a un estudiante que requiere constantemente de
        materiales y transporte.
    \item \textbf{Alimentación de baja calidad en la cantina} del \Gls{iab}, como
        consecuencia directa de lo anterior, las cantinas disponibles para los
        alumnos ofrecen menúes de baja calidad con la ventaja de que tienen un
        coste bajo.
    \item \textbf{Problemas familiares}, los alumnos reportan que tienen varios
        problemas familiares, muchas veces acentuados por el ingreso económico
        bajo.
\end{itemize}


Estos problemas son muchas veces causados por que los alumnos no pueden trabajar
durante su formación, lo que acentúa los problemas económicos, pues la familia
debe solventar el periodo académico.

Según~\cite{humphreys2013developing}, la enfermería es una profesión que atrae a
alumnos del tipo divergente\footnote{Son aquellos que aprenden mejor a través de
    la experimentación y de la reflexión acerca de lo experimentado.}, indicando
así que las materias teóricas donde no existe una experimentación activa, por
consiguiente, este tipo de enseñanza es menos productiva que la experimentación
en laboratorios, campos de práctica, etc.

Otros problemas reportados por profesores, es que ciertos profesores de campo
prefieren tener tiempo con los alumnos en el laboratorio, esto se debe
principalmente a:

\begin{itemize}
    \item \textbf{Falta de preparación de los alumnos}, ciertos detalles necesarios para
        la práctica de campo no son completamente cubiertos en el laboratorio.
    \item \textbf{Definición de un protocolo de comunicación}, en la práctica de campo,
        los profesores necesitan comunicarse con sus alumnos de una manera
        rápida y eficiente, entonces los profesores enseñan a sus alumnos
        ciertos códigos que son utilizados para corregir, notificar y enseñar
        durante la práctica.
    \item \textbf{Nerviosismo ante primera práctica}, ciertos alumnos reaccionan de
        manera inesperada la primera vez que deben realizar una práctica, esto
        se debe principalmente a que ciertos procedimientos son impactantes y
        ni el laboratorio ni el aula pueden preparar para este tipo de
        experiencias.
\end{itemize}

\pregunta{Estos son comentarios realizados por profesores, no se si pueden ir
    aquí, o como se deben citar}
