%! TEX root = ../main.tex
\section{Selección de escenas}
\label{sec:seleccion_escenas}

Con los criterios definidos, y como se menciono, se seleccionaron dos
procedimientos de enfermería, los cuales serán incluidos en la solución. A
continuación se fundamentan tales elecciones y se entra en detalle acerca de los
procedimientos seleccionados.

\subsection{Extracción de muestras de sangre}
\label{sec:hemocultivo}

Uno de los procedimientos mas comunes para tomar muestras de sangre es la
punción venosa, uno de los procedimientos invasivos mas frecuentes en los
hospitales, pues ofrece un medio directo de acceso al sistema vascular para
múltiples procesos diagnósticos y terapéuticos.

Este procedimiento fue considerado como uno de los apropiados para la solución
propuesta ya que el mismo posee pasos bien definidos que deben ser seguidos por
el profesional de enfermería. Además la complejidad del procedimiento no es muy
alta y posee pasos muy sensibles que son susceptibles de equivocaciones
especialmente en lo que se refiere a la bioseguridad que sera explicado mas
adelante. Estas apreciaciones fueron realizadas por los profesionales del
\Gls{iab} en las diversas reuniones. A continuación se explican las competencias
básicas de este procedimiento con respecto al plan de estudio de los estudiantes
de enfermería del \Gls{iab}, los criterios según la planilla de practica que
poseen los profesores para evaluar al alumno y el protocolo para llevar a cabo
el procedimiento según~\cite{oms:extraccion} y los profesores del \Gls{iab}.

\subsubsection{Competencias básicas}
\begin{itemize}
\item Ayudar en procedimientos invasivos.
\end{itemize}

Dentro de los procedimientos invasivos se encuentra el procedimiento de
extracción de muestras de sangre.

\subsubsection{Criterios de evaluación de la práctica}
\begin{itemize}
\item Informar al paciente acerca del procedimiento que va a ser realizado.
\item Preparar materiales para la técnica aséptica.
\item Lavado de manos.
\item Vestirse con chaleco estéril, tapaboca, gorro y calzarse guantes.
\item Realizar campo, dejar secar.
\item Punzar, extraer $10mm$ de sangre, comprimir zona de punción.
\item Cambiar la aguja.
\item Introducir la muestra en el frasco correspondiente.
\item Retirar los materiales y equipo de protección personal.
\item Etiquetar y enviar a laboratorio.
\end{itemize}

\subsubsection{Protocolo del procedimiento}
\label{sec:hemocultivo_protocolo}

\begin{enumerate}
\item Preparar el equipo, lo que incluye seleccionar la jeringa adecuada.
\item Identificar al paciente, presentarse y explicarle el procedimiento que va
    a ser realizado.
\item Asepsia de las manos.
\item Llevar el equipo a la unidad en donde se encuentra el paciente.
\item Vestirse con bata estéril, tapaboca y gorro.
\item Calzarse los guantes.
\item Ubicar al paciente en posición adecuada, esto es, el brazo debe estar
    extendido y lo mas relajado posible.
\item Elegir la zona a puncionar, para ello se debe palpar la vena para
    averiguar sus características.
\item Colocar el torniquete, 6 a 10 centímetros por encima de la zona de
    punción.
\item Solicitar al paciente que cierre el puño.
\item Esterilizar la zona de punción.
\item Extraer el protector de la aguja.
\item Tensar la zona de punción.
\item Puncionar la piel con la aguja hacia arriba. La aguja se introduce con un
    ángulo de 10 a 20 grados\todox{Ver si no debe ir el lugar de punción}.
\item Remover el torniquete.
\item Solicitar la apertura del puño.
\item Extraer la muestra se sangre necesaria.
\item Presionar y extraer la aguja.
\item Colocar algodón con alcohol en el punto de punción.
\item Sellar la muestra y enviarlo a su destinatario.
\item Retirar el equipo utilizado, incluyendo bata, tapaboca, gorro y guantes.
\item Asepsia de las manos.
\end{enumerate}


\subsection{Valoración de la escala de Glasgow}
\label{sec:glasgow}

La escala de Glasgow es una escala utilizada como una herramienta de valoración
objetiva del estado de conciencia para las víctimas de traumatismo
craneoencefálico. La escala esta compuesta por la exploración y cuantificación
de tres parámetros: la apertura ocular, la respuesta verbal y la respuesta
motora. Dando un puntaje a la mejor respuesta obtenida en cada ítem. El puntaje
obtenido para cada uno de los tres se suma, con lo que se obtiene el puntaje
final.

Este procedimiento fue considerado como uno de los apropiados para la solución
propuesta ya que el mismo posee pasos bien definidos que deben ser seguidos por
el profesional de enfermería, no se presentan muchos casos como estos en la
realidad durante las practicas de campo realizados por los estudiantes, el
factor de aleatoriedad para cada uno de las respuestas puede ser potenciado
altamente con la solución, además el procedimiento no es complejo. A
continuación se explican las competencias básicas de este procedimiento con
respecto al plan de estudios de los estudiantes de enfermería del \Gls{iab}, los
criterios según la planilla de practica que poseen los profesores para evaluar
al alumno y el protocolo para llevar a cabo el procedimiento según
\cite{protocolo} y los profesores del \Gls{iab}.

\subsubsection{Competencias básicas}
\begin{itemize}
\item Identificar actividades de cuidados según problemas urgentes principales.
\end{itemize}

\subsubsection{Criterios de evaluación de la practica}
\begin{itemize}
\item Cita las actividades que debe realizar según la necesidad de urgencia.
    \begin{enumerate*}
    \item Control de signos vitales.
    \item Inspección cefalocaudal, escala de Glasgow.
    \item Preparación del equipo según prioridad del problema
    \item Analizar su participación en las actividades.
    \item Fundamenta científicamente sus decisiones.
    \end{enumerate*}
\end{itemize}

\subsubsection{Protocolo de práctica}
\label{sec:glasgow_protocolo}

\begin{enumerate}
\item Preparación del material (Escala de Glasgow)
\item Preparación del paciente: comprobar su identidad, mantener una ambiente
    tranquilo evitando interrupciones, requerir la atención del paciente.
\item Colocar al paciente en posición cómoda.
\item Medir la apertura ocular, respuesta motora y respuesta verbal.
\item Registrar la puntuación final obtenida.
\end{enumerate}

Como se comento anteriormente, la escala de coma de Glasgow incluye tres
parámetros: la apertura ocular, la respuesta motora y la respuesta verbal. Cada
parámetro cuenta con items que poseen una puntuación como se muestra en la
tablas~\ref{tab:seleccion_glasgow_respuestas_ocular},~\ref{tab:seleccion_glasgow_respuestas_motor}
y~\ref{tab:seleccion_glasgow_respuestas_verbal}. 

\begin{table}[!hbt]
\centering
\begin{tabular}{lr}
\toprule
\textbf{Apertura ocular} & \textbf{Valor} \\
\midrule
Espontánea & 4 \\
Al hablar & 3 \\
Al dolor & 2 \\
Ausente & 1 \\
\bottomrule
\end{tabular}
\caption{Valoración de las distintas respuestas en la escala de Glasgow,
    respecto a la reacción ocular}
\label{tab:seleccion_glasgow_respuestas_ocular}
\end{table}

\begin{table}[!hbt]
\centering
\begin{tabular}{lr}
\toprule
\textbf{Respuesta motora} & \textbf{Valor} \\
\midrule
Obedece & 6 \\
Localiza & 5 \\
Retira & 4 \\
Flexión anormal & 3 \\
Extiende & 2 \\
Ausente & 1 \\
 & \\
\bottomrule
\end{tabular}
\caption{Valoración de las distintas respuestas en la escala de Glasgow,
    referentes a las respuestas motoras}
\label{tab:seleccion_glasgow_respuestas_motor}
\end{table}

\begin{table}[!hbt]
\centering
\begin{tabular}{lr}
\toprule
\textbf{Respuesta verbal} & \textbf{Valor} \\
\midrule
Orientada & 5 \\
Confusa & 4 \\
Palabras inapropiadas & 3 \\
Palabras incomprensibles & 2 \\
Ausente & 1 \\
\bottomrule
\end{tabular}
\caption{Valoración de las distintas respuestas en la escala de Glasgow
    referentes a la respuesta verbal}
\label{tab:seleccion_glasgow_respuestas_verbal}
\end{table}

Cada uno de los parámetros se evalúa mediante una sub-escala. Cada respuesta se
puntúa con un número, siendo cada una de las sub-escalas independientes. El
estado de conciencia del paciente se determina sumando la puntuaciones de los
tres parámetros. Una vez obtenida la puntuación final, éste se valora según lo
indica la tabla~\ref{tab:seleccion_glasgow_estado}\todox{Estos valores no
    tenemos nosotros, tenemos solo tres.}.

\begin{table}[!hbt]
\centering
\begin{tabular}{llr}
\toprule
\textbf{Nivel ECG} & 
\textbf{Condición clínica} & 
\textbf{Puntuación} \\ 
\midrule
 ECG 1 & Muerte & 3 \\
 ECG 2 & Estado vegetativo & 4 a 6 \\
 ECG 3 & Discapacidad severa & 7 a 9 \\
 ECG 4 & Discapacidad moderada & 10 a 11 \\
 ECG 5 & Recuperacion total & 12 a 15 \\
\bottomrule
\end{tabular}
\caption{Escala de valoración del estado del paciente}
\label{tab:seleccion_glasgow_estado}
\end{table}

\subsection{Bioseguridad}

Además de los procedimientos de enfermería seleccionados y descriptos, existe un
tema mas abordado implícitamente en el procedimiento de extracción de muestras
de sangre, la bioseguridad. La bioseguridad incluye procesos como el lavado de
manos, el uso de elementos de protección personal, entre otros.

Los profesores del \Gls{iab} mencionaron que la bioseguridad es un tema
transversal a todos lo demás y que representa un factor vulnerable por ser
susceptibles de olvido o equivocaciones.

En la solución no se abordara el tema como un procedimiento principal, sino solo
se buscara que los jugadores puedan recordar sin ninguna ayuda cuando y en que
orden dentro del procedimiento que este realizando debe llevar a cabo
actividades de bioseguridad.
