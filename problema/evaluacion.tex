%! TEX root = ../main.tex
\section{Evaluación de lo aprendido}
\label{sec:problema_evaluacion}
\observacio{Reformular el título, no se entiende (por el estudiante? O
    ustedes?)}


\observacion{Resumir más los siguientes 4 párrafos} 

Como dicta su perfil, un egresado debe ser capaz de desempeñar eficientemente su
profesión, el mecanismo que se utiliza para garantizar esto, son las
evaluaciones.

Una evaluación es un proceso que permite verificar el grado del progreso del
estudiante en el logro de los objetivos propuestos en cada
asignatura\cite{iab:est_enfemeria}, existen tres tipos de evaluaciones, exámenes
parciales, exámenes finales y evaluación de la práctica de campo.

La cantidad de evaluaciones parciales esta determinada por la materia y el
consenso de los profesores titulares\cite{iab:est_enfemeria}, la cantidad de
exámenes parciales varía desde dos hasta cuatro por asignatura\footnote{Se
    define examen parcial aquel que mide el rendimiento del periodo
    correspondiente\cite{iab:est_enfemeria}.}.

En cuanto a las evaluaciones finales, existen tres periodos en los cuales un
alumno puede rendir el examen final.

Cada alumno necesita de un $75\%$ de asistencia presencial para tener derecho a
las evaluaciones, así mismo, cada alumno requiere como mínimo $80\%$ de la carga
horaria en prácticas profesionales, el $20\%$ restante lo debe cumplir en un
periodo establecido por el \Gls{iab}.

\fixme{En cuanto a la evaluación de la práctica de campo, el enfoque es
    subjetivo, es decir depende exclusivamente del instructor de la práctica
    determinar si un alumno cuenta o no con la pericia necesaria.}{Concentrarse
    en esto}

La calificación final de materias no profesionales, se obtiene de la siguiente
manera\cite{iab:est_enfemeria}:

\begin{itemize}
    \item $30\%$ de la sumatoria de las pruebas parciales.
    \item $20\%$ de los trabajos prácticos.
    \item $50\%$ del examen final, siempre y cuando el alumno haya obtenido
        una calificación mínima de $2$.
\end{itemize}

\observacion{Que tan importantes son las prácticas? En la evaluación y en la
    vida real. Cocho: poner acá lo que dijo Miguela que la gente tiene miedo}

Para materias profesionales, se utiliza la misma escala descrita anteriormente,
y además se promedia el resultado con la calificación de la práctica
profesional\cite{iab:est_enfemeria}, teniendo en cuenta que un alumno con
calificación $1$ en el examen final tiene automáticamente la nota $1$ en la
materia\cite{iab:est_enfemeria}.

La escala es del $60\%$ para asignaturas no profesionales y del $75\%$ para
materias profesionales, la escala que se utiliza es del $1$ al $5$.

El objetivo final de estas evaluaciones es el de evaluar si un alumno comprende
y tiene la pericia necesaria en todas las competencias básicas de dicha
asignatura.
