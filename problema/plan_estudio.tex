\section{Plan de estudio}
\label{sec:plan_estudio}

La carrera de licenciatura en enfermería en el \Gls{iab} tiene una duración de 4
años, es presencial y tiene una carga total de 3745 horas. Cada alumno debe
aprobar 57 materias anuales.

Para completar las horas necesarias, las clases se desarrollan en el turno
mañana y turno tarde de manera continua, con excepción de los días donde existen
prácticas de campo. La mayoría de las materias son teóricas, desde segundo curso
acceden a los laboratorios especializados del instituto, y desde el tercer año
realizan prácticas de campo en hospitales escuela y hospitales con los cuales el
\Gls{iab} tiene convenios.

El perfil del egresado de la carrera de licenciatura en enfermería,
es\cite{iab:enfermeria}:

\begin{displayquote}

El profesional egresado de la Licenciatura en Enfermería será capaz de
desempeñar eficientemente el saber teórico y práctico en el campo de su
profesión, valorar las necesidades y problemas bio-psico-sociales y espirituales
del individuo, familia y comunidad, brindando apoyo y proponiendo alternativas
de solución, practicar los valores de honradez, solidaridad y respeto al ser
humano en la prestación de servicios de la salud.

\end{displayquote}

Existen tres formas principales de enseñanza dentro del \Gls{iab}, las clases
presenciales, las prácticas de laboratorio y las prácticas de campo.

Los alumnos se dividen en secciones, actualmente existen tres secciones, de $50$
alumnos cada una, la mayoría de las asignaturas cuentan con un trabajo práctico
que debe ser presentado y aprobado para obtener una habilitación para rendir el
examen final.

El plan de estudios se centra en las \emph{competencias básicas} que debe tener
cada alumno al finalizar la materia, estas competencias son facilitadas al
inicio de cada asignatura a los alumnos.

Las competencias básicas son los conocimientos teóricos y prácticos que debe
tener todo profesional de enfermería recién egresado, estas competencias son el
eje central de la carrera y en la obtención de las mismas se centran todas las
actividades curriculares y no curriculares (congresos, encuentros, etc)
realizadas por el \Gls{iab}.
