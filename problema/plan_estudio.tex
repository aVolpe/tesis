\section{Plan de estudio}
\label{sec:plan_estudio}

La carrera de licenciatura de enfermería en el \Gls{iab} dura 4 años, es
presencial y tiene una carga total de 3745 horas. Cada alumno debe aprobar 57
materias y las mismas son anuales.

Para poder completar todas las horas necesarias, las clases duran desde la
mañana hasta la tarde. La mayoría de las materias son teóricas, desde segundo
curso acceden a los laboratorios especializados del instituto, y desde el tercer
año realizan prácticas de campo en hospitales escuela y hospitales con los
cuales el \Gls{iab} tiene convenios.

El perfil del egresado de la carrera de licenciatura en enfermería,
es\cite{iab:enfermeria}:

\begin{displayquote}

El profesional egresado de la Licenciatura en Enfermería será capaz de
desempeñar eficientemente el saber teórico y práctico en el campo de su
profesión, valorar las necesidades y problemas bio-psico-sociales y espirituales
del individuo, familia y comunidad, brindando apoyo y proponiendo alternativas
de solución, practicar los valores de honradez, solidaridad y respeto al ser
humano en la prestación de servicios de la salud.

\end{displayquote}

Como dicta su perfil, un egresado debe ser capaz de desempeñar eficientemente su
profesión, el mecanismo que se utiliza para garantizar esto, son las
evaluaciones.

\subsection{Evaluaciones}

Una evaluación es un proceso que permite verificar el grado del progreso del
estudiante en el logro de los objetivos propuestos en cada
asignatura\cite{iab:est_enfemeria}.

La cantidad de evaluaciones parciales esta determinada por la materia y el
consenso de los profesores titulares\cite{iab:est_enfemeria}, la cantidad de
exámenes varía desde dos hasta cuatro exámenes parciales por
asignatura\footnote{Se define examen parcial aquel que mide el rendimiento del
    periodo correspondiente\cite{iab:est_enfemeria}.}.

En cuanto a las evaluaciones finales, existen tres periodos en los cuales un
alumno puede rendir el examen final.

Cada alumno necesita de un $75\%$ de asistencia presencial para tener derecho a
las evaluaciones, así mismo, cada alumno requiere como mínimo $80\%$ de la carga
horaria en prácticas profesionales, el $20\%$ restante lo debe cumplir en un
periodo establecido por el \Gls{iab}.
