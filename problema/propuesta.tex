%! TEX root = ../main.tex

\section{Propuesta de solución}

% NO VEO COMO EL JUEGO MEJORE ESO por que no se refiere a la falta de comunicacion a la
% hora de darle una retroalimentacion al alumno si no al lenguaje que van a utilizar
%
%\observacion{Estos problemas puede ser atacados por su solución. Se refiere
%a los problemas de comunicación entre profesor y alumno}

Una vez mencionado el estado actual de la formación de profesionales de
enfermería en el \Gls{iab} incluyendo sus problemáticas actuales, \fixme{no
    resulta}{Refinar} extraño pensar en una herramienta tecnológica que les
sirva de apoyo en su proceso de aprendizaje en forma de un juego serio.

Los principales problemas que puede abordar una solución con estas
características son los siguientes:

\observacion{Cambiar el formato, que sea algo así: Tema, solución y luego motivo
    (Párrafos separados)}

\begin{itemize}

\item \textbf{Evaluación}
    
    Las prácticas de campo y en laboratorio son un requisito para aprobar las
    materias que requieren prácticas, no aprobarlas significa volver a cursarlas
    por lo que es importante para un alumno en cuanto a su vida académica y para
    los profesores en cuanto a asegurar que los alumnos tengan los conocimientos
    requeridos.
    
    Una solución tecnológica permite un enfoque objetivo, esto es una diferencia
    sustancial con el mecanismo actual, en el cual la nota del alumno depende de
    la opinión del profesor, esto permite, entre otras cosas, que todos los
    alumnos sean evaluados de manera más similar.
    
\item \textbf{Progreso}

    Si bien existe una planilla de progreso del alumno, esta planilla almacena
    sólo los éxitos del alumno, es decir, cada vez que el instructor
    \emph{considere} que el alumno realizó una tarea de manera correcta, marca
    una casilla en su planilla de progreso.

    Una ventaja de la utilización de la tecnología para la misma tarea, es la
    capacidad que tiene para almacenar información, se podría almacenar no sólo
    cuantas veces cometió un error, sino también los detalles que llevaron al
    error entre múltiples datos interesantes para analizar el avance del alumno,
    como por ejemplo, en que parte del procedimiento encuentra más dificultades,
    cuanto tiempo tarda en realizar el procedimiento, etc.
    
\item \textbf{Tiempo de práctica}
     
    El tiempo que los estudiantes pasan en clases y prácticas es muy extenso por
    lo que no les queda casi tiempo para actividades extras.
    
    En cuanto al aspecto tecnológico, existen herramientas que permiten crear
    soluciones que puedan ser utilizadas de múltiples formas, en el celular, la
    computadora, etc. Es interesante contrastar esta posibilidad con uno de los
    problemas comunes, como es la falta de tiempo, ya que estas herramientas les
    puede permitir estudiar en el tiempo que están fuera de clases.
    
    Dado el problema de tiempo, una solución tecnológica a este debería poder
    utilizarse en cualquier momento y la experiencia no debe ser extensa.

    
\item \textbf{Factor psicológico}

    % A QUE CITAS TE REFERIS??? VOY A COMENTAR
    En el aspecto psicológico, actualmente, existen casos donde los alumnos no
    pueden manejar la primera experiencia con un paciente, la utilización de
    esta solución podría ayudar %(acá citas,
  %  hay varias que sirven, ver ventajas juegos serios) 
    al alumno a entender, interpretar y actuar en una situación realista.   
    
    Esta herramienta también puede enseñarle u orientarle cuando el estudiante
    no haga correctamente los procedimientos, dándole una retroalimentación, y
    permitiendo al mismo experimentar las situaciones sin poner en riesgo su
    vida y la del paciente, adicionalmente no hay riesgos económicos, como el
    desperdicio de material u herramientas.
    
\item \textbf{Ubicuidad}

    Actualmente las prácticas de laboratorio están centralizadas en el
    \Gls{iab}, y las prácticas de campo se realizan en diferentes hospitales.
    Los alumnos invierten gran parte de su tiempo en el transporte hasta el
    lugar de la práctica.
    
    Existen alternativas tecnológicas que permiten al usuario experimentar en
    entornos virtuales desde sus teléfonos móviles, lo que permitiría a los
    mismos utilizarlo en cualquier momento, siendo el único requisito tener el
    dispositivo móvil.
    
    
\item \textbf{Realismo}
    
    % COMO QUE?? VOY A COMENTAR
    %Las prácticas en los laboratorios tienen problemas de realismo, pues
    %utilizan un maniquí estático, el cual tiene ciertos características, como .
    
    Uno de los desafíos impuestos por perseguir la ubicuidad, es el nivel de
    realismo posible, al permitir que la solución corra en un dispositivo móvil,
    la cantidad de espacio para mostrar detalles es reducido y la forma de
    uso de la solución debe ser sencilla. Así, la solución no podrá representar
    un maniquí con la misma facilidad de manipulación que el maniquí del
    \Gls{iab}, si bien, algunos detalles pueden ser más realistas, la
    utilización del mismo no lo será.
    
    Es importante notar que no es un objetivo de la solución realizar una
    simulación detallada de los procedimientos, sino proveer una experiencia que
    permita al usuario comprender el procedimiento y sumergirse en el entorno.
    Según~\cite{videojuegos:gonzaleztardon} es necesario definir minuciosamente
    qué factores son simulados y cuáles no, pues si se simulan demasiados
    factores, el usuario podría perderse en los detalles. 
    
\item \textbf{Enfoque individual}
    
    La cantidad de alumnos dificulta la orientación individual por parte de los
    profesores, por ejemplo, en las materias no críticas, existen $7$ alumnos
    por instructor, en las practicas de laboratorio, existen $50$ alumnos por
    profesor.

    Un entorno virtual permite tratar al alumno individualmente, permitiendo al
    mismo experimentar en un entorno sin poner en riesgo a ningún ser humano.

\end{itemize}




% Una encuesta sobre el acceso a tecnología móvil que tienen los estudiantes de la
% carrera de licenciatura en enfermería detallado en el capitulo xxx y cuyos resultados 
% se muestran en el cap xxx nos indican que en su mayoria tienen acceso a un telefono 
% inteligente por lo que utilizar esta caracteristica como una propuesta para el 
% problema de tiempo resulta ser interesante.


%%%%%%%%%%% ANTES DE ESTO YA TIENE QUE DECIRSE LO DE LOS DISPOSITIVOS MOVILES POR QUE
%%%%%%%%% ACA YA DICE QUE VAMOS A HACER UNA SOLUCION MOVIL

%%%%%%%%%%%%%%%%%
%%%%%%%%%%%%%%%%%
% MIRTA VE ESTO %
%%%%%%%%%%%%%%%%%
%%%%%%%%%%%%%%%%%
Observando estos problemas se propone el desarrollo de una aplicación para
dispositivos móviles que se define como un juego serio llamado \Gls{yave}, el
cual toma prestados conceptos del construccionismo y de las simulaciones
educativas, con el objetivo de proveer un entorno virtual que permita a los
alumnos de enfermería realizar procedimientos en un entorno seguro.

El juego consiste en ofrecer a los usuarios, en este caso alumnos de enfermería,
un medio en el cual puedan realizar procedimientos de enfermería y cuyo objetivo
es servir como herramienta de apoyo en el aprendizaje.

Esta solución ayudaría a los estudiantes a tener más oportunidades de poner en
práctica sus conocimientos con un paciente virtual que incluso puede reaccionar
a sus acciones a diferencia de un maniquí de laboratorio, además les permite
poder hacerlo en cualquier lugar y momento.

\observacion{Donde quedo al final el tema de diseño y las limitaciones?}
\observacion{Las justificaciones están escondidas en cada item}

%\pregunta{Martín: tenemos que hablar más sobre las limitaciones tecnológicas}
%\observacion{En alguna parte de por acá tenemos que hablar de las limitaciones
%    tecnológicas, pues más adelante en cuanto se definen los criterios se
%    necesita.}
%\observacion{No se puede hablar de selección de procedimiento sin hablar de las
%    limitaciones tecnológicas. Hay que describir brevemente cual es el prototipo
%    de dispositivo e interfaz que se quiere utilizar. Y justificar (este es más
%    fácil)}
