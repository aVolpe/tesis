%! TEX root = ../main.tex

\section{Propuesta de solución}
\label{sec:motivacion}

Una vez mencionado el estado actual de la formación de profesionales de
enfermería en el \gls{iab}, incluyendo sus problemáticas actuales, se propone el
desarrollo de un juego serio que incluya la simulación de laboratorios virtuales
como una herramienta de apoyo al proceso de aprendizaje de los alumnos de las
carrera de Licenciatura en Enfermería. Esta solución permitirá realizar
procedimientos de enfermería con un paciente virtual.

Los principales problemas que puede abordar una solución con estas
características son los siguientes:

\begin{itemize}

\item \textbf{Evaluación}

	Una solución tecnológica permite un enfoque objetivo, esto es una diferencia
    sustancial con el mecanismo actual, en el cual la nota del alumno depende de
    la opinión del profesor, esto permite, entre otras cosas, que todos los
    alumnos sean evaluados de manera más similar.
    
    Las prácticas de campo y en laboratorio son un requisito para aprobar las
    materias que requieren prácticas, no aprobarlas significa volver a cursarlas
    por lo que es importante para un alumno en cuanto a su vida académica y para
    los profesores en cuanto a asegurar que los alumnos tengan los conocimientos
    requeridos.
    
    
\item \textbf{Progreso}

    Una ventaja de la utilización de la tecnología es la capacidad que tiene
    para almacenar información. Una solución tecnológica puede almacenar no sólo
    cuantas veces se cometió un error, sino también los detalles que llevaron al
    error, entre múltiples datos interesantes para analizar el avance del
    alumno, como por ejemplo, en que parte del procedimiento encuentra más
    dificultades, cuanto tiempo tarda en realizar el procedimiento, etc.

    En la actualidad existe una planilla de progreso del alumno, esta planilla
    almacena sólo los éxitos del alumno, es decir, cada vez que el instructor
    \emph{considere} que el alumno realizó una tarea de manera correcta, marca
    una casilla en su planilla de progreso.

   
    
\item \textbf{Disponibilidad de tiempo}

    Existen alternativas tecnológicas que permiten al usuario experimentar en
    entornos virtuales desde sus teléfonos móviles, lo que permitiría a los
    mismos utilizarlo en cualquier lugar y momento, siendo el único requisito
    tener el dispositivo móvil.
	
    La penetración dispositivos móviles inteligentes es grande en Paraguay, y
    aumenta considerablemente año por año. En el año $2012$ en el Paraguay existían
    $700.000$ dispositivos móviles inteligentes, en el año $2013$ existían
    $1.055.000$\cite{ultimahora:smartphones}, actualmente existen al menos
    $1.876.000$ dispositivos móviles inteligentes\cite{latamclick:2015}.

    El tiempo que los estudiantes pasan en clases y prácticas es muy extenso por
    lo que no les queda casi tiempo para actividades extras. Actualmente las
    prácticas de laboratorio están centralizadas en el \Gls{iab}, y las
    prácticas de campo se realizan en diferentes hospitales. Los alumnos
    invierten gran parte de su tiempo en el transporte hasta el lugar de la
    práctica.
    
\item \textbf{Factor psicológico}

    Una solución tecnológica puede enseñar u orientar al estudiante cuando no
    realice correctamente los procedimientos, dando una retroalimentación, y
    permitiendo experimentar las situaciones sin poner en riesgo su vida ni la
    del paciente. Adicionalmente no hay riesgos económicos, como el desperdicio
    de materiales o herramientas.

    En el aspecto psicológico, existen casos donde los alumnos no pueden manejar
    la primera experiencia con un paciente, la utilización de esta solución
    podría ayudar al alumno a entender, interpretar y actuar en una situación
    realista.   

\item \textbf{Enfoque individual}

    Un entorno virtual permite tratar al alumno individualmente. Actualmente en
    el \gls{iab}, la cantidad de alumnos dificulta la orientación individual por
    parte de los profesores, por ejemplo, en las materias no críticas, existen
    $10$ alumnos por instructor, en las practicas de laboratorio, existen $50$
    alumnos por profesor.

    

\end{itemize}


Observando estos problemas se propone el desarrollo de una aplicación para
dispositivos móviles que se define como un juego serio llamado
\textit{eTes\~{a}i}, el cual toma prestados conceptos del construccionismo y
de las simulaciones educativas, con el objetivo de proveer un entorno virtual
que permita a los alumnos de enfermería realizar procedimientos en un entorno
seguro.

Esta solución ayudaría a los estudiantes a tener más oportunidades de poner en
práctica sus conocimientos con un paciente virtual que incluso puede reaccionar
a sus acciones a diferencia de un maniquí de laboratorio, además les permite
poder hacerlo en cualquier lugar y momento.

