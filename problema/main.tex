\chapter{Definición del Problema}

Este capitulo define el problema en el cual se enfoca la presente tesis, el
capitulo define el estado actual de la enseñanza a futuros profesionales de
enfermería, y específicamente se centra a los profesionales formados por el
\Gls{iab}. 

Primeramente se muestra, en la sección~\ref{sec:plan_estudio}, como se
estructura la carrera de enfermería, mostrando el plan de estudios, y las
competencias que debe tener el profesional de enfermería recién egresado.

Como la enfermería es una profesión técnica, se hace una breve reseña de los
métodos de enseñanza fuera del aula que se utilizan actualmente, en el \Gls{iab}
se utilizan dos de ellos, prácticas en un laboratorio especializado
(sección~\ref{sec:practica_lab}) y prácticas de campo
(sección~\ref{sec:practica_hos})\footnote{Son prácticas que se realizan con
    pacientes reales en hospitales escuela y otros hospitales con los que el
    \Gls{iab} tiene convenios.}.

Para evaluar el rendimiento y aprendizaje de los alumnos, el \Gls{iab} utiliza
un examen teórico, además cada alumno debe completar una cantidad de horas
mínimas por materia técnica, y además debe completar ciertas actividades en
estas prácticas, esto es definido y mostrado en la
sección\ref{sec:problema_evaluacion}.

La definición del problema inicia en la sección~\ref{sec:problemas_actuales},
donde se describen los principales inconvenientes que tiene la metodología
actual, se utilizan dos puntos de vista para analizar los potenciales problemas,
la perspectiva del alumno y de las competencias básicas que debe adquirir.


La enfermería es un campo amplio, que cuenta con innumerables procedimientos que
los alumnos aprenden y perfeccionan durante su vida académica, en la
sección~\ref{sec:definicion_criterios} se definen los criterios utilizados para
seleccionar aquellas prácticas que serán simuladas como parte de la solución.

En la sección~\ref{sec:seleccion_escenas} se describen los motivos por los
cuales las escenas fueron seleccionadas, explicando sus ventajas y desventajas,
con respecto a las metodologías actuales, y finalmente, en la
sección~\ref{sec:problema_requisitos} se definen cuales son los requisitos a
tener en cuenta a la hora de desarrollar la solución\todox{Ver si no hay una
    mejor manera de definir la sección de requisitos.}.

\section{Plan de estudio}
\label{sec:plan_estudio}

La carrera de licenciatura en enfermería en el \Gls{iab} tiene una duración de 4
años, es presencial y tiene una carga total de 3745 horas. Cada alumno debe
aprobar 57 materias y las mismas son anuales.

Para completar todas las horas necesarias, las clases inician en la mañana y
culminan en la tarde. La mayoría de las materias son teóricas, recién desde
segundo curso acceden a los laboratorios especializados del instituto, y desde
el tercer año realizan prácticas de campo en hospitales escuela y hospitales con
los cuales el \Gls{iab} tiene convenios.

El perfil del egresado de la carrera de licenciatura en enfermería,
es\cite{iab:enfermeria}:

\begin{displayquote}

El profesional egresado de la Licenciatura en Enfermería será capaz de
desempeñar eficientemente el saber teórico y práctico en el campo de su
profesión, valorar las necesidades y problemas bio-psico-sociales y espirituales
del individuo, familia y comunidad, brindando apoyo y proponiendo alternativas
de solución, practicar los valores de honradez, solidaridad y respeto al ser
humano en la prestación de servicios de la salud.

\end{displayquote}

Existen tres formas principales de enseñanza dentro del \Gls{iab}, las clases
presenciales, las prácticas de laboratorio y las prácticas de campo.

Los alumnos se dividen en secciones, actualmente existen tres secciones, de $50$
alumnos cada una, la mayoría de las asignaturas cuentan con un trabajo práctico
que debe ser presentado y aprobado para obtener una habilitación para rendir el
examen final.

El plan de estudios se centra en las \emph{competencias básicas} que debe tener
cada alumno al finalizar la materia, estas competencias son facilitadas al
inicio de cada asignatura a los alumnos, y todo el desarrollo de la materia se
centra en su obtención.

Las competencias básicas son los conocimientos teóricos y prácticos que debe
tener todo profesional de enfermería recién egresado, estas competencias son el
eje central de la carrera y en la obtención de las mismas se centran todas las
actividades curriculares y no curriculares (congresos, encuentros, etc)
realizadas por el \Gls{iab}.

\section{Prácticas en laboratorios}
\section{Prácticas de enfermería}
\label{sec:practica_hos}

Se definen las prácticas profesionales como\cite{iab:est_enfemeria}:

\begin{displayquote}

Son prácticas de enfermería aquellas actividades de integración y de aplicación
de los conocimientos teóricos adquiridos y las destrezas y habilidades,
acercando al estudiante a una realidad concreta que le permita una experiencia
vivencial. Ellas pueden ser desarrolladas en arte de enfermería, así como en los
servicios de salud de diferentes niveles de complejidad, instituciones públicas
y privadas, establecimientos, hogares y comunidad.

\end{displayquote}

Las prácticas de campo son aquellas prácticas profesionales que son realizadas
por los alumnos con pacientes humanos y en hospitales, bajo supervisión de un
profesional y bajo una continua evaluación de sus acciones, las mismas son
llevadas a cabo una vez que los alumnos finalizan las prácticas de laboratorio.

Los alumnos del \Gls{iab} participan en prácticas de campo en diferentes
hospitales dependiendo de las necesidades de cada materia, por ejemplo, los
alumnos de \textit{Enfermería en Urgencias} realizan sus prácticas en el
\textit{Centro de Emergencias Médicas}, otros hospitales utilizados, son el
\textit{Hospital de Clínicas}, y diversos hospitales del \textit{Instituto de
    Previsión Social}.

Para controlar y medir la evolución de los estudiantes existe un grupo de
profesores cuya función es guiar a los alumnos durante las prácticas de campo,
este grupo de profesores son denominados \textbf{instructores}.

Las prácticas se realizan en grupos que varían de 4 a 10 alumnos, dependiendo de
la disponibilidad de instructores y de si el área es crítica o no\footnote{Se
    dice que un paciente esta en estado crítico si su vida depende de un
    procedimiento externo, como una transfusión de sangre, un área se considera
    crítica si los pacientes en su mayoría son críticos}, un instructor puede
manejar más de un grupo en diferentes horarios. 

Cada instructor posee un planilla por alumno donde se realiza el seguimiento de
sus actividades. La creación de esta planilla de actividades es responsabilidad
del instructor, el instructor debe basarse en las competencias básicas de la
asignatura y la misma es validada por la dirección de la carrera, y se considera
que un alumno ha adquirido la pericia necesaria para una asignatura solo sí pudo
completar la planilla del instructor de dicha asignatura. Son registradas todas
las actividades del alumno, pero solo cuentan para el progreso final aquellas
que son realizadas con la pericia necesaria.

La cantidad de alumnos, hace que la práctica de una asignatura rara vez se
realice en un solo hospital, para asignaturas críticas, debe haber
aproximadamente 35 grupos de estudiantes.

%! TEX root = ../main.tex
\section{Evaluación de lo aprendido}
\label{sec:problema_evaluacion}
\observacio{Reformular el título, no se entiende (por el estudiante? O
    ustedes?)}


\observacion{Resumir más los siguientes 4 párrafos} 

Como dicta su perfil, un egresado debe ser capaz de desempeñar eficientemente su
profesión, el mecanismo que se utiliza para garantizar esto, son las
evaluaciones.

Una evaluación es un proceso que permite verificar el grado del progreso del
estudiante en el logro de los objetivos propuestos en cada
asignatura\cite{iab:est_enfemeria}, existen tres tipos de evaluaciones, exámenes
parciales, exámenes finales y evaluación de la práctica de campo.

La cantidad de evaluaciones parciales esta determinada por la materia y el
consenso de los profesores titulares\cite{iab:est_enfemeria}, la cantidad de
exámenes parciales varía desde dos hasta cuatro por asignatura\footnote{Se
    define examen parcial aquel que mide el rendimiento del periodo
    correspondiente\cite{iab:est_enfemeria}.}.

En cuanto a las evaluaciones finales, existen tres periodos en los cuales un
alumno puede rendir el examen final.

Cada alumno necesita de un $75\%$ de asistencia presencial para tener derecho a
las evaluaciones, así mismo, cada alumno requiere como mínimo $80\%$ de la carga
horaria en prácticas profesionales, el $20\%$ restante lo debe cumplir en un
periodo establecido por el \Gls{iab}.

\fixme{En cuanto a la evaluación de la práctica de campo, el enfoque es
    subjetivo, es decir depende exclusivamente del instructor de la práctica
    determinar si un alumno cuenta o no con la pericia necesaria.}{Concentrarse
    en esto}

La calificación final de materias no profesionales, se obtiene de la siguiente
manera\cite{iab:est_enfemeria}:

\begin{itemize}
    \item $30\%$ de la sumatoria de las pruebas parciales.
    \item $20\%$ de los trabajos prácticos.
    \item $50\%$ del examen final, siempre y cuando el alumno haya obtenido
        una calificación mínima de $2$.
\end{itemize}

\observacion{Que tan importantes son las prácticas? En la evaluación y en la
    vida real. Cocho: poner acá lo que dijo Miguela que la gente tiene miedo}

Para materias profesionales, se utiliza la misma escala descrita anteriormente,
y además se promedia el resultado con la calificación de la práctica
profesional\cite{iab:est_enfemeria}, teniendo en cuenta que un alumno con
calificación $1$ en el examen final tiene automáticamente la nota $1$ en la
materia\cite{iab:est_enfemeria}.

La escala es del $60\%$ para asignaturas no profesionales y del $75\%$ para
materias profesionales, la escala que se utiliza es del $1$ al $5$.

El objetivo final de estas evaluaciones es el de evaluar si un alumno comprende
y tiene la pericia necesaria en todas las competencias básicas de dicha
asignatura.

\section{Problemas actuales}

\section{Definición de criterios}
\label{sec:definicion_criterios}

Teniendo en cuenta lo expuesto en las secciones anteriores, se realizaron varias reuniones con la coordinadora de la carrera de Enfermería del \Gls{iab}, Lic. Miguela Hermosilla y otros profesores encargados de las
prácticas de laboratorio y campo de las diversas materias de la carrera.

En estas reuniones se valoraron varios procedimientos que podían ser simulados o presentados en la
propuesta de solución que será descripta en la sección siguiente. De todos los procedimientos evaluados
y discutidos se eligieron dos ya que resulta imposible simular todos los procedimientos de enfermería
existentes. Los principales criterios para la selección fueron los siguientes:

\begin{itemize}
\item Las situaciones simuladas no deben ser muy afectadas por las limitaciones que conllevan las 
características que presentara la solución propuesta.
\item Las situaciones simuladas deben ser de utilidad para los usuarios de la población objetivo.
\item Las situaciones simuladas no deben ser muy complejas ni largas.
\item Las situaciones simuladas deben tener pasos bien definidos.
\item Las situaciones simuladas deben ser difíciles de representar con realismo en un aula o laboratorio.
\end{itemize}



\section{Selección de escenas}
\section{Requisitos de la solución}
\label{sec:problema_requisitos}

La solución propuesta debe cumplir con las condiciones que se citan a continuación:

\subsection{Generales}
\begin{itemize}
\item Se debe utilizar la plataforma Unity3D como herramienta para la creación del entorno.
\item Se debe permitir al jugador poder utilizar el entorno virtual en el momento y lugar que desee es decir, 
se le debe proveer ubicuidad.
\item Las escenas presentadas, junto con lo elementos dentro de ellas deben ser representados en tres dimensiones y
lo mas realista posible.
\item Cada escena representara un procedimiento de enfermería que debe ser realizado por el jugador.
\item El entorno debe permitir al jugador decidir libremente las acciones que quiere realizar.
\item El entorno no debe brindar pistas al jugador acerca de la forma en la que se deben realizar las procedimientos.
\item El entorno debe brindar al jugador información final acerca de su desempeño en la escena.
\item La selección de una escena debe ser realizada mediante un menú ubicado en la pantalla principal o de bienvenida.
\item Se debe permitir que el movimiento de la cámara sea manipulado por el jugador para rotación y aumento de tamaño de la escena.
\item El final de una partida debe indicarse mediante un botón en un menu ubicado en la pantalla principal de la escena.
\end{itemize}

\subsection{Interacción del jugador con los objetos}
\begin{itemize}
\item La selección de objetos se debe realizar mediante un menú ubicado en la pantalla principal de la escena.
\item El objeto actualmente seleccionado se representara mediante un imagen en la escena.
\item Los objetos utilizados como instrumentos necesarios para la realización de un procedimiento deben
utilizarse solo uno a la vez.
\item Un objeto seleccionado debe poder des-seleccionarse volviendo a presionar el botón que le representa en el 
menú.
\end{itemize}

\subsection{Interacción entre objetos}
\begin{itemize}
\item Para realizar acciones sobre el paciente haciendo uso del objeto seleccionado, se debe manipular el objeto
directamente.
\item Para ubicar el objeto seleccionado en la escena se debe tocar sobre el paciente en el lugar donde se desea
que aparezca.
\item Los objetos tendrán menú contextuales para aquellas acciones mas complejas de realizar.
\end{itemize}

\subsection{Acciones del jugador}
\begin{itemize}
\item Las acciones realizadas por los jugadores en el entorno virtual deben ser registradas.
\item Cada acción realizada por el usuario debe ser validada por el entorno, de forma a ofrecer información
correcta acerca de sus logros al final de la partida.
\item La validez de una acción puede requerir en algunos casos que se hayan hecho algunas acciones previas e incluso con cierto orden.
\item No se deben dar pistas o mensajes al jugador durante la partida cuando este realice incorrectamente una acción.
\item Se debe contar con botones para las acciones que modifican el estado del jugador.
\end{itemize}

