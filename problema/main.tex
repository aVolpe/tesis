%! TEX root = ../main.tex
\chapter{Definición del Problema}
\label{chap:problema}

%\observacion{\textbf{Mirta}: VNC (que grabe y haga streaming) para la presentación}
% Checkear estos:
%       https://play.google.com/store/apps/details?id=com.vlcforandroid.vlcdirectprofree
%       https://play.google.com/store/apps/details?id=com.mobzapp.screenstream&hl=es_419


%\observacion{\textbf{Arturo}: Preparar un resumen}
% Todavía no se hizo

En este capítulo se define el problema en el cual se basa el presente trabajo, 
empezando con la definición del estado actual de la enseñanza a estudiantes de la 
carrera de Licenciatura en Enfermería en el \Gls{iab}

Primeramente se muestra, como se estructura la carrera de enfermería, hablando sobre
el plan de estudios, y las competencias que debe tener el profesional de
enfermería recién egresado.

Como la enfermería es una profesión técnica, se hace una breve reseña de los
métodos de enseñanza fuera del aula que se utilizan actualmente.

Se describe como se realiza la evaluación de los estudiantes, tanto en el área
teórica como en el área práctica.

Luego se detallan los principales problemas a abordar, describiendo los
principales inconvenientes que tiene la metodología actual.

Finalmente se define la motivación tras este trabajo de grado, los problemas a los
cuales se busca una solución, citando las ventajas que podría tener una herramienta
que sirva de apoyo a las metodologías actuales.

%La enfermería es un campo amplio, que cuenta con una gran variedad de procedimientos 
%que los alumnos aprenden y perfeccionan durante su
%vida académica, en la sección~\ref{sec:definicion_criterios} se definen los criterios
%utilizados para seleccionar aquellas prácticas que serán simuladas como parte de
%la solución.

%En la sección~\ref{sec:seleccion_escenas} se describen los motivos por los
%cuales las escenas fueron seleccionadas y en que consisten las misma, y
%finalmente, en la sección~\ref{sec:problema_requisitos} se definen cuales son
%los requisitos a tener en cuenta a la hora de diseñar y desarrollar la
%solución.


Una observación importante es que la información no bibliográfica presentada 
en este capítulo, es fruto de reuniones con profesores, encargados y directores 
de la carrera de enfermería del \Gls{iab}, las mismas deben ser tomadas sólo 
como experiencias y apreciaciones, no es información oficial. 
\pregunta{esta bien que diga: no es Información Oficial}

%! TEX root = ../main.tex

\subsection{Estado actual}

La enfermería es una profesión técnica, en esta sección se hace una breve reseña
de los métodos de enseñanza fuera del aula que se utilizan actualmente. Se
describe como se realiza la evaluación de los estudiantes, tanto en el área
teórica como en el área práctica. Luego se detallan los principales problemas a
abordar, describiendo los inconvenientes que tiene la metodología actual.

Una observación importante es que la información no bibliográfica presentada en
esta sección, es fruto de reuniones con profesores, encargados y directores de
la carrera de enfermería del \Gls{iab}.

De este modo, se describe el contexto en el que se quiere aplicar una solución.


\subsubsection{Plan de estudio}
\label{sec:plan_estudio}

La carrera de licenciatura en enfermería en el \Gls{iab} tiene una duración de $4$
años, es completamente presencial y tiene una carga total de $3745$
horas. Cada alumno debe aprobar $57$ materias.

Para completar las horas necesarias, las clases se desarrollan en 
dos turnos de manera continua, con excepción de los días donde existen
prácticas de campo. La mayoría de las materias son teóricas, desde segundo curso
acceden a los laboratorios especializados del instituto, y desde el tercer curso
realizan prácticas de campo en hospitales escuela y hospitales con los cuales el
\Gls{iab} tiene convenios.

La carrera cuenta con $150$ alumnos nuevos por año, los mismos se dividen en tres secciones. 
El perfil del egresado de la carrera de licenciatura en enfermería 
es\cite{iab:enfermeria}:

\begin{displayquote}

El profesional egresado de la Licenciatura en Enfermería será capaz de
desempeñar eficientemente el saber teórico y práctico en el campo de su
profesión, valorar las necesidades y problemas bio-psico-sociales y espirituales
del individuo, familia y comunidad, brindando apoyo y proponiendo alternativas
de solución, practicar los valores de honradez, solidaridad y respeto al ser
humano en la prestación de servicios de la salud.

\end{displayquote}

Existen tres formas principales de enseñanza dentro del \Gls{iab}, 
\begin{enumerate*}[label=\itshape\alph*\upshape.]
\item las clases teóricas, 
\item las prácticas de laboratorio y, 
\item las prácticas de campo.
\end{enumerate*}


El plan de estudios se centra en las \emph{competencias básicas} que debe tener
cada alumno al finalizar la materia, estas competencias son facilitadas al
inicio de cada asignatura a los alumnos.

Las competencias básicas son los conocimientos teóricos y prácticos que debe
tener todo profesional de enfermería recién egresado, estas competencias son el
eje central de la carrera y en la obtención de las mismas se centran todas las
actividades curriculares y no curriculares (congresos, encuentros, etc)
realizadas por el \Gls{iab}.



\subsubsection{Prácticas en laboratorios}
\label{sec:practica_lab}

El \Gls{iab} cuenta con un laboratorio especializado para la práctica de los
estudiantes de enfermería. El laboratorio es utilizado por los alumnos 
desde su segundo año de formación, y en el mismo se desarrollan todas las materias 
prácticas, de manera a realizar una formación previa a las prácticas de campo 
explicadas más adelante.

El número de alumnos dificulta la enseñanza individual, por ello las prácticas se
dividen en dos partes, en la primera, similar a una aula tradicional, los
alumnos se sientan y observan al profesor realizar una simulación de
procedimientos sobre un voluntario, en este punto, el profesor realiza las
observaciones que crea son necesarias para llevar a cabo la práctica
profesional, da consejos y responde a las dudas de los alumnos.
Se utilizan modelos del cuerpo humano para simular
algunos procedimientos, en la figura~\ref{fig:iab_veno} se observa un modelo del
brazo humano utilizado para procedimientos de venopunción.

\begin{figure}[h!t] 
\centering 
\includegraphics[scale=0.2,natwidth=100,natheight=100]{problema/iab_sala_2.jpg}
\caption{Elementos utilizados para mostrar procedimientos de venopunción}
\label{fig:iab_veno}
\end{figure}

En la segunda parte, los alumnos pasan a un laboratorio que contiene las
herramientas necesarias para la práctica (maniquíes, camas de hospital, y otros
elementos, como se ven en la figura~\ref{fig:iab_lab}), donde pueden explorar y
practicar siempre bajo tutela del profesor. Esto se diferencia
principalmente de la primera parte, en que hay más material para las pruebas y los
alumnos pueden realizar por sí mismos una simulación de los procedimientos.

\begin{figure}[h!t] 
\centering 
\includegraphics[scale=0.3]{problema/iab_sala_1.jpg}
\caption{Laboratorio de enfermería del \Gls{iab}}
\label{fig:iab_lab}
\end{figure}


El maniquí que se observa en la figura~\ref{fig:iab_mani}, tiene ciertas
características que facilitan la práctica, por ejemplo, tiene un esquema de los
vasos sanguíneos en ambos brazos. Este maniquí se utiliza además para mostrar
las partes del cuerpo donde se puede realizar la venopunción, para mostrar la
zona específica donde se debe realizar la reanimación, y otras zonas importantes
para la práctica de enfermería.


\begin{figure}[b!t] 
\centering 
\includegraphics[scale=0.15]{problema/iab_sala_3.jpg}
\caption{Una instructora de laboratorio muestra las partes del maniquí utilizado
    para en el laboratorio de enfermería.}
\label{fig:iab_mani}
\end{figure}

Además existen varias camas de hospitales (como se observa en la
figura~\ref{fig:iab_lab} a la izquierda), donde se practica la higienización del
paciente, como utilizar los mecanismos de ajuste de la cama, las diferentes
telas utilizadas para las sábanas, y otros aspectos relacionados al cuidado de
un paciente en cama.


\subsubsection{Prácticas de campo}
\label{sec:practica_hos}

Las prácticas de campo son aquellas prácticas profesionales que son
realizadas por los alumnos con pacientes humanos y en hospitales, bajo
supervisión de un profesional y bajo una continua evaluación de sus
acciones, las mismas son llevadas a cabo una vez que los alumnos finalizan
las prácticas de laboratorio.


Los alumnos del \Gls{iab} participan en prácticas de campo 
en diferentes hospitales dependiendo de las necesidades de cada
materia, por ejemplo, los alumnos de \textit{Enfermería en Urgencias} realizan
sus prácticas en el \textit{Centro de Emergencias Médicas}, otros hospitales
utilizados, son el \textit{Hospital de Clínicas}, y diversos hospitales del
\textit{Instituto de Previsión Social}.


Para controlar y medir la evolución de los estudiantes existe un grupo de
profesores cuya función es guiar a los alumnos durante las prácticas de campo,
este grupo de profesores son denominados \textbf{instructores}.

Las prácticas se realizan en grupos que varían de $4$ a $10$ alumnos, dependiendo de
la disponibilidad de instructores y de si el área es crítica o no\footnote{Se
dice que un paciente esta en estado crítico si su vida depende de un
procedimiento externo, como una transfusión de sangre. Un área se considera
crítica si los pacientes en su mayoría son críticos}, un instructor puede
manejar más de un grupo en diferentes horarios. 

Cada instructor posee un planilla por alumno donde se realiza el seguimiento de
sus actividades. La creación de esta planilla de actividades es responsabilidad
del instructor, el instructor debe basarse en las competencias básicas de la
asignatura y la misma es validada por la dirección de la carrera, se considera
que un alumno ha adquirido la pericia\footnote{Sabiduría, práctica, experiencia y 
habilidad en una ciencia o arte.} necesaria para una asignatura solo si pudo
completar la planilla del instructor. Son registradas todas
las actividades del alumno, pero sólo son tomadas en cuenta para el progreso final 
aquellas que son realizadas con la pericia requerida.

\subsubsection{Evaluación a los estudiantes}
\label{sec:problema_evaluacion}

Como dicta su perfil, un egresado de Licenciatura en Enfermería 
debe ser capaz de desempeñar eficientemente su
profesión, el mecanismo que se utiliza para garantizar esto son las
evaluaciones.

Una evaluación es un proceso que permite verificar el grado del progreso del
estudiante en el logro de los objetivos propuestos en cada
asignatura\cite{iab:est_enfemeria}, existen tres tipos de evaluaciones, exámenes
parciales, exámenes finales y evaluación de la práctica de campo.

% VER PARA BORRAR
La cantidad de evaluaciones parciales\footnote{Se
    define examen parcial aquel que mide el rendimiento del período
    correspondiente\cite{iab:est_enfemeria}} está determinada por la materia y el
consenso de los profesores titulares\cite{iab:est_enfemeria}. En cuanto a las 
evaluaciones finales, existen tres períodos en los cuales un alumno puede rendir 
el examen final.


Cada alumno necesita de un $75\%$ de asistencia presencial para tener derecho a
las evaluaciones, así mismo, cada alumno requiere como mínimo $80\%$ de la carga
horaria en prácticas profesionales, el $20\%$ restante lo debe cumplir en un
periodo establecido por el \Gls{iab}.
% esto del 20% restante se puede borrar

En cuanto a la evaluación de la práctica de campo, el enfoque es
subjetivo, es decir depende exclusivamente del instructor de la práctica
determinar si un alumno cuenta o no con la pericia necesaria.
    
Cabe destacar que, si el alumno no aprueba sus prácticas de campo no tiene 
derecho al examen teórico de la materia en cuestión y por lo tanto debe volver 
a cursar la materia. De esta forma, aprobar la práctica de campo se convierte 
en un requisito importante para el progreso del alumno en su vida académica.

El objetivo final de estas evaluaciones es el de evaluar si un alumno comprende
y tiene la pericia necesaria en todas las competencias básicas de una
asignatura.



\subsection{Problemas actuales}
\label{sec:problemas_actuales}
%\observacion{Quizás convenga restar más, tipo sección 4.2}

Si bien el nivel actual de los egresados del \Gls{iab} en la carrera
Licenciatura en Enfermería es satisfactorio, existen ciertos inconvenientes, los
mismos son recabados de distintas fuentes, como tesis de
alumnos\cite{iab:tesis_atencion,iab:tesis_alumnos} y apreciaciones de los
profesores y de alumnos egresados.

En algunos casos los profesores de campo prefieren tener las primeras clases en
el laboratorio antes de ir a los hospitales debido a que se suelen presentar los
siguientes inconvenientes:

\begin{itemize}

    \item \textbf{Falta de preparación de los alumnos:} ciertos detalles
        necesarios para la práctica de campo no son completamente cubiertos en
        el laboratorio.

    \item \textbf{Nerviosismo ante primera práctica:} ciertos alumnos reaccionan
        de manera inesperada la primera vez que deben realizar una práctica,
        esto se debe principalmente a que ciertos procedimientos son impactantes
        y ni el laboratorio ni el aula pueden preparar para este tipo de
        experiencias.
                
    \item \textbf{Definición de un protocolo de comunicación:} en la práctica de
        campo, los profesores necesitan comunicarse con sus alumnos de una
        manera rápida y eficiente, debido a esto los profesores enseñan a sus
        alumnos ciertos códigos que son utilizados para corregir, notificar y
        enseñar durante la práctica.
          
\end{itemize}


En cuanto al punto de vista de los alumnos, los principales inconvenientes que
tienen los alumnos de enfermería del \Gls{iab} son:

\begin{itemize}

    \item \textbf{Carga horaria de trabajos prácticos:} se refiere al tiempo
        necesario por los estudiantes para llevar a cabo un trabajo
        práctico\cite{iab:tesis_alumnos}.
        
    \item \textbf{Carga horaria de materias teóricas:} se refiere al tiempo que
        consumen las materias teóricas, cuyo tiempo de estudio es reducido por
        la necesidad de acudir a prácticas de campo en horarios
        variados\cite{iab:tesis_alumnos}.
        

    \item \textbf{Falta de materiales para los profesores:} la enfermería es un
        área en constante evolución, los materiales se vuelven obsoletos
        rápidamente, y los profesores no cuentan con una fuente actualizada de
        información\cite{iab:tesis_alumnos}.
        
    \item \textbf{Problemas de transporte:} la ubicación del \Gls{iab} facilita
        el acceso al mismo desde rutas internacionales, pero no se puede decir
        lo mismo de los hospitales donde se realizan prácticas
        profesionales\cite{iab:tesis_alumnos}. 
        
        Este problema es acentuado por la gran cantidad de tiempo que deben
        pasar los alumnos en los medios de transporte para moverse desde sus
        respectivos hogares hasta el \Gls{iab} o a los campos de
        práctica\cite{iab:tesis_alumnos}.
        
        Adicionalmente, la población del \Gls{iab} esta compuesta en su gran
        mayoría por personas de niveles económicos medio-bajos y un gran
        porcentaje de los alumnos son del interior del
        país\cite{iab:tesis_alumnos}.
    
    \item \textbf{Preparación para las prácticas:} los alumnos rara vez están
        completamente preparados la primera vez que realizan una práctica de
        campo, muchos sufren ataques de pánico y no pueden reaccionar de manera
        correcta.
        
    \item \textbf{Cantidad de alumnos:} debido a la cantidad de alumnos, las
        prácticas de campo rara vez se realizan en un sólo hospital, para
        asignaturas críticas, se forman aproximadamente $35$ grupos de
        estudiantes.

\end{itemize}

%! TEX root = ../main.tex

\section{Propuesta de solución}

% NO VEO COMO EL JUEGO MEJORE ESO por que no se refiere a la falta de comunicacion a la
% hora de darle una retroalimentacion al alumno si no al lenguaje que van a utilizar
%
%\observacion{Estos problemas puede ser atacados por su solución. Se refiere
%a los problemas de comunicación entre profesor y alumno}

Una vez mencionado el estado actual de la formación de profesionales de
enfermería en el \Gls{iab} incluyendo sus problemáticas actuales, \fixme{no
    resulta}{Refinar} extraño pensar en una herramienta tecnológica que les
sirva de apoyo en su proceso de aprendizaje en forma de un juego serio.

Los principales problemas que puede abordar una solución con estas
características son los siguientes:

\observacion{Cambiar el formato, que sea algo así: Tema, solución y luego motivo
    (Párrafos separados)}

\begin{itemize}

\item \textbf{Evaluación}
    
    Las prácticas de campo y en laboratorio son un requisito para aprobar las
    materias que requieren prácticas, no aprobarlas significa volver a cursarlas
    por lo que es importante para un alumno en cuanto a su vida académica y para
    los profesores en cuanto a asegurar que los alumnos tengan los conocimientos
    requeridos.
    
    Una solución tecnológica permite un enfoque objetivo, esto es una diferencia
    sustancial con el mecanismo actual, en el cual la nota del alumno depende de
    la opinión del profesor, esto permite, entre otras cosas, que todos los
    alumnos sean evaluados de manera más similar.
    
\item \textbf{Progreso}

    Si bien existe una planilla de progreso del alumno, esta planilla almacena
    sólo los éxitos del alumno, es decir, cada vez que el instructor
    \emph{considere} que el alumno realizó una tarea de manera correcta, marca
    una casilla en su planilla de progreso.

    Una ventaja de la utilización de la tecnología para la misma tarea, es la
    capacidad que tiene para almacenar información, se podría almacenar no sólo
    cuantas veces cometió un error, sino también los detalles que llevaron al
    error entre múltiples datos interesantes para analizar el avance del alumno,
    como por ejemplo, en que parte del procedimiento encuentra más dificultades,
    cuanto tiempo tarda en realizar el procedimiento, etc.
    
\item \textbf{Tiempo de práctica}
     
    El tiempo que los estudiantes pasan en clases y prácticas es muy extenso por
    lo que no les queda casi tiempo para actividades extras.
    
    En cuanto al aspecto tecnológico, existen herramientas que permiten crear
    soluciones que puedan ser utilizadas de múltiples formas, en el celular, la
    computadora, etc. Es interesante contrastar esta posibilidad con uno de los
    problemas comunes, como es la falta de tiempo, ya que estas herramientas les
    puede permitir estudiar en el tiempo que están fuera de clases.
    
    Dado el problema de tiempo, una solución tecnológica a este debería poder
    utilizarse en cualquier momento y la experiencia no debe ser extensa.

    
\item \textbf{Factor psicológico}

    % A QUE CITAS TE REFERIS??? VOY A COMENTAR
    En el aspecto psicológico, actualmente, existen casos donde los alumnos no
    pueden manejar la primera experiencia con un paciente, la utilización de
    esta solución podría ayudar %(acá citas,
  %  hay varias que sirven, ver ventajas juegos serios) 
    al alumno a entender, interpretar y actuar en una situación realista.   
    
    Esta herramienta también puede enseñarle u orientarle cuando el estudiante
    no haga correctamente los procedimientos, dándole una retroalimentación, y
    permitiendo al mismo experimentar las situaciones sin poner en riesgo su
    vida y la del paciente, adicionalmente no hay riesgos económicos, como el
    desperdicio de material u herramientas.
    
\item \textbf{Ubicuidad}

    Actualmente las prácticas de laboratorio están centralizadas en el
    \Gls{iab}, y las prácticas de campo se realizan en diferentes hospitales.
    Los alumnos invierten gran parte de su tiempo en el transporte hasta el
    lugar de la práctica.
    
    Existen alternativas tecnológicas que permiten al usuario experimentar en
    entornos virtuales desde sus teléfonos móviles, lo que permitiría a los
    mismos utilizarlo en cualquier momento, siendo el único requisito tener el
    dispositivo móvil.
    
    
\item \textbf{Realismo}
    
    % COMO QUE?? VOY A COMENTAR
    %Las prácticas en los laboratorios tienen problemas de realismo, pues
    %utilizan un maniquí estático, el cual tiene ciertos características, como .
    
    Uno de los desafíos impuestos por perseguir la ubicuidad, es el nivel de
    realismo posible, al permitir que la solución corra en un dispositivo móvil,
    la cantidad de espacio para mostrar detalles es reducido y la forma de
    uso de la solución debe ser sencilla. Así, la solución no podrá representar
    un maniquí con la misma facilidad de manipulación que el maniquí del
    \Gls{iab}, si bien, algunos detalles pueden ser más realistas, la
    utilización del mismo no lo será.
    
    Es importante notar que no es un objetivo de la solución realizar una
    simulación detallada de los procedimientos, sino proveer una experiencia que
    permita al usuario comprender el procedimiento y sumergirse en el entorno.
    Según~\cite{videojuegos:gonzaleztardon} es necesario definir minuciosamente
    qué factores son simulados y cuáles no, pues si se simulan demasiados
    factores, el usuario podría perderse en los detalles. 
    
\item \textbf{Enfoque individual}
    
    La cantidad de alumnos dificulta la orientación individual por parte de los
    profesores, por ejemplo, en las materias no críticas, existen $7$ alumnos
    por instructor, en las practicas de laboratorio, existen $50$ alumnos por
    profesor.

    Un entorno virtual permite tratar al alumno individualmente, permitiendo al
    mismo experimentar en un entorno sin poner en riesgo a ningún ser humano.

\end{itemize}




% Una encuesta sobre el acceso a tecnología móvil que tienen los estudiantes de la
% carrera de licenciatura en enfermería detallado en el capitulo xxx y cuyos resultados 
% se muestran en el cap xxx nos indican que en su mayoria tienen acceso a un telefono 
% inteligente por lo que utilizar esta caracteristica como una propuesta para el 
% problema de tiempo resulta ser interesante.


%%%%%%%%%%% ANTES DE ESTO YA TIENE QUE DECIRSE LO DE LOS DISPOSITIVOS MOVILES POR QUE
%%%%%%%%% ACA YA DICE QUE VAMOS A HACER UNA SOLUCION MOVIL

%%%%%%%%%%%%%%%%%
%%%%%%%%%%%%%%%%%
% MIRTA VE ESTO %
%%%%%%%%%%%%%%%%%
%%%%%%%%%%%%%%%%%
Observando estos problemas se propone el desarrollo de una aplicación para
dispositivos móviles que se define como un juego serio llamado \Gls{yave}, el
cual toma prestados conceptos del construccionismo y de las simulaciones
educativas, con el objetivo de proveer un entorno virtual que permita a los
alumnos de enfermería realizar procedimientos en un entorno seguro.

El juego consiste en ofrecer a los usuarios, en este caso alumnos de enfermería,
un medio en el cual puedan realizar procedimientos de enfermería y cuyo objetivo
es servir como herramienta de apoyo en el aprendizaje.

Esta solución ayudaría a los estudiantes a tener más oportunidades de poner en
práctica sus conocimientos con un paciente virtual que incluso puede reaccionar
a sus acciones a diferencia de un maniquí de laboratorio, además les permite
poder hacerlo en cualquier lugar y momento.

\observacion{Donde quedo al final el tema de diseño y las limitaciones?}
\observacion{Las justificaciones están escondidas en cada item}

%\pregunta{Martín: tenemos que hablar más sobre las limitaciones tecnológicas}
%\observacion{En alguna parte de por acá tenemos que hablar de las limitaciones
%    tecnológicas, pues más adelante en cuanto se definen los criterios se
%    necesita.}
%\observacion{No se puede hablar de selección de procedimiento sin hablar de las
%    limitaciones tecnológicas. Hay que describir brevemente cual es el prototipo
%    de dispositivo e interfaz que se quiere utilizar. Y justificar (este es más
%    fácil)}


%\section{Plan de estudio}
\label{sec:plan_estudio}

La carrera de licenciatura en enfermería en el \Gls{iab} tiene una duración de 4
años, es presencial y tiene una carga total de 3745 horas. Cada alumno debe
aprobar 57 materias anuales.

Para completar las horas necesarias, las clases se desarrollan en el turno
mañana y turno tarde de manera continua, con excepción de los días donde existen
prácticas de campo. La mayoría de las materias son teóricas, desde segundo curso
acceden a los laboratorios especializados del instituto, y desde el tercer año
realizan prácticas de campo en hospitales escuela y hospitales con los cuales el
\Gls{iab} tiene convenios.

El perfil del egresado de la carrera de licenciatura en enfermería,
es\cite{iab:enfermeria}:

\begin{displayquote}

El profesional egresado de la Licenciatura en Enfermería será capaz de
desempeñar eficientemente el saber teórico y práctico en el campo de su
profesión, valorar las necesidades y problemas bio-psico-sociales y espirituales
del individuo, familia y comunidad, brindando apoyo y proponiendo alternativas
de solución, practicar los valores de honradez, solidaridad y respeto al ser
humano en la prestación de servicios de la salud.

\end{displayquote}

Existen tres formas principales de enseñanza dentro del \Gls{iab}, las clases
presenciales, las prácticas de laboratorio y las prácticas de campo.

Los alumnos se dividen en secciones, actualmente existen tres secciones, de $50$
alumnos cada una, la mayoría de las asignaturas cuentan con un trabajo práctico
que debe ser presentado y aprobado para obtener una habilitación para rendir el
examen final.

El plan de estudios se centra en las \emph{competencias básicas} que debe tener
cada alumno al finalizar la materia, estas competencias son facilitadas al
inicio de cada asignatura a los alumnos.

Las competencias básicas son los conocimientos teóricos y prácticos que debe
tener todo profesional de enfermería recién egresado, estas competencias son el
eje central de la carrera y en la obtención de las mismas se centran todas las
actividades curriculares y no curriculares (congresos, encuentros, etc)
realizadas por el \Gls{iab}.

%\section{Prácticas en laboratorios}
\label{sec:practica_lab}

El \Gls{iab} cuenta con un laboratorio especializado para la práctica de los
estudiantes de enfermería.

El laboratorio es utilizado por los alumnos desde su segundo año de formación, y
en el mismo se desarrollan todas las materias prácticas, de manera a realizar
una formación previa a las prácticas de campo explicadas más adelante.

La cantidad de alumnos no permite que sea factible la enseñanza individual, por
ello las prácticas se dividen en dos partes, en la primera, similar a una aula
tradicional, los alumnos se sientan y observan al profesor realizar una
simulación del procedimiento sobre un voluntario, en este punto, el profesor
realiza las observaciones que crea son necesarias para llevar a cabo la práctica
profesional, da consejos y responde a las dudas de los alumnos, existen modelos
del cuerpo humano utilizados para simular algunos procedimientos, en la
figura~\ref{fig:iab_veno} se observa un modelo del brazo humano utilizado para
procedimientos de venopunción.

\begin{figure}[h!t] 
\centering 
\includegraphics[scale=0.2,natwidth=100,natheight=100]{problema/iab_sala_2.jpg}
\caption{Elementos utilizados para mostrar procedimientos de venopunción}
\label{fig:iab_veno}
\end{figure}

En la segunda parte, los alumnos pasan a un laboratorio que contiene las
herramientas necesarias para la práctica (maniquís, camas de hospital, y otros
elementos, como se ven en la figura~\ref{fig:iab_lab}), donde pueden explorar y
practicar siempre bajo tutela del profesor. Esta parte se diferencia
principalmente de la primera, en que hay más materiales para las pruebas y los
alumnos pueden realizar por sí mismos una simulación de los procedimientos.

\begin{figure}[h!t] 
\centering 
\includegraphics[scale=0.3,natwidth=100,natheight=100]{problema/iab_sala_1.jpg}
\caption{Laboratorio de enfermería del \Gls{iab}}
\label{fig:iab_lab}
\end{figure}

Se describen algunos de los elementos que componen al laboratorio, estos
elementos son utilizados de ejemplo para mostrar como se lleva a cabo la
práctica\martin{No se como decir que hay más elementos pero no es importante
    describirlos todos}.

El maniquí que se observa en la figura~\ref{fig:iab_mani}, tiene ciertas
características que facilitan la práctica, por ejemplo, tiene un esquema de los
vasos sanguíneos en ambos brazos. Este maniquí se utiliza además para mostrar
las partes del cuerpo donde se puede realizar la venopunción, para mostrar la
zona específica donde se debe realizar la reanimación, y otras zonas importantes
para la práctica de enfermería.

\begin{figure}[h!t] 
\centering 
\includegraphics[scale=0.2,natwidth=400,natheight=200]{problema/iab_sala_3.jpg}
\caption{Una instructora de laboratorio muestra las partes del maniquí utilizado
    para en el laboratorio de enfermería.}
\label{fig:iab_mani}
\end{figure}

Además existen varias camas de hospitales (como se observa en la
figura~\ref{fig:iab_lab} a la izquierda), donde se practica la higienización del
paciente, como utilizar los mecanismos de ajuste de la cama, las diferentes
telas utilizadas para las sabanas, y otros aspectos relacionados al cuidado de
un paciente en cama.

%\section{Prácticas de enfermería}
\label{sec:practica_hos}

Se definen las prácticas profesionales como\cite{iab:est_enfemeria}:

\begin{displayquote}

Son prácticas de enfermería aquellas actividades de integración y de aplicación
de los conocimientos teóricos adquiridos y las destrezas y habilidades,
acercando al estudiante a una realidad concreta que le permita una experiencia
vivencial. Ellas pueden ser desarrolladas en arte de enfermería, así como en los
servicios de salud de diferentes niveles de complejidad, instituciones públicas
y privadas, establecimientos, hogares y comunidad.

\end{displayquote}

Las prácticas de campo son aquellas prácticas profesionales que son realizadas
por los alumnos con pacientes humanos y en hospitales, bajo supervisión de un
profesional y bajo una continua evaluación de sus acciones, las mismas son
llevadas a cabo una vez que los alumnos finalizan las prácticas de laboratorio.

Los alumnos del \Gls{iab} participan en prácticas de campo en diferentes
hospitales dependiendo de las necesidades de cada materia, por ejemplo, los
alumnos de \textit{Enfermería en Urgencias} realizan sus prácticas en el
\textit{Centro de Emergencias Médicas}, otros hospitales utilizados, son el
\textit{Hospital de Clínicas}, y diversos hospitales del \textit{Instituto de
    Previsión Social}.

Para controlar y medir la evolución de los estudiantes existe un grupo de
profesores cuya función es guiar a los alumnos durante las prácticas de campo,
este grupo de profesores son denominados \textbf{instructores}.

Las prácticas se realizan en grupos que varían de 4 a 10 alumnos, dependiendo de
la disponibilidad de instructores y de si el área es crítica o no\footnote{Se
    dice que un paciente esta en estado crítico si su vida depende de un
    procedimiento externo, como una transfusión de sangre, un área se considera
    crítica si los pacientes en su mayoría son críticos}, un instructor puede
manejar más de un grupo en diferentes horarios. 

Cada instructor posee un planilla por alumno donde se realiza el seguimiento de
sus actividades. La creación de esta planilla de actividades es responsabilidad
del instructor, el instructor debe basarse en las competencias básicas de la
asignatura y la misma es validada por la dirección de la carrera, y se considera
que un alumno ha adquirido la pericia necesaria para una asignatura solo sí pudo
completar la planilla del instructor de dicha asignatura. Son registradas todas
las actividades del alumno, pero solo cuentan para el progreso final aquellas
que son realizadas con la pericia necesaria.

La cantidad de alumnos, hace que la práctica de una asignatura rara vez se
realice en un solo hospital, para asignaturas críticas, debe haber
aproximadamente 35 grupos de estudiantes.

%%! TEX root = ../main.tex
\section{Evaluación de lo aprendido}
\label{sec:problema_evaluacion}
\observacion{Reformular el título, no se entiende (por el estudiante? O
    ustedes?)}


\observacion{Resumir más los siguientes 4 párrafos} 

Como dicta su perfil, un egresado debe ser capaz de desempeñar eficientemente su
profesión, el mecanismo que se utiliza para garantizar esto, son las
evaluaciones.

Una evaluación es un proceso que permite verificar el grado del progreso del
estudiante en el logro de los objetivos propuestos en cada
asignatura\cite{iab:est_enfemeria}, existen tres tipos de evaluaciones, exámenes
parciales, exámenes finales y evaluación de la práctica de campo.

La cantidad de evaluaciones parciales esta determinada por la materia y el
consenso de los profesores titulares\cite{iab:est_enfemeria}, la cantidad de
exámenes parciales varía desde dos hasta cuatro por asignatura\footnote{Se
    define examen parcial aquel que mide el rendimiento del periodo
    correspondiente\cite{iab:est_enfemeria}.}.

En cuanto a las evaluaciones finales, existen tres periodos en los cuales un
alumno puede rendir el examen final.

Cada alumno necesita de un $75\%$ de asistencia presencial para tener derecho a
las evaluaciones, así mismo, cada alumno requiere como mínimo $80\%$ de la carga
horaria en prácticas profesionales, el $20\%$ restante lo debe cumplir en un
periodo establecido por el \Gls{iab}.

\fixme{En cuanto a la evaluación de la práctica de campo, el enfoque es
    subjetivo, es decir depende exclusivamente del instructor de la práctica
    determinar si un alumno cuenta o no con la pericia necesaria.}{Concentrarse
    en esto}

La calificación final de materias no profesionales, se obtiene de la siguiente
manera\cite{iab:est_enfemeria}:

\begin{itemize}
    \item $30\%$ de la sumatoria de las pruebas parciales.
    \item $20\%$ de los trabajos prácticos.
    \item $50\%$ del examen final, siempre y cuando el alumno haya obtenido
        una calificación mínima de $2$.
\end{itemize}

\observacion{Que tan importantes son las prácticas? En la evaluación y en la
    vida real. Cocho: poner acá lo que dijo Miguela que la gente tiene miedo}

Para materias profesionales, se utiliza la misma escala descrita anteriormente,
y además se promedia el resultado con la calificación de la práctica
profesional\cite{iab:est_enfemeria}, teniendo en cuenta que un alumno con
calificación $1$ en el examen final tiene automáticamente la nota $1$ en la
materia\cite{iab:est_enfemeria}.

La escala es del $60\%$ para asignaturas no profesionales y del $75\%$ para
materias profesionales, la escala que se utiliza es del $1$ al $5$.

El objetivo final de estas evaluaciones es el de evaluar si un alumno comprende
y tiene la pericia necesaria en todas las competencias básicas de dicha
asignatura.

%%! TEX root = ../main.tex
\section{Problemas actuales}
\label{sec:problemas_actuales}

Sí bien el nivel actual de los egresados del \Gls{iab} en la carrera
Licenciatura en Enfermería es satisfactorio, existen ciertos inconvenientes, los
mismos son recabados de distintas fuentes, como tesis de
alumnos\cite{iab:tesis_alumnos}, comentarios de los profesores y opiniones de
alumnos egresados.

La información no bibliográfica presentada, es fruto de reuniones con
profesores, encargados y directores de la carrera de enfermería, las mismas
deben ser tomadas como experiencias de los mismos, \textbf{no representan un
    compendio de todos los problemas del \Gls{iab}}.

La población del \Gls{iab} esta compuesta en su gran mayoría por personas de
niveles económicos medio-bajos y un gran porcentaje de los alumnos son del
interior del país.

Desde el factor humano, los alumnos tienen varios inconvenientes para asistir a
clases, la mayoría reporto en~\cite{iab:tesis_alumnos} que los principales
inconvenientes son:

\begin{itemize}
    \item \textbf{Carga horaria de trabajos prácticos}, se refiere al tiempo necesario
        por los estudiantes para llevar a cabo un trabajo práctico. 
    \item \textbf{Carga horaria de materias teóricas}, se refiere al tiempo que consumen
        las materias teóricas, cuyo tiempo de estudio es reducido por la
        necesidad de acudir a prácticas de campo en horarios variados.
    \item \textbf{Poca flexibilidad de docentes}, los docentes no siempre toman en cuenta
        las dificultades que debe pasar un alumno para poder acudir a clase, o a
        una práctica de laboratorio.
    \item \textbf{Falta de materiales en los docentes}, la enfermería es un campo muy
        dinámico, los materias se vuelven obsoletos rápidamente, y los docentes
        no cuentan con una fuente actualizada de información.
    \item \textbf{Problemas de transporte}, la ubicación del \Gls{iab} facilita el acceso al
        mismo desde rutas internacionales, pero no se puede decir lo mismo de
        los hospitales donde se realizan prácticas profesionales. Este problema
        es acentuado por la gran cantidad de tiempo que deben pasar los alumnos
        en los medios de transporte para moverse desde sus respectivos hogares
        hasta el \Gls{iab} o a los campos de práctica.
\end{itemize}

Además de estos inconvenientes, existen otros problemas, relacionados al ámbito
familiar de los estudiantes\cite{iab:tesis_alumnos}:

\begin{itemize}
    \item \textbf{Ingreso familiar económico bajo}, la mayoría de las familias que
        soportan a alumnos son de escasos recursos económicos, lo que se acentúa
        cuando deben mantener a un estudiante que requiere constantemente de
        materiales y transporte.
    \item \textbf{Alimentación de baja calidad en la cantina} del \Gls{iab}, como
        consecuencia directa de lo anterior, las cantinas disponibles para los
        alumnos ofrecen menúes de baja calidad con la ventaja de que tienen un
        coste bajo.
    \item \textbf{Problemas familiares}, los alumnos reportan que tienen varios
        problemas familiares, muchas veces acentuados por el ingreso económico
        bajo.
\end{itemize}


Estos problemas son muchas veces causados por que los alumnos no pueden trabajar
durante su formación, lo que acentúa los problemas económicos, pues la familia
debe solventar el periodo académico.

Según~\cite{humphreys2013developing}, la enfermería es una profesión que atrae a
alumnos del tipo divergente\footnote{Son aquellos que aprenden mejor a través de
    la experimentación y de la reflexión acerca de lo experimentado.}, indicando
así que las materias teóricas donde no existe una experimentación activa, por
consiguiente, este tipo de enseñanza es menos productiva que la experimentación
en laboratorios, campos de práctica, etc.

Otros problemas reportados por profesores, es que ciertos profesores de campo
prefieren tener tiempo con los alumnos en el laboratorio, esto se debe
principalmente a:

\begin{itemize}
    \item \textbf{Falta de preparación de los alumnos}, ciertos detalles necesarios para
        la práctica de campo no son completamente cubiertos en el laboratorio.
    \item \textbf{Definición de un protocolo de comunicación}, en la práctica de campo,
        los profesores necesitan comunicarse con sus alumnos de una manera
        rápida y eficiente, entonces los profesores enseñan a sus alumnos
        ciertos códigos que son utilizados para corregir, notificar y enseñar
        durante la práctica.
    \item \textbf{Nerviosismo ante primera práctica}, ciertos alumnos reaccionan de
        manera inesperada la primera vez que deben realizar una práctica, esto
        se debe principalmente a que ciertos procedimientos son impactantes y
        ni el laboratorio ni el aula pueden preparar para este tipo de
        experiencias.
\end{itemize}

\pregunta{Estos son comentarios realizados por profesores, no se si pueden ir
    aquí, o como se deben citar}

%%! TEX root = ../main.tex
\section{Definición de criterios}
\label{sec:definicion_criterios}

Teniendo en cuenta lo expuesto en las secciones anteriores, se realizaron varias
reuniones con la coordinadora de la carrera de Licenciatura en Enfermería del
\Gls{iab}, \textit{Lic. Miguela Hermosilla} y otros profesores encargados de las
prácticas de laboratorio y campo de diversas materias de la carrera.

En estas reuniones se valoraron los procedimientos que pueden ser simulados,
\fixme{los mismos}{Estos procedimientos son} fueron extraídos del plan de
estudio de la carrera, y posteriormente fueron validados y aceptados por los
profesores.

De los procedimientos evaluados y discutidos \fixme{se eligieron}{eligen} dos ya
que la variedad y cantidad de procedimientos de enfermería existentes es muy
alta e incluirlos a la solución está fuera del alcance de este trabajo. Los
principales criterios para la selección de los procedimientos son los
siguientes:

\observacion{No se puede hablar de selección de procedimiento sin hablar de las
    limitaciones tecnológicas. Hay que describir brevemente cual es el prototipo
    de dispositivo e interfaz que se quiere utilizar. Y justificar (este es más
    fácil)}

\begin{itemize}
\item \textbf{El entorno simulado no deben ser muy afectado por las limitaciones
        de la tecnología}\revisar{Reformular}, escenarios que requieren la
    simulación de órganos internos, psicología humana, etc, pueden requerir una
    complejidad adicional a la hora de implementar la simulación, complejidad
    que escapa al alcance del presente trabajo.
\item \textbf{Deben de ser de utilidad para los alumnos}, es decir, deben ser
    procedimientos comunes en la vida profesional de un enfermero. \item
    \textbf{Los procedimientos seleccionados no deben ser complejos ni
        extensos}, facilitando simulaciones cortas, simulaciones \fixme{largas pueden
        tener muchas ramificaciones y se pueden volver complejas}{Yo me
        enfocaría más en el hecho que es difícil que se use si la simulación son
        largas}.
\item \textbf{Debe tener pasos definidos}, si el procedimiento tiene un objetivo claro,
    y un conjunto de pasos previamente definidos y previsibles, se facilita la
    aceptación del mismo por los \fixme{profesores}{Docentes? Unificar}, es
    decir, es más fácil realizar una validación de la simulación.
\item \textbf{Debe ser difíciles de representar con realismo en un aula o
        laboratorio}, para mostrar las ventajas de la tecnología y técnicas
    propuestas, el procedimiento tiene que representar un desafío a las técnicas
    actuales, este desafío puede ser tanto técnico (falta de herramientas, como
    equipos médicos, maniquís, etc) como humanos (falta de pacientes con la
    patología deseada, procedimiento que pone en riesgo la vida del
    paciente)\revisar{No se entiende}.
\end{itemize}



%\section{Selección de escenas}
\label{sec:seleccion_escenas}
\todox{Falta agregar las referencias}
Con los criterios definidos, y como se menciono, se seleccionaron dos procedimientos de enfermería, los cuales serán simuladas en la solución. A continuación se fundamentan tales elecciones y se entra en 
detalle acerca de los procedimientos seleccionados.

\subsection{Extracción de muestras de sangre}

Uno de los procedimientos mas comunes para tomar muestras de sangre es la punción venosa, uno de los 
procedimientos invasivos mas frecuentes en los hospitales, pues ofrece un medio directo de acceso al 
sistema vascular para múltiples procesos diagnósticos y terapéuticos.

Este procedimiento fue considerado como uno de los apropiados para la solución propuesta ya que el mismo
posee pasos bien definidos que deben ser seguidos por el profesional de enfermería. Además la complejidad
del procedimiento no es muy alta y posee pasos muy sensibles que son susceptibles de equivocaciones especialmente en lo que se refiere a la bioseguridad que sera explicado mas adelante. Estas apreciaciones
fueron realizadas por los profesionales del \Gls{iab} en las diversas reuniones. A continuación se
explican las competencias básicas de este procedimiento con respecto al plan de estudio de los estudiantes
de enfermería del \Gls{iab}, los criterios según la planilla de practica que poseen los profesores para evaluar al alumno y el protocolo para llevar a cabo el procedimiento.

\subsubsection{Competencias básicas}
\begin{itemize}
\item Ayudar en procedimientos invasivos.
\end{itemize}

Dentro de los procedimientos invasivos se encuentra el procedimiento de extracción de muestras de sangre.

\subsubsection{Criterios de evaluación de la práctica}
\begin{itemize}
\item Informar al paciente acerca del procedimiento que va a ser realizado.
\item Preparar materiales para la técnica aséptica.
\item Lavado de manos.
\item Vestirse con chaleco estéril, tapaboca, gorro y calzarse guantes.
\item Realizar campo, dejar secar.
\item Punzar, extraer 10mm de sangre, comprimir zona de punción.
\item Cambiar la aguja.
\item Introducir la muestra en el frasco correspondiente.
\item Retirar los materiales y equipo de protección personal.
\item Etiquetar y enviar a laboratorio.
\end{itemize}

\subsubsection{Protocolo del procedimiento}

\begin{enumerate}
\item Preparar el equipo, lo que incluye seleccionar la jeringa adecuada.
\item Identificar al paciente, presentarse y explicarle el procedimiento que va a ser realizado.
\item Asepsia de las manos.
\item Llevar el equipo a la unidad en donde se encuentra el paciente.
\item Vestirse con bata estéril, tapaboca y gorro.
\item Calzarse los guantes.
\item Ubicar al paciente en posición adecuada, esto es, el brazo debe estar extendido y lo mas relajado 
posible.
\item Elegir la zona a puncionar, para ello se debe palpar la vena para averiguar sus características.
\item Colocar el torniquete, 6 a 10 centímetros por encima de la zona de punción.
\item Solicitar al paciente que cierre el puño.
\item Esterilizar la zona de punción.
\item Extraer el protector de la aguja.
\item Tensar la zona de punción.
\item Puncionar la piel con la aguja hacia arriba. La aguja se introduce con un ángulo de 10 a 20 grados.
\item Remover el torniquete.
\item Solicitar la apertura del puño.
\item Extraer la muestra se sangre necesaria.
\item Presionar y extraer la aguja.
\item Colocar algodón con alcohol en el punto de punción.
\item Sellar la muestra y enviarlo a su destinatario.
\item Retirar el equipo utilizado, incluyendo bata, tapaboca, gorro y guantes.
\item Asepsia de las manos.
\end{enumerate}


\subsection{Valoración de la escala de Glasgow}

La escala de Glasgow es una escala utilizada como una herramienta de valoración objetiva del estado de conciencia para las víctimas de traumatismo craneoencefálico. La escala esta compuesta por la exploración
y cuantificación de tres parámetros: la apertura ocular, la respuesta verbal y la respuesta motora. Dando 
un puntaje a la mejor respuesta obtenida en cada ítem. El puntaje obtenido para cada uno de los
tres se suma, con lo que se obtiene el puntaje final.

Este procedimiento fue considerado como uno de los apropiados para la solución propuesta ya que el mismo
posee pasos bien definidos que deben ser seguidos por el profesional de enfermería, no se presentan
muchos casos como estos en la realidad durante las practicas de campo realizados por los estudiantes,
el factor de aleatoriedad para cada uno de las respuestas puede ser potenciado altamente con la solución,
además el procedimiento no es complejo. A continuación se explican las competencias básicas de este
procedimiento con respecto al plan de estudios de los estudiantes de enfermería del \Gls{iab}, los criterios
según la planilla de practica que poseen los profesores para evaluar al alumno y el protocolo para llevar a 
cabo el procedimiento.

\subsubsection{Competencias básicas}
\begin{itemize}
\item Identificar actividades de cuidados según problemas urgentes principales.
\end{itemize}

\subsubsection{Criterios de evaluación de la practica}
\begin{itemize}
\item Cita las actividades que debe realizar según la necesidad de urgencia. \begin{enumerate*}
\item Control de signos vitales.
\item Inspección cefalocaudal - escala de Glasgow.
\item Preparación del equipo según prioridad del problema
\item Analizar su participación en las actividades.
\item Fundamenta científicamente sus decisiones.
\end{enumerate*}
\end{itemize}


\subsubsection{Protocolo de práctica}

\begin{enumerate}
\item Preparación del material (Escala de Glasgow)
\item Preparación del paciente: comprobar su identidad, mantener una ambiente tranquilo evitando interrupciones, 
requerir la atención del paciente.
\item Colocar al paciente en posición cómoda.
\item Medir la apertura ocular, respuesta motora y respuesta verbal.
\item Registrar la puntuación final obtenida.
\end{enumerate}

Como se comento anteriormente, la escala de coma de Glasgow incluye tres parámetros: la apertura ocular, la respuesta motora y la respuesta verbal. Cada parámetro cuenta con ítems que poseen una puntuación como se muestra en la tabla
\ref{tab:seleccion_glasgow_respuestas}. 

\begin{table}[!hbt]
\centering
\begin{tabular}{lr}
%\toprule
\textbf{Apertura ocular} & \textbf{Valor} \\
Espontánea & 4 \\
Al hablar & 3 \\
Al dolor & 2 \\
Ausente & 1 \\
%\midrule
\textbf{Respuesta motora} & \textbf{Valor} \\
Obedece & 6 \\
Localiza & 5 \\
Retira & 4 \\
Flexión anormal & 3 \\
Extiende & 2 \\
Ausente & 1 \\
%\midrule
\textbf{Respuesta verbal} & \textbf{Valor} \\
Orientada & 5 \\
Confusa & 4 \\
Palabras inapropiadas & 3 \\
Palabras incomprensibles & 2 \\
Ausente & 1 \\
%\bottomrule
\end{tabular}
\caption{Valoración de las distintas respuestas en la escala de Glasgow}
\label{tab:seleccion_glasgow_respuestas}
\end{table}

Cada uno de los parámetros se evalúa mediante una sub-escala. Cada respuesta se puntúa con un número, siendo cada
una de las sub-escalas independientes. El estado de conciencia del paciente se determina sumando la puntuaciones
de los tres parámetros. Una vez obtenido la puntuación final, este se valora según lo indica la tabla \ref{tab:seleccion_glasgow_estado}.

\begin{table}[!hbt]
\centering
\begin{tabular}{llr}
\toprule
\textbf{Nivel ECG} & 
\textbf{Condición clínica} & 
\textbf{Puntuación} \\ 
\midrule
 ECG 1 & Muerte & 3 \\
 ECG 2 & Estado vegetativo & 4 a 6 \\
 ECG 3 & Discapacidad severa & 7 a 9 \\
 ECG 4 & Discapacidad moderada & 10 a 11 \\
 ECG 5 & Recuperacion total & 12 a 15 \\
\bottomrule
\end{tabular}
\caption{Escala de valoración del estado del paciente}
\label{tab:seleccion_glasgow_estado}
\end{table}

\subsection{Bioseguridad}

Además de los procedimientos de enfermería seleccionados y descriptos, existe un tema mas abordado implícitamente
en el procedimiento de extracción de muestras de sangre, la bioseguridad. La bioseguridad incluye procesos como el
lavado de manos, el uso de elementos de protección personal, entre otros.

Los profesores del \Gls{iab} mencionaron que la bioseguridad es un tema transversal a todos lo demás y que 
representa un factor vulnerable por ser susceptibles de olvido o equivocaciones.

En la solución no se abordara el tema como un procedimiento principal, sino solo se buscara que los jugadores
puedan recordar sin ninguna ayuda cuando y en que orden dentro del procedimiento que este realizando
debe llevar a cabo actividades de bioseguridad.
%\section{Requisitos de la solución}
\label{sec:problema_requisitos}

A fin de que la solución propuesta pueda cumplir con las competencias básicas
definidas por el plan de estudio, se definen varios tipos de requisitos.

Las condiciones que afectan a la solución como una herramienta, son:

\begin{itemize}
\item Se debe permitir al jugador poder utilizar el entorno virtual en el
    momento y lugar que desee es decir, es decir, se le debe proveer ubicuidad.
\item Las escenas presentadas, junto con lo elementos dentro de ellas deben ser
    representados en tres dimensiones y deben ser lo mas realista posible.
\item Cada escena representará un procedimiento de enfermería que debe ser
    realizado por el jugador.
\item El entorno debe permitir al jugador decidir libremente las acciones que
    quiere realizar, favoreciendo la exploración del entorno.
\item El entorno no debe brindar pistas al jugador acerca de la forma en la que
    se deben realizar las procedimientos.
\item El entorno debe brindar al jugador información final acerca de su
    desempeño en la escena.
\item La vision del jugador deberá poder se manipulada en tres dimensiones,
    permitiendo acercar, alejar, mover y rotar la vision para poder observar el
    entorno sin limitaciones.
\item El final de una partida debe indicarse mediante un botón en un menu
    ubicado en la pantalla principal de la escena.
\end{itemize}

Con respecto a la interacción del jugador con los diferentes elementos que
pueden ser utilizados dentro de la simulación, denominados objetos, son:

\begin{itemize}
\item La selección de objetos debe ser homogénea, y los mismos pueden ser
    accedidos en cualquier momento, representando la mesa de elementos que
    poseen los enfermeros durante la práctica profesional.
\item Debe ser claro que objeto actualmente esta en uso.
\item Debe ser posible simular la selección y des-selección de objetos.
\item Cada objeto disponible durante la simulación debe ser utilizado de
    manera independiente de los demás, y solamente un objeto a la vez.
\end{itemize}

La interacción entre el jugador y el paciente se toman los siguientes
requisitos:

\begin{itemize}
\item El jugador puede manipular el estado del paciente a través de objetos, la
    utilización de los mismos debe ser uniforme e intuitiva.
\item Las acciones sobre los objetos no deben ser muy detalladas, no deben
    distraer al jugador del objetivo principal de la simulación.
\item Todas las acciones sobre objetos que no son relevantes para el objetivo de
    la simulación, no deben ser simuladas, pero sí deben ser realizadas (por
    ejemplo puede existir una opción que indique la realización de cierta
    acción, pero la acción en sí no se simula).
\end{itemize}

En cuanto a las acciones que realiza el jugador mientras utiliza la simulación,
se consideran los siguientes requisitos:

\begin{itemize}
\item Las acciones realizadas por los jugadores en el entorno virtual deben ser
    registradas.
\item Cada acción realizada por el usuario debe ser validada por el entorno, de
    forma a ofrecer información correcta acerca de sus logros al final de la
    partida.
\item La validez de una acción puede requerir en algunos casos que se hayan
    hecho algunas acciones previas, pueden requerir un orden definido, o pueden
    depender del entorno en el cual se realizan las mismas.
\item No se deben dar pistas o mensajes al jugador durante la partida cuando
    este realice incorrectamente una acción.
\end{itemize}



% TICS
%       Diciendo que los juegos serios son una buena forma de utilizar las TICS
% Juegos Serios
%       Diciendo que la enfermería es un campo propicio para la utilización de
%       los juegos serios
% Problema
%   Estado actual
%       Plan de estudio
%       Practicas lab
%       Practicas hos
%       Evaluación
%       Problemas
%   Propuesta de solución           -- Descripción del problema a encarar
%      Motivación
%       Diciendo que se definiio en manera global la solución y que faltna más detalles
% Requisitos de la solución
%   Hipótesis
%   --Alcance
%       Diciendo que se debe definir como se va a encarar la propuesta
% Tecnología
%       Diciendo que falta describir como se uso esa tecnología
% Solución
%       Diciendo que hace falta evaluar la solución
% Evaluación
%       Diciendo que hace falta analizar los datos
% Análisis
% Conclusión
