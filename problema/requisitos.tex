\section{Requisitos de la solución}
\label{sec:problema_requisitos}

La solución propuesta debe cumplir con las condiciones que se citan a continuación:

\subsection{Generales}
\begin{itemize}
\item Se debe utilizar la plataforma Unity3D como herramienta para la creación del entorno.
\item Se debe permitir al jugador poder utilizar el entorno virtual en el momento y lugar que desee es decir, 
se le debe proveer ubicuidad.
\item Las escenas presentadas, junto con lo elementos dentro de ellas deben ser representados en tres dimensiones y
lo mas realista posible.
\item Cada escena representara un procedimiento de enfermería que debe ser realizado por el jugador.
\item El entorno debe permitir al jugador decidir libremente las acciones que quiere realizar.
\item El entorno no debe brindar pistas al jugador acerca de la forma en la que se deben realizar las procedimientos.
\item El entorno debe brindar al jugador información final acerca de su desempeño en la escena.
\item La selección de una escena debe ser realizada mediante un menú ubicado en la pantalla principal o de bienvenida.
\item Se debe permitir que el movimiento de la cámara sea manipulado por el jugador para rotación y aumento de tamaño de la escena.
\item El final de una partida debe indicarse mediante un botón en un menu ubicado en la pantalla principal de la escena.
\end{itemize}

\subsection{Interacción del jugador con los objetos}
\begin{itemize}
\item La selección de objetos se debe realizar mediante un menú ubicado en la pantalla principal de la escena.
\item El objeto actualmente seleccionado se representara mediante un imagen en la escena.
\item Los objetos utilizados como instrumentos necesarios para la realización de un procedimiento deben
utilizarse solo uno a la vez.
\item Un objeto seleccionado debe poder des-seleccionarse volviendo a presionar el botón que le representa en el 
menú.
\end{itemize}

\subsection{Interacción entre objetos}
\begin{itemize}
\item Para realizar acciones sobre el paciente haciendo uso del objeto seleccionado, se debe manipular el objeto
directamente.
\item Para ubicar el objeto seleccionado en la escena se debe tocar sobre el paciente en el lugar donde se desea
que aparezca.
\item Los objetos tendrán menú contextuales para aquellas acciones mas complejas de realizar.
\end{itemize}

\subsection{Acciones del jugador}
\begin{itemize}
\item Las acciones realizadas por los jugadores en el entorno virtual deben ser registradas.
\item Cada acción realizada por el usuario debe ser validada por el entorno, de forma a ofrecer información
correcta acerca de sus logros al final de la partida.
\item La validez de una acción puede requerir en algunos casos que se hayan hecho algunas acciones previas e incluso con cierto orden.
\item No se deben dar pistas o mensajes al jugador durante la partida cuando este realice incorrectamente una acción.
\item Se debe contar con botones para las acciones que modifican el estado del jugador.
\end{itemize}
