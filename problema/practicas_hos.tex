\section{Prácticas de enfermería}
\label{sec:practica_hos}

Se definen las prácticas profesionales como\cite{iab:est_enfemeria}:

\begin{displayquote}

Son prácticas de enfermería aquellas actividades de integración y de aplicación
de los conocimientos teóricos adquiridos y las destrezas y habilidades,
acercando al estudiante a una realidad concreta que le permita una experiencia
vivencial. Ellas pueden ser desarrolladas en arte de enfermería, así como en los
servicios de salud de diferentes niveles de complejidad, instituciones públicas
y privadas, establecimientos, hogares y comunidad.

\end{displayquote}

Las practicas de campo son aquellas prácticas que son realizadas por los alumnos
con pacientes humanos y en hospitales, bajo supervisión de un profesional y bajo
una continua evaluación de sus acciones, las mismas son llevadas a cabo una vez
que los alumnos finalizan las prácticas de laboratorio.

Los alumnos del \Gls{iab} participan en prácticas de campo en diferentes
hospitales dependiendo de las necesidades de cada materia, por ejemplo, los
alumnos de \textit{Enfermería en Urgencias} realizan sus prácticas en el
\textit{Centro de emergencias médicas}, otros hospitales utilizados, son el
\textit{Hospital de Clínicas}, y diversos hospitales del \textit{Instituto de
    previsión social}.

Para controlar y medir la evolución de los estudiantes existe un grupo de
profesores cuya función es guiar a los alumnos durante las prácticas de campo,
este grupo de profesores son denominados \textbf{instructores}.

Las prácticas se realizan en grupos que varían de 4 a 10 alumnos, dependiendo de
la disponibilidad de instructores y de si el área es crítica o no\footnote{Se
    dice que un paciente esta en estado crítico si su vida depende de un
    procedimiento externo, como una transfusión de sangre.}, un instructor puede
manejar más de un grupo en diferentes horarios. 

Cada instructor posee un planilla por alumno donde se realiza el seguimiento de
las actividades de cada alumno. La creación de esta planilla de actividades es
responsabilidad del instructor, el instructor debe basarse en las competencias
básicas de la asignatura y la misma es validada por la dirección de la carrera,
y se considera que un alumno ha adquirido la pericia necesaria para una
asignatura solo sí pudo completar la planilla del instructor de dicha
asignatura. Son registradas todas las actividades del alumno, pero solo cuentan
para el progreso final aquellas que son realizadas con la pericia necesaria.

La cantidad de alumnos, hace que la práctica de una asignatura rara vez se
realice en un solo hospital, para asignaturas críticas, debe haber
aproximadamente 35 grupos de estudiantes.
