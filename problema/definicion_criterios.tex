\section{Definición de criterios}
\label{sec:definicion_criterios}

Teniendo en cuenta lo expuesto en las secciones anteriores, se realizaron varias
reuniones con la coordinadora de la carrera de Licenciatura en Enfermería del
\Gls{iab}, \textit{Lic. Miguela Hermosilla} y otros profesores encargados de las
prácticas de laboratorio y campo de diversas materias de la carrera.

En estas reuniones se valoraron varios procedimientos que pueden ser simulados,
los mismos fueron extraídos del plan de estudio de la carrera, y posteriormente
fueron validados y aceptados por los profesores.

De los procedimientos evaluados y discutidos se eligieron dos ya que la variedad
y cantidad de procedimientos de enfermería existentes es muy alta e incluirlos
en la solución está fuera del alcance de este trabajo. Los principales criterios
para la selección de los procedimientos son los siguientes:

\begin{itemize}
\item \textbf{El entorno simulado no deben ser muy afectado por las limitaciones
        de la tecnología}, escenarios que requieren la simulación de órganos
    internos, psicología humana, etc, pueden requerir una complejidad adicional
    a la hora de implementar la simulación, complejidad que escapa al alcance
    del presente trabajo.
\item \textbf{Deben de ser de utilidad para los alumnos}, es decir, deben ser
    procedimientos comunes en la vida profesional de un enfermero. \item
    \textbf{Los procedimientos seleccionados no deben ser complejos ni
        extensos}, facilitando simulaciones cortas, simulaciones largas pueden
    tener muchas ramificaciones y se pueden volver complejas.
\item \textbf{Debe tener pasos definidos}, si el procedimiento tiene un objetivo claro,
    y un conjunto de pasos previamente definidos y previsibles, se facilita la
    aceptación del mismo por los profesores, es decir, es más fácil realizar una
    validación de la simulación.
\item \textbf{Debe ser difíciles de representar con realismo en un aula o
        laboratorio}, para mostrar las ventajas de la tecnología y técnicas
    propuestas, el procedimiento tiene que representar un desafío a las técnicas
    actuales, este desafío puede ser tanto técnico (falta de herramientas, como
    equipos médicos, maniquís, etc) como humanos (falta de pacientes con la
    patología deseada, procedimiento que pone en riesgo la vida del paciente).
\end{itemize}


