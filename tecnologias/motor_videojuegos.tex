\section{Motor de videojuego}
\label{sec:motores}

El término \enquote{motor de videojuego} hace referencia a un \textit{framework}
 que permite el diseño, la creación, el desarrollo y la
representación gráfica de un videojuego\cite{videojuego:telechea}.

La mayoría de los motores ofrecen características y
funciones que facilitan la construcción del videojuego, como un motor de física
\footnote{Un motor de física es un software capaz de realizar \enquote{simulaciones} de ciertos sistemas físicos
como la dinámica de un cuerpo rígido, el movimiento de un fluido o la
elasticidad\cite{videojuego:telechea}.}, detector de colisiones, animaciones,
inteligencia artificial, comunicación a través de redes,
administración de memoria, etc\cite{videojuego:telechea}.

El motor de videojuego a utilizar depende de las características que posea el
videojuego que se quiere desarrollar, las cuales
fueron descritas en el capítulo \ref{chap:requerimientos}. A continuación se da una breve
descripción de los motores de videojuegos más utilizados actualmente, se definen
los criterios de selección y se realiza una comparación entre los mismos, para
la elección del motor de videojuego que más se adecue a las necesidades de la
solución. Entre los aspectos que son comparados se encuentran la distribución,
librerías, tiendas, licencias, curva de aprendizaje, lenguajes de programación,
entre otros.

\subsection{Unreal Development Kit}

Es la edición gratuita de \textit{Unreal Engine 3}. \textit{Unreal Engine} es el
motor de videojuego desarrollado por \textit{Epic Games}\cite{unrealengine}.

\Gls{udk} proporciona acceso al motor de juegos 3D y a la herramienta
profesional que se utiliza en el desarrollo de videojuegos, visualización arquitectónica, el desarrollo de juegos
para móviles, modelos 3D, películas digitales y más. Utilizando \Gls{udk} se
pueden implementar juegos y aplicaciones en \textit{Windows}, \textit{iOS} y
\textit{Mac}\cite{unrealengine}.

Posee su propio entorno de desarrollo, las rutinas de programación pueden ser escritas 
en los lenguajes de programación Unreal Script y C++, y una comunidad 
donde se puede recurrir. Además soporta formatos de modelos 
3D como fbx, dds, raw y ASE\cite{unrealengine}.

\subsection{Blender Game Engine}

\enquote{Blender Game Engine} es el motor de videojuego de \textit{Blender Foundation} 
que permite crear aplicaciones 3D interactivas o simulaciones, desarrollado bajo 
la licencia \Gls{gnu}\cite{blender}.

\textit{Blender Game Engine} genera las escenas de forma continua en tiempo real
e incorpora facilidades para la interacción del usuario durante el proceso de
\textit{renderizado},\footnote{Proceso que genera imágenes a partir de modelos 3D.} procesa la lógica de sonido, de la física y la 
representación de simulaciones en orden secuencial\cite{blender}.

Posee la posibilidad de exportar en plataformas como
\textit{Windows, Linux y Mac OS}. También incluye básico en desarrollo para 
plataformas Android\cite{blender}.

Posee su propio entorno de desarrollo, las rutinas de programación pueden ser 
escritas en los lenguajes de programación Python y C++, y
una comunidad grande donde se puede recurrir. Además soporta formatos 3D 
como 3ds, dae, fbx, dxf\cite{blender}.

\subsection{CryEngine}

\enquote{CryEngine} es el motor de videojuego desarrollado por \textit{Crytek}.
Existe una versión gratuita denominada \enquote{CryEngine Free SDK} con todas
las funcionalidades, esta versión esta disponible para su descarga, pero ya ha
sido descontinuada\cite{cryengine:sdk}.

Este motor de videojuego permite exportar a plataformas como iOS y
Android\cite{cryengine}. Posee su propio entorno de desarrollo, las rutinas de
programación pueden ser escritas en los lenguajes de programación C++ y Lua, y
una comunidad de tamaño moderado donde se puede recurrir. Además sólo soporta
sus propios formatos 3D\cite{cryengine}.

\subsection{ShiVa3D}

\textit{ShiVa3D} es un motor para el desarrollo de videojuegos y aplicaciones 
3D desarrollado por ShiVa Technologies\cite{shiva}.

Un producto relacionado es el \enquote{ShiVa Server}, el cual permite el
desarrollo de aplicaciones multijugador. Las características de este servidor,
incluyen comunicación \textit{VoIp}, etc\cite{shiva}, esta es
una opción interesante para el desarrollo del \textit{back-end}.

\textit{ShiVa} puede exportar juegos y aplicaciones a una cantidad variada de 
plataformas, incluyendo móviles como \textit{iOS, Android, BlackBerry y Windows
Phone}, escritorios como \textit{Windows, Mac OS X y Linux}, los
navegadores web con soporte \textit{Flash y HTML5}, así como consolas que incluyen la
\textit{Xbox 360, PlayStation3 y Nintendo Wii}. El \Gls{ide} se ejecuta en
\textit{Windows} y \textit{Mac OS X}\cite{shiva}. 

Posee su propio entorno de desarrollo, las rutinas de programación pueden ser 
escritas en los lenguajes de programación FlowGraph y Lua, y una 
comunidad moderada donde se puede recurrir. Además sólo soporta el formato 
3D, dae\cite{shiva}.


\subsection{Unity3D}

\textit{Unity} es un motor para el desarrollo de juegos, desarrollado por
\textit{Unity Technologies}. Posee una versión paga y una
gratuita\cite{unity3d}.

Posee un motor de \textit{renderizado} y un flujo de trabajo para la creación de
contenido 3D interactivo, además permite mezclar contenido 3D, 2D, sonidos y
animaciones de manera sencilla\cite{unity3d}.

La versión gratuita permite desarrollar juegos para múltiples plataformas. 
Entre las plataformas móviles soportadas, encontramos a \textit{iOS, Andriod, 
Windows Phone 8, BlackBerry 10}, entre las plataformas de escritorio a 
\textit{Windows, Mac y Linux}, plataformas web como \textit{Internet Explorer,
    Mozzilla Firefox, Google Chrome}\footnote{Requiere un \textit{plugin} para
    el navegador que está disponible para \textit{Windows y Mac}.}, y entre
plataformas de consolas a \textit{Xbox 360, Xbox One, Wii, Wii U, Nintendo
    3DS}\cite{unity3d}.

Posee su propio entorno de desarrollo, las rutinas de programación pueden ser 
escritas en los lenguajes de programación \cs{}, UnityScript y Boo, y una 
comunidad grande donde recurrir. Además soporta formatos 3D como fbx, obj, max, 
blend, dae, 3ds, dxf, MB, MA\cite{unity3d}.

