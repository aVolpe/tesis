\section{Entorno de desarrollo de la solución}

Este proceso se compone de varios pasos necesarios, para llevar a cabo el
desarrollo completo se debe recurrir a una gran cantidad de tecnologías,
herramientas, \textit{frameworks}, y recursos.

En esta sección se describen todos las partes involucradas en el desarrollo de
la solución propuesta. 

La solución se compone de dos partes, la primera es la aplicación que los
usuarios utilizan para realizar las prácticas, denominada \textit{Frontend}, y la segunda
parte, es un servidor que se encarga de almacenar la información sobre los
usuarios de la solución y como la utilizan, denominado \textit{Backend}.

Primeramente se definen las herramientas de gestión de código, pues ellas son
utilizadas tanto por la solución como por el \textit{backend}, luego se definen
las herramientas específicas de la creación de la simulación y por último, se
citan y describen de manera breve las herramientas utilizadas por el
\textit{backend}.

La idea detrás de separar la solución en \textit{Frontend} y \textit{Backend} 
es que ambas partes de la solución se podrán mantener de forma separada sin que 
el cambio de una afecte el funcionamiento de la otra. 

\subsection{Herramientas de gestión de código}

La gestión del código fuente desarrollado como parte de esta tesis fue realizado
mediante la utilización de la herramienta de control de código fuente
\textit{Git}, \textit{Git} es un software de control de versiones distribuido,
de código abierto bajo la licencia \Gls{gnu}\cite{git}. El proveedor del
servicio \textit{Git} utilizado es \textit{BitBucket}\cite{bitbucket}, el cual
almacena repositorios \textit{Git}, la principal característica y motivación por
la cual se utiliza este servicio es que el mismo permite mantener varios
repositorios privados de manera gratuita\cite{bitbucket}.

\subsection{Desarrollo del \textit{frontend}}

El desarrollo de la solución requiere de una variedad importante de tecnologías
aparte del motor de videojuegos, las herramientas descritas en esta sección complementan al
motor seleccionado y facilitan la creación de contenido.

Se utilizó el \Gls{ide} \textit{Unity Editor} para la creación de las escenas,
el \Gls{ide} \textit{MonoDevelop} y el lenguaje \cs{} para la programación de la
interacción entre los componentes de la solución.

Adicionalmente se utilizaron varias herramientas de diseño para crear
componentes 3D y 2D, \textit{Make Human} es utilizado para la creación de los
pacientes, \textit{3Ds Max} permite la creación de objetos como gazas, y otros
elementos utilizados dentro de la solución. En cuanto a los gráficos 2D, se
utilizaron \textit{Photoshop}, y diversas páginas web que proveían contenido
gratuito.

\subsubsection{Unity Editor}

El \Gls{ide} de \textit{Unity} es la herramienta en la se crean los videojuegos
o simulaciones. Este \Gls{ide} importa todos los \textit{asset}\footnote{Un
    \textit{Asset} es un paquete \textit{Unity} que puede contener modelos,
    librerias, sonidos, etc.}. Permite compilar las escenas con los terrenos,
luces, audios, personajes, física, entre otros. Se puede agregar interacción a
través de \textit{scripting}. En este \Gls{ide} se puede probar y editar en
forma simultánea los videojuegos y desplegarlos en las plataformas
elegidas\cite{unity3d}. 

Este editor es la única herramienta que permite crear escenas en \textit{Unity},
sin embargo, este \Gls{ide}, no permite la edición de código\footnote{Existen
    \textit{Assets} que permiten la edición de código dentro del \textit{Unity
        Editor}, pero son pagas y de baja reputación}, por ello se necesita un
editor externo de código.


\subsubsection{MonoDevelop}

\textit{MonoDevelop} es un \Gls{ide} de código abierto, bajo la licencia
\Gls{gnu} apoyado principalmente por la comunidad \textit{Mono}. Es el \Gls{ide}
utilizado por defecto en el desarrollo de aplicaciones para \textit{Unity3D}, el
mismo soporta varios lenguajes de programación, como \cs{},
\textit{UnityScript}, y \textit{Boo}. Es un \Gls{ide} multiplataforma que
soporta \textit{Windows} al igual que \textit{Unity3D}.

Existen otros editores que pueden ser utilizados para el desarrollo del código
fuente, pero los mismos no cuentan con el mismo nivel de integración y no son
gratuítos\footnote{\textit{Microsoft Visual Studio} permite el mismo nivel de
    integración que \textit{MonoDevelop} con un \textit{plugin}, pero el mismo
    era de pago durante el desarrollo de la solución}.

\subsubsection{Lenguaje de programación}

\textit{Unity3D} utiliza versiones limitadas\footnote{La definición del lenguaje
    es la misma, pero las librerías estándar no están completas} de tres
lenguajes de programación: \cs{}, \textit{UnityScript}, y
\textit{Boo}\cite{unity:script}. Estos lenguajes son compilados y orientados a
objetos.

Otra característica interesante es que, por el orden de compilación de los
proyectos \textit{Unity3D}, los archivos \textit{UnityScript} y \textit{Boo} son
compilados antes que los \cs{}, esto provoca que, las clases
\textit{UnityScript} sean utilizables desde \cs{}, lo que no se cumple en el
caso contrario, es decir, las clases \cs{} no son accesibles desde código
\textit{UnityScript} o \textit{Boo}.

Por los criterios mencionados se selecciona a \cs{} como el lenguaje de
implementación, por ser el lenguaje con más ayuda en línea, y por que las
librerías no diseñadas específicamente para \textit{Unity3d} pueden ser
utilizadas. Otro factor que influye en la elección es la familiaridad de los
autores con lenguajes similares. 

\subsubsection{Herramientas de diseño}

Las herramientas utilizadas para crear modelos 3D son las siguientes:

\begin{itemize}
\item \textbf{MakeHuman}: es un software de código abierto bajo la licencia
    \Gls{agnu} para crear personajes humanos 3D. Es una herramienta diseñada
    para simplificar la creación de seres humanos virtuales utilizando una
    interfaz gráfica de usuario\cite{makehuman}. 
\item \textbf{3DS Max}: es un software privado de modelado 3D, que además posee
    herramientas para animación, simulación y renderización. Esta herramienta
    fue utilizada para crear objetos 3D que no fueran personajes humanos y para
    exportar modelos de un formato a otro que fuera compatible con
    \textit{Unity3D}\cite{3dsmax}.
\item \textbf{Photoshop}: es una herramienta de edición de gráficos 2D de
    \textit{Adobe}, permite la creación y edición de gráficos, es utilizada para
    la creación de iconos, botones y demás contenido 2D que forma parte de la
    solución.
\end{itemize}

\subsection{Desarrollo del \textit{backend}}


A fin de registrar las actividades del usuario, se necesita de un servidor que
almacene los datos de todos los usuarios y las actividades que estos realizan
dentro de la solución.

\begin{figure}[ht]
\centering
\includegraphics[scale=0.5]{tecnologias/images/backend_diagrama.png}
\caption{Diagrama de la interacción de los usuarios con el \textit{BackEnd}, se
    puede observar a grandes rasgos, los componentes del sistema y los servicios
    que ofrece.}
\label{fig:backend_diagrama}
\end{figure}


Se puede observar en lineas generales como funciona este servicio
en~\ref{fig:backend_diagrama}, desde el registro de actividades del usuario, a
la comunicación con el \textit{backend}, su posterior traducción y persistencia
en una base datos, así, este servicio debe proveer:

\begin{itemize}
    \item \textbf{Alta disponibilidad}: el servidor debe estar disponible en
        todo momento, cualquier día de la semana y a cualquier hora. Los
        requisitos de accesibilidad son estrictos, pues se necesita que los
        usuarios envíen datos sin inconvenientes cuando crean necesario.
    \item \textbf{Accesibilidad}: el servidor debe poder ser accesible desde
        cualquier red móvil.
    \item \textbf{Bajo costo de comunicación}: la comunicación del usuario con
        el \textit{backend} debe ser lo menos costosa posible, pues se utilizan
        recursos del usuario.
\end{itemize}

Para el desarrollo de la aplicación web que almacena los datos se utiliza
\Gls{javaee} en su versión $6$, la misma se utiliza por la familiarización de
los autores con la tecnología, y la facilidad que provee la misma para la
realización de servicios web que permitan la interacción con la solución.

Para los servicios se utiliza la arquitectura \Gls{rest}, la principal
motivación para utilizar \Gls{rest} es la eficiencia en el uso de la
red\cite{pautasso2008restful}, la cual es también la motivación para la
utilización de \Gls{json}. La implementación del lado del servidor de la
arquitectura \Gls{rest} es \textit{RestEasy}, de parte del \textit{frontend} se utiliza
la implementación por defecto de \textit{Unity3D}.

El almacenamiento permanente de los datos se logra con la utilización de
\textit{PostgreSQL}, el cual es un motor de bases de datos de código abierto
dirigido por una comunidad de desarrolladores llamada \textit{PostgreSQL Global
    Development Group}. La versión elegida es la $9.1$.

\subsubsection{OpenShift}

A fin de obtener las características necesarias, de alta disponibilidad y
accesibilidad, se utiliza la herramienta de plataforma como servicio de
\textit{RedHat} llamada \textit{OpenShift}, la cual es un producto de código
abierto dirigido por \textit{RedHat}.

Además de dirigir el proyecto, \textit{RedHat} provee un servicio limitado y
gratuito\cite{openshift:pricing}.\footnote{Existen versiones completas del
    producto mantenidas por \textit{RedHat}, las cuales tienen un costo mensual
    y de acuerdo a las funcionalidades utilizadas\cite{openshift:pricing}} Para
esta tesis se utilizo el servicio gratuito con la plataforma \textit{JBoss
    Application Server 7.1} y \textit{PostgreSQL 9.1}.
