\section{Entorno de desarrollo}

\observacion{El entorno de quien}

\observacion{En este punto medio que se necesita entender la solución de una
    manera abstracta primreo. Por que un backend?, por que un frontend? por que
    photoshop?, falta la intro que explique}

Este proceso se compone de varios pasos necesarios, para llevar a cabo el
desarrollo completo se debe recurrir a una gran cantidad de tecnologías,
herramientas, \textit{frameworks}, y recursos.

En esta sección se describen todos las partes involucradas en el desarrollo de
la solución propuesta. 

La solución se compone de dos partes, la primera es la aplicación que los
usuarios utilizan para realizar las prácticas, denominada solución, y la segunda
parte, es un servidor que se encarga de almacenar la información sobre los
usuarios de la solución y como la utilizan, denominado \textit{Backend}.

Primeramente se definen las herramientas de gestión de código, pues ellas son
utilizadas tanto por la solución como por el \textit{backend}, luego se definen
las herramientas específicas de la creación de la simulación y por último, se
citan y describen de manera breve las herramientas utilizadas por el
\textit{backend}.

\subsection{Herramientas de gestión de código}
\observacion{Resumir}

La gestión del código fuente desarrollado como parte de esta tesis se realizo
mediante la utilización de la herramienta de control de código fuente
\textit{Git}, \textit{Git} es un software de control de versiones distrubido, de
código abierto bajo la licencia \Gls{gnu}\cite{git}.

El proveedor del servicio \textit{Git} utilizado es
\textit{BitBucket}\cite{bitbucket}, el cual almacena repositorios \textit{Git},
la principal característica y motivación por la cual se utiliza este servicio es
que el mismo permite mantener varios repositorios privados de manera
gratuita\cite{bitbucket}, de esta manera, se utilizan distintos repositorios,
específicos para cada parte del presente trabajo, un repositorio para los
documentos de las reuniones y la propuesta de tesis, un repositorio que mantiene
el código fuente del libro de la tesis, un repositorio que mantiene el código
fuente, imágenes y diferentes archivos multimedia de la solución, y finalmente
un repositorio que mantiene el código fuente del servidor que almacena los datos
de los usuario.

\subsection{Desarrollo del \textit{frontend}}

El desarrollo de la solución requiere de una variedad importante de tecnologías
aparte del motor, las herramientas descritas en esta sección complementan al
motor seleccionado y facilitan la creación de contenido.

Se utilizo el \Gls{ide} \textit{Unity Editor} para la creación de las escenas,
el \Gls{ide} \textit{MonoDevelop} y el lenguaje \cs{} para la programación de la
interacción entre los componentes de la solución.

Adicionalmente se utilizaron varias herramientas de diseño para crear
componentes 3D y 2D, \textit{Make Human} es utilizado para la creación de los
pacientes, \textit{3Ds Max} permite la creación de objetos como gazas, y otros
elementos utilizados dentro de la solución. En cuanto a los gráficos 2D, se
utilizo \textit{Photoshop}, y diversas páginas web que proveían contenido
gratuito.

\subsubsection{Unity Editor}

\observacion{Explicar esto con un lenguaje natural}

El \Gls{ide} de \textit{Unity} es la herramienta en la se crean \fixme{los
    juegos}{desambiguar} o simulaciones. Este \Gls{ide} importa todos los
\fixme{activos}{?} que se hayan seleccionado y los organiza dentro de él.
Permite compilar las escenas con los terrenos, luces, audio, personajes, física,
entre otros. Se puede agregar interacción a través de \textit{scripting}. En
este \Gls{ide} se puede probar y editar en forma simultánea los juegos y
desplegarlos en las plataformas elegidas\cite{unity3d}. 

Existen cinco vistas en el editor: explorador de proyectos, inspector,
jerarquía, escena y juego. Cada una de estas vistas cuenta con las herramientas
necesarias que permiten el desarrollo del juego\cite{unity3d}.

El editor se utiliza por que es única herramienta que permite crear escenas en
\textit{Unity}, sin embargo, este \Gls{ide}, no permite la edición de código
\footnote{Existen \textit{Assets} que permiten la edición de código dentro del
    \textit{Unity Editor}, pero son pagas y de baja reputación}, por ello se
necesita un editor externo de código.


\subsubsection{MonoDevelop}

\textit{MonoDevelop} es un \Gls{ide} de código abierto, bajo la licencia
\Gls{gnu} apoyado principalmente por la comunidad \textit{Mono}. Es el \Gls{ide}
utilizado por defecto en el desarrollo de aplicaciones para \textit{Unity3D}, el
mismo soporta varios lenguajes de programación, como \cs{},
\textit{UnityScript}, y \textit{Boo}. Es un \Gls{ide} multiplataforma que
soporta \textit{Windows} al igual que \textit{Unity3D}.

Existen otros editores que pueden ser utilizados para el desarrollo del código
fuente, pero los mismos no cuentan con el mismo nivel de integración y no son
gratuítos\footnote{\textit{Microsoft Visual Studio} permite el mismo nivel de
    integración que \textit{MonoDevelop} con un \textit{plugin}, pero el mismo
    era de pago durante el desarrollo de la solución}.

\subsubsection{Lenguaje de programación}

\observacion{Explicar el que se usa nomas.}
\observacion{Buscar algo mejor para \cs{}}

\textit{Unity3D} utiliza versiones limitadas\footnote{La definición del lenguaje
    es la misma, pero las librerías estándar no están completas} de tres
lenguajes de programación: \cs{}, \textit{UnityScript}, y
\textit{Boo}\cite{unity:script}. Estos lenguajes son compilados y orientados a
objetos.

\cs{} es un lenguaje de programación orientado a objetos diseñado por
\textit{Microsoft}, es el lenguaje más utilizado por la comunidad, y el
subconjunto de funcionalidades implementadas en \textit{Unity3D} hace que la
mayoría de las librerías disponibles para \cs{} funcionen correctamente en
\textit{Unity3D}.

\textit{UnityScript} es un lenguaje normalmente confundido con
\textit{JavaScript}, lenguaje del que esta inspirado, una de las principales
diferencias es que \textit{UnityScript} es orientado a objetos, a través de
clases, al contrario que \textit{JavaScript} que es orientado a objetos a través
de prototipos. Otras diferencias menores con \textit{JavaScript}, es que en
\textit{UnityScript}, el nombre del archivo es el nombre de una clase implícita
que engloba el contenido del archivo\cite{us_vs_js}, por ello, las librerías
disponibles para \textit{JavaScript} no son compatibles con
\textit{UnityScript}.

\textit{Boo} es un lenguaje inspirado en \textit{Python}, desarrollado bajo la
licencia \Gls{mit}, la cantidad de librerías portables a \textit{Unity3D}
disponibles es muy limitada y la aceptación de la comunidad hacia el lenguaje es
pobre.

Otra característica interesante es que, por el orden de compilación de los
proyectos \textit{Unity3D}, los archivos \textit{UnityScript} y \textit{Boo} son
compilados antes que los \cs{}, esto provoca que, las clases
\textit{UnityScript} sean utilizables desde \cs{}, lo que no se cumple en el
caso contrario, es decir, las clases \cs{} no son accesibles desde código
\textit{UnityScript} o \textit{Boo}.

Por los criterios mencionados se selecciona a \cs{} como el lenguaje de
implementación, por ser el lenguaje con más ayuda en línea, y por que las
librerías no diseñadas específicamente para \textit{Unity3d} pueden ser
utilizadas. Otro factor que influye en la elección es la familiaridad de los
autores con lenguajes similares. 

\subsubsection{Herramientas de diseño}

Las herramientas utilizadas para crear modelos 3D son las siguientes:

\begin{itemize}
\item \textbf{MakeHuman}: es un software de código abierto bajo la licencia
    \Gls{agnu} para crear personajes humanos 3D. Es una herramienta diseñada
    para simplificar la creación de seres humanos virtuales utilizando una
    interfaz gráfica de usuario\cite{makehuman}. 
\item \textbf{3DS Max}: es un software privado de modelado 3D, que además posee
    herramientas para animación, simulación y renderización. Esta herramienta
    fue utilizada para crear objetos 3D que no fueran personajes humanos y para
    exportar modelos de un formato a otro que fuera compatible con
    \textit{Unity3D}\cite{3dsmax}.
\item \textbf{Photoshop}: es una herramienta de edición de gráficos 2D de
    \textit{Adobe}, permite la creación y edición de gráficos, es utilizada para
    la creación de iconos, botones y demás contenido 2D que forma parte de la
    solución.
\end{itemize}

\subsection{Desarrollo del \textit{backend}}

\observacion{Resumir}

A fin de registrar las actividades del usuario, se necesita de un servidor que
almacene los datos de todos los usuarios y las actividades que estos realizan
dentro de la solución.

Este servicio debe proveer:

\begin{itemize}
    \item \textbf{Alta disponibilidad}: el servidor debe estar disponible en
        todo momento, cualquier día de la semana y a cualquier hora. Los
        requisitos de accesibilidad son estrictos, pues se necesita que los
        usuarios envíen datos sin inconvenientes cuando crean necesario.
    \item \textbf{Accesibilidad}: el servidor debe poder ser accesible desde
        cualquier red móvil.
    \item \textbf{Bajo costo de comunicación}: la comunicación del usuario con
        el \textit{backend} debe ser lo menos costosa posible, pues se utilizan
        recursos del usuario.
\end{itemize}

Para el desarrollo de la aplicación web que almacena los datos se utiliza
\Gls{javaee} en su versión $6$, la misma se utiliza por la familiarización de
los autores con la tecnología, y la facilidad que provee la misma para la
realización de servicios web que permitan la interacción con la solución.

\Gls{javaee} es un estándar de software empresarial de código abierto cuyo
desarrollo es dirigido por la comunidad\cite{javaee}, la implementación
utilizada es de \textit{RedHat} llamada \textit{JBoss} en su versión $7.1$,
\textit{JBoss} es un esfuerzo de la comunidad dirigido por \textit{RedHat}. 

Para los servicios se utiliza la arquitectura \Gls{rest}, la principal
motivación para utilizar \Gls{rest} es la eficiencia en el uso de la
red\cite{pautasso2008restful}, la cual es también la motivación para la
utilización de \Gls{json}. 

La implementación del lado del servidor de la arquitectura \Gls{rest} es
\textit{RestEasy}, \textit{RestEasy} es un proyecto de código abierto dirigido
por \textit{RedHat} y que forma parte de la plataforma \textit{JBoss}, de parte
de la solución se utiliza la implementación por defecto de \textit{Unity3D}.

Otro factor determinante es la facilidad de interacción que existe entre
\textit{Unity3D} y los servicios \Gls{rest}, debido a que \Gls{rest} cumple con
el protocolo HTTP, y en ambos ecosistemas (\textit{JavaEE} y \textit{Unity3d})
existen implementaciones disponibles y de fácil utilización.

Para el desarrollo del servidor web se utilizo el \Gls{ide} \textit{Eclipse}, el
cual es un proyecto de código abierto dirigido por la \textit{Eclipse
    Fundation}\cite{eclipse}, el mismo provee facilidades para el desarrollo de
servicios web \Gls{rest}.

El almacenamiento permanente de los datos se logra con la utilización de
\textit{PostgreSQL}, el cual es un motor de bases de datos de código abierto
dirigido por una comunidad de desarrolladores llamada \textit{PostgreSQL Global
    Development Group}. La versión elegida es la $9.1$.

\subsubsection{OpenShift}

A fin de obtener las características necesarias, de alta disponibilidad y
accesibilidad, se utiliza la herramienta de plataforma como servicio de
\textit{RedHat} llamada \textit{OpenShift}, la cual es un producto de código
abierto dirigido por \textit{RedHat}.

\textit{OpenShift} provee diferentes plataformas, entre las cuales se encuentran
las herramientas seleccionadas \textit{PostgreSQL} 9.1 y \textit{JBoss
    Application Server} 7.1.

Además de dirigir el proyecto, \textit{RedHat} provee un servicio limitado y
gratuito\cite{openshift:pricing}.\footnote{Existen versiones completas del
    producto mantenidas por \textit{RedHat}, las cuales tienen un costo mensual
    y de acuerdo a las funcionalidades utilizadas\cite{openshift:pricing}} Para
esta tesis se utilizo el servicio gratuito con la plataforma \textit{JBoss
    Application Server 7.1} y \textit{PostgreSQL 9.1}.

Otra característica importante de \textit{OpenShift} es la facilidad con la cual
se pueden desplegar nuevas versiones de la aplicación, utiliza un repositorio
\textit{GIT} para mantener el código de la última versión, y cada vez que se
actualiza este repositorio, la aplicación se despliega con la versión
actualizada.

% NOTA DEL TUTOR: no es necesario describir todas las librerías, hay que tratar
% de explicar a más alto nivel al menos y mencionar que se necesita, pero sin
% dedicar un párrafo completo.

\observacion{Vendría bien esa explicación más general de la arquitectura con un
    grafoo por lo menos, de como funciona : frontend -> datos -> nube ->
    backend}
