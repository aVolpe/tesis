\chapter{Tecnologías}
\label{chap:tecnologias}

Luego de definir los requerimientos que debe cumplir la solución propuesta 
se deben seleccionar de manera adecuada las tecnologías que será utilizadas para 
implementarla de tal forma que cumpla con todos los requerimientos.

En este capítulo se describen en detalle estas tecnologías haciendo énfasis especial a 
la selección del motor de videojuegos, el cual es la principal herramienta requerida, por 
lo mismo, se describen los principales motores utilizados en la actualidad y se definen 
criterios para seleccionar de entre ellos el más adecuado.

Entrando un poco más en detalle dentro de la implementación de la solución, se describen 
las herramientas que son necesarias para el desarrollo del juego a más bajo nivel como el 
lenguaje de programación, herramientas para modelado 3D, gestores de proyectos, entre otros. 
Estas tecnologías son agrupadas en tres partes principales:

\begin{itemize}
    \item Herramientas de gestión de código: se describen las tecnologías requeridas 
    para gestionar y mantener proyectos, en este caso la implementación de la solución.
    \item Desarrollo del frontend: se describen aquellas tecnologías utilizadas 
    para el desarrollo del videojuego en sí. El frontend es la parte de la solución 
    que interactúa con el usuario y por lo tanto es lo que él puede percibir.
    \item Desarrollo del backend: se describen aquellas tecnologías utilizadas para 
    procesar las acciones que los usuarios realizan dentro de la solución.
\end{itemize}


\section{Motores de videojuegos}

%\observacion{Falta una descripción de más alto nivel del tipo de tecnología que se
%    necesita}

Para el desarrollo de videojuegos se utilizan programas o herramientas
especializadas en ello llamadas \enquote{Motores de videojuegos}. A continuación
se da una breve introducción de lo que es un motor de videojuego.

El término \enquote{motor de videojuegos} hace referencia a una serie de rutinas
de programación que permiten el diseño, la creación, el desarrollo y la
representación gráfica de un videojuego\cite{videojuego:telechea}.

Además, la gran mayoría de estos motores ofrecen a su vez características y
funciones que facilitan la construcción del videojuego, como el motor físico
(software capaz de realizar \enquote{simulaciones} de ciertos sistemas físicos
como la dinámica de un cuerpo rígido, el movimiento de un fluido o la
elasticidad) o detector de colisiones, sonidos, \textit{scripting}, animaciones,
inteligencia artificial, comunicación a través de redes, \textit{streaming},
administración de memoria, etc\cite{videojuego:telechea}.

El motor de videojuego a utilizar depende de las características que posea el
videojuego que se quiere desarrollar. A continuación
se da una breve descripción de los motores de videojuegos más utilizados actualmente, se
definen los criterios de selección y se realiza una comparación entre los
mismos, para la elección del motor de videojuegos que más se adecue a las necesidades de la 
solución. Entre los aspectos que son comparados se encuentran la distribución, librerías, 
tiendas, licencias, curva de aprendizaje, lenguajes de programación, entre otros.

%\observacion{Comentar que puntos hay en comparación}

\subsection{Unreal Development Kit}

Es la edición gratuita de \textit{Unreal Engine 3}. \textit{Unreal Engine} es el 
motor de videojuegos desarrollado por \textit{Epic Games}, que en su versión pagada se 
ofrece bajo un plan de suscripción mensual\cite{unrealengine}.

\Gls{udk} proporciona acceso al motor de juegos 3D y a la herramienta profesional 
que se utiliza en el desarrollo de videojuegos \textit{blockbuster}, visualización 
arquitectónica, el desarrollo de juegos para móviles, modelos 3D, películas digitales 
y más. Utilizando \Gls{udk} se pueden implementar juegos y aplicaciones en
\textit{Windows PC}, \textit{iOS} y \textit{Mac}\cite{unrealengine}.

Posee su propio entorno de desarrollo, las rutinas de programación pueden ser escritas 
en los lenguajes de programación Unreal Script y C++, y una comunidad 
grande donde se puede recurrir. Además soporta formatos de modelos 
3D como fbx, dds, raw y ASE\cite{unrealengine}.


%\textit{Unreal Engine} es el motor de juegos desarrollado por \textit{Epic
%    Games}, se ofrece bajo un plan de suscripción mensual. El servicio de
%suscripción permite a los desarrolladores unirse a una comunidad dedicada a la
%construcción de grandes juegos y evolución del \textit{Unreal
%    Engine}\cite{unrealengine}.

%\textit{Unreal Engine} permite desarrollar juegos para plataformas como
%\textit{Windows PC, Mac, iOS y Android}. También es compatible con \textit{Xbox
%    One} y \textit{PlayStation 4}. \fixme{Existen}{} además trabajos recientes
%sobre otras plataformas como \textit{HTLM5} y \textit{Linux}\cite{unrealengine}.

%Sin embargo \fixme{existe}{} una versión gratuita del \textit{Unreal Engine}, el
%\Gls{udk}. El \Gls{udk} es la edición gratuita de \textit{Unreal Engine
%    3}\footnote{Versión previa a la actual versión comercial.} que proporciona
%acceso al motor de juegos 3D y a la herramienta profesional que se utiliza en el
%desarrollo de videojuegos \textit{blockbuster}, visualización arquitectónica, el
%desarrollo de juegos para móviles, modelos 3D, películas digitales y más.
%Utilizando \Gls{udk} se pueden implementar juegos y aplicaciones en
%\textit{Windows PC}, \textit{iOS} y \textit{Mac}.


\subsection{Blender Game Engine}

%\observacion{No hace falta hablar de blender}

\enquote{Blender Game Engine} es el motor de videojuegos de \textit{Blender Foundation} 
que permite crear aplicaciones 3D interactivas o simulaciones, desarrollado bajo 
la licencia \Gls{gnu}\cite{blender}.

\textit{Blender Game Engine} genera las escenas de forma continua en tiempo real
e incorpora facilidades para la interacción del usuario durante el proceso de
\textit{renderización}, procesa la lógica de sonido, de la física y la 
representación de simulaciones en orden secuencial\cite{blender}.

Posee la posibilidad de exportar en plataformas como
\textit{Windows, Linux y Mac OS}. También incluye básico en desarrollo para 
plataformas Android\cite{blender}.

Posee su propio entorno de desarrollo, las rutinas de programación pueden ser 
escritas en los lenguajes de programación Python y C++, y
una comunidad grande donde se puede recurrir. Además soporta formatos 3D 
como 3ds, dae, fbx, dxf\cite{blender}.

%\enquote{Blender Game Engine} es el motor de juego de \textit{Blender
%    Foundation} que permite crear aplicaciones 3D interactivas o simulaciones,
%desarrollado bajo la licencia \Gls{gnu}\cite{blender}.

%\textit{Blender Game Engine} genera las escenas de forma continua en tiempo real
%e incorpora facilidades para la interacción del usuario durante el proceso de
%\textit{renderización}\cite{blender}.

%Procesa la lógica de sonido, de la física y la representación de simulaciones en
%orden secuencial. El motor esta escrito en \textit{C++}. Posee un editor que
%proporciona una profunda interacción con la simulación, su funcionalidad se
%puede ampliar con \textit{scripts} \textit{Python} y esta diseñado para abstraer
%las características complejas del motor en una interfaz de usuario simple, que
%no requiere programación\cite{blender}.

%El motor de juego puede simular contenido dentro de \textit{Blender}, sin
%embargo, también incluye la posibilidad de exportar en plataformas como
%\textit{Windows, Linux y Mac OS}. También hay soporte básico para plataformas
%móviles con el proyecto \textit{Android Blender Player \fixme{GSOC}{?} 2012}.

\subsection{CryEngine}

%\observacion{Las descripciones tienen que restar lo mejor y lo pero de cada
%    engine, osino parece una propaganda}

\enquote{CryEngine} es el motor de videojuegos desarrollado por \textit{Crytek}. Existe 
una versión gratuita denominada \enquote{CryEngine Free SDK} con todas las 
funcionalidades, esta versión esta disponible para su descarga, pero ya ha sido descontinuada\cite{cryengine:sdk}.

Permite el desarrollo de videojuegos, películas, simulaciones 
y aplicaciones interactivas. Posee un conjunto de herramientas utilizadas para el 
análisis de rendimiento. Este motor de videojuegos permite exportar a plataformas 
como iOS y Android\cite{cryengine}.

Posee su propio entorno de desarrollo, las rutinas de programación pueden ser 
escritas en los lenguajes de programación C++ y Lua, y una comunidad de 
tamaño moderado donde se puede recurrir. Además sólo soporta sus propios formatos 
3D\cite{cryengine}.



%\enquote{CryEngine 3} es el motor de juegos desarrollado por \textit{Crytek}.
%\textit{CryEngine} es un motor avanzado para el desarrollo de juegos, películas,
%simulaciones de alta calidad y aplicaciones interactivas. Tiene características
%diseñadas específicamente para \textit{Windows PC, PlayStation 3 y Xbox
%    360}\cite{cryengine}.

%Existe una versión gratuita de \textit{CryEngine}, denominada \enquote{CryEngine
%    Free SDK} con todas las funcionalidades, esta versión esta disponible para
%su descarga, pero ya ha sido descontinuada\cite{cryengine:sdk}.

%\textit{CryEngine} posee además un conjunto de herramientas utilizadas para el
%análisis de rendimiento\cite{cryengine}, además de un \Gls{ide}, que permite
%editar texturas, interacciones, vehículos, etc.

\subsection{ShiVa3D}

\textit{ShiVa3D} es un motor para el desarrollo de videojuegos y aplicaciones 
3D desarrollado por ShiVa Technologies\cite{shiva}.

Un producto relacionado es el \enquote{ShiVa Server}, el cual permite el
desarrollo de aplicaciones multijugador. Las características de este servidor,
incluyen buen rendimiento, comunicación \textit{VoIp}, etc\cite{shiva}.

\textit{ShiVa} puede exportar juegos y aplicaciones a una cantidad variada de 
plataformas, incluyendo móviles como \textit{iOS, Android, BlackBerry y Windows
Phone}, de escritorio como \textit{Windows, Mac OS X y Linux}, los
navegadores web con soporte \textit{Flash y HTML5}, así como consolas como la
\textit{Xbox 360, PlayStation3 y Nintendo Wii}. El \Gls{ide} se ejecuta en
\textit{Windows} y \textit{Mac OS X}\cite{shiva}. 

Posee su propio entorno de desarrollo, las rutinas de programación pueden ser 
escritas en los lenguajes de programación FlowGraph y Lua, y una 
comunidad moderada donde se puede recurrir. Además sólo soporta el formato 
3D, dae\cite{shiva}.


%\textit{ShiVa3D} es un paquete para el desarrollo de juegos y aplicaciones 3D,
%posee un \Gls{ide} \Gls{wysiwyg} potente\cite{shiva}.

%\textit{ShiVa} puede exportar juegos y aplicaciones para más de $20$ plataformas
%de destino, incluyendo móviles como \textit{iOS, Android, BlackBerry y Windows
%    Phone}, de escritorio como \textit{Windows, Mac OS X y Linux}, los
%navegadores web con soporte \textit{Flash y HTML5}, así como consolas como la
%\textit{Xbox 360, PlayStation3 y Nintendo Wii}. El \Gls{ide} se ejecuta en
%\textit{Windows} y \textit{Mac OS X}\cite{shiva}. 

%Un producto relacionado es el \enquote{ShiVa Server}, el cual permite el
%desarrollo de aplicaciones multijugador. Las carácteristicas de este servidor,
%incluyen alto rendimiento, comunicación \textit{VoIp}, etc. \enquote{ShiVa
%    Server} se distribuye con una licencia distinta a
%\textit{Shiva3D}\cite{shiva}.

\subsection{Unity3D}

\textit{Unity} es un motor para el desarrollo de juegos,
desarrollado por \textit{Unity Technologies}. Posee una versión pagada 
y una gratuita\cite{unity3d}.

Posee un motor de \textit{renderizado} y un flujo de trabajo para la creación 
de contenido 3D interactivo, además permite mezclar contenido 3D, 2D, sonidos 
y animaciones de manera sencilla\cite{unity3d}.

La versión gratuita permite desarrollar juegos para múltiples plataformas. 
Entre las plataformas móviles soportadas, encontramos a \textit{iOS, Andriod, 
Windows Phone 8, BlackBerry 10}, entre las plataformas de escritorio a 
\textit{Windows, Mac y Linux}, plataformas web como \textit{Internet Explorer, 
Mozzilla Firefox, Google Chrome}\footnote{Requiere un plugin para el navegador que está
disponible para \textit{Windows y Mac}}, y entre plataformas de consolas a
\textit{Xbox 360, Xbox One, Wii, Wii U, Nintendo 3DS}\cite{unity3d}.

Posee su propio entorno de desarrollo, las rutinas de programación pueden ser 
escritas en los lenguajes de programación C-Sharp, UnityScript y Boo, y una 
comunidad grande donde recurrir. Además soporta formatos 3D como fbx, obj, max, 
blend, dae, 3ds, dxf, MB, MA\cite{unity3d}.\todox{Ver lo de 
C sharp}

%La versión paga de \textit{Unity}, \textit{Unity Pro} permite además plataformas
%como \textit{PlayStation 4, PlayStation 3 y PlayStation VITA}.


%\textit{Unity}\cite{unity3d} es un motor para el desarrollo de juegos,
%desarrollado por \textit{Unity Technologies}, incluye un \Gls{ide} con un un
%motor de \textit{renderizado} y flujos de trabajo para la creación de contenido
%3D interactivo, desarrollo multiplataforma. Además cuenta con una gran cantidad
%de \textit{assets}\footnote{Un \textit{Asset} es un paquete \textit{Unity} que
%    puede contener modelos, librerias, sonidos, etc.} disponibles en un
%\enquote{Asset Store} y una gran comunidad donde se intercambian conocimientos.

%En \textit{Unity} se pueden desarrollar de forma sencilla elementos 2D y 3D.
%Posee un \Gls{ide} intuitivo y flexible, el nivel visual y de audio son de gran
%calidad, un sistema de animación poderoso. 

%Permite desarrollar juegos para múltiples plataformas. Entre las plataformas
%móviles soportadas, encontramos a \textit{iOS, Andriod, Windows Phone 8,
%    BlackBerry 10}, entre las plataformas de escritorio a \textit{Windows, Mac y
%    Linux}, plataformas web como \textit{Internet Explorer, Mozzilla Firefox,
%    Google Chrome}\footnote{Requiere un plugin para el navegador que está
%    disponible para \textit{Windows y Mac}}, y entre plataformas de consolas a
%\textit{Xbox 360, Xbox One, Wii, Wii U, Nintendo 3DS}.

%La versión paga de \textit{Unity}, \textit{Unity Pro} permite además plataformas
%como \textit{PlayStation 4, PlayStation 3 y PlayStation VITA}.

%Debido a la alta popularidad de \textit{Unity}, un paquete fue desarrollado por
%\textit{Facebook} que permite una interacción sencilla con la \Gls{api} de la
%red social \textit{Facebook} en un \textit{Asset} escrito en \cs{}, este paquete
%se encuentra en \textit{Asset Store} de \textit{Unity}.

\input{tecnologias/seleccion_motor}
\section{Entorno de desarrollo de la solución}

Este proceso se compone de varios pasos necesarios, para llevar a cabo el
desarrollo completo se debe recurrir a una gran cantidad de tecnologías,
herramientas, \textit{frameworks}, y recursos.

En esta sección se describen todos las partes involucradas en el desarrollo de
la solución propuesta. 

La solución se compone de dos partes, la primera es la aplicación que los
usuarios utilizan para realizar las prácticas, denominada \textit{Frontend}, y la segunda
parte, es un servidor que se encarga de almacenar la información sobre los
usuarios de la solución y como la utilizan, denominado \textit{Backend}.

Primeramente se definen las herramientas de gestión de código, pues ellas son
utilizadas tanto por la solución como por el \textit{backend}, luego se definen
las herramientas específicas de la creación de la simulación y por último, se
citan y describen de manera breve las herramientas utilizadas por el
\textit{backend}.

La idea detrás de separar la solución en \textit{Frontend} y \textit{Backend} 
es que ambas partes de la solución se podrán mantener de forma separada sin que 
el cambio de una afecte el funcionamiento de la otra. 

\subsection{Herramientas de gestión de código}

La gestión del código fuente desarrollado como parte de esta tesis fue realizado
mediante la utilización de la herramienta de control de código fuente
\textit{Git}, \textit{Git} es un software de control de versiones distribuido,
de código abierto bajo la licencia \Gls{gnu}\cite{git}. El proveedor del
servicio \textit{Git} utilizado es \textit{BitBucket}\cite{bitbucket}, el cual
almacena repositorios \textit{Git}, la principal característica y motivación por
la cual se utiliza este servicio es que el mismo permite mantener varios
repositorios privados de manera gratuita\cite{bitbucket}.

\subsection{Desarrollo del \textit{frontend}}

El desarrollo de la solución requiere de una variedad importante de tecnologías
aparte del motor de videojuegos, las herramientas descritas en esta sección complementan al
motor seleccionado y facilitan la creación de contenido.

Se utilizó el \Gls{ide} \textit{Unity Editor} para la creación de las escenas,
el \Gls{ide} \textit{MonoDevelop} y el lenguaje \cs{} para la programación de la
interacción entre los componentes de la solución.

Adicionalmente se utilizaron varias herramientas de diseño para crear
componentes 3D y 2D, \textit{Make Human} es utilizado para la creación de los
pacientes, \textit{3Ds Max} permite la creación de objetos como gazas, y otros
elementos utilizados dentro de la solución. En cuanto a los gráficos 2D, se
utilizaron \textit{Photoshop}, y diversas páginas web que proveían contenido
gratuito.

\subsubsection{Unity Editor}

El \Gls{ide} de \textit{Unity} es la herramienta en la se crean los videojuegos
o simulaciones. Este \Gls{ide} importa todos los \textit{asset}\footnote{Un
    \textit{Asset} es un paquete \textit{Unity} que puede contener modelos,
    librerias, sonidos, etc.}. Permite compilar las escenas con los terrenos,
luces, audios, personajes, física, entre otros. Se puede agregar interacción a
través de \textit{scripting}. En este \Gls{ide} se puede probar y editar en
forma simultánea los videojuegos y desplegarlos en las plataformas
elegidas\cite{unity3d}. 

Este editor es la única herramienta que permite crear escenas en \textit{Unity},
sin embargo, este \Gls{ide}, no permite la edición de código\footnote{Existen
    \textit{Assets} que permiten la edición de código dentro del \textit{Unity
        Editor}, pero son pagas y de baja reputación}, por ello se necesita un
editor externo de código.


\subsubsection{MonoDevelop}

\textit{MonoDevelop} es un \Gls{ide} de código abierto, bajo la licencia
\Gls{gnu} apoyado principalmente por la comunidad \textit{Mono}. Es el \Gls{ide}
utilizado por defecto en el desarrollo de aplicaciones para \textit{Unity3D}, el
mismo soporta varios lenguajes de programación, como \cs{},
\textit{UnityScript}, y \textit{Boo}. Es un \Gls{ide} multiplataforma que
soporta \textit{Windows} al igual que \textit{Unity3D}.

Existen otros editores que pueden ser utilizados para el desarrollo del código
fuente, pero los mismos no cuentan con el mismo nivel de integración y no son
gratuítos\footnote{\textit{Microsoft Visual Studio} permite el mismo nivel de
    integración que \textit{MonoDevelop} con un \textit{plugin}, pero el mismo
    era de pago durante el desarrollo de la solución}.

\subsubsection{Lenguaje de programación}

\textit{Unity3D} utiliza versiones limitadas\footnote{La definición del lenguaje
    es la misma, pero las librerías estándar no están completas} de tres
lenguajes de programación: \cs{}, \textit{UnityScript}, y
\textit{Boo}\cite{unity:script}. Estos lenguajes son compilados y orientados a
objetos.

Otra característica interesante es que, por el orden de compilación de los
proyectos \textit{Unity3D}, los archivos \textit{UnityScript} y \textit{Boo} son
compilados antes que los \cs{}, esto provoca que, las clases
\textit{UnityScript} sean utilizables desde \cs{}, lo que no se cumple en el
caso contrario, es decir, las clases \cs{} no son accesibles desde código
\textit{UnityScript} o \textit{Boo}.

Por los criterios mencionados se selecciona a \cs{} como el lenguaje de
implementación, por ser el lenguaje con más ayuda en línea, y por que las
librerías no diseñadas específicamente para \textit{Unity3d} pueden ser
utilizadas. Otro factor que influye en la elección es la familiaridad de los
autores con lenguajes similares. 

\subsubsection{Herramientas de diseño}

Las herramientas utilizadas para crear modelos 3D son las siguientes:

\begin{itemize}
\item \textbf{MakeHuman}: es un software de código abierto bajo la licencia
    \Gls{agnu} para crear personajes humanos 3D. Es una herramienta diseñada
    para simplificar la creación de seres humanos virtuales utilizando una
    interfaz gráfica de usuario\cite{makehuman}. 
\item \textbf{3DS Max}: es un software privado de modelado 3D, que además posee
    herramientas para animación, simulación y renderización. Esta herramienta
    fue utilizada para crear objetos 3D que no fueran personajes humanos y para
    exportar modelos de un formato a otro que fuera compatible con
    \textit{Unity3D}\cite{3dsmax}.
\item \textbf{Photoshop}: es una herramienta de edición de gráficos 2D de
    \textit{Adobe}, permite la creación y edición de gráficos, es utilizada para
    la creación de iconos, botones y demás contenido 2D que forma parte de la
    solución.
\end{itemize}

\subsection{Desarrollo del \textit{backend}}


A fin de registrar las actividades del usuario, se necesita de un servidor que
almacene los datos de todos los usuarios y las actividades que estos realizan
dentro de la solución.

\begin{figure}[ht]
\centering
\includegraphics[scale=0.5]{tecnologias/images/backend_diagrama.png}
\caption{Diagrama de la interacción de los usuarios con el \textit{BackEnd}, se
    puede observar a grandes rasgos, los componentes del sistema y los servicios
    que ofrece.}
\label{fig:backend_diagrama}
\end{figure}


Se puede observar en lineas generales como funciona este servicio
en~\ref{fig:backend_diagrama}, desde el registro de actividades del usuario, a
la comunicación con el \textit{backend}, su posterior traducción y persistencia
en una base datos, así, este servicio debe proveer:

\begin{itemize}
    \item \textbf{Alta disponibilidad}: el servidor debe estar disponible en
        todo momento, cualquier día de la semana y a cualquier hora. Los
        requisitos de accesibilidad son estrictos, pues se necesita que los
        usuarios envíen datos sin inconvenientes cuando crean necesario.
    \item \textbf{Accesibilidad}: el servidor debe poder ser accesible desde
        cualquier red móvil.
    \item \textbf{Bajo costo de comunicación}: la comunicación del usuario con
        el \textit{backend} debe ser lo menos costosa posible, pues se utilizan
        recursos del usuario.
\end{itemize}

Para el desarrollo de la aplicación web que almacena los datos se utiliza
\Gls{javaee} en su versión $6$, la misma se utiliza por la familiarización de
los autores con la tecnología, y la facilidad que provee la misma para la
realización de servicios web que permitan la interacción con la solución.

Para los servicios se utiliza la arquitectura \Gls{rest}, la principal
motivación para utilizar \Gls{rest} es la eficiencia en el uso de la
red\cite{pautasso2008restful}, la cual es también la motivación para la
utilización de \Gls{json}. La implementación del lado del servidor de la
arquitectura \Gls{rest} es \textit{RestEasy}, de parte del \textit{frontend} se utiliza
la implementación por defecto de \textit{Unity3D}.

El almacenamiento permanente de los datos se logra con la utilización de
\textit{PostgreSQL}, el cual es un motor de bases de datos de código abierto
dirigido por una comunidad de desarrolladores llamada \textit{PostgreSQL Global
    Development Group}. La versión elegida es la $9.1$.

\subsubsection{OpenShift}

A fin de obtener las características necesarias, de alta disponibilidad y
accesibilidad, se utiliza la herramienta de plataforma como servicio de
\textit{RedHat} llamada \textit{OpenShift}, la cual es un producto de código
abierto dirigido por \textit{RedHat}.

Además de dirigir el proyecto, \textit{RedHat} provee un servicio limitado y
gratuito\cite{openshift:pricing}.\footnote{Existen versiones completas del
    producto mantenidas por \textit{RedHat}, las cuales tienen un costo mensual
    y de acuerdo a las funcionalidades utilizadas\cite{openshift:pricing}} Para
esta tesis se utilizo el servicio gratuito con la plataforma \textit{JBoss
    Application Server 7.1} y \textit{PostgreSQL 9.1}.
