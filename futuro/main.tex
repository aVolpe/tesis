\chapter{Trabajos Futuros}
\label{chap:futuro}

La utilización de las \Gls{tic} en la educación es un área de estudio
interesante, si a esto se le añade el nuevo rol asumido por las \Gls{tic} en las
corrientes pedagógicas contemporáneas, y la enseñanza de profesionales de la
salud, los temas para trabajos de investigación son prácticamente ilimitados. 

En este capítulo se describen posibles temas para trabajos futuros, que utilicen
a los juegos serios como área de investigación.

\section{Nuevos escenarios de práctica}

Este trabajo presenta dos procedimientos relacionados a la enfermería, en el
área existen innumerables procedimientos cuya simulación puede tener un impacto
positivo. Durante las reuniones con los profesionales varios de estos
procedimientos fueron discutidos, entre los que podemos encontrar a:

\begin{itemize}
    \item Colocación del collarín
    \item Recepción del recién nacido
    \item Venoclisis
\end{itemize}

Cuanto más procedimientos sean simulados, mayor será el beneficio para los alumnos.

\section{Visión de progreso}

Una de las características de la solución propuesta es la retroalimentación que
recibe el usuario al terminar una partida, dándole una lista de pasos
realizados correcta e incorrectamente, además de una breve explicación de la
razón por la que un paso fue realizado erróneamente.

Un añadido a la retroalimentación, sería el progreso del alumno, un lugar donde
el mismo pueda ver como fue mejorando a través de diversas sesiones, donde se
observen cuales son los puntos débiles recurrentes y otros aspectos que pueden
ser extraídos cuando se estudian los datos de varias sesiones de manera
conjunta. 

Se puede utilizar esta información para proveer una retroalimentación aún más
específica, por ejemplo, si se observa que el alumno no puede realizar un paso
de manera correcta aún cuando lo intentó en reiteradas ocasiones.

\section{Integración con sistemas de monitoreo}

El presente trabajo no propone mecanismos de monitoreo del progreso de los
alumnos por parte de los docentes, la cual es un área interesante, pues la
información recabada acerca del desempeño de los alumnos podría servir como una
alerta al profesor.

Si se estudia el comportamiento de todos los alumnos de manera simultánea, se
podría obtener información acerca de las debilidades y fortalezas del grupo de
alumnos, y así los profesores tendrían una herramienta adicional para el
desarrollo de sus actividades académicas.

\section{Multijugador}

El ser humano es un ser social, el construccionismo indica que el conocimiento
es fruto de la interacción social, crear simulaciones donde varios alumnos
participen al mismo tiempo, interactuando entre sí, y creando conocimiento,
permitirá explotar áreas que no son posibles con un sólo jugador, como:

\begin{itemize}
    \item \textbf{Comunicación especializada}, los profesionales de la salud se
        comunican con señas y palabras claves.
    \item \textbf{Sincronización de actividades}, dos profesionales de salud que
        participan en el mismo procedimiento no realizan las mismas actividades,
        la coordinación y sincronización de sus acciones es un factor clave para
        la realización del procedimiento.
    \item \textbf{Trabajos multidisciplinarios}, los profesionales de salud trabajan
        constantemente con personas de diferente especialidad y función, por
        ejemplo, durante una cirugía enfermeros y médicos trabajan en forma
        conjunta.
\end{itemize}

\section{Escenarios dinámicos}

Las simulaciones se centran en los procedimientos. El entorno es una herramienta
auxiliar que aumenta el realismo y la inmersión. Simulaciones centradas en crear
escenas con entornos complejos, donde se deban realizar diferentes
procedimientos de acuerdo a la situación, permitirán entrenar el poder y la
velocidad de reacción, el nerviosismo y otros aspectos intrínsecos a situaciones
desconocidas. 

Este tipo de simulaciones es un área interesante de estudio, si bien requieren
un mayor tiempo de desarrollo, el potencial de las mismas es mayor que la
solución propuesta en este trabajo. 

\section{Exploración de plataformas de realidad virtual}

En la actualidad las herramientas de realidad virtual permiten un nivel de
inmersión muy alto, aumentando el nivel de realismo de las simulaciones y
videojuegos.     

Herramientas como el \emph{Oculus Rift}, permiten crear entornos virtuales donde
el jugador se puede desplazar e incluso utilizar elementos de forma
natural\cite{makerbot}, cabe mencionar que desde finales del $2014$, estas
herramientas pueden ser utilizadas de manera gratuita con
\emph{Unity3d}\cite{unity:vr}.

Si bien la utilización de este tipo de herramientas no permite la movilidad 
buscada en esta tesis, se considera un área interesante para evaluar posibles
herramientas de apoyo en entornos completamente simulados, creando así un nivel
de interacción similar al de la realidad.


\section{Dificultad de acuerdo al alumno}

El nivel de dificultad de los diferentes desafíos debe ser acorde al nivel de
preparación de los usuarios. En este trabajo la dificultad es siempre la misma,
pues los alumnos seleccionados provienen del mismo entorno y aprobaron la misma
cantidad de asignaturas en su carrera.

Si se consideran alumnos de distintos entornos y niveles, una simulación debería
adaptarse a la preparación y capacidad de cada alumno, por ello, se podrían
utilizar técnicas para que se adapte al nivel del usuario, por ejemplo, la
dificultad de la simulación puede ser progresiva, es decir, basada en objetivos
o metas cada vez más difíciles.

Un aspecto interesante a analizar en este punto, son los sistemas de tutoría
inteligente, que pueden ayudar a determinar contenido cognitivo preciso para los
usuarios de acuerdo a su nivel de conocimiento y de aptitud.
