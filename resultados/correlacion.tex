\section{Correlación entre variables}

% Datos globales
En la tabla~\ref{tab:all_correlation} se observa la correlación entre cinco
variables estudiadas, a fin de observar si existe alguna correlación entre los
valores, se utiliza la correlación de \emph{Pearson}, descrita
en~\ref{sec:correlacion}.

\begin{table}[H]
\centering
\begin{tabular}{lrrrrrr}
\toprule
        &
\begin{sideways}\textbf{Tiempo de Uso}\end{sideways}             &
\begin{sideways}\textbf{Encuesta subjetiva}\end{sideways}        &
\begin{sideways}\textbf{Encuesta objetiva}\end{sideways}         &
\begin{sideways}\textbf{Puntaje Máximo Extracción}\end{sideways} &
\begin{sideways}\textbf{Puntaje Máximo Glasgow}\end{sideways}    \\
\midrule
Tiempo de Uso             & 1    & -0.2  & 0.15  & 0.62 & 0.41 & 0.78 \\
Encuesta subjetiva        & -0.2 & 1     & -0.07 & 0.04 & 0.11 & -0.28\\
Encuesta objetiva         & 0.15 & -0.07 & 1     & 0.44 & 0.44 & 0.02 \\
Puntaje máximo Extracción & 0.62 & 0.04  & 0.44  & 1    & 0.96 & 0.44 \\
Puntaje máximo Glasgow    & 0.41 & 0.11  & 0.44  & 0.96 & 1    & 0.27 \\
\bottomrule               & 0.78 & -0.28 & 0.02  & 0.44 & 0.27 & 1    \\
\end{tabular}
\caption{Correlación entre factores estudiados} 
\label{tab:all_correlation}
\end{table}

Las correlaciones fuertes, que se observan en la
tabla~\ref{tab:all_correlation}, son:

\begin{itemize}
    \item Tiempo de uso y puntaje máximo extracción, $0,62$, correlación
        positiva fuerte.
    \item Tiempo de uso y puntaje máximo Glasgow, $0,78$, correlación positiva
        muy fuerte.
    \item Puntaje máximo extracción y encuesta objetiva, $0,44$, correlación
        positiva fuerte.
\end{itemize}


La tabla~\ref{tab:all_correlation} indica que existe una correlación positiva
fuerte ($0,62$ y $0,78$) entre el tiempo de uso y el puntaje más alto obtenido,
lo que sugiere que mientras más se utiliza la solución, se obtienen mejores
resultados. 

Una correlación positiva fuerte entre el puntaje máximo obtenido en la
Extracción y la encuesta objetiva ($0,44$), sugiere que existe una relación entre
el nivel de conocimientos de los alumnos y su desempeño en la práctica.

