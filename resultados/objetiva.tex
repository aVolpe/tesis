\section{Encuesta Objetiva}
\label{sec:res_OBJETIVA}

\observacion{Hay que separar bien la validez de tener la persona para las
    pruebas de diseño y para las pruebas de objetivos}

Como se detalló en la sección~\ref{sec:objetiva}, la encuesta realizada a cada
usuario, parte del experimento, es utilizada para obtener una comparación en
cuanto al rendimiento de los usuarios que forman parte de la muestra y los que
forman parte del grupo de control.

La tabla~\ref{tab:objetiva_rendimiento_por_pregunta} muestra el nivel de acierto
en promedio por pregunta de los usuarios que forman parte de la muestra y de los que
forman parte del grupo de control, con sus respectivas desviaciones estándar. Según
estos datos, en el $60\%$ de los casos hay una leve mejoría en cuanto al nivel de acierto
para los usuarios que forman parte de la muestra.

\begin{table}[!hbt]
\centering
\begin{tabular}{lrrr}
\toprule
\textbf{Pregunta} & 
\textbf{promsujetos} & 
\textbf{promcontrol} & 
\textbf{promtotal} \\ 
\midrule
1         & 0.36 & 0.18 & 0.20 \\
2         & 0.64 & 0.60 & 0.60 \\
3         & 0.09 & 0.14 & 0.13 \\
4         & 0.27 & 0.25 & 0.26 \\
5         & 0.82 & 0.56 & 0.59 \\
6         & 0.00 & 0.18 & 0.16 \\
7         & 0.64 & 0.51 & 0.53 \\
8         & 0.45 & 0.27 & 0.29 \\
9         & 0.18 & 0.32 & 0.31 \\
10        & 0.36 & 0.45 & 0.45 \\
\midrule
Promedio: & 3.82 & 3.47 & 3.49  \\
\bottomrule
\end{tabular}
\caption{Rendimiento promedio de usuarios por pregunta}
\label{tab:objetiva_rendimiento_por_pregunta}
\end{table}

Los datos mostrados sólo sugieren levemente una tendencia a la mejoría de los puntajes 
para los usuarios que forman parte de la muestra, sin embargo, estos datos no pueden ser 
tomados para realizar conclusiones ya que la cantidad de sesiones de juego por usuario no 
se considera suficiente para que el uso de la solución propuesta afecte realmente en el 
aprendizaje del mismo.
