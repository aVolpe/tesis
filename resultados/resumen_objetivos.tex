\section{Objetivos}
\label{sec:resultados_objetivos}

Esta sección presenta un resumen de los resultados de los objetivos propuestos
para la evaluación en la sección~\label{sec:evaluacion_objetivos}. 

\subsection{Validar las hipótesis asumidas durante el desarrollo de la solución}

En la tabla~\ref{tab:resultado_resumen_hipotesis} se observa la opinion de los
alumnos con respecto a las hipótesis asumidas en~\ref{sec:hipotesis}. Se observa
una aceptación a las hipótesis asumidas.

\begin{table}[!hbt]
\centering
\begin{tabular}{lcr}
\toprule
Hipótesis                        & Promedio Subjetiva      & Promedio estandarizado \\
\midrule
Comandos de voz con interfaz     & De acuerdo              & $0,55$ \\
Extracción uniforme de elementos & Parcialmente de acuerdo & $0,65$ \\
Acciones de bioseguridad         & De acuerdo              & $0,58$ \\
Representación iconográfica      & Parcialmente de acuerdo & $0,53$ \\
Factores motivadores             & De acuerdo              & $0,65$ \\
Falta de pistas                  & De acuerdo              & $0,61$ \\
\bottomrule
\end{tabular}
\caption{Hipótesis con su aceptación}\label{tab:resultado_resumen_hipotesis}
\end{table}


\subsection{Evaluar los puntos fuertes y débiles de la solución}

Utilizando los datos con doble estandarización de la
tabla~\ref{tab:subjetiva_conformidad_corregida}, se crea la
tabla~\ref{tab:resultado_resumen_aspectos_aceptacion}, donde se observa la
apreciación de los usuarios por cada aspecto estudiado.

\begin{table}[!hbt]
\centering
\begin{tabular}{lcr}
\toprule
Hipótesis         & Promedio Subjetiva      & Promedio estandarizado \\
\midrule
Motivación        & De acuerdo              & $0.67$  \\
Exploración       & De acuerdo              & $0.68$  \\
Inmersión         & De acuerdo              & $0.63$  \\
Pedagogía         & De acuerdo              & $0.67$  \\
Representación    & Parcialmente de acuerdo & $0.53$  \\
Retroalimentación & Parcialmente de acuerdo & $0.60$  \\
Utilidad          & De acuerdo              & $0.69$  \\
\bottomrule
\end{tabular}
\caption{Aceptación por aspecto de la solución}
\label{tab:resultado_resumen_aspectos_aceptacion}
\end{table}

Para  obtener una mejor visión de las fortalezas y debilidades de la solución
propuesta, se presenta el gráfico de \emph{kiviat}~\ref{fig:subjetiva_kiviat},
en la misma se puede observar cuales son los puntos débiles de la solución.

\begin{figure}[!ht]
\begin{tikzpicture}[label distance=.15cm]
\tkzKiviatDiagram[radial=2,
                    lattice=2, step=2,
                    scale=2.3]%
                {Motivación,
                 Exploración,
                 Inmersión,
                 Pedagogía,
                 Representación,
                 Retroalimentación,
                 Utilidad}
\tkzKiviatLine[thick,
                color=blue,
                mark=ball,
                ball color=red,
                mark size=1pt,opacity=.2, 
                fill=red!20](0.67,0.68,0.63,0.67,0.53,0.60,0.69)
\end{tikzpicture}
\label{fig:subjetiva_kiviat}
\caption{Gráfico de Kiviat de los factores evaluados}
\end{figure}

Se observa que las principales debilidades de la solución son la representación
y la retroalimentación, y las fortalezas la utilidad, pedagogía, exploración, y
la motivación.

\subsection{Determinar el nivel de aceptación de la solución}

En la sección~\ref{sec:res_subjetiva_abiertas} se muestra un resumen de la
apreciación de los alumnos hacia la solución, en este punto se observa que el
$100\%$ de los alumnos cree que es beneficioso contar con este tipo de
soluciones.


\subsection{Evaluar la utilización de la solución, y el progreso de los
    usuarios}

Se observa en las tablas~\ref{tab:log_hemocultivo_puntaje}
y~\ref{tab:log_glasgow_random_puntaje} los alumnos que participaron de la prueba
mejoran su desempeño a medida que aumenta el número de partidas. 

Es importante notar que la cantidad de partidas no es uniforme entre los
alumnos, es decir hay alumnos con más de $10$ partidas y usuarios con menos de
$5$, por ello, es difícil demostrar que existe un progreso a medida que aumenta
el número de partidas.

\subsection{Identificar la influencia de la utilización de la solución en el ámbito
    pedagógico}

Los datos mostrados en la sección~\ref{sec:res_objetiva} sólo sugieren levemente
una tendencia a la mejoría de los puntajes para los usuarios que forman parte de
la muestra, sin embargo, estos datos no pueden ser tomados para realizar
conclusiones ya que la cantidad de sesiones de juego por usuario no se considera
suficiente para que el uso de la solución propuesta afecte realmente en el
aprendizaje del mismo.

\subsection{Determinar correlaciones entre variables estudiadas}

% Datos globales
En la tabla~\ref{tab:all_correlation} se observa la correlación entre cinco
variables estudiadas, a fin de observar si existe alguna correlación entre los
valores, se utiliza la correlación de \emph{Pearson}, descrita
en~\ref{sec:correlacion}.

\begin{table}[H]
\centering
\begin{tabular}{lrrrrrr}
\toprule
        &
\begin{sideways}\textbf{Tiempo de Uso}\end{sideways}             &
\begin{sideways}\textbf{Encuesta subjetiva}\end{sideways}        &
\begin{sideways}\textbf{Encuesta objetiva}\end{sideways}         &
\begin{sideways}\textbf{Puntaje Máximo Extracción}\end{sideways} &
\begin{sideways}\textbf{Puntaje Máximo Glasgow}\end{sideways}    \\
\midrule
Tiempo de Uso             & 1    & -0.2  & 0.15  & 0.62 & 0.41 & 0.78 \\
Encuesta subjetiva        & -0.2 & 1     & -0.07 & 0.04 & 0.11 & -0.28\\
Encuesta objetiva         & 0.15 & -0.07 & 1     & 0.44 & 0.44 & 0.02 \\
Puntaje máximo Extracción & 0.62 & 0.04  & 0.44  & 1    & 0.96 & 0.44 \\
Puntaje máximo Glasgow    & 0.41 & 0.11  & 0.44  & 0.96 & 1    & 0.27 \\
\bottomrule               & 0.78 & -0.28 & 0.02  & 0.44 & 0.27 & 1    \\
\end{tabular}
\caption{Correlación entre factores estudiados} 
\label{tab:all_correlation}
\end{table}

Las correlaciones fuertes, que se observan en la
tabla~\ref{tab:all_correlation}, son:

\begin{itemize}
    \item Tiempo de uso y puntaje máximo extracción, $0,62$, correlación
        positiva fuerte.
    \item Tiempo de uso y puntaje máximo Glasgow, $0,78$, correlación positiva
        muy fuerte.
    \item Puntaje máximo extracción y encuesta objetiva, $0,44$, correlación
        positiva fuerte.
\end{itemize}


La tabla~\ref{tab:all_correlation} indica que existe una correlación positiva
fuerte ($0,62$ y $0,78$) entre el tiempo de uso y el puntaje más alto obtenido,
lo que sugiere que mientras más se utiliza la solución, se obtienen mejores
resultados. 

Una correlación positiva fuerte entre el puntaje máximo obtenido en la
Extracción y la encuesta objetiva ($0,44$), sugiere que existe una relación entre
el nivel de conocimientos de los alumnos y su desempeño en la práctica.
