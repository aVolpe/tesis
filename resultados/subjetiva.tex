%! TEX root = ../main.tex
%! TEX root = ../main.tex

\section{Encuesta Subjetiva}


\replantear{En el análisis de los resultados de las encuestas subjetivas,
    existieron alumnos que no respondieron todas
    las preguntas, para tratar este tipo de casos, es importante analizar la
    naturaleza del patrón de datos faltantes\cite{carpita2011imputation}.}

La información recogida por la encuesta muestra que hay datos faltantes, como se
explico en~\ref{sec:informacion_faltante}, esta información faltante es
completamente aleatoria en relación a la variable medida y a las demás
variables, de hecho, una sola encuesta tiene información faltante, así, se
establece que el tipo de información faltante es \emph{Información faltante
    completamente aleatoria}.

En su resumen de las diferentes técnicas y cuando se deben utilizar,
\cite{tsikriktsis2005review}, recomienda la utilización de la sustitución basada
en promedio por caso. Así, se completan los valores faltantes con el promedio de
respuestas completadas por el usuario.

\subsection{Resultados}

Se presentan a continuación los resultados de las encuestas, agrupados por los
factores definidos en~\ref{sec:variables}.

La tabla~\ref{tab:subjetiva_conformidad_exploracion} nos muestra las respuestas
de los alumnos a las preguntas relacionadas al factor exploración, son cuatro
preguntas, las cuales fueron descritas en~\ref{sec:sub_exploracion}. 

\begin{table}[!hbt]
\centering
\begin{tabular}{@{} *{5}{r} @{}}
\toprule
& \multicolumn{4}{c}{Exploración} \\
\cmidrule(lr){2-5}
Alumno &
\parbox{3cm}{Funciones realizadas por los elementos del juego} &
\parbox{3cm}{Aleatoriedad para afianzar conocimientos} &
\parbox{2.5cm}{Aleatoriedad para representar realismo} &
\parbox{2.5cm}{Facilidad de uso}  \\
\midrule
1         & 2   & 6   & 5   & 6  \\
2         & 6   & 6   & 4   & 6  \\
3         & 3   & 3   & 5   & 5  \\
4         & 6   & 6   & 6   & 6  \\
5         & 6   & 6   & 2   & 5  \\
6         & 6   & 6   & 6   & 6  \\
7         & 7   & 7   & 7   & 7  \\
8         & 6   & 6   & 7   & 7  \\
9         & 5   & 7   & 7   & 7  \\
10        & 6   & 7   & 6   & 6  \\
11        & 7   & 6   & 7   & 6  \\
\bottomrule
\end{tabular}
\caption{Resultados de la encuesta subjetiva relacionados al factor Exploración}
\label{tab:subjetiva_conformidad_exploracion}
\end{table}

La tabla~\ref{tab:subjetiva_conformidad_gamificacion} muestra las respuestas de
los alumnos a las preguntas relacionadas al factor \textit{Gamificación}, son
cinco preguntas, las cuales fueron descritas en~\ref{sec:sub_gamificacion}. 

\begin{table}[!hbt]
\centering
\begin{tabular}{@{} *{5}{r} @{}}
\toprule
& \multicolumn{4}{c}{Gamificación} \\
\cmidrule(lr){2-5}
Alumno &
\parbox{2.5cm}{Motivación del puntaje} &
\parbox{2.5cm}{Importancia del puntaje} &
\parbox{3cm}{Socialización de los puntajes} &
\parbox{3cm}{Medición del tiempo como motivación} \\
\midrule
1  & 6 & 4 & 4 & 7  \\
2  & 7 & 4 & 6 & 6  \\
3  & 6 & 6 & 5 & 6  \\
4  & 1 & 4 & 6 & 1  \\
5  & 2 & 2 & 7 & 7  \\
6  & 6 & 5 & 4 & 6  \\
7  & 7 & 7 & 6 & 7  \\
8  & 7 & 7 & 7 & 7  \\
9  & 7 & 7 & 7 & 7  \\
10 & 7 & 4 & 5 & 7  \\
11 & 5 & 4 & 5 & 6  \\
\bottomrule
\end{tabular}
\caption{Resultados de la encuesta subjetiva relacionados al factor Gamificación}
\label{tab:subjetiva_conformidad_gamificacion}
\end{table}

La tabla~\ref{tab:subjetiva_conformidad_inmersion} muestra las respuestas de
los alumnos a las preguntas relacionadas al factor \textit{Inmersión}, son
cinco preguntas, las cuales fueron descritas en~\ref{sec:sub_inmersion}. 

\begin{table}[!hbt]
\centering
\begin{tabular}{@{} *{6}{r} @{}}
\toprule
& \multicolumn{5}{c}{Inmersión} \\
\cmidrule(lr){2-6}
Alumno &
\parbox{2.5cm}{Escenografía para entrar en ambiente} &
\parbox{2.5cm}{Juegos cortos como ayuda para la repetición} &
\parbox{2.5cm}{Gráficos en tres dimensiones para entender el entorno} &
\parbox{2.5cm}{Realismo a través de ordenes verbales} &
\parbox{2.5cm}{Simulación como herramienta} \\
\midrule
1  & 4 & 6 & 4 & 5 & 3  \\
2  & 6 & 6 & 6 & 6 & 6  \\
3  & 6 & 6 & 6 & 5 & 6  \\
4  & 4 & 6 & 7 & 5 & 6  \\
5  & 6 & 6 & 5 & 6 & 6  \\
6  & 6 & 6 & 6 & 4 & 4  \\
7  & 7 & 7 & 7 & 7 & 7  \\
8  & 6 & 7 & 7 & 7 & 7  \\
9  & 6 & 7 & 7 & 7 & 7  \\
10 & 6 & 3 & 4 & 6 & 6  \\
11 & 5 & 3 & 5 & 5 & 4  \\
\bottomrule
\end{tabular}
\caption{Resultados de la encuesta subjetiva relacionados al factor Inmersión}
\label{tab:subjetiva_conformidad_inmersion}
\end{table}


La tabla~\ref{tab:subjetiva_conformidad_pedagogia} agrupa las respuestas de los
alumnos según el factor pedagógico, son tres preguntas, las cuales fueron
descritas en~\ref{sec:sub_pedagogia}. 

\observacion{En el capítulo anterior faltaba muchos más, donde están?}
\observacion{Agregar promedios generales}
\begin{table}[!hbt]
\centering
\begin{tabular}{@{} *{4}{r} @{}}
\toprule
& \multicolumn{3}{c}{Pedagogía} \\
\cmidrule(lr){2-4}
Alumno &
\parbox{4cm}{La solución para memorizar y comprender el procedimiento} &
\parbox{4cm}{Falta de pistas como ayuda al aprendizaje} &
\parbox{4cm}{Suficiencia de los botones que indican acciones} \\
\midrule
1  & 6 & 6 & 6  \\
2  & 6 & 6 & 7  \\
3  & 4 & 6 & 6  \\
4  & 6 & 7 & 6  \\
5  & 7 & 5 & 6  \\
6  & 4 & 4 & 6  \\
7  & 7 & 6 & 7  \\
8  & 6 & 7 & 7  \\
9  & 7 & 7 & 7  \\
10 & 6 & 7 & 7  \\
11 & 5 & 6 & 5  \\
\bottomrule
\end{tabular}
\caption{Resultados de la encuesta subjetiva relacionados al factor Pedagogía}
\label{tab:subjetiva_conformidad_pedagogia}
\end{table}


La tabla~\ref{tab:subjetiva_conformidad_representacion} agrupa las respuestas de
los alumnos según la calidad de presentación, son cinco preguntas, las cuales
fueron descritas en~\ref{sec:sub_representacion}. 

\begin{table}[!hbt]
\centering
\begin{tabular}{@{} *{6}{r} @{}}
\toprule
& \multicolumn{5}{c}{Representación} \\
\cmidrule(lr){2-6}
Alumno &
\parbox{2.5cm}{Movimientos motrices del paciente} &
\parbox{2.5cm}{Movimientos oculares del paciente} &
\parbox{2.5cm}{Reacción verbal del paciente} &
\parbox{2.5cm}{Distinción entre los estados del paciente} &
\parbox{2.5cm}{Acciones las herramientas} \\
\midrule
1  & 6 & 6 & 2 & 5 & 2  \\
2  & 4 & 5 & 5 & 6 & 4  \\
3  & 5 & 3 & 3 & 3 & 3  \\
4  & 6 & 5 & 2 & 4 & 2  \\
5  & 2 & 2 & 6 & 6 & 6  \\
6  & 6 & 4 & 6 & 6 & 6  \\
7  & 7 & 6 & 5 & 7 & 5  \\
8  & 6 & 7 & 7 & 7 & 5  \\
9  & 5 & 6 & 2 & 7 & 6  \\
10 & 6 & 4 & 4 & 4 & 5  \\
11 & 6 & 4 & 6 & 6 & 5  \\
\bottomrule
\end{tabular}
\caption{Resultados de la encuesta subjetiva relacionados al factor
    Representación}
\label{tab:subjetiva_conformidad_representacion}
\end{table}


La tabla~\ref{tab:subjetiva_conformidad_retroalimentacion} agrupa las respuestas
de los alumnos según la calidad de retroalimentación, son tres preguntas, las
cuales fueron descritas en~\ref{sec:sub_retroalimentacion}. 

\begin{table}[!hbt]
\centering
\begin{tabular}{@{} *{4}{r} @{}}
\toprule
& \multicolumn{3}{c}{Retroalimentación} \\
\cmidrule(lr){2-4}
Alumno &
\parbox{4cm}{Detalles de los pasos realizados incorrectamente} &
\parbox{4cm}{Suficiencia de los detalles de los pasos realizados incorrectamente} &
\parbox{4cm}{Iconos para representar el estado del jugador} \\
\midrule
1  & 3 & 2 & 7  \\
2  & 5 & 4 & 6  \\
3  & 3 & 6 & 6  \\
4  & 6 & 6 & 6  \\
5  & 6 & 1 & 6  \\
6  & 2 & 6 & 6  \\
7  & 6 & 7 & 7  \\
8  & 6 & 6 & 7  \\
9  & 6 & 6 & 7  \\
10 & 5 & 4 & 6  \\
11 & 4 & 5 & 6  \\
\bottomrule
\end{tabular}
\caption{Resultados de la encuesta subjetiva relacionados al factor
    Retroalimentación}
\label{tab:subjetiva_conformidad_retroalimentacion}
\end{table}


La tabla~\ref{tab:subjetiva_conformidad_utilidad} agrupa las respuestas de los
alumnos según la utilidad de la solución, son tres preguntas, las cuales fueron
descritas en~\ref{sec:sub_utilidad}. 


\begin{table}[!hbt]
\centering
\begin{tabular}{@{} *{6}{r} @{}}
\toprule
& \multicolumn{3}{c}{Utilidad} \\
\cmidrule(lr){2-4}
Alumno &
\parbox{4cm}{Simulación para complementar el estudio en clase y laboratorio} &
\parbox{4cm}{Simulación provee más facilidades para el estudio} &
\parbox{4cm}{Interacción con el paciente} \\
\midrule
1  & 7 & 5 & 7  \\
2  & 6 & 6 & 6  \\
3  & 6 & 6 & 6  \\
4  & 2 & 6 & 6  \\
5  & 2 & 6 & 6  \\
6  & 6 & 6 & 6  \\
7  & 7 & 6 & 7  \\
8  & 5 & 6 & 7  \\
9  & 7 & 7 & 7  \\
10 & 1 & 7 & 7  \\
11 & 6 & 4 & 5  \\
\bottomrule
\end{tabular}
\caption{Resultados de la encuesta subjetiva relacionados al factor Utilidad}
\label{tab:subjetiva_conformidad_utilidad}
\end{table}

\section{Agrupamiento de datos}

Los resultados se resumen en la tabla~\ref{tab:subjetiva_conformidad_resumen},
se muestra el número de alumno para identificar a un alumno y el promedio de sus
respuestas en la encuesta, se muestra el promedio de las mismas.

\begin{table}[!hbt]
\begin{tabular}{llllllllr}
\toprule
\textbf{\shortstack{Número de \\alumno}}         &
\begin{sideways}\textbf{Gamificación}                    \end{sideways}        &
\begin{sideways}\textbf{Exploración}                     \end{sideways}        &
\begin{sideways}\textbf{Inmersión}                       \end{sideways}        &
\begin{sideways}\textbf{Pedagogía}                       \end{sideways}        &
\begin{sideways}\textbf{Representación}                  \end{sideways}        &
\begin{sideways}\textbf{Retroalimentación}               \end{sideways}        &
\begin{sideways}\textbf{Utilidad}                        \end{sideways}        &
\textbf{\shortstack{Promedio\\de respuestas}}\\
\midrule
1              & 5 & 5 & 4 & 6 & 4 & 4 & 6 & 5 \\
2              & 6 & 6 & 6 & 6 & 5 & 5 & 6 & 6 \\
3              & 4 & 6 & 6 & 5 & 3 & 5 & 6 & 5 \\
4              & 6 & 3 & 6 & 6 & 4 & 6 & 5 & 5 \\
5              & 5 & 5 & 6 & 6 & 4 & 4 & 5 & 5 \\
6              & 6 & 5 & 5 & 5 & 6 & 5 & 6 & 5 \\
7              & 7 & 7 & 7 & 7 & 6 & 7 & 7 & 7 \\
8              & 7 & 7 & 7 & 7 & 6 & 6 & 6 & 7 \\
9              & 7 & 7 & 7 & 7 & 5 & 6 & 7 & 6 \\
10             & 6 & 6 & 5 & 7 & 5 & 5 & 5 & 5 \\
11             & 7 & 5 & 4 & 5 & 5 & 5 & 5 & 5 \\
\midrule
Promedio Total & 6 & 6 & 6 & 6 & 5 & 5 & 6 & 6 \\
\bottomrule
\end{tabular}
\caption{Resultados de la encuesta subjetiva}
\label{tab:subjetiva_conformidad_resumen}
\observacion{Utilizar aleatoriamente este}
\observacion{Y los valore reescalados?}
\end{table}

Se observa que el el puntaje más bajo en el promedio final es 5 que significa
\textit{Parcialmente de acuerdo}, y el más alto es 7, que significa
\textit{Totalmente de acuerdo}, se observa además el puntaje 6, que significa
\textit{De acuerdo}. 


Como se explica en la sección~\ref{sec:likert}, estos resultados están sujetos a
tendencias, para ello se aplica el método de doble
estandarización\cite{Pagolu2011}.

Con el resultado final de la estandarización diferenciamos cuales son los puntos
fuertes y cuales los puntos débiles de la solución propuesta con respecto a las
respuestas dadas por los usuarios. Estos valores son relativos a las respuestas
originales dadas en la encuesta, los resultados se muestran en la
tabla~\ref{tab:subjetiva_conformidad_corregida}.

\begin{table}[!hbt]
\centering
\begin{tabular}{lrrrrrrrr}
\toprule
\textbf{\shortstack{Número de \\alumno}}                                &
\begin{sideways}\textbf{Gamificación}                    \end{sideways} &
\begin{sideways}\textbf{Exploración}                     \end{sideways} &
\begin{sideways}\textbf{Inmersión}                       \end{sideways} &
\begin{sideways}\textbf{Pedagogía}                       \end{sideways} &
\begin{sideways}\textbf{Representación}                  \end{sideways} &
\begin{sideways}\textbf{Retroalimentación}               \end{sideways} &
\begin{sideways}\textbf{Utilidad}                        \end{sideways} &
\textbf{\shortstack{Promedio\\de respuestas}}\\
\midrule
1              & 0.45 & 0.55 & 0.20 & 0.63 & 0.44 & 0.41 & 0.82 & 0.47 \\
2              & 0.33 & 0.53 & 0.49 & 0.61 & 0.27 & 0.13 & 0.52 & 0.41 \\
3              & 0.17 & 0.86 & 0.87 & 0.67 & 0.13 & 0.67 & 1.00 & 0.60 \\
4              & 0.75 & 0.31 & 0.63 & 0.81 & 0.47 & 0.78 & 0.54 & 0.59 \\
5              & 0.46 & 0.58 & 0.69 & 0.67 & 0.57 & 0.50 & 0.54 & 0.58 \\
6              & 1.00 & 0.73 & 0.68 & 0.42 & 0.90 & 0.67 & 1.00 & 0.78 \\
7              & 1.00 & 0.79 & 1.00 & 0.67 & 0.50 & 0.87 & 0.78 & 0.80 \\
8              & 0.75 & 1.00 & 0.83 & 0.75 & 0.70 & 0.70 & 0.44 & 0.75 \\
9              & 0.90 & 1.00 & 0.93 & 1.00 & 0.64 & 0.92 & 1.00 & 0.90 \\
10             & 0.79 & 0.74 & 0.54 & 0.92 & 0.60 & 0.60 & 0.67 & 0.68 \\
11             & 0.75 & 0.42 & 0.08 & 0.25 & 0.60 & 0.35 & 0.25 & 0.39 \\
\midrule
\textbf{Promedio Total} & 0.67 & 0.68 & 0.63 & 0.67 & 0.53 & 0.60 & 0.69 & 0.63 \\
\bottomrule
\end{tabular}
\caption{Resultados de la encuesta subjetiva con doble estandarización}
\label{tab:subjetiva_conformidad_corregida}
\end{table}

Para obtener una mejor visión de las fortalezas y debilidades de la solución
propuesta, se presenta el gráfico de kiviat~\ref{fig:subjetiva_kiviat}, en la
misma se puede observar cuales son los puntos débiles de la solución.

\begin{figure}[!ht]
\begin{tikzpicture}[label distance=.15cm]
\tkzKiviatDiagram[radial=2,
                    lattice=2, step=2,
                    scale=2.3]%
                {Gamificación,
                 Exploración,
                 Inmersión,
                 Pedagogía,
                 Representación,
                 Retroalimentación,
                 Utilidad}
\tkzKiviatLine[thick,
                color=blue,
                mark=ball,
                ball color=red,
                mark size=1pt,opacity=.2, 
                fill=red!20](0.67,0.68,0.63,0.67,0.53,0.60,0.69)
\end{tikzpicture}
\label{fig:subjetiva_kiviat}
\caption{Gráfico de Kiviat de los factores evaluados}
\end{figure}

En el gráfico se observa en la figura, que las principales debilidades de
la solución son la representación y la retroalimentación.

Es importante notar que los datos la figura~\ref{fig:subjetiva_kiviat}
y~\ref{tab:subjetiva_conformidad_corregida} son relativas a los datos de la
tabla~\ref{tab:subjetiva_conformidad_resumen}, es decir, que la representación
es el punto más débil, pero se ve en~\ref{tab:subjetiva_conformidad_resumen}
que el valor es 5 de 7, lo que indica que es un punto aceptable, y entre
los factores analizados es el que menos aprobación obtuvo.



\subsection{Correlacion}

En la tabla~\ref{tab:subjetiva_correlacion} se muestra la correlación entre los
factores analizados, las mismas fueron calculadas según lo explicado en la
sección \todox{Agregar sección de correlación}. En resumen se pueden notar las
siguientes correlaciones:

\observacion{Algún comentario sobre esto?}
\observacion{Que dicen estas correlaciones  para que se buscan?}
\observacion{Como se interpretan y cuales son relevantes?}

\begin{itemize}
\item \textbf{Gamification y Exploración, 0.33} Correlación positiva moderada.
\item \textbf{Gamification y Pedagogía, 0.27} Correlación positiva débil.
\item \textbf{Gamificatoin e Inmersión, 0.48} Correlación positiva fuerte.
\item \textbf{Gamification y Representación, 0.81} Correlación positiva muy fuerte.
\item \textbf{Gamification y Retroalimentación, 0.66} Correlación positiva fuerte.
\item \textbf{Gamification y Utilidad, 0.26} Correlación positiva débil.
\item \textbf{Exploración y Pedagogía, 0.52} Correlación positiva fuerte.
\item \textbf{Exploración e Inmersión, 0.53} Correlación positiva fuerte.
\item \textbf{Exploración y Representación, 0.44} Correlación positiva fuerte.
\item \textbf{Exploración y Retroalimentación, 0.37} Correlación positiva moderada.
\item \textbf{Exploración y Utilidad, 0.69} Correlación positiva fuerte.
\item \textbf{Pedagogía e Inmersión, 0.57} Correlación positiva fuerte.
\item \textbf{Pedagogía y Representación, 0.23} Correlación positiva débil.
\item \textbf{Pedagogía y Retroalimentación, 0.68} Correlación positiva fuerte.
\item \textbf{Pedagogía y Utilidad, 0.54} Correlación positiva fuerte.
\item \textbf{Inmersión y Representación, 0.39} Correlación positiva moderada.
\item \textbf{Inmersión y Retroalimentación, 0.49} Correlación positiva fuerte.
\item \textbf{Inmersión y Utilidad, 0.35} Correlación positiva moderada.
\item \textbf{Representación y Retroalimentación, 0.51} Correlación positiva fuerte.
\item \textbf{Representación y Utilidad, 0.36} Correlación positiva moderada.
\item \textbf{Retroalimentación y Utilidad, 0.52} Correlación positiva fuerte.
\end{itemize}

\begin{table}[!hbt]
\centering
\begin{tabular}{l|lllllllr}
\toprule
        &
\begin{sideways}\textbf{Gamificación}                    \end{sideways}        &
\begin{sideways}\textbf{Exploración}                     \end{sideways}        &
\begin{sideways}\textbf{Inmersión}                       \end{sideways}        &
\begin{sideways}\textbf{Pedagogía}                       \end{sideways}        &
\begin{sideways}\textbf{Representación}                  \end{sideways}        &
\begin{sideways}\textbf{Retroalimentación}               \end{sideways}        &
\begin{sideways}\textbf{Utilidad}                        \end{sideways}      \\
\midrule
\textbf{Gamification}      & 1    & 0.33 & 0.27 & 0.48 & 0.81 & 0.66  & 0.27 \\
\textbf{Exploración}       & 0.33 & 1    & 0.52 & 0.53 & 0.44 & 0.37  & 0.69 \\
\textbf{Inmersión}         & 0.27 & 0.52 & 1    & 0.57 & 0.23 & 0.68  & 0.54 \\
\textbf{Pedagogía}         & 0.48 & 0.53 & 0.57 & 1    & 0.39 & 0.49  & 0.35 \\
\textbf{Representación}    & 0.81 & 0.44 & 0.23 & 0.39 & 1    & 0.51  & 0.36 \\
\textbf{Retroalimentación} & 0.66 & 0.37 & 0.68 & 0.49 & 0.51 & 1     & 0.52 \\
\textbf{Utilidad}          & 0.27 & 0.69 & 0.54 & 0.35 & 0.36 & 0.52  & 1 \\
\bottomrule
\end{tabular}
\caption{Correlación entre los factores analizados} 
\label{tab:subjetiva_correlacion}
\observacion{Todo estaba correlacionado?}
\end{table}
