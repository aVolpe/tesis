\section{Encuesta Subjetiva}

\todox{Añadir más explicación acerca de los resultados}
En el análisis de los resultados de las encuestas subjetivas, existieron alumnos
que no respondieron todas las preguntas, para tratar este tipo de casos, es
importante analizar la naturaleza del patrón de datos
faltantes\cite{carpita2011imputation}. 


\subsection{Datos faltantes}

Existen tres posibles formas de categorizar el patrón de ocurrencia de falta de
respuestas\cite{leite2010performance}
\cite{leite2010performance}\cite{tsikriktsis2005review}:


\begin{description}
    \item[Información faltante completamente aleatoria] Cuando la información
        faltante es independiente de la variable medida y de otras variables.
    \item[Información faltante aleatoria] Cuando la información faltante depende
        de otras variables, pero no de la variable en sí. 
    \item[Información faltante no aleatoria] Cuando hay una relación entre la
        información faltante y el valor de la variable.
\end{description}

Los datos muestran que la información faltante es completamente aleatoria en
relación a la variable medida y a las demás variables, de hecho, una sola
encuesta tiene información faltante, así, se establece que el tipo de
información faltante es \emph{Información faltante completamente aleatoria}.

Existen tres mecanismos\cite{tsikriktsis2005review} principales para lidiar con
información faltante, eliminación, reemplazo, y procedimientos basados en
modelo.~\cite{tsikriktsis2005review} recomienda utilizar un mecanismo de
reemplazo para escalas del tipo Likert.

Las técnicas de reemplazo se clasifican en tres grandes
grupos\cite{tsikriktsis2005review}:
\begin{enumerate*}[label=\itshape\alph*\upshape.]
\item basadas en el promedio,
\item basadas en regresión, y,
\item imputación \emph{hot deck}.
\end{enumerate*}

La sustitución basada por promedio, se divide nuevamente en tres grupos;
promedio
\begin{enumerate*}[label=\itshape\alph*\upshape.]
\item total,
\item del subgrupo, y,
\item por caso.
\end{enumerate*}
La sustitución del promedio total se realiza obteniendo el promedio de todas las
respuestas de esta pregunta, la sustitución de subgrupo es similar, solo que se
limita a aquellos sujetos del mismo subgrupo del sujeto que no respondió, y
finalmente, la sustitución por caso, es el promedio de las respuestas válidas
del sujeto.

En su resumen de las diferentes técnicas y cuando se deben utilizar,
\cite{tsikriktsis2005review}, recomienda la utilización de la sustitución basada
en promedio por caso. 

De esta forma se reemplazan los valores faltantes en la encuesta, con el
promedio del sujeto.

\subsection{Resultados}

Los resultados se resumen en la tabla~\ref{tab:subjetiva_conformidad_resumen},
se muestra el número de alumno para identificar a un alumno y el promedio de sus
respuestas en la encuesta, existen 27 preguntas y por motivos de espacio,
solamente se muestra el promedio de las mismas, los detalles completos se pueden
ver en el apéndice.

\begin{table}[!hbt]
\begin{tabular}{llllllllr}
\toprule
\textbf{\shortstack{Número de \\alumno}}         &
\begin{sideways}\textbf{Gamificación}                    \end{sideways}        &
\begin{sideways}\textbf{Exploración}                     \end{sideways}        &
\begin{sideways}\textbf{Inmersión}                       \end{sideways}        &
\begin{sideways}\textbf{Pedagogía}                       \end{sideways}        &
\begin{sideways}\textbf{Representación}                  \end{sideways}        &
\begin{sideways}\textbf{Retroalimentación}               \end{sideways}        &
\begin{sideways}\textbf{Utilidad}                        \end{sideways}        &
\textbf{\shortstack{Promedio\\de respuestas}}\\
\midrule
1              & 5 & 5 & 4 & 6 & 4 & 4 & 6 & 5 \\
2              & 6 & 6 & 6 & 6 & 5 & 5 & 6 & 6 \\
3              & 4 & 6 & 6 & 5 & 3 & 5 & 6 & 5 \\
4              & 6 & 3 & 6 & 6 & 4 & 6 & 5 & 5 \\
5              & 5 & 5 & 6 & 6 & 4 & 4 & 5 & 5 \\
6              & 6 & 5 & 5 & 5 & 6 & 5 & 6 & 5 \\
7              & 7 & 7 & 7 & 7 & 6 & 7 & 7 & 7 \\
8              & 7 & 7 & 7 & 7 & 6 & 6 & 6 & 7 \\
9              & 7 & 7 & 7 & 7 & 5 & 6 & 7 & 6 \\
10             & 6 & 6 & 5 & 7 & 5 & 5 & 5 & 5 \\
11             & 7 & 5 & 4 & 5 & 5 & 5 & 5 & 5 \\
\midrule
Promedio Total & 6 & 6 & 6 & 6 & 5 & 5 & 6 & 6 \\
\bottomrule
\end{tabular}
\caption{Resultados de la encuesta subjetiva}
\label{tab:subjetiva_conformidad_resumen}
\end{table}
\martin{~\ref{tab:subjetiva_conformidad_resumen} Es mejor poner las iniciales o un número de alumno?}

Se observa que el el puntaje más bajo en el promedio final es 5 que significa
\textit{Parcialmente de acuerdo}, y el más alto es 7, que significa
\textit{Totalmente de acuerdo}, se observa además el puntaje 6, que significa
\textit{De acuerdo}.

Como se explica en la sección~\ref{sec:likert}, estos resultados están sujetos a
tendencias, para ello se aplica el método de doble
estandarización\cite{Pagolu2011}, los resultados se muestran en la
tabla~\ref{tab:subjetiva_conformidad_corregida}

\begin{table}[!hbt]
\centering
\begin{tabular}{lrrrrrrrr}
\toprule
\textbf{\shortstack{Número de \\alumno}}         &
\begin{sideways}\textbf{Gamificación}                    \end{sideways}        &
\begin{sideways}\textbf{Exploración}                     \end{sideways}        &
\begin{sideways}\textbf{Inmersión}                       \end{sideways}        &
\begin{sideways}\textbf{Pedagogía}                       \end{sideways}        &
\begin{sideways}\textbf{Representación}                  \end{sideways}        &
\begin{sideways}\textbf{Retroalimentación}               \end{sideways}        &
\begin{sideways}\textbf{Utilidad}                        \end{sideways}        &
\textbf{\shortstack{Promedio\\de respuestas}}\\
\midrule
1              & 0.45 & 0.55 & 0.20 & 0.63 & 0.44 & 0.41 & 0.82 & 0.47 \\
2              & 0.33 & 0.53 & 0.49 & 0.61 & 0.27 & 0.13 & 0.52 & 0.41 \\
3              & 0.17 & 0.86 & 0.87 & 0.67 & 0.13 & 0.67 & 1.00 & 0.60 \\
4              & 0.75 & 0.31 & 0.63 & 0.81 & 0.47 & 0.78 & 0.54 & 0.59 \\
5              & 0.46 & 0.58 & 0.69 & 0.67 & 0.57 & 0.50 & 0.54 & 0.58 \\
6              & 1.00 & 0.73 & 0.68 & 0.42 & 0.90 & 0.67 & 1.00 & 0.78 \\
7              & 1.00 & 0.79 & 1.00 & 0.67 & 0.50 & 0.87 & 0.78 & 0.80 \\
8              & 0.75 & 1.00 & 0.83 & 0.75 & 0.70 & 0.70 & 0.44 & 0.75 \\
9              & 0.90 & 1.00 & 0.93 & 1.00 & 0.64 & 0.92 & 1.00 & 0.90 \\
10             & 0.79 & 0.74 & 0.54 & 0.92 & 0.60 & 0.60 & 0.67 & 0.68 \\
11             & 0.75 & 0.42 & 0.08 & 0.25 & 0.60 & 0.35 & 0.25 & 0.39 \\
\midrule
\textbf{Promedio Total} & 0.67 & 0.68 & 0.63 & 0.67 & 0.53 & 0.60 & 0.69 & 0.63 \\
\bottomrule
\end{tabular}
\caption{Resultados de la encuesta subjetiva con doble estandarización}
\label{tab:subjetiva_conformidad_corregida}
\end{table}

Para obtener una mejor visión de las fortalezas y debilidades de la solución
propuesta, se presenta la estrella de kiviat~\ref{fig:subjetiva_kiviat}, en la
misma se puede observar cuales son los puntos débiles de la solución.

\begin{figure}[!ht]
\begin{tikzpicture}[label distance=.15cm]
\tkzKiviatDiagram[radial=2,
                    lattice=2,
                    step=2,
                    scale=2.3]%
                {Gamificación,
                Exploración,
                Inmersión,
                Pedagogía,
                Representación,
                Retroalimentación,
                Utilidad }
\tkzKiviatLine[thick,
                color=blue,
                mark=ball,
                ball color=red,
                mark size=1pt,opacity=.2, 
                fill=red!20](0.67,0.68,0.63,0.67,0.53,0.60,0.69)
\end{tikzpicture}
\label{fig:subjetiva_kiviat}
\caption{Estrella de Kiviat de los aspectos simulados}
\end{figure}
