%! TEX root = ../main.tex
\section{Encuesta Subjetiva}
\label{sec:res_subjetiva}

%\replantear{En el análisis de los resultados de las encuestas subjetivas,
    %existieron alumnos que no respondieron todas
    %las preguntas, para tratar este tipo de casos, es importante analizar la
    %naturaleza del patrón de datos faltantes\cite{carpita2011imputation}.}

La información recogida por la encuesta muestra que hay datos faltantes, como se
explico en~\ref{sec:informacion_faltante}, esta información faltante es
completamente aleatoria en relación a la variable medida y a las demás
variables, de hecho, una sola encuesta tiene información faltante, así, se
establece que el tipo de información faltante es \emph{Información faltante
    completamente aleatoria}.

En su resumen de las diferentes técnicas y cuando se deben utilizar,
\cite{tsikriktsis2005review}, recomienda la utilización de la sustitución basada
en promedio por caso. Así, se completan los valores faltantes con el promedio de
respuestas completadas por el usuario.

\subsection{Resultados}

Se presentan a continuación los resultados de las encuestas, agrupados por los
factores definidos en~\ref{sec:variables}.

La tabla~\ref{tab:subjetiva_conformidad_exploracion} nos muestra las respuestas
de los alumnos a las preguntas relacionadas al factor exploración, son cuatro
preguntas, las cuales fueron descritas en~\ref{sec:sub_exploracion}. 

\begin{table}[H]
\centering
\begin{tabular}{@{} *{5}{r} @{}}
\toprule
& \multicolumn{4}{c}{Exploración} \\
\cmidrule(lr){2-5}
Alumno &
\parbox{3cm}{Funciones realizadas por los elementos del juego} &
\parbox{3cm}{Aleatoriedad para afianzar conocimientos} &
\parbox{2.5cm}{Aleatoriedad para representar realismo} &
\parbox{2.5cm}{Facilidad de uso}  \\
\midrule
1         & 2   & 6   & 5   & 6  \\
2         & 6   & 6   & 4   & 6  \\
3         & 3   & 3   & 5   & 5  \\
4         & 6   & 6   & 6   & 6  \\
5         & 6   & 6   & 2   & 5  \\
6         & 6   & 6   & 6   & 6  \\
7         & 7   & 7   & 7   & 7  \\
8         & 6   & 6   & 7   & 7  \\
9         & 5   & 7   & 7   & 7  \\
10        & 6   & 7   & 6   & 6  \\
11        & 7   & 6   & 7   & 6  \\
\midrule
\textbf{Promedio}  & \textbf{5}   & \textbf{6}   & \textbf{6}   & \textbf{6} \\
\bottomrule
\end{tabular}
\caption{Resultados de la encuesta subjetiva relacionados al factor Exploración}
\label{tab:subjetiva_conformidad_exploracion}
\end{table}

La tabla~\ref{tab:subjetiva_conformidad_representacion} agrupa las respuestas de
los alumnos según la calidad de presentación, son cinco preguntas, las cuales
fueron descritas en~\ref{sec:sub_representacion}. 

\begin{table}[H]
\centering
\begin{tabular}{@{} *{6}{r} @{}}
\toprule
& \multicolumn{5}{c}{Representación} \\
\cmidrule(lr){2-6}
& \multicolumn{3}{c}{Respuestas del paciente} & & \\
\cmidrule(lr){2-4}
Alumno &
\parbox{2.5cm}{Movimientos motrices del paciente} &
\parbox{2.5cm}{Movimientos oculares del paciente} &
\parbox{2.5cm}{Reacción verbal del paciente} &
\parbox{2.5cm}{Distinción entre los estados del paciente} &
\parbox{2.5cm}{Acciones con las herramientas} \\
\midrule
1  & 6 & 6 & 2 & 5 & 2  \\
2  & 4 & 5 & 5 & 6 & 4  \\
3  & 5 & 3 & 3 & 3 & 3  \\
4  & 6 & 5 & 2 & 4 & 2  \\
5  & 2 & 2 & 6 & 6 & 6  \\
6  & 6 & 4 & 6 & 6 & 6  \\
7  & 7 & 6 & 5 & 7 & 5  \\
8  & 6 & 7 & 7 & 7 & 5  \\
9  & 5 & 6 & 2 & 7 & 6  \\
10 & 6 & 4 & 4 & 4 & 5  \\
11 & 6 & 4 & 6 & 6 & 5  \\
\midrule
\textbf{Promedio}  & \textbf{5} & \textbf{5} & \textbf{4} & \textbf{6} & \textbf{4} \\
\bottomrule
\end{tabular}
\caption{Resultados de la encuesta subjetiva relacionados al factor
    Representación}
\label{tab:subjetiva_conformidad_representacion}
\end{table}

La tabla~\ref{tab:subjetiva_conformidad_motivacion} muestra las respuestas de
los alumnos a las preguntas relacionadas al factor \textit{Motivación}, son
cinco preguntas, las cuales fueron descritas en~\ref{sec:sub_motivacion}. 

\begin{table}[H]
\centering
\begin{tabular}{@{} *{5}{r} @{}}
\toprule
& \multicolumn{4}{c}{Motivación} \\
\cmidrule(lr){2-5}
Alumno &
\parbox{2.5cm}{Motivación del puntaje} &
\parbox{2.5cm}{Importancia del puntaje} &
\parbox{3cm}{Socialización de los puntajes} &
\parbox{3cm}{Medición del tiempo} \\
\midrule
1  & 6 & 4 & 4 & 7  \\
2  & 7 & 4 & 6 & 6  \\
3  & 6 & 6 & 5 & 6  \\
4  & 1 & 4 & 6 & 1  \\
5  & 2 & 2 & 7 & 7  \\
6  & 6 & 5 & 4 & 6  \\
7  & 7 & 7 & 6 & 7  \\
8  & 7 & 7 & 7 & 7  \\
9  & 7 & 7 & 7 & 7  \\
10 & 7 & 4 & 5 & 7  \\
11 & 5 & 4 & 5 & 6  \\
\midrule
\textbf{Promedio}  & \textbf{6}   & \textbf{5}   & \textbf{6}   & \textbf{6} \\
\bottomrule
\end{tabular}
\caption{Resultados de la encuesta subjetiva relacionados al factor motivación}
\label{tab:subjetiva_conformidad_motivacion}
\end{table}

La tabla~\ref{tab:subjetiva_conformidad_inmersion} muestra las respuestas de
los alumnos a las preguntas relacionadas al factor \textit{Inmersión}, son
cinco preguntas, las cuales fueron descritas en~\ref{sec:sub_inmersion}. 

\begin{table}[H]
\centering
\begin{tabular}{@{} *{6}{r} @{}}
\toprule
& \multicolumn{5}{c}{Inmersión} \\
\cmidrule(lr){2-6}
Alumno &
\parbox{2.5cm}{Escenografía para entrar en ambiente} &
\parbox{2.5cm}{Juegos cortos como ayuda para la repetición} &
\parbox{2.5cm}{Gráficos en tres dimensiones para entender el entorno} &
\parbox{2.5cm}{Realismo a través de ordenes verbales} &
\parbox{2.5cm}{Simulación como herramienta} \\
\midrule
1  & 4 & 6 & 4 & 5 & 3  \\
2  & 6 & 6 & 6 & 6 & 6  \\
3  & 6 & 6 & 6 & 5 & 6  \\
4  & 4 & 6 & 7 & 5 & 6  \\
5  & 6 & 6 & 5 & 6 & 6  \\
6  & 6 & 6 & 6 & 4 & 4  \\
7  & 7 & 7 & 7 & 7 & 7  \\
8  & 6 & 7 & 7 & 7 & 7  \\
9  & 6 & 7 & 7 & 7 & 7  \\
10 & 6 & 3 & 4 & 6 & 6  \\
11 & 5 & 3 & 5 & 5 & 4  \\
\midrule
\textbf{Promedio}  & \textbf{6} & \textbf{6} & \textbf{6} & \textbf{6} & \textbf{6} \\
\bottomrule
\end{tabular}
\caption{Resultados de la encuesta subjetiva relacionados al factor Inmersión}
\label{tab:subjetiva_conformidad_inmersion}
\end{table}

La tabla~\ref{tab:subjetiva_conformidad_utilidad} agrupa las respuestas de los
alumnos según la utilidad de la solución, son tres preguntas, las cuales fueron
descritas en~\ref{sec:sub_utilidad}. 


\begin{table}[H]
\centering
\begin{tabular}{@{} *{6}{r} @{}}
\toprule
& \multicolumn{3}{c}{Utilidad} \\
\cmidrule(lr){2-4}
Alumno &
\parbox{4cm}{Simulación para complementar el estudio en clase y laboratorio} &
\parbox{4cm}{Simulación como proveedor de facilidades para el estudio} &
\parbox{4cm}{Interacción con el paciente} \\
\midrule
1  & 7 & 5 & 7  \\
2  & 6 & 6 & 6  \\
3  & 6 & 6 & 6  \\
4  & 2 & 6 & 6  \\
5  & 2 & 6 & 6  \\
6  & 6 & 6 & 6  \\
7  & 7 & 6 & 7  \\
8  & 5 & 6 & 7  \\
9  & 7 & 7 & 7  \\
10 & 1 & 7 & 7  \\
11 & 6 & 4 & 5  \\
\midrule
\textbf{Promedio}  & \textbf{5} & \textbf{6} & \textbf{6} \\
\bottomrule
\end{tabular}
\caption{Resultados de la encuesta subjetiva relacionados al factor Utilidad}
\label{tab:subjetiva_conformidad_utilidad}
\end{table}

La tabla~\ref{tab:subjetiva_conformidad_retroalimentacion} agrupa las respuestas
de los alumnos según la calidad de retroalimentación, son tres preguntas, las
cuales fueron descritas en~\ref{sec:sub_retroalimentacion}. 

\begin{table}[H]
\centering
\begin{tabular}{@{} *{4}{r} @{}}
\toprule
& \multicolumn{3}{c}{Retroalimentación} \\
\cmidrule(lr){2-4}
Alumno &
\parbox{4cm}{Detalles de los pasos realizados incorrectamente} &
\parbox{4cm}{Retroalimentación suficiente respecto a los pasos realizados} &
\parbox{4cm}{Representación iconográfica de conceptos y acciones en la \Gls{gui}} \\
\midrule
1  & 3 & 2 & 7  \\
2  & 5 & 4 & 6  \\
3  & 3 & 6 & 6  \\
4  & 6 & 6 & 6  \\
5  & 6 & 1 & 6  \\
6  & 2 & 6 & 6  \\
7  & 6 & 7 & 7  \\
8  & 6 & 6 & 7  \\
9  & 6 & 6 & 7  \\
10 & 5 & 4 & 6  \\
11 & 4 & 5 & 6  \\
\midrule
\textbf{Promedio}  & \textbf{5} & \textbf{5} & \textbf{6} \\
\bottomrule
\end{tabular}
\caption{Resultados de la encuesta subjetiva relacionados al factor
    Retroalimentación}
\label{tab:subjetiva_conformidad_retroalimentacion}
\end{table}

La tabla~\ref{tab:subjetiva_conformidad_pedagogia} agrupa las respuestas de los
alumnos según el factor pedagógico, son tres preguntas, las cuales fueron
descritas en~\ref{sec:sub_pedagogia}. 

%\observacion{En el capítulo anterior faltaba muchos más, donde están?}
\begin{table}[H]
\centering
\begin{tabular}{@{} *{4}{r} @{}}
\toprule
& \multicolumn{3}{c}{Pedagogía} \\
\cmidrule(lr){2-4}
Alumno &
\parbox{4cm}{Potencial para memorizar y comprender el procedimiento} &
\parbox{4cm}{Falta de pistas como ayuda al aprendizaje} &
\parbox{4cm}{Suficiencia de los botones que indican acciones} \\
\midrule
1  & 6 & 6 & 6  \\
2  & 6 & 6 & 7  \\
3  & 4 & 6 & 6  \\
4  & 6 & 7 & 6  \\
5  & 7 & 5 & 6  \\
6  & 4 & 4 & 6  \\
7  & 7 & 6 & 7  \\
8  & 6 & 7 & 7  \\
9  & 7 & 7 & 7  \\
10 & 6 & 7 & 7  \\
11 & 5 & 6 & 5  \\
\midrule
\textbf{Promedio}  & \textbf{6} & \textbf{6} & \textbf{6} \\
\bottomrule
\end{tabular}
\caption{Resultados de la encuesta subjetiva relacionados al factor Pedagogía}
\label{tab:subjetiva_conformidad_pedagogia}
\end{table}

\subsection{Agrupamiento de datos}

Los resultados se resumen en la tabla~\ref{tab:subjetiva_conformidad_resumen},
se muestra el número de alumno para identificar a un alumno y el promedio de sus
respuestas en la encuesta, se muestra el promedio de las mismas.

\begin{table}[H]
\begin{tabular}{llllllllr}
\toprule
\textbf{\shortstack{Número de \\alumno}}         &
\begin{sideways}\textbf{Motivación}                    \end{sideways}        &
\begin{sideways}\textbf{Exploración}                     \end{sideways}        &
\begin{sideways}\textbf{Inmersión}                       \end{sideways}        &
\begin{sideways}\textbf{Pedagogía}                       \end{sideways}        &
\begin{sideways}\textbf{Representación}                  \end{sideways}        &
\begin{sideways}\textbf{Retroalimentación}               \end{sideways}        &
\begin{sideways}\textbf{Utilidad}                        \end{sideways}        &
\textbf{\shortstack{Promedio\\de respuestas}}\\
\midrule
1              & 5 & 5 & 4 & 6 & 4 & 4 & 6 & 5 \\
2              & 6 & 6 & 6 & 6 & 5 & 5 & 6 & 6 \\
3              & 4 & 6 & 6 & 5 & 3 & 5 & 6 & 5 \\
4              & 6 & 3 & 6 & 6 & 4 & 6 & 5 & 5 \\
5              & 5 & 5 & 6 & 6 & 4 & 4 & 5 & 5 \\
6              & 6 & 5 & 5 & 5 & 6 & 5 & 6 & 5 \\
7              & 7 & 7 & 7 & 7 & 6 & 7 & 7 & 7 \\
8              & 7 & 7 & 7 & 7 & 6 & 6 & 6 & 7 \\
9              & 7 & 7 & 7 & 7 & 5 & 6 & 7 & 6 \\
10             & 6 & 6 & 5 & 7 & 5 & 5 & 5 & 5 \\
11             & 7 & 5 & 4 & 5 & 5 & 5 & 5 & 5 \\
\midrule
Promedio Total & 6 & 6 & 6 & 6 & 5 & 5 & 6 & 6 \\
\bottomrule
\end{tabular}
\caption{Resultados de la encuesta subjetiva}
\label{tab:subjetiva_conformidad_resumen}
\end{table}

Se observa que el el puntaje más bajo en el promedio final es 5 que significa
\textit{Parcialmente de acuerdo}, y el más alto es 7, que significa
\textit{Totalmente de acuerdo}, se observa además el puntaje 6, que significa
\textit{De acuerdo}. 


Como se explica en la sección~\ref{sec:likert}, estos resultados están sujetos a
tendencias, para ello se aplica el método de doble
estandarización\cite{Pagolu2011}.

Con el resultado final de la estandarización diferenciamos cuales son los puntos
fuertes y cuales los puntos débiles de la solución propuesta con respecto a las
respuestas dadas por los usuarios. Estos valores son relativos a las respuestas
originales dadas en la encuesta, los resultados se muestran en la
tabla~\ref{tab:subjetiva_conformidad_corregida}.

\begin{table}[H]
\centering
\begin{tabular}{lrrrrrrrr}
\toprule
\textbf{\shortstack{Número de \\alumno}}                                &
\begin{sideways}\textbf{Motivación}                    \end{sideways} &
\begin{sideways}\textbf{Exploración}                     \end{sideways} &
\begin{sideways}\textbf{Inmersión}                       \end{sideways} &
\begin{sideways}\textbf{Pedagogía}                       \end{sideways} &
\begin{sideways}\textbf{Representación}                  \end{sideways} &
\begin{sideways}\textbf{Retroalimentación}               \end{sideways} &
\begin{sideways}\textbf{Utilidad}                        \end{sideways} &
\textbf{\shortstack{Promedio\\de respuestas}}\\
\midrule
1              & 0.45 & 0.55 & 0.20 & 0.63 & 0.44 & 0.41 & 0.82 & 0.47 \\
2              & 0.33 & 0.53 & 0.49 & 0.61 & 0.27 & 0.13 & 0.52 & 0.41 \\
3              & 0.17 & 0.86 & 0.87 & 0.67 & 0.13 & 0.67 & 1.00 & 0.60 \\
4              & 0.75 & 0.31 & 0.63 & 0.81 & 0.47 & 0.78 & 0.54 & 0.59 \\
5              & 0.46 & 0.58 & 0.69 & 0.67 & 0.57 & 0.50 & 0.54 & 0.58 \\
6              & 1.00 & 0.73 & 0.68 & 0.42 & 0.90 & 0.67 & 1.00 & 0.78 \\
7              & 1.00 & 0.79 & 1.00 & 0.67 & 0.50 & 0.87 & 0.78 & 0.80 \\
8              & 0.75 & 1.00 & 0.83 & 0.75 & 0.70 & 0.70 & 0.44 & 0.75 \\
9              & 0.90 & 1.00 & 0.93 & 1.00 & 0.64 & 0.92 & 1.00 & 0.90 \\
10             & 0.79 & 0.74 & 0.54 & 0.92 & 0.60 & 0.60 & 0.67 & 0.68 \\
11             & 0.75 & 0.42 & 0.08 & 0.25 & 0.60 & 0.35 & 0.25 & 0.39 \\
\midrule
\textbf{Promedio Total} & 0.67 & 0.68 & 0.63 & 0.67 & 0.53 & 0.60 & 0.69 & 0.63 \\
\bottomrule
\end{tabular}
\caption{Resultados de la encuesta subjetiva con doble estandarización}
\label{tab:subjetiva_conformidad_corregida}
\end{table}

Es importante notar que los datos la
tabla~\ref{tab:subjetiva_conformidad_corregida} son relativas a los datos de la
tabla~\ref{tab:subjetiva_conformidad_resumen}, es decir, que la representación
es el punto más débil, aún así, se ve que
en~\ref{tab:subjetiva_conformidad_resumen} que el valor es $5$ de $7$, lo que
indica que es un punto aceptable, y entre los factores analizados es el que
menos aprobación obtuvo.


Con la información obtenida, es posible \emph{Validar las hipótesis asumidas
    durante el desarrollo de la solución}, el cual es uno de los objetivos de
este capítulo. En la tabla~\ref{tab:resultado_resumen_hipotesis} se observa la
opinion de los alumnos con respecto a las hipótesis asumidas
en~\ref{sec:hipotesis}. Se observa una aceptación a las hipótesis asumidas.

\begin{table}[!hbt]
\centering
\begin{tabular}{lcr}
\toprule
Hipótesis                        & Promedio Subjetiva      & Promedio estandarizado \\
\midrule
Comandos de voz con interfaz     & De acuerdo              & $0,55$ \\
Extracción uniforme de elementos & Parcialmente de acuerdo & $0,65$ \\
Acciones de bioseguridad         & De acuerdo              & $0,58$ \\
Representación iconográfica      & Parcialmente de acuerdo & $0,53$ \\
Factores motivadores             & De acuerdo              & $0,65$ \\
Falta de pistas                  & De acuerdo              & $0,61$ \\
\bottomrule
\end{tabular}
\caption{Hipótesis con su aceptación}\label{tab:resultado_resumen_hipotesis}
\end{table}

Adicionalmente, se puede \emph{Evaluar los puntos fuertes y débiles de la
    solución}, utilizando los datos con doble estandarización de la
tabla~\ref{tab:subjetiva_conformidad_corregida}, se crea la
tabla~\ref{tab:resultado_resumen_aspectos_aceptacion}, donde se observa la
apreciación de los usuarios por cada aspecto estudiado.

\begin{table}[!hbt]
\centering
\begin{tabular}{lcr}
\toprule
Hipótesis         & Promedio Subjetiva      & Promedio estandarizado \\
\midrule
Motivación        & De acuerdo              & $0.67$  \\
Exploración       & De acuerdo              & $0.68$  \\
Inmersión         & De acuerdo              & $0.63$  \\
Pedagogía         & De acuerdo              & $0.67$  \\
Representación    & Parcialmente de acuerdo & $0.53$  \\
Retroalimentación & Parcialmente de acuerdo & $0.60$  \\
Utilidad          & De acuerdo              & $0.69$  \\
\bottomrule
\end{tabular}
\caption{Aceptación por aspecto de la solución}
\label{tab:resultado_resumen_aspectos_aceptacion}
\end{table}

Para  obtener una mejor visión de las fortalezas y debilidades de la solución
propuesta, se presenta el gráfico de \emph{kiviat}~\ref{fig:subjetiva_kiviat},
en la misma se puede observar cuales son los puntos débiles de la solución.

\begin{figure}[!ht]
\begin{tikzpicture}[label distance=.15cm]
\tkzKiviatDiagram[radial=2,
                    lattice=2, step=2,
                    scale=2.3]%
                {Motivación,
                 Exploración,
                 Inmersión,
                 Pedagogía,
                 Representación,
                 Retroalimentación,
                 Utilidad}
\tkzKiviatLine[thick,
                color=blue,
                mark=ball,
                ball color=red,
                mark size=1pt,opacity=.2, 
                fill=red!20](0.67,0.68,0.63,0.67,0.53,0.60,0.69)
\end{tikzpicture}
\label{fig:subjetiva_kiviat}
\caption{Gráfico de Kiviat de los factores evaluados}
\end{figure}

Se observa que las principales debilidades de la solución son la representación
y la retroalimentación, y las fortalezas la utilidad, pedagogía, exploración, y
la motivación.

\subsection{Preguntas abiertas}
\label{sec:res_subjetiva_abiertas}

En la parte final de la encuesta que completaron los alumnos, que formaron parte
de la prueba, cuenta con preguntas abiertas, donde los alumnos expresaron sus
opiniones sobre los aspectos que rodean al uso de este tipo de soluciones al
aprendizaje de enfermería.


\begin{itemize}
    \item El $100\%$ de los alumnos menciono que este tipo de soluciones son
        beneficiosas para el aprendizaje de procedimientos de enfermería.
    \item El $64\%$ de los alumnos menciono que la principal dificultad para
        utilizar la solución es el factor tiempo.
    \item El $45\%$ de los alumnos menciono que la solución esta completa,
        mientras que el $18\%$ sugirió más elementos e interacción con el
        paciente.
\end{itemize}


%\begin{description}[style=unboxed]
    %\item[¿En qué cree beneficia y en que perjudica el uso de estas herramientas
        %como apoyo al aprendizaje?] La mayoría menciono que beneficia poder
        %realizar la práctica en un entorno seguro, y que permite comprender mejor
        %el procedimiento.

        %En cuanto a las desventajas, puede provocar adicción, y no puede ser
        %comparado con un escenario real.

    %\item[¿Cree que son necesarios métodos para complementar al
        %laboratorio/aula?, si es sí, ¿Que tipo de herramientas podrían ayudar?]
        %Se menciono que este tipo de solución es útil para el aprendizaje,
        %algunos alumnos reportaron que no son necesarios métodos adicionales

    %\item[¿Que factores le impidió utilizar más a menudo la aplicación?] el
        %principal factor que no permitió la utilización frecuente de la
        %solución, es el tiempo disponible, se menciona la cantidad de horas
        %requeridas por la facultad y el campo de prácticas.

    %\item[¿Qué le hubiera gustado que incluyera la aplicación?, con respecto al
        %contenido de las escenas, escenas simuladas y aspectos generales] Se
        %mencionan otros procedimientos, una mejor visualización del enfermero,
        %más interacción con el paciente, y en cuanto al entorno, más elementos.

        %Es importante notar que casi la mitad de los alumnos menciono que los
        %escenarios están completos y no necesitan ser expandidos.
%\end{description}

%se observa en general que la opinion de los alumnos es positiva en cuanto a la
%utilización de este tipo de soluciones, y que una mayor cantidad de
%procedimientos podría ayudar a la utilización más frecuente de la misma.

%otro factor importante es el tiempo que disponen los alumnos, si bien uno de los
%objetivos de este trabajo es que los usuarios puedan utilizar la solución en
%cualquier momento, esto no se cumple para individuos que están condicionados por
%factores externos, por ejemplo, el miedo a utilizar un dispositivo móvil en el
%transporte público, y exámenes periódicos.

Con esta información se puede \emph{determinar el nivel de aceptación de la
    solución}, se observa que el $100\%$ de los alumnos cree que es beneficioso
contar con este tipo de soluciones.
