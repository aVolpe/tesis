%! TEX root = ../main.tex


\section{Encuesta de Ubicación}
\label{sec:res_UBICACION}

Como se indicó en la sección~\ref{sec:ubicacion}, se agrupa a los alumnos
encuestados de acuerdo a las características de sus dispositivos móviles y del
acceso a internet.

En la figura~\ref{fig:ubicacion_acceso_internet} se puede observar que de 93
encuestados, el $94,6\%$ tiene acceso a internet al menos en algún momento y que
solo el $5.4\%$ no tiene acceso a internet en sus dispositivos móviles.

\begin{figure}[ht!]
\centering
\includegraphics[scale=0.8]{resultados/imagenes/ubicacion_acceso_internet.png}
\caption{Acceso a internet desde dispositivos móviles}
\label{fig:ubicacion_acceso_internet}
\observacion{no habia otras caracteristicas}
\end{figure}

Por otro lado, en la figura\ref{fig:ubicacion_sistemas_operativos} se muestra
los sistemas operativos móviles utilizados por los usuarios encuestados. Se
puede observar que Android lidera con un $61.3\%$, le sigue Windows Phone con un
$12.9\%$.

\begin{figure}[ht!]
\centering
\includegraphics[scale=0.8]{resultados/imagenes/ubicacion_sistemas_operativos.png}
\caption{Sistemas operativos móviles utilizados}
\label{fig:ubicacion_sistemas_operativos}
\observacion{Meter a la par de otros}
\end{figure}

Por último, se discrimina a los encuestados para determinar cuantos de ellos
tiene dispositivos móviles que cumplen los requisitos mínimos para utilizar la
solución propuesta según lo descrito en la sección~\ref{sec:ubicacion}. En la
figura~\ref{fig:ubicacion_requisitos_minimos} se puede observar que el $18,3\%$
de los encuestados cumplen con los requisitos.

\begin{figure}[ht!]
\centering
\includegraphics[scale=0.8]{resultados/imagenes/ubicacion_requisitos_minimos.png}
\caption{Dispositivos que cumplen con los requisitos mínimos para la prueba}
\label{fig:ubicacion_requisitos_minimos}
\observacion{Que conclusiones podría quitar de esto?}
\end{figure}

\observacion{Acercar más los gráficos}
