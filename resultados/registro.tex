\section{Registro de actividad}

\martin{Ponemos a los alumnos que no jugaron con 0 partidas y 0 segundos, o
    los omitimos?}

Las actividad de los usuarios es registrada y almacenada para su análisis, a
continuación se presentan los resultados de ese análisis, el mismo fue descrito
en~\ref{sec:registro}.


\begin{table}[!hbt]
\centering
\begin{tabular}{lrrrrrrrr}
\toprule
                   & \multicolumn{2}{c}{Extracción de sangre} \\
                   \cmidrule(lr){2-3} 
Número de alumno   & Sesiones jugadas                            & Tiempo jugado \\
\midrule
 1       & 5  & 1202 \\
 2       & 19 & 2507 \\
 4       & 5  & 398  \\
 5       & 6  & 768  \\
 6       & 17 & 2371 \\
 7       & 7  & 707  \\
 9       & 1  & 126  \\
10       & 8  & 960  \\
\midrule
Total:   & 68 & 9039 \\
\bottomrule
\end{tabular}
\caption{Número de partidas y tiempo total por alumno, en la escena de
    extracción de sangre.}
\label{tab:log_hemocultivo_partida}
\end{table}

La cantidad de partidas jugadas por usuario, se ven en la
tabla~\ref{tab:log_hemocultivo_partida}, se observa que existen 3 alumnos que
participaron del experimento o no se registro su actividad.

Los registros pueden no ser registrados sí
\begin{enumerate*}[label=\itshape\alph*\upshape)]
    \item El usuario utilizo la solución, pero no envió los datos y, luego
        desinstalo la solución o borro los datos de la misma, o,
    \item El usuario no utilizo la solución.
\end{enumerate*}

En la tabla~\ref{tab:log_glasgow_random_partida}, se observa la cantidad de
sesiones y tiempo total por alumno, en la escena de \textit{Glasgow}, en modo de
evaluación.

Se observa que $5$ alumnos participaron en $22$ sesiones, en total jugaron
$1768$ segundos.

\begin{table}[!hbt]
\centering
\begin{tabular}{lrrrrrrrr}
\toprule
& \multicolumn{2}{c}{Glasgow (Evaluación)} \\
                   \cmidrule(lr){2-3} 
Número de alumno   & Sesiones jugadas                            & Tiempo jugado \\
\midrule
1     & 4  & 211 \\
2     & 8  & 738 \\
4     & 3  & 132 \\
6     & 1  & 97  \\
7     & 6  & 590 \\
\midrule
Total & 22 & 1768 \\
\bottomrule
\end{tabular}
\caption{Número de partidas y tiempo total por alumno, en la escena
    \textit{Glasgow}, en modo evaluación}
\label{tab:log_glasgow_random_partida}
\end{table}


\begin{table}[!hbt]
\centering
\begin{tabular}{lrrrrrrrr}
\toprule
& \multicolumn{2}{c}{Glasgow (Exploración)} \\
                   \cmidrule(lr){2-3} 
Número de alumno   & Sesiones jugadas                            & Tiempo jugado \\
\midrule
1        & 2 & 79 \\
2        & 3 & 80 \\
4        & 3 & 89 \\
6        & 1 & 79 \\
\midrule
Total:   & 9 & 327 \\
\bottomrule
\end{tabular}
\caption{Número de partidas y tiempo total por alumno, en la escena
    \textit{Glasgow}, en modo exploración}
\label{tab:log_glasgow_custom_partida}
\end{table}
