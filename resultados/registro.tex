%! TEX root = ../main.tex
\section{Registro de actividad}

Las actividad de los usuarios es registrada y almacenada para su análisis, a
continuación se presentan los resultados de ese análisis, el mismo fue descrito
en~\ref{sec:registro}, en la tabla~\ref{tab:log_total} se observa un resumen del
experimento, en cuanto a tiempo, partidas y acciones.


\begin{table}[H]
\centering
\begin{tabular}{lrrrrrrrr}
\toprule
\textbf{Variable}                         & \textbf{Valor} \\
\midrule
Tiempo total                     & 11134 \\
Partidas                         & 99 \\
Acciones                         & 2944 \\
Promedio de tiempo por partida   & 112 \\
Promedio de acciones por partida & 30 \\
Promedio de partidas por usuario & 12,37 \\
Usuarios                         & 8 \\
\bottomrule
\end{tabular}
\caption{Resumen de la información extraída del registro de actividades.}
\label{tab:log_total}
\end{table}

\begin{table}[H]
\centering
\begin{tabular}{lrrrrrrrr}
\toprule
& \multicolumn{2}{c}{Extracción de sangre} \\
\cmidrule(lr){2-3} 
Alumno   & Sesiones jugadas & Tiempo jugado \\
\midrule
 1       & 5                & 1202 \\
 2       & 19               & 2507 \\
 4       & 5                & 398  \\
 5       & 6                & 768  \\
 6       & 17               & 2371 \\
 7       & 7                & 707  \\
 9       & 1                & 126  \\
10       & 8                & 960  \\
\midrule
Total   & 68               & 9039 \\
\bottomrule
\end{tabular}
\caption{Número de partidas y tiempo total por alumno en segundos, en la escena
    de extracción de sangre.}
\label{tab:log_hemocultivo_partida}
\end{table}

La cantidad de partidas jugadas por usuario, se ven en la
tabla~\ref{tab:log_hemocultivo_partida}, se observa que existen $3$ alumnos que no
participaron de la prueba o no se registro su actividad.

Los registros pueden no ser registrados sí
\begin{enumerate*}[label=\itshape\alph*\upshape)]
    \item el usuario utilizo la solución, no envió los datos y, luego
        desinstalo la solución o borro los datos de la misma, o,
    \item el usuario no utilizo la solución.
\end{enumerate*}

En la tabla~\ref{tab:log_glasgow_random_partida}, se observa la cantidad de
sesiones y tiempo total por alumno, en la escena de \textit{Glasgow}, en modo de
evaluación. Se observa que $5$ alumnos participaron en $22$ sesiones, en total
jugaron $1768$ segundos.

\begin{table}[H]
\centering
\begin{tabular}{lrrrrrrrr}
\toprule
& \multicolumn{2}{c}{Glasgow (Evaluación)} \\
                   \cmidrule(lr){2-3} 
Número de alumno   & Sesiones jugadas                            & Tiempo jugado \\
\midrule
1     & 4  & 211 \\
2     & 8  & 738 \\
4     & 3  & 132 \\
6     & 1  & 97  \\
7     & 6  & 590 \\
\midrule
Total & 22 & 1768 \\
\bottomrule
\end{tabular}
\caption{Número de partidas y tiempo total por alumno en segundos, en la escena
    \textit{Glasgow}, en modo evaluación}
\label{tab:log_glasgow_random_partida}
\end{table}


\begin{table}[H]
\centering
\begin{tabular}{lrrrrrrrr}
\toprule
& \multicolumn{2}{c}{Glasgow (Exploración)} \\
                   \cmidrule(lr){2-3} 
Número de alumno   & Sesiones jugadas                            & Tiempo jugado \\
\midrule
1        & 2 & 79 \\
2        & 3 & 80 \\
4        & 3 & 89 \\
6        & 1 & 79 \\
\midrule
Total   & 9 & 327 \\
\bottomrule
\end{tabular}
\caption{Número de partidas y tiempo total por alumno en segundos, en la escena
    \textit{Glasgow}, en modo exploración}
\label{tab:log_glasgow_custom_partida}
\end{table}


En las tablas~\ref{tab:log_hemocultivo_puntaje}
y~\ref{tab:log_glasgow_random_puntaje} se muestran los primeros puntajes y un
promedio de los puntajes siguientes obtenidos por cada usuario en los
procedimientos de extracción de sangre y de la evaluación de la escala de
Glasgow. Se debe tener en cuenta el tiempo y las cantidades de veces que cada
alumno jugó cada uno de los procedimientos para valorar los resultados
mostrados. 

\begin{table}[H]
\centering
\begin{tabular}{lrrrrrrrr}
\toprule
& \multicolumn{2}{c}{Extracción de sangre} \\
\cmidrule(lr){2-3} 
Número de alumno  & Primer Puntaje & Siguientes Puntajes \\
\midrule
 1                & 11             & 14.3 \\
 2                & 9              & 10.6 \\
 4                & 3              & 3.3  \\
 5                & 3              & 6.8  \\
 6                & 3              & 5.8  \\
 7                & 4              & 4    \\
 9                & 16             & \\
10                & 3              & 7.2  \\
\midrule
\textbf{Promedio} & 6.5            & 7.42 \\
\bottomrule
\end{tabular}
\caption{Puntaje obtenido la primera vez y el promedio de las siguientes veces
    por alumno, en la escena de extracción de sangre.}
\label{tab:log_hemocultivo_puntaje}
\end{table}


\begin{table}[H]
\centering
\begin{tabular}{lrrrrrrrr}
\toprule
& \multicolumn{2}{c}{Glasgow (Evaluación)} \\
                   \cmidrule(lr){2-3} 
Número de alumno   & Primer Puntaje & Siguientes Puntajes \\
\midrule
1     & 1 & 1.5 \\
2     & 2 & 2.3 \\
4     & 1 & 1.5 \\
6     & 2 & 2 \\
7     & 0 & 1 \\
\midrule
\textbf{Promedio} & 1.2 & 1.66 \\
\bottomrule
\end{tabular}
\caption{Puntaje obtenido la primera vez y el promedio de las siguientes veces
    por alumno, en la escena \textit{Glasgow}, en modo evaluación}
\label{tab:log_glasgow_random_puntaje}
\end{table}

Se observa en las tablas~\ref{tab:log_hemocultivo_puntaje}
y~\ref{tab:log_glasgow_random_puntaje} los alumnos que participaron de la prueba
mejoran su desempeño a medida que aumenta el número de partidas. 

Es importante notar que la cantidad de partidas no es uniforme entre los
alumnos, es decir hay alumnos con más de $10$ partidas y usuarios con menos de
$5$, por ello, no es posible demostrar que existe un progreso a medida que
aumenta el número de partidas.

