\section{Recomendaciones}

En esta sección se busca dar una orientación en las diversas fases del
desarrollo de aplicaciones educativas similares a la de este trabajo de grado de
acuerdo a la experiencia de los autores y los resultados obtenidos.

\subsection{Diseño de la aplicación}

Es este apartado se detallaran las recomendaciones obtenidas del proceso de
diseño de la solución en diferentes aspectos. 

\begin{itemize}

\item \textbf{Retroalimentación constante con profesores y alumnos.} Es
    importante tener una buena relación con los profesores, y además mantener su
    interés en el proyecto, pues es necesario obtener la mayor cantidad de
    conocimientos y detalles en cuanto al contenido de la aplicación y, sobre
    todo,para que  evalúen cada paso realizado. Los alumnos pueden brindar
    información sobre el contenido y sus necesidades desde otro punto de vista
    totalmente válido pues son los usuarios finales de la aplicación.

\item \textbf{Aprovechar las reuniones con los profesionales.} Debido a la
    apretada agenda que poseen los profesionales, poder reunirse con ellos
    frecuentemente y en un largo periodo de tiempo para validar ideas y
    contenido de la aplicación resulta ser muy difícil, por lo cual es
    importante recabar la máxima cantidad de información en cada reunión.

\item \textbf{Realismo y facilidad de uso de la \Gls{gui}.} En cuanto a la
    interacción del usuario con la aplicación, siempre es preferible que sea lo
    más natural posible. La forma de uso de la aplicación no debería ser un
    obstáculo para utilizarla. La forma en que los elementos son representados y
    utilizados dentro de la aplicación debe ser realista, o en todo caso debe
    representar de la mejor manera la realidad de acuerdo a las limitaciones y
    sin distraer al alumno del objetivo pedagógico.

\end{itemize}


\subsection{Desarrollo de la aplicación}

Es este apartado se detallaran las recomendaciones obtenidas del proceso de
implementación de la solución en diferentes aspectos. 

\begin{itemize}

\item \textbf{Importancia del uso de los motores de videojuegos.} La utilización
    de motores gráficos modernos facilita la creación de juegos serios,
    relacionados a procedimientos del área de enfermería, permiten crear y
    manipular entornos realistas sin demasiadas complicaciones. 

    Una buena elección del motor gráfico de acuerdo a las características de la
    aplicación que se desea implementar es sumamente importante y los criterios
    de selección utilizados en este trabajo y mencionados
    en~\ref{sec:seleccion_plataforma} pueden ser de gran ayuda en la selección.

\item \textbf{Pruebas para mejoras durante el desarrollo.} La realización de
    prueba de usuarios es interesante durante el desarrollo de una aplicación ya
    que permiten visualizarlos errores en su funcionamiento y de esta manera se
    pueden corregir las debilidades encontradas de manera temprana, siempre es
    bueno detectar la mayor cantidad de errores en el menor tiempo posible.

\item \textbf{Validaciones de contenido de la aplicación.} La realización de
    validaciones tanto del contenido como de la representación de la aplicación
    por parte de los profesionales es de suma importancia, debido a que son a
    los estudiantes del área a los que va dirigido la solución. Estas
    validaciones no sólo permiten encontrar errores sino que permiten asegurar
    que la forma de implementación de ciertos aspectos es correcta. 

    Más allá de lo expuesto, estas validaciones con profesionales permiten que
    se sientan parte del proyecto, manteniendo el interés y predisposición de
    los profesionales.

\end{itemize}


\subsection{Evaluación de la aplicación}

Es este apartado se detallaran las recomendaciones obtenidas del proceso de
evaluación de la solución en diferentes aspectos. 

\begin{itemize}

\item \textbf{Evaluación de conocimiento con apoyo de profesionales.} Cuando se
    diseñan encuestas relacionadas con la medición del conocimiento es
    imprescindible elaborarlas con la ayuda y opinión de profesionales en el
    área, ya que tienen experiencia en la medición del conocimiento, esto no
    sólo ayuda a realizar una mejor medición sino que permite controlar que el
    contenido se encuentre dentro del grado de dificultad acorde a los
    estudiantes a los que va dirigido. 

\item \textbf{Importancia de los registros de actividades.} El registro de
    actividades del usuario es una herramienta importante, pues permite
    contrastar los datos obtenidos con otras metodologías, obteniendo
    correlaciones. Además permiten evaluar información interesante sobre la
    utilización de una aplicación, como frecuencia de uso, tiempo de uso, etc.

\item \textbf{Evaluación de la solución por los usuarios.} Se deben diseñar
    encuestas que puedan obtener información que no se puedan apreciar con las
    pruebas de conocimiento y los registros de actividades, como es la
    apreciación subjetiva de los estudiantes en cuanto a la aplicación.

\item \textbf{Mantener interés de los estudiantes.} Los estudiantes que formarán
    parte de la prueba de la aplicación deben estar bien informados sobre el
    objetivo de la aplicación y la validez de la misma, ya que deben estar
    interesados en probarla y ayudar para facilitar la evaluación. 

    Esto es importante sobre todo cuando la naturaleza de la aplicación no
    permite una evaluación controlada, como es el caso de este trabajo.
\end{itemize}
