%! TEX root = ../main.tex
\section{Diseño de juego serios}

En esta sección se detallan las conclusiones obtenidas en cuanto al proceso de
diseño de la solución en diferentes aspectos. La etapa de diseño se basa en 
crear un prototipo de la aplicación acorde a las características del problema.

%\observacion{Ver lo de que no hace falta simular todo}

\subsection{Validar constantemente el contenido con profesionales del área}

Una buena relación con los profesionales del área de estudio permite la
adquisición de valiosa información a través de reuniones, comentarios y
validaciones. La perspectiva con la que se analiza un juego serio es distinta de
acuerdo al profesional, es útil la perspectiva de los profesionales del área en
aspectos pedagógicos, el nivel de detalle de cada paso simulado y las
características del entorno.

En este trabajo se mantuvo una buena relación con los profesionales, frecuentes
reuniones ayudaron a validar y detectar errores en el diseño. Los profesionales
permitieron encontrar errores en las reglas definidas para la evaluación, y la
presentación de la escenografía de manera a mejorar su nivel de realismo.

Más allá de lo expuesto, estas validaciones con profesionales permiten que se
sientan parte del proyecto, manteniendo el interés y la predisposición para
ayudar al proyecto.


\subsection{Motivar a través de la utilización de aspectos lúdicos, como el
    puntaje y la medición del tiempo}


Los juegos serios no sólo permiten al estudiante experimentar, poner a prueba y
adquirir conocimientos sino que, debido a sus características lúdicas, ayuda en
la motivación siendo además una opción diferente con respecto a las demás
opciones tecnológicas utilizadas en el ámbito educativo.
    
En este sentido, en los resultados obtenidos de la evaluación de la solución se
muestra que en promedio los usuarios manifestaron estar \enquote{De acuerdo} con
el efecto motivacional de la solución en cuanto al uso de puntaje, tiempo y
socialización, como se observa en la sección~\ref{sec:res_subjetiva}.

Los datos mostrados en la tabla~\ref{tab:subjetiva_conformidad_motivacion}
permiten concluir que el principal factor motivador es la visualización de un
puntaje que resuma el desempeño del usuario. Adicionalmente se observa que otros
factores que influyen son la medición del tiempo y, en menor medida, la
posibilidad de compartir a través de redes sociales el rendimiento.

\subsection{Utilizar escenarios aleatorios para favorecer la exploración}

Permitir al usuario explorar y descubrir las diferentes posibilidades que ofrece
el entorno, es un factor clave en la metodología construccionista. Utilizar
entornos que varían de forma aleatoria entre las distintas sesiones de juego
permite al usuario la posibilidad de explorar.

En la solución propuesta, el factor aleatorio es lo que más favorece a la
exploración. Como se observa en la sección~\ref{sec:res_subjetiva}, la
utilización de un paciente con un estado no determinado permite al usuario
explorar las diversas formas que tiene para interactuar con el mismo.

%El alcance de la simulación de las herramientas debe incluir solamente lo
%necesario para llevar a cabo el objetivo, incluso ciertas actividades pueden no
%ser incluidas completamente para mantener la consistencia entre los distintos
%elementos.


\subsection{Proveer retroalimentación al finalizar cada escenario}

El momento en el que se provee información al usuario acerca de sus acciones
tiene un impacto en el potencial pedagógico de un juego serio, si se provee
información muy frecuente se puede obtener una herramienta del tipo \emph{Prueba
    y Error} que poco aporte al aprendizaje. Si la retroalimentación se posterga
demasiado, el usuario puede olvidar el contexto en que se generó dicha
retroalimentación.

En la solución, la retroalimentación es mostrada al terminar una sesión, se
evita que el usuario pierda el contexto creando simulaciones cortas y mostrando
información acerca de los pasos realizados incorrectamente. Proveer información
acerca del rendimiento al final de cada escenario, como se indica en la
sección~\ref{sec:res_subjetiva}, tiene un impacto positivo.

\subsection{Proveer ubicuidad para aumentar la utilización}


Los alumnos están de acuerdo con la utilización de este tipo de herramientas
para el apoyo al aprendizaje, como se observa en la
sección~\ref{sec:res_subjetiva} y~\ref{sec:res_subjetiva_abiertas}.

La ubicuidad permite a los usuarios una experiencia más frecuente, como se
observa en la sección~\ref{sec:res_subjetiva}, lo que es considerado por los
usuarios como útil. Esta es una de las principales ventajas de los juegos serios
frente a otras formas de apoyo al aprendizaje como los laboratorios.

\subsection{Imágenes representativas son útiles para indicar el estado de
    entidades}

La forma de presentar al usuario información de objetos que no son visibles
en la simulación 

Utilizar imágenes representativas para presentar al usuario información de
estados que no son visibles en el escenario es un factor que debe ser estudiado
al momento de diseñar un juego serio, pues existe información que es necesaria
mostrar cuya simulación es muy compleja, por ejemplo, las manos correctamente
esterilizadas.

La utilización de imágenes representativas tiene aprobación positiva de los
usuarios, como se observa en la sección~\ref{sec:res_subjetiva}.
