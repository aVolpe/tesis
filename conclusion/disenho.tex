
\section{Diseño de la aplicación}

En esta sección se detallan las conclusiones obtenidas en cuanto al proceso de
diseño de la solución en diferentes aspectos. La etapa de diseño se basa en 
crear un prototipo de la aplicación acorde a las características del problema.

\subsection{Retroalimentación constante con profesores y alumnos}

Es valiosa la retroalimentación que pueda obtenerse de parte de los profesores,
así como también la retroalimentación de los estudiantes ya que son estos
últimos los que utilizan la herramienta. Es importante tener una buena relación
con los profesores, y además mantener su interés en el proyecto, pues es
necesario obtener la mayor cantidad de conocimientos y detalles en cuanto al
contenido de la aplicación y, sobre todo, para que los profesores evalúen cada
paso realizado. Los estudiantes pueden brindar información sobre el contenido y
sus necesidades desde otro punto de vista totalmente válido pues son los
usuarios finales de la aplicación.

% COMUNICACIÓN CON PROFESIONALES
Debido a la apretada agenda que poseen los profesionales de salud, poder
reunirse con ellos frecuentemente y en un largo periodo de tiempo para validar
ideas y contenido de la aplicación resulta ser muy difícil, por lo cual es
importante recabar la máxima cantidad de información en cada reunión.

\subsection{Realismo y facilidad de uso de la \Gls{gui}}

En cuanto a la interacción del usuario con la aplicación, siempre es preferible
que sea lo más natural posible. La forma de uso de la aplicación no debería ser
un obstáculo para utilizarla. La forma en que los elementos son representados y
utilizados dentro de la aplicación debe ser realista, o en todo caso debe
representar de la mejor manera la realidad de acuerdo a las limitaciones y sin
distraer al alumno del objetivo pedagógico.

Las hipótesis relacionadas que se detallaron en este trabajo fueron confirmadas
en~\ref{sec:res_subjetiva} con los resultados de las pruebas realizadas. En
cuanto a las limitaciones, tanto tecnológicas como de utilización, presentadas
por los dispositivos móviles dificultan la interacción con el entorno. 

La interacción a través de dispositivos móviles debe ser cuidadosamente diseñada
para proveer de una interacción fluida y significativa.

\subsection{Ubicuidad de los dispositivos móviles}

Según datos obtenidos en la \emph{Encuesta de ubicación} el $98\%$ de los estudiantes 
encuestados cuenta con al menos un dispositivo móvil inteligente por lo que la idea 
de brindarles ubicuidad como uno de los objetivos del presenta trabajo resultó factible, 
no sólo por que contaban con este medio sino por que ello significaba brindarles la 
oportunidad de utilizar la herramienta en cualquier lugar y momento.

De hecho, los usuarios que evaluaron la solución mencionaron en promedio estar
\enquote{De acuerdo} con el uso de dispositivos móviles como tecnología de apoyo
como se muestra en~\ref{sec:res_subjetiva}.

\subsection{Aspectos lúdicos como motivación}

Los juegos serios no sólo permiten al estudiante experimentar, poner a prueba y
adquirir conocimientos sino que, debido a sus características lúdicas, ayuda en
la motivación siendo además una opción diferente con respecto a las demás
opciones tecnológicas utilizadas en el ámbito educativo.
    
En este sentido, en los resultados obtenidos de la evaluación de la solución se
muestra que en promedio los usuarios manifestaron estar \enquote{De acuerdo} con
el efecto motivacional de la solución en cuanto al uso de puntaje, tiempo y
socialización, como se observa en~\ref{sec:res_subjetiva}.



    
%\subsection{Pedagogía}
% (reuniones, escena, opciones, etc)
 
 
%\subsection{Accesibilidad}


%\subsection{Comunicación con profesionales}


%\subsection{Retroalimentación}

%\subsection{Utilización de las \Gls{tic}}

%\subsection{Aspectos Lúdicos}
