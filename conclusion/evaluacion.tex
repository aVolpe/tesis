\section{Evaluación de la aplicación}

En esta sección se detallan las conclusiones obtenidas en cuanto al proceso de
de evaluación de la solución en diferentes aspectos.

\subsection{Evaluación de conocimiento con apoyo de profesionales}

Cuando se diseñan encuestas relacionadas con la medición del conocimiento, en
nuestro caso de estudiantes de la carrera de Licenciatura en Enfermería, es
imprescindible elaborarlas con la ayuda y opinión de profesionales en el área,
ya que tienen experiencia en la medición del conocimiento, esto no sólo ayuda a
realizar una mejor medición sino que permite controlar que el contenido se
encuentre dentro del grado de dificultad acorde a los estudiantes a los que va
dirigido. 

En el caso de este trabajo, la \emph{Encuesta Objetiva} está orientada a este
objetivo, debido a que las preguntas se basaban en respuestas de selección
múltiple con una sola respuesta verdadera, los profesionales ayudaron a evitar
que las formulaciones y la lista de respuestas sean ambiguas o confusas.


\subsection{Importancia de los registros de actividades}

El registro de actividades del usuario es una herramienta importante, pues
permite contrastar los datos obtenidos con otras metodologías, obteniendo
correlaciones. Además permiten evaluar información interesante sobre la
utilización de una aplicación, como frecuencia de uso, tiempo de uso, etc.

Estas herramientas deben ser transparentes para el usuario y deben ser capaces
de funcionar aún sin acceso constante a internet.

\subsection{Evaluación de la solución por los usuarios}

Se deben diseñar encuestas que puedan obtener información que no se puedan
apreciar con las pruebas de conocimiento y los registros de actividades, como es
la apreciación subjetiva de los estudiantes en cuanto a la aplicación en
diferentes aspectos como utilidad, motivación, representación, pedagogía, entre
otros. En el caso de este trabajo, sirvió para validar algunas hipótesis
planteadas.


%\subsection{Como crear la evaluación (Opinión de los putitos)}
%\subsection{Logs}
%\subsection{Utilidad de las encuestas}
%\subsection{Proponer criterios que puedan servir para el desarrollo de trabajos
%    similares}
