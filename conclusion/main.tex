\chapter{Conclusión}
\label{chap:conclusion}

\observacion{\begin{itemize}
    \item Agrupar por temas
    \item Agregar referencias hacia atrás
    \item Evitar mapear términos
    \item Mapear objetivos específicos con la conclusión.
\end{itemize}}

% INTRO
Durante este trabajo \fixme{de grado}{esutdio el ... y} se desarrolló una
aplicación para dispositivos móviles con el fin de que sirva de apoyo en el
proceso de aprendizaje de los estudiantes de la carrera de enfermería,
utilizando a los juegos serios y el construccionismo como base.

% ACCESIBILIDAD
\fixme{
    La mayoría de los estudiantes cuenta con al menos un dispositivo móvil con el
    cual tienen acceso a internet por lo que la idea de darle ubicuidad nos resulto
    factible, no sólo por que contaban con este medio sino por que ello significaba
    brindarles la oportunidad de utilizar la herramienta en cualquier lugar y
    momento. La mayoría de los estudiantes que formaron parte del experimento
    concluyó que el uso de la solución les parecía útil para complementar el estudio
    en clase o laboratorio y que ayuda a comprender los procedimientos.
}{que quieren decir?}

% VENTAJAS
Los estudiantes mencionan como ventajas de la solución, el poder realizar
prácticas de procedimientos de manera más confiada ya que el paciente con el que
se interactúa es sólo virtual, la oportunidad de poder practicar en todo momento
y además, también mencionaron que el uso de la solución los acercaba más a la
tecnología.

Los profesores encargados del proceso de aprendizaje de los estudiantes de
enfermería con los que trabajamos en el desarrollo de este trabajo en todo
momento estuvieron abiertos al uso de las \Gls{tic} en forma de juegos serios,
incluso mencionaron la idea de poder utilizarlos en clase. Esto nos lleva a
pensar en que están abiertas las posibilidades de inclusión de la tecnología en
formas mas innovadoras con respecto a la forma de utilización en la actualidad
en nuestro país.

\fixme{Los juegos serios no sólo permiten al estudiante experimentar, poner a
    prueba y adquirir conocimientos sino que, debido a sus características
    lúdicas, ayuda en la motivación siendo además una opción diferente con
    respecto a las demás opciones tecnológicas utilizadas en el ámbito
    educativo. }{Mencionar datos del experimento}

% DESARROLLO

% CARACTERISTICAS
En cuanto al diseño de juego serio con las características de la solución, es
importante la retroalimentación que pueda tenerse de parte de los profesores,
así como es importante también la retroalimentación de los estudiantes ya que
son estos últimos los que utilizan la herramienta. Es importante mantener una
buena relación con los profesores, y además se debe mantener el interés de estos
en el proyecto, pues es necesario obtener la mayor cantidad de conocimientos y
detalles que puedan brindar en cuanto al contenido de la aplicación y, sobre
todo, para que evalúen cada paso realizado en la solución.

% COMUNICACIÓN CON PROFESIONALES
Debido a la apretada agenda que poseen los profesionales de salud, poder
reunirse con ellos frecuentemente y en un largo periodo de tiempo para validar
ideas y contenido de la aplicación resulta ser muy difícil, por lo cual es
importante recabar la máxima cantidad de información posible.

% INTERACCIÓN
En cuanto a la interacción con la solución implementada, siempre es preferible
que sea lo más natural posible, la forma de uso de la aplicación no debería ser
un obstáculo para utilizarla, la forma en que los elementos son representados y
utilizados dentro de la aplicación debe ser realista, o en todo caso debe
representar de la mejor manera la realidad de acuerdo a las limitaciones.

% ENCUESTA
En cuanto a las pruebas de la aplicación, es importante elaborarlas con la
opinión profesional de los expertos, sus validaciones son indispensables, ya que
ellos son especialistas en el contenido. Los estudiantes que formarán parte de
la prueba deben estar bien informados sobre el objetivo de la solución y la
validez de la misma, ya que deben estar interesados en probarla y ayudar para
facilitar la evaluación. 

% LOGS
Es importante registrar la actividad de los usuarios, sobre todo con
herramientas que ofrecen ubicuidad y son probadas a distancia. Además de las
pruebas, se deben diseñar encuestas que puedan obtener información que no se
puedan apreciar con las pruebas y registros de actividades, sobre todo la
apreciación final de los estudiantes de la solución como una herramienta útil
para el aprendizaje.


% PUEDE IR CONCLUSION SOBRE EL USO DE LA HERRAMIENTA ESPECIFICAMENTE EN EL AREA DE ENFERMERIA
La utilización de motores gráficos modernos facilita la creación de juegos
serios, relacionados a procedimientos del área de enfermería, permiten crear y
manipular entornos realistas sin demasiadas complicaciones. En cuanto a las
limitaciones, tanto tecnológicas como de utilización, presentadas por los
dispositivos móviles dificultan la interacción con el entorno, la interacción a
través de dispositivos móviles debe ser cuidadosamente diseñada para proveer de
una interacción fluida. 

% ESTO DEFINITIVAMENTE DEBE IR AL FINAL
Como conclusión final, consideramos interesante la investigación de herramientas
como la solución propuesta en este trabajo ya que la tendencia actual es que las
tecnologías tengan un papel más activo en el proceso de enseñanza-aprendizaje, y
las nuevas pedagogías requieren que sea así. Como se reflejan en varios
artículos, los estudiantes del nuevo milenio están acostumbrados a las nuevas
tecnologías y tienen otros estilos de aprendizaje, por lo que debe asumirse el
desafío de incorporar a la tecnología con mayor fuerza, aportando aún más
dinamismo en los procesos de enseñanza-aprendizaje. Creemos además, que el
factor motivacional que puede brindar este tipo de herramientas influirá
positivamente en los estudiantes y en el profesores.

% APRECIACIÓN PERSONAL

\observacion{\begin{itemize}
\item Agrupar las conclusiones por categoría
\item Referenciar secciones cuando se afirma algo respecto a los
    resultados
\item Enumerar o bullets
\end{itemize}}

% Puntos de los que se podria dar una conclusion


% - La mayoria de los estudiantes cuenta con un dispositivo movil desde el cual tienen acceso a internet
% - La mayoria de los estudiantes opino que el uso de herramientas como la solucion es util para complementar 
% el estudio en clase o laboratorio.
% - La mayoria de los estudiantes opino que la solucion es una herramienta que ayuda a memorizar y comprender 
% los procedimientos
% - La mayoria considero sumamente importante que las herramientas le proporcionen una retroalimentacion sobre 
% su rendimiento pero desean mucha informacion acerca de los errores cometidos, no solo breves justificaciones.
% - La mayoria considero que la herramienta provee facilidades para el estudio
% - En preguntas abiertas los estudiantes dieron como ventaja de la solucion el poder realizar practicas de 
% procedimientos de manera mas confiada ya que el paciente es ficticio, la oportunidad de poder practicar 
% en todo momento, mencionaron tambien que los acerca mas a la tecnologia
% - En el diseño y desarrollo de herramientas de este tipo es importante la retroalimentacion que tengamos de 
% parte de los maestros encargados del aprendizaje pero tambien de los alumnos ya son ellos los que la mayoria
% de las veces las utilizaran.
% - Los profesores del area estuvieron abiertos al uso de las TIC presentados de esta forma, incluso 
% mencionaron poder utilizarlos en clase.
% - Los juegos serios no solo permiten al estudiante experimentar, poner a prueba y adquirir conocimientos sino que
% debido a las caracteristicas ludicas que posee ayuda en la motivacion de los mismos, siendo ademas una opcion 
% diferente e innovadora con respecto a las demas
% - En cuanto al diseño, es preferible que la interaccion con la aplicacion sea lo mas natural posible.
% - En cuanto al diseño, es preferible que la forma de uso de los elementos dentro de la aplicacion sean lo 
% mas real posible o en todo caso represente de la mejor manera la realidad de acuerdo a las limitaciones.
% - En el diseño y desarrollo, es importante mantener una buena relacion con los profesionales encargados de 
% la formacion de los estudiantes ya que se debe obtener la mayor cantidad de conocimientos y detalles de los 
% mismos, y ademas debe matenerse el interes de estos profesionales para que evaluen cada paso tomado en 
% la realizacion de la solucion
% - Debido a la apretada agenda que poseen los profesionales de salud poner reunirse con ellos todo el tiempo y 
% un largo periodo resulta muy dificil, por lo cual es importante ir preparados a estas reuniones y ser los
% mas breves y concisos posible.
% - En cuanto a las pruebas, es importante elaborarlas con la opinion profesional de los maestros, su validacion es 
% muy importante ya que ellos son los especialistas en el contenido en la que la herramienta esta basada.
% - En cuanto a las pruebas, es importante que los estudiantes esten informados del objetivo de la herramienta, 
% de la validez de la misma, ya que deben estar interesados en probarla y ayudar en su evaluacion para que 
% pueda ser valorada de manera correcta.
% - En cuanto a las pruebas, es importante tener un registro de actividad de los usuarios, sobre todo con 
% herramientas que ofrecen ubicuidad y son probados a distancia dada esta caracteristica importante, de 
% otra manera seria imposible evaluar su uso.
% - En cuanto a las pruebas, es importante diseñar encuestas que puedan obtener informacion que no puedan ser 
% apreciadas solo con el uso de registro de actividad de usuarios, es importante la apreciacion final de los 
% estudiantes sobre todo en cuanto a lo que implica la solucion como una herramienta util
% - Actualmente existen motores de juegos completos que ofrecen grandes facilidades para el desarrollo de videojuegos.
% - Se considera interesante la investigacion de herramientas como la solucion ya que la tendencia actual es que 
% las tecnologias tengan un papel mas activo en el aprendizaje, y las nuevas pedagogias requiren que sea asi. 
% Los estudiantes del nuevo milenio estan acostumbrado a las nuevas tecnologias y tiene otros estilos de aprendizaje, 
% ACA PUEDE IR ALGO DE LA IMPORTANCIA DE LAS TICS EN GENERAL
% http://www.oei.es/divulgacioncientifica/?Cuando-las-TIC-irrumpen-en-las&utm_content=bufferf064b&utm_medium=social&utm_source=facebook.com&utm_campaign=buffer
