\chapter{Conclusión}
\label{chap:conclusion}


Durante esta trabajo de grado se estudió la utilización de las \Gls{tic} en la
educación, en especial a los juegos serios, y la teoría construccionista del
aprendizaje. Como resultado se diseñó y desarrolló una aplicación para
dispositivos móviles cuyo fin es el de servir como herramienta de apoyo en el
proceso de aprendizaje de los estudiantes de la carrera de enfermería.

Consideramos interesante la investigación de herramientas como la solución
propuesta en este trabajo ya que la tendencia actual es que las tecnologías
tengan un papel más activo en el proceso de enseñanza-aprendizaje, y las nuevas
corrientes pedagógicas requieren que sea así. 

Como se refleja en varios artículos, los estudiantes del nuevo milenio están
acostumbrados a las nuevas tecnologías y tienen otros estilos de aprendizaje,
por lo que debe asumirse el desafío de incorporar a la tecnología con mayor
fuerza, aportando aún más dinamismo en los procesos de enseñanza-aprendizaje.
Creemos además, que el factor motivacional que puede brindar este tipo de
herramientas influirá positivamente en los estudiantes y en el profesores.

A continuación se detallan las conclusiones obtenidas en cada una de las fases del 
desarrollo de la aplicación.

%! TEX root = ../main.tex
\section{Diseño de juego serios}

En esta sección se detallan las conclusiones obtenidas en cuanto al proceso de
diseño de la solución en diferentes aspectos. La etapa de diseño se basa en 
crear un prototipo de la aplicación acorde a las características del problema.

%\observacion{Ver lo de que no hace falta simular todo}

\subsection{Validar constantemente el contenido con profesionales del área}

Una buena relación con los profesionales del área de estudio permite la
adquisición de valiosa información a través de reuniones, comentarios y
validaciones. La perspectiva con la que se analiza un juego serio es distinta de
acuerdo al profesional, es útil la perspectiva de los profesionales del área en
aspectos pedagógicos, el nivel de detalle de cada paso simulado y las
características del entorno.

En este trabajo se mantuvo una buena relación con los profesionales, frecuentes
reuniones ayudaron a validar y detectar errores en el diseño. Los profesionales
permitieron encontrar errores en las reglas definidas para la evaluación, y la
presentación de la escenografía de manera a mejorar su nivel de realismo.

Más allá de lo expuesto, estas validaciones con profesionales permiten que se
sientan parte del proyecto, manteniendo el interés y la predisposición para
ayudar al proyecto.


\subsection{Motivar a través de la utilización de aspectos lúdicos, como el
    puntaje y la medición del tiempo}

Los juegos serios no sólo permiten al estudiante experimentar, poner a prueba y
adquirir conocimientos sino que, debido a sus características lúdicas, ayuda en
la motivación siendo además una opción diferente con respecto a las demás
opciones tecnológicas utilizadas en el ámbito educativo.
    
En este sentido, en los resultados obtenidos de la evaluación de la solución se
muestra que en promedio los usuarios manifestaron estar \enquote{De acuerdo} con
el efecto motivacional de la solución en cuanto al uso de puntaje, tiempo y
socialización, como se observa en la sección~\ref{sec:res_subjetiva}.

Los datos mostrados en la tabla~\ref{tab:subjetiva_conformidad_motivacion}
permiten concluir que el principal factor motivador es la visualización de un
puntaje que resuma el desempeño del usuario. Adicionalmente se observa que otros
factores que influyen son la medición del tiempo y, en menor medida, la
posibilidad de compartir a través de redes sociales el rendimiento.

\subsection{Utilizar escenarios aleatorios para favorecer la exploración}

Permitir al usuario explorar y descubrir las diferentes posibilidades que ofrece
el entorno, es un factor clave en la metodología construccionista. Utilizar
entornos que varían de forma aleatoria entre las distintas sesiones de juego
permite al usuario la posibilidad de explorar.

En la solución propuesta, el factor aleatorio es lo que más favorece a la
exploración. Como se observa en la sección~\ref{sec:res_subjetiva}, la
utilización de un paciente con un estado no determinado permite al usuario
explorar las diversas formas que tiene para interactuar con el mismo.

%El alcance de la simulación de las herramientas debe incluir solamente lo
%necesario para llevar a cabo el objetivo, incluso ciertas actividades pueden no
%ser incluidas completamente para mantener la consistencia entre los distintos
%elementos.


\subsection{Proveer retroalimentación al finalizar cada escenario}

El momento en el que se provee información al usuario acerca de sus acciones
tiene un impacto en el potencial pedagógico de un juego serio, si se provee
información muy frecuente se puede obtener una herramienta del tipo \emph{Prueba
    y Error} que poco aporte al aprendizaje. Si la retroalimentación se posterga
demasiado, el usuario puede olvidar el contexto en que se generó dicha
retroalimentación.

En la solución, la retroalimentación es mostrada al terminar una sesión, se
evita que el usuario pierda el contexto creando simulaciones cortas y mostrando
información acerca de los pasos realizados incorrectamente. Proveer información
acerca del rendimiento al final de cada escenario, como se indica en la
sección~\ref{sec:res_subjetiva}, tiene un impacto positivo.

\subsection{Proveer ubicuidad para aumentar la utilización}


Los alumnos están de acuerdo con la utilización de este tipo de herramientas
para el apoyo al aprendizaje, como se observa en la
sección~\ref{sec:res_subjetiva} y~\ref{sec:res_subjetiva_abiertas}.

La ubicuidad permite a los usuarios una experiencia más frecuente, como se
observa en la sección~\ref{sec:res_subjetiva}, lo que es considerado por los
usuarios como útil. Esta es una de las principales ventajas de los juegos serios
frente a otras formas de apoyo al aprendizaje como los laboratorios.

\subsection{Imágenes representativas son útiles para indicar el estado de
    entidades}

La forma de presentar al usuario información de objetos que no son visibles
en la simulación 

Utilizar imágenes representativas para presentar al usuario información de
objetos que no son visibles en el escenario es un factor que debe ser estudiado
al momento de diseñar un juego serio, pues existe información que es necesaria
mostrar cuya simulación es muy compleja, por ejemplo, las manos correctamente
esterilizadas.

La utilización de imágenes representativas tiene aprobación positiva de los
usuarios, como se observa en la sección~\ref{sec:res_subjetiva}.


\section{Desarrollo de la aplicación}

En esta sección se detallan las conclusiones obtenidas en cuanto al proceso de
implementación de la solución en diferentes aspectos.

\subsection{Importancia del uso de motores de videojuegos}

La utilización de motores gráficos modernos facilita la creación de juegos
serios, relacionados a procedimientos del área de enfermería, permiten crear y
manipular entornos realistas sin demasiadas complicaciones. 

Una buena elección del motor gráfico de acuerdo a las características de la
aplicación que se desea implementar es sumamente importante y los criterios de
selección utilizados en este trabajo y mencionados
en~\ref{sec:seleccion_plataforma} pueden ser de gran ayuda en la selección.

\subsection{Pruebas para mejoras durante el desarrollo}

La realización de prueba de usuarios es interesante durante el desarrollo de una
aplicación ya que permiten visualizar los errores en su funcionamiento y de esta
manera se pueden corregir las debilidades encontradas de manera temprana,
siempre es mejor detectar la mayor cantidad de errores en el menor tiempo
posible.

En el caso de la solución, se realizaron pruebas de \Gls{gui} para mejorar la
interacción de los usuarios con ella, estas pruebas revelaron principalmente
problemas con la interacción con el entorno y con los objetos. Como resultado se
aplicaron mejoras que permitieron mejorar el uso y la intuitividad de la
interfaz.

\subsection{Validaciones de contenido de la aplicación}

La realización de validaciones tanto del contenido como de la representación de
la aplicación por parte de los profesionales es de suma importancia, debido a
que son a los estudiantes del área a los que va dirigido la solución. Estas
validaciones deben centrarse en los detalles en los que los mismos ponen mayor
interés.

Estas validaciones no sólo permiten encontrar errores sino que permiten asegurar
que la forma de implementación de ciertos aspectos es correcta. En el caso de la
solución, esto permitió corregir errores en las reglas definidas para la
evaluación del procedimiento de extracción de muestras de sangre, y la
presentación de la escenografía de manera tal que fuera más realista.

Más allá de lo expuesto, estas validaciones con profesionales permiten que se
sientan parte del proyecto, manteniendo el interés y predisposición de los
profesionales.


%En cuanto a las
%limitaciones, tanto tecnológicas como de utilización, presentadas por los
%dispositivos móviles dificultan la interacción con el entorno, la interacción a
%través de dispositivos móviles debe ser cuidadosamente diseñada para proveer de
%una interacción fluida.

%\subsection{Tecnología}
%\subsection{Nivel de detalle}
%\subsection{Interacción con objetos}
%\subsection{Partes que son necesarias pero no simulables}
    
%(lo de bioseguridad)

\section{Evaluación de la aplicación}

En esta sección se detallan las conclusiones obtenidas en cuanto al proceso de
de evaluación de la solución en diferentes aspectos.

\subsection{Evaluación de conocimiento con apoyo de profesionales}

Cuando se diseñan encuestas relacionadas con la medición del conocimiento, en
nuestro caso de estudiantes de la carrera de Licenciatura en Enfermería, es
imprescindible elaborarlas con la ayuda y opinión de profesionales en el área,
ya que tienen experiencia en la medición del conocimiento, esto no sólo ayuda a
realizar una mejor medición sino que permite controlar que el contenido se
encuentre dentro del grado de dificultad acorde a los estudiantes a los que va
dirigido. 

En el caso de este trabajo, la \emph{Encuesta Objetiva} está orientada a este
objetivo, debido a que las preguntas se basaban en respuestas de selección
múltiple con una sola respuesta verdadera, los profesionales ayudaron a evitar
que las formulaciones y la lista de respuestas sean ambiguas o confusas.


\subsection{Importancia de los registros de actividades}

El registro de actividades del usuario es una herramienta importante, pues
permite contrastar los datos obtenidos con otras metodologías, obteniendo
correlaciones. Además permiten evaluar información interesante sobre la
utilización de una aplicación, como frecuencia de uso, tiempo de uso, etc.

Estas herramientas deben ser transparentes para el usuario y deben ser capaces
de funcionar aún sin acceso constante a internet.

\subsection{Evaluación de la solución por los usuarios}

Se deben diseñar encuestas que puedan obtener información que no se puedan
apreciar con las pruebas de conocimiento y los registros de actividades, como es
la apreciación subjetiva de los estudiantes en cuanto a la aplicación en
diferentes aspectos como utilidad, motivación, representación, pedagogía, entre
otros. En el caso de este trabajo, sirvió para validar algunas hipótesis
planteadas.


%\subsection{Como crear la evaluación (Opinión de los putitos)}
%\subsection{Logs}
%\subsection{Utilidad de las encuestas}
%\subsection{Proponer criterios que puedan servir para el desarrollo de trabajos
%    similares}

\section{Otras conclusiones}

En esta sección se detallan conclusiones sobre aspectos transversales 
a los demás puntos tratados.

\subsection{Utilidad de la simulación como herramienta de apoyo}

La mayoría de los estudiantes que formaron parte de la prueba concluyó que el
uso de la solución les parecía útil para complementar el estudio en clase o
laboratorio y que ayuda a comprender los procedimientos.

Los profesores encargados del proceso de aprendizaje de los estudiantes de
enfermería con los que se trabajó en el desarrollo de este trabajo en todo
momento estuvieron abiertos al uso de las \Gls{tic} en la forma de juegos
serios, incluso mencionaron la idea de poder utilizarlos en clase. 

Esto permite concluir que están abiertas las posibilidades de inclusión de la
tecnología en formas más innovadoras con respecto a la forma de utilización en
la actualidad en nuestro país.

\subsection{Mantener interés de los estudiantes}

Los estudiantes que formarán parte de la prueba de la aplicación deben estar
bien informados sobre el objetivo de la aplicación y la validez de la misma, ya
que deben estar interesados en probarla y ayudar para facilitar la evaluación. 

Esto es importante sobre todo cuando la naturaleza de la aplicación no permite
una evaluación controlada, como es el caso de este trabajo.







%\subsection{Comunicación del objetivo a los alumnos}
%\subsection{Utilidad y ventajas}
%\subsection{Que todos tienen móviles}

