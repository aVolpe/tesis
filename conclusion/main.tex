\chapter{Conclusión}
\label{chap:conclusion}


Durante esta trabajo de grado se analizó y evalúo la utilización de las
\Gls{tic} en la educación, en especial a los juegos serios. Como resultado se
diseñó y desarrolló una aplicación para dispositivos móviles cuyo fin es el de
servir como herramienta de apoyo en el proceso de aprendizaje de los estudiantes
de la carrera de enfermería. Así como también para identificar y valorar
los factores pedagógicos, de diseño, de implementación y de evaluación que
influyen en la creación de juegos serios.

Se considera interesante la investigación de herramientas como la solución
propuesta en este trabajo, debido a la tendencia actual de promover tecnologías
que tengan un papel más activo en el proceso de enseñanza-aprendizaje, abriendo
camino a las nuevas corrientes pedagógicas. 

Los estudiantes del nuevo milenio están acostumbrados a las nuevas tecnologías y
tienen otros estilos de aprendizaje. Por lo tanto se debe asumir el desafío de
incorporar a la tecnología con mayor fuerza, aportando aún más dinamismo en los
procesos de enseñanza-aprendizaje. Se considera además, que el factor motivacional
que puede brindar este tipo de herramientas influirá positivamente en los
estudiantes y en los profesores.

A continuación se detallan las conclusiones obtenidas en cada una de las fases
del desarrollo de la aplicación.

%
\section{Diseño de la aplicación}

En esta sección se detallan las conclusiones obtenidas en cuanto al proceso de
diseño de la solución en diferentes aspectos. La etapa de diseño se basa en 
crear un prototipo de la aplicación acorde a las características del problema.

\subsection{Retroalimentación constante con profesores y alumnos}

Es valiosa la retroalimentación que pueda obtenerse de parte de los profesores,
así como también la retroalimentación de los estudiantes ya que son estos
últimos los que utilizan la herramienta. Es importante tener una buena relación
con los profesores, y además mantener su interés en el proyecto, pues es
necesario obtener la mayor cantidad de conocimientos y detalles en cuanto al
contenido de la aplicación y, sobre todo, para que los profesores evalúen cada
paso realizado. Los estudiantes pueden brindar información sobre el contenido y
sus necesidades desde otro punto de vista totalmente válido pues son los
usuarios finales de la aplicación.

% COMUNICACIÓN CON PROFESIONALES
Debido a la apretada agenda que poseen los profesionales de salud, poder
reunirse con ellos frecuentemente y en un largo periodo de tiempo para validar
ideas y contenido de la aplicación resulta ser muy difícil, por lo cual es
importante recabar la máxima cantidad de información en cada reunión.

\subsection{Realismo y facilidad de uso de la \Gls{gui}}

En cuanto a la interacción del usuario con la aplicación, siempre es preferible
que sea lo más natural posible. La forma de uso de la aplicación no debería ser
un obstáculo para utilizarla. La forma en que los elementos son representados y
utilizados dentro de la aplicación debe ser realista, o en todo caso debe
representar de la mejor manera la realidad de acuerdo a las limitaciones y sin
distraer al alumno del objetivo pedagógico.

Las hipótesis relacionadas que se detallaron en este trabajo fueron confirmadas
en~\ref{sec:res_subjetiva} con los resultados de las pruebas realizadas. En
cuanto a las limitaciones, tanto tecnológicas como de utilización, presentadas
por los dispositivos móviles dificultan la interacción con el entorno. 

La interacción a través de dispositivos móviles debe ser cuidadosamente diseñada
para proveer de una interacción fluida y significativa.

\subsection{Ubicuidad de los dispositivos móviles}

Según datos obtenidos en la \emph{Encuesta de ubicación} el $98\%$ de los estudiantes 
encuestados cuenta con al menos un dispositivo móvil inteligente por lo que la idea 
de brindarles ubicuidad como uno de los objetivos del presenta trabajo resultó factible, 
no sólo por que contaban con este medio sino por que ello significaba brindarles la 
oportunidad de utilizar la herramienta en cualquier lugar y momento.

De hecho, los usuarios que evaluaron la solución mencionaron en promedio estar
\enquote{De acuerdo} con el uso de dispositivos móviles como tecnología de apoyo
como se muestra en~\ref{sec:res_subjetiva}.

\subsection{Aspectos lúdicos como motivación}

Los juegos serios no sólo permiten al estudiante experimentar, poner a prueba y
adquirir conocimientos sino que, debido a sus características lúdicas, ayuda en
la motivación siendo además una opción diferente con respecto a las demás
opciones tecnológicas utilizadas en el ámbito educativo.
    
En este sentido, en los resultados obtenidos de la evaluación de la solución se
muestra que en promedio los usuarios manifestaron estar \enquote{De acuerdo} con
el efecto motivacional de la solución en cuanto al uso de puntaje, tiempo y
socialización, como se observa en~\ref{sec:res_subjetiva}.



    
%\subsection{Pedagogía}
% (reuniones, escena, opciones, etc)
 
 
%\subsection{Accesibilidad}


%\subsection{Comunicación con profesionales}


%\subsection{Retroalimentación}

%\subsection{Utilización de las \Gls{tic}}

%\subsection{Aspectos Lúdicos}

%%! TEX root = ../main.tex

\section{Desarrollo de la aplicación}

En esta sección se detallan las conclusiones obtenidas en cuanto al proceso de
implementación de la solución.

\subsection{Utilizar un motor de videojuegos}

La utilización de motores de videojuegos facilita la creación de juegos serios.
Los motores actuales permiten crear y manipular entornos complejos, desarrollar
aplicaciones para diversas plataformas, y cuentan con librerías para las tareas
comunes (como manipulación de perspectiva, creación de animaciones, etc). 

La utilización de motores de videojuegos permiten al desarrollador centrarse en
aspectos específicos del juego serio y a no invertir el tiempo en aspectos
comunes a los motores, como, un motor de física, detectores de colisiones,
reproducción de contenido multimedia, motores de animaciones, inteligencia
artificial, comunicación, \textit{streaming}, administración de memoria, etc.

En este trabajo se utiliza \textit{Unity3D}, el cual facilita las tareas a
través de un \gls{ide} complejo, diversas herramientas para la creación de
animaciones, y gestión de librerías. Uno de los factores más interesantes de
\textit{Unity3D} es su tienda de librerías, donde se puede adquirir modelos en
tres dimensiones de calidad a precios bajos e incluso de manera gratuita.

\subsection{Seleccionar herramientas de acuerdo a las características necesarias}

Una buena elección del motor gráfico de acuerdo a las características de la
aplicación que se desea implementar es sumamente importante y los criterios de
selección utilizados en este trabajo y mencionados en la
sección~\ref{sec:seleccion_plataforma} pueden ser de gran ayuda en la selección.

Las limitaciones técnicas, de tiempo y monetarias del equipo de desarrollo
también deben ser tenidas en cuenta al momento de la selección de una
herramienta.

Junto a estos criterios, la información proveída en la
sección~\ref{sec:motores}, permitió seleccionar a \textit{Unity3D} como el motor
de videojuego necesario para desarrollar la solución utilizada para evaluar a
los juegos serios en el aprendizaje en este trabajo.

\subsection{Probar el desarrollo de manera frecuente}

En el desarrollo de juegos serios es útil realizar pruebas preliminares para
analizar y evaluar las decisiones de diseño, a fin de minimizar los obstáculos
presentes para los usuarios finales. Los aspectos que se deben probar y validar
son:

\begin{itemize}
    \item Calidad gráfica.
    \item Nivel de detalle del entorno.
    \item Integración con el hardware.
    \item Desenvolvimiento en el entorno a través de la interfaz.
    \item Interacción con objetos y entidades.
    \item Usabilidad de la \gls{gui}.
\end{itemize}

Las pruebas preliminares realizadas a la solución, cuyos resultados se observan
en la sección~\ref{sec:res_interfaz}, permite identificar fortalezas y
debilidades. 

En el caso específico de la solución, los problemas detectados incluyen falta de
naturalidad en la utilización de objetos y falta de intuitividad en el uso de la
\Gls{gui}. Estos problemas son solucionados en la versión que se presenta a los
usuarios finales.

La realización de prueba con usuarios es útil durante el desarrollo de una
aplicación ya que permiten visualizar los errores en su funcionamiento y de esta
manera se pueden corregir las debilidades encontradas de manera temprana,
siempre es mejor detectar la mayor cantidad de errores en el menor tiempo
posible.

En el caso de la solución, se realizaron pruebas de \Gls{gui} para mejorar la
interacción de los usuarios con ella, estas pruebas revelaron principalmente
problemas con la interacción con el entorno y con los objetos. Como resultado se
aplicaron mejoras que permitieron mejorar el uso y la intuitividad de la
interfaz.

\subsection{Crear entornos en tres dimensiones para aumentar la inmersión}

Escenarios en tres dimensiones similares a los lugares conocidos por los
usuarios ayudan permite a los mismos sentirse parte de la simulación. 

En la solución presentada en este trabajo, se utiliza un escenario en tres
dimensiones similar a un escenario conocido por los usuarios finales (un
laboratorio de prácticas), el mismo fue diseño en base a observaciones
realizadas en una visita guiada por profesionales del \gls{iab}. En la
sección~\ref{sec:res_subjetiva} se observa que los usuarios valoran
positivamente la influencia del escenario en la inmersión.

\subsection{Proveer retroalimentación clara al realizar una acción}

La forma de notificar al usuario en el momento en el que realiza una acción,
debe ser fácilmente observable y distinguible, de esta manera el usuario puede
tener la certeza de que se realizo la acción deseada.  

En la solución, cuando el usuario interactúa con un objeto, en la
sección~\ref{sec:res_subjetiva} se observa que mientras más vistosa es la
reacción del paciente, mejor es la recepción de los usuarios. Movimientos
discretos como la apertura ocular y el movimiento de los ojos, tienen una pobre
aceptación, en cambio, movimientos vistosos como el movimiento de las piernas,
brazos y el cuerpo en general, tienen una aceptación positiva.

La forma de representar las alteraciones del entorno luego de realizar una
acción deben ser claras.


%\section{Evaluación de la aplicación}

En esta sección se detallan las conclusiones obtenidas en cuanto al proceso de
de evaluación de la solución en diferentes aspectos.

\subsection{Evaluación de conocimiento con apoyo de profesionales}

Cuando se diseñan encuestas relacionadas con la medición del conocimiento, en
nuestro caso de estudiantes de la carrera de Licenciatura en Enfermería, es
imprescindible elaborarlas con la ayuda y opinión de profesionales en el área,
ya que tienen experiencia en la medición del conocimiento, esto no sólo ayuda a
realizar una mejor medición sino que permite controlar que el contenido se
encuentre dentro del grado de dificultad acorde a los estudiantes a los que va
dirigido. 

En el caso de este trabajo, la \emph{Encuesta Objetiva} está orientada a este
objetivo, debido a que las preguntas se basaban en respuestas de selección
múltiple con una sola respuesta verdadera, los profesionales ayudaron a evitar
que las formulaciones y la lista de respuestas sean ambiguas o confusas.


\subsection{Importancia de los registros de actividades}

El registro de actividades del usuario es una herramienta importante, pues
permite contrastar los datos obtenidos con otras metodologías, obteniendo
correlaciones. Además permiten evaluar información interesante sobre la
utilización de una aplicación, como frecuencia de uso, tiempo de uso, etc.

Estas herramientas deben ser transparentes para el usuario y deben ser capaces
de funcionar aún sin acceso constante a internet.

\subsection{Evaluación de la solución por los usuarios}

Se deben diseñar encuestas que puedan obtener información que no se puedan
apreciar con las pruebas de conocimiento y los registros de actividades, como es
la apreciación subjetiva de los estudiantes en cuanto a la aplicación en
diferentes aspectos como utilidad, motivación, representación, pedagogía, entre
otros. En el caso de este trabajo, sirvió para validar algunas hipótesis
planteadas.


%\subsection{Como crear la evaluación (Opinión de los putitos)}
%\subsection{Logs}
%\subsection{Utilidad de las encuestas}
%\subsection{Proponer criterios que puedan servir para el desarrollo de trabajos
%    similares}

\section{Estado del arte}

Las conclusiones obtenidas en cuanto a la investigación del estado del arte son 
las siguientes:

\begin{itemize}

% REVISAR
\item \textbf{Los beneficios de las \gls{tic} aún no hay sido explotados
        completamente en el área  educativa}: el rol de las \gls{tic} en la
    educación sigue siendo mayormente como proveedor de conocimiento debido a
    que el instruccionismo es aún la corriente pedagógica con mayor vigencia.

\item \textbf{Los juegos serios tienden a ofrecer una retroalimentación muy
        guiada}: la mayoría de los juegos serios existentes brindan una
    retroalimentación muy guiada al usuario induciéndolo a hacer lo correcto sin
    brindarle suficiente espacio para que desarrolle su pensamiento crítico y
    toma de decisiones.

\item \textbf{Los juegos serios permiten una experiencia sin riesgos reales}:
    permiten  al usuario experimentar, poner a prueba y construir conocimientos
    sin los riesgos económicos y de salud presentes en la vida real.

% Ver si vale la pena citar
\item \textbf{La enseñanza de profesionales de enfermería es un área propicia
        para la aplicación de los juegos serios}: los estudiantes de enfermería
    requieren un alto grado de prácticas y poseen poco tiempo disponible para
    actividades ajenas a las académicas impuestas por su centro de estudios.

\item \textbf{Se debe utilizar herramientas alternativas de bajo costo}: el
    nivel de acceso a la tecnología de los estudiantes del \gls{iab} es bajo.
    Proveer soluciones que requieran una inversión en hardware o software por
    parte de los usuarios no es una alternativa viable actualmente.

\end{itemize}

\section{Diseño del juego serio}

Las conclusiones obtenidas en cuanto al diseño del juego serio son las siguientes:

\begin{itemize}


\item \textbf{La definición de los aspectos pedagógicos, nivel de detalle
        requerido por procedimiento y las características del entorno debe ser
        realizada con profesores de prácticas y directores de carrera}: las
    validaciones constantes de estos aspectos ayudan a la detección de errores
    en el diseño, un ejemplo es la evaluación del rendimiento del alumno ya que
    existen maneras adecuadas de realizar los procedimientos.

\item \textbf{La utilización del puntaje por procedimiento motiva a los
        usuarios}: los juegos serios poseen aspectos lúdicos inherentes que
    motivan al usuario en su uso, agregar indicadores de rendimiento ayudan a
    esta motivación. Los datos mostrados en la sección~\ref{sec:subjetiva}
    permiten concluir que el principal factor motivacional es la visualización
    de un puntaje que resuma el desempeño del usuario y, en menor medida, la
    medición del tiempo y la posibilidad de compartir su rendimiento en las
    redes sociales. 

\item \textbf{Los estados aleatorios y funciones simplificadas facilitan la
        exploración}: permitir al usuario explorar el entorno para resolver
    problemas ayuda en el desarrollo de su pensamiento crítico, así también, se
    debe ofrecer una interfaz intuitiva y simple para no obstaculizar esta
    exploración. Los datos mostrados en la sección~\ref{sec:subjetiva} permiten
    concluir que los principales factores que favorecen la exploración son la
    aleatoriedad en el estado del paciente y la representación simplificada de
    las funciones de los elementos. Sin embargo, esta simplificación debe
    diseñarse con mucho cuidado para no perder intuitividad en la interfaz. 

\item \textbf{Los gráficos en tres dimensiones y las partidas cortas aumentan la
        inmersión del usuario}: es importante que el usuario sienta que está
    inmerso en el juego para que de este modo pueda percibir el contexto que se
    le está presentado y actúe de acuerdo a esto. Los datos mostrados en la
    sección~\ref{sec:subjetiva} permiten concluir que la utilización de entornos
    en tres dimensiones para la representación de elementos y lugares a los que
    están familiarizado el usuario, así como proveer partidas cortas para
    evitar que el usuario pierda el contexto de sus acciones, favorecen a la
    inmersión. 

\item \textbf{Se debe proveer retroalimentación sobre el desempeño del usuario
        sólo al finalizar la partida}: el momento en el que se provee
    información acerca del desempeño del usuario tiene un impacto en el
    potencial pedagógico de un juego serio, si se provee información de manera
    muy frecuente se puede obtener herramientas del tipo prueba y error que poco
    aporte al aprendizaje. Los datos mostrados en la sección~\ref{sec:subjetiva}
    permiten concluir que ofrecerles a los usuarios una simulación de los
    procedimientos con una retroalimentación limitada les ayuda a poner en
    práctica sus conocimientos y a comprender el procedimiento. 

\item \textbf{La información sobre el rendimiento del usuario debe ser
        detallada}: los datos mostrados en la sección~\ref{sec:subjetiva}
    permiten concluir que los usuarios están de acuerdo con el hecho de
    proporcionarles una retroalimentación indicándoles los pasos que realizó de
    manera correcta e incorrecta dentro del procedimiento. Sin embargo, una
    breve causa acerca de las equivocaciones no es suficiente, se requiere
    información detallada. 
    
\item \textbf{Se deben utilizar indicadores de realización de acciones}: en la
    simulación de entornos de enfermería se debe diseñar un esquema que
    notifique al usuario sobre la realización de una acción, sobre todo para 
    aquellas acciones que no son visibles ante el ojo humano en la vida real. 
    En la solución se utilizan indicadores transparentes.

\item \textbf{Se debe limitar la manipulación del punto de vista al utilizar
        elementos en dispositivos móviles}. El usuario debería ser capaz de
    realizar una sola acción a la vez a través de la interfaz gráfica, así, se
    debería permitir al mismo alterar el punto de vista, o utilizar los
    elementos, pero no realizar ambas acciones al mismo tiempo. En las pruebas
    preliminares de la interfaz se detectó que los usuarios tienen problemas al
    manipular el punto de vista mientras utilizan elementos.

\end{itemize}

\section{Implementación del juego serio}

Las conclusiones obtenidas en cuanto a la implementación del juego serio son las 
siguientes:

\begin{itemize}

\item \textbf{Las diferencias principales entre el desarrollo tradicional de software 
        y el desarrollo de juegos serios son la interacción y la utilización de
        gráficos en tres dimensiones}: 

    \begin{itemize}

    \item \textbf{Interacción}: existen diversas formas de interacción con la
        solución, esto implica un desafío al momento de coordinar y realizar las
        pruebas pertinentes.

    \item \textbf{Gráficos en tres dimensiones}: la utilización de gráficos en
        tres dimensiones implica un desafío mayor para los aspectos de diseño,
        en cuanto al aspecto estético como a las posibilidades que posee el
        usuario dentro de la simulación.

    \end{itemize}

\item \textbf{El uso de un motor de videojuego facilita el desarrollo}: crear
    un juego serio desde cero es un proceso sumamente complejo y costoso, la
    utilización de un motor moderno, como \emph{Unity3D}, facilita en gran parte
    el proceso. 

\item \textbf{Se recomienda tener en cuenta el costo, requisitos mínimos,
        familiaridad, librerías, tienda y comunidad al seleccionar un motor de
        videojuego}:
        
    \begin{itemize}

    \item \textbf{Costo de la utilización}: motores como \emph{Unity3D} y
    	\emph{UnrealEngine} poseen planes gratuitos que permiten acceder a 
        la funcionalidad completa para fines educativos.

    \item Familiaridad de los desarrolladores con las tecnologías
        utilizadas con el motor.

    \item \textbf{Librerías}: compatibilidad de librerías de terceros con el
        motor. Durante el desarrollo de la solución existieron librerías que no
        pudieron ser utilizadas con \emph{Unity3D}, estos problemas estaban
        relacionados con la versión de \emph{Mono} utilizada.

    \item \textbf{Tienda y comunidad}: el soporte brindado  por la comunidad,
        así con las librerías y componentes disponibles en la tienda sirven para
        acelerar el desarrollo de un juego serio. Durante el desarrollo de la
        solución se utilizaron varias librerías gratuitas de la tienda, y la
        comunidad es fuente de guías y tutoriales.

    \end{itemize}


\item \textbf{Se recomienda utilizar la guía para el desarrollo de un juego
        serio definida por Pereira\cite{pereira2009design}}: teniendo en cuenta
    que no todos los pasos se aplican a todos los juegos serios. Al desarrollar
    la solución se utilizó la guía proveída en~\cite{pereira2009design}, con
    adecuaciones de acuerdo al contexto de la solución.

\item \textbf{El uso de un motor de reglas condicionado por eventos es
        suficiente para evaluar al usuario}: un motor del tipo \gls{eca} permite
    la evaluación del usuario al momento de realizar las acciones, lo que a su
    vez permite tener acceso al contexto de la acción. 

\item \textbf{Es necesario evaluar al usuario en el front-end}: la evaluación
    del usuario como parte del front-end permite brindar una mayor movilidad y
    posibilita una mayor fluidez en la experiencia al no depender de una
    conexión a internet. 



\item \textbf{Se deben diseñar personajes sólo cuando se requiere un alto nivel
        de detalle o interacción}: las diferentes fuentes de personajes, como
    las tiendas o comunidades en línea, permiten acceder a una gran cantidad de
    modelos de seres humanos. Estos modelos son adecuados para la mayoría de los
    casos, en la solución se requiere un nivel de detalle no encontrado, es por
    ello que en casos particulares se debe diseñar un personaje o adaptar uno ya
    existente. No se encontraron pacientes con el nivel de detalle necesario
    para la simulación del procedimiento de \emph{Glasgow}, por esto, se utilizó
    un personaje base y se agregaron detalles como los ojos y los labios.

\item \textbf{Es necesario enviar automáticamente los registros de utilización}:
    para asegurar el envió de los datos por parte del usuario, los registros
    deben ser enviados automáticamente. En la solución, el usuario debe
    seleccionar la opción de enviar datos, lo que provocó que los datos no sean
    siempre enviados.

\end{itemize}

\section{Evaluación}

Las conclusiones obtenidas en cuanto a la evaluación del juego serio son las 
siguientes:

\begin{itemize}

\item \textbf{Se deben validar las pruebas de conocimiento con profesionales para
        determinar la dificultad y la relevancia de los temas a tratar}: para
    realizar evaluaciones acerca del conocimiento de los alumnos, se debe
    consultar con los profesionales. No basta con utilizar los manuales y otras
    herramientas, pues los profesores son capaces de determinar la dificultad y
    la importancia de cada punto tratado.

\item \textbf{Es necesario registrar todas las acciones del usuario}: el nivel
    de granularidad de las acciones registradas por el juego serio debe ser la
    mayor posible. No sólo sirven para determinar el rendimiento del usuario,
    sino además permiten evaluar el uso de la interfaz, la frecuencia de
    utilización, entre otros aspectos. Es especialmente útil almacenar la mayor
    información posible cuando se realiza el análisis de variables no previstas.

\item \textbf{Los juegos serios ayudan a los estudiantes de enfermería a poner a
        prueba sus conocimientos}: los datos mostrados en la
    sección~\ref{sec:subjetiva} permiten concluir que los estudiantes de
    enfermería consideran que el uso de la solución apoya al estudio en clase y
    laboratorio brindándoles más oportunidades de poner a prueba sus
    conocimientos con respecto a otros materiales utilizados que poseen
    limitaciones físicas y ofreciéndoles un paciente que reacciona a sus
    acciones. 

\item \textbf{Todas las consideraciones definidas en el presente trabajo con respecto
        al diseño y la utilidad de la solución fueron confirmadas}: Las
    valoraciones obtenidas en cada una de las consideraciones pueden observarse en la
    sección~\ref{sec:subjetiva} y nos permiten concluir que los únicos puntos
    débiles fueron la representación iconográfica y la interacción por voz, sin
    embargo todas las consideraciones fueron valoradas positivamente.

\end{itemize}

\section{Correlaciones}

Las correlaciones obtenidas, que se observan en la sección~\ref{sec:correlacion}, muestran
que:

\begin{itemize}

\item Existe una relación entre el tiempo que se utiliza la solución y el
    puntaje obtenido dentro de la solución. Esto podría sugerir que mientras
    más se utiliza la solución, mejor rendimiento se obtiene. Es un punto
    positivo pues muestra que los usuarios aprenden a utilizarla y mejoran con
    el tiempo.

\item Existe una relación positiva fuerte entre el puntaje máximo obtenido
    dentro de la solución y el puntaje en la prueba de conocimiento. Esto podría
    sugerir que los alumnos con mejor rendimiento en la solución, obtuvieron el
    mejor rendimiento en la evaluación.

\item Existe una relación positiva moderada entre el tiempo de utilización del
    procedimiento Venopunción, y la utilización del procedimiento Glasgow, lo
    que sugiere que los usuarios dedicaron un tiempo similar en ambos
    procedimientos.

\item Existe una relación positiva moderada entre el puntaje mayor en el
    procedimiento Venopunción, y el tiempo de juego en el procedimiento Glasgow,
    lo que parece indicar que los usuarios que completaron la mayor parte del
    procedimiento Venopunción, dedicaron más tiempo al procedimiento Glasgow. 

\item La relación positiva muy fuerte que existe entre el puntaje promedio de
    las preguntas de conocimiento relacionadas al procedimiento Glasgow, y el
    puntaje promedio de las preguntas de conocimiento relacionadas al
    procedimiento Venopunción, podría sugerir que el nivel de conocimiento de
    los alumnos sobre ambos procedimientos está relacionado.

\end{itemize}

%\section{Factores}


\subsection{Motivación}

Los datos mostrados en la tabla~\ref{tab:subjetiva_conformidad_motivacion}
permiten concluir el principal factor motivador es la visualización de un
puntaje que resuma el desempeño del usuario. Adicionalmente se observa que otros
factores que influyen son la medición del tiempo y, en menor medida, la
posibilidad de compartir a través de redes sociales el rendimiento.

\subsection{Exploración}

La utilización de factores aleatorios es la que más favorece a la exploración,
como se observa en la sección~\ref{sec:res_subjetiva}, por ejemplo, la
utilización de un paciente con un estado no determinado permite al usuario
explorar las diversas formas que tiene para interactuar con el mismo.

El alcance de la simulación de las herramientas debe incluir solamente lo
necesario para llevar a cabo el objetivo, incluso ciertas actividades pueden no
ser incluidas completamente para mantener la consistencia entre los distintos
elementos.

\subsection{Inmersión}

Escenarios en tres dimensiones similares a los lugares conocidos por los
usuarios ayudan permite a los mismos sentirse parte de la simulación, como se
puede observar en la sección~\ref{sec:res_subjetiva}.

La utilización de varias formas de interacción, como ordenes verbales,
respuestas sonoras, vibraciones, etc, aumentan el nivel de inmersión de los
usuarios.

\subsection{Pedagogía}

El momento en el que se provee información al usuario acerca de sus acciones
tiene un impacto en el potencial pedagógico de un juego serio, si se provee
información muy frecuente se puede obtener una herramienta del tipo \emph{Prueba
    y Error} que poco aporte al aprendizaje. Proveer información acerca del
rendimiento al final de cada escenario, como se indica en la
sección~\ref{sec:res_subjetiva}, tiene un impacto positivo.


\subsection{Representación}

Cuando el usuario interactúa con un objeto, el comportamiento de este objeto
deben ser fácilmente reconocibles, en la sección~\ref{sec:res_subjetiva} se
observa que mientras más vistosa es la reacción del paciente, mejor es la
recepción de los usuarios. Movimientos discretos como la apertura ocular y el
movimiento de los ojos, tienen una pobre aceptación, en cambio, movimientos
vistosos como el movimiento de las piernas, brazos y el cuerpo en general,
tienen una aceptación positiva.

La forma de representar las alteraciones del entorno luego de realizar una
acción deben ser claras.

\subsection{Retroalimentación}

La utilización de imágenes representativas, como mostrar la imagen de un guante
para indicar que el usuario tiene los guantes puestos, es suficiente para
indicar el estado de los objetos.

\subsection{Utilidad}

Los alumnos están de acuerdo con la utilización de este tipo de herramientas
para el apoyo al aprendizaje, como se observa en la
sección~\ref{sec:res_subjetiva} y~\ref{sec:res_subjetiva_abiertas}.

La ubicuidad permite a los usuarios una experiencia más frecuente, como se
observa en la sección~\ref{sec:res_subjetiva}, lo que es considerado por los
usuarios como útil. Esta es una de las principales ventajas de los juegos serios
frente a otras formas de apoyo al aprendizaje como los laboratorios.


%\section{Recomendaciones}

En esta sección se busca dar una orientación en las diversas fases del
desarrollo de aplicaciones educativas similares a la de este trabajo de grado de
acuerdo a la experiencia de los autores y los resultados obtenidos.

\subsection{Diseño de la aplicación}

Es este apartado se detallaran las recomendaciones obtenidas del proceso de
diseño de la solución en diferentes aspectos. 

\begin{itemize}

\item \textbf{Retroalimentación constante con profesores y alumnos.} Es
    importante tener una buena relación con los profesores, y además mantener su
    interés en el proyecto, pues es necesario obtener la mayor cantidad de
    conocimientos y detalles en cuanto al contenido de la aplicación y, sobre
    todo,para que  evalúen cada paso realizado. Los alumnos pueden brindar
    información sobre el contenido y sus necesidades desde otro punto de vista
    totalmente válido pues son los usuarios finales de la aplicación.

\item \textbf{Aprovechar las reuniones con los profesionales.} Debido a la
    apretada agenda que poseen los profesionales, poder reunirse con ellos
    frecuentemente y en un largo periodo de tiempo para validar ideas y
    contenido de la aplicación resulta ser muy difícil, por lo cual es
    importante recabar la máxima cantidad de información en cada reunión.

\item \textbf{Realismo y facilidad de uso de la \Gls{gui}.} En cuanto a la
    interacción del usuario con la aplicación, siempre es preferible que sea lo
    más natural posible. La forma de uso de la aplicación no debería ser un
    obstáculo para utilizarla. La forma en que los elementos son representados y
    utilizados dentro de la aplicación debe ser realista, o en todo caso debe
    representar de la mejor manera la realidad de acuerdo a las limitaciones y
    sin distraer al alumno del objetivo pedagógico.

\end{itemize}


\subsection{Desarrollo de la aplicación}

Es este apartado se detallaran las recomendaciones obtenidas del proceso de
implementación de la solución en diferentes aspectos. 

\begin{itemize}

\item \textbf{Importancia del uso de los motores de videojuegos.} La utilización
    de motores gráficos modernos facilita la creación de juegos serios,
    relacionados a procedimientos del área de enfermería, permiten crear y
    manipular entornos realistas sin demasiadas complicaciones. 

    Una buena elección del motor gráfico de acuerdo a las características de la
    aplicación que se desea implementar es sumamente importante y los criterios
    de selección utilizados en este trabajo y mencionados
    en~\ref{sec:seleccion_plataforma} pueden ser de gran ayuda en la selección.

\item \textbf{Pruebas para mejoras durante el desarrollo.} La realización de
    prueba de usuarios es interesante durante el desarrollo de una aplicación ya
    que permiten visualizarlos errores en su funcionamiento y de esta manera se
    pueden corregir las debilidades encontradas de manera temprana, siempre es
    bueno detectar la mayor cantidad de errores en el menor tiempo posible.

\item \textbf{Validaciones de contenido de la aplicación.} La realización de
    validaciones tanto del contenido como de la representación de la aplicación
    por parte de los profesionales es de suma importancia, debido a que son a
    los estudiantes del área a los que va dirigido la solución. Estas
    validaciones no sólo permiten encontrar errores sino que permiten asegurar
    que la forma de implementación de ciertos aspectos es correcta. 

    Más allá de lo expuesto, estas validaciones con profesionales permiten que
    se sientan parte del proyecto, manteniendo el interés y predisposición de
    los profesionales.

\end{itemize}


\subsection{Evaluación de la aplicación}

Es este apartado se detallaran las recomendaciones obtenidas del proceso de
evaluación de la solución en diferentes aspectos. 

\begin{itemize}

\item \textbf{Evaluación de conocimiento con apoyo de profesionales.} Cuando se
    diseñan encuestas relacionadas con la medición del conocimiento es
    imprescindible elaborarlas con la ayuda y opinión de profesionales en el
    área, ya que tienen experiencia en la medición del conocimiento, esto no
    sólo ayuda a realizar una mejor medición sino que permite controlar que el
    contenido se encuentre dentro del grado de dificultad acorde a los
    estudiantes a los que va dirigido. 

\item \textbf{Importancia de los registros de actividades.} El registro de
    actividades del usuario es una herramienta importante, pues permite
    contrastar los datos obtenidos con otras metodologías, obteniendo
    correlaciones. Además permiten evaluar información interesante sobre la
    utilización de una aplicación, como frecuencia de uso, tiempo de uso, etc.

\item \textbf{Evaluación de la solución por los usuarios.} Se deben diseñar
    encuestas que puedan obtener información que no se puedan apreciar con las
    pruebas de conocimiento y los registros de actividades, como es la
    apreciación subjetiva de los estudiantes en cuanto a la aplicación.

\item \textbf{Mantener interés de los estudiantes.} Los estudiantes que formarán
    parte de la prueba de la aplicación deben estar bien informados sobre el
    objetivo de la aplicación y la validez de la misma, ya que deben estar
    interesados en probarla y ayudar para facilitar la evaluación. 

    Esto es importante sobre todo cuando la naturaleza de la aplicación no
    permite una evaluación controlada, como es el caso de este trabajo.
\end{itemize}

%\section{Otras conclusiones}

En esta sección se detallan conclusiones sobre aspectos transversales 
a los demás puntos tratados.

\subsection{Utilidad de la simulación como herramienta de apoyo}

La mayoría de los estudiantes que formaron parte de la prueba concluyó que el
uso de la solución les parecía útil para complementar el estudio en clase o
laboratorio y que ayuda a comprender los procedimientos.

Los profesores encargados del proceso de aprendizaje de los estudiantes de
enfermería con los que se trabajó en el desarrollo de este trabajo en todo
momento estuvieron abiertos al uso de las \Gls{tic} en la forma de juegos
serios, incluso mencionaron la idea de poder utilizarlos en clase. 

Esto permite concluir que están abiertas las posibilidades de inclusión de la
tecnología en formas más innovadoras con respecto a la forma de utilización en
la actualidad en nuestro país.

\subsection{Mantener interés de los estudiantes}

Los estudiantes que formarán parte de la prueba de la aplicación deben estar
bien informados sobre el objetivo de la aplicación y la validez de la misma, ya
que deben estar interesados en probarla y ayudar para facilitar la evaluación. 

Esto es importante sobre todo cuando la naturaleza de la aplicación no permite
una evaluación controlada, como es el caso de este trabajo.







%\subsection{Comunicación del objetivo a los alumnos}
%\subsection{Utilidad y ventajas}
%\subsection{Que todos tienen móviles}


