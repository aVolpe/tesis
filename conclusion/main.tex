\chapter{Conclusión}
\label{chap:conclusion}


Durante esta trabajo de grado se estudió la utilización de las \Gls{tic} en la
educación, en especial a los juegos serios, y la teoría construccionista del
aprendizaje. Como resultado se diseñó y desarrolló una aplicación para
dispositivos móviles cuyo fin es el de servir como herramienta de apoyo en el
proceso de aprendizaje de los estudiantes de la carrera de enfermería.

Consideramos interesante la investigación de herramientas como la solución
propuesta en este trabajo ya que la tendencia actual es que las tecnologías
tengan un papel más activo en el proceso de enseñanza-aprendizaje, y las nuevas
corrientes pedagógicas requieren que sea así. 

Como se refleja en varios artículos, los estudiantes del nuevo milenio están
acostumbrados a las nuevas tecnologías y tienen otros estilos de aprendizaje,
por lo que debe asumirse el desafío de incorporar a la tecnología con mayor
fuerza, aportando aún más dinamismo en los procesos de enseñanza-aprendizaje.
Creemos además, que el factor motivacional que puede brindar este tipo de
herramientas influirá positivamente en los estudiantes y en el profesores.

A continuación se detallan las conclusiones obtenidas en cada una de las fases del 
desarrollo de la aplicación.


\section{Diseño de la aplicación}

En esta sección se detallan las conclusiones obtenidas en cuanto al proceso de
diseño de la solución en diferentes aspectos. La etapa de diseño se basa en 
crear un prototipo de la aplicación acorde a las características del problema.

\subsection{Retroalimentación constante con profesores y alumnos}

Es valiosa la retroalimentación que pueda obtenerse de parte de los profesores,
así como también la retroalimentación de los estudiantes ya que son estos
últimos los que utilizan la herramienta. Es importante tener una buena relación
con los profesores, y además mantener su interés en el proyecto, pues es
necesario obtener la mayor cantidad de conocimientos y detalles en cuanto al
contenido de la aplicación y, sobre todo, para que los profesores evalúen cada
paso realizado. Los estudiantes pueden brindar información sobre el contenido y
sus necesidades desde otro punto de vista totalmente válido pues son los
usuarios finales de la aplicación.

% COMUNICACIÓN CON PROFESIONALES
Debido a la apretada agenda que poseen los profesionales de salud, poder
reunirse con ellos frecuentemente y en un largo periodo de tiempo para validar
ideas y contenido de la aplicación resulta ser muy difícil, por lo cual es
importante recabar la máxima cantidad de información en cada reunión.

\subsection{Realismo y facilidad de uso de la \Gls{gui}}

En cuanto a la interacción del usuario con la aplicación, siempre es preferible
que sea lo más natural posible. La forma de uso de la aplicación no debería ser
un obstáculo para utilizarla. La forma en que los elementos son representados y
utilizados dentro de la aplicación debe ser realista, o en todo caso debe
representar de la mejor manera la realidad de acuerdo a las limitaciones y sin
distraer al alumno del objetivo pedagógico.

Las hipótesis relacionadas que se detallaron en este trabajo fueron confirmadas
en~\ref{sec:res_subjetiva} con los resultados de las pruebas realizadas. En
cuanto a las limitaciones, tanto tecnológicas como de utilización, presentadas
por los dispositivos móviles dificultan la interacción con el entorno. 

La interacción a través de dispositivos móviles debe ser cuidadosamente diseñada
para proveer de una interacción fluida y significativa.

\subsection{Ubicuidad de los dispositivos móviles}

Según datos obtenidos en la \emph{Encuesta de ubicación} el $98\%$ de los estudiantes 
encuestados cuenta con al menos un dispositivo móvil inteligente por lo que la idea 
de brindarles ubicuidad como uno de los objetivos del presenta trabajo resultó factible, 
no sólo por que contaban con este medio sino por que ello significaba brindarles la 
oportunidad de utilizar la herramienta en cualquier lugar y momento.

De hecho, los usuarios que evaluaron la solución mencionaron en promedio estar
\enquote{De acuerdo} con el uso de dispositivos móviles como tecnología de apoyo
como se muestra en~\ref{sec:res_subjetiva}.

\subsection{Aspectos lúdicos como motivación}

Los juegos serios no sólo permiten al estudiante experimentar, poner a prueba y
adquirir conocimientos sino que, debido a sus características lúdicas, ayuda en
la motivación siendo además una opción diferente con respecto a las demás
opciones tecnológicas utilizadas en el ámbito educativo.
    
En este sentido, en los resultados obtenidos de la evaluación de la solución se
muestra que en promedio los usuarios manifestaron estar \enquote{De acuerdo} con
el efecto motivacional de la solución en cuanto al uso de puntaje, tiempo y
socialización, como se observa en~\ref{sec:res_subjetiva}.



    
%\subsection{Pedagogía}
% (reuniones, escena, opciones, etc)
 
 
%\subsection{Accesibilidad}


%\subsection{Comunicación con profesionales}


%\subsection{Retroalimentación}

%\subsection{Utilización de las \Gls{tic}}

%\subsection{Aspectos Lúdicos}

%! TEX root = ../main.tex

\section{Desarrollo de la aplicación}

En esta sección se detallan las conclusiones obtenidas en cuanto al proceso de
implementación de la solución.

\subsection{Utilizar un motor de videojuegos}

La utilización de motores de videojuegos facilita la creación de juegos serios.
Los motores actuales permiten crear y manipular entornos complejos, desarrollar
aplicaciones para diversas plataformas, y cuentan con librerías para las tareas
comunes (como manipulación de perspectiva, creación de animaciones, etc). 

La utilización de motores de videojuegos permiten al desarrollador centrarse en
aspectos específicos del juego serio y a no invertir el tiempo en aspectos
comunes a los motores, como, un motor de física, detectores de colisiones,
reproducción de contenido multimedia, motores de animaciones, inteligencia
artificial, comunicación, \textit{streaming}, administración de memoria, etc.

En este trabajo se utiliza \textit{Unity3D}, el cual facilita las tareas a
través de un \gls{ide} complejo, diversas herramientas para la creación de
animaciones, y gestión de librerías. Uno de los factores más interesantes de
\textit{Unity3D} es su tienda de librerías, donde se puede adquirir modelos en
tres dimensiones de calidad a precios bajos e incluso de manera gratuita.

\subsection{Seleccionar herramientas de acuerdo a las características necesarias}

Una buena elección del motor gráfico de acuerdo a las características de la
aplicación que se desea implementar es sumamente importante y los criterios de
selección utilizados en este trabajo y mencionados en la
sección~\ref{sec:seleccion_plataforma} pueden ser de gran ayuda en la selección.

Las limitaciones técnicas, de tiempo y monetarias del equipo de desarrollo
también deben ser tenidas en cuenta al momento de la selección de una
herramienta.

Junto a estos criterios, la información proveída en la
sección~\ref{sec:motores}, permitió seleccionar a \textit{Unity3D} como el motor
de videojuego necesario para desarrollar la solución utilizada para evaluar a
los juegos serios en el aprendizaje en este trabajo.

\subsection{Probar el desarrollo de manera frecuente}

En el desarrollo de juegos serios es útil realizar pruebas preliminares para
analizar y evaluar las decisiones de diseño, a fin de minimizar los obstáculos
presentes para los usuarios finales. Los aspectos que se deben probar y validar
son:

\begin{itemize}
    \item Calidad gráfica.
    \item Nivel de detalle del entorno.
    \item Integración con el hardware.
    \item Desenvolvimiento en el entorno a través de la interfaz.
    \item Interacción con objetos y entidades.
    \item Usabilidad de la \gls{gui}.
\end{itemize}

Las pruebas preliminares realizadas a la solución, cuyos resultados se observan
en la sección~\ref{sec:res_interfaz}, permite identificar fortalezas y
debilidades. 

En el caso específico de la solución, los problemas detectados incluyen falta de
naturalidad en la utilización de objetos y falta de intuitividad en el uso de la
\Gls{gui}. Estos problemas son solucionados en la versión que se presenta a los
usuarios finales.

La realización de prueba con usuarios es útil durante el desarrollo de una
aplicación ya que permiten visualizar los errores en su funcionamiento y de esta
manera se pueden corregir las debilidades encontradas de manera temprana,
siempre es mejor detectar la mayor cantidad de errores en el menor tiempo
posible.

En el caso de la solución, se realizaron pruebas de \Gls{gui} para mejorar la
interacción de los usuarios con ella, estas pruebas revelaron principalmente
problemas con la interacción con el entorno y con los objetos. Como resultado se
aplicaron mejoras que permitieron mejorar el uso y la intuitividad de la
interfaz.

\subsection{Crear entornos en tres dimensiones para aumentar la inmersión}

Escenarios en tres dimensiones similares a los lugares conocidos por los
usuarios ayudan permite a los mismos sentirse parte de la simulación. 

En la solución presentada en este trabajo, se utiliza un escenario en tres
dimensiones similar a un escenario conocido por los usuarios finales (un
laboratorio de prácticas), el mismo fue diseño en base a observaciones
realizadas en una visita guiada por profesionales del \gls{iab}. En la
sección~\ref{sec:res_subjetiva} se observa que los usuarios valoran
positivamente la influencia del escenario en la inmersión.

\subsection{Proveer retroalimentación clara al realizar una acción}

La forma de notificar al usuario en el momento en el que realiza una acción,
debe ser fácilmente observable y distinguible, de esta manera el usuario puede
tener la certeza de que se realizo la acción deseada.  

En la solución, cuando el usuario interactúa con un objeto, en la
sección~\ref{sec:res_subjetiva} se observa que mientras más vistosa es la
reacción del paciente, mejor es la recepción de los usuarios. Movimientos
discretos como la apertura ocular y el movimiento de los ojos, tienen una pobre
aceptación, en cambio, movimientos vistosos como el movimiento de las piernas,
brazos y el cuerpo en general, tienen una aceptación positiva.

La forma de representar las alteraciones del entorno luego de realizar una
acción deben ser claras.


\section{Evaluación de la aplicación}

En esta sección se detallan las conclusiones obtenidas en cuanto al proceso de
de evaluación de la solución en diferentes aspectos.

\subsection{Evaluación de conocimiento con apoyo de profesionales}

Cuando se diseñan encuestas relacionadas con la medición del conocimiento, en
nuestro caso de estudiantes de la carrera de Licenciatura en Enfermería, es
imprescindible elaborarlas con la ayuda y opinión de profesionales en el área,
ya que tienen experiencia en la medición del conocimiento, esto no sólo ayuda a
realizar una mejor medición sino que permite controlar que el contenido se
encuentre dentro del grado de dificultad acorde a los estudiantes a los que va
dirigido. 

En el caso de este trabajo, la \emph{Encuesta Objetiva} está orientada a este
objetivo, debido a que las preguntas se basaban en respuestas de selección
múltiple con una sola respuesta verdadera, los profesionales ayudaron a evitar
que las formulaciones y la lista de respuestas sean ambiguas o confusas.


\subsection{Importancia de los registros de actividades}

El registro de actividades del usuario es una herramienta importante, pues
permite contrastar los datos obtenidos con otras metodologías, obteniendo
correlaciones. Además permiten evaluar información interesante sobre la
utilización de una aplicación, como frecuencia de uso, tiempo de uso, etc.

Estas herramientas deben ser transparentes para el usuario y deben ser capaces
de funcionar aún sin acceso constante a internet.

\subsection{Evaluación de la solución por los usuarios}

Se deben diseñar encuestas que puedan obtener información que no se puedan
apreciar con las pruebas de conocimiento y los registros de actividades, como es
la apreciación subjetiva de los estudiantes en cuanto a la aplicación en
diferentes aspectos como utilidad, motivación, representación, pedagogía, entre
otros. En el caso de este trabajo, sirvió para validar algunas hipótesis
planteadas.


%\subsection{Como crear la evaluación (Opinión de los putitos)}
%\subsection{Logs}
%\subsection{Utilidad de las encuestas}
%\subsection{Proponer criterios que puedan servir para el desarrollo de trabajos
%    similares}

\section{Otras conclusiones}

En esta sección se detallan conclusiones sobre aspectos transversales 
a los demás puntos tratados.

\subsection{Utilidad de la simulación como herramienta de apoyo}

La mayoría de los estudiantes que formaron parte de la prueba concluyó que el
uso de la solución les parecía útil para complementar el estudio en clase o
laboratorio y que ayuda a comprender los procedimientos.

Los profesores encargados del proceso de aprendizaje de los estudiantes de
enfermería con los que se trabajó en el desarrollo de este trabajo en todo
momento estuvieron abiertos al uso de las \Gls{tic} en la forma de juegos
serios, incluso mencionaron la idea de poder utilizarlos en clase. 

Esto permite concluir que están abiertas las posibilidades de inclusión de la
tecnología en formas más innovadoras con respecto a la forma de utilización en
la actualidad en nuestro país.

\subsection{Mantener interés de los estudiantes}

Los estudiantes que formarán parte de la prueba de la aplicación deben estar
bien informados sobre el objetivo de la aplicación y la validez de la misma, ya
que deben estar interesados en probarla y ayudar para facilitar la evaluación. 

Esto es importante sobre todo cuando la naturaleza de la aplicación no permite
una evaluación controlada, como es el caso de este trabajo.







%\subsection{Comunicación del objetivo a los alumnos}
%\subsection{Utilidad y ventajas}
%\subsection{Que todos tienen móviles}

