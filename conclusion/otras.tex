\section{Otras conclusiones}

En esta sección se detallan conclusiones sobre aspectos transversales 
a los demás puntos tratados.

\subsection{Utilidad de la simulación como herramienta de apoyo}

La mayoría de los estudiantes que formaron parte de la prueba concluyó que el
uso de la solución les parecía útil para complementar el estudio en clase o
laboratorio y que ayuda a comprender los procedimientos.

Los profesores encargados del proceso de aprendizaje de los estudiantes de
enfermería con los que se trabajó en el desarrollo de este trabajo en todo
momento estuvieron abiertos al uso de las \Gls{tic} en la forma de juegos
serios, incluso mencionaron la idea de poder utilizarlos en clase. 

Esto permite concluir que están abiertas las posibilidades de inclusión de la
tecnología en formas más innovadoras con respecto a la forma de utilización en
la actualidad en nuestro país.

\subsection{Mantener interés de los estudiantes}

Los estudiantes que formarán parte de la prueba de la aplicación deben estar
bien informados sobre el objetivo de la aplicación y la validez de la misma, ya
que deben estar interesados en probarla y ayudar para facilitar la evaluación. 

Esto es importante sobre todo cuando la naturaleza de la aplicación no permite
una evaluación controlada, como es el caso de este trabajo.







%\subsection{Comunicación del objetivo a los alumnos}
%\subsection{Utilidad y ventajas}
%\subsection{Que todos tienen móviles}
