\section{Factores}


\subsection{Motivación}

Los datos mostrados en la tabla~\ref{tab:subjetiva_conformidad_motivacion}
permiten concluir el principal factor motivador es la visualización de un
puntaje que resuma el desempeño del usuario. Adicionalmente se observa que otros
factores que influyen son la medición del tiempo y, en menor medida, la
posibilidad de compartir a través de redes sociales el rendimiento.

\subsection{Exploración}

La utilización de factores aleatorios es la que más favorece a la exploración,
como se observa en la sección~\ref{sec:res_subjetiva}, por ejemplo, la
utilización de un paciente con un estado no determinado permite al usuario
explorar las diversas formas que tiene para interactuar con el mismo.

El alcance de la simulación de las herramientas debe incluir solamente lo
necesario para llevar a cabo el objetivo, incluso ciertas actividades pueden no
ser incluidas completamente para mantener la consistencia entre los distintos
elementos.

\subsection{Inmersión}

Escenarios en tres dimensiones similares a los lugares conocidos por los
usuarios ayudan permite a los mismos sentirse parte de la simulación, como se
puede observar en la sección~\ref{sec:res_subjetiva}.

La utilización de varias formas de interacción, como ordenes verbales,
respuestas sonoras, vibraciones, etc, aumentan el nivel de inmersión de los
usuarios.

\subsection{Pedagogía}

El momento en el que se provee información al usuario acerca de sus acciones
tiene un impacto en el potencial pedagógico de un juego serio, si se provee
información muy frecuente se puede obtener una herramienta del tipo \emph{Prueba
    y Error} que poco aporte al aprendizaje. Proveer información acerca del
rendimiento al final de cada escenario, como se indica en la
sección~\ref{sec:res_subjetiva}, tiene un impacto positivo.


\subsection{Representación}

Cuando el usuario interactúa con un objeto, el comportamiento de este objeto
deben ser fácilmente reconocibles, en la sección~\ref{sec:res_subjetiva} se
observa que mientras más vistosa es la reacción del paciente, mejor es la
recepción de los usuarios. Movimientos discretos como la apertura ocular y el
movimiento de los ojos, tienen una pobre aceptación, en cambio, movimientos
vistosos como el movimiento de las piernas, brazos y el cuerpo en general,
tienen una aceptación positiva.

La forma de representar las alteraciones del entorno luego de realizar una
acción deben ser claras.

\subsection{Retroalimentación}

La utilización de imágenes representativas, como mostrar la imagen de un guante
para indicar que el usuario tiene los guantes puestos, es suficiente para
indicar el estado de los objetos.

\subsection{Utilidad}

Los alumnos están de acuerdo con la utilización de este tipo de herramientas
para el apoyo al aprendizaje, como se observa en la
sección~\ref{sec:res_subjetiva} y~\ref{sec:res_subjetiva_abiertas}.

La ubicuidad permite a los usuarios una experiencia más frecuente, como se
observa en la sección~\ref{sec:res_subjetiva}, lo que es considerado por los
usuarios como útil. Esta es una de las principales ventajas de los juegos serios
frente a otras formas de apoyo al aprendizaje como los laboratorios.

