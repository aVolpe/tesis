\chapter{Conclusion}

\section{Capítulo 1}

\textbf{Nakore}

\section{Capítulo 2}

\begin{itemize}
\item El rol de las \gls{tic} en la educación sigue siendo mayormente como
    proveedor de conocimiento. Existen proyectos como \gls{olpc} que fomentan el
    uso de las \gls{tic} con un rol activo en el aprendizaje pero esto aún no se
    da en la mayoría de los casos.
\item La utilización de las \gls{tic} en la educación aumento pero no es una
    tendencia.
\item Aumento de publicaciones relacionadas a los juegos serios en un $25\%$
    (pasa de 13900 en 2013 a 17500 en el 2015). 
\end{itemize}

\section{Capítulo 3}

\begin{itemize}
\item Los juegos serios permiten educar sin riesgos a la salud y a la
    integridad.
\item La mayoría de los juegos serio actuales brindan una retroalimentación muy
    guiada, induciendo a los usuarios a hacer lo que es correcto sin darle
    demasiado espacio para desarrollar su pensamiento critico y toma de
    decisiones. Estos juegos serios tienden a ser \textbf{edutainment}.
\item Los juegos serios permiten al estudiante experimentar, poner a prueba y
    adquirir conocimientos, y son especialmente efectivos para el aprendizaje
    del pensamiento de alto nivel.
\end{itemize}

\section{Capítulo 4}

\begin{itemize}
\item La enseñanza de profesionales de enfermería es un buen área de aplicación
    de los juegos serios, especialmente en Paraguay\todox{Justificar esto}.
\item Las herramientas actuales de enseñanza de profesionales de enfermería en
    el IAB no son suficientes para los estudiantes\cite{iab:tesis_atencion}.
\item La cantidad de alumnos en el IAB dificulta la enseñanza
    personalizada\cite{iab:tesis_alumnos}.
\item Los alumnos no siempre están suficientemente preparados en su primera
    práctica con personas\cite{iab:tesis_alumnos}.
\item Los alumnos invierten mucho tiempo en el transporte\cite{iab:tesis_alumnos}.
\item Los estudiantes de enfermaría requieren una alternativa a la
    personalización de la enseñanza ya que la no personalización debido a la
    gran cantidad de alumnos es un problema para ellos.
\item Las estudiantes disponen de poco tiempo libre debido a sus actividades
    académicas, por lo que proveerles una herramienta para usarla en este tiempo
    disponible y para apoyar su aprendizaje resulta interesante.
\item El nivel de acceso a la tecnología es bajo con respecto a las demás
    carreras, por lo que dar soluciones como laboratorios complejos de realidad
    virtual no es factible actualmente.
    \begin{itemize}
        \item Si bien aumento, no es muy alto.
        \item Aguantaría software con menos requisitos.
    \end{itemize}

\end{itemize}

\section{Capítulo 5}

\subsection{Selección de temas}

\begin{itemize}

\item Estudiar las limitaciones tecnológicas.
\item Simular actividades que puedan ser corregidas fácilmente.
\item Estudiar el plan de estudios conjuntamente con los profesionales para
    investigar temas, los roles necesarios son:
    \begin{itemize}
        \item Profesor instructor de laboratorio, se encarga de preparar a los
            alumnos para las prácticas.
        \item Profesor instructor de práctica de campo, pues es el encargado de
            monitorizar al alumno en su primera práctica.
        \item Director de carrera, tiene una visión global de los puntos débiles
            y fuerte, si bien no con mucho detalle, puede guiar hacia
            asignaturas específicas.
    \end{itemize}.
\item Esclarecer los aspectos que son de un procedimiento y cuales son de otro
    procedimiento, por ejemplo, en al enfermería la mayoría del los
    procedimientos requiere que el enfermero se calce los guantes, el proceso de
    calzarse los guantes, es de por sí otro procedimiento.
\item Definir los criterios para la selección de los temas a simular con
    profesionales, aspectos como la importancia deben ser apreciados por los
    instructores de campo.

\end{itemize}

\subsection{Alcance} 

\begin{itemize}

\item Utilizar más de un medio de interacción (tacto, vista, audio, voz,
    vibraciones, etc) para aumentar la inmersión.
\item Exceso de detalles perjudica al objetivo pedagógico.
    \begin{itemize}
    \item Limitar la libertad de la cámara hacia los aspectos deseados, por
        ejemplo no permitir recorrer todo el salon, por que esto requiere un
        mayor nivel de detalle y distrae, \emph{permitir siempre un nivel de
            exploración}.
    \item Es suficiente solo la representación de las funciones de un elemento
        que van a ser usadas en el procedimiento, no se requiere la complejidad
        de simular todo lo que puede hacerse con él. Seleccionar las funciones
        de acuerdo al objetivo pedagógico.
    \end{itemize}
\item Buscar que el usuario pueda utilizar la solución en cualquier momento, los
    datos muestra un pico en horario extracurricular y actividad los días
    sábados y domingos.
\item Todas las hipótesis acerca de representación, facilidad de uso,
    exploración e intuitividad referente a la interfaz de usuario fueron
    confirmadas.
\item Para usar comandos verbales se deben considerar factores como el sonido
    ambiente y el volumen.

\end{itemize}

\section{Capítulo 6}

\begin{itemize}

\item Utilizar tecnología transparente para el usuario al momento de recabar
    datos. Me refiero a registrar datos automáticamente.
\item Características deseadas de un motor:
\begin{itemize}
    \item Tienda y comunidad grande.
    \item Lenguaje de programación familiar y con soporte a librerías creadas
        fueras del motor.
    \item Integración con software de diseño de terceros.
    \item Bajos requerimientos mínimos del motor en la plataforma deseada.
    \item Si se utiliza para desarrollar móvil ver la madurez de sus
        herramientas de depuración.
\end{itemize}
\item Buscar soluciones gratuitas si es necesario desplegar contenido que este
    disponible 24 horas (como OpenShift)

\end{itemize}
\section{Capítulo 7}

\begin{itemize}

\item Seguir la guía definida en \textbf{LivingForest}, pero tener cuidado que no todos
    los pasos se aplican a todos los juegos serios.
\item Prestar especial atención al desarrollo de la interacción con el punto de
    vista.
  \item Utilizar librerías ya disponibles.
  \item Permitir mover y zoom, no rotar.
\item Utilizar un Motor de reglas del tipo ECA para evaluar al usuario.
\item Utilizar eventos para facilitar la integración con motores ECA y registros de
  acciones.
\item Utilizar iconos para describir el estado de las entidades.
\item Si se utiliza un sistema de reconocimiento de sonido, tener en cuenta el
  sonido ambiente base.
\item Utilizar re saltadores para hacer visibles aspectos invisibles (como la
  esterilización de una zona)
\item Evaluar al usuario lo más pronto posible para no perder información de
  contexto.
\item Cada acción del usuario debe ser visible para informar al usuario que fue
  realizada.
\item Generar toda la información posible para diagnosticar el error del usuario,
  pero solo dar pistas del error.
\item Bloquear la manipulación del punto de vista del usuario mientras el mismo
  utiliza objetos y herramientas.
\item Utilizar una puntuación para motivar (el algoritmo no importa, lo que importa
  es que refleja el puntaje, es decir no dar puntos por pasos sino dar una
  puntuación global)
\item Utilizar tiempo  de juego para motivar
\item Registrar la actividad por sesión, de manera transparente y enviar si se
  puede.
\item Registrar no solo la actividad dentro de la simulación, sino la interacción
  con los menús y GUI en general.
  
\item La creación de pacientes es preferible cuando existen detalles muy específicos
  requeridos para los procedimientos como movimientos de ojos, bocas, parpados,
  los cuales ademas ayudan al realismo. 
\item La evaluación de las reglas como parte de una fronted es la alternativa
  adecuada cuando se tiene por objetivo darle al usuario la mayor ubicuidad
  posible, además afectarle económicamente lo menos posible, esto también
  posibilita una mayor fluidez en la experiencia de usuario.
\item Incluir el factor de aleatoriedad en el paciente ayuda a poner a prueba los
  conocimientos del usuario y a aumentar el realismo.
\item Incluir factores lúdicos como un puntaje o medición del tiempo de la partida
  motiva a que los usuarios sigan jugando para perfeccionarse.
\item La retroalimentación brindada a los usuarios debe detallar las razones por las
  cuales el jugador realizo erróneamente un paso, no solo informarle de su error
  diciéndole cual fue. 
\end{itemize}

\section{Capítulo 8}


\todox{Diferenciar de la IS genérica}

\begin{itemize}

\item Evaluar rápido
\item Las hipótesis (son 6)
\item Realizar pruebas con muchas personas es difícil por la falta de tiempo de
    los profesionales.
\item Evaluar la velocidad de aprendizaje para ver intuitividad de la interfaz.
    (primera vez vs siguientes veces)
\item Encuestas libres para medir opinión.


\item El uso de simulaciones ayuda en el complemento del estudio en clase y
    laboratorio, brindar una herramienta que puede utilizarse en cualquier lugar
    y momento provee más facilidades que prestar un libro o intentar practicar
    en laboratorio en el tiempo libre.
\item El resultado de la encuesta para medir el conocimiento, si bien no posee
    números estadísticamente aceptables, sugieren una leve mejoría en el puntaje
    obtenido por la muestra con respecto a sus demás compañeros en las preguntas
    relacionadas al procedimiento de extracción de muestras de sangre, la cual
    fue superiormente la mas jugada, lo que puede indicar que la solución, ya
    sea por utilidad o por motivación o por interés ante  una herramienta poco
    usual, podría influir positivamente en el ámbito académico de sus usuarios.
\item La solución beneficia como apoyo al proceso de aprendizaje brindando una
    posibilidad más de poner en practica los conocimientos y acercando a los
    estudiantes al ámbito tecnológico.
  
\item La correlación sugiere una relación entre el puntaje obtenido en el examen
    de conocimiento y el tiempo de utilización de la solución.
\end{itemize}

\section{Capítulo 9}

\subsection{Diseño de juego serios}

\begin{itemize}
\item Validar constantemente el contenido con profesionales del área
\item Corroborar con profesionales los aspectos pedagógicos, el nivel de detalle
    por paso y las características del entorno para validar y detectar errores
    en el diseño. 
\item Motivar a través de la utilización de aspectos lúdicos, como el puntaje y
    la medición del tiempo
\item Utilizar escenas generadas aleatoriamente para favorecer la exploración
\item Proveer retroalimentación al finalizar cada escenario
\item Proveer ubicuidad para aumentar la utilización
\item Es suficiente una representación gráfica para representar el estado de una
    entidad.
\end{itemize}
 
\subsection{Desarrollo de la aplicación}

\begin{itemize}
\item Diferencias con el desarrollo tradicional de software (3D, input 3d)
\item Utilizar un motor de videojuegos
\item Criterios de selección del motor: costo, familiaridad, librerías, tienda,
    plataformas soportadas, y comunidad.
\item Probar el desarrollo de manera frecuente
\item Toda acción debe tener un indicador visual evidente de que fue realizada.
\end{itemize}
 
\subsection{Evaluación de la aplicación}

\begin{itemize}
\item Evaluación de conocimiento con apoyo de profesionales
\item Importancia de los registros de actividades
\item Evaluación de la solución por los usuarios
\end{itemize}

\subsection{Puntos a ver}

\begin{itemize}
\item Ver lo de juegos serios de corta duración
\item Expandir la sección de la selección de motor (lo de motor, lo que dice
    sobre las librerias y el \gls{ide})
\end{itemize}

\section{Capítulo 10}

\subsection{Diseño de juego serios}

\begin{itemize}
\item Validar constantemente el contenido con profesionales del área
\item Motivar a través de la utilización de aspectos lúdicos, como el puntaje y la medición del tiempo
\item Utilizar escenarios aleatorios para favorecer la exploración
\item Proveer retroalimentación al finalizar cada escenario
\item Proveer ubicuidad para aumentar la utilización
\item Imágenes representativas son útiles para indicar el estado de entidades
\end{itemize}

\subsection{Desarrollo de la aplicación}

\begin{itemize}
\item Utilizar un motor de videojuegos
\item Seleccionar herramientas de acuerdo a las características necesarias
\item Probar el desarrollo de manera frecuente
\item Crear entornos en tres dimensiones para aumentar la inmersión
\item Proveer retroalimentación clara al realizar una acción
\end{itemize}

\subsection{Evaluación de la aplicación}

\begin{itemize}
\item Evaluación de conocimiento con apoyo de profesionales
\item Importancia de los registros de actividades
\item Evaluación de la solución por los usuarios
\end{itemize}

\section{Capítulo 11}
