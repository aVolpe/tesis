\section{Estado del arte}

Las conclusiones obtenidas en cuanto a la investigación del estado del arte son 
las siguientes:

\begin{itemize}

% REVISAR
\item \textbf{Los beneficios de las \gls{tic} aún no hay sido explotados
        completamente en el área  educativa}: El rol de las \gls{tic} en la
    educación sigue siendo mayormente como proveedor de conocimiento debido a
    que el instruccionismo es aún la corriente pedagógica con mayor vigencia.

\item \textbf{Los juegos serios tienden a ofrecer una retroalimentación muy
        guiada}: La mayoría de los juegos serios existentes brindan una
    retroalimentación muy guiada al usuario induciéndolo a hacer lo correcto sin
    brindarle suficiente espacio para que desarrolle su pensamiento crítico y
    toma de decisiones.

\item \textbf{Los juegos serios permiten una experiencia sin riesgos reales}:
    permiten  al usuario experimentar, poner a prueba y construir conocimientos
    sin los riesgos económicos y de salud presentes en la vida real.

% Ver si vale la pena citar
\item \textbf{La enseñanza de profesionales de enfermería es un área propicia
        para la aplicación de los juegos serios}: los estudiantes de enfermería
    requieren un alto grado de prácticas y poseen poco tiempo disponible para
    actividades ajenas a las académicas impuestas por su centro de estudios.

\item \textbf{Utilizar herramientas alternativas de bajo costo}: el nivel de
    acceso a la tecnología de los estudiantes del \gls{iab} es bajo. Proveer
    soluciones que requieran una inversión importante no es una alternativa
    viable actualmente.

\end{itemize}

\section{Diseño del juego serio}

Las conclusiones obtenidas en cuanto al diseño del juego serio son las siguientes:

\begin{itemize}


\item \textbf{Definir los aspectos pedagógicos, nivel de detalle requerido por
        procedimiento y las características del entorno con profesores de
        prácticas y directores de carrera}:
    Las validaciones constantes de estos aspectos ayudan a la detección de
    errores en el diseño, un ejemplo es la evaluación del rendimiento del alumno
    ya que existen maneras adecuadas de realizar los procedimientos.

\item \textbf{La utilización del puntaje por procedimiento motiva a los
        usuarios}: Los juegos serios poseen aspecto lúdicos inherentes que
    motivan al usuario en su uso, agregar indicadores de rendimiento ayudan a
    esta motivación. Los datos mostrados en la sección~\ref{sec:subjetiva}
    permiten concluir que el principal factor motivacional es la visualización
    de un puntaje que resuma el desempeño del usuario y, en menor medida, la
    medición del tiempo y la posibilidad de compartir su rendimiento en las
    redes sociales. 

\item \textbf{Facilitar la exploración con estados aleatorios y funciones
        simplificadas}: Permitir al usuario explorar el entorno para resolver
    problemas ayuda en el desarrollo de su pensamiento crítico, así también, se
    debe ofrecer una interfaz intuitiva y simple para no obstaculizar esta
    exploración. Los datos mostrados en la sección~\ref{sec:subjetiva} permiten
    concluir que los principales factores que favorecen la exploración son la
    aleatoriedad en el estado del paciente y la representación simplificada de
    las funciones de los elementos. Sin embargo, esta simplificación debe
    diseñarse con mucho cuidado para no perder intuitividad en la interfaz. 

\item \textbf{Utilizar gráficos en tres dimensiones y partidas cortas para
        aumentar la inmersión del usuario}: Es importante que el usuario sienta
    que está inmerso en el juego para que de este modo pueda percibir el
    contexto que se le está presentado y actúe de acuerdo a esto. Los datos
    mostrados en la sección~\ref{sec:subjetiva} permiten concluir que la
    utilización de entornos en tres dimensiones para la representación de
    elementos y lugares a los que están familiarizados el usuario, así como
    proveer partidas cortas para evitar que el usuario pierda el contexto de sus
    acciones, favorecen a la inmersión. 

\item \textbf{Proveer retroalimentación sobre el desempeño del usuario sólo al finalizar 
    la partida}: El momento en el que se provee información acerca del desempeño 
    del usuario tiene un impacto en el potencial pedagógico de un juego serio, 
    si se provee información de manera muy frecuente se puede obtener herramientas 
    del tipo prueba y error que poco aporte al aprendizaje. Los datos mostrados en la
    sección~\ref{sec:subjetiva} permiten concluir que ofrecerles a los usuarios
    una simulación de los procedimientos con una retroalimentación limitada les
    ayuda a poner en práctica sus conocimientos y a comprender el procedimiento. 

\item \textbf{Ofrecer información detallada sobre el rendimiento del usuario}:
    Los datos mostrados en la sección~\ref{sec:subjetiva} permiten concluir que
    los usuarios están de acuerdo con el hecho de proporcionarles una
    retroalimentación indicándoles los pasos que realizó de manera correcta e
    incorrecta dentro del procedimiento. Sin embargo, una breve causa acerca de
    las equivocaciones no es suficiente, se requiere información detallada. 

\end{itemize}

\section{Implementación del juego serio}

Las conclusiones obtenidas en cuanto a la implementación del juego serio son las 
siguientes:

\begin{itemize}

\item \textbf{Las diferencias principales entre el desarrollo tradicional de software 
        y el desarrollo de juegos serios son la interacción y la utilización de
        gráficos en tres dimensiones}: 

    \begin{itemize}

    \item \textbf{Interacción}: existen diversas formas de interacción con la
        solución, esto implica un desafío al momento de coordinar y realizar las
        pruebas pertinentes.

    \item \textbf{Gráficos en tres dimensiones}: La utilización de gráficos en
        tres dimensiones implica un desafío mayor para los aspectos de diseño,
        en cuanto al aspecto estético como a las posibilidades que posee el
        usuario dentro de la simulación.

    \end{itemize}

    \item \textbf{Utilizar un motor de videojuegos para facilitar el
            desarrollo}: crear un juego serio desde cero es un proceso sumamente
        complejo y costoso, la utilización de un motor moderno, como
        \emph{Unity3D}, facilita en gran parte el proceso. 

    \item \textbf{Tener en cuenta el costo, requisitos mínimos, familiaridad,
        librerías, tienda y comunidad al seleccionar un motor de videojuegos}:
        
    \begin{itemize}

    \item \textbf{Costo de la utilización}: Muchos motores, como \emph{Unity3D}
        , y \emph{UnrealEngine} poseen planes educativos, los cuales permiten acceder a 
        la funcionalidad completa para fines no comerciales.

    \item Familiaridad de los desarrolladores con las tecnologías
        utilizadas con el motor.

    \item \textbf{Librerías}: compatibilidad de librerías de terceros con el
        motor. Durante el desarrollo de la solución existieron librerías que no
        pudieron ser utilizadas con \emph{Unity3D}, estos problemas estaban
        relacionados con la versión de \emph{Mono} utilizada.

    \item \textbf{Tienda y comunidad}: el soporte brindado  por la comunidad,
        así con las librerías y componentes disponibles en la tienda sirven para
        acelerar el desarrollo de un juego serio. Durante el desarrollo de la
        solución se utilizaron varias librerías gratuitas de la tienda, y la
        comunidad es fuente de guías y tutoriales.

    \end{itemize}


\item \textbf{Utilizar indicadores de realización de acciones}: en la simulación de
    entornos de enfermería, no todas las acciones son visibles ante el ojo
    humano, para estos casos, se debe diseñar un esquema que notifique al
    usuario sobre la realización de una acción. En la solución se utilizan
    indicadores transparentes.

\item \textbf{Utilizar la guía definida por~\cite{pereira2009design}, tener en
        cuenta que no todos los pasos se aplican a todos los juegos serios}: Una
    de las características de los juegos serios es su dependencia del contexto,
    esto es también aplicable a su desarrollo. Al desarrollar la solución se
    utilizó la guía proveída en~\cite{pereira2009design}, con adecuaciones de
    acuerdo al contexto de la solución.

\item \textbf{Utilizar un motor de reglas condicionado por eventos}: un motor
    del tipo \gls{eca} permite la evaluación del usuario al momento de realizar
    las acciones, lo que a su vez permite tener acceso al contexto de la acción. 

\item \textbf{Evaluar al usuario en el front-end}: la evaluación del usuario
    como parte del front-end permite brindar una mayor movilidad y posibilita
    una mayor fluidez en la experiencia al no depender de una conexión a
    internet. 

\item \textbf{Limitar manipulación del punto de vista al utilizar elementos en dispositivos móviles}: el
    usuario debería ser capaz de realizar una sola acción a la vez a través de
    la interfaz gráfica, así, se debería permitir al mismo alterar el punto de
    vista, o utilizar los elementos, pero no realizar ambas acciones al mismo
    tiempo. En las pruebas preliminares de la interfaz se detectó que los
    usuarios tienen problemas al manipular el punto de vista mientras
    utilizan elementos.

\item \textbf{Diseñar personajes solo cuando se requiere un alto nivel de
        detalle o interacción}: las diferentes fuentes de personajes, como las
    tiendas o comunidades en línea, permiten acceder a una gran cantidad de
    modelos de seres humanos. Estos modelos son adecuados para la mayoría de los
    casos, en la solución se requiere un nivel de detalle no encontrado, es por
    ello que en casos particulares se debe diseñar un personaje o adaptar uno ya
    existente. No se encontraron pacientes con el nivel de detalle necesario
    para la simulación del procedimiento de \emph{Glasgow}, por esto, se utilizó
    un personaje base y se agregaron detalles como los ojos y los labios.

\item \textbf{Enviar automáticamente los registros de utilización}: para asegurar el envió de los datos por parte del 
    usuario, los registros deben ser enviados automáticamente. En la solución, el 
    usuario debe seleccionar la opción de enviar datos, lo que provocó que los datos 
    no sean siempre enviados.

\end{itemize}

\section{Evaluación}

Las conclusiones obtenidas en cuanto a la evaluación del juego serio son las 
siguientes:

\begin{itemize}

\item \textbf{Validar pruebas de conocimiento con profesionales para determinar la
        dificultad y la relevancia de los temas a tratar}: Para realizar
    evaluaciones acerca del conocimiento de los alumnos, se debe consultar con
    los profesionales. No basta con utilizar los manuales y otras herramientas,
    pues los profesores son capaces de determinar la dificultad y la importancia
    de cada punto tratado.

\item \textbf{Registrar todas las acciones del usuario}: El nivel de
    granularidad de las acciones registradas por el juego serio debe ser la
    mayor posible. No solo sirven para determinar el rendimiento del usuario,
    sino además permiten evaluar el uso de la interfaz, la frecuencia de
    utilización, entre otros aspectos. Es especialmente útil almacenar la mayor
    información posible cuando se realiza el análisis de variables no previstas.

\item \textbf{Los juegos serios ayudan a los estudiantes de enfermería a poner a
        prueba sus conocimientos}: Los datos mostrados en la
    sección~\ref{sec:subjetiva} permiten concluir que los estudiantes de
    enfermería consideran que el uso de la solución apoya al estudio en clase y
    laboratorio brindándoles más oportunidades de poner a prueba sus
    conocimientos con respecto a otros materiales utilizados que poseen
    limitaciones físicas y ofreciéndoles un paciente que reacciona a sus
    acciones. 

\item \textbf{Todas las hipótesis definidas en el presente trabajo con respecto
        al diseño y la utilidad de la solución fueron confirmadas}: Las
    valoraciones obtenidas en cada una de las hipótesis pueden observarse en la
    sección~\ref{sec:subjetiva} y nos permiten concluir que los únicos puntos
    débiles fueron la representación iconográfica y la interacción por voz, sin
    embargo todas las hipótesis fueron valoradas positivamente.

\end{itemize}

\section{Correlaciones}

Las correlaciones que se observan en la sección~\ref{sec:correlacion}, muestran
que:

\begin{itemize}

\item Existe una relación entre el tiempo que se utiliza la solución y el
    puntaje obtenido dentro de la solución. Esto sugiere que mientras más se
    utiliza la solución, mejor rendimiento se obtiene. Es un punto positivo pues
    muestra que los usuarios aprenden a utilizarla y mejoran con el tiempo.

\item Existe una relación positiva fuerte entre el puntaje máximo
    obtenido dentro de la solución y el puntaje en la prueba de conocimiento. Es
    decir, mientras mejor rendimiento se tiene dentro de la solución, mejor
    rendimiento se tiene en la evaluación.

\item Existe una relación moderada entre el tiempo de utilización del
    procedimiento Venopunción, y la utilización del procedimiento Glasgow, lo
    que sugiere que los usuarios dedicaron un tiempo similar en ambos
    procedimientos.

\item Existe una relación positiva moderada entre el puntaje mayor en el
    procedimiento Venopunción, y el tiempo de juego en el procedimiento Glasgow,
    lo que parece indicar que los usuarios que completaron la mayor parte del
    procedimiento Venopunción, dedicaron más tiempo al procedimiento Glasgow. 

\item La relación positiva muy fuerte que existe entre el puntaje promedio de
    las preguntas de conocimiento relacionadas al procedimiento Glasgow, y el
    puntaje promedio de las preguntas de conocimiento relacionadas al
    procedimiento Venopunción, sugiere que el nivel de conocimiento de los
    alumnos sobre ambos procedimientos esta relacionado.

\end{itemize}
