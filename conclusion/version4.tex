\section[TIC en la educación]%
{TIC en la educación}
%{Proveer una visión actualizada de los fundamentos y estado del arte de las nuevas corriente pedagógicas y su relación con las \gls{tic}}

\begin{itemize}

\item \textbf{Los beneficios de las \gls{tic} aún no hay sido explotados completamente en el área educativa}: esto se debe principalmente a la cantidad de inversión necesaria en infraestructura y a la necesidad de adopción de nuevas corrientes pedagógicas que requieren un cambio de roles de los profesores y alumnos\cite{unesco2013tics}.

%\item \textbf{El construccionismo y el constructivismo son pedagogías útiles para el desarrollo de habilidades profesionales}: poseen un efecto motivador intrínseco, lo que permite una enseñanza más eficaz\cite{kim2005effects}. La principal ventaja sobre los métodos tradicionales es la capacidad para el desarrollo del pensamiento de alto nivel\cite{wilson2012constructivism}.

\end{itemize}

\section[Juegos Serios]%
{Juegos Serios como herramientas educativas}
%{Proveer una visión actualizada de los juegos serios, sus principales características y sus ventajas y desventajas como herramientas pedagógicas.}

\begin{itemize}

\item \textbf{Los juegos serios permiten una experiencia sin riesgos}: permiten  al usuario experimentar, poner a prueba y construir conocimientos sin los riesgos económicos y de salud presentes en la vida real\cite{sg:aoverview}.

\item \textbf{Los juegos serios actuales tienden a ofrecer una retroalimentación muy guiada}: la mayoría de los juegos serios existentes brindan una retroalimentación muy guiada al usuario induciéndolo a hacer lo correcto sin brindarle suficiente espacio para que desarrolle su pensamiento crítico y toma de decisiones.

\end{itemize}

\section[Áreas de aplicación]%
{Áreas de aplicación de los juegos serios}
%{Identificar áreas de aplicación de los juegos serios, y seleccionar un área local para su evaluación}

\begin{itemize}

\item \textbf{La enseñanza de profesionales de enfermería es un área propicia para la aplicación de los juegos serios}: los estudiantes de enfermería requieren un alto grado de prácticas, pues adquieren el conocimiento a través de la experimentación y la reflexión\cite{humphreys2013developing}.

\item \textbf{En el \gls{iab}, los juegos serios poseen potencial para resolver los problemas de la formación de los profesionales de enfermería}: las técnicas utilizadas actualmente no son suficientes para la preparación de los alumnos ante las primeras prácticas con pacientes humanos.
%Adicionalmente los alumnos poseen poco tiempo disponible para actividades ajenas a las académicas impuestas por su centro de estudios.

\item \textbf{Se debe utilizar herramientas alternativas de bajo costo}: el nivel de acceso a la tecnología de los estudiantes del \gls{iab} es bajo. Proveer soluciones que requieran una inversión en hardware o software por parte de los usuarios no es una alternativa viable actualmente.

\end{itemize}

\section[Selección de herramientas]%
{Selección de herramientas tecnológicas para el desarrollo de juegos serios}
%{Seleccionar las herramientas tecnológicas disponibles para el desarrollo de soluciones que involucran a los juegos serios}

\begin{itemize}

\item \textbf{Se recomienda utilizar Unity3D para el desarrollo de juegos serios para dispositivos móviles}: crear un juego serio desde cero es un proceso sumamente complejo y costoso, la utilización de un motor moderno, como \emph{Unity3D}, facilita en gran parte el proceso.

\item \textbf{Se recomienda tener en cuenta el costo, requisitos mínimos, familiaridad, librerías, tienda y comunidad al seleccionar un motor de videojuego}:

    \begin{itemize}

    \item \textbf{Costo de la utilización}: motores como \emph{Unity3D} y \emph{UnrealEngine} poseen planes gratuitos que permiten acceder a  la funcionalidad completa para fines educativos.

    \item Familiaridad de los desarrolladores con las tecnologías utilizadas con el motor.

    \item \textbf{Librerías}: compatibilidad de librerías de terceros con el motor. Durante el desarrollo de la solución existieron librerías que no pudieron ser utilizadas con \emph{Unity3D}, estos problemas estaban relacionados con la versión de \emph{Mono} utilizada.

    \item \textbf{Tienda y comunidad}: el soporte brindado  por la comunidad, así con las librerías y componentes disponibles en la tienda sirven para acelerar el desarrollo de un juego serio. Durante el desarrollo de la solución se utilizaron varias librerías gratuitas de la tienda, y la comunidad es fuente de guías y tutoriales.

    \end{itemize}

\item \textbf{El uso de un motor de reglas condicionado por eventos es suficiente para evaluar al usuario}: un motor del tipo \gls{eca} permite la evaluación del usuario al momento de realizar las acciones, lo que a su vez permite tener acceso al contexto de la acción.

\end{itemize}

\section[Puesta en práctica de conocimientos teóricos]%
{Puesta en práctica de los conocimientos teóricos adquiridos}
%{Contrastar en la práctica los conocimientos teóricos adquiridos a través del diseño e implementación de un juego serio}

Durante el desarrollo del juego serio, se concluyeron la siguientes ventajas y desventajas relacionadas a la utilización de los juegos serios, y su aplicabilidad en un área específica en el Paraguay.

\subsection{Ventajas}

Las principales ventajas encontradas son:

\begin{itemize}

\item La utilización de juegos serios en dispositivos móviles permite su utilización en cualquier lugar y momento, haciendo frente a las limitaciones de tiempo y espacio.

% La verdad que no me gusta este
\item Las soluciones basadas en dispositivos móviles son factibles de aplicación en el Paraguay por el aumento constante de la penetración de los dispositivos móviles inteligentes como se menciona en la sección~\ref{sec:motivacion}.

\item La solución es beneficiosa para el aprendizaje de procedimientos de enfermería según el $100\%$ de los alumnos que la evaluaron.

\item La solución agrega un nivel adicional de preparación entre las clases teóricas y la práctica con pacientes, sirviendo así como una herramienta de apoyo a las técnicas actuales.

\item Existe una recepción positiva ante la utilización de los juegos serios. Tanto los alumnos, como las autoridades del \gls{iab} mencionan que la utilización de este tipo de soluciones para la formación de profesionales puede ser ventajosa.

\item Los juegos serios ayudan a los estudiantes de enfermería a poner a prueba sus conocimiento. Los datos mostrados en la sección~\ref{sec:subjetiva} permiten concluir que los estudiantes de enfermería consideran que el uso de la solución apoya al estudio en clase y laboratorio brindándoles más oportunidades de poner a prueba sus conocimientos con respecto a otros materiales utilizados que poseen limitaciones físicas y ofreciéndoles un paciente que reacciona a sus acciones.

\end{itemize}

\subsection{Desventajas}

Las desventajas encontradas durante el desarrollo y el análisis de los datos son:

\begin{itemize}

\item Alto costo de implementación. La simulación de cada escenario es costosa en cuanto a tiempo de diseño e implementación, la curva de aprendizaje es alta ya que la elaboración de un videojuego con las características de la solución requiere de un trabajo multidisciplinario.

\item No existen generadores de contenido para juegos serios. En la actualidad los profesores del área no pueden generar el contenido, existe una dependencia hacia los programadores y diseñadores, lo que incrementa el costo del desarrollo.

\item Los alumnos no cuentan con dispositivos móviles de altas prestaciones. Como se mencionó en la sección~\ref{sec:motivacion}, en la actualidad la penetración es muy alta, pero las prestaciones de estos dispositivos no es la necesaria para juegos serios complejos como la solución. Actualmente, los nuevos dispositivos móviles de gama baja ya cuentan con las prestaciones mínimas requeridas por \textit{Unity3d}, lo que indica que dentro de un tiempo, cuando estos teléfonos lleguen al mercado, la cantidad de usuarios potenciales aumentará.

%\item Requiere capacitación de los profesionales, pero están motivados>

\item La principal dificultad para utilizar en mayor medida la solución es el factor tiempo según el $64\%$ de los alumnos que la evaluaron.
%\item El $64\%$ de los alumnos menciono que la principal dificultad para utilizar la solución es el factor tiempo.

\end{itemize}

\section[Evaluación de solución propuesta]%
{Evaluación de la solución propuesta}

Las conclusiones obtenidas en cuanto al diseño, implementación y evaluación del juego serio
desarrollado son las siguientes:
%{Evaluar la solución propuesta para la obtención de datos que permitan identificar aspectos de diseño, desarrollo y evaluación a tener en cuenta para la creación de un juego serio}

\subsection{Diseño del juego serio}

%Las conclusiones obtenidas en cuanto al diseño del juego serio son las siguientes:

\begin{itemize}

\item \textbf{La definición de los aspectos pedagógicos, nivel de detalle requerido por procedimiento y las características del entorno debe ser realizada con profesores de prácticas y directores de carrera}: las validaciones constantes de estos aspectos ayudan a la detección de errores en el diseño, un ejemplo es la evaluación del rendimiento del alumno ya que existen maneras adecuadas de realizar los procedimientos.

\item \textbf{La motivación se incrementa al utilizar puntaje por procedimientos}: los juegos serios poseen aspectos lúdicos inherentes que motivan al usuario en su uso, agregar indicadores de rendimiento ayudan a esta motivación. Los datos mostrados en la sección~\ref{sec:subjetiva} permiten concluir que el principal factor motivacional es la visualización de un puntaje que resuma el desempeño del usuario y, en menor medida, la medición del tiempo y la posibilidad de compartir su rendimiento en las redes sociales.

\item \textbf{La exploración es facilitada por los estados aleatorios y las funciones simplificadas}: permitir al usuario explorar el entorno para resolver problemas ayuda en el desarrollo de su pensamiento crítico, así también, se debe ofrecer una interfaz intuitiva y simple para no obstaculizar esta exploración. Los datos mostrados en la sección~\ref{sec:subjetiva} permiten concluir que los principales factores que favorecen la exploración son la aleatoriedad en el estado del paciente y la representación simplificada de las funciones de los elementos. Sin embargo, esta simplificación debe diseñarse con mucho cuidado para no perder intuitividad en la interfaz.

\item \textbf{La inmersión es aumentada con la utilización de gráficos en tres dimensiones y partidas cortas}: es importante que el usuario sienta que está inmerso en el juego para que de este modo pueda percibir el contexto que se le está presentado y actúe de acuerdo a esto. Los datos mostrados en la sección~\ref{sec:subjetiva} permiten concluir que la utilización de entornos en tres dimensiones para la representación de elementos y lugares a los que están familiarizado el usuario, así como proveer partidas cortas para evitar que el usuario pierda el contexto de sus acciones, favorecen a la inmersión.

\item \textbf{Se debe proveer retroalimentación sobre el desempeño del usuario sólo al finalizar la partida}: el momento en el que se provee información acerca del desempeño del usuario tiene un impacto en el potencial pedagógico de un juego serio, si se provee información de manera muy frecuente se puede obtener herramientas del tipo prueba y error que poco aporte al aprendizaje. Los datos mostrados en la sección~\ref{sec:subjetiva} permiten concluir que ofrecerles a los usuarios una simulación de los procedimientos con una retroalimentación limitada les ayuda a poner en práctica sus conocimientos y a comprender el procedimiento.

\item \textbf{La información sobre el rendimiento del usuario debe ser detallada}: los datos mostrados en la sección~\ref{sec:subjetiva} permiten concluir que los usuarios están de acuerdo con el hecho de proporcionarles una retroalimentación indicándoles los pasos que realizó de manera correcta e incorrecta dentro del procedimiento. Sin embargo, una breve causa acerca de las equivocaciones no es suficiente, se requiere información detallada.

\item \textbf{Se deben utilizar indicadores de realización de acciones}: en la simulación de entornos de enfermería se debe diseñar un esquema que notifique al usuario sobre la realización de una acción, sobre todo para  aquellas acciones que no son visibles ante el ojo humano en la vida real.  En la solución se utilizan indicadores transparentes.

\item \textbf{Limitar la manipulación del punto de vista al utilizar elementos}: el usuario debería ser capaz de realizar una sola acción a la vez a través de la interfaz gráfica, así, se debería permitir al mismo alterar el punto de vista, o utilizar los elementos, pero no realizar ambas acciones al mismo tiempo. En las pruebas preliminares de la interfaz se detectó que los usuarios tienen problemas al manipular el punto de vista mientras utilizan elementos.

\end{itemize}

\subsection{Implementación del juego serio}

%Las conclusiones obtenidas en cuanto a la implementación del juego serio son las  siguientes:

\begin{itemize}

\item \textbf{Las diferencias principales entre el desarrollo tradicional de software  y el desarrollo de juegos serios son la interacción y la utilización de gráficos en tres dimensiones}:

    \begin{itemize}

    \item \textbf{Interacción}: existen diversas formas de interacción con la solución, esto implica un desafío al momento de coordinar y realizar las pruebas pertinentes.

    \item \textbf{Gráficos en tres dimensiones}: la utilización de gráficos en tres dimensiones implica un desafío mayor para los aspectos de diseño, en cuanto al aspecto estético como a las posibilidades que posee el usuario dentro de la simulación.

    \end{itemize}


\item \textbf{Se recomienda utilizar la guía para el desarrollo de un juego serio definida por Pereira\cite{pereira2009design}}: teniendo en cuenta que no todos los pasos se aplican a todos los juegos serios. Al desarrollar la solución se utilizó la guía proveída en~\cite{pereira2009design}, con adecuaciones de acuerdo al contexto de la solución.


\item \textbf{Es necesario evaluar al usuario sin conexión a Internet}: la evaluación del usuario como parte del front-end permite brindar una mayor movilidad y posibilita una mayor fluidez en la experiencia al no depender de una conexión a Internet.

\item \textbf{Es costoso diseñar personajes y entornos en tres dimensiones}: las diferentes fuentes de modelos 3D, como las tiendas o comunidades en línea, permiten acceder a una gran cantidad de modelos. Estos modelos son adecuados para la mayoría de los casos, crear componentes es un proceso largo y costoso.

\item \textbf{Es necesario enviar automáticamente los registros de utilización}: para asegurar el envió de los datos por parte del usuario, los registros deben ser enviados automáticamente, por ejemplo, cuando se detecta una conexión \textit{Wi-Fi}. En la solución, el usuario debe seleccionar la opción de enviar datos, lo que provocó que los datos no sean siempre enviados.

\end{itemize}

\subsection{Evaluación del juego serio}

%Las conclusiones obtenidas en cuanto a la evaluación del juego serio son las  siguientes:

\begin{itemize}

\item \textbf{Validar relevancia y dificultad de temas a tratar en las pruebas de conocimientos con los profesores de cátedra}: para realizar evaluaciones acerca del conocimiento de los alumnos, se debe consultar con los profesionales. No basta con utilizar los manuales y otras herramientas, pues los profesores son capaces de determinar la dificultad y la importancia de cada punto tratado.

\item \textbf{Es necesario poder reproducir las sesiones de juego del usuario con los registros de uso}: el nivel de granularidad de las acciones registradas por el juego serio debe ser la mayor posible. No sólo sirven para determinar el rendimiento del usuario, sino además permiten evaluar el uso de la interfaz, la frecuencia de utilización, entre otros aspectos. Es especialmente útil almacenar la mayor información posible cuando se realiza el análisis de variables no previstas.


\end{itemize}
