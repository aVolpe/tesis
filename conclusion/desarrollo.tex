
\section{Desarrollo de la aplicación}

En esta sección se detallan las conclusiones obtenidas en cuanto al proceso de
implementación de la solución en diferentes aspectos.

\subsection{Importancia del uso de motores de videojuegos}

La utilización de motores gráficos modernos facilita la creación de juegos
serios, relacionados a procedimientos del área de enfermería, permiten crear y
manipular entornos realistas sin demasiadas complicaciones. 

Una buena elección del motor gráfico de acuerdo a las características de la
aplicación que se desea implementar es sumamente importante y los criterios de
selección utilizados en este trabajo y mencionados
en~\ref{sec:seleccion_plataforma} pueden ser de gran ayuda en la selección.

\subsection{Pruebas para mejoras durante el desarrollo}

La realización de prueba de usuarios es interesante durante el desarrollo de una
aplicación ya que permiten visualizar los errores en su funcionamiento y de esta
manera se pueden corregir las debilidades encontradas de manera temprana,
siempre es mejor detectar la mayor cantidad de errores en el menor tiempo
posible.

En el caso de la solución, se realizaron pruebas de \Gls{gui} para mejorar la
interacción de los usuarios con ella, estas pruebas revelaron principalmente
problemas con la interacción con el entorno y con los objetos. Como resultado se
aplicaron mejoras que permitieron mejorar el uso y la intuitividad de la
interfaz.

\subsection{Validaciones de contenido de la aplicación}

La realización de validaciones tanto del contenido como de la representación de
la aplicación por parte de los profesionales es de suma importancia, debido a
que son a los estudiantes del área a los que va dirigido la solución. Estas
validaciones deben centrarse en los detalles en los que los mismos ponen mayor
interés.

Estas validaciones no sólo permiten encontrar errores sino que permiten asegurar
que la forma de implementación de ciertos aspectos es correcta. En el caso de la
solución, esto permitió corregir errores en las reglas definidas para la
evaluación del procedimiento de extracción de muestras de sangre, y la
presentación de la escenografía de manera tal que fuera más realista.

Más allá de lo expuesto, estas validaciones con profesionales permiten que se
sientan parte del proyecto, manteniendo el interés y predisposición de los
profesionales.


%En cuanto a las
%limitaciones, tanto tecnológicas como de utilización, presentadas por los
%dispositivos móviles dificultan la interacción con el entorno, la interacción a
%través de dispositivos móviles debe ser cuidadosamente diseñada para proveer de
%una interacción fluida.

%\subsection{Tecnología}
%\subsection{Nivel de detalle}
%\subsection{Interacción con objetos}
%\subsection{Partes que son necesarias pero no simulables}
    
%(lo de bioseguridad)
