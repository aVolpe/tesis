%! TEX root = ../main.tex

\section{Desarrollo de la aplicación}

En esta sección se detallan las conclusiones obtenidas en cuanto al proceso de
implementación de la solución.

\subsection{Utilizar un motor de videojuegos}

La utilización de motores de videojuegos facilita la creación de juegos serios.
Los motores actuales permiten crear y manipular entornos complejos, desarrollar
aplicaciones para diversas plataformas, y cuentan con librerías para las tareas
comunes (como manipulación de perspectiva, creación de animaciones, etc). 

La utilización de motores de videojuegos permiten al desarrollador centrarse en
aspectos específicos del juego serio y a no invertir el tiempo en aspectos
comunes a los motores, como, un motor de física, detectores de colisiones,
reproducción de contenido multimedia, motores de animaciones, inteligencia
artificial, comunicación, \textit{streaming}, administración de memoria, etc.

En este trabajo se utiliza \textit{Unity3D}, el cual facilita las tareas a
través de un \gls{ide} complejo, diversas herramientas para la creación de
animaciones, y gestión de librerías. Uno de los factores más interesantes de
\textit{Unity3D} es su tienda de librerías, donde se puede adquirir modelos en
tres dimensiones de calidad a precios bajos e incluso de manera gratuita.

\subsection{Seleccionar herramientas de acuerdo a las características necesarias}

Una buena elección del motor gráfico de acuerdo a las características de la
aplicación que se desea implementar es sumamente importante y los criterios de
selección utilizados en este trabajo y mencionados en la
sección~\ref{sec:seleccion_plataforma} pueden ser de gran ayuda en la selección.

Las limitaciones técnicas, de tiempo y monetarias del equipo de desarrollo
también deben ser tenidas en cuenta al momento de la selección de una
herramienta.

Junto a estos criterios, la información proveída en la
sección~\ref{sec:motores}, permitió seleccionar a \textit{Unity3D} como el motor
de videojuego necesario para desarrollar la solución utilizada para evaluar a
los juegos serios en el aprendizaje en este trabajo.

\subsection{Probar el desarrollo de manera frecuente}

En el desarrollo de juegos serios es útil realizar pruebas preliminares para
analizar y evaluar las decisiones de diseño, a fin de minimizar los obstáculos
presentes para los usuarios finales. Los aspectos que se deben probar y validar
son:

\begin{itemize}
    \item Calidad gráfica.
    \item Nivel de detalle del entorno.
    \item Integración con el hardware.
    \item Desenvolvimiento en el entorno a través de la interfaz.
    \item Interacción con objetos y entidades.
    \item Usabilidad de la \gls{gui}.
\end{itemize}

Las pruebas preliminares realizadas a la solución, cuyos resultados se observan
en la sección~\ref{sec:res_interfaz}, permite identificar fortalezas y
debilidades. 

En el caso específico de la solución, los problemas detectados incluyen falta de
naturalidad en la utilización de objetos y falta de intuitividad en el uso de la
\Gls{gui}. Estos problemas son solucionados en la versión que se presenta a los
usuarios finales.

La realización de prueba con usuarios es útil durante el desarrollo de una
aplicación ya que permiten visualizar los errores en su funcionamiento y de esta
manera se pueden corregir las debilidades encontradas de manera temprana,
siempre es mejor detectar la mayor cantidad de errores en el menor tiempo
posible.

En el caso de la solución, se realizaron pruebas de \Gls{gui} para mejorar la
interacción de los usuarios con ella, estas pruebas revelaron principalmente
problemas con la interacción con el entorno y con los objetos. Como resultado se
aplicaron mejoras que permitieron mejorar el uso y la intuitividad de la
interfaz.

\subsection{Crear entornos en tres dimensiones para aumentar la inmersión}

Escenarios en tres dimensiones similares a los lugares conocidos por los
usuarios ayudan permite a los mismos sentirse parte de la simulación. 

En la solución presentada en este trabajo, se utiliza un escenario en tres
dimensiones similar a un escenario conocido por los usuarios finales (un
laboratorio de prácticas), el mismo fue diseño en base a observaciones
realizadas en una visita guiada por profesionales del \gls{iab}. En la
sección~\ref{sec:res_subjetiva} se observa que los usuarios valoran
positivamente la influencia del escenario en la inmersión.

\subsection{Proveer retroalimentación clara al realizar una acción}

La forma de notificar al usuario en el momento en el que realiza una acción,
debe ser fácilmente observable y distinguible, de esta manera el usuario puede
tener la certeza de que se realizo la acción deseada.  

En la solución, cuando el usuario interactúa con un objeto, en la
sección~\ref{sec:res_subjetiva} se observa que mientras más vistosa es la
reacción del paciente, mejor es la recepción de los usuarios. Movimientos
discretos como la apertura ocular y el movimiento de los ojos, tienen una pobre
aceptación, en cambio, movimientos vistosos como el movimiento de las piernas,
brazos y el cuerpo en general, tienen una aceptación positiva.

La forma de representar las alteraciones del entorno luego de realizar una
acción deben ser claras.

