\section{Áreas de aplicación}
\label{sec:areas_aplicacion}

A continuación se definen las áreas de utilización  más frecuentes de los juegos
serios, son en estas áreas donde los juegos serios demuestran su fortalezas.

\begin{itemize}

\item \textbf{Militar}: durante más de $30$ años los videojuegos han sido
    reconocidos como herramientas factibles en el entrenamiento de militares, es
    más, los primeros videojuegos se basaban principalmente en lucha o combate.
    En $1996$ fue lanzado un videojuego llamado \emph{Marine Doom} en donde la
    tarea de los jugadores era el aprendizaje de formas de ataque, conservación
    de municiones, comunicación eficaz, 
    entre otros. De esta manera tuvo lugar una forma de entrenamiento más
    atractivo, sin el costo, dificultad, riesgos e inconvenientes que implica el
    mismo entrenamiento en un entorno real. Se pueden crear situaciones
    que en el mundo real son muy difíciles de replicar y permiten
    la repetición hasta alcanzar la maestría\cite{education:games}.

\item \textbf{Salud}: los juegos
    de salud se utilizan para la formación de profesionales basada en la
    simulación. En $2008$ el Centro de Simulación \emph{Hollier} en
    \emph{Birmingham}, Reino Unido, realizó una prueba que permitió a médicos
    jóvenes experimentar y entrenar para diversos escenarios médicos a través de
    maniquíes virtuales como pacientes, de este modo el aprendizaje se da por la
    experiencia. Con respecto a los videojuegos en el área de la salud
    \emph{Roger D. Smith} en su disertación, realizó una comparación entre la
    enseñanza tradicional y la formación mediante realidad virtual y el uso de
    herramientas basadas en la tecnología de videojuegos en cuanto a la cirugía
    laparoscópica. Como conclusión afirmó que lo último era más barato, requería
    menos tiempo y que permitió menos errores médicos cuando los médicos se
    presentaban en una cirugía real debido a, entre otras cosas, la posibilidad
    de repetición de la experiencia sin riesgo alguno\cite{education:games}. 

\item \textbf{Juegos corporativos}: este tipo de videojuegos se han utilizado
    para la selección de personal, la mejora de comunicación entre los
    directivos y su personal de confianza, y la formación de nuevos empleados.
    Un ejemplo de estos videojuegos es el \emph{INNOV8} de \emph{IBM} que ayuda
    en el entrenamiento de los estudiantes acerca de la gestión de procesos de
    negocios. Los juegos serios pueden ser utilizados incluso para elaborar
    planes de negocios\cite{education:games}. 

\end{itemize}
