\section{Aplicaciones}

Los siguientes son ejemplos de algunas áreas que utilizan \textit{Serious Game}
\todox{Arreglar esta introducción}


\subsection{Militar}

Los primeros juegos a menudo se basaban en lucha o combate.
Durante más de 30 años los juegos han sido reconocidos como herramientas
factibles en el entrenamiento de militares. En 1996 fue lanzado un juego
llamado \emph{Marine Doom} en donde la tarea de los jugadores era el
aprendizaje de formas de ataque, conservación de municiones, comunicarse
con eficacia, dar órdenes al equipo de trabajo entre otros. De esta
manera tuvo lugar una forma de entrenamiento más atractivo, sin el
costo, dificultad, riesgos e inconvenientes que implicaría el mismo
entrenamiento en un entorno real. Además se podían crear situaciones que
en el mundo real serían muy difíciles de replicar y donde los errores
pueden ser catastróficos además, permite la repetición hasta alcanzar la
maestría\cite{education:games}.

\subsection{Salud}

Este tipo de juegos son cada vez mayores, los juegos de salud se
utilizan para la formación de profesionales basada en la simulación. En
2008 el Centro de Simulación Hollier en Birmingham, Reino Unido, realizó
una prueba que permitió a médicos jóvenes experimentar y entrenar para
diversos escenarios médicos a través de maniquíes virtuales como
pacientes, de este modo el aprendizaje se da por la experiencia. En su
disertación, Roger D. Smith, realizó una comparación entre la enseñanza
tradicional y la formación mediante realidad virtual y el uso de
herramientas basadas en la tecnología de juegos en cuanto a la cirugía
laparoscópica. Como conclusión afirmó que lo último era más barato,
requería menos tiempo y que permitió menos errores médicos cuando los
médicos se presentaban en una cirugía real debido a, entre otras cosas,
la posibilidad de repetición de la experiencia sin riesgo
alguno\cite{education:games}.


\subsection{Juegos corporativos}

Este tipo de juegos se han utilizado para la
selección de personal, la mejora de comunicación entre los directivos y
su personal de confianza, y la formación de nuevos empleados. Un ejemplo
de estos juegos es el INNOV8 de IBM que ayuda en el entrenamiento de los
estudiantes acerca de la gestión de procesos de negocios. Los Serious
Game pueden ser utilizados incluso para elaborar planes de
negocios\cite{education:games}. 

