\section{Concepto}

Un \emph{Serious Game} es un vídeo juego elaborado con el propósito primario que
no es el de entretener\cite{sg:aoverview}, sino tienen una finalidad educativa
explícita y cuidadosamente pensada, utiliza la tecnología y los conceptos de la
industria de los vídeojuegos para encontrar solución a problemas reales. Es
decir, se utilizan para definir los juegos que poseen una pedagogía incluida,
algún tipo de evaluación ya sea interna o externa y lo que hay que aprender
(contenido) integrado\cite{damien:sg}.

\subsection{Características}

Los \emph{Serious Game} proveen una oportunidad muy importante para ayudar en la
enseñanza y desarrollo de profesionales, por que ayudan a crear el tipo de
educación que los adultos prefieren, proveen mecanismos para que los estudiantes
cometan errores y experimenten con sus ideas, con su conocimiento y con la
teoría en un ambiente protegido sin riesgos para la vida o la identidad\todox{Buscar
referencia}. 

Los beneficios que brindan los \emph{Serious Game} se acentúan en la medida en
la que los mismos proveen entornos más completos en donde realmente se puedan
poner en práctica la teoría, esto ayuda a una comprensión más profunda del área
de interés\todox{Referenciar}.

La principal diferencia entre los \emph{Serious Game} y otras aplicaciones de
\emph{E-Learing} es su enfoque en la creación de una experiencia de aprendizaje
significativo, relevante y atractivo. En un \emph{Serious Game} existen metas
claras de aprendizaje pero las mismas se encuentran en un contexto significativo
en donde se deben aplicar los conocimientos y hacer uso de herramientas que
están a disposición para obtener éxito en la resolución de los problemas
presentados. Estos problemas se equilibran a través de la retroalimentación y
otras estrategias para mantener el interés del estudiante\cite{papertian:const}.
%. Todo esto hace que en los \emph{Serious Game} el principal objetivo sea ganar
%el juego no aprender, sin embargo sólo se puede hacer esto dominando el
%aprendizaje

\fixme{El campo de los \emph{Serious Game} rechaza la idea de que los profesionales de
    la educación pueden ser reemplazados fácilmente}{Obs: que es cada sección?,
    un enfoque? Una técnica? Un buzzword?}, para ellos la labor de estos
profesionales es imprescindible para la reflexión y orientación del aprendizaje.
Es cierto que se puede llegar a aprender sin el apoyo de un profesional de la
educación pero se corre el riesgo de perder el enfoque y la eficacia
\cite{elearning:seiousgames}. 

El \emph{serious Game} no se trata de una modelo de aprendizaje pasajero. Varios
autores como \emph{Johan Huizinga}, \emph{Jean Piaget}, \emph{Wittgenstin} y
\emph{Seymour Papert} han reconocido su importancia  como objeto de aprendizaje.
Los juegos deben ser elaborados teniendo en cuenta el nivel cognitivo del
estudiante, es decir, su etapa de aprendizaje y en que el aprendizaje difiere de
acuerdo a la etapa de vida en la que se encuentre un estudiante. Mediante la
práctica repetida de actividades relacionadas al área de interés se desarrollan
habilidades y destrezas\cite{education:games}. 

\observacion{Se podría hacer una comparación? (entre todos)}

\subsection{Diferencia entre juegos serios, simulaciones y mundos virtuales}
\todox{Esto agregue, ver si corresponde}

Estas tres alternativas son entornos virtuales altamente interactivos, todos con
sus propias posibilidades y fines. Los tres pueden ser similares pero
\cite{education:games}:

\begin{itemize}
\item \textbf{Simulaciones educativas}: utilizan escenarios rigurosamente
    estructurados con un conjunto altamente refinado de normas, retos y
    estrategias que son cuidadosamente diseñados para desarrollar las
    competencias específicas que se pueden transferir directamente al mundo
    real.
\item Los juegos son actividades atractivas y divertidas que habitualmente se
    utilizan exclusivamente para el entrenamiento pero también permiten una
    exposición con un conjunto determinado de herramientas, argumentos o ideas.
    Todas las partidas se juegan en un mundo estructurado por normas
    específicas, mecanismos de retroalimentación, y las herramientas necesarias,
    aunque no están tan definidas con en las simulaciones.
\item \textbf{Mundos virtuales:} son entornos sociales 3D multijugador, pero sin
    el enfoque en un objetivo en particular, como avanzar al siguiente nivel o
    navegar con éxito el escenario.
\end{itemize}

