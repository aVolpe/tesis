\section{Características}

Los \emph{Juegos Serios} proveen una oportunidad muy importante para ayudar en la
enseñanza y desarrollo de profesionales, por que ayudan a crear el tipo de
educación que los adultos prefieren, proveen mecanismos para que los estudiantes
cometan errores y experimenten con sus ideas, con su conocimiento y con la
teoría en un ambiente protegido sin riesgos para la vida o la identidad. 

Los beneficios que brindan los \emph{Juegos Serios} se acentúan en la medida en
la que los mismos proveen entornos más completos en donde realmente se puedan
poner en práctica la teoría, esto ayuda a una comprensión más profunda del área
de interés.

La principal diferencia entre los \emph{Juegos Serios} y otras aplicaciones de
\emph{E-Learing} es su enfoque en la creación de una experiencia de aprendizaje
significativo, relevante y atractivo. En un \emph{Juego Serio} existen metas
claras de aprendizaje pero las mismas se encuentran en un contexto significativo
en donde se deben aplicar los conocimientos y hacer uso de herramientas que
están a disposición para obtener éxito en la resolución de los problemas
presentados. Estos problemas se equilibran a través de la retroalimentación y
otras estrategias para mantener el interés del estudiante\cite{papertian:const}.

El campo de los \emph{Juegos Serios} rechaza la idea de que los profesionales de
la educación pueden ser reemplazados fácilmente, para ellos la labor de estos
profesionales es imprescindible para la reflexión y orientación del aprendizaje.
Es cierto que se puede llegar a aprender sin el apoyo de un profesional de la
educación pero se corre el riesgo de perder el enfoque y la eficacia
\cite{elearning:seiousgames}. 

El \emph{Juego Serio} no se trata de una modelo de aprendizaje pasajero. Varios
autores como \emph{Johan Huizinga}, \emph{Jean Piaget}, \emph{Wittgenstin} y
\emph{Seymour Papert} han reconocido su importancia como objeto de aprendizaje.
Los juegos deben ser elaborados teniendo en cuenta el nivel cognitivo del
estudiante, es decir, su etapa de aprendizaje y en que el aprendizaje difiere de
acuerdo a la etapa de vida en la que se encuentre un estudiante. Mediante la
práctica repetida de actividades relacionadas al área de interés se desarrollan
habilidades y destrezas\cite{education:games}.

\subsection{Ventajas}


Las \Gls{tic} y los juegos serios en particular son herramientas de inestimable
valor para apoyar los nuevos procesos de enseñanza-aprendizaje y
evaluación\cite{guenaga2013serious}.

Los juegos serios, constituyen un escenario privilegiado para el desarrollo de
todos los componentes de las competencias (conceptos, habilidades, actitudes,
motivaciones, valores, etc.) ya que permiten desarrollar vivencias en las que
ponerlos en práctica, permitiendo el entrenamiento en situaciones que en muchas
ocasiones son similares a las que se encuentran en entornos
reales\cite{guenaga2013serious}.

Además, favorecen la autoestima y tienen un factor motivacional, así como la
posibilidad de desarrollar destrezas y estrategias cognitivas como la capacidad
de resolución de problemas, toma de decisiones, búsqueda y organización de la
información, habilidades perceptivo-motrices y razonamiento
abstracto\cite{guenaga2013serious}.

Se puede añadir también que aumentan la capacidad de coordinación, percepción
espacial y ampliación del campo visual, lo que tiene una incidencia en la
lectura y el manejo eficiente en ambientes 3D\cite{guenaga2013serious}. 

Más allá del logro de competencias puntuales, las teorías modernas de
aprendizaje sugieren que el aprendizaje es más efectivo cuando es activo,
experiencial, situado, basado en problemas y se recibe retroalimentación
inmediata y los juegos basado en aprendizaje se fundamentan en esos
principios\cite{guenaga2013serious}.



\subsection{Desafíos}


El potencial de un juego serio no es ilimitado, los desarrolladores se
encuentran con múltiples desafíos que deben ser superados para poder obtener un
juego serio que obtenga las ventajas citadas previamente y pueda ser de utilidad
en la educación formal.

Un juego serio, como el resto de la \textit{media}, no puede cambiar el
comportamiento de una persona por sí solo, un juego acerca de hábitos
saludables, no hará del jugador un nutricionista, pero si permiten al jugador
explorar las opciones, tener en cuenta las consecuencias de sus actos y poner en
práctica su conocimiento\cite{education:games}, adicionalmente es importante
definir lo que forma parte del juego serio y lo que no, pues un juego serio no
debe incluir todas las características de la
realidad\cite{stapleton2004serious,videojuegos:gonzaleztardon}. 

La forma tradicional de evaluación presenta dificultades a los juegos serios,
por ejemplo, las pruebas tradicionales contienen un grupo de preguntas, las
cuales son vistas de manera independiente, en cambio en un juego serio, las
acciones son dependientes del contexto y las acciones previamente
realizadas\cite{shute2009melding}.

En cuanto al objeto pedagógico, el área en la cual se utiliza un juego serio es
un factor determinante para el éxito del mismo, es decir, se debe responder a la
pregunta: \emph{¿Es necesaria una solución basada en juegos
    serios?}\cite{stapleton2004serious}, las áreas de aplicación de un juego
serio se describen con más detalle en~\ref{sec:areas_aplicacion}.

Uno de los factores más complicados a la hora del desarrollo de juegos serios es
la limitación de recursos financieros, esto no quiere decir que no existan
recursos para su desarrollo, sino que, comparados con los recursos invertidos en
otras \textit{media} es insignificante\cite{stapleton2004serious}. Como
consecuencia de las limitaciones financieras, los desarrolladores no siempre
pueden acceder a tecnología de última generación\cite{stapleton2004serious}
