\section{Características de los juegos serios}

Las \Gls{tic} y los juegos serios en particular son herramientas de gran valor
para apoyar los nuevos procesos de enseñanza-aprendizaje y
evaluación\cite{guenaga2013serious}. 


Los \emph{Juegos Serios} proveen una oportunidad muy importante para ayudar en
la enseñanza y desarrollo de profesionales\cite{mariluz:seiousgames}, por que
ayudan a crear el tipo de educación que los adultos prefieren, proveen
mecanismos para que los estudiantes cometan errores y experimenten con sus
ideas, con su conocimiento y con la teoría en un ambiente protegido sin riesgos
para la vida o la identidad\cite{sg:aoverview,education:games}. 

Los beneficios que brindan los \emph{Juegos Serios} se acentúan en la medida en
la que los mismos proveen entornos más completos en donde realmente se puedan
poner en práctica la teoría, esto ayuda a una comprensión más profunda del área
de interés\cite{sg:aoverview}.

% revisar la cita de este párrafo, estaba con papertian:const
La diferencia entre los \emph{Juegos Serios} y otras aplicaciones de
\emph{E-Learning}, está en el enfoque que tienen los juegos serios en la creación
de una experiencia de aprendizaje significativa, relevante y
atractiva\cite{sg:aoverview}.

% revisar la cita de este párrafo, estaba con papertian:const
En un \emph{Juego Serio} existen metas claras de aprendizaje pero las mismas se
encuentran en un contexto significativo en donde se deben aplicar los
conocimientos y hacer uso de herramientas que están a disposición para obtener
éxito en la resolución de los problemas presentados. Estos problemas se
equilibran a través de la retroalimentación y otras estrategias para mantener el
interés del estudiante\cite{sg:aoverview}.

El campo de los \emph{Juegos Serios} rechaza la idea de que los profesionales de
la educación pueden ser reemplazados fácilmente, la labor de estos profesionales
es imprescindible para la reflexión y orientación del
aprendizaje\cite{elearning:seiousgames}. Es cierto que se puede llegar a
aprender sin el apoyo de un profesional de la educación pero se corre el riesgo
de perder el enfoque y la eficacia\cite{elearning:seiousgames}. 

Es importante notar que los juegos serios deben ser elaborados teniendo en cuenta el
nivel cognitivo del estudiante, es decir, su etapa de aprendizaje, ya que el
aprendizaje difiere de acuerdo a la etapa de vida en la que se encuentre un
estudiante\cite{education:games}.

\subsection{Ventajas de los juegos serios}

Los juegos serios, al ser parte de la corriente de las \Gls{tic}, posee las
mismas ventajas descritas en~\ref{sec:tics_ventajas}, y enfatiza especialmente
la motivación del usuario.

Las ventajas que se acentúan en los juegos serios son:

\begin{itemize}

\item \textbf{Motivación interna}: favorecen la autoestima y tienen un factor
    motivacional\cite{guenaga2013serious}, permitiendo una implicación mayor del
    usuario en la actividad\cite{sg:aoverview}. La implicación del usuario
    dentro de la actividad, es un tema central en el desarrollo de los juegos
    serios\cite{charsky:2010}.

    La eficacia de un juego serio depende en gran parte del usuario, y de su
    compromiso con las metas del mismo\cite{sg:aoverview}.

\item \textbf{Apoyo al aprendizaje}: el motivo por el cual los juegos serios
    ayudan al aprendizaje es por que los mismos se desarrollan en un entorno
    significativo y relevante al contexto, y esto es mejor que un aprendizaje
    fuera de un entorno significativo\cite{sg:aoverview}.
    
   Adicionalmente al logro de competencias puntuales, las teorías modernas de
   aprendizaje sugieren que el aprendizaje es más efectivo cuando es activo,
   experiencial y basado en problemas, los juegos serios se fundamentan en esos
   principios\cite{guenaga2013serious}.

\item \textbf{Menos limitaciones}: la utilización de juegos serios permite a sus usuarios
    experimentar en entornos y sistemas que no son posibles en la vida real, por
    cuestiones de costo, tiempo y aspectos relacionados a la
    seguridad\cite{sg:aoverview}.

\item \textbf{Similitud a la realidad}: los juegos serios constituyen un
    escenario privilegiado para el desarrollo de todos los componentes de las
    competencias (conceptos, habilidades, actitudes, motivaciones, valores,
    etc.), ya que permiten desarrollar vivencias en las que ponerlos en practica,
    permitiendo el entrenamiento en situaciones que en muchas ocasiones son
    similares a las que se encuentran en entornos
    reales\cite{guenaga2013serious,sg:aoverview}.
    
\item \textbf{Estimulación sensorial}: aumentan la capacidad de coordinación,
    percepción espacial y ampliación del campo visual, lo que tiene una
    incidencia en la lectura y el manejo eficiente en ambientes
    3D\cite{guenaga2013serious}. 

\end{itemize}


\subsection{Desafíos de los juegos serios}
%\observacion{De los juegos serios, muy repetido ya}

El potencial de un juego serio no es ilimitado, los desarrolladores se
encuentran con múltiples desafíos que deben ser superados para poder obtener un
juego serio que obtenga las ventajas citadas previamente y pueda ser de utilidad
en la educación formal.

Adicionalmente a los desafíos de las \Gls{tic} definidos
en~\ref{sec:tics_ventajas}, los juegos serios cuentan con desafíos particulares,
los cuales son:

\begin{itemize}

\item \textbf{Falta de investigación}: aunque en los últimos años los estudios
    del impacto de los juegos serios han aumentado considerablemente, son
    necesarios más estudios para probar su eficiencia\cite{sg:aoverview}.
    Existen estudios que muestran que las simulaciones, entornos virtuales,
    etc.\ promueven la educación, pero en el área específica de los juegos
    serios, aún faltan más estudios\cite{sg:aoverview}.

\item \textbf{Expectativas muy altas}: un juego serio, como el resto de la
    \textit{media}, no puede cambiar el comportamiento de una persona por sí
    solo, un videojuego acerca de hábitos saludables, no hará del jugador un
    nutricionista, pero sí permiten al jugador explorar las opciones, tener en
    cuenta las consecuencias de sus actos y poner en práctica sus
    conocimientos\cite{education:games}, adicionalmente es importante definir lo
    que forma parte del juego serio y lo que no, pues un juego serio no debe
    incluir todas las características de la
    realidad\cite{stapleton2004serious,videojuegos:gonzaleztardon}. 

\item \textbf{Evaluación tradicional}: la forma tradicional de evaluación presenta
    dificultades a los juegos serios, por ejemplo, las pruebas tradicionales
    contienen un grupo de preguntas, las cuales son vistas de manera
    independiente, en cambio en un juego serio, las acciones son dependientes
    del contexto y las acciones previamente realizadas\cite{shute2009melding}.

\item \textbf{Utilización incorrecta}: en cuanto al objeto pedagógico, el área en
    la cual se utiliza un juego serio es un factor determinante para el éxito
    del mismo, es decir, se debe responder a la pregunta: \emph{¿Es necesaria
        una solución basada en juegos serios?}\cite{stapleton2004serious}, las
    áreas de aplicación de un juego serio se describen con más detalle
    en~\ref{sec:areas_aplicacion}.

\item \textbf{Falta de recursos}: uno de los factores más complicados a la hora
    del desarrollo de juegos serios es la limitación de recursos financieros,
    esto no quiere decir que no existan recursos para su desarrollo, sino que,
    comparados con los recursos invertidos en otras \textit{media}, el
    presupuesto es insignificante\cite{stapleton2004serious,sg:aoverview}. Como
    consecuencia de las limitaciones financieras, los desarrolladores no siempre
    pueden acceder a tecnología de última generación\cite{stapleton2004serious}


\end{itemize}

