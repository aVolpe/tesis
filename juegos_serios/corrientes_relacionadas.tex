\section{Corrientes relacionadas}

\subsection{Simulación}

La simulación se define como el proceso de diseñar un modelo de un sistema real
y, llevar a cabo experimentos con este modelo, con el fin o bien de entender el
comportamiento del sistema o de la evaluación de distintas estrategias para la
operación del sistema\cite{ingalls2008introduction}. 
%[ingalls2008introduction]

Un juego y una simulación podrían llegar a ser muy parecidos, a veces los juegos
tienen motores de simulación\footnote{Un motor de simulación es un conjunto de
objetos y métodos que se utilizan para la construcción de modelos de
simulación que están dentro de las aplicaciones}, una de las diferencias
es que la simulación es muy dependiente del contexto. 

La simulación en el ámbito de la educación fue evolucionando desde simples
motores de reglas hasta complejos entornos, la simulación demostró ser una
herramienta muy útil en el ámbito laboral\cite{mariluz:seiousgames}, pues enseña
al alumno a encarar situaciones muy difíciles de representar en entornos
completamente controlados y provee mecanismos para comprobar la efectividad de
la herramienta. 

Actualmente la simulación se utiliza más en el ámbito empresarial pues las
empresas son las más necesitadas de innovar en el ámbito de la enseñanza. Un
ejemplo de esta necesidad se da, por ejemplo, en el entrenamiento de nuevos
vendedores, es muy difícil enseñar a un vendedor como debe vender los productos
con un pizarrón y/o una presentación, en cambio la simulación permite que el
mismo pueda probar cosas nuevas y experiencias de sus compañeros (o instructor),
convirtiendo así el aprendizaje en colectivo\cite{mariluz:seiousgames}. En el
ámbito académico la simulación es más utilizada en campos físicos (como simulación
de fluidos), meteorología (simulación de tormentas y fenómenos climáticos), etc. 

Existen dos tipos de simulaciones, en primer lugar están las experimentales que
ponen al estudiante en el lugar de un profesional y requieren que el mismo tome
decisiones para alcanzar los objetivos y en segundo lugar están las simbólicas
que buscan que el estudiante deduzca eventos, principios y mejores prácticas
\cite{charsky:2010}. 


\subsubsection{Partes de una simulación}

Una simulación esta conformada por:

\begin{description}

\item[Entidades] Cualquier objeto o componente en el sistema que requiera
	representación explícita en el modelo se define como
	entidad\cite{banks2000dm}. Las entidades poseen atributos. Los atributos
	son las características de una determinada entidad que son exclusivos de
	esa entidad. Por último, son aquellas que cambian el estado de una
	simulación. Ejemplo de entidades son: un médico o una jeringa en una
	simulación médica.

\item[Acciones] Las entidades interactúan entre sí a través de acciones. Estas
	acciones puede causar cambios en el estado de la simulación además de
	eventos. Ejemplo de una acción en una simulación médica es la
	esterilización de un instrumento.

\item[Eventos] Los eventos son hechos que ocurren de manera controlada pero no
	siempre predecible en el entorno simulado, los mismos afectan a las
	entidades y deben obligar a realizar alguna de las acciones disponibles
	para tal evento. Ejemplo de un evento en un simulación médica es un paro
	cardíaco del paciente.

\end{description}

La confianza en el modelo o la simulación según\cite{DoDSysEng2001} se establece
mediante:

\begin{itemize}

\item \textbf{La verificación} Es el proceso de determinar si la implementación
	representa con precisión las especificaciones del diseño. 

\item \textbf{La validación} Es el proceso de determinar el grado en el que el modelo
	representa de forma exacta la realidad de acuerdo al uso que se tiene
	previsto darle y el nivel de confianza que debe tenerse en la
	evaluación.

\item[La acreditación] Es el proceso de certificación de un modelo para su uso
	con un propósito específico.
%[DoDSysEng2001]

\end{itemize}



\subsection{Gamification}

%\observacion{Empezar explicando qué es}

El uso de mecánicas tradicionalmente usadas en los videojuegos en
contextos distintos a los juegos es denominado gamification.


%\fixme{ Es el uso de mecánicas tradicionalmente usadas en los videojuegos en
%    contextos distintos a los juegos. Según Shell~\cite{hj:gamification} un
%    juego es una actividad cuyo fin es resolver un problema de manera
%    entretenida. }{reordenar}

La \emph{Gamification}, mejora la actividad del usuario, el \emph{engagement}
(enganchamiento o compromiso con el juego), el aprendizaje, la puntualidad
(capacidad de completar una tarea o asignación antes del tiempo designado), el
retorno a la inversión, la calidad y la colaboración.

\subsubsection{Principios}

Los principios de la gamificación moderna según~\cite{hj:gamification} son los
siguientes:

\begin{itemize}
    \item \fixme{Objetivos claramente definidos}{explicar más}.
    \item Mejor registro de resultados y tablas de puntuación.
    \item Retroalimentación más frecuente.
    \item Un mayor grado de elección personal de los métodos.\revisar{Abstracto}
    \item Entrenamiento consistente.
\end{itemize}

\observacion{Agregar ejemplo o ver otra manera de enriquecer la explicación de
    los principios.}

Cuando se mide el desempeño, el rendimiento mejora, cuando el rendimiento se
mide y además se informa sobre esto, la tasa de mejora acelera. Cuando la
retroalimentación se presenta en forma de tablas y gráficos el impacto es aún
mayor.

\subsubsection{Propiedades gamification}

Esencialmente, gamification intenta aplicar la mecánica de los juegos en otros
entornos, como el ambiente educativo. Este concepto no está directamente
relacionado con el diseño del juego, sino que trata de involucrar al usuario a
través de pequeñas dosis de desafíos y recompensas con el fin de conseguir que
el usuario realice ciertas acciones en diferentes
ambientes\cite{breaking:gamification}.

Gamification trabaja para satisfacer algunos de los deseos humanos más
fundamentales: el reconocimiento y la recompensa, de estado, de logros,
competencia y colaboración, la auto-expresión, y el
altruismo.\cite{breaking:gamification}.

La mecánica del juego pueden ser de diferentes
tipos\cite{breaking:gamification}, tales como:

\begin{itemize}
    \item Comportamiento (centrado en el comportamiento humano y la psiquis
        humana),
    \item Retroalimentación (en relación con el ciclo de retroalimentación en la
        mecánica de juego, y
    \item La progresión (utilizada para estructurar y extender la acumulación de
        habilidades significativas).
\end{itemize}


Existen otros mecanismos de juego que se pueden utilizar para los materiales
gamification y actividades educativas\cite{breaking:gamification}, tales como:

\begin{itemize}
    \item El tiempo (los jugadores tienen un tiempo limitado para realizar una
        tarea).
    \item La exploración (los jugadores tienen que explorar y descubrir cosas
        que les sorprenderán).
    \item Los desafíos entre los usuarios (los jugadores pueden darse desafíos
        unos a otros y competir para el logro de los objetivos, los objetos,
        medallas, etc.).
\end{itemize}

Para que sea eficaz a largo plazo, gamification debe ser algo más que la adición
de este tipo de elementos para un contexto no-juego, también debe actuar sobre
la motivación intrínseca de los jugadores\cite{framework:gamification}. 

Con el fin de tener una motivación intrínseca para realizar una tarea, la
persona debe mantenerse en un estado entre la ansiedad (si el desafío supera las
capacidades de la persona) y el aburrimiento (si la persona siente que la tarea
es demasiado fácil ). Este es un estado conocido como flujo. Objetivos claros,
un sentido de control, retroalimentación inmediata y, sobre todo, un equilibrio
entre habilidad y reto son algunos de los factores que contribuyen a
fluir\cite{framework:gamification}.

La relación, el deseo de interactuar y conectarse con otras personas, es una de
las necesidades humanas innatas que conducen a la motivación
intrínseca\cite{framework:gamification}.

Por lo tanto, los sistemas con gamification no sólo deben abordar la motivación
extrínseca de los jugadores, sino también considerar la forma de conducir a los
jugadores la motivación intrínseca. Debería centrarse en cómo crear experiencias
significativas, proporcionar un sentido de relación entre los jugadores, mejorar
su reconocimiento social, y dar la autonomía y el propósito de sus acciones.
También debe mantener a los jugadores en un estado de flujo y proporcionar una
experiencia divertida conjunto\cite{framework:gamification}. 


\subsubsection{Elementos del juego}

Los elementos del juego son el conjunto de componentes y características de los
juegos de vídeo que se pueden utilizar en contextos
no-juego\cite{framework:gamification}.

El flujo y la diversión deben ser considerados en un diseño como sección
transversal del sistema, transversal a los otros
componentes\cite{framework:gamification}.

A continuación se describe como se implementarían estos conceptos con los elementos
del juego, según\cite{framework:gamification}:

\begin{itemize}
    \item Retroalimentación y recompensas: puntos, barras de progreso,
        insignias, trofeos, tabla de calificación.
    \item Amigos: compartir, invitar a amigos, dar/comercializar/vender bienes
        virtuales, tablas de clasificación (gráfico social).
    \item Jugabilidad: niveles, objetivos intermedios, objetivos claros, fracaso
        divertido, reglas, economía virtual, calendarios de recompensas.
\end{itemize}

Los componentes transversales de flujo y la diversión se logran a través de la
forma en que las actividades se establecen en el sistema. El dominio y el
progreso son los que hacen que las experiencias sean divertidas. La sensación de
dominio y el progreso se puede implementar a través de los elementos de la
jugabilidad, los amigos y conceptos de retroalimentación y recompensas. Lo mismo
ocurre con el flujo. El jugador puede mantenerse en un canal de flujo cuando él
o ella está óptimamente desafiado proporcionando tareas que no son ni demasiado
fácil ni demasiado difícil. Esto podría lograrse proporcionando
retroalimentación inmediata, objetivos intermedios y diferentes niveles de
progresión. De esta manera, el reto es equilibrado con las habilidades del
jugador\cite{framework:gamification}.

\observacion{\textbf{Inssito}, tienen que expandir la primera sección como para
    darle contexto a todas estas secciones}
