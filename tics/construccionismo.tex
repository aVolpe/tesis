\section{Construccionismo}
\label{sec:tics_CONSTRUCCIONISMO}

El construccionismo es una corriente pedagógica que parte de una concepción del
aprendizaje según la cual la persona aprende por medio de su interacción
dinámica con el mundo físico, social y cultural en el que está
inmerso\cite{valdivia:sg}.

Posee un enfoque diferente en cuanto al uso de las \Gls{tic} en la educación.
Esta pedagogía se diferencia de la educación tradicional en que el estudiante ya
no es un receptor pasivo de información, en cambio, el mismo participa
activamente del proceso de aprendizaje construyendo su propio conocimiento. Se
diferencia del instruccionismo en que el construccionismo utiliza la tecnología
como medio cognitivo y no para la entrega de contenido.

Se considera al construccionismo como una alternativa prometedora a la educación
tradicional. Desde el punto de vista tecnológico, es ideal pues el mismo
requiere un alto dinamismo en el traspaso del conocimiento
\cite{sasha:construtivism}. 

El construccionismo y las \Gls{tic} siempre han estado relacionados, ya que el
mismo se originó con un lenguaje de programación (LOGO)\cite{ict:ttc}. Un
característica importante de esta relación es que tienen la capacidad de
eliminar los problemas de distancia\cite{mariluz:seiousgames}.


\subsection{Historia}

A mediados de la década de $1960$ \textit{Seymour Papert}, un matemático
sudafricano, llegó a los Estados Unidos de América, donde fue co-fundador del
Laboratorio de Inteligencia Artificial del \Gls{mit} con \textit{Marvin
    Minsky}\cite{logo:sg}. 


En la década de $1980$, \textit{Seymour Papert} acuñó el término
construccionismo en su aplicación a la Fundación Nacional de Ciencia de los
Estados Unidos titulada \enquote{Constructionism: A New Opportunity for
    Elementary Science Education} (Construccionismo: Una nueva oportunidad para
la enseñanza de ciencia element aria) para presentar un método pedagógico que se
basaba en muchas de las ideas de la educación progresiva estudiadas por el
estadounidense \textit{John Dewey} en el inicio del siglo $20$ en su escuela experimental
en la Universidad de Chicago. \textit{Dewey} quería poner gran parte de la
responsabilidad de aprender en los estudiantes que han nacido con el don de
aprendizaje y la creación de conocimiento en sus propios
términos\cite{historia:2014}.

\textit{Papert} también fue influenciado por \textit{Maria Montessori} quien
luego de grados en pediatría, medicina, psicología y filosofía comenzó su propia
escuela experimental para niños pequeños. En lugar de que ella misma
estableciera formalmente las tareas, vio como sus estudiantes actuaron por su
cuenta. \textit{Montessori} siguió los intereses de los estudiantes y observó como
respondieron a sus entornos especialmente preparados. \textit{Montessori}
preparó el escenario pero no ofreció una guía explícita\cite{historia:2014}.

\textit{Papert} trabajó directamente con el psicólogo suizo \textit{Jean
    Piaget}, fundador del constructivismo. \textit{Piaget}, al igual que
\textit{Dewey}, \textit{Montessori} y otros, desarrolló su teoría de la
educación y la construcción del conocimiento observando e interactuando con los
niños. De estas observaciones nació el movimiento constructivista. El
contructivismo se basa en que el conocimiento debe ser construido por el
estudiante y los nuevos significados deben ser obtenidos relacionándolos a
significados anteriores en el propio sistema de relaciones del
estudiante\cite{historia:2014}. 

\textit{Papert} admite que jugó más con la palabra construcción. El
contruccionismo de \textit{Papert} se diferencia del constructivismo de
\textit{Piaget} en que los estudiantes construyen las ideas o partes del mundo
utilizando herramientas. Para \textit{Papert} los estudiantes necesitan
construir modelos de partes de su mundo con el fin de comprender más plenamente
el significado, el contenido y la dinámica de las partes. La elaboración de
representaciones mentales mediante la construcción y el intercambio es la
metáfora del marco construccionista\cite{historia:2014}.

\textit{Papert} trabajó con el equipo de \textit{Bolt, Beranek y Newman},
liderado por \textit{Wallace Feurzeig}, que creó la primera versión del lenguaje
de programación \enquote{LOGO} en $1967$. \enquote{LOGO} es un dialecto de
\textit{Lisp}, fue diseñado como una herramienta para el aprendizaje. Sus
características, como la modularidad, extensibilidad, interactividad y
flexibilidad, derivan de este objetivo\cite{logo:sg}.

Los desarrolladores de \enquote{LOGO} no solo alentaron la promoción de formas
construccionistas de enseñanza y aprendizaje sino también alentaron otra forma
de aprendizaje nueva y no tradicional con las diferentes herramientas
tecnológicas\cite{historia:2014}. 

Por lo tanto, se puede decir que la creación de \enquote{LOGO} permitió la
creación del construccionismo\cite{historia:2014}.

\subsection{Bases Pedagógicas}

Para el construccionismo, el conocimiento es construido por el estudiante en
lugar de ser trasmitido por el profesor\cite{moses:2003} y esto sucede
particularmente cuando el mismo se compromete en la elaboración de un producto o
artefacto que tenga un significado y pueda ser compartido\cite{valdivia:sg}. De
esta manera, se permite a los estudiantes elaborar sus propias interpretaciones
razonadas del mundo mediante la interacción con el mismo. El profesor actúa como
guía para el estudiante en la construcción de su conocimiento, aportando
conocimiento y experiencia.

Según \textit{Papert}, los alumnos estarán mucho más involucrados en su aprendizaje si
construyen artefactos que los demás pueden ver, criticar y tal vez utilizar. Y
además, el alumno se enfrenta a problemas complejos con estas construcciones,
harán el esfuerzo por resolver problemas y aprender ya que la construcción les
motivará\cite{const:vs}.

El enfoque construccionista establece que los seres humanos conocen y aprenden
de formas diferentes por lo tanto, no se puede elaborar una jerarquía de estilos
de aprendizajes\cite{valdivia:sg}.

\subsection{Actualidad}

Existen varios proyectos o iniciativas que incluyen al construccionismo como
base pedagógica, para la mayoría de ellos las computadoras son esenciales
mientras que para otros el mayor esfuerzo está en la incorporación de la
tecnología en su práctica educativa\cite{papertian:const}.

Algunos de estos proyectos son:

\begin{itemize}

\item \textbf{Lenguaje de programación \enquote{LOGO}}: El lenguaje
    \enquote{LOGO} es la cuna del construccionismo. Fundamentalmente consiste en
    presentar a los niños retos intelectuales que puedan ser resueltos mediante
    el desarrollo de programas en \enquote{LOGO}. El proceso de revisión manual de los
    errores contribuye a que el niño desarrolle habilidades metacognitivas al
    poner en práctica procesos de auto-corrección\cite{logo:sg}.

    Las actividades de programación \enquote{LOGO} se realizan en las áreas de
    matemática, lenguaje, música, robótica, telecomunicaciones y ciencias.
    \enquote{LOGO} es accesible para principantes, especialmente niños pequeños,
    y es compatible con exploraciones complejas y proyectos sofisticados
    realizados por usuarios experimentados\cite{logo:sg}.
    
\item \textbf{\Gls{olpc}}: es una asociación sin ánimos de lucro cuyo esfuerzo
    se centra en dotar a los niños de una computadora duradera, accesible y
    potente en los países en desarrollo, se dice que es un descendiente directo
    del construccionismo\cite{papertian:const}.
	
    Surgió dentro del \gls{mit}, \Gls{olpc} propone un cambio de paradigma
    basado en un modelo de aprendizaje en el que cada alumno disponga de su
    propia computadora portátil y que se pueda conectar a Internet, de forma
    totalmente gratuita, desde su escuela. A partir de esta política se pretende
    disminuir la brecha tecnológica y de acceso de información en países más
    desfavorecidos en comparación con los países del primer
    mundo\cite{videojuegos:gonzaleztardon}.
	
\item \textbf{Fabricación personal}: \textit{Neil Gershenfeld}, colega de
    \textit{Papert} en el Media Lab del \Gls{mit} dictó un curso titulado
    \emph{Cómo hacer casi cualquier cosa}. La idea se centraba en la creación de
    la tecnología que se necesita para resolver los problemas que se poseen.
    Esta auto-confianza, la autonomía personal y la agencia sobre la tecnología
    han estado en el centro de trabajo de \textit{Papert} durante años.
    \textit{Papert} no sólo defendió la idea de que los niños posean
    computadoras personales, sino también que a la larga ellos debían
    mantenerlas, repararlas e incluso construirlas.

    Junto con la capacidad para utilizar la tecnología para inventar soluciones
    a los problemas de significado personal, los estudiantes no sólo tienen
    acceso a la información, sino que tienen una mayor capacidad para darle
    forma a su mundo. La fabricación personal promueve la visión de
    \textit{Papert} \emph{Si se puede utilizar la tecnología para hacer las
        cosas, usted puede hacer las cosas mucho más interesantes y usted puede
        aprender mucho más haciéndolo}\cite{papertian:const}.

\end{itemize}

%http://constructingmodernknowledge.com/cmk08/wp-content/uploads/2012/10/StagerConstructionism2012.pdf

