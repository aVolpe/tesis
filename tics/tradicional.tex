%! TEX root = ../main.tex
\section{Educación Tradicional}

La educación tradicional o instruccionismo se basa en la transferencia de 
conocimiento del profesor al alumno, se enfoca más en el profesor, en la 
capacidad del mismo, y en el producto final como resultado de un proceso 
no interactivo y bien documentado\cite{igi:instructionism}. Los mecanismos 
tradicionales para probar la efectividad de este tipo de enseñanza son los exámenes.

%\fixme{La educación tradicional o instruccionismo se basa en el concepto de que
%   existe un profesor y un alumno. El profesor transfiere el conocimiento que
%    ha adquirido de diferentes métodos (educación, experiencia, etc) a un alumno
%    que es un receptor pasivo de información }{No copy/paste en la
%    intro}\cite{johnson2005instructionism}.

%Se enfoca más en el profesor, y en la enseñanza, y en el producto final como
%resultado de un proceso no interactivo y bien
%documentado\cite{igi:instructionism}. Los mecanismos tradicionales para poder
%probar la efectividad de este tipo de enseñanza son los exámenes.

%\fixme{La principal}{Ablandar} critica a este modelo es que se enfoca la
%enseñanza y no el aprendizaje, mientras más se enseña, más se aprende.
%\textbf{Esto contradice al sentido común en el sentido de que, cosas básicas
%    como caminar o hablar, aprendemos sin la necesidad de un
%    profesor\cite{ackoff:education}\cite{johnson2005instructionism}.}{Más
%    énfasis en el aprender haciendo y usar referencias, no ``el sentido común''}

Una de las principales criticas a este modelo de enseñanza por parte de las
demás corrientes pedagógicas es que según ellas el instruccionismo se basa en la
idea de que  \enquote{cuanto mas se enseña mas se aprende} y esto contradice al
enfoque de \enquote{aprender haciendo} teniendo en cuenta que cosas como caminar
o hablar se aprende sin la necesidad de un profesor
\cite{ackoff:education,johnson2005instructionism}. 

%Tradicionalmente el rol de las TIC's en la educación se vio relegada a la de
%sustituto del libro y/o de presentaciones en clase, es decir, es un mecanismo
%más para trasmitir el conocimiento del maestro al alumno.

