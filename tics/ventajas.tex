\section{Ventajas y desafíos}
\label{sec:tics_ventajas}

Durante la historia de las \Gls{tic} en la educación, se han encontrado
diferentes dificultades a la hora de aplicar los nuevos conceptos en la
educación, desde los primeros enfoques que carecían de bases pedagógicas válidas
hasta la actualidad, el principal problema es falta de motivación de los
profesionales de la educación para emplear las
\Gls{tic}\cite{punie:ict,ict:romeo}.

%\todox{Reconsiderar el párrafo que esta comentado bajo este todo}
%\fixme{El contenido proveído actualmente puede ser considerado como un conjunto
%    de buenas prácticas\cite{punie:ict} y así, omiten completa o parcialmente el
%    contexto donde esa buena práctica fue generado. }{No se entiende de donde
%    sale esto, de que habla y para qué?}

Aún así, las \Gls{tic} han tenido un impacto positivo en la educación, pero el
mismo no es el esperado\cite{punie:ict}, por ejemplo, iniciativas como el
\emph{edutainment} que prometían ser la solución a los problemas
educacionales no cumplieron las expectativas.

Sucesivos fracasos en los resultados obtenidos dotaron a los \emph{edutainment}
de una reputación negativa, y hoy en día son considerados como un método educativo 
ineficiente, pues son un ejercicio de \emph{prueba-error} ocultos bajo un juego poco 
entretenido, además de su incapacidad de enseñar como aplicar conceptos aprendidos 
a un entorno real\cite{resnick:2004}.

Mientras que la utilización de las \Gls{tic} puede eliminar problemas actuales
como el aislamiento y la falta de pensamiento de alto nivel\cite{punie:ict}, la
brecha social existente implica otro riesgo para la utilización de las \Gls{tic}
en la educación, aquellos que no posean los recursos económicos necesarios para
acceder a la misma no se verán beneficiados por las \Gls{tic}\cite{punie:ict}.

Las empresas involucradas en el área de las \Gls{tic} en
educación siguen en la época donde los juegos son prueba y error, esto no
significa que los mismos no funcionen, sino que pueden ser mejorados
considerablemente\cite{egenfeldt2007third}.

Otro de los desafíos actuales es la dificultad comercial impuesta por la
historia de los mismos, es muy difícil para los juegos actuales presentar
promesas realistas, principalmente por el antecedente sentado por los
\emph{edutainment}\cite{egenfeldt2007third}

Las principales ventajas de la utilización de las \Gls{tic} en la educación son:

\begin{itemize}

\item \textbf{Nuevos modelos pedagógicos:} teorías como el constructivismo moderno
    enfatizan el proceso de como adquirir (aprendizaje) conocimiento y no
    solamente como transmitir el conocimiento en sí
    (enseñanza)\cite{guenaga2013serious}.

\item \textbf{Eliminación de distancias:} Con la aparición de las computadoras y los
    satélites,  el mundo se ha convertido en una aldea global, y las distancias
    en cuestiones de transmisión de información se han vuelto
    insignificantes\cite{mohammed2013information},  los medios tradicionales
    como bibliotecas, o escuelas están limitados a un espacio  físico, con el
    uso de las \Gls{tic}, esta restricción física desaparece\cite{tinio:ict}.

\item \textbf{Colaboración distribuida:} como consecuencia del punto anterior, los
    alumnos pueden colaborar de manera más sencilla pues no tienen limitaciones
    físicas. Además, los alumnos pueden consultar con expertos que están en
    linea, e incluso tener mentores en linea, estas tutorías pueden ser uno a
    uno, por ejemplo mediante comunicaciones por correo electrónico. Además
    permite la colaboración masiva entre estudiantes de intereses comunes,
    mediante foros y redes sociales\cite{unesco:ict}.

\item \textbf{Motivación para aprender:} Las \Gls{tic} tienen un impacto positivo en el
    proceso de aprendizaje especialmente en lo referente al compromiso
    con\cite{passey2004motivational,egenfeldt2007third}:
	    
    \begin{itemize}
    \item \textbf{La actividad}: a través de estímulos visuales, auditivos, etc.
    \item \textbf{La capacidad de investigación}: es más fácil acceder a gran cantidad de
        información bibliográfica.
    \item \textbf{La capacidad de escritura y lectura}: emitiendo compartir  ideas de
        manera más legible y mejorarlas iterativamente.
    \item \textbf{La capacidad de presentación}: es más fácil presentar trabajos
        profesionalmente a un público mayor.
    \end{itemize}
	    
\item \textbf{Adquisición de habilidades básicas:} las habilidades necesarias para
    utilizar de manera efectiva las \Gls{tic} se están convirtiendo en una
    necesidad básica, un aprendizaje guiado por las mismas puede ayudar a una
    rápida asimilación de los conceptos relacionados.

\end{itemize}

Uno de los desafíos más importantes que enfrentan las \Gls{tic} para convertirse
en una alternativa viable es la inversión en infraestructura
necesaria\cite{unesco:ict}.

