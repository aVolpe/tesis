\section[Ventajas y desafíos]{Ventajas y desafíos del uso de las TIC en la educación}
\label{sec:tics_ventajas}

Las principales ventajas de la utilización de las \Gls{tic} en la educación son:

\begin{itemize}

\item \textbf{Nuevos modelos pedagógicos:} las nuevas corrientes pedagógicas
    enfatizan el proceso de como adquirir conocimiento (aprendizaje) y no
    solamente como transmitir el conocimiento
    (enseñanza)\cite{guenaga2013serious}.

\item \textbf{Eliminación de distancias:} con la aparición de las computadoras y
    los satélites,  el mundo se ha convertido en una aldea global, y las
    distancias en cuestiones de transmisión de información se han vuelto
    insignificantes\cite{mohammed2013information},  los medios tradicionales
    como bibliotecas, o escuelas están limitados a un espacio  físico, con el
    uso de las \Gls{tic}, esta restricción física
    desaparece\cite{tinio:ict,punie:ict}.

\item \textbf{Colaboración distribuida:} como consecuencia del punto anterior,
    los alumnos pueden colaborar de manera más sencilla pues no tienen
    limitaciones físicas. Las \gls{tic} permiten consultar con expertos o tener
    mentores a través de Internet. Permiten además la colaboración entre
    estudiantes con intereses comunes, mediante foros y redes
    sociales\cite{unesco:ict}.

\item \textbf{Motivación para aprender:} las \Gls{tic} tienen un impacto
    positivo en el proceso de aprendizaje especialmente en lo referente al
    compromiso
    con\cite{passey2004motivational,egenfeldt2007third,martin2008modelo}:
	    
    \begin{itemize}
    \item \textbf{La actividad}: a través de estímulos visuales, auditivos, etc.
    \item \textbf{La capacidad de investigación}: es más fácil acceder a gran cantidad de
        información bibliográfica.
    \item \textbf{La capacidad de escritura y lectura}: permitiendo compartir  ideas de
        manera más legible y mejorarlas iterativamente.
    \item \textbf{La capacidad de presentación}: es más fácil presentar trabajos
        profesionalmente a un público mayor.
    \end{itemize}
	    
\item \textbf{Adquisición de habilidades básicas:} las habilidades necesarias para
    utilizar de manera efectiva las \Gls{tic} se están convirtiendo en una
    necesidad básica, un aprendizaje guiado por las mismas puede ayudar a una
    rápida asimilación de los conceptos relacionados\cite{martin2008modelo}.

\end{itemize}

Durante la historia de las \Gls{tic} en la educación, se han encontrado
diferentes dificultades a la hora de aplicar los nuevos conceptos en la
educación, los principales desafíos son:

\begin{itemize}

\item \textbf{Falta de motivación de los profesionales}: desde los primeros
    enfoques que carecían de bases pedagógicas válidas hasta la
    actualidad\cite{punie:ict,ict:romeo}.

\item \textbf{Brecha social}: la brecha social existente implica otro riesgo
    para la utilización de las \Gls{tic} en la educación, aquellos que no posean
    los recursos económicos necesarios para acceder a la misma no se verán
    beneficiados por las \Gls{tic}\cite{punie:ict}.

\item \textbf{Altas expectativas}: las \Gls{tic} han tenido un impacto positivo
    en la educación, pero el mismo no es el esperado\cite{punie:ict}, por
    ejemplo, iniciativas como el \emph{edutainment} que prometían ser la
    solución a los problemas educacionales no cumplieron las expectativas. 

\item \textbf{Aspectos financieros}: uno de los desafíos más importantes que
    enfrentan las \Gls{tic} para convertirse en una alternativa viable es la
    inversión en infraestructura necesaria\cite{unesco:ict}.

\end{itemize}

