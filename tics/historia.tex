\section{Historia de las TIC en la educación}

La historia de las \Gls{tic} en educación comienza con la Universidad Abierta
del Reino Unido\footnote{Open University of United Kingdom} que en 1969 se
establece como la primera institución educativa dedicada a la enseñanza a
distancia utilizando las, para aquel entonces, nuevas
tecnologías\cite{tinio:ict}.

En 1973 Vint Cerf creo el protocolo TCP/IP y es considerado el nacimiento de
Internet\cite{white:ict}, lo que permitió que la información pueda ser
transmitida de manera más sencilla, tiempo después con la aparición de las
computadoras personales en 1977\cite{white:ict}. 

Otro hito tecnológico se dio en la \Gls{cern} en el año 1989 cuando se concibió
lo que hoy se conoce como \emph{World Wide Web}, permitiendo que los usuarios de
la \emph{Web} puedan compartir archivos mediante un protocolo
estándar\cite{white:ict}. 

Con las principales eventos que marcaron la evolución tecnológica de las
\Gls{tic} en la educación, se divide su historia en cinco partes. Las mismas se
pueden dividir en dos secciones, las primeras tres corresponden a los comienzos
y donde los alumnos eran receptores de información, época denominada, y la
segunda denominada\emph{push}\cite{white:ict} que es aquella donde los alumnos
participan de su educación y son creadores activos de conocimiento.

\subsection{Programación, ejercicios y prácticas}

Este periodo que abarca desde la aparición de las primeras computadoras
personales hasta el final de la década de 1980, este periodo se caracterizo por
computadoras muy limitadas, nula interacción multimedia y escasez de programas
especializados. Se enseñaba programación básica\cite{leinonen:ict}, no por la
necesidad de educar programadores, sino por la creencia de que así se
desarrollarían habilidades matemáticas y lógicas en los alumnos. Los programas
eran muy simples y se basaban en matemáticas y nociones básicas del idioma. 

Esta clase de ejercicios no ayudaron a los alumnos a obtener un aprendizaje
profundo, pues era fácil resolverlos a través de la prueba y el error, la mayor
parte del tiempo servían para distraer a los alumnos no interesados en la
programación mientras el profesor enseñaba programación a aquellos que parecían
interesados\cite{leinonen:ict}.


\subsection{Entrenamiento basado en computadoras}

Cuando aparecieron en el mercado computadoras con multimedia, se argumento que
los ejercicios de la era anterior fallaron en su objetivo de una educación
profunda por que no contenían multimedia\cite{leinonen:ict}, las aplicaciones
eran distribuidas por CD-ROM, y así se actualizaban de manera más frecuente, y
podían contener gran cantidad de contenido multimedia.

En este periodo se desarrollaron una gran cantidad de aplicaciones educativas
que más tarde serían conocidas como \emph{Edutainment}\footnote{Education +
	Enteirtainment, se traduce como educación entretenida}, estas pretendían
agregarle entretenimiento a la educación, se veía al como un receptor pasivo de
información que debía asimilarla, y para aumentar el compromiso, el
entretenimiento era agregado\cite{resnick:2004}.

Las bases pedagógicas de esta se basada en la capacidad de ciertos estudiantes
de aprender mejor cuando interactúan con contenido multimedia, la \emph{prueba y
	error} aún estaban presentes, pero no eran presentados inmediatamente,
sino más bien una vez que el alumno ya debería haber asimilado los conceptos y
funcionaban como pruebas de adquisición de conocimiento. Este tipo de contenido
tampoco logro la enseñanza profunda, solamente fueron efectivos en el
aprendizaje de idiomas, fallando en todos los demás campos\cite{leinonen:ict},
además los contenidos muchas veces estaban desactualizados y obtener nuevas
versiones no era una tarea sencilla.

Varios gobiernos apoyaron de manera agresiva la introducción de las \Gls{tic} en
educación\cite{mcdougall2006theory} y se realizo un importante avance teórico
con los trabajos sobre el aprendizaje construccionista de Papert y Harel (1991),
y la influencia de las computadoras sobre el aprendizaje y la mente de Marvin
Minksy (1987)~\cite{mcdougall2006theory}.

A comienzos de la decada de 1990, con la popularización de Internet, se le vio
como solución al problema de las poco frecuentes actualizaciones de aplicaciones
educativas, su utilización no tenia bases pedagógicas, más bien se basaban en la
facilidad de distribuir contenido por la \emph{Web}, el principal inconveniente
era la velocidad del Internet, no era suficiente para proveer entornos ricos en
multimedia como lo hacían los CD-ROM\cite{leinonen:ict}.

\subsection{e-Learning}

La bases pedagógicas de esta son similares a la era del entrenamiento basado en
computadoras, se distribuye contenido masivamente a los alumnos, y luego, de
manera muy discreta se permite a los mismos colaborar, dejando siempre en claro
que primero se debe asimilar toda la información posible y luego relacionarse
con los demás\cite{leinonen:ict}.

El \emph{e-Learning} apareció a finales de la década de 1990 y tubo su apogeo en
mediados de la década del 2000, apoyada por la gran penetración de las \Gls{tic}
en la población\cite{punie:ict}.

Todos los paradigmas anteriores viven dentro del \emph{e-Learning}, permitiendo
compartir contenido multimedia y realizar pruebas del tipo \emph{prueba-error}. 

Si bien en las anteriores épocas, el uso de las \Gls{tic} estaba más orientado
hacia la educación básica y secundaría, el \emph{e-Learning} actualmente es más
utilizado en la educación terciaría\cite{punie:ict}.

La utilización del \emph{e-Learning} tiene varios grados de aplicación en
entornos reales\cite{punie:ict}, que van desde ser simples elementos
complementarios a la clase, como por ejemplo un repositorio para las
diapositivas y otros materiales de clase, hasta cursos completamente en linea,
donde la clase ha sido completamente sustituída.

