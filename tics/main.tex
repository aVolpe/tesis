\section{TIC's en la educación}

Las Tecnologías de la información y la comunicación (TIC's) son un conjunto de
herramientas tecnológicas y recursos utilizados para comunicar, crear,
diseminar, almacenar y manejar la información\cite{unesco:ict}. Estas
tecnologías abarcan computadoras personales, internet, radio, televisión y
telefonía\cite{tinio:ict}.

Las TIC's fueron utilizadas como complemento a la educación desde los inicias de
la misma con la radio y la televisión. Fueron vistas como un complemento a las
herramientas utilizadas en clase, como complemento del libro, o como una
herramienta que elimina la distancia física entre el profesor y el
alumno\cite{unesco:ict}. 

Las TIC's 
La utilización de las TIC's no mostró una utilidad clara hasta la utilización
masiva de las computadoras, es en esta área donde se encontraron los resultados
más prometedores\cite{unesco:ict}. 

Las principales ventajas de la utilización de las TIC's en la educación es su
aplicabilidad en áreas que no pueden ser cubiertas por otras alternativas, como
son:
\begin{description}
    \item[Nuevos modelos pedagógicos] teorías como el constructivismo moderno
	    enfatizan el proceso de como adquirir conocimiento y no solamente el
	    conocimiento en sí.
    \item[Recursos remotos] Las herramientas tradicionales como las bibliotecas,
	    escuelas y universidades están limitadas a un espacio físico, con
	    las TIC's este requerimiento físico desaparece, prueba de ello es el
	    Internet, que es la colección más grande de información y esta
	    disponible prácticamente a cualquier estudiante.
    \item[Colaboración distribuida] como consecuencia del punto anterior, los
	    alumnos pueden colaborar de manera más sencilla pues no tienen
	    limitaciones físicas.
    \item[Ayuda profesional] Los alumnos pueden consultar con expertos que están
	    en linea, e incluso tener mentores en linea, estas tutorías pueden
	    ser uno a uno, por ejemplo mediante comunicaciones por correo
	    electrónico. Además permite la colaboración masiva entre estudiantes
	    de intereses comunes, mediante foros y redes sociales.
\end{description}

Uno de los desafíos más importantes que enfrentan las TIC's para convertirse en
una alternativa viable es la inversión en infraestructura necesaria.
