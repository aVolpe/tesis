%! TEX root = ../main.tex
\chapter[TIC's en la Educación]{Tecnologías de la Información y Comunicación en
    la Educación}
\label{chap:tics}

Las \Gls{tic} son el conjunto de herramientas tecnológicas y recursos utilizados
para comunicar, crear, diseminar, almacenar y manejar la
información\cite{unesco:ict}. Estas tecnologías abarcan computadores personales,
Internet, radio, televisión y telefonía\cite{tinio:ict}.

La utilización actual de las \Gls{tic} en la educación no es un fenómeno
asilado, responde a una evolución constante de la tecnología y metodología
utilizada\cite{egenfeldt2007third}. Las expectativas iniciales acerca del
impacto de las \Gls{tic} en la educación han sido ampliamente superiores a los
resultados obtenidos\cite{unesco:ict}. Con el advenimiento de las computadoras
se redujo la diferencia entre las expectativas y lo obtenido, en mayor medida
por la utilización de las \Gls{tic} en conjunto con tecnologías como Internet,
así, los efectos positivos en la educación han aumentando
gradualmente\cite{unesco:ict}.

Desde sus inicios, varias corrientes pedagógicas emplean a las \gls{tic} en
mayor o menor medida, desde su utilización como una herramienta para suplantar a
libros impresos, diapositivas, etc\cite{nanjappa2003constructing}; hasta su
utilización como una herramienta de aprendizaje de pensamiento de alto
nivel\cite{egenfeldt2007third,white:ict,nanjappa2003constructing}.

Las corrientes pedagógicas que utilizan a las \Gls{tic} de manera activa son el
instruccionismo o educación tradicional, el conductismo, el constructivismo, y
el construccionismo. Esto no implica que las \Gls{tic} no sean utilizadas por
otras pedagogías, es más, existen otras corrientes que emplean a las \Gls{tic}
de diversas maneras como el
cognoscitivismo\cite{egenfeldt2007third,martin2008modelo} y el
conectivismo\cite{white:ict}. 

Cada corriente pedagógica utiliza las \Gls{tic} de manera diferente, si bien
comparten características, no deben ser consideradas como una sucesión de
pedagogías que desembocan en una pedagogía utilizada actualmente.

En este capítulo se describen las corrientes pedagógicas que utilizan las
\gls{tic} de manera activa, incluyendo sus características, ventajas y
desventajas. Finalmente se describen las ventajas y desafíos que presenta la
utilización de las \Gls{tic} en la educación.

%%! TEX root = ../main.tex
\section{Educación Tradicional}

La educación tradicional o instruccionismo se basa en la transferencia de 
conocimiento del profesor al alumno, se enfoca más en el profesor, en la 
capacidad del mismo, y en el producto final como resultado de un proceso 
no interactivo y bien documentado\cite{igi:instructionism}. Los mecanismos 
tradicionales para probar la efectividad de este tipo de enseñanza son los exámenes.

%\fixme{La educación tradicional o instruccionismo se basa en el concepto de que
%   existe un profesor y un alumno. El profesor transfiere el conocimiento que
%    ha adquirido de diferentes métodos (educación, experiencia, etc) a un alumno
%    que es un receptor pasivo de información }{No copy/paste en la
%    intro}\cite{johnson2005instructionism}.

%Se enfoca más en el profesor, y en la enseñanza, y en el producto final como
%resultado de un proceso no interactivo y bien
%documentado\cite{igi:instructionism}. Los mecanismos tradicionales para poder
%probar la efectividad de este tipo de enseñanza son los exámenes.

%\fixme{La principal}{Ablandar} critica a este modelo es que se enfoca la
%enseñanza y no el aprendizaje, mientras más se enseña, más se aprende.
%\textbf{Esto contradice al sentido común en el sentido de que, cosas básicas
%    como caminar o hablar, aprendemos sin la necesidad de un
%    profesor\cite{ackoff:education}\cite{johnson2005instructionism}.}{Más
%    énfasis en el aprender haciendo y usar referencias, no ``el sentido común''}

Una de las principales criticas a este modelo de enseñanza por parte de las
demás corrientes pedagógicas es que según ellas el instruccionismo se basa en la
idea de que  \enquote{cuanto mas se enseña mas se aprende} y esto contradice al
enfoque de \enquote{aprender haciendo} teniendo en cuenta que cosas como caminar
o hablar se aprende sin la necesidad de un profesor
\cite{ackoff:education,johnson2005instructionism}. 

%Tradicionalmente el rol de las TIC's en la educación se vio relegada a la de
%sustituto del libro y/o de presentaciones en clase, es decir, es un mecanismo
%más para trasmitir el conocimiento del maestro al alumno.



\section[Educación con TIC's]{Educación con las Tecnologías de la Información y
    Comunicación}
\label{sec:tics_educacion}

Las expectativas iniciales acerca del impacto de las \Gls{tic} en la educación
fueron ampliamente superiores a los resultados obtenidos\cite{unesco:ict}. Con
el advenimiento de las computadoras se redujo esta diferencia entre las
expectativas  y lo obtenido, en mayor medida por que se utilizo a las \Gls{tic}
en conjunto con  tecnologías como Internet, y los efectos positivos en la
educación fueron aumentando gradualmente\cite{unesco:ict}.

Una de las principales ventajas de la utilización de las \Gls{tic} en la
educación es su aplicabilidad en áreas como:

\begin{itemize}

\item \textbf{Nuevos modelos pedagógicos:} teorías como el constructivismo moderno
    enfatizan el proceso de como adquirir (aprendizaje) conocimiento y no
    solamente como transmitir el conocimiento en sí (enseñanza).

\item \textbf{Eliminación de distancias:} Con la aparición de las computadoras y los
    satélites,  el mundo se ha convertido en una aldea global, y las distancias
    en cuestiones de transmisión de información se han vuelto
    insignificantes\cite{mohammed2013information},  los medios tradicionales
    como bibliotecas, o escuelas están limitados a un espacio  físico, con el
    uso de las \Gls{tic}, esta restricción física desaparece\cite{tinio:ict}.

\item \textbf{Colaboración distribuida:} como consecuencia del punto anterior, los
    alumnos pueden colaborar de manera más sencilla pues no tienen limitaciones
    físicas. Además, los alumnos pueden consultar con expertos que están en
    linea, e incluso tener mentores en linea, estas tutorías pueden ser uno a
    uno, por ejemplo mediante comunicaciones por correo electrónico. Además
    permite la colaboración masiva entre estudiantes de intereses comunes,
    mediante foros y redes sociales\cite{unesco:ict}.

\item \textbf{Motivación para aprender:} Las \Gls{tic} tienen un impacto positivo en el
    proceso de aprendizaje especialmente en lo referente al compromiso
    con\cite{passey2004motivational,egenfeldt2007third}:
	    
    \begin{itemize}
    \item La actividad: a través de estímulos visuales, auditivos, etc.
    \item La capacidad de investigación: es más fácil acceder a gran cantidad de
        información bibliográfica.
    \item La capacidad de escritura y lectura: emitiendo compartir  ideas de
        manera más legible y mejorarlas iterativamente.
    \item La capacidad de presentación: es más fácil presentar trabajos
        profesionalmente a un público mayor.
    \end{itemize}
	    
\item \textbf{Adquisición de habilidades básicas:} las habilidades necesarias para
    utilizar de manera efectiva las \Gls{tic} se están convirtiendo en una
    necesidad básica, un aprendizaje guiado por las mismas puede ayudar a una
    rápida asimilación de los conceptos \fixme{relacionados}{Por qué?}.

\end{itemize}

Uno de los desafíos más importantes que enfrentan las \Gls{tic} para convertirse
en una alternativa viable es la inversión en infraestructura
necesaria\cite{unesco:ict}.

\subsection{Evolución}

La historia de las \Gls{tic} en educación comienza en la \enquote{Open
    University of United Kingdom} que en $1969$ se establece como la primera
institución educativa dedicada a la enseñanza a distancia utilizando las, para
aquel entonces, nuevas tecnologías\cite{tinio:ict}. 

El análisis de la historia de las \Gls{tic} en educación es indispensable,
aunque existe una corriente que tiende a desestimar las experiencias pasadas,
cuyo principal fundamente es la velocidad con la que la tecnología evoluciona,
es importante el estudio de la evolución de la misma pues los errores
pedagógicos cometidos, aunque puedan parecer evidentes hoy en día, condujeron a
nuevos modelos y conclusiones que son la base de la utilización de las \Gls{tic}
hoy en día\cite{mcdougall2006theory}.

El impacto de las \Gls{tic} en la educación no ha sido constante durante su
historia, sino más bien, ha evolucionado de ser un medio más de traspaso de
información, hasta hoy en día, donde permite generar
conocimiento\cite{tinio:ict}.

\begin{figure}
    \centering
    \includegraphics[scale=0.75]{tics/images/tics_history.png}
    \caption{Utilización de las \Gls{tic} en la educación desde el año $1975$}
    \label{fig:history_tics}
\end{figure}

Para entender la historia de las \Gls{tic} en la educación, se presenta el
gráfico~\ref{fig:history_tics}, en el cual se observa la evolución que sufrió la
utilización de las \Gls{tic} como herramienta en la educación. Se observa que se
parte la historia en cinco corrientes definidas, y a la vez, estas corrientes se
agrupan según el mecanismo de obtención de información, las tres primeras
corrientes se denominan \textit{pull} y las siguientes dos se denominan
\textit{push}. \textit{Pull} se refiere a que los estudiantes obtenían la
información sin participar en la creación de la misma, y \textit{push} es cuando
los alumnos son creados activos de conocimiento\cite{white:ict,leinonen:ict}.

Aunque la figura~\ref{fig:history_tics} muestre un progreso lineal de las
corrientes, este progreso no es igual en todo el mundo, y la el grado de impacto
de las \Gls{tic} varia entre países, lo que se conoce como una brecha
tecnológica. Las fechas utilizadas en el figura~\ref{fig:history_tics} son
relacionadas a la evolución en los Estados Unidos de Norte America.

\subsubsection{Pull}

A finales de la década de $1970$ e inicios de la década de $1980$,  la
complejidad técnica de las computadoras limitaba la cantidad de herramientas
disponibles, los programas eran desarrollados por profesores, y su objetivo era
que los alumnos puedan poner en práctica lo aprendido en el aula. El campo de
aplicación de las herramientas se limitaban a matemáticas y lenguaje, donde se
podía evaluar inmediatamente los resultados proveídos por los alumnos, pues,
normalmente era un enunciado y una lista posible de opciones. La cantidad
limitada de opciones para responder, provocó que los alumnos no interpreten los
resultados, sino prueben todas las posibles opciones hasta pasar al siguiente
enunciado, sin realizar ningún aprendizaje significativo\cite{leinonen:ict},
este tipo de juegos se denomina \enquote{Prueba y Error}.

Cuando aparecieron en el mercado computadoras con multimedia, a finales de la
década de $1980$, la utilización de las \Gls{tic} se simplificaron y dieron
contenido a la posibilidad de incluir contenido multimedia, se argumentó que los
ejercicios de tipo \enquote{Prueba y Error} no cumplieron su objetivo de una
educación profunda por que no contenían multimedia\cite{leinonen:ict}, así, en
se empezaron a distribuir las aplicaciones eran distribuidas por
\textit{CD-ROM}, se actualizaban de manera frecuente, y podían contener gran
cantidad de contenido multimedia.

Con la creación de los juegos del tipo \enquote{Prueba y Error} y el contenido
multimedia, se inicio a un nueva corriente denominada \emph{Edutainment},
palabra que representa la unión de la educación y el entretenimiento. Así el
\emph{edutainment} pretende agregar entretenimiento a la educación, se ve al
alumno como un receptor pasivo de información que debe asimilarla, y para
aumentar la implicación de los alumnos, el entretenimiento era
agregado\cite{resnick:2004}.

Los \emph{edutainment} se basan principalmente en el conductismo y el
cognoscitivismo, se enfoca en juegos sencillos que transmiten información simple
al usuario, su estructura se basa en un objetivo claro que está separado de la
experiencia educativa\cite{egenfeldt2007third}. 

\emph{Math Blaster} (ver~\ref{fig:math_blaster}) es un \emph{edutainment} donde
el alumno debe responder repetitivamente preguntas aritméticas para obtener
municiones, luego con esas municiones debe completar diferentes misiones en una
nave\cite{bruckman1999can}. Como todas las preguntas se responden mediante un
mecanismo de selección múltiple, y no existe penalización por fallar una
respuesta, los alumnos no reflexionan sobre las respuestas elegidas, seleccionan
una opción aleatoria y si no es la correcta, prueban otra, tras una cantidad
finita de intentos, siempre se obtiene la recompensa deseada.

\begin{figure}[ht!] 
\centering 
\includegraphics[scale=0.5,natwidth=296,natheight=217]{tics/images/math_blaster.jpg}
\caption{Math Blaster, \emph{edutainment} del año 1987}
\label{fig:math_blaster} 
\end{figure}

\begin{figure}[ht!] 
\centering 
\includegraphics[scale=0.5]{tics/images/carmen.jpg}
\caption{Donde en el mundo esta Carmen Sandiego} 
\label{fig:carmen}
\end{figure}

\enquote{Donde en el mundo esta Carmen Sandiego} (ver~\ref{fig:carmen}) es un
juego que representa el potencial multimedia de esta época, el objetivo del
juego era detener a una serie de criminales mediante una serie de pistas que
eran provistas en forma de texto. Este exitoso juego demuestra las falencias de
esta época, siendo visualmente muy atractivo, y con contenido multimedia acorde
a su tiempo, no era más que \emph{prueba y error}, cada nivel del juego podía
ser completado sin leer la información proveída educativa\cite{charsky:2010}.

Los \emph{edutainment} no logran enseñar habilidades complejas, se enfocan
principalmente en enseñar tareas extremadamente repetitivas que no dependen de
un contexto\cite{charsky:2010,egenfeldt2007third,bruckman1999can}, son
excelentes para enseñar a sumar, pero no para aplicar ese conocimiento, analizar
y obtener conclusiones, o evaluar lo que aprendieron.

Las principales causas por del fracaso de los \emph{edutainment} en su intento
de ser una alternativa viable a la educación son según\cite{egenfeldt2007third}: 

\begin{itemize}

\item \textbf{Falta de motivación interna:} los \emph{edutainment} se centran en
    motivaciones externas, y dejan de lado la motivación interna. Se centran en
    dar recompensas por acciones lo que es una motivación externa, que en que el
    alumno se sienta emocionado al finalizar un nivel, lo que es una forma de
    motivación interna.

\item \textbf{Aprendizaje como anexo:} el principal objetivo del desarrollo de un
    \emph{edutainment} es el de entretener, los objetivos pedagógicos son
    agregados al final. Adicionalmente, este aprendizaje se provee a través de
    largos textos que normalmente son omitidos.

\item \textbf{Interacción limitada:} son construidos con una jugabilidad pobre,
    sin la posibilidad de realizar multiples acciones, normalmente limitados a
    seleccionar respuestas o moverse en un pequeño mundo. 

\item \textbf{Ejercicios de prueba y error sistemáticos:} todas las debilidades
    anteriores se pueden fundamentar en el hecho de que los juegos permiten al
    alumno intentar varias veces sin ser penalizados, además
    de que los alumnos no están motivados, provocaba que todas las opciones sean
    probadas sin el proceso de reflexión necesario para aprender, por ejemplo,
    varios juegos aritméticos solicitaban pruebas del tipo $2+2$ donde el alumno
    probaba diferentes resultados y luego memorizaba el mismo. Se enseñaba a
    probar opciones sin sentido antes que entender y analizar la experiencia.

\end{itemize}

Los \emph{edutainment} se limitan a enseñar tareas mecánicas, sin entender el
contexto en el que se aplican y por qué se realizan\cite{egenfeldt2007third}.

El contenido distribuido, aunque contribuyo a la calidad del contenido proveído,
no resolvió el problema de contenido actualizado, pues a comienzos de la década
de $1990$, con la popularización de Internet\footnote{Internet se concibió lo
    que hoy se conoce como \emph{World Wide Web}\cite{white:ict}, pero se
    masifico en la década de $1990$\cite{leinonen:ict}}, se genera más contenido
que el que puede ser distribuido por medios físicos, así se accede al tercer
hito de la figura~\ref{fig:history_tics}, donde el contenido es distribuido por
internet.

Las bases pedagógicas de los \emph{Edutainment} siguen siendo las mismas, la
incursión de internet solo permite contenido actualizado, los problemas
persisten y se crean nuevos factores que incrementan la brecha tecnológica,
ahora, además de poseer una computadora, es necesaria una conexión permanente.
La velocidad inicial de internet no es suficiente para proveer los mismos
entornos ricos en multimedia que sí lo proveían los CD-ROM\cite{leinonen:ict}.

Así se crea una nueva generación de \emph{edutainment}, que son distribuidos por
internet, actualizados constantemente, pero con los mismos problemas
\cite{leinonen:ict}.

Un cambio de paradigma permite poner fin a la época de los \emph{Pull}, se pone
especial énfasis en el aprendiz y no en el contenido, así se inicia la época
\emph{Push}.

\subsubsection{Push}

Se inicia con el \emph{E-Learning}, \emph{E-Learning} se define como la
educación y capacitación a través de medios digitales, incluye todo tipo de
medio capaz de distribuir información, puede ser de en tiempo real como
salas de conversaciones y videoconferencias o puede ser diferido, como por
ejemplo foros, enciclopedias. Es particularmente útil para educación a distancia
y con horarios flexibles. Se originó a finales de la década de $1990$ y tuvo su
apogeo a mediados de la década del $2000$, apoyada por la gran penetración de
las \Gls{tic} en la población\cite{punie:ict}.

Se distribuye contenido masivamente a los alumnos, y luego, de manera discreta
se permite a los mismos colaborar, dejando siempre en claro que primero se debe
asimilar toda la información posible y luego relacionarse con los
demás\cite{leinonen:ict}, esto evidencia algunos problemas pedagógicos heredados
de la época \emph{Pull}.


\begin{figure}[h] 
\centering 
\includegraphics[scale=0.5]{tics/images/moodle.jpg}
\caption{Moodle, plataforma de e Learning} 
\label{fig:moodle}
\end{figure}


La plataforma \emph{Moodle} (ver~\ref{fig:moodle}) cuya primera versión salió en
el $2002$, es una de las principales herramientas del \emph{e-Learning} hoy en
día, permite la creación de cursos específicos por materia y sitios
especializados por instituciones académicas\cite{perkins2006using}. 

La utilización del \emph{e-Learning} tiene varios grados de aplicación en
entornos reales\cite{punie:ict}, que van desde ser simples elementos
complementarios a la clase, como por ejemplo un repositorio para las
diapositivas y otros materiales de clase, hasta cursos completamente en línea,
donde la clase ha sido completamente sustituida.

Sí bien, el \emph{e-Learning} permite la distribución y colaboración en
distintos niveles, no hace un enfoque en el aspecto pedagógico, y se centra en
la forma de transmitir información, y no como la recibe el
alumno\cite{leinonen:ict}.

Ideas que surgieron en la época \emph{Pull} son tomadas en cuenta a la hora de
diseñar nuevos esquemas de educación\cite{mcdougall2006theory}, las teorías del
construccionismo de \textit{Papert}, permiten enfocarse en el alumno, además del
ambiente en el que ocurre el aprendizaje\cite{egenfeldt2007third}.


%\section{Construccionismo y las TIC's}
\label{sec:tics_CONSTRUCCIONISMO}

El construccionismo es una corriente pedagógica que parte de una concepción del
aprendizaje según la cual la persona aprende por medio de su interacción
dinámica con el mundo físico, social y cultural en el que está
inmerso\cite{valdivia:sg}.

Posee un enfoque diferente en cuanto al uso de las \Gls{tic} en la educación.
Esta pedagogía se diferencia de la educación tradicional en que el estudiante ya
no es un receptor pasivo de información, en cambio, el mismo participa
activamente del proceso de aprendizaje construyendo su propio conocimiento. Se
diferencia del instruccionismo en que el construccionismo utiliza la tecnología
como medio cognitivo y no para la entrega de contenido.

Se considera al construccionismo como una alternativa prometedora a la educación
tradicional. Desde el punto de vista tecnológico, es ideal pues el mismo
requiere un alto dinamismo en el traspaso del conocimiento
\cite{sasha:construtivism}. 

El construccionismo y las \Gls{tic} siempre han estado relacionados, ya que el
mismo se originó con un lenguaje de programación (LOGO)\cite{ict:ttc}. Un
característica importante de esta relación es que tienen la capacidad de
eliminar los problemas de distancia\cite{mariluz:seiousgames}.


\subsection{Historia}

A mediados de la década de 1960 Seymour Papert, un matemático sudafricano, llegó
a los Estados Unidos, donde fue co-fundador del Laboratorio de Inteligencia
Artificial del \Gls{mit} con Marvin Minsky\cite{logo:sg}. 


En la década de $1980$, Seymour Papert acuñó el término construccionismo en la
Fundación Nacional de Ciencia de los Estados Unidos titulada
\enquote{Constructionism: A New Opportunity for Elementary Science Education}
para presentar un método pedagógico que se basaba en muchas de las ideas de la
educación progresiva estudiadas por el estadounidense John Dewey en el inicio
del siglo 20 en su escuela experimental en la Universidad de Chicago. Dewey
quería poner gran parte de la responsabilidad de aprender en los estudiantes que
han nacido con el don de aprendizaje y la creación de conocimiento en sus
propios términos\cite{historia:2014}.

Papert también fue influenciado por Maria Montessori quien luego de grados en
pediatría, medicina, psicología y filosofía comenzó su propia escuela
experimental para niños pequeños. En lugar de que ella misma estableciera
formalmente las tareas, vio como sus estudiantes actuaron por su cuenta. Ella
siguió los intereses de los estudiantes y observó como respondieron a sus
entornos especialmente preparados. Ella preparó el escenario pero no ofreció una
guía explícita\cite{historia:2014}.

Papert trabajó directamente con el psicólogo suizo Jean Piaget a quien Ernst von
Glasersfeld llamó \enquote{El pionero de la teoría constructivista del
    aprendizaje}. 
% esto de glaserfeld pio importa?
Piaget, al igual que Dewey, Montessori y otros, desarrolló su teoría de la
educación y la construcción del conocimiento observando e interactuando con los
niños. De estas observaciones nació el movimiento constructivista. El
contructivismo se basa en que el conocimiento debe ser construído por el
estudiante y los nuevos significados deben ser obtenidos relacionándolos a
significados anteriores en el propio sistema de relaciones del
estudiante\cite{historia:2014}. 

Papert admitió que jugó más con la palabra construcción. El contruccionismo de
Papert se diferencia del constructivismo de Piaget en que los estudiantes
construyen las ideas o partes del mundo utilizando herramientas. Para Papert los
estudiantes necesitan contruir modelos de partes de su mundo con el fin de
comprender más plenamente el significado, el 
% la definición de modelos podemos poner un footnote?
contenido y la dinámica de las partes. La elaboración de representaciones
mentales mediante la construcción y el intercambio es la metáfora del marco
construccionista\cite{historia:2014}.

%Durante $1980$, Seymor Papert, Wally Feurzeig, Marvin Minsky y John McCarthy y los
%miembros del Departamento de Inteligencia Artificial del \Gls{mit} y una
%compañía de tecnología en Cambridge, Massachusetts, desarrollaron un nuevo
%lenguaje de programación llamado LOGO que tenía por objeto que los estudiantes
%construyeran sus modelos mentales en notación LOGO. Este juego
%introduciría de forma natural las ideas de los procedimientos, funciones,
%variables, recursividad, la modularidad, simulación, verificación, entre %otros\cite{historia:2014}.

Papert trabajó con el equipo de Bolt, Beranek y Newman, liderado por Wallace
Feurzeig, que creó la primera versión del lenguaje de programación LOGO en 1967.
LOGO es un dialecto de Lisp, fue diseñado como una herramienta para el
aprendizaje. Sus características, como la modularidad, extensibilidad,
interactividad y flexibilidad, derivan de este objetivo \cite{logo:sg}.

Los desarrolladores de LOGO no solo alentaron la promoción de formas
construccionistas de enseñanza y aprendizaje sino también alentaron otra forma
de aprendizaje nueva y no tradicional con las diferentes herramientas
tecnológicas\cite{historia:2014}. 

Por lo tanto, se puede decir que la creación de LOGO permitió la creación del
construccionismo\cite{historia:2014}.

    
    
% ESTE
% Cuando se produjo LOGO y se acuñó el construccionismo, la comunidad del 
% construccionismo era en su mayoría ingenieros informáticos y matemáticos\cite{historia:2014}.
% Esto es relevante?

\observacion{Toda la sección hay que reordernar}

\subsection{Bases Pedagógicas}

Para el construccionismo, el conocimiento es construido por el estudiante en
lugar de ser trasmitido por el profesor \cite{moses:2003} y esto sucede particularmente cuando
el mismo se compromete en la elaboración de un producto o artefacto que tenga un
significado y pueda ser compartido\cite{valdivia:sg}. De esta manera, se permite
a los estudiantes elaborar sus propias interpretaciones razonadas del mundo
mediante la interacción con el mismo. El profesor actúa como guía para el estudiante 
en la construcción de su conocimiento, aportando conocimiento y experiencia.

Según Papert, los alumnos estarán mucho más involucrados en su aprendizaje si
construyen artefactos que los demás pueden ver, criticar y tal vez utilizar. Y
además, el alumno se enfrenta a problemas complejos con estas construcciones,
harán el esfuerzo por resolver problemas y aprender ya que la construcción les
motivará\cite{const:vs}.

El enfoque construccionista establece que los seres humanos conocen y aprenden
de formas diferentes por lo tanto, no se puede elaborar una jerarquía de estilos
de aprendizajes\cite{valdivia:sg}.

\subsection{Actualidad}

%El construccionismo pone énfasis en el \emph{Aprender haciendo}, esta idea
%\fixme{mejora}{ref} la práctica educativa tradicional o instruccionismo. %El instruccionismo
%se basa en el concepto de que existe un profesor y un estudiante, el profesor
%transfiere el conocimiento que ha adquirido a un alumno que es receptor pasivo
%de información de esta manera, se enfoca más en la capacidad del profesor. 

Existen varios proyectos o emprendimientos que incluyen al contruccionismo como
base pedagógica, para la mayoría de ellos las computadoras son esenciales
mientras que para otros el mayor esfuerzo está en la incorporación de la
tecnología en su práctica educativa\cite{papertian:const}.

Algunos de estos emprendimientos son:

\begin{itemize}

\item \textbf{Lenguaje de programación LOGO}: El lenguaje LOGO es la cuna del construccionismo. %, se basa en el
%	principio de que se aprende mejor haciendo, pero se aprende todavía
%	mejor si se combina la acción con la verbalización  y la reflexión
%	acerca de lo que se ha hecho. 
	Fundamentalmente consiste en presentar a
	los niños retos intelectuales que puedan ser resueltos mediante el
	desarrollo de programas en LOGO. El proceso de revisión manual de los
	errores contribuye a que el niño desarrolle habilidades metacognitivas
	al poner en práctica procesos de auto-corrección\cite{logo:sg}.
	%http://es.wikipedia.org/wiki/Logo_(lenguaje_de_programaci%C3%B3n)


    %A mediados de la década de 1960 Seymour
	%Papert, que había estado trabajando con Piaget en Ginebra, llegó a
	%Estados Unidos donde co-fundó el Laboratorio de Inteligencia Artifical
	%del MIT con Marvin Minsky. Papert trabajó con el equipo de Bolt, Beranek
	%y Newman, liderado por Wallace Feurzeig, que creó la primera versión del
	%logotipo en 1967. A lo largo de la década de 1970 Logo fuen incubado en
	%el MIT y algunos otros sitios de investigación. El lenguaje de
	%programación Logo, un dialecto de Lisp, fue diseñado como una
	%herramienta para el aprendizaje. Sus características como la
	%modularidad, extensibilidad, interactividad y flexibilidad se derivan de
	%este objetivo. 
	%http://el.media.mit.edu/logo-foundation/logo/index.html

    Las actividades de programación LOGO se realizan en las áreas de matemática, lenguaje, 
    música, robótica, telecomunicaciones y ciencias. LOGO es accesible para novatos, 
    incluyendo niños pequeños, y también es compatible con exploraciones complejas y 
    proyectos sofisticados realizados por usuarios experimentados\cite{logo:sg}.
    
    Uno de los ambientes más populares de LOGO incluye a la tortuga, originalmente 
    era un robot que era puesto en el suelo y se podía dirigir su movimiento 
    escribiendo comandos en el ordenador. Pronto la tortuga emigró a las 
    pantallas gráficas de los ordenadores en donde se las utiliza para dibujar 
    formas y diseños\cite{logo:sg}.

    

%\item[Simulación] La simulación en el ámbito de la educación fue evolucionando
%desde simples motores de reglas hasta complejos entornos, la simulación
%demostró ser una herramienta muy útil el ámbito laboral
%\cite{mariluz:seiousgames}, pues enseña al alumno a encarar situaciones muy
%difíciles de representar en entornos completamente controlados y provee
%mecanismos para comprobar la efectividad de la herramienta. 

%Actualmente la simulación se utiliza más en el ámbito empresarial pues las
%empresas son las más necesitas de innovar en el ámbito de la enseñanza. Un
%ejemplo de esta necesidad se da, por ejemplo, en el entrenamiento de nuevos
%vendedores, es muy difícil enseñar a un vendedor como debe vender los productos
%con un pizzarón y/o una presentación, en cambio la simulación permite que el
%mismo pueda probar cosas nuevas y experiencias de sus compañeros (o
%instructor), convirtiendo así el aprendizaje en
%colectivo\cite{mariluz:seiousgames}. En el ámbito académico la simulación mas
%utilizada en campos físicos (como simulación de fluidos), meteorología
%(simulación de tormentas y fenómenos climáticos), etc. 

%\item[Serious Games] Diseñado con el propósito de aprender. Generalmente hace
%uso de la simulación para permitir un aprendizaje más realista.

%\item[Lego Serious Play] Es una iniciativa de Lego que busca fomentar el
%pensamiento creativo por medio de la construcción por parte de los estudiantes
%de su identidad y experiencias utilizando legos. 

\item \textbf{\Gls{olpc}}: es una asociación sin ánimos de lucro cuyo esfuerzo
    se centra en dotar a los niños de una computadora duradera, accesible y
    potente en los países en desarrollo, se dice que es un descendiente directo
    del construccionismo\cite{papertian:const}.
	
    Surgió dentro del MIT Media Lab, \Gls{olpc} propone un cambio de paradigma
    basado en un modelo de aprendizaje en el que cada alumno disponga de su
    propia computadora portátil y que se pueda conectar a internet, de forma
    totalmente gratuita, desde su escuela. A partir de esta política se pretende
    disminuir la brecha tecnológica y de acceso de información en países más
    desfavorecidos en comparación con los países del primer
    mundo\cite{videojuegos:gonzaleztardon}.
	
	%Con esto se busca que
	%la computadora personal sea utilizada como un laboratorio intelectual y
	%un vehículo para la auto-expresión. OLPC no tiene que ver con la
	%escolarización o la escuela, más bien las utiliza como medio de
	%distribución de las computadoras a los niños, los cuales pueden
	%utilizarlas para aprender en cualquier lugar y momento. Se busca
	%fomentar el aprendizaje natural, es decir, aquel aprendizaje sin
	%enseñanza.

    %\fixme{Los problemas atribuidos al experimento}{experimento?} OLPC son
    %predominantemente las críticas a la política, el liderazgo o de la
    %intransigencia de la escuela en vez del construccionismo o computadora
    %personal para los niños pobres. El experimento audaz de Nicholas Negroponte
    %(co-fundador de \Gls{olpc}) y Sugata Mitra para dejar las computadoras desde
    %un helicóptero sobre una aldea de África se basa en la creencia en el
    %construccionismo\cite{papertian:const}.

\item \textbf{Fabricación personal}: Neil Gershenfeld, colega de Papert en el
    Media Lab del \Gls{mit} dictó un curso titulado \emph{Cómo hacer casi
        cualquier cosa}. La idea se centraba en la creación de  la tecnología
    que se necesita para resolver los problemas que se poseen. Esta
    auto-confianza, la autonomía personal y la agencia sobre la tecnología han
    estado en el centro de trabajo de Papert durante años. Papert no sólo
    defendió la idea de que los niños posean computadoras personales, sino
    también que a la larga ellos debían mantenerlas, repararlas e incluso
    construirlas.

	Junto con la capacidad para utilizar la tecnología para inventar
	soluciones a los problemas de significado personal, los estudiantes no
	sólo tienen acceso a la información, sino que tienen una mayor capacidad
	para darle forma a su mundo. La fabricación personal promueve la visión
	de Papert \emph{Si se puede utilizar la tecnología para hacer las cosas,
	usted puede hacer las cosas mucho más interesantes y usted
	puede aprender mucho más haciéndolo}\cite{papertian:const}.

\end{itemize}

%http://constructingmodernknowledge.com/cmk08/wp-content/uploads/2012/10/StagerConstructionism2012.pdf






\section{Historia de las TIC en la educación}

La historia de las \Gls{tic} en educación comienza con la Universidad Abierta
del Reino Unido\footnote{Open University of United Kingdom} que en 1969 se
establece como la primera institución educativa dedicada a la enseñanza a
distancia utilizando las, para aquel entonces, nuevas
tecnologías\cite{tinio:ict}.

En 1973 Vint Cerf creo el protocolo TCP/IP y es considerado el nacimiento de
Internet\cite{white:ict}, lo que permitió que la información pueda ser
transmitida de manera más sencilla, tiempo después con la aparición de las
computadoras personales en 1977\cite{white:ict}. 

Otro hito tecnológico se dio en la \Gls{cern} en el año 1989 cuando se concibió
lo que hoy se conoce como \emph{World Wide Web}, permitiendo que los usuarios de
la \emph{Web} puedan compartir archivos mediante un protocolo
estándar\cite{white:ict}. 

Con las principales eventos que marcaron la evolución tecnológica de las
\Gls{tic} en la educación, se divide su historia en cinco partes. Las mismas se
pueden dividir en dos secciones, las primeras tres corresponden a los comienzos
y donde los alumnos eran receptores de información, época denominada, y la
segunda denominada\emph{push}\cite{white:ict} que es aquella donde los alumnos
participan de su educación y son creadores activos de conocimiento.

\subsection{Programación, ejercicios y prácticas}

Este periodo que abarca desde la aparición de las primeras computadoras
personales hasta el final de la década de 1980, este periodo se caracterizo por
computadoras muy limitadas, nula interacción multimedia y escasez de programas
especializados. Se enseñaba programación básica\cite{leinonen:ict}, no por la
necesidad de educar programadores, sino por la creencia de que así se
desarrollarían habilidades matemáticas y lógicas en los alumnos. Los programas
eran muy simples y se basaban en matemáticas y nociones básicas del idioma. 

Esta clase de ejercicios no ayudaron a los alumnos a obtener un aprendizaje
profundo, pues era fácil resolverlos a través de la prueba y el error, la mayor
parte del tiempo servían para distraer a los alumnos no interesados en la
programación mientras el profesor enseñaba programación a aquellos que parecían
interesados\cite{leinonen:ict}.


\subsection{Entrenamiento basado en computadoras}

Cuando aparecieron en el mercado computadoras con multimedia, se argumento que
los ejercicios de la era anterior fallaron en su objetivo de una educación
profunda por que no contenían multimedia\cite{leinonen:ict}, las aplicaciones
eran distribuidas por CD-ROM, y así se actualizaban de manera más frecuente, y
podían contener gran cantidad de contenido multimedia.

En este periodo se desarrollaron una gran cantidad de aplicaciones educativas
que más tarde serían conocidas como \emph{Edutainment}\footnote{Education +
	Enteirtainment, se traduce como educación entretenida}, estas pretendían
agregarle entretenimiento a la educación, se veía al como un receptor pasivo de
información que debía asimilarla, y para aumentar el compromiso, el
entretenimiento era agregado\cite{resnick:2004}.

Las bases pedagógicas de esta se basada en la capacidad de ciertos estudiantes
de aprender mejor cuando interactúan con contenido multimedia, la \emph{prueba y
	error} aún estaban presentes, pero no eran presentados inmediatamente,
sino más bien una vez que el alumno ya debería haber asimilado los conceptos y
funcionaban como pruebas de adquisición de conocimiento. Este tipo de contenido
tampoco logro la enseñanza profunda, solamente fueron efectivos en el
aprendizaje de idiomas, fallando en todos los demás campos\cite{leinonen:ict},
además los contenidos muchas veces estaban desactualizados y obtener nuevas
versiones no era una tarea sencilla.

Varios gobiernos apoyaron de manera agresiva la introducción de las \Gls{tic} en
educación\cite{mcdougall2006theory} y se realizo un importante avance teórico
con los trabajos sobre el aprendizaje construccionista de Papert y Harel (1991),
y la influencia de las computadoras sobre el aprendizaje y la mente de Marvin
Minksy (1987)~\cite{mcdougall2006theory}.

A comienzos de la decada de 1990, con la popularización de Internet, se le vio
como solución al problema de las poco frecuentes actualizaciones de aplicaciones
educativas, su utilización no tenia bases pedagógicas, más bien se basaban en la
facilidad de distribuir contenido por la \emph{Web}, el principal inconveniente
era la velocidad del Internet, no era suficiente para proveer entornos ricos en
multimedia como lo hacían los CD-ROM\cite{leinonen:ict}.

\subsection{e-Learning}

La bases pedagógicas de esta son similares a la era del entrenamiento basado en
computadoras, se distribuye contenido masivamente a los alumnos, y luego, de
manera muy discreta se permite a los mismos colaborar, dejando siempre en claro
que primero se debe asimilar toda la información posible y luego relacionarse
con los demás\cite{leinonen:ict}.

El \emph{e-Learning} apareció a finales de la década de 1990 y tubo su apogeo en
mediados de la década del 2000, apoyada por la gran penetración de las \Gls{tic}
en la población\cite{punie:ict}.

Todos los paradigmas anteriores viven dentro del \emph{e-Learning}, permitiendo
compartir contenido multimedia y realizar pruebas del tipo \emph{prueba-error}. 

Si bien en las anteriores épocas, el uso de las \Gls{tic} estaba más orientado
hacia la educación básica y secundaría, el \emph{e-Learning} actualmente es más
utilizado en la educación terciaría\cite{punie:ict}.

La utilización del \emph{e-Learning} tiene varios grados de aplicación en
entornos reales\cite{punie:ict}, que van desde ser simples elementos
complementarios a la clase, como por ejemplo un repositorio para las
diapositivas y otros materiales de clase, hasta cursos completamente en linea,
donde la clase ha sido completamente sustituída.


\section{Teorías del aprendizaje: Modelo Push}


La primera manifestación de las \gls{tic} en la educación se da como un
sustituto a los medios tradicionales. Los materiales didácticos son
responsabilidad de los profesores, imprentas y
academias\cite{leinonen:ict,white:ict}.

El modelo \textit{push} favorece a las corrientes pedagógicas instruccionismo y
conductismo. El instruccionismo utiliza a las \gls{tic} para resolver problemas
de distancia y de costo\cite{igi:instructionism,johnson2005instructionism},
mientras que el conductismo utiliza a las \gls{tic} como un medio para proveer
refuerzos positivos\cite{weegar2012comparison}.

\subsection{Instruccionismo}

La educación tradicional o instruccionismo se basa en la transferencia de
conocimiento del profesor al alumno, se enfoca más en el profesor, en la
capacidad del mismo, y en el producto final como resultado de un proceso no
interactivo y bien
documentado\cite{igi:instructionism,johnson2005instructionism}. Los mecanismos
tradicionales para probar la efectividad de este tipo de enseñanza son los
exámenes escritos.

El instruccionismo es conocido además como enseñanza sistemática, enseñanza
explícita, enseñanza directa, y enseñanza
activa\cite{johnson2005instructionism}. Siempre se enfatiza en el
profesor\cite{johnson2005instructionism}.

Epistemológicamente se puede observar al instruccionismo como objetivo, pues
considera que el conocimiento es independiente del entorno, se asume que el
mismo es isomorfo, si el profesor puede enseñar, el alumno puede
aprender\cite{johnson2005instructionism}.

En el instruccionismo, la utilización de las \Gls{tic} se centra principalmente
en mecanismos para proveer contenido, se utilizan plataformas que
permiten a los profesores distribuir contenido y otras actividades relacionadas.

\subsubsection{Ejemplos}

Las aplicaciones de las \gls{tic} en el instruccionismo son:

\begin{itemize}

\item \textbf{E-Learning}: El \emph{E-Learning} se define como la educación y
    capacitación a través de medios digitales, incluye todo tipo de medio capaz
    de distribuir información, puede ser en tiempo real como salas de
    conversaciones y videoconferencias o puede ser diferido, como por ejemplo
    foros y enciclopedias\cite{punie:ict}.

    Una de las desventajas del \emph{e-Learning} es que se distribuye contenido
    masivamente a los alumnos, y luego, de manera discreta se permite a los
    mismos colaborar, dejando siempre en claro que primero se debe asimilar toda
    la información posible y luego relacionarse con los
    demás\cite{leinonen:ict}, es decir, los alumnos no forman parte de la
    creación del conocimiento.

    \begin{figure}[h] 
    \centering 
    \includegraphics[scale=0.5]{tics/images/moodle.jpg}
    \caption{Moodle, plataforma de e Learning}\label{fig:moodle}
    \end{figure}

    La plataforma \emph{Moodle} (ver figura~\ref{fig:moodle}) cuya primera
    versión salió en el $2002$, es una de las principales herramientas del
    \emph{e-Learning} hoy en día, permite la creación de cursos específicos por
    materia y sitios especializados por instituciones
    académicas\cite{perkins2006using}. 

\item \textbf{Sustituto de medios tradicionales}: la utilización de las
    \gls{tic} desde sus inicios es como sustituto de los medios físicos, como
    libros y diapositivas por sus versiones digitales\cite{tinio:ict}.

\item \textbf{Clases a distancia}: las clases por videoconferencias y clases
    grabadas en video son herramientas utilizadas en reemplazo de las clases
    tradicionales. Permiten eliminar las distancias de tiempo y
    espacio\cite{tinio:ict}.

\end{itemize}

\subsection{Conductismo}

El conductismo es una corriente de la psicología, creada por \textit{Jhon
    Watson}, y posteriormente perfeccionada por \textit{Pavlov},
\textit{Skinner}, y \textit{Thorndik}. El conductismo defiende la idea de que
todas las acciones que realizan los seres vivos son consecuencia de un
estímulo\cite{weegar2012comparison}.

La primera utilización del conductismo con las \Gls{tic} es presentada por
\textit{Skinner}, en $1958$\cite{weegar2012comparison}, donde se describe una
máquina que contiene botones y una pantalla donde se presenta una pregunta, para
responder, el usuario dispone de varias opciones, cada opción esta relacionada
con un botón, si el aprendiz no presiona el botón correcto, debe seguir
intentando hasta responder correctamente y así
poder avanzar\cite{weegar2012comparison}, este es el inicio de lo que se conoce como
\enquote{Prueba y Error}.

Una característica del conductismo, es la ley de \textit{Thorndike}, la cual
indica que una acción, cuya consecuencia es un estímulo favorable, es más
probable que sea repetida\cite{weegar2012comparison}.

%A finales de la década de $1970$ e inicios de la década de $1980$,  la
%complejidad técnica de las computadoras limitaba la cantidad de herramientas
%disponibles, los programas eran desarrollados por profesores, y su objetivo era
%que los alumnos puedan poner en práctica lo aprendido en el aula. 

%\textbf{Aprendizaje como anexo:} el principal objetivo del desarrollo de un
%\emph{edutainment} es el de entretener, los objetivos pedagógicos son agregados
%al final. Adicionalmente, este aprendizaje se provee a través de largos textos
%que normalmente son omitidos.

\subsubsection{Ejemplos}
\label{sec:edutainment}

Entre los ejemplos de aplicación del conductismo en las \gls{tic} se encuentran:

\begin{itemize}

\item \textbf{Ejercicios de prueba y error}: son ejercicios en los cuales se
    presenta una pregunta y una lista de respuestas al alumno, y este debe
    responder correctamente. Si el alumno no responde correctamente, se repite
    la pregunta\cite{weegar2012comparison}.

\item \textbf{Edutainment}: son juegos sencillos que transmiten información
    simple al usuario, su estructura se basa en un objetivo claro que está
    separado de la experiencia educativa\cite{egenfeldt2007third}. El
    \emph{edutainment} pretende agregar entretenimiento a la educación, se ve al
    alumno como un receptor pasivo de información que debe asimilarla, y para
    aumentar la implicación de los alumnos, el entretenimiento es
    agregado\cite{resnick:2004}. Los \emph{edutainment} son el primer intento de
    unir el entretenimiento y la educación dentro de las
    \gls{tic}\cite{leinonen:ict}.

    Entre ejemplos de los \emph{edutainment}, podemos encontrar a:

    \begin{itemize}

    \item \textbf{Math Blaster}: (ver figura~\ref{fig:math_blaster}) es un
        \emph{edutainment} donde el alumno debe responder repetitivamente
        preguntas aritméticas para obtener municiones, luego con esas municiones
        debe completar diferentes misiones en una nave\cite{bruckman1999can}.
        Como todas las preguntas se responden mediante un mecanismo de selección
        múltiple, y no existe penalización por respuestas incorrectas, los
        alumnos no reflexionan sobre las respuestas elegidas, seleccionan una
        opción aleatoria y si no es la correcta, prueban otra, tras una cantidad
        finita de intentos, siempre se obtiene la recompensa deseada.

        \begin{figure}[ht!] 
        \centering 
        \includegraphics[scale=0.5]{tics/images/math_blaster.jpg}
        \caption{Math Blaster, \emph{edutainment} del año 1987}\label{fig:math_blaster} 
        \end{figure}

    \item \textbf{Donde en el mundo esta Carmen Sandiego} (ver
        figura~\ref{fig:carmen}) el objetivo del juego es detener a una serie de
        criminales mediante indicios que son proveídos en forma de texto. Este
        exitoso juego demuestra las falencias del \textit{Edutainment}, siendo
        visualmente muy atractivo, y con contenido multimedia acorde a su
        tiempo, no era más que \enquote{Prueba y Error}, cada nivel del juego
        podía ser completado sin leer la información
        proveída\cite{charsky:2010}.

        \begin{figure}[ht!] 
        \centering 
        \includegraphics[scale=0.5]{tics/images/carmen.jpg}
        \caption{Donde en el mundo esta Carmen Sandiego}\label{fig:carmen}
        \end{figure}

    \end{itemize}

    Las aplicaciones se limitaban a matemáticas, lenguaje y geografía, donde se
    podía evaluar inmediatamente los resultados proveídos por los alumnos, pues,
    normalmente era un enunciado y una lista posible de opciones del tipo
    \enquote{Prueba y Error}\cite{leinonen:ict}. 

\end{itemize}

\subsection{Ventajas y desventajas del modelo Push}

Las ventajas del modelo \textit{push} son:

\begin{itemize}

\item Permite controlar el entorno del aprendizaje utilizando un enfoque
    científico. Se controla el entorno de los alumnos, los estímulos que recibe
    y las consecuencias de estos estímulos. Se ignora a los pensamientos y
    experiencias previas de las personas\cite{weegar2012comparison}.

\item Permite educar a personas con desafíos de aprendizaje y
    comportamiento\cite{johnson2005instructionism}.

\item Permite la enseñanza de pensamiento de bajo nivel\footnote{El pensamiento
        de bajo nivel está relacionado con las prácticas educacionales que
        incluyen la capacidad de memorizar y procesar, al contrario, el
        pensamiento de alto nivel incluye la capacidad de crear y
        evaluar\cite{edwards2000higher}}. Son excelentes para enseñar matemática
    básica, geografía y lenguaje\cite{charsky:2010}.

\end{itemize}

Las desventajas que presentan las corrientes del tipo \textit{push} son:

\begin{itemize}

\item No tienen en cuenta la experiencia previa de los alumnos ni el factor
    social. Los medios utilizados para la educación se centran en el profesor y
    en la manera de dar la clase\cite{siemens2008learning,sawyer2005cambridge}.

    El comportamiento humano no puede ser reducido, no solo responde a los
    estímulos y al entorno, sino, además responde a la experiencia
    previa\cite{weegar2012comparison}.

\item No se enfoca en los alumnos, no todos los alumnos aprenden al mismo
    nivel\cite{johnson2005instructionism}.

\item Desde el aspecto pedagógico, los alumnos tienen problemas para comprender
    ideas diferentes a las encontradas en clase\cite{sawyer2005cambridge}.

\item Los alumnos tienden a memorizar contenido sin entender el contexto en el
    cual fue creado el conocimiento\cite{sawyer2005cambridge}. 
    
\item Los alumnos tratan a los hechos y procedimientos como conocimiento
    estático, que no puede ser transformado, y que siempre es
    verdad\cite{sawyer2005cambridge}.

\item Se centra en motivaciones externas, y deja de lado la motivación interna.
    Las motivaciones externas son las recompensas que reciben los alumnos al
    llevar a cabo un ejercicio de manera correcta. La motivación interna es de
    gran importancia\cite{weegar2012comparison}.
    
\item Prueba y error, las aplicaciones creadas utilizando estas corrientes
    permiten al alumno intentar varias veces sin ser penalizados, además de que
    los alumnos no están motivados, provocan que el alumno pruebe las opciones
    sin el proceso de reflexión necesario para aprender. Es decir, se enseñaba a
    probar opciones sin sentido antes que entender y analizar la
    experiencia\cite{charsky:2010,egenfeldt2007third,bruckman1999can}.

\end{itemize}

\section{Teorías del aprendizaje: Modelo Pull}

En el modelo \textit{pull}, los usuarios participan de forma activa en la
construcción del conocimiento. El rol de los profesionales de la educación es
guiar a los alumnos. Entre las corrientes pedagógicas podemos encontrar al
constructivismo y al construccionismo.

\subsection{Constructivismo}

El constructivismo es una corriente pedagógica creada por \textit{Jean Piaget} y
\textit{Lev Vygotsky}, cuya idea central es que el aprendizaje humano se
construye, que la mente de las personas elabora nuevos conocimientos a partir de
la base de enseñanzas anteriores\cite{martin2008modelo}. Predica que el
aprendizaje de los estudiantes debe ser activo, deben participar en actividades
en lugar de permanecer de manera pasiva observando lo que se les
explica\cite{hernandez:constructivismo,johnson2005instructionism}.

El constructivismo es un conjunto de prácticas que se enfocan en el alumno,
basados en el contenido, orientados al proceso, interactivos y que responden a
las necesidades e intereses personales de los
alumnos\cite{johnson2005instructionism}.

Se basa en que las personas no entienden, ni utilizan de manera inmediata la
información que se les proporciona, en cambio, el individuo construye su propio
conocimiento. El conocimiento se construye a través de la experiencia y esto
conduce a la creación de modelos mentales que se almacenan en la mente. Estos
esquemas van evolucionando, ampliándose y volviéndose más complejos a través de
dos procesos complementarios: la asimilación y el
alojamiento\cite{hernandez:constructivismo,johnson2005instructionism}.

Epistemológicamente el constructivismo es subjetivo, pues considera que el
conocimiento depende de las experiencias\cite{johnson2005instructionism}. 

Las aulas constructivistas crean un mundo realista donde lo más importante es el
aprendizaje, el profesor es un facilitator del aprendizaje del
alumno\cite{johnson2005instructionism,nanjappa2003constructing}.

\subsubsection{Ejemplos}

Entre los ejemplos del constructivismo podemos encontrar:

\begin{itemize}

\item \textbf{Redes sociales:} funcionan como una continuación del aula escolar,
    pero de carácter virtual, ampliando el espacio de interacción entre los
    estudiantes y el profesor, permitiendo el contacto continuo con los
    integrantes, y proporcionando nuevos materiales para la comunicación entre
    ellos. Esta tecnología presenta las características de interacción, elevados
    parámetros de calidad de imagen y sonidos, instantaneidad, interconexión y
    diversidad\cite{hernandez:constructivismo}. 

\item \textbf{Enciclopedias online o wikis:} genera un cambio drástico en la
    manera tradicional de obtener información para los temas impartidos en el
    aula; con las wikis los alumnos no sólo obtienen información, sino que ellos
    mismos pueden crearla. Los estudiantes pasan de ser simples observadores y
    trabajar de manera pasiva, a estar involucrados activamente en la
    construcción de su conocimiento, escuchando en clase, investigando fuera de
    ella  y después redactando artículos en la wiki que reflejen sus
    investigaciones, lo que han aprendido y la forma cómo lo han
    hecho\cite{hernandez:constructivismo}.

\item \textbf{Blogs:} ofrecen un espacio en el que los usuarios tienen la
    oportunidad de expresar sus ideas sobre cualquier tema que les interese. Los
    usuarios que acceden a los blogs pueden comentar sobre los escritos y dejar
    sus opiniones, consiguiendo un diálogo entre el propietario del blog y los
    que acceden a él\cite{hernandez:constructivismo}.
 
\end{itemize}

\subsection{Construccionismo}
\label{sec:tics_construccionismo}

El construccionismo es una corriente pedagógica que parte de una concepción del
aprendizaje según la cual la persona aprende por medio de su interacción
dinámica con el mundo físico, social y cultural en el que está
inmerso\cite{valdivia:sg}. El construccionismo nace en la década de $1980$ de la
mano de \textit{Seymour Papert}\cite{historia:2014,ict:ttc}.

Para el construccionismo, el conocimiento es construido por el estudiante en
lugar de ser trasmitido por el profesor\cite{moses:2003} y esto sucede
particularmente cuando el mismo se compromete en la elaboración de un producto o
artefacto que tenga un significado y pueda ser compartido\cite{valdivia:sg}. De
esta manera, se permite a los estudiantes elaborar sus propias interpretaciones
razonadas del mundo mediante la interacción con el mismo. El profesor actúa como
guía para el estudiante en la construcción de su conocimiento, aportando
conocimiento y experiencia. El construccionismo utiliza a las \gls{tic} como
medio cognitivo y no para la entrega de contenido\cite{sasha:construtivism}.

Según \textit{Papert}, los alumnos estarán mucho más involucrados en su
aprendizaje si construyen artefactos que los demás pueden ver, criticar y tal
vez utilizar. Y además, el alumno se enfrenta a problemas complejos con estas
construcciones, harán el esfuerzo por resolver problemas y aprender ya que la
construcción les motivará\cite{const:vs}.

\subsubsection{Ejemplos} 

Existen varios proyectos o iniciativas que incluyen al construccionismo como
base pedagógica, para la mayoría de ellos las computadoras son esenciales
mientras que para otros el mayor esfuerzo está en la incorporación de la
tecnología en su práctica educativa\cite{papertian:const}.

Algunos de estos proyectos son:

\begin{itemize}

\item \textbf{Lenguaje de programación \enquote{LOGO}}: El lenguaje
    \enquote{LOGO} es la cuna del construccionismo. Fundamentalmente consiste en
    presentar a los niños retos intelectuales que puedan ser resueltos mediante
    el desarrollo de programas en \enquote{LOGO}. El proceso de revisión manual
    de los errores contribuye a que el niño desarrolle habilidades
    metacognitivas al poner en práctica procesos de
    auto-corrección\cite{logo:sg,ict:ttc}.

    Las actividades de programación \enquote{LOGO} se realizan en las áreas de
    matemática, lenguaje, música, robótica, telecomunicaciones y ciencias.
    \enquote{LOGO} es accesible para principantes, especialmente niños pequeños,
    y es compatible con exploraciones complejas y proyectos sofisticados
    realizados por usuarios experimentados\cite{logo:sg}.
    
\item \textbf{\Gls{olpc}}: es una asociación sin ánimos de lucro cuyo esfuerzo
    se centra en dotar a los niños de una computadora duradera, accesible y
    potente en los países en desarrollo, se dice que es un descendiente directo
    del construccionismo\cite{papertian:const}.
	
    Surgió dentro del \gls{mit}, \Gls{olpc} propone un cambio de paradigma
    basado en un modelo de aprendizaje en el que cada alumno disponga de su
    propia computadora portátil y que se pueda conectar a Internet, de forma
    totalmente gratuita, desde su escuela. A partir de esta política se pretende
    disminuir la brecha tecnológica y de acceso a la información en países más
    desfavorecidos en comparación con los países del primer
    mundo\cite{videojuegos:gonzaleztardon}.
    
    \textit{Sugar} es el principal factor construccionista de \Gls{olpc}, consiste en una 
    interfaz que captura gráficamente el mundo de los estudiantes y maestros, 
    haciendo énfasis en las conexiones dentro de la comunidad, entre las personas y 
    sus actividades\cite{olpc:sugar}.

	
\item \textbf{Fabricación personal}: \textit{Neil Gershenfeld}, colega de
    \textit{Papert} en el Media Lab del \Gls{mit} dictó un curso titulado
    \emph{Cómo hacer casi cualquier cosa}. La idea se centraba en la creación de
    la tecnología que se necesita para resolver los problemas que se poseen.
    Esta auto-confianza, la autonomía personal y la agencia sobre la tecnología
    han estado en el centro de trabajo de \textit{Papert} durante años.
    \textit{Papert} no sólo defendió la idea de que los niños posean
    computadoras personales, sino también que a la larga ellos debían
    mantenerlas, repararlas e incluso construirlas.

    Junto con la capacidad para utilizar la tecnología para inventar soluciones
    a los problemas de significado personal, los estudiantes no sólo tienen
    acceso a la información, sino que tienen una mayor capacidad para darle
    forma a su mundo. La fabricación personal promueve la visión de
    \textit{Papert} \emph{Si se puede utilizar la tecnología para hacer las
        cosas, usted puede hacer las cosas mucho más interesantes y usted puede
        aprender mucho más haciéndolo}\cite{papertian:const}.

\end{itemize}

\subsection{Ventajas y desventajas del modelo Pull}

Las principales ventajas de los modelos \emph{pull} son:

\begin{itemize}

\item La enseñanza es más eficaz que la enseñanza tradicional en términos de
    logros académicos de los estudiantes\cite{kim2005effects}.

\item Tiene un efecto positivo sobre la motivación para aprender las tareas
    académicas causando ansiedad en el proceso de
    aprendizaje\cite{kim2005effects}.

%\item El estudiante asume un papel activo en la construcción del
%    conocimiento\cite{hernandez:constructivismo,johnson2005instructionism}.

\item Permite desarrollar el pensamiento de alto nivel lo que involucra la
    resolución de problemas y el pensamiento
    crítico\cite{wilson2012constructivism}.

\item Se basa en la totalidad de la persona y conduce a representaciones más
    realistas de la experiencia\cite{wilson2012constructivism}.

\end{itemize}

Las principales desventajas de los modelos \emph{pull} son:

\begin{itemize}

\item Las técnicas tradicionales de evaluación no se ajustan a estas teorías.
    Tanto el constructivismo como el construccionismo predican que el estudiante
    es el más apropiado para evaluar su conocimiento, por lo que se deben
    desarrollar nuevas técnicas de evaluación\cite{lai1995implications}.

\item No pueden ser aplicadas a todos los contextos del
    aprendizaje\cite{lai1995implications}, especialmente en los aspectos que
    involucran pensamiento de bajo nivel.

\end{itemize}

\section[Ventajas y desafíos]{Ventajas y desafíos del uso de las TIC en la educación}
\label{sec:tics_ventajas}

Las principales ventajas de la utilización de las \Gls{tic} en la educación son:

\begin{itemize}

\item \textbf{Nuevos modelos pedagógicos:} las nuevas corrientes pedagógicas
    enfatizan el proceso de como adquirir conocimiento (aprendizaje) y no
    solamente como transmitir el conocimiento
    (enseñanza)\cite{guenaga2013serious}.

\item \textbf{Eliminación de distancias:} con la aparición de las computadoras y
    los satélites,  el mundo se ha convertido en una aldea global, y las
    distancias en cuestiones de transmisión de información se han vuelto
    insignificantes\cite{mohammed2013information},  los medios tradicionales
    como bibliotecas, o escuelas están limitados a un espacio  físico, con el
    uso de las \Gls{tic}, esta restricción física
    desaparece\cite{tinio:ict,punie:ict}.

\item \textbf{Colaboración distribuida:} como consecuencia del punto anterior,
    los alumnos pueden colaborar de manera más sencilla pues no tienen
    limitaciones físicas. Las \gls{tic} permiten consultar con expertos o tener
    mentores a través de Internet. Permiten además la colaboración entre
    estudiantes con intereses comunes, mediante foros y redes
    sociales\cite{unesco:ict}.

\item \textbf{Motivación para aprender:} las \Gls{tic} tienen un impacto
    positivo en el proceso de aprendizaje especialmente en lo referente al
    compromiso
    con\cite{passey2004motivational,egenfeldt2007third,martin2008modelo}:
	    
    \begin{itemize}
    \item \textbf{La actividad}: a través de estímulos visuales, auditivos, etc.
    \item \textbf{La capacidad de investigación}: es más fácil acceder a gran cantidad de
        información bibliográfica.
    \item \textbf{La capacidad de escritura y lectura}: permitiendo compartir  ideas de
        manera más legible y mejorarlas iterativamente.
    \item \textbf{La capacidad de presentación}: es más fácil presentar trabajos
        profesionalmente a un público mayor.
    \end{itemize}
	    
\item \textbf{Adquisición de habilidades básicas:} las habilidades necesarias para
    utilizar de manera efectiva las \Gls{tic} se están convirtiendo en una
    necesidad básica, un aprendizaje guiado por las mismas puede ayudar a una
    rápida asimilación de los conceptos relacionados\cite{martin2008modelo}.

\end{itemize}

Durante la historia de las \Gls{tic} en la educación, se han encontrado
diferentes dificultades a la hora de aplicar los nuevos conceptos en la
educación, los principales desafíos son:

\begin{itemize}

\item \textbf{Falta de motivación de los profesionales}: desde los primeros
    enfoques que carecían de bases pedagógicas válidas hasta la
    actualidad\cite{punie:ict,ict:romeo}.

\item \textbf{Brecha social}: la brecha social existente implica otro riesgo
    para la utilización de las \Gls{tic} en la educación, aquellos que no posean
    los recursos económicos necesarios para acceder a la misma no se verán
    beneficiados por las \Gls{tic}\cite{punie:ict}.

\item \textbf{Altas expectativas}: las \Gls{tic} han tenido un impacto positivo
    en la educación, pero el mismo no es el esperado\cite{punie:ict}, por
    ejemplo, iniciativas como el \emph{edutainment} que prometían ser la
    solución a los problemas educacionales no cumplieron las expectativas. 

\item \textbf{Aspectos financieros}: uno de los desafíos más importantes que
    enfrentan las \Gls{tic} para convertirse en una alternativa viable es la
    inversión en infraestructura necesaria\cite{unesco:ict}.

\end{itemize}



% TIC's en la educación
%% Breve reseña histórica
%% Push
%%% Instruccionismo
%%%% Ventajas y desventajas
%%%% Ejemplos (e-learning)
%%% Conductismo
%%%% Ventajas y desventajas
%%%% Ejemplos (edutainment)
%% Pull
%%% Constructivismo
%%%% Ventajas y desventajas
%%%% Ejemplos (wikis, redes sociales, blogs, e-learing, simulaciones educativas)
%%% Construccionismo
%%%% Ventajas y desventajas
%%%% Ejemplos (logo, olpc)
%% Ventajas y desventajas
