\chapter{TIC's en la educación}

Las \Gls{tic} son un conjunto de herramientas tecnológicas y recursos utilizados
para comunica, crear, diseminar, almacenar y manejar la
información\cite{unesco:ict}. Estas tecnologías abarcan computadoras personales,
internet, radio, televisión y telefonía\cite{tinio:ict}.

Las \Gls{tic} fueron utilizadas como complemento a la educación desde los
inicias de la misma con la radio y la televisión. Fueron vistas como un
complemento a las herramientas utilizadas en clase, como complemento del libro,
o como una herramienta que elimina la distancia física entre el profesor y el
alumno\cite{unesco:ict}. 

Las expectativas iniciales acerca del impacto de las \Gls{tic} en la educación
fueron ampliamente superiores a los resultados obtenidos\cite{unesco:ict}, con
el advenimiento de las computadoras esta brecha se redujo, en mayor medida por
que se las utilizo en conjunto con tecnologías como Internet, y los efectos
positivos en la educación fueron aumentando gradualmente\cite{unesco:ict}.

Las principales ventajas de la utilización de las \Gls{tic} en la educación es
su aplicabilidad en áreas que no pueden ser cubiertas por otras alternativas,
como son:

\begin{description}

    \item[Nuevos modelos pedagógicos] teorías como el constructivismo moderno
	    enfatizan el proceso de como adquirir conocimiento y no solamente el
	    conocimiento en sí.

    \item[Eliminación de distancias] Con el advenimiento de las computadoras y los
	    satélites, el mundo se ha convertido en una aldea global, y las
	    distancias en cuestiones de transmisión de información se han vuelto
	    insignificantes\cite{mohammed2013information}, los medios
	    tradicionales como bibliotecas, o escuelas están limitados a un
	    espacio físico, con el uso de las \Gls{tic}, esta restricción
	    física desaparece\cite{tinio:ict}.

    \item[Colaboración distribuida] como consecuencia del punto anterior, los
	    alumnos pueden colaborar de manera más sencilla pues no tienen
	    limitaciones físicas. Además, los alumnos pueden consultar con
	    expertos que están en linea, e incluso tener mentores en linea,
	    estas tutorías pueden ser uno a uno, por ejemplo mediante
	    comunicaciones por correo electrónico. Además permite la
	    colaboración masiva entre estudiantes de intereses comunes, mediante
	    foros y redes sociales\cite{unesco:ict}.

    \item[Motivación para aprender] Las \Gls{tic} tienen un impacto positivo en
	    el proceso de aprendizaje especialmente en lo referente al
	    compromiso con la actividad (a través de estímulos visuales,
	    auditivos, etc), capacidad de investigación (es más fácil acceder a
	    gran cantidad de información bibliográfica), capacidad de escritura
	    y lectura (permitiendo compartir ideas de manera más legible y
	    mejorarlas iterativamente) y capacidad de presentación (es más fácil
	    presentar trabajos profesionalmente a un público
	    mayor)\cite{passey2004motivational}\cite{egenfeldt2007third}.

    \item[Adquisición de habilidades básicas] las habilidades necesarias para
	    utilizar de manera efectiva las \Gls{tic} se están convirtiendo en
	    una necesidad básica, un aprendizaje guiado por las mismas puede
	    ayudar a una rápida asimilación de los conceptos relacionados.

\end{description}

Uno de los desafíos más importantes que enfrentan las \Gls{tic} para convertirse
en una alternativa viable es la inversión en infraestructura
necesaria\cite{unesco:ict}. 


\section{Historia de las TIC en la educación}

La historia de las \Gls{tic} en educación comienza con la Universidad Abierta
del Reino Unido\footnote{Open University of United Kingdom} que en 1969 se
establece como la primera institución educativa dedicada a la enseñanza a
distancia utilizando las, para aquel entonces, nuevas
tecnologías\cite{tinio:ict}.

En 1973 Vint Cerf creo el protocolo TCP/IP y es considerado el nacimiento de
Internet\cite{white:ict}, lo que permitió que la información pueda ser
transmitida de manera más sencilla, tiempo después con la aparición de las
computadoras personales en 1977\cite{white:ict}. 

Otro hito tecnológico se dio en la \Gls{cern} en el año 1989 cuando se concibió
lo que hoy se conoce como \emph{World Wide Web}, permitiendo que los usuarios de
la \emph{Web} puedan compartir archivos mediante un protocolo
estándar\cite{white:ict}. 

Con las principales eventos que marcaron la evolución tecnológica de las
\Gls{tic} en la educación, se divide su historia en cinco partes. Las mismas se
pueden dividir en dos secciones, las primeras tres corresponden a los comienzos
y donde los alumnos eran receptores de información, época denominada, y la
segunda denominada\emph{push}\cite{white:ict} que es aquella donde los alumnos
participan de su educación y son creadores activos de conocimiento.

\subsection{Programación, ejercicios y prácticas}

Este periodo que abarca desde la aparición de las primeras computadoras
personales hasta el final de la década de 1980, este periodo se caracterizo por
computadoras muy limitadas, nula interacción multimedia y escasez de programas
especializados. Se enseñaba programación básica\cite{leinonen:ict}, no por la
necesidad de educar programadores, sino por la creencia de que así se
desarrollarían habilidades matemáticas y lógicas en los alumnos. Los programas
eran muy simples y se basaban en matemáticas y nociones básicas del idioma. 

Esta clase de ejercicios no ayudaron a los alumnos a obtener un aprendizaje
profundo, pues era fácil resolverlos a través de la prueba y el error, la mayor
parte del tiempo servían para distraer a los alumnos no interesados en la
programación mientras el profesor enseñaba programación a aquellos que parecían
interesados\cite{leinonen:ict}.


\subsection{Entrenamiento basado en computadoras}

Cuando aparecieron en el mercado computadoras con multimedia, se argumento que
los ejercicios de la era anterior fallaron en su objetivo de una educación
profunda por que no contenían multimedia\cite{leinonen:ict}, las aplicaciones
eran distribuidas por CD-ROM, y así se actualizaban de manera más frecuente, y
podían contener gran cantidad de contenido multimedia.

En este periodo se desarrollaron una gran cantidad de aplicaciones educativas
que más tarde serían conocidas como \emph{Edutainment}\footnote{Education +
	Enteirtainment, se traduce como educación entretenida}, estas pretendían
agregarle entretenimiento a la educación, se veía al como un receptor pasivo de
información que debía asimilarla, y para aumentar el compromiso, el
entretenimiento era agregado\cite{resnick:2004}.

Las bases pedagógicas de esta se basada en la capacidad de ciertos estudiantes
de aprender mejor cuando interactúan con contenido multimedia, la \emph{prueba y
	error} aún estaban presentes, pero no eran presentados inmediatamente,
sino más bien una vez que el alumno ya debería haber asimilado los conceptos y
funcionaban como pruebas de adquisición de conocimiento. Este tipo de contenido
tampoco logro la enseñanza profunda, solamente fueron efectivos en el
aprendizaje de idiomas, fallando en todos los demás campos\cite{leinonen:ict},
además los contenidos muchas veces estaban desactualizados y obtener nuevas
versiones no era una tarea sencilla.

Varios gobiernos apoyaron de manera agresiva la introducción de las \Gls{tic} en
educación\cite{mcdougall2006theory} y se realizo un importante avance teórico
con los trabajos sobre el aprendizaje construccionista de Papert y Harel (1991),
y la influencia de las computadoras sobre el aprendizaje y la mente de Marvin
Minksy (1987)~\cite{mcdougall2006theory}.

A comienzos de la decada de 1990, con la popularización de Internet, se le vio
como solución al problema de las poco frecuentes actualizaciones de aplicaciones
educativas, su utilización no tenia bases pedagógicas, más bien se basaban en la
facilidad de distribuir contenido por la \emph{Web}, el principal inconveniente
era la velocidad del Internet, no era suficiente para proveer entornos ricos en
multimedia como lo hacían los CD-ROM\cite{leinonen:ict}.

\subsection{e-Learning}

La bases pedagógicas de esta son similares a la era del entrenamiento basado en
computadoras, se distribuye contenido masivamente a los alumnos, y luego, de
manera muy discreta se permite a los mismos colaborar, dejando siempre en claro
que primero se debe asimilar toda la información posible y luego relacionarse
con los demás\cite{leinonen:ict}.

El \emph{e-Learning} apareció a finales de la década de 1990 y tubo su apogeo en
mediados de la década del 2000, apoyada por la gran penetración de las \Gls{tic}
en la población\cite{punie:ict}.

Todos los paradigmas anteriores viven dentro del \emph{e-Learning}, permitiendo
compartir contenido multimedia y realizar pruebas del tipo \emph{prueba-error}. 

Si bien en las anteriores épocas, el uso de las \Gls{tic} estaba más orientado
hacia la educación básica y secundaría, el \emph{e-Learning} actualmente es más
utilizado en la educación terciaría\cite{punie:ict}.

La utilización del \emph{e-Learning} tiene varios grados de aplicación en
entornos reales\cite{punie:ict}, que van desde ser simples elementos
complementarios a la clase, como por ejemplo un repositorio para las
diapositivas y otros materiales de clase, hasta cursos completamente en linea,
donde la clase ha sido completamente sustituída.


%\input{tics/arte}
\section{Problemas actuales}
\label{tics:problemas}

Durante la historia de las \Gls{tic} en la educación, se han encontrado
diferentes dificultades a la hora de aplicar los nuevos conceptos en la
educación, desde los primeros enfoques que carecían de bases pedagógicas válidas
hasta la actualidad.

El principal problema es falta de motivación de los profesionales de la
educación para emplear las \Gls{tic}\cite{punie:ict}\cite{ict:romeo}.

El contenido proveído actualmente puede ser considerado como un conjunto de
buenas prácticas\cite{punie:ict} y así, omiten completa o parcialmente el
contexto donde esa buena práctica fue generado.

A la hora de educar a los educadores en la utilización de las \Gls{tic} no se
debe medir medidas cuantitavias, como las notas o el número de cursos, sino más
bien el impacto cualitativo de la educación

Las \Gls{tic} han tenido un impacto positivo en la educación\cite{punie:ict},
pero no han obtenido el impacto esperado.

Iniciativas como el \emph{Edutainment}, prometían ser la solución a los
problemas educacionales, sin embargo, su implementación no cumplio con las
expectativas, obtuvieron una reputación negativa, y hoy en día son considerados
como el peor tipo de educación, pues son un ejercicio de \emph{prueba-error}
ocultos bajo un juego poco entretenido\cite{resnick:2004}. La principal critica
contra los \emph{Edutainment} es su incapacidad de enseñar como aplicar
conceptos aprendidos a un entorno real\cite{resnick:2004}.

Mientras que la utilización de las \Gls{tic} puede eliminar problemas actuales
como el aislamiento y la falta de pensamiento de alto nivel\cite{punie:ict}, la
brecha social existente implica otro riesgo para la utilización de las \Gls{tic}
en la educación, aquellos que no posean los recursos económicos necesarios para
acceder a la misma no se verán beneficiados por las \Gls{tic}\cite{punie:ict}.


 

\section{Construccionismo y las TIC's}
\label{sec:tics_CONSTRUCCIONISMO}

El construccionismo es una corriente pedagógica que parte de una concepción del
aprendizaje según la cual la persona aprende por medio de su interacción
dinámica con el mundo físico, social y cultural en el que está
inmerso\cite{valdivia:sg}.

Posee un enfoque diferente en cuanto al uso de las \Gls{tic} en la educación.
Esta pedagogía se diferencia de la educación tradicional en que el estudiante ya
no es un receptor pasivo de información, en cambio, el mismo participa
activamente del proceso de aprendizaje construyendo su propio conocimiento. Se
diferencia del instruccionismo en que el construccionismo utiliza la tecnología
como medio cognitivo y no para la entrega de contenido.

Se considera al construccionismo como una alternativa prometedora a la educación
tradicional. Desde el punto de vista tecnológico, es ideal pues el mismo
requiere un alto dinamismo en el traspaso del conocimiento
\cite{sasha:construtivism}. 

El construccionismo y las \Gls{tic} siempre han estado relacionados, ya que el
mismo se originó con un lenguaje de programación (LOGO)\cite{ict:ttc}. Un
característica importante de esta relación es que tienen la capacidad de
eliminar los problemas de distancia\cite{mariluz:seiousgames}.


\subsection{Historia}

A mediados de la década de 1960 Seymour Papert, un matemático sudafricano, llegó
a los Estados Unidos, donde fue co-fundador del Laboratorio de Inteligencia
Artificial del \Gls{mit} con Marvin Minsky\cite{logo:sg}. 


En la década de $1980$, Seymour Papert acuñó el término construccionismo en la
Fundación Nacional de Ciencia de los Estados Unidos titulada
\enquote{Constructionism: A New Opportunity for Elementary Science Education}
para presentar un método pedagógico que se basaba en muchas de las ideas de la
educación progresiva estudiadas por el estadounidense John Dewey en el inicio
del siglo 20 en su escuela experimental en la Universidad de Chicago. Dewey
quería poner gran parte de la responsabilidad de aprender en los estudiantes que
han nacido con el don de aprendizaje y la creación de conocimiento en sus
propios términos\cite{historia:2014}.

Papert también fue influenciado por Maria Montessori quien luego de grados en
pediatría, medicina, psicología y filosofía comenzó su propia escuela
experimental para niños pequeños. En lugar de que ella misma estableciera
formalmente las tareas, vio como sus estudiantes actuaron por su cuenta. Ella
siguió los intereses de los estudiantes y observó como respondieron a sus
entornos especialmente preparados. Ella preparó el escenario pero no ofreció una
guía explícita\cite{historia:2014}.

Papert trabajó directamente con el psicólogo suizo Jean Piaget a quien Ernst von
Glasersfeld llamó \enquote{El pionero de la teoría constructivista del
    aprendizaje}. 
% esto de glaserfeld pio importa?
Piaget, al igual que Dewey, Montessori y otros, desarrolló su teoría de la
educación y la construcción del conocimiento observando e interactuando con los
niños. De estas observaciones nació el movimiento constructivista. El
contructivismo se basa en que el conocimiento debe ser construído por el
estudiante y los nuevos significados deben ser obtenidos relacionándolos a
significados anteriores en el propio sistema de relaciones del
estudiante\cite{historia:2014}. 

Papert admitió que jugó más con la palabra construcción. El contruccionismo de
Papert se diferencia del constructivismo de Piaget en que los estudiantes
construyen las ideas o partes del mundo utilizando herramientas. Para Papert los
estudiantes necesitan contruir modelos de partes de su mundo con el fin de
comprender más plenamente el significado, el 
% la definición de modelos podemos poner un footnote?
contenido y la dinámica de las partes. La elaboración de representaciones
mentales mediante la construcción y el intercambio es la metáfora del marco
construccionista\cite{historia:2014}.

%Durante $1980$, Seymor Papert, Wally Feurzeig, Marvin Minsky y John McCarthy y los
%miembros del Departamento de Inteligencia Artificial del \Gls{mit} y una
%compañía de tecnología en Cambridge, Massachusetts, desarrollaron un nuevo
%lenguaje de programación llamado LOGO que tenía por objeto que los estudiantes
%construyeran sus modelos mentales en notación LOGO. Este juego
%introduciría de forma natural las ideas de los procedimientos, funciones,
%variables, recursividad, la modularidad, simulación, verificación, entre %otros\cite{historia:2014}.

Papert trabajó con el equipo de Bolt, Beranek y Newman, liderado por Wallace
Feurzeig, que creó la primera versión del lenguaje de programación LOGO en 1967.
LOGO es un dialecto de Lisp, fue diseñado como una herramienta para el
aprendizaje. Sus características, como la modularidad, extensibilidad,
interactividad y flexibilidad, derivan de este objetivo \cite{logo:sg}.

Los desarrolladores de LOGO no solo alentaron la promoción de formas
construccionistas de enseñanza y aprendizaje sino también alentaron otra forma
de aprendizaje nueva y no tradicional con las diferentes herramientas
tecnológicas\cite{historia:2014}. 

Por lo tanto, se puede decir que la creación de LOGO permitió la creación del
construccionismo\cite{historia:2014}.

    
    
% ESTE
% Cuando se produjo LOGO y se acuñó el construccionismo, la comunidad del 
% construccionismo era en su mayoría ingenieros informáticos y matemáticos\cite{historia:2014}.
% Esto es relevante?

\observacion{Toda la sección hay que reordernar}

\subsection{Bases Pedagógicas}

Para el construccionismo, el conocimiento es construido por el estudiante en
lugar de ser trasmitido por el profesor \cite{moses:2003} y esto sucede particularmente cuando
el mismo se compromete en la elaboración de un producto o artefacto que tenga un
significado y pueda ser compartido\cite{valdivia:sg}. De esta manera, se permite
a los estudiantes elaborar sus propias interpretaciones razonadas del mundo
mediante la interacción con el mismo. El profesor actúa como guía para el estudiante 
en la construcción de su conocimiento, aportando conocimiento y experiencia.

Según Papert, los alumnos estarán mucho más involucrados en su aprendizaje si
construyen artefactos que los demás pueden ver, criticar y tal vez utilizar. Y
además, el alumno se enfrenta a problemas complejos con estas construcciones,
harán el esfuerzo por resolver problemas y aprender ya que la construcción les
motivará\cite{const:vs}.

El enfoque construccionista establece que los seres humanos conocen y aprenden
de formas diferentes por lo tanto, no se puede elaborar una jerarquía de estilos
de aprendizajes\cite{valdivia:sg}.

\subsection{Actualidad}

%El construccionismo pone énfasis en el \emph{Aprender haciendo}, esta idea
%\fixme{mejora}{ref} la práctica educativa tradicional o instruccionismo. %El instruccionismo
%se basa en el concepto de que existe un profesor y un estudiante, el profesor
%transfiere el conocimiento que ha adquirido a un alumno que es receptor pasivo
%de información de esta manera, se enfoca más en la capacidad del profesor. 

Existen varios proyectos o emprendimientos que incluyen al contruccionismo como
base pedagógica, para la mayoría de ellos las computadoras son esenciales
mientras que para otros el mayor esfuerzo está en la incorporación de la
tecnología en su práctica educativa\cite{papertian:const}.

Algunos de estos emprendimientos son:

\begin{itemize}

\item \textbf{Lenguaje de programación LOGO}: El lenguaje LOGO es la cuna del construccionismo. %, se basa en el
%	principio de que se aprende mejor haciendo, pero se aprende todavía
%	mejor si se combina la acción con la verbalización  y la reflexión
%	acerca de lo que se ha hecho. 
	Fundamentalmente consiste en presentar a
	los niños retos intelectuales que puedan ser resueltos mediante el
	desarrollo de programas en LOGO. El proceso de revisión manual de los
	errores contribuye a que el niño desarrolle habilidades metacognitivas
	al poner en práctica procesos de auto-corrección\cite{logo:sg}.
	%http://es.wikipedia.org/wiki/Logo_(lenguaje_de_programaci%C3%B3n)


    %A mediados de la década de 1960 Seymour
	%Papert, que había estado trabajando con Piaget en Ginebra, llegó a
	%Estados Unidos donde co-fundó el Laboratorio de Inteligencia Artifical
	%del MIT con Marvin Minsky. Papert trabajó con el equipo de Bolt, Beranek
	%y Newman, liderado por Wallace Feurzeig, que creó la primera versión del
	%logotipo en 1967. A lo largo de la década de 1970 Logo fuen incubado en
	%el MIT y algunos otros sitios de investigación. El lenguaje de
	%programación Logo, un dialecto de Lisp, fue diseñado como una
	%herramienta para el aprendizaje. Sus características como la
	%modularidad, extensibilidad, interactividad y flexibilidad se derivan de
	%este objetivo. 
	%http://el.media.mit.edu/logo-foundation/logo/index.html

    Las actividades de programación LOGO se realizan en las áreas de matemática, lenguaje, 
    música, robótica, telecomunicaciones y ciencias. LOGO es accesible para novatos, 
    incluyendo niños pequeños, y también es compatible con exploraciones complejas y 
    proyectos sofisticados realizados por usuarios experimentados\cite{logo:sg}.
    
    Uno de los ambientes más populares de LOGO incluye a la tortuga, originalmente 
    era un robot que era puesto en el suelo y se podía dirigir su movimiento 
    escribiendo comandos en el ordenador. Pronto la tortuga emigró a las 
    pantallas gráficas de los ordenadores en donde se las utiliza para dibujar 
    formas y diseños\cite{logo:sg}.

    

%\item[Simulación] La simulación en el ámbito de la educación fue evolucionando
%desde simples motores de reglas hasta complejos entornos, la simulación
%demostró ser una herramienta muy útil el ámbito laboral
%\cite{mariluz:seiousgames}, pues enseña al alumno a encarar situaciones muy
%difíciles de representar en entornos completamente controlados y provee
%mecanismos para comprobar la efectividad de la herramienta. 

%Actualmente la simulación se utiliza más en el ámbito empresarial pues las
%empresas son las más necesitas de innovar en el ámbito de la enseñanza. Un
%ejemplo de esta necesidad se da, por ejemplo, en el entrenamiento de nuevos
%vendedores, es muy difícil enseñar a un vendedor como debe vender los productos
%con un pizzarón y/o una presentación, en cambio la simulación permite que el
%mismo pueda probar cosas nuevas y experiencias de sus compañeros (o
%instructor), convirtiendo así el aprendizaje en
%colectivo\cite{mariluz:seiousgames}. En el ámbito académico la simulación mas
%utilizada en campos físicos (como simulación de fluidos), meteorología
%(simulación de tormentas y fenómenos climáticos), etc. 

%\item[Serious Games] Diseñado con el propósito de aprender. Generalmente hace
%uso de la simulación para permitir un aprendizaje más realista.

%\item[Lego Serious Play] Es una iniciativa de Lego que busca fomentar el
%pensamiento creativo por medio de la construcción por parte de los estudiantes
%de su identidad y experiencias utilizando legos. 

\item \textbf{\Gls{olpc}}: es una asociación sin ánimos de lucro cuyo esfuerzo
    se centra en dotar a los niños de una computadora duradera, accesible y
    potente en los países en desarrollo, se dice que es un descendiente directo
    del construccionismo\cite{papertian:const}.
	
    Surgió dentro del MIT Media Lab, \Gls{olpc} propone un cambio de paradigma
    basado en un modelo de aprendizaje en el que cada alumno disponga de su
    propia computadora portátil y que se pueda conectar a internet, de forma
    totalmente gratuita, desde su escuela. A partir de esta política se pretende
    disminuir la brecha tecnológica y de acceso de información en países más
    desfavorecidos en comparación con los países del primer
    mundo\cite{videojuegos:gonzaleztardon}.
	
	%Con esto se busca que
	%la computadora personal sea utilizada como un laboratorio intelectual y
	%un vehículo para la auto-expresión. OLPC no tiene que ver con la
	%escolarización o la escuela, más bien las utiliza como medio de
	%distribución de las computadoras a los niños, los cuales pueden
	%utilizarlas para aprender en cualquier lugar y momento. Se busca
	%fomentar el aprendizaje natural, es decir, aquel aprendizaje sin
	%enseñanza.

    %\fixme{Los problemas atribuidos al experimento}{experimento?} OLPC son
    %predominantemente las críticas a la política, el liderazgo o de la
    %intransigencia de la escuela en vez del construccionismo o computadora
    %personal para los niños pobres. El experimento audaz de Nicholas Negroponte
    %(co-fundador de \Gls{olpc}) y Sugata Mitra para dejar las computadoras desde
    %un helicóptero sobre una aldea de África se basa en la creencia en el
    %construccionismo\cite{papertian:const}.

\item \textbf{Fabricación personal}: Neil Gershenfeld, colega de Papert en el
    Media Lab del \Gls{mit} dictó un curso titulado \emph{Cómo hacer casi
        cualquier cosa}. La idea se centraba en la creación de  la tecnología
    que se necesita para resolver los problemas que se poseen. Esta
    auto-confianza, la autonomía personal y la agencia sobre la tecnología han
    estado en el centro de trabajo de Papert durante años. Papert no sólo
    defendió la idea de que los niños posean computadoras personales, sino
    también que a la larga ellos debían mantenerlas, repararlas e incluso
    construirlas.

	Junto con la capacidad para utilizar la tecnología para inventar
	soluciones a los problemas de significado personal, los estudiantes no
	sólo tienen acceso a la información, sino que tienen una mayor capacidad
	para darle forma a su mundo. La fabricación personal promueve la visión
	de Papert \emph{Si se puede utilizar la tecnología para hacer las cosas,
	usted puede hacer las cosas mucho más interesantes y usted
	puede aprender mucho más haciéndolo}\cite{papertian:const}.

\end{itemize}

%http://constructingmodernknowledge.com/cmk08/wp-content/uploads/2012/10/StagerConstructionism2012.pdf


