\section{Push}


La primera manifestación de las \gls{tic} en la educación se da como un
sustituto a los medios tradicionales, el libro impreso por ejemplo. La creación
de la información es responsabilidad de los profesores, imprentas y
academias\cite{leinonen:ict,white:ict}.

El modelo \emph{push} favorece a dos corrientes pedagógicas, al instruccionismo
y al conductismo. El instruccionismo utiliza a las \gls{tic} para resolver
problems de distancia y de
costo\cite{igi:instructionism,johnson2005instructionism}, mientras que el
conductismo utiliza a las \gls{tic} como un medio para proveer refuerzos
positivos\cite{weegar2012comparison}.

\subsection{Instruccionismo}

La educación tradicional o instruccionismo se basa en la transferencia de
conocimiento del profesor al alumno, se enfoca más en el profesor, en la
capacidad del mismo, y en el producto final como resultado de un proceso no
interactivo y bien
documentado\cite{igi:instructionism,johnson2005instructionism}. Los mecanismos
tradicionales para probar la efectividad de este tipo de enseñanza son los
exámenes escritos.

El instruccionismo es conocido además como enseñanza sistemática, enseñanza
explícita, enseñanza directa, y enseñanza activa, siempre enfatizando al
profesor\cite{johnson2005instructionism}.

Epistemológicamente se puede observar al instruccionismo como objetivo, pues
considera que el conocimiento es independiente del entorno, se asume que el
mismo es isomorfo, si el profesor puede enseñar, el alumno puede
aprender\cite{johnson2005instructionism}.

En el instruccionismo, la utilización de las \Gls{tic} en la actualidad se
centra principalmente en mecanismos para proveer contenido, se utilizan
plataformas complejas que permiten a los profesores distribuir contenido y otras
actividades relacionadas, estas plataformas se centran bajo el nombre de
\emph{E-Learning}.




\subsubsection{E-Learing} 
\observacion{No se entiende bien la relación entre las corriente pull/push, y
    las corrientes específicas (respecto a su grafo 2.1)}
\observacion{E-Learing, es pull o push. No se refleja bien la relación}

El \emph{E-Learning} se define como la educación y capacitación a través de
medios digitales, incluye todo tipo de medio capaz de distribuir información,
puede ser de en tiempo real como salas de conversaciones y videoconferencias o
puede ser diferido, como por ejemplo foros, enciclopedias. Es particularmente
útil para educación a distancia y con horarios flexibles. Se originó a finales
de la década de $1990$ y tuvo su apogeo a mediados de la década del $2000$,
apoyada por la gran penetración de las \Gls{tic} en la
población\cite{punie:ict}.

Se distribuye contenido masivamente a los alumnos, y luego, de manera discreta
se permite a los mismos colaborar, dejando siempre en claro que primero se debe
asimilar toda la información posible y luego relacionarse con los
demás\cite{leinonen:ict}.


\begin{figure}[h] 
\centering 
\includegraphics[scale=0.5]{tics/images/moodle.jpg}
\caption{Moodle, plataforma de e Learning} 
\label{fig:moodle}
\end{figure}


La plataforma \emph{Moodle} (ver~\ref{fig:moodle}) cuya primera versión salió en
el $2002$, es una de las principales herramientas del \emph{e-Learning} hoy en
día, permite la creación de cursos específicos por materia y sitios
especializados por instituciones académicas\cite{perkins2006using}. 

La utilización del \emph{e-Learning} tiene varios grados de aplicación en
entornos reales\cite{punie:ict}, que van desde ser simples elementos
complementarios a la clase, como por ejemplo un repositorio para las
diapositivas y otros materiales de clase, hasta cursos completamente en línea,
donde la clase ha sido completamente sustituida.

Sí bien, el \emph{e-Learning} permite la distribución y colaboración en
distintos niveles, no hace un enfoque en el aspecto pedagógico, y se centra en
la forma de transmitir información, y no como la recibe el
alumno\cite{leinonen:ict}, tampoco se centra en la reacción del alumno ante la
información recibida, lo que sí es estudiado por el
conductismo\cite{weegar2012comparison}.

\subsubsection{Ventajas y desafíos}

Las ventajas que presenta la utilización del instruccionismo son:

\subsection{Conductismo}
\observacion{?}
\observacion{Resumir más, balancer las descripciones}
% Mencionar lo de los marcos (ver martin2008modelo)

El conductismo es una corriente de la psicología, creada por \textit{Jhon
    Watson}, y posteriormente perfeccionada por \textit{Pavlov},
\textit{Skinner}, y \textit{Thorndik}. El conductismo defiende la idea de que
todas las acciones que realizan los seres vivos son consecuencia de un estímulo.
Un ejemplo de esta técnica es el experimento de \textit{Pavlov}, \fixme{donde un
    perro es alimentado cada vez que suena una campana, provocando que el perro
    salive cuando suena la campana, incluso si no existe una recompensa
    (alimento)}{algún mejor ejemplo?}\cite{weegar2012comparison}.

\fixme{El conductismo permite a la epistemología utilizar un enfoque
    científico, permitiendo controlar todas las variables controlables, como el
    estímulo y la reacción, e ignorando los pensamientos y experiencias de las
    personas\cite{weegar2012comparison}. }{Traducir en algo más extendible sin
    usar términos como variables controladas, etc.}

La primera incursión del conductismo con las \Gls{tic}, fue presentada por
\textit{Skinner}, en $1958$\cite{weegar2012comparison}, donde se describe una
máquina que contiene botones y una pantalla donde se presenta una pregunta,
para responder el usuario dispone de varias opciones, cada opción esta
relacionada con un botón, si el aprendiz no presiona el botón correcto, debe
seguir intentando hasta acertar y así avanzar\cite{weegar2012comparison}, este
es el inicio de lo que se conoce como \enquote{Prueba y Error}.

Una característica del conductismo, es la ley de \textit{Thorndike}, indica que
una acción cuya consecuencia es un estímulo favorable, es más probable que sea
repetida\cite{weegar2012comparison}, en la
tabla~\ref{tab:conductismo_estimulo}, se observa los distintos mecanismos 
que propone el conductismo para alentar o desalentar un comportamiento.

\begin{table}[!hbt]
\begin{center}
\begin{tabulary}{\textwidth}{|L|C|C|}
\hline
& Comportamiento alentado & Comportamiento reprimido \\
\hline
Estímulo presente & Refuerzo positivo, por ejemplo, buenas notas & Castigo
Presente, por ejemplo, tiempo después de clase \\
\hline
Estímulo eliminado & Refuerzo negativo, por ejemplo, no hacer quehaceres &
Castigo eliminado, por ejemplo, no permitir utilizar la computadora. \\
\hline
\end{tabulary}
\end{center}
\caption{Tipos de estímulos}
\label{tab:conductismo_estimulo}
\end{table}

A finales de la década de $1970$ e inicios de la década de $1980$,  la
complejidad técnica de las computadoras limitaba la cantidad de herramientas
disponibles, los programas eran desarrollados por profesores, y su objetivo era
que los alumnos puedan poner en práctica lo aprendido en el aula. 


El campo de aplicación de las herramientas, basadas en el experimento de
\textit{Skinner}, se limitaban a matemáticas y lenguaje, donde se podía evaluar
inmediatamente los resultados proveídos por los alumnos, pues, normalmente era
un enunciado y una lista posible de opciones del tipo \enquote{Prueba y
    Error}\cite{leinonen:ict}. 

La cantidad limitada de opciones para responder, provocó que los alumnos no
interpreten los resultados, sino prueben todas las posibles opciones hasta
pasar al siguiente enunciado, sin obtener ningún aprendizaje
significativo\cite{leinonen:ict}.

Cuando aparecieron en el mercado computadoras con multimedia, a finales de la
década de $1980$, la utilización de las \Gls{tic} se simplificaron y dieron
contenido a la posibilidad de incluir contenido multimedia, se argumentó que los
ejercicios de tipo \enquote{Prueba y Error} no cumplieron su objetivo de una
educación profunda por que no contenían multimedia\cite{leinonen:ict}, así, en
se empezaron a distribuir las aplicaciones por \textit{CD-ROM} y contener gran
cantidad de contenido multimedia.

Con la creación de los juegos del tipo \enquote{Prueba y Error} y el contenido
multimedia, se inicio a un nueva corriente denominada \emph{Edutainment},
palabra que representa la unión de la educación y el entretenimiento. 

\subsubsection{Edutainment}
\label{sec:edutainment}

\observacion{Esquema Global
\begin{itemize}
    \item Que es?
    \item Quien creo y cuando?
    \item Ejemplos
    \item Fortalezas y desventajas
\end{itemize}
Tratar de hacer las descripciones más simétricas tirando hacia el resumen}

Los \emph{edutainment} se basan principalmente en el conductismo y el
cognoscitivismo, se enfoca en juegos sencillos que transmiten información simple
al usuario, su estructura se basa en un objetivo claro que está separado de la
experiencia educativa\cite{egenfeldt2007third}. 

Así el \emph{edutainment} pretende agregar entretenimiento a la educación, se
ve al alumno como un receptor pasivo de información que debe asimilarla, y para
aumentar la implicación de los alumnos, el entretenimiento era
agregado\cite{resnick:2004}.

\observacion{Conectar mejor \enquote{Un ejemplo es Math Blaster}}

\emph{Math Blaster} (ver~\ref{fig:math_blaster}) es un \emph{edutainment} donde
el alumno debe responder repetitivamente preguntas aritméticas para obtener
municiones, luego con esas municiones debe completar diferentes misiones en una
nave\cite{bruckman1999can}. Como todas las preguntas se responden mediante un
mecanismo de selección múltiple, y no existe penalización por fallar una
respuesta, los alumnos no reflexionan sobre las respuestas elegidas, seleccionan
una opción aleatoria y si no es la correcta, prueban otra, tras una cantidad
finita de intentos, siempre se obtiene la recompensa deseada.

\begin{figure}[ht!] 
\centering 
\includegraphics[scale=0.5,natwidth=296,natheight=217]{tics/images/math_blaster.jpg}
\caption{Math Blaster, \emph{edutainment} del año 1987}
\label{fig:math_blaster} 
\end{figure}

\begin{figure}[ht!] 
\centering 
\includegraphics[scale=0.5]{tics/images/carmen.jpg}
\caption{Donde en el mundo esta Carmen Sandiego} 
\label{fig:carmen}
\end{figure}

\enquote{Donde en el mundo esta Carmen Sandiego} (ver~\ref{fig:carmen}) es un
juego que representa el potencial multimedia de esta época, el objetivo del
juego era detener a una serie de criminales mediante varias pistas que eran
provistas en forma de texto. Este exitoso juego demuestra las falencias del
\textit{Edutainment}, siendo visualmente muy atractivo, y con contenido
multimedia acorde a su tiempo, no era más que \enquote{Prueba y Error}, cada
nivel del juego podía ser completado sin leer la información proveída
educativa\cite{charsky:2010}.

Los \emph{edutainment} no logran enseñar habilidades complejas, se enfocan
principalmente en enseñar tareas extremadamente repetitivas que no dependen de
un contexto\cite{charsky:2010,egenfeldt2007third,bruckman1999can}, son
excelentes para enseñar a sumar, pero no para aplicar ese conocimiento, analizar
y obtener conclusiones, o evaluar lo que aprendieron.

Las principales causas por del fracaso de los \emph{edutainment} en su intento
de ser una alternativa viable a la educación son según\cite{egenfeldt2007third}: 

\begin{itemize}

\item \textbf{Falta de motivación interna:} los \emph{edutainment} se centran en
    motivaciones externas, y dejan de lado la motivación interna. Se centran en
    dar recompensas por acciones lo que es una motivación externa, que en que el
    alumno se sienta emocionado al finalizar un nivel, lo que es una forma de
    motivación interna.

\item \textbf{Aprendizaje como anexo:} el principal objetivo del desarrollo de un
    \emph{edutainment} es el de entretener, los objetivos pedagógicos son
    agregados al final. Adicionalmente, este aprendizaje se provee a través de
    largos textos que normalmente son omitidos.

\item \textbf{Interacción limitada:} son construidos con una jugabilidad pobre,
    sin la posibilidad de realizar multiples acciones, normalmente limitados a
    seleccionar respuestas o moverse en un pequeño mundo. 

\item \textbf{Ejercicios de prueba y error sistemáticos:} todas las debilidades
    anteriores se pueden fundamentar en el hecho de que los juegos permiten al
    alumno intentar varias veces sin ser penalizados, además
    de que los alumnos no están motivados, provocaba que todas las opciones sean
    probadas sin el proceso de reflexión necesario para aprender, por ejemplo,
    varios juegos aritméticos solicitaban pruebas del tipo $2+2$ donde el alumno
    probaba diferentes resultados y luego memorizaba el mismo. Se enseñaba a
    probar opciones sin sentido antes que entender y analizar la experiencia.

\end{itemize}

La distribución por medios físicos, aunque contribuyo a la calidad de material
proveído, no resolvió el problema de información actualizada, pues a comienzos
de la década de $1990$, con la popularización de Internet, se genera más
contenido que el que puede ser distribuido por medios físicos, así se accede al
tercer hito de la figura~\ref{fig:history_tics}, donde el contenido es
distribuido por Internet.

La incursión de internet solo permite contenido actualizado, los problemas
persisten y se crean nuevos factores que ensanchan la brecha tecnológica,
ahora, además de poseer una computadora, es necesaria una conexión permanente.
Adicionalmente, la velocidad inicial de Internet no es suficiente para proveer
los mismos entornos ricos en multimedia que sí lo proveían los
CD-ROM\cite{leinonen:ict}.

\textit{Skinner}, se dio cuenta que el compartimiento humano no puede ser
reducido al conductismo, no solo responde a los estímulos, sino, que además
responde a su experiencia previa\cite{weegar2012comparison}, así se estudia el
constructivismo, que se centra en el alumno y sus experiencias.

