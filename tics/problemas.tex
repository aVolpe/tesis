%! TEX root = ../main.tex
\section{Problemas actuales}
\label{tics:problemas}

Durante la historia de las \Gls{tic} en la educación, se han encontrado
diferentes dificultades a la hora de aplicar los nuevos conceptos en la
educación, desde los primeros enfoques que carecían de bases pedagógicas válidas
hasta la actualidad.

El principal problema es falta de motivación de los profesionales de la
educación para emplear las \Gls{tic}\cite{punie:ict}\cite{ict:romeo}.

El contenido proveído actualmente puede ser considerado como un conjunto de
buenas prácticas\cite{punie:ict} y así, omiten completa o parcialmente el
contexto donde esa buena práctica fue generado.

Las \Gls{tic} han tenido un impacto positivo en la educación\cite{punie:ict},
pero no han obtenido el impacto esperado.

Iniciativas como el \emph{edutainment} que prometían ser la solución a los
problemas educacionales no cumplieron las expectativas. Sucesivos fracasos en
los resultados obtenidos dotaron a los \emph{edutainment} de una reputación
negativa, y hoy en día son considerados como el peor tipo de educación, pues son
un ejercicio de \emph{prueba-error} ocultos bajo un juego poco
entretenido\cite{resnick:2004}. La principal critica contra los
\emph{edutainment} es su incapacidad de enseñar como aplicar conceptos
aprendidos a un entorno real\cite{resnick:2004}.

Mientras que la utilización de las \Gls{tic} puede eliminar problemas actuales
como el aislamiento y la falta de pensamiento de alto nivel\cite{punie:ict}, la
brecha social existente implica otro riesgo para la utilización de las \Gls{tic}
en la educación, aquellos que no posean los recursos económicos necesarios para
acceder a la misma no se verán beneficiados por las \Gls{tic}\cite{punie:ict}.

Muchas empresas que están en el área de las \Gls{tic} en educación siguen en la
época donde los juegos son prueba y error, esto no significa que los mismos no
funcionen, sino que pueden ser mejorados
considerablemente\cite{egenfeldt2007third}.

Otro de los problemas actuales es la dificultad comercial impuesta por la
historia de los mismos, es muy difícil para los juegos actuales presentar
promesas realistas, principalmente por el antecedente sentado por los
edutainment\cite{egenfeldt2007third}.

