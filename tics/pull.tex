\section{Teorías del aprendizaje: Modelo Pull}

En el modelo \textit{pull}, los usuarios participan de forma activa en la
construcción del conocimiento. El rol de los profesionales de la educación es
guiar a los alumnos. Entre las corrientes pedagógicas podemos encontrar al
constructivismo y al construccionismo.

\subsection{Constructivismo}

El constructivismo es una corriente pedagógica creada por \textit{Jean Piaget} y
\textit{Lev Vygotsky}, cuya idea central es que el aprendizaje humano se
construye, que la mente de las personas elabora nuevos conocimientos a partir de
la base de enseñanzas anteriores\cite{martin2008modelo}. Predica que el
aprendizaje de los estudiantes debe ser activo, deben participar en actividades
en lugar de permanecer de manera pasiva observando lo que se les
explica\cite{hernandez:constructivismo,johnson2005instructionism}.

El constructivismo es un conjunto de prácticas que se enfocan en el alumno,
basados en el contenido, orientados al proceso, interactivos y que responden a
las necesidades e intereses personales de los
alumnos\cite{johnson2005instructionism}.

Se basa en que las personas no entienden, ni utilizan de manera inmediata la
información que se les proporciona, en cambio, el individuo construye su propio
conocimiento. El conocimiento se construye a través de la experiencia y esto
conduce a la creación de modelos mentales que se almacenan en la mente. Estos
esquemas van evolucionando, ampliándose y volviéndose más complejos a través de
dos procesos complementarios: la asimilación y el
alojamiento\cite{hernandez:constructivismo,johnson2005instructionism}.

Epistemológicamente el constructivismo es subjetivo, pues considera que el
conocimiento depende de las experiencias\cite{johnson2005instructionism}. 

Las aulas constructivistas crean un mundo realista donde lo más importante es el
aprendizaje, el profesor es un facilitator del aprendizaje del
alumno\cite{johnson2005instructionism,nanjappa2003constructing}.

\subsubsection{Ejemplos}

Entre los ejemplos del constructivismo podemos encontrar:

\begin{itemize}

\item \textbf{Redes sociales:} funcionan como una continuación del aula escolar,
    pero de carácter virtual, ampliando el espacio de interacción entre los
    estudiantes y el profesor, permitiendo el contacto continuo con los
    integrantes, y proporcionando nuevos materiales para la comunicación entre
    ellos. Esta tecnología presenta las características de interacción, elevados
    parámetros de calidad de imagen y sonidos, instantaneidad, interconexión y
    diversidad\cite{hernandez:constructivismo}. 

\item \textbf{Enciclopedias online o wikis:} genera un cambio drástico en la
    manera tradicional de obtener información para los temas impartidos en el
    aula; con las wikis los alumnos no sólo obtienen información, sino que ellos
    mismos pueden crearla. Los estudiantes pasan de ser simples observadores y
    trabajar de manera pasiva, a estar involucrados activamente en la
    construcción de su conocimiento, escuchando en clase, investigando fuera de
    ella  y después redactando artículos en la wiki que reflejen sus
    investigaciones, lo que han aprendido y la forma cómo lo han
    hecho\cite{hernandez:constructivismo}.

\item \textbf{Blogs:} ofrecen un espacio en el que los usuarios tienen la
    oportunidad de expresar sus ideas sobre cualquier tema que les interese. Los
    usuarios que acceden a los blogs pueden comentar sobre los escritos y dejar
    sus opiniones, consiguiendo un diálogo entre el propietario del blog y los
    que acceden a él\cite{hernandez:constructivismo}.
 
\end{itemize}

\subsection{Construccionismo}
\label{sec:tics_construccionismo}

El construccionismo es una corriente pedagógica que parte de una concepción del
aprendizaje según la cual la persona aprende por medio de su interacción
dinámica con el mundo físico, social y cultural en el que está
inmerso\cite{valdivia:sg}. El construccionismo nace en la década de $1980$ de la
mano de \textit{Seymour Papert}\cite{historia:2014,ict:ttc}.

Para el construccionismo, el conocimiento es construido por el estudiante en
lugar de ser trasmitido por el profesor\cite{moses:2003} y esto sucede
particularmente cuando el mismo se compromete en la elaboración de un producto o
artefacto que tenga un significado y pueda ser compartido\cite{valdivia:sg}. De
esta manera, se permite a los estudiantes elaborar sus propias interpretaciones
razonadas del mundo mediante la interacción con el mismo. El profesor actúa como
guía para el estudiante en la construcción de su conocimiento, aportando
conocimiento y experiencia. El construccionismo utiliza a las \gls{tic} como
medio cognitivo y no para la entrega de contenido\cite{sasha:construtivism}.

Según \textit{Papert}, los alumnos estarán mucho más involucrados en su aprendizaje si
construyen artefactos que los demás pueden ver, criticar y tal vez utilizar. Y
además, el alumno se enfrenta a problemas complejos con estas construcciones,
harán el esfuerzo por resolver problemas y aprender ya que la construcción les
motivará\cite{const:vs}.

\subsubsection{Ejemplos} 

Existen varios proyectos o iniciativas que incluyen al construccionismo como
base pedagógica, para la mayoría de ellos las computadoras son esenciales
mientras que para otros el mayor esfuerzo está en la incorporación de la
tecnología en su práctica educativa\cite{papertian:const}.

Algunos de estos proyectos son:

\begin{itemize}

\item \textbf{Lenguaje de programación \enquote{LOGO}}: El lenguaje
    \enquote{LOGO} es la cuna del construccionismo. Fundamentalmente consiste en
    presentar a los niños retos intelectuales que puedan ser resueltos mediante
    el desarrollo de programas en \enquote{LOGO}. El proceso de revisión manual
    de los errores contribuye a que el niño desarrolle habilidades
    metacognitivas al poner en práctica procesos de
    auto-corrección\cite{logo:sg,ict:ttc}.

    Las actividades de programación \enquote{LOGO} se realizan en las áreas de
    matemática, lenguaje, música, robótica, telecomunicaciones y ciencias.
    \enquote{LOGO} es accesible para principantes, especialmente niños pequeños,
    y es compatible con exploraciones complejas y proyectos sofisticados
    realizados por usuarios experimentados\cite{logo:sg}.
    
\item \textbf{\Gls{olpc}}: es una asociación sin ánimos de lucro cuyo esfuerzo
    se centra en dotar a los niños de una computadora duradera, accesible y
    potente en los países en desarrollo, se dice que es un descendiente directo
    del construccionismo\cite{papertian:const}.
	
    Surgió dentro del \gls{mit}, \Gls{olpc} propone un cambio de paradigma
    basado en un modelo de aprendizaje en el que cada alumno disponga de su
    propia computadora portátil y que se pueda conectar a Internet, de forma
    totalmente gratuita, desde su escuela. A partir de esta política se pretende
    disminuir la brecha tecnológica y de acceso a la información en países más
    desfavorecidos en comparación con los países del primer
    mundo\cite{videojuegos:gonzaleztardon}.
    
    \textit{Sugar} es el principal factor construccionista de \Gls{olpc}, consiste en una 
    interfaz que captura gráficamente el mundo de los estudiantes y maestros, 
    haciendo énfasis en las conexiones dentro de la comunidad, entre las personas y 
    sus actividades\cite{olpc:sugar}.

	
\item \textbf{Fabricación personal}: \textit{Neil Gershenfeld}, colega de
    \textit{Papert} en el Media Lab del \Gls{mit} dictó un curso titulado
    \emph{Cómo hacer casi cualquier cosa}. La idea se centraba en la creación de
    la tecnología que se necesita para resolver los problemas que se poseen.
    Esta auto-confianza, la autonomía personal y la agencia sobre la tecnología
    han estado en el centro de trabajo de \textit{Papert} durante años.
    \textit{Papert} no sólo defendió la idea de que los niños posean
    computadoras personales, sino también que a la larga ellos debían
    mantenerlas, repararlas e incluso construirlas.

    Junto con la capacidad para utilizar la tecnología para inventar soluciones
    a los problemas de significado personal, los estudiantes no sólo tienen
    acceso a la información, sino que tienen una mayor capacidad para darle
    forma a su mundo. La fabricación personal promueve la visión de
    \textit{Papert} \emph{Si se puede utilizar la tecnología para hacer las
        cosas, usted puede hacer las cosas mucho más interesantes y usted puede
        aprender mucho más haciéndolo}\cite{papertian:const}.

\end{itemize}

\subsection{Ventajas y desventajas del modelo Pull}

Las principales ventajas de los modelos \emph{pull} son:

\begin{itemize}

\item La enseñanza es más eficaz que la enseñanza tradicional en términos de
    logros académicos de los estudiantes\cite{kim2005effects}.

\item Tiene un efecto positivo sobre la motivación
    para aprender las tareas académicas causando ansiedad en el proceso de
    aprendizaje\cite{kim2005effects}.

%\item El estudiante asume un papel activo en la construcción del
%    conocimiento\cite{hernandez:constructivismo,johnson2005instructionism}.

\item Permite desarrollar el pensamiento de alto nivel lo que involucra la 
	resolución de problemas y el
    pensamiento crítico\cite{wilson2012constructivism}.

\item Se basa en la totalidad de la persona y
    conduce a representaciones más realistas de la
    experiencia\cite{wilson2012constructivism}.

\end{itemize}

Las principales desventajas de los modelos \emph{pull} son:

\begin{itemize}

\item Las técnicas tradicionales de evaluación no se ajustan a estas
    teorías. Tanto el constructivismo como el construccionismo predican 
    que el estudiante es el más apropiado para evaluar su conocimiento, por
    lo que se deben desarrollar nuevas técnicas de
    evaluación\cite{lai1995implications}.

\item No pueden ser aplicadas a todos los contextos del
    aprendizaje\cite{lai1995implications}, especialmente en los aspectos
    que involucran pensamiento de bajo nivel.

\end{itemize}
