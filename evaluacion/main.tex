%! TEX root = ../main.tex

\chapter{Evaluación}
\label{chap:evaluacion}

\observacion{El único detalle que veo es que hay (naturalmente) demasiada
información, discutamos sobre si juntar o no los capítulos}

\observacion{\begin{itemize}
\item Poner descripción al nombre de las pruebas por ejemplo no objetiva o
subjetiva
\item Juntar cap 8 y 9
\item  Poner Juego Serio en el titulo de la tesis en lugar de construccionismo
\item Explicar a quien se le hizo el test, resumen, correlación
\item Borrar los Nos
\item Resultado objetiva en cuanto a que algunos son
\item Ver lo de aleatoriedad (variable) en subjetiva
\end{itemize}}

En este capitulo se definen los \fixme{mecanismos}{Cambiar} utilizados para
evaluar la solución propuesta, estos mecanismos están orientados a la validación
de las hipótesis planteadas durante el desarrollo de la solución, así como la
evaluación de aspectos pedagógicos, de utilidad y de la participación activa del
usuario. Como parte de la evaluación se miden variables relacionadas a los
aspectos mencionados.

La evaluación se divide en cinco partes principales:

\begin{itemize}
    \item \textbf{Prueba de interfaz de usuario:} Es una prueba inicial para
    medir la calidad de la interfaz y la interacción con la misma, esta
    evaluación es realizada con personas no relacionadas al área de enfermería.
    La prueba es llevada a cabo durante el desarrollo de la solución a
    diferencia de las demás, las cuales son realizadas una vez terminada la
    solución.

    \observacion{No queda claro a quien le hacen usar}

    \item \textbf{Encuesta de ubicación:} Es una encuesta acerca del nivel de
    acceso a la tecnología que poseen los alumnos del 4to año del \Gls{iab}, de
    ahora en más \textit{el Universo}\revisar{Población?}, esta encuesta sirve
    para definir la muestra.

    \item \textbf{Encuesta Subjetiva:} Es una encuesta realizada a cada sujeto
    de la población que participa haciendo uso de la solución, donde se busca la
    opinión del mismo acerca de la solución y factores relacionados a la misma. 

    \item \textbf{Encuesta Objetiva:} Es un cuestionario que es completado por
    el universo de alumnos, donde se mide el conocimiento de los mismos, se
    utilizan a los \revisar{También participan} alumnos que no forman parte de
    la \fixme{muestra}{}, como grupo de control.

    \item \textbf{Registro de actividades:} Es información almacenada por la
    solución automáticamente, y contiene datos acerca del uso y el desempeño del
    alumno.
        
\end{itemize}

El capitulo define los objetivos de la evaluación, describe brevemente conceptos
transversales a las técnicas utilizadas y luego define las metodologías,
métricas y variables utilizadas en cada parte de la evaluación.

\section{Objetivos}
\label{sec:objetivos}

Los \emph{Registros de actividades} y \emph{Encuesta objetiva} buscan obtener 
información sobre el aprendizaje y la utilización de la solución, mientras que 
la \emph{Encuesta subjetiva} busca obtener información acerca de las fortalezas y 
debilidades de una simulación para el entrenamiento de enfermeros y de la solución 
propuesta.

Se definen los objetivos principales de la evaluación como siguen:

\begin{enumerate}

\item Validar las hipótesis asumidas durante el desarrollo de la
    solución.
    % Revisar

\item Evaluar los puntos fuertes y débiles de la solución.
%en cuanto a los criterios definidos.
    %   Usabilidad
    %   Ubicuidad
    %   Realismo
    %   Mirar la encuesta pare ver a que se refiere esto

\item Determinar los factores que afectan al uso de herramientas similares para
    apoyo a profesionales de enfermería
    %   Estos son los puntos que me parecen que pueden responder a este objetivo
    %%   SC7.Es importante mostrar los detalles de los pasos incorrectos ya que no sería suficiente sólo decir qué pasos se hicieron correctamente.
    %%   SC14.La opción de hablar para que aparezca el menú de ordenes verbales hace que el juego sea mas interactivo, siendo más similar a la realidad
    %%   SC15.La escenografía en los juegos permite que entremos en ambiente para realizar los procedimientos.
    %%   SC16.Los gráficos en tres dimensiones nos ayudan a entender mejor el entorno y las posibles acciones
    %%   SC17.La falta de pistas durante el desarrollo de una partida permite plasmar y medir el conocimiento acerca del tema
    %%   SC18.Es importante dar una puntuación total para ver el rendimiento
    %%   SC19.Motiva compartir y ver el progreso con otras personas a través del Facebook
    %%   SC20.La cantidad de tiempo de cada procedimiento jugado, motiva a seguir jugando para mejorarlo
    %%   SC21.El puntaje de cada procedimiento jugado, motiva a seguir jugando para mejorarlo
    %%   SC22.Interactuar con un paciente que reacciona a mis acciones, es mejor que utilizar un maniquí inmóvil
    %%   SC23.La utilización de la simulación en todo momento provee más facilidades para poner en práctica los conocimientos con respecto a las demás alternativas.
    %%   SC24.La utilización de herramientas alternativas, como la simulación, es útil para complementar el estudio en clase y laboratorio
    %%   SC27.Juegos cortos permiten jugar varias veces de seguido
    
    % Los factores serían
    %%  Nivel de detalle de retroalimentación       SC 7
    %%  Interfaz a través de voz                    SC14
    %%  Entorno realista                            SC15 - SC16 - SC22 - SC23
    %%  Interacción social                          SC19 
    %%  Puntuación y tiempo                         SC18 - SC21
    %%  Duración                                    SC20 - SC27
    %%  Utilidad                                    SC23 - SC24
    
\item Determinar el nivel de aceptación de la solución.

\item Evaluar la utilización de la solución, y el progreso de los usuarios.

\item Identificar la influencia de la utilización de la solución en el ámbito
    pedagógico.

\item Determinar correlaciones entre variables estudiadas, a fin de determinar
    la influencia de la utilización de la solución.
    
\end{enumerate}


%! TEX root = ../main.tex

\section{Métricas generales}

Se definen métricas generales como aquellas que son utilizadas por más de una
prueba o encuesta que forma parte de la evaluación. Se describen aquellas que se
consideran que son requeridas detallar.

A continuación se describe la escala de Likert que es una métrica utilizada en
la \emph{Encuesta subjetiva} y en la encuesta correspondiente a la \emph{Prueba
    de interfaz de usuario}. Por otra parte se describe también en que consiste
la correlación de Pearson, métrica que es utilizada para medir el grado de
relación entre variables de las encuestas realizadas, registros de actividades,
entre otros.

%Existen métricas que son usadas por más de una prueba, a continuación se
%describen estas métricas.

Cabe destacar que en~\cite{norman2010likert} se demuestra que, aunque el tamaño
de la muestra sea pequeña, los datos no puedan ser distribuidos normalmente o
los datos sean de escalas de tipo Likert, los métodos paramétricos como el
análisis de varianza, la regresión y la correlación pueden ser utilizados.


\subsection{Escala de Likert}
\label{sec:likert}

Para la valoración de las variables medidas en la \emph{Prueba de interfaz de
    usuario} y la {Encuesta subjetiva} se utiliza la escala de
Likert\cite{Allen:2007} de 7 valores posibles. La escala de Likert es utilizada
para permitir a las personas indicar cuánto están de acuerdo o en desacuerdo con
respecto a ciertos puntos. Los valores utilizados, son:

\begin{enumerate}
    \item Totalmente en desacuerdo.
    \item En desacuerdo.
    \item Parcialmente en desacuerdo.
    \item Neutral.
    \item Parcialmente de acuerdo.
    \item De acuerdo.
    \item Totalmente de acuerdo.
\end{enumerate}

Una vez valoradas y registradas todas las respuestas y con el objetivo de
eliminar las tendencias en la forma en la que son completadas las
encuestas\cite{Fischer2010} se utiliza el método de \emph{Doble Estandarización}
recomendado en~\cite{Pagolu2011}. Este método, consiste en dos
estandarizaciones, la primera por fila, que en este caso representa a los
individuos y la segunda por columna donde cada columna representa una de las
diferentes preguntas de la encuesta.

Siendo:
\begin{itemize}
	\item $\min_i$ la respuesta de menor valor del usuario $i$.
	\item $\max_i$ la respuesta de mayor valor del usuario $i$.
\end{itemize}

Para cada respuesta $s$ del usuario $i$, el valor ajustado, por la primera 
normalización, $s_1$ se define como:

\begin{equation}
s_1{_i}=\frac{s-\min_i}{\max_i-\min_i}
\end{equation}

Y luego siendo:
\begin{itemize}
	\item $groupmin_i$ la respuesta ajustada de menor valor en el grupo $i$.
	\item $groupmax_i$ la respuesta ajustada de mayor valor en el grupo $i$
\end{itemize}

Para cada respuesta ajustada $s_1{_i}$ del usuario $i$, el valor ajustado $sa_i$ se
define como:	

\begin{equation}
sa_i=\frac{s_{1_i}-groupmin_i}{groupmax_i-groupmin_i}
\end{equation}

Obteniendo así un valor normalizado, tanto por individuo, como por pregunta, en
el rango $0$ y $1$.

Para la valoración absoluta de cada  item se utiliza la media de cada columna o
respuesta a una pregunta de la encuesta.

Siendo:
\begin{itemize} 
\item $r_{k_i}$ la respuesta del usuario $i$ a la pregunta $k$.
\item $t_k$ la cantidad total de usuarios que respondieron la pregunta $k$.
\end{itemize}

El puntaje promedio de cada pregunta o item evaluado  $p_k$ en la encuesta se
define como:

\begin{equation}
p_k = \frac{\sum_{i=1}^n{r_{k_i}}}{t_k}
\end{equation}

%Cabe destacar que en \cite{norman2010likert} se demuestra que aunque el tamaño de la muestra 
%sea pequeña, los datos no puedan ser distribuidos normalmente o los datos sean de 
%escalas de tipo Likert los métodos paramétricos como el análisis de varianza, la 
%regresión y la correlación pueden ser utilizados.


\subsubsection{Manejo de información faltante}
\label{sec:informacion_faltante}

En toda encuesta pueden existir preguntas que no son respondidas por los encuestados, 
en este tipo de situaciones existen tres posibles formas de categorizar el 
patrón de ocurrencia de la falta de 
respuestas\cite{leite2010performance,tsikriktsis2005review}:

\begin{description}
    \item[Información faltante completamente aleatoria:] cuando la información
        faltante es independiente de la variable medida y de otras variables.
    \item[Información faltante aleatoria:] cuando la información faltante depende
        de otras variables, pero no de la variable en sí. 
    \item[Información faltante no aleatoria:] cuando hay una relación entre la
        información faltante y el valor de la variable.
\end{description}

Una vez categorizado el patrón de ocurrencia, existen a su vez tres
mecanismos~\cite{tsikriktsis2005review} principales para lidiar con información
faltante como son la eliminación, el reemplazo y los  procedimientos basados en
modelo.~\cite{tsikriktsis2005review} recomienda utilizar un mecanismo de
reemplazo para escalas del tipo Likert.

Las técnicas de reemplazo se clasifican en tres grandes
grupos\cite{tsikriktsis2005review}:
\begin{enumerate*}[label=\itshape\alph*\upshape.]
\item basadas en el promedio,
\item basadas en regresión, e,
\item imputación \emph{hot deck}.
\end{enumerate*}

De estas técnicas se seleccionó la sustitución basada por promedio ya que las
relaciones entre las variables son bajas y los datos faltantes son menos del
$10\%$. La sustitución basada por promedio se divide nuevamente en tres
grupos\cite{tsikriktsis2005review}; promedio
\begin{enumerate*}[label=\itshape\alph*\upshape.]
\item total,
\item del subgrupo, y,
\item por caso.
\end{enumerate*}

La sustitución por promedio total es elegida debido a que la relación entre la
variable que falta y las demás variables en los datos es relativamente baja, es
fácil de usar y retiene la muestra. La sustitución por promedio total se realiza
obteniendo el promedio de todas las respuestas de la pregunta cuya respuesta
falte, la sustitución de subgrupo es similar, solo que se limita a aquellos
sujetos del mismo subgrupo del sujeto que no respondió, y finalmente, la
sustitución por caso, es el promedio de las respuestas válidas del sujeto.

\subsection{Correlación de variables aleatorias}
\label{sec:correlacion}

Las correlaciones se utilizan durante una etapa exploratoria o de observación de
la investigación para determinar las variables que tienen al menos una relación
estadística con cada uno de los diseños experimentales. Las correlaciones
también se utilizan para determinar el grado de asociación entre variables
dependientes e independientes. Por otro lado, el coeficiente de correlación se
utiliza comúnmente para cuantificar el grado de asociación entre dos variables
\cite{BoslaughStatistics2008}.

La correlación de Pearson\cite{BoslaughStatistics2008} mide la relación que
existe entre dos variables, $X$ e $Y$, el mismo esta comprendido entre $-1$ y
$1$, en su punto más bajo ($-1$) indica que una de las dos variables crece mientras
la otra decrece, y en su punto más alto ($1$), indica que ambas crecen o
decrecen conjuntamente, el valor $0$, indica que no existe una relación entre
ambas variables.

El coeficiente para las variables $X$ e $Y$ está dado por:

\begin{equation}
r = \frac{\sum_{i=1}^n{(\frac{x_i-\bar{x}}{s_x})({\frac{y_i-\bar{y}}{s_y}})}}%
{n - 1}
\end{equation}

donde:

\begin{itemize}
    \item ($x_i$, $y_i$) es el conjunto de coordenadas de las variables $X$ e $Y$.
    \item $\bar{x}$ es la media de la variable $X$.
    \item $\bar{y}$ es la media de la variable $Y$.
    \item $s_x$ es la desviación estándar de la variable $X$.
    \item $s_y$ es la desviación estándar de la variable $Y$.
    \item $n - 1$ son los grados de libertad.
\end{itemize}

\section{Interfaz de usuario}
\label{sec:interfaz}

Durante el desarrollo de la solución se realizó una prueba para evaluar la 
interfaz de usuario, específicamente buscando la retroalimentación de usuarios 
acostumbrados a tecnología similar a la utilizada en la solución.

Esta prueba ayuda en el proceso de diseño e implementación de la solución con 
las características mencionadas en los objetivos del trabajo y acorde a los 
requerimientos. De esta manera se pueden identificar los aspectos que deben 
ser mejorados.

La prueba consta de dos partes importantes involucradas en la recolección
de datos para su posterior análisis. Estas partes son las siguientes:

\begin{description}

\item[Simulación:] Luego de una explicación acerca de las funciones y manejos
    generales de la solución por parte de los encargados de la prueba, cada usuario
    completa una tarea que consiste en realizar el procedimiento de extracción de 
    muestra de sangre, como ayuda, recibe una hoja con una lista de todos los pasos 
    necesarios para llevar a cabo el procedimiento utilizando la solución.
    	
    Las simulaciones son grabadas mediante programas de captura de pantalla, así
    como por detectores de eventos táctiles.
    	
\item[Encuesta:] Posteriormente se le provee una encuesta a cada
    usuario la cuál es utilizada para obtener una idea general acerca de la
    calidad de la simulación según la percepción de los mismos. Esta encuesta 
    contiene preguntas que son medidas mediante la escala de tipo Likert. 

\end{description} 

\subsection{Muestra}

La prueba de interfaz de usuario se realiza con alumnos de la carrera de
Ingeniería en Informática de la \Gls{fpuna}, sin experiencia previa tanto con la
solución como con los procedimientos simulados, pero sí familiarizados con la
utilización de dispositivos móviles. La muestra no requiere de sujetos que sean
parte del \emph{Universo} ya que sólo está orientada a mejorar aspectos de
interfaz de usuario y no el contenido de la solución, además se considera que la
muestra puede brindar una evaluación más crítica debido a su familiarización con
interfaces similares a la de la solución.

El número de muestras tomadas fue 8, ya que según~\cite{nielsen2000} son
necesarios al menos $5$ participantes para poder obtener resultados
significativos en una prueba de usabilidad. Además,~\cite{ritch2009} asegura que
la teoría de~\cite{nielsen2000} es verdadera especialmente para pruebas simples. 

Se fundamenta el número de participantes, y que es una prueba sencilla, por que:

\begin{itemize}

\item La prueba que se está ejecutando es sencilla ya no debería tomar más de $10$
    minutos.

\item Se busca solamente obtener información acerca de la interfaz, y no el
    funcionamiento en sí de la simulación, pues los usuarios no son expertos en
    el área y no tienen conocimiento acerca de qué debería pasar. 

\item No se busca medir el aprendizaje del usuario en temas no relacionados a la
    interfaz, es decir, no se mide el aprendizaje del usuario en el tema
    simulado.

\item El procedimiento esta bien definido y los pasos necesarios están a
    disposición del usuario en todo momento.

\end{itemize}

\subsection{Variables}
\label{sec:evaluacion_interfaz_variables}

Antes de definir las variables, se deben primero definir los conceptos 
relacionados a los tipos de acciones que pueden realizarse sobre el paciente 
virtual en la solución, los mismos son:

\begin{itemize}
\item \textbf{Acción por menú contextual:} se refiere a las acciones que el usuario 
    puede realizar utilizando el menú contextual que aparece sobre cada una de las
    herramientas disponibles en la solución.
\item \textbf{Acción por menú de la \Gls{gui}:} se refiere a las 
    acciones que el usuario puede realizar seleccionando una opción en los menús 
    principales que presenta la interfaz.
\item \textbf{Acción por herramienta:} se refiere a las actividades que el usuario 
    puede realizar cuando tiene seleccionada una herramienta y que no involucre el 
    uso del menú contextual.
\end{itemize}


Las variables medidas durante las simulaciones y realización de la tarea son las
siguientes:

\begin{itemize}

\item \textbf{Tiempo de realización de la primera acción por menú contextual:} 
    cuanto tiempo le toma al usuario realizar una acción por menú contextual la 
    primera vez.

\item \textbf{Tiempo de realización de la primera acción por \Gls{gui}:} cuanto 
    tiempo le toma al usuario realizar una acción por menú de 
    interfaz gráfica de usuario la primera vez.
    
\item \textbf{Tiempo de realización de la primera acción por herramienta:} cuanto 
    tiempo le toma al usuario realizar una acción por herramienta la primera vez.
    
\item \textbf{Tiempo de realización de las siguientes acciones por menú contextual:} 
    cuanto tiempo le toma al usuario realizar una acción por menú 
    contextual las siguientes veces.

\item \textbf{Tiempo de realización de las siguientes acciones por \Gls{gui}:} 
    cuanto tiempo le toma al usuario realizar una acción 
    por interfaz gráfica de usuario las siguientes veces.

\item \textbf{Tiempo de realización de las siguientes acciones por herramienta:} 
    cuanto tiempo le toma al usuario realizar una acción por herramienta 
    las siguientes veces.

\item \textbf{Tiempo total:} se refiere al tiempo empleado por el usuario para 
    completar la tarea asignada.

\item \textbf{Número de pasos realizados:} cantidad de pasos requeridos en la tarea 
    que son realizados por el usuario en la simulación. 

\item \textbf{Cantidad de movimientos espaciales por tipo:} número de veces en que se 
    modifica el estado de la cámara para realizar las acciones deseadas agrupados por 
    tipo (desplazamiento, acercamiento/desplazamiento).

\end{itemize}

En cuanto a la encuesta, las siguientes son las variables que fueron consideradas 
y medidas:

\begin{itemize}

\item \textbf{Calidad gráfica:} realismo y calidad de los modelos utilizados.

\item \textbf{Interacción:} desenvolvimiento en el entorno y utilización del 
    hardware.

\item \textbf{Interacción con objetos:} utilización errónea de objetos.

\item \textbf{Características del entorno:} realismo del escenario y de los 
    objetos utilizados.

\item \textbf{Usabilidad de la interfaz:} facilidad de uso de las opciones 
    proveídas por la interfaz.

\item \textbf{Integración con el hardware:} facilidad de uso de la solución con 
    un dispositivo móvil. 

\end{itemize}

\subsection{Métricas}

Para el análisis de la encuesta realizada a los usuarios, se utiliza la escala de Likert
con la \emph{Doble estandarización} explicada en~\ref{sec:likert}, y en el análisis de la 
interacción del usuario con la solución se utilizan las grabaciones registradas durante la 
prueba.

Haciendo uso de las variables descriptas anteriormente, las métricas
utilizadas son las siguientes:

\begin{itemize}
    
\item \textbf{Tiempo promedio de realización de las siguientes acciones por menú contextual:} 
    se obtiene dividiendo la cantidad total de tiempo empleado en realizar acciones por menú 
    contextual por el número de veces que se realizaron esas acciones, a excepción de la primera 
    vez. 
    
\item \textbf{Tiempo promedio de realización de las siguientes acciones por \Gls{gui}:} 
    se obtiene dividiendo la cantidad total de tiempo empleado en realizar acciones por \Gls{gui} 
    por el número de veces que se realizaron esas acciones, a excepción de la primera 
    vez. 
    
\item \textbf{Tiempo promedio de realización de las siguientes acciones por herramienta:} 
    se obtiene dividiendo la cantidad total de tiempo empleado en realizar acciones por 
    herramienta por el número de veces que se realizaron esas acciones, a excepción de la 
    primera vez. 
    
\item \textbf{Promedio de pasos correctos:} se obtiene dividiendo la cantidad de 
    pasos requeridos realizados por los usuarios sobre la cantidad de pasos requeridos. 
    
\item \textbf{Promedio de movimientos por tipo:} se obtiene dividiendo el número de 
    movimientos que fueron realizados agrupados por tipo (desplazamiento, acercamiento/
    desplazamiento) por la cantidad de usuarios.
    
\item \textbf{Promedio del tiempo total:} se obtiene dividiendo el tiempo total empleado 
    por los usuario para completar la tarea asignada por el número de usuarios.

\end{itemize}


%! TEX root = ../main.tex

\section{Encuesta de ubicación}
\label{sec:ubicacion}

\observacion{No repitan (se refiere al obtener información)}

A fin de obtener información acerca del nivel de acceso  de los alumnos a la
tecnología, se realiza una encuesta que cuenta con diez preguntas, las cuales
buscan obtener información acerca del modelo de dispositivo móvil, el acceso a
Internet, y la predisposición de cada alumno a ayudar en el experimento.

En el año $2014$, el \Gls{iab} cuenta con $124$ alumnos en el cuarto año distribuidos en
tres secciones, el cual es considerado el Universo. De los 124, 93 de
ellos estuvieron dispuestos a participar de la prueba y completaron la encuesta.

Con los resultados de la encuesta de ubicación tecnológica, se seleccionan
aquellos alumnos que posean dispositivos móviles que superan o igualan las
especificaciones.


\subsection{Variables}

Se definen $2$ factores necesarios para que un alumno pueda ser considerado como
sujeto de experimento\revisar{Utilizar termino prueba}, el primero es la
predisposición del mismo a participar del experimento y el segundo es que posea
un dispositivo móvil que supere los requisitos mínimos.

Los requisitos mínimos para que la solución tenga un desempeño que garantice una
experiencia fluida a la hora de utilizarla son:

\begin{itemize}
    \item Sistema Operativo Android $4.0$ o superior
    \item Memoria ram de $512$MB o superior.
    \item Velocidad de procesador de $800$ GHz o superior.
    \item \Gls{gpu} Mali 400 o superior.
    \item Conexión frecuente a internet.
\end{itemize}

El conjunto de funcionalidades utilizadas por la solución requiere un \Gls{api}
Android de nivel 14\cite{android:api} o superior, lo cual corresponde a un
sistema operativo Android 4.0.

Los requisitos de \textit{hardware} mencionados, son requeridos por las
características de la simulación, y la plataforma utilizada, es requerida una
\Gls{gpu} por los gráficos en tres dimensiones.

La conexión a Internet es requerida, pues los registros de actividad de cada
dispositivo deben ser enviados y almacenados para su posterior interpretación y
análisis.

\section{Encuesta subjetiva}
\label{sec:subjetiva}

Al final del período de prueba, cada alumno que forma parte de la muestra
completa una encuesta con $31$ preguntas que se utilizan para validar las
hipótesis, las cuales fueron explicadas en el capítulo~\ref{chap:requerimientos}
y para medir sus apreciaciones sobre otros aspectos de la solución que serán
detallados más adelante en esta sección. 

Las preguntas están agrupadas en dos, el primer grupo cuenta con $27$ preguntas
cerradas, es decir de una sola respuesta en una lista de opciones, el segundo
grupo cuenta con $4$ preguntas abiertas, es decir los encuestados pueden dar
respuestas libres a las preguntas. 

De esta forma, se busca identificar las fortalezas y debilidades de la solución,
además de evaluar la solución en cuanto a factores de exploración,
representación, motivación, inmersión, retroalimentación y pedagogía, de acuerdo
a las apreciaciones de los miembros de la muestra.

\subsection{Muestra}

Cada encuesta es entregada a los $11$ alumnos del universo que acordaron
participar en la prueba y que fueron seleccionados como resultado de la 
\emph{Encuesta de ubicación}, mientras completan la encuesta, un guía está presente
para responder cualquier duda.

La utilización de $11$ alumnos es suficiente, ya que según estudios presentados
en~\cite{nielsen2000}, mientras menos experiencia tengan los sujetos de estudio
con la solución planteada, serán necesarios menos para detectar un gran
porcentaje de errores y fortalezas, y según~\cite{ritch2009}, una base de $10$ a
$12$ es suficiente para obtener resultados estadísticamente válidos.

\subsection{Variables}
\label{sec:variables}

A continuación se describen las variables que tienen por objetivo demostrar la
validez de las hipótesis planteadas en este trabajo descritas en el
capítulo~\ref{chap:requerimientos} y la medición de otros aspectos de la
solución relacionados con los objetivos de este trabajo descritos en la
sección~\ref{sec:objetivos_generales}. Estas variables son agrupadas en
factores, los cuales representan aquellos aspectos de la solución propuesta que
buscan ser evaluados.

Cabe volver a resaltar que la medición de estas variables se realizan
exclusivamente de acuerdo a las valoraciones dadas por la muestra en cada uno de
las preguntas que forman parte de la encuesta.
%De acuerdo a los objetivos planteados en la sección~\ref{sec:objetivos}, se
%busca describir los factores\revisar{?} analizados en las pruebas y las variables
%relacionadas a los mismos, las cuales, tienen por objetivo demostrar la validez
%de las hipótesis planteadas en este trabajo.

%Las variables se presentaran agrupadas en factores, los mismos representan
%aquellos aspectos de la solución propuesta que buscan ser evaluados.

\subsubsection{Exploración}
\label{sec:sub_exploracion}

Este factor esta relacionado con la característica que posee la solución en
cuanto a la oportunidad que brinda al usuario para explorar cada uno de los
elementos del entorno simulado (paciente, herramientas propias del
procedimiento). En este sentido, se busca proveer facilidad de uso, intuitividad
y realismo en cuanto a las acciones y situaciones que se presentan en la
solución para que de esta manera, los elementos que la componen no representen
para el jugador un obstáculo que impida su uso.

Las variables que miden este aspecto son las siguientes:

\begin{description}

\item[Funciones realizadas por los elementos del juego:] se refiere a la
    correctitud con la que una herramienta o elemento del juego representa las
    funciones que el mismo puede realizar en la vida real, en este sentido, se
    evalúa el realismo con el que es representado tal elemento.

\item[Aleatoriedad para afianzar conocimientos:] se refiere al beneficio que
    puede traer el hecho de que el estado del paciente en el juego sea aleatorio
    en cuanto a la posibilidad que esto brinda al jugador para poner a prueba
    sus conocimientos teóricos.

\item[Aleatoriedad para representar realismo:] se refiere al uso de estados
    aleatorios en el paciente para que de esta forma el procedimiento se asemeje
    mas a una situación real.

\item[Facilidad de uso:] se refiere a lo fácil e intuitivo  que puede ser la
    utilización de los elementos del juego.

\end{description}

\subsubsection{Representación}
\label{sec:sub_representacion}

Este factor esta relacionado con la calidad y suficiencia con la que se
representan los diferentes objetos que son simulados en la solución. La
representación abarca tanto funcionalidad como aspecto del objeto.

De esta manera, se busca permitir al jugador realizar con los objetos las
acciones que requiera para llevar a cabo el procedimiento que se le presente en
la solución, y además, representar estos elementos de la mejor manera posible.

Las variables que miden estos aspectos son las siguientes:

\begin{description}

\item [Respuestas del paciente:] se refiere a la suficiencia de las respuestas 
    motrices, oculares y verbales que realiza el paciente en la escena 
    correspondiente a la valoración de la escala de Glasgow.

%\item [Movimientos motrices del paciente] se refiere a la suficiencia de los
%    movimientos motrices que realiza el paciente en el escena correspondiente a
%    la valoración de la escala de Glasgow.

%\item [Movimientos oculares del paciente] se refiere a la suficiencia de los
%    movimientos oculares que realiza el paciente en la escena correspondiente a
%    la valoración del escala de Glasgow.

%\item [Reacción verbal del paciente] se refiere a la suficiencia de las
%    reacciones o respuestas verbales que realiza el paciente en la escena
%    correspondiente a la valoración de la escala de Glasgow.

\item[Distinción entre los estados del paciente:] se refiere a si los diferentes
    estados del paciente son distinguidos correctamente en el procedimiento de
    valoración de la Escala de Glasgow ya que esto es importante para que el
    jugador pueda diagnosticar correctamente al paciente.

\item[Acciones con las herramientas:] se refiere a si las diferentes acciones que
    pueden realizarse con los elementos o herramientas del juego en un
    determinado procedimiento de enfermería son suficientes para ese
    procedimiento, ya que, debido a las limitaciones de la tecnología estas
    acciones son limitadas.

\end{description}

\subsubsection{Motivación}
\label{sec:sub_motivacion}

Este factor esta relacionado con la importancia de incluir en la solución
aquellas características que son propias de un videojuego convencional. Se
busca conocer el valor de estas características en cuanto a la motivación que
puedan producir en los jugadores tanto para volver a utilizar la solución como
para superarse en cada juego.

Las variables que miden estos aspectos son las siguientes:

\begin{description}

\item[Motivación del puntaje:] se refiere a que tanto motiva al jugador que la
    solución le proporcione un puntaje total al final de cada partida para poder
    mejorar constantemente siendo este puntaje como una evaluación final de todo
    lo que realizo dentro de la partida.

\item[Importancia del puntaje:] se refiere a que tan importante es para un
    jugador que se le proporcione un puntaje total al final de cada partida para
    poder visualizar su rendimiento.

\item[Socialización de los puntajes:] se refiere a si el hecho de que las
    personas del mismo entorno compartan sus puntajes, experiencias y logros en
    las partidas a través de redes sociales pueda ser motivador.

\item[Medición del tiempo:] se refiere a que tanto motiva al
    jugador que la solución le proporcione el tiempo que duro su partida
    sirviendo este tiempo como una evaluación de su precisión a la hora de
    realizar el procedimiento que se le presente.

\end{description} 


\subsubsection{Inmersión}
\label{sec:sub_inmersion}

Este factor esta relacionado con el sentimiento de formar parte de la escena. Es
decir, se trata de evaluar que tanto un jugador puede sentir que realmente se
encuentra dentro del juego para que de este modo el pueda entrar en ambiente
para realizar los procedimientos que se le presenten en sus partidas de juego.

Las variables que miden este aspecto son las siguientes:

\begin{description}

\item[Escenografía para entrar en ambiente:] se refiere a la importancia de la
    escenografía de la partida para que el jugador entre en ambiente para
    realizar el procedimiento que se le presente.

\item[Juegos cortos como ayuda para la repetición:] se refiere a si el hecho de
    que los procedimientos presentados en las partidas sean cortos contribuye a
    repetir las partidas varias veces de seguido entrando en un estado de
    inmersión.

\item[Gráficos en tres dimensiones para entender el entorno:] se refiere a la
    importancia que tiene el uso de gráficos en tres dimensiones para que el
    jugador pueda entender mejor el entorno y las posibles acciones que puede
    realizar.

\item[Realismo a través de ordenes verbales:] se refiere a si el hecho de que la
    solución brinde la posibilidad de que aparezca un menú de ordenes verbales
    en el momento en que el jugador habla hace que la acción de dar ordenes
    verbales se asemeje mas a la realidad.

\item[Simulación como herramienta:] se refiere a si la simulación ayuda al
    jugador a sentirse parte del laboratorio, dando cierto realismo a la escena
    que se le presenta.

\end{description}

\subsubsection{Utilidad}
\label{sec:sub_utilidad}

%Decia que replantee cambie utilidad por potencial y saque lo de "que pueda tener la solucion"
Este factor está relacionado con el potencial de la solución como herramienta 
de apoyo al proceso de aprendizaje de los estudiantes de enfermería.

Las variables que miden este aspecto son las siguientes: 

\begin{description}
% que tanto potencial, pedia que replantee
\item[Simulación para complementar el estudio en clase y laboratorio:] se
    refiere a que tanto potencial tienen las herramientas alternativas como la
    simulación pueden complementar a los métodos de aprendizaje tradicionales
    que son el estudio en clase y en el laboratorio.

\item[Simulación como proveedor de facilidades para el estudio:] se refiere a si las
    herramientas alternativas como la solución proveen más facilidades para
    poner en practica los conocimientos con respecto a los demás métodos de
    aprendizaje que son los libros, laboratorios y el campo de practicas.

\item[Interacción con el paciente:] se refiere a si el hecho de que el jugador
    pueda interactuar con un paciente que responde a las acciones del jugador 
    implica una mejora con respecto a otros materiales utilizados en los 
    laboratorios de práctica.
    %\revisar{Mejor que el maniqui? lo puede complicar por que no se
    %    midio esto.}.

\end{description}

\subsubsection{Retroalimentación}
\label{sec:sub_retroalimentacion}

Este factor esta relacionado con la importancia de ofrecer al jugador
información acerca de sus logros y errores de manera tal que el pueda estar
consciente de sus puntos fuertes y sus puntos débiles en los diversos
procedimientos que realice en la solución.

Las variables que miden este aspecto son las siguientes:

\begin{description}[style=unboxed]

\item[Detalles de los pasos realizados incorrectamente:] se refiere a 
    la importancia que tiene para el jugador que la solución no sólo le 
    diga los pasos que realizó de manera incorrecta sino también el por qué 
    de ello.

\item[Retroalimentación suficiente respecto a los pasos realizados:] se refiere 
    a sí son suficientes las justificaciones breves acerca de las causas por las 
    cuales se realizó incorrectamente un paso.

\item[Representación iconográfica de conceptos y acciones en la \Gls{gui}:] 
    se refiere a la suficiencia de mostrar iconos en la interfaz de 
    la solución para representar el estado actual del jugador.

\end{description}

\subsubsection{Pedagogía}
\label{sec:sub_pedagogia}

Este factor esta relacionado a la utilidad y al beneficio que puede traer la
solución para apoyar el aprendizaje del jugador. De esta manera, se busca
obtener la validez real de este tipo de herramientas como aporte al aprendizaje,
proveyendo mas interacción al jugador.

Las variables que miden este aspecto son las siguientes:

\begin{description}

\item[Potencial para memorizar y comprender el procedimiento:] se refiere a
    que tanto ayuda la solución al jugador para entender los procedimientos que se
    le presenten y para memorizar los pasos de cada uno de ellos.

\item[Falta de pistas como ayuda al aprendizaje:] se refiere a que tan efectivo
    resulta no dar pistas al jugador en el momento de realizar un procedimiento
    para que pueda plasmar y medir sus conocimientos.

\item[Suficiencia de los botones que indican acciones:] se refiere a que tan
    suficiente es representar determinadas acciones  con un botón debido a
    limitaciones en la tecnología.

\end{description}


\subsection{Métricas}

La métrica utilizada en las preguntas cerradas es la escala de Likert haciendo uso de 
la \emph{Doble estandarización}, como se describe en la sección~\ref{sec:likert}. Esto ayuda 
a determinar los puntos fuertes y débiles de los aspectos evaluados.

Además se utilizan promedios hallados teniendo en cuenta las respuestas de los usuarios 
en cada una de las preguntas cerradas para determinar el nivel de aceptación promedio en cuanto a los 
temas abordados en las preguntas.




\section{Encuesta objetiva}
\label{sec:objetiva}

A fin de obtener información acerca del conocimiento de los alumnos que
utilizaron la solución propuesta y los que no la utilizaron, los cuales
constituyen el grupo de control, se realiza un examen que consta de diez
preguntas.

El examen mide el nivel de conocimiento del alumno sobre los dos temas
simulados, contiene preguntas de nivel básico, medio y avanzado. Las mismas son
formuladas utilizando la lista de competencias básicas que debe tener un alumno
para aprobar la materia \textbf{Enfermería en Urgencias II}, y además están
aprobadas por los profesores de la cátedra.

Cada pregunta tiene el mismo peso, así la puntuación más baja obtenible es $0$, y
la más alta es $10$.

\subsection{Muestra}

El universo cuenta con 124 alumnos, de los cuales 11 son la muestra seleccionada
para el experimento\revisar{Prueba}, y los 113 alumnos restantes son utilizados
como grupo de control.

La utilización de 11 alumnos es suficiente, ya que según estudios presentados en
\cite{nielsen2000}, mientras menos experiencia tengan los sujetos de estudio con
la solución planteada, serán necesarios menos para detectar un gran porcentaje
de errores y fortalezas, y según~\cite{ritch2009}, una base de 10 a 12 es
suficiente para obtener resultados estadísticamente válidos.

\observacion{Prueba vs experimento}

\subsection{Variables}

\observacion{Falta texto de nexo}

Se mide el rendimiento de los alumnos, para ello se utiliza el promedio de las
notas, tanto del conjunto total de alumnos, como de los que participaron del
experimento, y del grupo de control.

Adicionalmente, se busca medir el rendimiento por tema, para así poder
contrastar con los resultados del registro de actividades, que se explica en la
siguiente sección.

Siendo:

\begin{itemize}
    \item $po_i{_k}$ la respuesta del usuario $i$ a la pregunta $k$
    \item $n$ total de preguntas, es igual a 10
    \item $tc$ total de alumnos en el grupo de control, que es igual a 113.
    \item $t$ total de alumnos, que es 124
    \item $ts$ total de sujetos de estudio, que es igual a 11.
\end{itemize}

Se define el puntaje total de de cada alumno $pto_i$ del alumno $i$ como, 

\begin{equation*}
    pto_i = \sum_{j=1}^n{po_i{_j}}
\end{equation*}

Se define el promedio total de los alumnos, $promtotal$ como:

\begin{equation*}
    promtotal = \frac{\sum_{i=1}^t{pto_i}}{t}
\end{equation*}

Se obtienen los promedios del grupo de control ($promcontrol$) y del grupo de alumno que
participaron en el experimento ($promsujetos$) de la misma manera.

\section{Registro de actividades}
\label{sec:registro}

Las metodologías anteriormente descritas incluyen encuestas que miden
el conocimiento del alumno y su opinión con respecto a la solución propuesta,
para poder formar una opinión válida primero deben experimentar con la misma, para
ello se instala la solución en los dispositivos móviles de los alumnos que forman 
parte de la población objetivo.

La instalación de la solución se lleva a cabo en el \Gls{iab}, y se procede 
a mostrar un vídeo de la simulación, explicar la interfaz y realizar una muestra 
de como desenvolverse en el entorno.

El período de prueba se extiende por 20 días, el mismo no es
\fixme{controlado}{Asistido}, es decir que existen factores que no pueden ser
controlados, como:

\begin{itemize}
    \item Tiempo dedicado a la simulación.
    \item Que todas las acciones provengan del alumno.
    \item Solamente el conocimiento del alumno es puesto a prueba, es decir, no
        se puede controlar que no reciba ayuda externa.
\end{itemize}

Por estos motivos, el uso de la solución propuesta no puede ser considerado
el único factor relacionado con los resultados de la encuesta objetiva
descrita en~\ref{sec:objetiva}.

La solución propuesta almacena información relacionada a la actividad del
usuario, incluyendo cuando y como utiliza las acciones, los pasos que realiza,
el orden y las condiciones de la escena cuando realiza cada acción.

El registro como un todo es enviado cada vez que el usuario desee, este envío
requiere una conexión a internet por ello no es automático. Adicionalmente el
último día de la prueba, todos los registros fueron enviados para que sean
analizados.

El registro de actividades ayuda a identificar las  fortalezas y debilidades 
de la solución en cuanto al diseño y utilidad. Sobre todo, ayuda a medir 
el impacto pedagógico al permitir contrarrestar el uso y desempeño del usuario 
con el puntaje obtenido por el mismo en la \emph{Encuesta Objetiva}.

\subsection{Muestra}

La muestra esta conformada por los $11$ alumnos que aceptaron formar parte de 
la prueba y poseían dispositivos móviles que cumplen con los requisitos
mínimos.

\subsection{Variables}

La utilización de la simulación, y el registro de las actividades genera una
gran cantidad de información, los factores que se desean medir están
relacionados a aquellos que pueden ser contrastados con los resultados de la
encuesta objetiva.

La utilización de la simulación nos permite obtener información relevante acerca
de como se utilizo la misma, se definen los criterios a medir:

\begin{description}

\item[Cantidad de partidas:] se define como la cantidad de veces que un alumno
    inicia una escena. 

\item[Tiempo total:] es la suma del tiempo empleado en todas las partidas.

\item[Tiempo total de partidas por usuario y por tipo:] es el tiempo total 
    empleado para jugar las partidas discriminadas por tipo y por usuario.

\item[Cantidad de acciones:] es la cantidad total de acciones realizadas por 
    los usuarios.
 
\item[Cantidad de partidas realizadas por usuario y por tipo:] es el número de 
    partidas jugadas por usuario discriminado por el procedimiento al que 
    corresponde.

\item[Cantidad de usuarios:] es el número de usuarios que utilizaron la solución.
    
\item[Puntuación de las partidas:] Dado el registro de reglas cumplidas en una partida 
    del procedimiento de extracción de sangre o el diagnóstico dado por el usuario 
    en una partida del procedimiento de valoración de la escala de Glasgow, se 
    puede obtener el desempeño del usuario en las partida. Esto puede ser contrastado 
    con la puntuación obtenida por el usuario en la \emph{Encuesta Objetiva}.

%\item[Puntuación por regla cumplida] Las variables definidas
%    en~\ref{sec:objetiva}, pueden ser contrastadas con la puntuación obtenida
%    por los alumnos en la simulación.

\end{description}

\subsection{Métricas}

\begin{description}
\item[Promedio de tiempo por partida:] se obtiene dividiendo el tiempo total empleado 
    en las partidas por el número de partidas.
\item[Promedio de acciones por partida:] se obtiene dividiendo el cantidad total de 
    acciones realizas por el usuario por el número de partidas.
\item[Promedio de partidas por usuario:] se obtiene dividiendo el número total de partidas
    por el número de usuarios que utilizaron la solución.
\item[Total de sesiones jugadas por tipo:] es la suma de la cantidad de partidas jugadas por 
    usuario y tipo.
\item[Total de tiempo jugado por tipo:] es la suma de la cantidad de tiempo empleado en una 
    partida por usuario y tipo.
\item[Promedio de siguientes puntajes por tipo y por usuario:] se obtiene dividiendo la suma 
    de los puntajes obtenidos en cada tipo de escenario por la cantidad de veces que jugó el 
    usuario, a excepción de la primera vez.
\end{description}

Además de las métricas descriptas también se utiliza la correlación de Pearson como se 
explica en \ref{sec:correlacion} para identificar las relaciones entre los datos obtenidos en 
la \emph{Encuesta Objetiva} y los obtenidos en el \emph{Registro de actividades}.


