%! TEX root = ../main.tex

\chapter{Evaluación}
\label{chap:evaluacion}

\observacion{El único detalle que veo es que hay (naturalmente) demasiada
información, discutamos sobre si juntar o no los capítulos}

\observacion{\begin{itemize}
\item Poner descripción al nombre de las pruebas por ejemplo no objetiva o
subjetiva
\item Juntar cap 8 y 9
\item  Poner Juego Serio en el titulo de la tesis en lugar de construccionismo
\item Explicar a quien se le hizo el test, resumen, correlación
\item Borrar los Nos
\item Resultado objetiva en cuanto a que algunos son
\item Ver lo de aleatoriedad (variable) en subjetiva
\end{itemize}}

En este capitulo se definen los \fixme{mecanismos}{Cambiar} utilizados para
evaluar la solución propuesta, estos mecanismos están orientados a la validación
de las hipótesis planteadas durante el desarrollo de la solución, así como la
evaluación de aspectos pedagógicos, de utilidad y de la participación activa del
usuario. Como parte de la evaluación se miden variables relacionadas a los
aspectos mencionados.

La evaluación se divide en cinco partes principales:

\begin{itemize}
    \item \textbf{Prueba de interfaz de usuario:} Es una prueba inicial para
    medir la calidad de la interfaz y la interacción con la misma, esta
    evaluación es realizada con personas no relacionadas al área de enfermería.
    La prueba es llevada a cabo durante el desarrollo de la solución a
    diferencia de las demás, las cuales son realizadas una vez terminada la
    solución.

    \observacion{No queda claro a quien le hacen usar}

    \item \textbf{Encuesta de ubicación:} Es una encuesta acerca del nivel de
    acceso a la tecnología que poseen los alumnos del 4to año del \Gls{iab}, de
    ahora en más \textit{el Universo}\revisar{Población?}, esta encuesta sirve
    para definir la muestra.

    \item \textbf{Encuesta Subjetiva:} Es una encuesta realizada a cada sujeto
    de la población que participa haciendo uso de la solución, donde se busca la
    opinión del mismo acerca de la solución y factores relacionados a la misma. 

    \item \textbf{Encuesta Objetiva:} Es un cuestionario que es completado por
    el universo de alumnos, donde se mide el conocimiento de los mismos, se
    utilizan a los \revisar{También participan} alumnos que no forman parte de
    la \fixme{muestra}{}, como grupo de control.

    \item \textbf{Registro de actividades:} Es información almacenada por la
    solución automáticamente, y contiene datos acerca del uso y el desempeño del
    alumno.
        
\end{itemize}

El capitulo define los objetivos de la evaluación, describe brevemente conceptos
transversales a las técnicas utilizadas y luego define las metodologías,
métricas y variables utilizadas en cada parte de la evaluación.

\section{Objetivos}
\label{sec:objetivos}

Los \emph{Registros de actividades} y \emph{Encuesta objetiva} buscan obtener 
información sobre el aprendizaje y la utilización de la solución, mientras que 
la \emph{Encuesta subjetiva} busca obtener información acerca de las fortalezas y 
debilidades de una simulación para el entrenamiento de enfermeros y de la solución 
propuesta.

Se definen los objetivos principales de la evaluación como siguen:

\begin{enumerate}

\item Validar las hipótesis asumidas durante el desarrollo de la
    solución.

\item Evaluar los puntos fuertes y débiles de la solución.

\item Determinar los factores que afectan al uso de herramientas similares para
    apoyo a profesionales de enfermería
    
\item Determinar el nivel de aceptación de la solución.

\item Evaluar la utilización de la solución, y el progreso de los usuarios.

\item Identificar la influencia de la utilización de la solución en el ámbito
    pedagógico.

\item Determinar correlaciones entre variables estudiadas, a fin de determinar
    la influencia de la utilización de la solución.
    
\end{enumerate}

\observacion{Mejorar el 1 a 1 con los objetivos}

%! TEX root = ../main.tex
%! TEX root = ../main.tex

\section{Métricas generales}

Existen métricas que son usadas por más de un experimento\revisar{Ver el termino
    correcto}, a continuación se describen estas métricas:

\subsection{Escala de Likert}
\label{sec:likert}

Para la valoración de las variables medidas se utiliza la escala de
Likert\cite{Allen:2007} de 7 valores posibles. La escala de Likert es utilizada
para permitir a las personas indicar cuánto están de acuerdo o en desacuerdo con
respecto a ciertos puntos. Los valores utilizados, son:

\begin{enumerate}
    \item Totalmente en desacuerdo
    \item En desacuerdo
    \item Parcialmente en desacuerdo
    \item Neutral
    \item Parcialmente de acuerdo
    \item De acuerdo
    \item Totalmente de acuerdo
\end{enumerate}

Una vez valoradas y registradas todas las respuestas y con el objetivo de
eliminar las tendencias en la forma en la que son completadas las
encuestas\cite{Fischer2010} se utiliza el método de Doble Estandarización
recomendado en~\cite{Pagolu2011}. Este método, consiste en dos
estandarizaciones, la primera por fila, que en este caso representa a los
individuos y la segunda por columna donde cada columna representa una de las
diferentes preguntas de la encuesta.

Siendo:
\begin{itemize}
	\item $\min_i$ la respuesta de menor valor del usuario $i$.
	\item $\max_i$ la respuesta de mayor valor del usuario $i$.
\end{itemize}

Para cada respuesta $s$ del usuario $i$, el valor ajustado, por la primera 
normalización, $s_1$ se define como:

\begin{equation*}
s_1{_i}=\frac{s-\min_i}{\max_i-\min_i}
\end{equation*}

Y luego siendo:
\begin{itemize}
	\item $groupmin_i$ la respuesta ajustada de menor valor en el grupo $i$.
	\item $groupmax_i$ la respuesta ajustada de mayor valor en el grupo $i$
\end{itemize}

Para cada respuesta ajustada $s_1{_i}$ del usuario $i$, el valor ajustado $sa_i$ se
define como:	

\begin{equation*}
sa_i=\frac{s_{1_i}-groupmin_i}{groupmax_i-groupmin_i}
\end{equation*}

Obteniendo así un valor normalizado, tanto por individuo, como por pregunta, en
el rango $0$ y $1$.

Para la valoración absoluta de cada  item se utiliza la media de cada columna o
respuesta a una pregunta de la encuesta.

Siendo:
\begin{itemize} 
\item $r_{k_i}$ la respuesta del usuario $i$ a la pregunta $k$.
\item $t_k$ la cantidad total de usuarios que respondieron la pregunta $k$.
\end{itemize}

El puntaje promedio de cada pregunta o item evaluado  $p_k$ en la encuesta se
define como:

\begin{equation*}
p_k = \frac{\sum_{i=1}^n{r_{k_i}}}{t_k}
\end{equation*}

\subsubsection{Manejo de información faltante}
\label{sec:informacion_faltante}

\observacion{Falta mejorar}
En toda encuesta pueden existir preguntas que no sean respondidas, y existen
tres posibles formas de categorizar el patrón de ocurrencia de la falta de
respuestas\cite{leite2010performance,tsikriktsis2005review}:

\begin{description}
    \item[Información faltante completamente aleatoria] Cuando la información
        faltante es independiente de la variable medida y de otras variables.
    \item[Información faltante aleatoria] Cuando la información faltante depende
        de otras variables, pero no de la variable en sí. 
    \item[Información faltante no aleatoria] Cuando hay una relación entre la
        información faltante y el valor de la variable.
\end{description}

Existen tres mecanismos\revisar{No repitan}\cite{tsikriktsis2005review}
principales para lidiar con información faltante, eliminación, reemplazo, y
procedimientos basados en modelo.~\cite{tsikriktsis2005review} recomienda
utilizar un mecanismo de reemplazo para escalas del tipo Likert.

Las técnicas de reemplazo se clasifican en tres grandes
grupos\cite{tsikriktsis2005review}:
\begin{enumerate*}[label=\itshape\alph*\upshape.]
\item basadas en el promedio,
\item basadas en regresión, y,
\item imputación \emph{hot deck}.
\end{enumerate*}

La sustitución basada por promedio, se divide nuevamente en tres
grupos\cite{tsikriktsis2005review}; promedio
\begin{enumerate*}[label=\itshape\alph*\upshape.]
\item total,
\item del subgrupo, y,
\item por caso.
\end{enumerate*}

La sustitución del promedio total se realiza obteniendo el promedio de todas las
respuestas de esta pregunta, la sustitución de subgrupo es similar, solo que se
limita a aquellos sujetos del mismo subgrupo del sujeto que no respondió, y
finalmente, la sustitución por caso, es el promedio de las respuestas válidas
del sujeto.

\subsection{Correlación de variables aleatorias}
\label{sec:correlacion}

\observacion{Falta un mini parrafo que explique en forma general que es la
    correlación y después mencionar a pearson}

La correlación de Pearson\cite{BoslaughStatistics2008} mide la relación que
existe entre dos variables, $X$ e $Y$, el mismo esta comprendido entre $-1$ y
$1$, en su punto más bajo ($-1$) indica una de las dos variables crece mientras
la otra decrece, y en su punto más alto ($1$), indica que ambas crecen o
decrecen conjuntamente, el valor $0$, indica que no existe una relación entre
ambas variables.

\replantear{\cite{norman2010likert} menciona que la misma puede ser utilizada
    para variables medidas con la escala de Likert, aún cuando la misma es
    utilizada normalmente para variables cuantitativas.}


\section{Interfaz de usuario}
\label{sec:interfaz}

Durante el desarrollo de la solución se realizó una prueba para evaluar la 
interfaz de usuario, específicamente buscando la retroalimentación de usuarios 
acostumbrados a tecnología similar a la utilizada en la solución.

Esta prueba ayuda en el proceso de diseño e implementación de la solución con 
las características mencionadas en los objetivos del trabajo y acorde a los 
requerimientos. De esta manera se pueden identificar los aspectos que deben 
ser mejorados.

La prueba consta de dos partes importantes involucradas en la recolección
de datos para su posterior análisis. Estas partes son las siguientes:

\begin{description}

\item[Simulación:] Luego de una explicación acerca de las funciones y manejos
    generales de la solución por parte de los encargados de la prueba, cada usuario
    completa una tarea que consiste en realizar el procedimiento de extracción de 
    muestra de sangre, como ayuda, recibe una hoja con una lista de todos los pasos 
    necesarios para llevar a cabo el procedimiento utilizando la solución.
    	
    Las simulaciones son grabadas mediante programas de captura de pantalla, así
    como por detectores de eventos táctiles.
    	
\item[Encuesta:] Posteriormente se le provee una encuesta a cada
    usuario la cuál es utilizada para obtener una idea general acerca de la
    calidad de la simulación según la percepción de los mismos. Esta encuesta 
    contiene preguntas que son medidas mediante la escala de tipo Likert. 

\end{description} 

\subsection{Muestra}

La prueba de interfaz de usuario se realiza con alumnos de la carrera de
Ingeniería en Informática de la \Gls{fpuna}, sin experiencia previa tanto con la
solución como con los procedimientos simulados, pero sí familiarizados con la
utilización de dispositivos móviles. La muestra no requiere de sujetos que sean
parte del \emph{Universo} ya que sólo está orientada a mejorar aspectos de
interfaz de usuario y no el contenido de la solución, además se considera que la
muestra puede brindar una evaluación más crítica debido a su familiarización con
interfaces similares a la de la solución.

El número de muestras tomadas fue 8, ya que según~\cite{nielsen2000} son
necesarios al menos $5$ participantes para poder obtener resultados
significativos en una prueba de usabilidad. Además,~\cite{ritch2009} asegura que
la teoría de~\cite{nielsen2000} es verdadera especialmente para pruebas simples. 

Se fundamenta el número de participantes, y que es una prueba sencilla, por que:

\begin{itemize}

\item La prueba que se está ejecutando es sencilla ya no debería tomar más de $10$
    minutos.

\item Se busca solamente obtener información acerca de la interfaz, y no el
    funcionamiento en sí de la simulación, pues los usuarios no son expertos en
    el área y no tienen conocimiento acerca de qué debería pasar. 

\item No se busca medir el aprendizaje del usuario en temas no relacionados a la
    interfaz, es decir, no se mide el aprendizaje del usuario en el tema
    simulado.

\item El procedimiento esta bien definido y los pasos necesarios están a
    disposición del usuario en todo momento.

\end{itemize}

\subsection{Variables}
\label{sec:evaluacion_interfaz_variables}

Antes de definir las variables, se deben primero definir los conceptos 
relacionados a los tipos de acciones que pueden realizarse sobre el paciente 
virtual en la solución, los mismos son:

\begin{itemize}
\item \textbf{Acción por menú contextual:} se refiere a las acciones que el usuario 
    puede realizar utilizando el menú contextual que aparece sobre cada una de las
    herramientas disponibles en la solución.
\item \textbf{Acción por menú de la \Gls{gui}:} se refiere a las 
    acciones que el usuario puede realizar seleccionando una opción en los menús 
    principales que presenta la interfaz.
\item \textbf{Acción por herramienta:} se refiere a las actividades que el usuario 
    puede realizar cuando tiene seleccionada una herramienta y que no involucre el 
    uso del menú contextual.
\end{itemize}


Las variables medidas durante las simulaciones y realización de la tarea son las
siguientes:

\begin{itemize}

\item \textbf{Tiempo de realización de la primera acción por menú contextual:} 
    cuanto tiempo le toma al usuario realizar una acción por menú contextual la 
    primera vez.

\item \textbf{Tiempo de realización de la primera acción por \Gls{gui}:} cuanto 
    tiempo le toma al usuario realizar una acción por menú de 
    interfaz gráfica de usuario la primera vez.
    
\item \textbf{Tiempo de realización de la primera acción por herramienta:} cuanto 
    tiempo le toma al usuario realizar una acción por herramienta la primera vez.
    
\item \textbf{Tiempo de realización de las siguientes acciones por menú contextual:} 
    cuanto tiempo le toma al usuario realizar una acción por menú 
    contextual las siguientes veces.

\item \textbf{Tiempo de realización de las siguientes acciones por \Gls{gui}:} 
    cuanto tiempo le toma al usuario realizar una acción 
    por interfaz gráfica de usuario las siguientes veces.

\item \textbf{Tiempo de realización de las siguientes acciones por herramienta:} 
    cuanto tiempo le toma al usuario realizar una acción por herramienta 
    las siguientes veces.

\item \textbf{Tiempo total:} se refiere al tiempo empleado por el usuario para 
    completar la tarea asignada.

\item \textbf{Número de pasos realizados:} cantidad de pasos requeridos en la tarea 
    que son realizados por el usuario en la simulación. 

\item \textbf{Cantidad de movimientos espaciales por tipo:} número de veces en que se 
    modifica el estado de la cámara para realizar las acciones deseadas agrupados por 
    tipo (desplazamiento, acercamiento/desplazamiento).

\end{itemize}

En cuanto a la encuesta, las siguientes son las variables que fueron consideradas 
y medidas:

\begin{itemize}

\item \textbf{Calidad gráfica:} realismo y calidad de los modelos utilizados.

\item \textbf{Interacción:} desenvolvimiento en el entorno y utilización del 
    hardware.

\item \textbf{Interacción con objetos:} utilización errónea de objetos.

\item \textbf{Características del entorno:} realismo del escenario y de los 
    objetos utilizados.

\item \textbf{Usabilidad de la interfaz:} facilidad de uso de las opciones 
    proveídas por la interfaz.

\item \textbf{Integración con el hardware:} facilidad de uso de la solución con 
    un dispositivo móvil. 

\end{itemize}

\subsection{Métricas}

Para el análisis de la encuesta realizada a los usuarios, se utiliza la escala de Likert
con la \emph{Doble estandarización} explicada en~\ref{sec:likert}, y en el análisis de la 
interacción del usuario con la solución se utilizan las grabaciones registradas durante la 
prueba.

Haciendo uso de las variables descriptas anteriormente, las métricas
utilizadas son las siguientes:

\begin{itemize}
    
\item \textbf{Tiempo promedio de realización de las siguientes acciones por menú contextual:} 
    se obtiene dividiendo la cantidad total de tiempo empleado en realizar acciones por menú 
    contextual por el número de veces que se realizaron esas acciones, a excepción de la primera 
    vez. 
    
\item \textbf{Tiempo promedio de realización de las siguientes acciones por \Gls{gui}:} 
    se obtiene dividiendo la cantidad total de tiempo empleado en realizar acciones por \Gls{gui} 
    por el número de veces que se realizaron esas acciones, a excepción de la primera 
    vez. 
    
\item \textbf{Tiempo promedio de realización de las siguientes acciones por herramienta:} 
    se obtiene dividiendo la cantidad total de tiempo empleado en realizar acciones por 
    herramienta por el número de veces que se realizaron esas acciones, a excepción de la 
    primera vez. 
    
\item \textbf{Promedio de pasos correctos:} se obtiene dividiendo la cantidad de 
    pasos requeridos realizados por los usuarios sobre la cantidad de pasos requeridos. 
    
\item \textbf{Promedio de movimientos por tipo:} se obtiene dividiendo el número de 
    movimientos que fueron realizados agrupados por tipo (desplazamiento, acercamiento/
    desplazamiento) por la cantidad de usuarios.
    
\item \textbf{Promedio del tiempo total:} se obtiene dividiendo el tiempo total empleado 
    por los usuario para completar la tarea asignada por el número de usuarios.

\end{itemize}


%! TEX root = ../main.tex

\section{Encuesta de ubicación}
\label{sec:ubicacion}

\observacion{No repitan (se refiere al obtener información)}

A fin de obtener información acerca del nivel de acceso  de los alumnos a la
tecnología, se realiza una encuesta que cuenta con diez preguntas, las cuales
buscan obtener información acerca del modelo de dispositivo móvil, el acceso a
Internet, y la predisposición de cada alumno a ayudar en el experimento.

En el año $2014$, el \Gls{iab} cuenta con $124$ alumnos en el cuarto año distribuidos en
tres secciones, el cual es considerado el Universo. De los 124, 93 de
ellos estuvieron dispuestos a participar de la prueba y completaron la encuesta.

Con los resultados de la encuesta de ubicación tecnológica, se seleccionan
aquellos alumnos que posean dispositivos móviles que superan o igualan las
especificaciones.


\subsection{Variables}

Se definen $2$ factores necesarios para que un alumno pueda ser considerado como
sujeto de prueba, el primero es la predisposición del mismo a participar de la
prueba y el segundo es que posea un dispositivo móvil que supere los requisitos
mínimos.

Los requisitos mínimos para que la solución tenga un desempeño que garantice una
experiencia fluida a la hora de utilizarla son:

\begin{itemize}
    \item Sistema Operativo Android $4.0$ o superior
    \item Memoria ram de $512$MB o superior.
    \item Velocidad de procesador de $800$ GHz o superior.
    \item \Gls{gpu} Mali 400 o superior.
    \item Conexión frecuente a internet.
\end{itemize}

El conjunto de funcionalidades utilizadas por la solución requiere un \Gls{api}
Android de nivel $14$\cite{android:api} o superior, lo cual corresponde a un
sistema operativo Android $4.0$.

Los requisitos de \textit{hardware} mencionados, son requeridos por las
características de la simulación, y la plataforma utilizada, es requerida una
\Gls{gpu} por los gráficos en tres dimensiones.

La conexión a Internet es requerida, pues los registros de actividad de cada
dispositivo deben ser enviados y almacenados para su posterior interpretación y
análisis.

%! TEX root = ../main.tex
\section{Encuesta Subjetiva}
\label{sec:res_subjetiva}
\observacion{Describir mejor la \enquote{Prueba de opinion}}

La información recogida por la encuesta muestra que hay datos faltantes, como se
explico en~\ref{sec:informacion_faltante}, esta información faltante es
completamente aleatoria en relación a la variable medida y a las demás
variables, de hecho, una sola encuesta tiene información faltante, así, se
establece que el tipo de información faltante es \emph{Información faltante
    completamente aleatoria}.

En su resumen de las diferentes técnicas y cuando se deben utilizar,
\cite{tsikriktsis2005review}, recomienda la utilización de la sustitución basada
en promedio por caso. Así, se completan los valores faltantes con el promedio de
respuestas completadas por el usuario.

\subsection{Resultados}
\label{sec:res_subjetiva}

Se presentan a continuación los resultados de las encuestas, agrupados por los
factores definidos en~\ref{sec:variables}.

La tabla~\ref{tab:subjetiva_conformidad_exploracion} \fixme{nos muestra}{tiempo}
las respuestas de los alumnos a las preguntas relacionadas al factor
exploración, son cuatro preguntas, las cuales fueron descritas
en~\ref{sec:sub_exploracion}. 
\observacion{Revisar bien los tiempos}
\observacion{En este punto uno ya se olvida de la escala}
\observacion{Algo que resaltar de todas estas tablas?}

\begin{table}[H]
\centering
\begin{tabular}{@{} *{5}{r} @{}}
\toprule
& \multicolumn{4}{c}{Exploración} \\
\cmidrule(lr){2-5}
Alumno &
\parbox{2.5cm}{Facilidad de uso}  &
\parbox{3cm}{Funciones realizadas por los elementos del juego} &
\parbox{3cm}{Aleatoriedad para afianzar conocimientos} &
\parbox{2.5cm}{Aleatoriedad para representar realismo} \\
\midrule
1         & 2   & 6   & 5   & 6  \\
2         & 6   & 6   & 4   & 6  \\
3         & 3   & 3   & 5   & 5  \\
4         & 6   & 6   & 6   & 6  \\
5         & 6   & 6   & 2   & 5  \\
6         & 6   & 6   & 6   & 6  \\
7         & 7   & 7   & 7   & 7  \\
8         & 6   & 6   & 7   & 7  \\
9         & 5   & 7   & 7   & 7  \\
10        & 6   & 7   & 6   & 6  \\
11        & 7   & 6   & 7   & 6  \\
\midrule
\textbf{Promedio}  & \textbf{5}   & \textbf{6}   & \textbf{6}   & \textbf{6} \\
\bottomrule
\end{tabular}
\caption{Resultados de la encuesta subjetiva relacionados al factor Exploración}
\label{tab:subjetiva_conformidad_exploracion}
\end{table}

La tabla~\ref{tab:subjetiva_conformidad_representacion} agrupa las respuestas de
los alumnos según la calidad de presentación, son cinco preguntas, las cuales
fueron descritas en~\ref{sec:sub_representacion}. 

\begin{table}[H]
\centering
\begin{tabular}{@{} *{6}{r} @{}}
\toprule
& \multicolumn{5}{c}{Representación} \\
\cmidrule(lr){2-6}
& & \multicolumn{3}{c}{Respuestas del paciente} & \\
\cmidrule(lr){3-5}
Alumno &
\parbox{2.5cm}{Acciones con las herramientas} &
\parbox{2.5cm}{Movimientos oculares del paciente} &
\parbox{2.5cm}{Reacción verbal del paciente} &
\parbox{2.5cm}{Movimientos motrices del paciente} &
\parbox{2.5cm}{Distinción entre los estados del paciente} \\
\midrule
1  & 6 & 6 & 2 & 5 & 2  \\
2  & 4 & 5 & 5 & 6 & 4  \\
3  & 5 & 3 & 3 & 3 & 3  \\
4  & 6 & 5 & 2 & 4 & 2  \\
5  & 2 & 2 & 6 & 6 & 6  \\
6  & 6 & 4 & 6 & 6 & 6  \\
7  & 7 & 6 & 5 & 7 & 5  \\
8  & 6 & 7 & 7 & 7 & 5  \\
9  & 5 & 6 & 2 & 7 & 6  \\
10 & 6 & 4 & 4 & 4 & 5  \\
11 & 6 & 4 & 6 & 6 & 5  \\
\midrule
\textbf{Promedio}  & \textbf{5} & \textbf{5} & \textbf{4} & \textbf{6} & \textbf{4} \\
\bottomrule
\end{tabular}
\caption{Resultados de la encuesta subjetiva relacionados al factor
    Representación}
\label{tab:subjetiva_conformidad_representacion}
\end{table}

La tabla~\ref{tab:subjetiva_conformidad_motivacion} muestra las respuestas de
los alumnos a las preguntas relacionadas al factor \textit{Motivación}, son
cinco preguntas, las cuales fueron descritas en~\ref{sec:sub_motivacion}. 

\begin{table}[H]
\centering
\begin{tabular}{@{} *{5}{r} @{}}
\toprule
& \multicolumn{4}{c}{Motivación} \\
\cmidrule(lr){2-5}
Alumno &
\parbox{2.5cm}{Importancia del puntaje} &
\parbox{3cm}{Socialización de los puntajes} &
\parbox{3cm}{Medición del tiempo} &
\parbox{2.5cm}{Motivación del puntaje} \\
\midrule
1  & 6 & 4 & 4 & 7  \\
2  & 7 & 4 & 6 & 6  \\
3  & 6 & 6 & 5 & 6  \\
4  & 1 & 4 & 6 & 1  \\
5  & 2 & 2 & 7 & 7  \\
6  & 6 & 5 & 4 & 6  \\
7  & 7 & 7 & 6 & 7  \\
8  & 7 & 7 & 7 & 7  \\
9  & 7 & 7 & 7 & 7  \\
10 & 7 & 4 & 5 & 7  \\
11 & 5 & 4 & 5 & 6  \\
\midrule
\textbf{Promedio}  & \textbf{6}   & \textbf{5}   & \textbf{6}   & \textbf{6} \\
\bottomrule
\end{tabular}
\caption{Resultados de la encuesta subjetiva relacionados al factor Motivación}
\label{tab:subjetiva_conformidad_motivacion}
\end{table}

La tabla~\ref{tab:subjetiva_conformidad_inmersion} muestra las respuestas de
los alumnos a las preguntas relacionadas al factor \textit{Inmersión}, son
cinco preguntas, las cuales fueron descritas en~\ref{sec:sub_inmersion}. 

\begin{table}[H]
\centering
\begin{tabular}{@{} *{6}{r} @{}}
\toprule
& \multicolumn{5}{c}{Inmersión} \\
\cmidrule(lr){2-6}
Alumno &
\parbox{2.5cm}{Realismo a través de ordenes verbales} &
\parbox{2.5cm}{Escenografía para entrar en ambiente} &
\parbox{2.5cm}{Gráficos en tres dimensiones para entender el entorno} &
\parbox{2.5cm}{Simulación como herramienta} &
\parbox{2.5cm}{Juegos cortos como ayuda para la repetición} \\
\midrule
1  & 4 & 6 & 4 & 5 & 3  \\
2  & 6 & 6 & 6 & 6 & 6  \\
3  & 6 & 6 & 6 & 5 & 6  \\
4  & 4 & 6 & 7 & 5 & 6  \\
5  & 6 & 6 & 5 & 6 & 6  \\
6  & 6 & 6 & 6 & 4 & 4  \\
7  & 7 & 7 & 7 & 7 & 7  \\
8  & 6 & 7 & 7 & 7 & 7  \\
9  & 6 & 7 & 7 & 7 & 7  \\
10 & 6 & 3 & 4 & 6 & 6  \\
11 & 5 & 3 & 5 & 5 & 4  \\
\midrule
\textbf{Promedio}  & \textbf{6} & \textbf{6} & \textbf{6} & \textbf{6} & \textbf{6} \\
\bottomrule
\end{tabular}
\caption{Resultados de la encuesta subjetiva relacionados al factor Inmersión}
\label{tab:subjetiva_conformidad_inmersion}
\end{table}

La tabla~\ref{tab:subjetiva_conformidad_utilidad} agrupa las respuestas de los
alumnos según la utilidad de la solución, son tres preguntas, las cuales fueron
descritas en~\ref{sec:sub_utilidad}. 


\begin{table}[H]
\centering
\begin{tabular}{@{} *{6}{r} @{}}
\toprule
& \multicolumn{3}{c}{Utilidad} \\
\cmidrule(lr){2-4}
Alumno &
\parbox{4cm}{Interacción con el paciente} &
\parbox{4cm}{Complementar el estudio en clase y laboratorio} &
\parbox{4cm}{Proveedor de facilidades para el estudio} \\
\midrule
1  & 7 & 5 & 7  \\
2  & 6 & 6 & 6  \\
3  & 6 & 6 & 6  \\
4  & 2 & 6 & 6  \\
5  & 2 & 6 & 6  \\
6  & 6 & 6 & 6  \\
7  & 7 & 6 & 7  \\
8  & 5 & 6 & 7  \\
9  & 7 & 7 & 7  \\
10 & 1 & 7 & 7  \\
11 & 6 & 4 & 5  \\
\midrule
\textbf{Promedio}  & \textbf{5} & \textbf{6} & \textbf{6} \\
\bottomrule
\end{tabular}
\caption{Resultados de la encuesta subjetiva relacionados al factor Utilidad}
\label{tab:subjetiva_conformidad_utilidad}
\end{table}

La tabla~\ref{tab:subjetiva_conformidad_retroalimentacion} agrupa las respuestas
de los alumnos según la calidad de retroalimentación, son tres preguntas, las
cuales fueron descritas en~\ref{sec:sub_retroalimentacion}. 

\begin{table}[H]
\centering
\begin{tabular}{@{} *{4}{r} @{}}
\toprule
& \multicolumn{3}{c}{Retroalimentación} \\
\cmidrule(lr){2-4}
Alumno &
\parbox{4cm}{Representación iconográfica de conceptos y acciones en la \Gls{gui}}  &
\parbox{4cm}{Retroalimentación suficiente respecto a los pasos realizados} &
\parbox{4cm}{Detalles de los pasos realizados incorrectamente} \\
\midrule
1  & 3 & 2 & 7  \\
2  & 5 & 4 & 6  \\
3  & 3 & 6 & 6  \\
4  & 6 & 6 & 6  \\
5  & 6 & 1 & 6  \\
6  & 2 & 6 & 6  \\
7  & 6 & 7 & 7  \\
8  & 6 & 6 & 7  \\
9  & 6 & 6 & 7  \\
10 & 5 & 4 & 6  \\
11 & 4 & 5 & 6  \\
\midrule
\textbf{Promedio}  & \textbf{5} & \textbf{5} & \textbf{6} \\
\bottomrule
\end{tabular}
\caption{Resultados de la encuesta subjetiva relacionados al factor
    Retroalimentación}
\label{tab:subjetiva_conformidad_retroalimentacion}
\end{table}

La tabla~\ref{tab:subjetiva_conformidad_pedagogia} agrupa las respuestas de los
alumnos según el factor pedagógico, son tres preguntas, las cuales fueron
descritas en~\ref{sec:sub_pedagogia}. 

\observacion{Habrá que replantear algunos nombres (falta de pistas como)}
\begin{table}[H]
\centering
\begin{tabular}{@{} *{4}{r} @{}}
\toprule
& \multicolumn{3}{c}{Pedagogía} \\
\cmidrule(lr){2-4}
Alumno &
\parbox{4cm}{Suficiencia de los botones que indican acciones} &
\parbox{4cm}{Falta de pistas como ayuda al aprendizaje} &
\parbox{4cm}{Potencial para comprender el procedimiento} \\
\midrule
1  & 6 & 6 & 6  \\
2  & 6 & 6 & 7  \\
3  & 4 & 6 & 6  \\
4  & 6 & 7 & 6  \\
5  & 7 & 5 & 6  \\
6  & 4 & 4 & 6  \\
7  & 7 & 6 & 7  \\
8  & 6 & 7 & 7  \\
9  & 7 & 7 & 7  \\
10 & 6 & 7 & 7  \\
11 & 5 & 6 & 5  \\
\midrule
\textbf{Promedio}  & \textbf{6} & \textbf{6} & \textbf{6} \\
\bottomrule
\end{tabular}
\caption{Resultados de la encuesta subjetiva relacionados al factor Pedagogía}
\label{tab:subjetiva_conformidad_pedagogia}
\end{table}

\subsection{Agrupamiento de datos}

Los resultados se resumen en la tabla~\ref{tab:subjetiva_conformidad_resumen},
se muestra el número de alumno para identificar a un alumno y el promedio de sus
respuestas en la encuesta, se muestra el promedio de las mismas.

\observacion{Retroalimentación esta marcado con un circulo}

\begin{table}[H]
\begin{tabular}{llllllllr}
\toprule
\textbf{\shortstack{Número de \\alumno}}         &
\begin{sideways}\textbf{Motivación}                    \end{sideways}        &
\begin{sideways}\textbf{Exploración}                     \end{sideways}        &
\begin{sideways}\textbf{Inmersión}                       \end{sideways}        &
\begin{sideways}\textbf{Pedagogía}                       \end{sideways}        &
\begin{sideways}\textbf{Representación}                  \end{sideways}        &
\begin{sideways}\textbf{Retroalimentación}               \end{sideways}        &
\begin{sideways}\textbf{Utilidad}                        \end{sideways}        &
\textbf{\shortstack{Promedio\\de respuestas}}\\
\midrule
1              & 5 & 5 & 4 & 6 & 4 & 4 & 6 & 5 \\
2              & 6 & 6 & 6 & 6 & 5 & 5 & 6 & 6 \\
3              & 4 & 6 & 6 & 5 & 3 & 5 & 6 & 5 \\
4              & 6 & 3 & 6 & 6 & 4 & 6 & 5 & 5 \\
5              & 5 & 5 & 6 & 6 & 4 & 4 & 5 & 5 \\
6              & 6 & 5 & 5 & 5 & 6 & 5 & 6 & 5 \\
7              & 7 & 7 & 7 & 7 & 6 & 7 & 7 & 7 \\
8              & 7 & 7 & 7 & 7 & 6 & 6 & 6 & 7 \\
9              & 7 & 7 & 7 & 7 & 5 & 6 & 7 & 6 \\
10             & 6 & 6 & 5 & 7 & 5 & 5 & 5 & 5 \\
11             & 7 & 5 & 4 & 5 & 5 & 5 & 5 & 5 \\
\midrule
Promedio Total & 6 & 6 & 6 & 6 & 5 & 5 & 6 & 6 \\
\bottomrule
\end{tabular}
\caption{Resultados de la encuesta subjetiva}
\label{tab:subjetiva_conformidad_resumen}
\end{table}

Se observa que el el puntaje más bajo en el promedio final es 5 que significa
\textit{Parcialmente de acuerdo}, y el más alto es 7, que significa
\textit{Totalmente de acuerdo}, se observa además el puntaje 6, que significa
\textit{De acuerdo}. 


Como se explica en la sección~\ref{sec:likert}, estos resultados están sujetos a
tendencias, para ello se aplica el método de doble
estandarización\cite{Pagolu2011}.

Con el resultado final de la estandarización diferenciamos cuales son los puntos
fuertes y cuales los puntos débiles de la solución propuesta con respecto a las
respuestas dadas por los usuarios. Estos valores son relativos a las respuestas
originales dadas en la encuesta, los resultados se muestran en la
tabla~\ref{tab:subjetiva_conformidad_corregida}.

\begin{table}[H]
\centering
\begin{tabular}{lrrrrrrrr}
\toprule
\textbf{\shortstack{Número de \\alumno}}                                &
\begin{sideways}\textbf{Motivación}                    \end{sideways} &
\begin{sideways}\textbf{Exploración}                     \end{sideways} &
\begin{sideways}\textbf{Inmersión}                       \end{sideways} &
\begin{sideways}\textbf{Pedagogía}                       \end{sideways} &
\begin{sideways}\textbf{Representación}                  \end{sideways} &
\begin{sideways}\textbf{Retroalimentación}               \end{sideways} &
\begin{sideways}\textbf{Utilidad}                        \end{sideways} &
\textbf{\shortstack{Promedio\\de respuestas}}\\
\midrule
1              & 0.45 & 0.55 & 0.20 & 0.63 & 0.44 & 0.41 & 0.82 & 0.47 \\
2              & 0.33 & 0.53 & 0.49 & 0.61 & 0.27 & 0.13 & 0.52 & 0.41 \\
3              & 0.17 & 0.86 & 0.87 & 0.67 & 0.13 & 0.67 & 1.00 & 0.60 \\
4              & 0.75 & 0.31 & 0.63 & 0.81 & 0.47 & 0.78 & 0.54 & 0.59 \\
5              & 0.46 & 0.58 & 0.69 & 0.67 & 0.57 & 0.50 & 0.54 & 0.58 \\
6              & 1.00 & 0.73 & 0.68 & 0.42 & 0.90 & 0.67 & 1.00 & 0.78 \\
7              & 1.00 & 0.79 & 1.00 & 0.67 & 0.50 & 0.87 & 0.78 & 0.80 \\
8              & 0.75 & 1.00 & 0.83 & 0.75 & 0.70 & 0.70 & 0.44 & 0.75 \\
9              & 0.90 & 1.00 & 0.93 & 1.00 & 0.64 & 0.92 & 1.00 & 0.90 \\
10             & 0.79 & 0.74 & 0.54 & 0.92 & 0.60 & 0.60 & 0.67 & 0.68 \\
11             & 0.75 & 0.42 & 0.08 & 0.25 & 0.60 & 0.35 & 0.25 & 0.39 \\
\midrule
\textbf{Promedio Total} & 0.67 & 0.68 & 0.63 & 0.67 & 0.53 & 0.60 & 0.69 & 0.63 \\
\bottomrule
\end{tabular}
\caption{Resultados de la encuesta subjetiva con doble estandarización}
\label{tab:subjetiva_conformidad_corregida}
\end{table}

Es importante notar que los datos la
tabla~\ref{tab:subjetiva_conformidad_corregida} son relativas a los datos de la
tabla~\ref{tab:subjetiva_conformidad_resumen}, es decir, que la representación
es el punto más débil, aún así, se ve que
en~\ref{tab:subjetiva_conformidad_resumen} que el valor es $5$ de $7$, lo que
indica que es un punto aceptable, y entre los factores analizados es el que
menos aprobación obtuvo.


Con la información obtenida, es posible \emph{Validar las hipótesis asumidas
    durante el desarrollo de la solución}, el cual es uno de los objetivos de
este capítulo. En la tabla~\ref{tab:resultado_resumen_hipotesis} se observa la
opinion de los alumnos con respecto a las hipótesis asumidas
en~\ref{sec:hipotesis}. Se observa una aceptación a las hipótesis asumidas.

\begin{table}[!hbt]
\centering
\begin{tabular}{lcr}
\toprule
Hipótesis                        & Promedio Subjetiva      & Promedio estandarizado \\
\midrule
Comandos de voz con interfaz     & De acuerdo              & $0,55$ \\
Extracción uniforme de elementos & Parcialmente de acuerdo & $0,65$ \\
Acciones de bioseguridad         & De acuerdo              & $0,58$ \\
Representación iconográfica      & Parcialmente de acuerdo & $0,53$ \\
Factores motivadores             & De acuerdo              & $0,65$ \\
Falta de pistas                  & De acuerdo              & $0,61$ \\
\bottomrule
\end{tabular}
\caption{Hipótesis con su aceptación}\label{tab:resultado_resumen_hipotesis}
\end{table}

Adicionalmente, se puede \emph{Evaluar los puntos fuertes y débiles de la
    solución}, utilizando los datos con doble estandarización de la
tabla~\ref{tab:subjetiva_conformidad_corregida}, se crea la
tabla~\ref{tab:resultado_resumen_aspectos_aceptacion}, donde se observa la
apreciación de los usuarios por cada aspecto estudiado.

\begin{table}[!hbt]
\centering
\begin{tabular}{lcr}
\toprule
Factores        & Promedio Subjetiva      & Promedio estandarizado \\
\midrule
Motivación        & De acuerdo              & $0.67$  \\
Exploración       & De acuerdo              & $0.68$  \\
Inmersión         & De acuerdo              & $0.63$  \\
Pedagogía         & De acuerdo              & $0.67$  \\
Representación    & Parcialmente de acuerdo & $0.53$  \\
Retroalimentación & Parcialmente de acuerdo & $0.60$  \\
Utilidad          & De acuerdo              & $0.69$  \\
\bottomrule
\end{tabular}
\caption{Aceptación por aspecto de la solución}
\label{tab:resultado_resumen_aspectos_aceptacion}
\end{table}

Para  obtener una mejor visión de las fortalezas y debilidades de la solución
propuesta, se presenta el gráfico de \emph{kiviat}~\ref{fig:subjetiva_kiviat},
en la misma se puede observar cuales son los puntos débiles de la solución.

\observacion{Pulir la manera en la que hacen referencia a ete tópico, aclarando
que son percepciones desde el punto de vista del usuario}
\begin{figure}[!ht]
\begin{tikzpicture}[label distance=.15cm]
\tkzKiviatDiagram[radial=2,
                    lattice=2, step=2,
                    scale=2.3]%
                {Motivación,
                 Exploración,
                 Inmersión,
                 Pedagogía,
                 Representación,
                 Retroalimentación,
                 Utilidad}
\tkzKiviatLine[thick,
                color=blue,
                mark=ball,
                ball color=red,
                mark size=1pt,opacity=.2, 
                fill=red!20](0.67,0.68,0.63,0.67,0.53,0.60,0.69)
\end{tikzpicture}
\label{fig:subjetiva_kiviat}
\caption{Gráfico de Kiviat de los factores evaluados}
\end{figure}

Se observa que las principales debilidades de la solución son la representación
y la retroalimentación, y las fortalezas la utilidad, pedagogía, exploración, y
la motivación.

\subsection{Preguntas abiertas}
\label{sec:res_subjetiva_abiertas}

En la parte final de la encuesta que completaron los alumnos, que formaron parte
de la prueba, cuenta con preguntas abiertas, donde los alumnos expresaron sus
opiniones sobre los aspectos que rodean al uso de este tipo de soluciones al
aprendizaje de enfermería.


\begin{itemize}
    \item El $100\%$ de los alumnos menciono que este tipo de soluciones son
        beneficiosas para el aprendizaje de procedimientos de enfermería.
    \item El $64\%$ de los alumnos menciono que la principal dificultad para
        utilizar la solución es el factor tiempo.
    \item El $45\%$ de los alumnos menciono que la solución esta completa,
        mientras que el $18\%$ sugirió más elementos e interacción con el
        paciente.
\end{itemize}


Con esta información se puede \emph{determinar el nivel de aceptación de la
    solución}, se observa que el $100\%$ de los alumnos cree que es beneficioso
contar con este tipo de soluciones.

\section{Encuesta objetiva}
\label{sec:objetiva}

A fin de obtener información acerca del conocimiento de los alumnos que utilizaron 
la solución propuesta y los que no la utilizaron, los cuales constituyen el grupo 
de control, se realiza una encuesta que consta de diez preguntas.

La encuesta mide el nivel de conocimiento del alumno sobre los dos temas
simulados, contiene preguntas de nivel básico, medio y avanzado. Las mismas son
formuladas utilizando la lista de competencias básicas que debe tener un alumno
para aprobar la materia \textbf{Enfermería en Urgencias II}. Las preguntas son
verificadas  por los profesores de la cátedra. Cada pregunta tiene el mismo
peso, así la puntuación más baja obtenible es $0$, y la más alta es $10$.

De esta manera se busca evaluar la influencia pedagógica y la utilidad de la 
solución como herramienta de apoyo al aprendizaje.


\subsection{Muestra}
\observacion{Se repite mucho lo de las muestras hay 2 universos nomas?}

El universo cuenta con $124$ alumnos, de los cuales $11$ son la muestra seleccionada
para la prueba, y los $113$ alumnos restantes son utilizados
como grupo de control.

\subsection{Variables}

Se busca medir el puntaje total de los alumnos en la \emph{Encuesta objetiva}. Esto 
se obtiene de la siguiente manera.

Siendo:

\begin{itemize}
    \item $po_i{_k}$ la respuesta del usuario $i$ a la pregunta $k$
    \item $n$ total de preguntas, es igual a 10
    \item $tc$ total de alumnos en el grupo de control, que es igual a 113.
    \item $t$ total de alumnos, que es 124
    \item $ts$ total de sujetos de estudio, que es igual a 11.
\end{itemize}

Se define el puntaje total de de cada alumno $pto_i$ del alumno $i$ como, 

\begin{equation*}
    pto_i = \sum_{j=1}^n{po_i{_j}}
\end{equation*}


\subsection{Métricas}

Como se mencionó, la \emph{Encuesta Objetiva} busca medir el rendimiento de los 
alumnos, para ello se utiliza como métrica principal el promedio de acierto, 
tanto del conjunto total de alumnos, como de los que participaron de la
prueba, y del grupo de control.

Se define el promedio total de los alumnos, $promtotal$ como:

\begin{equation*}
    promtotal = \frac{\sum_{i=1}^t{pto_i}}{t}
\end{equation*}

Se obtienen los promedios del grupo de control ($promcontrol$) y del grupo de alumno que
participaron en la prueba ($promsujetos$) de la misma manera.

\section{Registro de actividades}
\label{sec:registro}

Las metodologías anteriormente descritas incluyen encuestas que miden
el conocimiento del alumno y su opinión con respecto a la solución propuesta,
para poder formar una opinión válida primero deben experimentar con la misma, para
ello se instala la solución en los dispositivos móviles de los alumnos que forman 
parte de la población objetivo.

La instalación de la solución se lleva a cabo en el \Gls{iab}, y se procede 
a mostrar un vídeo de la simulación, explicar la interfaz y realizar una muestra 
de como desenvolverse en el entorno.

El período de prueba se extiende por 20 días, el mismo no es
\fixme{controlado}{Asistido}, es decir que existen factores que no pueden ser
controlados, como:

\begin{itemize}
    \item Tiempo dedicado a la simulación.
    \item Que todas las acciones provengan del alumno.
    \item Solamente el conocimiento del alumno es puesto a prueba, es decir, no
        se puede controlar que no reciba ayuda externa.
\end{itemize}

Por estos motivos, el uso de la solución propuesta no puede ser considerado
el único factor relacionado con los resultados de la encuesta objetiva
descrita en~\ref{sec:objetiva}.

La solución propuesta almacena información relacionada a la actividad del
usuario, incluyendo cuando y como utiliza las acciones, los pasos que realiza,
el orden y las condiciones de la escena cuando realiza cada acción.

El registro como un todo es enviado cada vez que el usuario desee, este envío
requiere una conexión a internet por ello no es automático. Adicionalmente el
último día de la prueba, todos los registros fueron enviados para que sean
analizados.

El registro de actividades ayuda a identificar las  fortalezas y debilidades 
de la solución en cuanto al diseño y utilidad. Sobre todo, ayuda a medir 
el impacto pedagógico al permitir contrarrestar el uso y desempeño del usuario 
con el puntaje obtenido por el mismo en la \emph{Encuesta Objetiva}.

\subsection{Muestra}

La muestra esta conformada por los $11$ alumnos que aceptaron formar parte de 
la prueba y poseían dispositivos móviles que cumplen con los requisitos
mínimos.

\subsection{Variables}

La utilización de la simulación, y el registro de las actividades genera una
gran cantidad de información, los factores que se desean medir están
relacionados a aquellos que pueden ser contrastados con los resultados de la
encuesta objetiva.

La utilización de la simulación nos permite obtener información relevante acerca
de como se utilizo la misma, se definen los criterios a medir:

\begin{description}

\item[Cantidad de partidas:] se define como la cantidad de veces que un alumno
    inicia una escena. 

\item[Tiempo total:] es la suma del tiempo empleado en todas las partidas.

\item[Tiempo total de partidas por usuario y por tipo:] es el tiempo total 
    empleado para jugar las partidas discriminadas por tipo y por usuario.

\item[Cantidad de acciones:] es la cantidad total de acciones realizadas por 
    los usuarios.
 
\item[Cantidad de partidas realizadas por usuario y por tipo:] es el número de 
    partidas jugadas por usuario discriminado por el procedimiento al que 
    corresponde.

\item[Cantidad de usuarios:] es el número de usuarios que utilizaron la solución.
    
\item[Puntuación de las partidas:] Dado el registro de reglas cumplidas en una partida 
    del procedimiento de extracción de sangre o el diagnóstico dado por el usuario 
    en una partida del procedimiento de valoración de la escala de Glasgow, se 
    puede obtener el desempeño del usuario en las partida. Esto puede ser contrastado 
    con la puntuación obtenida por el usuario en la \emph{Encuesta Objetiva}.

%\item[Puntuación por regla cumplida] Las variables definidas
%    en~\ref{sec:objetiva}, pueden ser contrastadas con la puntuación obtenida
%    por los alumnos en la simulación.

\end{description}

\subsection{Métricas}

\begin{description}
\item[Promedio de tiempo por partida:] se obtiene dividiendo el tiempo total empleado 
    en las partidas por el número de partidas.
\item[Promedio de acciones por partida:] se obtiene dividiendo el cantidad total de 
    acciones realizas por el usuario por el número de partidas.
\item[Promedio de partidas por usuario:] se obtiene dividiendo el número total de partidas
    por el número de usuarios que utilizaron la solución.
\item[Total de sesiones jugadas por tipo:] es la suma de la cantidad de partidas jugadas por 
    usuario y tipo.
\item[Total de tiempo jugado por tipo:] es la suma de la cantidad de tiempo empleado en una 
    partida por usuario y tipo.
\item[Promedio de siguientes puntajes por tipo y por usuario:] se obtiene dividiendo la suma 
    de los puntajes obtenidos en cada tipo de escenario por la cantidad de veces que jugó el 
    usuario, a excepción de la primera vez.
\end{description}

Además de las métricas descriptas también se utiliza la correlación de Pearson como se 
explica en \ref{sec:correlacion} para identificar las relaciones entre los datos obtenidos en 
la \emph{Encuesta Objetiva} y los obtenidos en el \emph{Registro de actividades}.


