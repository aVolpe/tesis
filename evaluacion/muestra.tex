\section{Muestra}

Se realizaron cuatro evaluaciones de la aplicación, la primera fue con personas
no relacionadas con el área de enfermería, y las siguientes tres con alumnos del
4to año de Licenciatura en enfermería del \Gls{iab}.

\section{Evaluación inicial de la interfaz de usuario}

La primera fue realizada con alumnos de la carrera de Ingeniería en Informática
de la Facultad Politécnica que pertenece a la Universidad Nacional de Asunción,
sin experiencia previa tanto con la solución como con los procedimientos
simulados, pero si familizarizados con la utilización de dispositivos móviles.

\cite{nielsen2000} recomienda una muestra de 5 personas para pruebas de
usabilidad.~\cite{ritch2009} menciona que con 5 individuos, se encuentran 85\%
de los errores en promedio, y que un grupo de 5 a 10 personas es adecuado para
pruebas de usabilidad sencillas.

Esta prueba no es de gran complejidad, el procedimiento es sencillo y esta
bien definido, se busca determinar que problemas presenta la interfaz, que
impedimentos encuentran usuarios acostumbrados a la tecnología pero no al
procedimiento, por ello se elige una muestra de 8 alumnos.

\subsection{Universo}
\label{subsec:universo}

Para las pruebas con usuarios familiarizados con los procedimientos simulados,
se utilizaron alumnos del 4to año de Licenciatura en Enfermería del \Gls{iab}.

El \Gls{iab} contó en el 2014 con 124 alumnos distribuidos en tres secciones, 
el cual es considerado el Universo, para seleccionar a la población que utilizaría
la simulación se realizo una encuesta de ubicación definida en \ref{subsec:ubicacion}.

Se agruparon los alumnos en diferentes grupos para determinar si sus dispositivos
celulares eran lo suficientemente potentes\todox{Buscar mejor expresión} para 
ejecutar la simulación, los requisitos son:

\begin{itemize}
\item Sistema Operativo Android 4.0 o superior.
\item Memoria ram de 512MB o superior.
\item Velocidad de procesador de 1 GHz o superior.
\item GPU \todox{No se como poner la GPU, hay demasiada variedad}
\end{itemize}

Con los resultados de la encuesta de ubicación tecnológica, se seleccionan 
aquellos alumnos que posean dispositivos móviles que superen las especificaciones, 
se seleccionan un total de 11 estudiantes\todox{Agregar citas a estudios donde 
dice que es suficiente 11 personas}.

Estos 11 alumnos seleccionados, representan la muestra que se utiliza para el 
experimento.


\section{Grupo de control}

El universo cuenta con 124 alumnos, de los cuales 11 son la muestra seleccionada
para el experimento, entonces se utilizan a los 113 alumnos restantes como un 
grupo de control.
