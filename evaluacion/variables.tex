
\section{Variables}
\label{sec:variables}

De acuerdo a los objetivos planteados en la sección~\ref{sec:objetivos}, se
busca describir los factores analizados en las pruebas y las variables
relacionadas a los mismos, las cuales, tienen por objetivo demostrar la validez
de las hipótesis planteadas en este trabajo.

Las variables se presentaran agrupadas en factores, los mismos representan
aquellos aspectos de la aplicación que buscan ser evaluados.

\subsection{Exploración}

Este factor esta relacionado con la característica que posee la aplicación en
cuanto a la oportunidad que brinda al usuario para explorar cada uno de los
elementos del entorno simulado (paciente, herramientas propias del
procedimiento). En este sentido, se busca proveer facilidad de uso, intuitividad
y realismo en cuanto a las acciones y situaciones que se presentan en la
aplicación para que de esta manera, los elementos que componen la aplicación no
representen para el jugador un obstáculo que impida su uso.

Las variables que miden este aspecto son las siguiente:

\begin{description}

\item [Funciones realizadas por los elementos del juego:] se refiere a la
    correctitud con la que una herramienta o elemento del juego representa las
    funciones que el mismo puede realizar en la vida real, en este sentido, se
    evalúa el realismo con el que es representado tal elemento.

\item [Aleatoriedad para afianzar conocimientos:] se refiere al beneficio que
    puede traer el hecho de que el estado del paciente en el juego sea aleatorio
    en cuanto a la posibilidad que esto brinda al jugador para poner a prueba
    sus conocimientos teóricos.

\item [Aleatoriedad para representar realismo:] se refiere al uso de estados
    aleatorios en el paciente para que de esta forma el procedimiento se asemeje
    mas a una situación real.

\item [Facilidad de uso:] se refiere a lo fácil e intuitivo  que puede ser la
    utilización de los elementos del juego.

\end{description}

\subsection{Representación}

Este factor esta relacionado con la calidad y suficiencia con la que se
representan los diferentes objetos que son simulados en la aplicación. La
representación abarca tanto funcionalidad como aspecto del objeto.

De esta manera, se busca permitir al jugador realizar con los objetos las
acciones que requiera para llevar a cabo el procedimiento que se le presente en
la aplicación, y además, representar estos elementos de la mejor manera posible,
de forma realista.

Las variables que miden estos aspectos son las siguientes:

\begin{description}

\item [Movimientos motrices del paciente:] se refiere a la suficiencia de los
    movimientos motrices que realiza el paciente en el escena correspondiente a
    la valoración de la escala de Glasgow.

\item [Movimientos oculares del paciente:] se refiere a la suficiencia de los
    movimientos oculares que realiza el paciente en la escena correspondiente a
    la valoración del escala de Glasgow.

\item [Reacción verbal del paciente:] se refiere a la suficiencia de las
    reacciones o respuestas verbales que realiza el paciente en la escena
    correspondiente a la valoración de la escala de Glasgow.

\item[Acciones suficientes de los elementos o herramientas:] se refiere a si las
    diferentes acciones que pueden realizarse con los elementos o herramientas
    del juego en un determinado procedimiento de enfermería son suficientes para
    ese procedimiento, ya que, debido a las limitaciones de la tecnología estas
    acciones son limitadas.

\item[Distinción entre los estados del paciente:] se refiere a si los diferentes
    estados del paciente son distinguidos correctamente en el procedimiento de
    valoración de la Escala de Glasgow ya que esto es importante para que el
    jugador pueda diagnosticar correctamente al paciente.

\end{description}

\subsection{Gamificacion}

Este factor esta relacionado con la importancia de incluir en la aplicación
aquellas características que son propias de un juego de vídeo convencional. Se
busca conocer el valor de estas características en cuanto a la motivación que
puedan producir en los jugadores tanto para volver a utilizar la aplicación como
para superarse en cada juego.

Las variables que miden estos aspectos son las siguientes:

\begin{description}
   
\item[Puntaje del juego como motivación:] se refiere a que tanto motiva al
    jugador que la aplicación le proporcione un puntaje total al final de cada
    partida para poder mejorar constantemente siendo este puntaje como una
    evaluación final de tod lo que realizo dentro de la partida.

\item[Importancia del puntaje:] se refiere a que tan importante es para un
    jugador que se le proporcione un puntaje total al final de cada partida para
    poder visualizar su rendimiento.

\item[Socialización de los puntajes:] se refiere a si el hecho de que las
    personas del mismo entorno compartan sus puntajes, experiencias y logros en
    las partidas a través de redes sociales pueda ser motivador.

\item[Medición del tiempo como motivación:] se refiere a que tanto motiva al
    jugador que la aplicación le proporcione el tiempo que duro su partida
    siendo este tiempo como una evaluación de su precisión a la hora de realizar
    el procedimiento que se le presente.

\end{description} \subsection{Inmersión}

Este factor esta relacionado con el sentimiento de formar parte de la escena. Es
decir, se trata de evaluar que tanto un jugador puede sentir que realmente se
encuentra dentro del juego para que de este modo el pueda entrar en ambiente
para realizar los procedimientos que se le presenten en sus partidas de juego.

Las variables que miden este aspecto son las siguientes:

\begin{description}

\item[Escenografía para entrar en ambiente:] se refiere a la importancia de la
    escenografía de la partida para que el jugador entre en ambiente para
    realizar el procedimiento que se le presente.

\item[Juegos cortos como ayuda para la repetición:] se refiere a si el hecho de
    que los procedimientos presentados en las partidas sean cortos contribuye a
    repetir las partidas varias veces de seguido entrando en un estado de
    inmersión.

\item[Gráficos en tres dimensiones para entender el entorno:] se refiere a la
    importancia que tiene el uso de gráficos en tres dimensiones para que el
    jugador pueda entender mejor el entorno y las posibles acciones que puede
    realizar.

\item[Realismo a través de ordenes verbales:] se refiere a si el hecho de que la
    aplicación brinde la posibilidad de que aparezca un menú de ordenes verbales
    en el momento en que el jugador habla hace que la acción de dar ordenes
    verbales se asemeje mas a la realidad.

\item[Simulación como herramienta:] se refiere a si la simulación ayuda al
    jugador a sentirse parte del laboratorio, dando cierto realismo a la escena
    que se le presenta.

\end{description}

\subsection{Utilidad}

Este factor esta relacionado con lo útil que puede ser la aplicación como
herramienta de apoyo al proceso de aprendizaje de los estudiantes de enfermería.

Las variables que miden este aspecto son las siguientes:

\begin{description}

\item[Simulación para complementar el estudio en clase y laboratorio:] se
    refiere a que tanto las herramientas alternativas como la simulación pueden
    complementar a los métodos de aprendizaje tradicionales que son el estudio
    en clase y en el laboratorio.

\item[Simulación provee mas facilidades para el estudio:] se refiere a si las
    herramientas alternativas como la simulación proveen mas facilidades para
    poner en practica los conocimientos con respecto a los demás métodos de
    aprendizaje que son los libros, laboratorios,  campo de practicas.

\item[Interacción con el paciente:] se refiere a si el hecho de que el jugador
    pueda interactuar con un paciente que responde a las acciones del jugador es
    mejor que utilizar un maniquí inmóvil como el de los laboratorios de
    practica.

\end{description}

\subsection{Retroalimentación}

Este factor esta relacionado con la importancia de ofrecer al jugador
información acerca de sus logros y errores de manera tal que el pueda estar
consciente de sus puntos fuertes y sus puntos débiles en los diversos
procedimientos que realice en la aplicación.

Las variables que miden este aspecto son las siguientes:

\begin{description}

\item[Importancia de los detalles de los pasos realizados incorrectamente:] se
    refiere a que tan importante es para el jugador que la aplicación no solo le
    diga los pasos que hizo de manera incorrecta sino también las causas por las
    cuales no los realizo correctamente.

\item[Suficiencia de los detalles de los pasos realizados incorrectamente:]se
    refiere a que tan suficiente es para el jugador mostrar justificaciones
    breves acerca de las causas por las que realizo incorrectamente un paso.

\item[Iconos para representar el estado del jugador:] se refiere a la
    suficiencia de mostrar iconos en la interfaz de la aplicación para
    representar el estado actual del jugador.

\end{description}

\subsection{Pedagogía}

Este factor esta relacionado a la utilidad y al beneficio que puede traer la
aplicación para apoyar el aprendizaje del jugador. De esta manera, se busca
obtener la validez real de este tipo de herramientas como aporte al aprendizaje,
proveyendo mas interacción al jugador.

Las variables que miden este aspecto son las siguientes:

\begin{description}

\item[Aplicación para memorizar y comprender el procedimiento:] se refiere a que
    tanto ayuda la aplicación al jugador para entender los procedimientos que se
    le presenten y para memorizar los pasos de cada uno de ellos.

\item[Falta de pistas como ayuda al aprendizaje:] se refiere a que tan efectivo
    resulta no dar pistas al jugador en el momento de realizar un procedimiento
    para que pueda plasmar y medir sus conocimientos.

\item[Suficiencia de los botones que indican acciones:] se refiere a que tan
    suficiente es representar determinadas acciones que deben realizarse con un
    botón debido a limitaciones en la tecnología.

\end{description}
