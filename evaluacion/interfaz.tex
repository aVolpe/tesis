\section{Interfaz de usuario}
\label{sec:interfaz}

Durante el desarrollo de la solución se realizó una prueba para evaluar la 
interfaz de usuario, específicamente buscando la retroalimentación de usuarios 
acostumbrados a tecnología similar a la utilizada en la solución.

Esta prueba ayuda en el proceso de diseño e implementación de la solución con 
las características mencionadas en los objetivos del trabajo y acorde a los 
requerimientos. De esta manera se pueden identificar los aspectos que deben 
ser mejorados.

La prueba consta de dos partes importantes involucradas en la recolección
de datos para su posterior análisis. Estas partes son las siguientes:

\begin{description}

\item[Simulación:] Luego de una explicación acerca de las funciones y manejos
    generales de la solución por parte de los encargados de la prueba, cada usuario
    completa una tarea que consiste en realizar el procedimiento de extracción de 
    muestra de sangre, como ayuda, recibe una hoja con una lista de todos los pasos 
    necesarios para llevar a cabo el procedimiento utilizando la solución.
    	
    Las simulaciones son grabadas mediante programas de captura de pantalla, así
    como por detectores de eventos táctiles.
    	
\item[Encuesta:] Posteriormente se le provee una encuesta a cada
    usuario la cuál es utilizada para obtener una idea general acerca de la
    calidad de la simulación según la percepción de los mismos. Esta encuesta 
    contiene preguntas que son medidas mediante la escala de tipo Likert. 

\end{description} 

\subsection{Muestra}

La prueba de interfaz de usuario se realiza con alumnos de la carrera de
Ingeniería en Informática de la \Gls{fpuna}, sin experiencia previa tanto con la
solución como con los procedimientos simulados, pero sí familiarizados con la
utilización de dispositivos móviles. La muestra no requiere de sujetos que sean
parte del \emph{Universo} ya que sólo está orientada a mejorar aspectos de
interfaz de usuario y no el contenido de la solución, además se considera que la
muestra puede brindar una evaluación más crítica debido a su familiarización con
interfaces similares a la de la solución.

El número de muestras tomadas fue 8, ya que según~\cite{nielsen2000} son
necesarios al menos $5$ participantes para poder obtener resultados
significativos en una prueba de usabilidad. Además,~\cite{ritch2009} asegura que
la teoría de~\cite{nielsen2000} es verdadera especialmente para pruebas simples. 

Se fundamenta el número de participantes, y que es una prueba sencilla, por que:

\begin{itemize}

\item La prueba que se está ejecutando es sencilla ya no debería tomar más de $10$
    minutos.

\item Se busca solamente obtener información acerca de la interfaz, y no el
    funcionamiento en sí de la simulación, pues los usuarios no son expertos en
    el área y no tienen conocimiento acerca de qué debería pasar. 

\item No se busca medir el aprendizaje del usuario en temas no relacionados a la
    interfaz, es decir, no se mide el aprendizaje del usuario en el tema
    simulado.

\item El procedimiento esta bien definido y los pasos necesarios están a
    disposición del usuario en todo momento.

\end{itemize}

\subsection{Variables}
\label{sec:evaluacion_interfaz_variables}

Antes de definir las variables, se deben primero definir los conceptos 
relacionados a los tipos de acciones que pueden realizarse sobre el paciente 
virtual en la solución, los mismos son:

\begin{itemize}
\item \textbf{Acción por menú contextual:} se refiere a las acciones que el usuario 
    puede realizar utilizando el menú contextual que aparece sobre cada una de las
    herramientas disponibles en la solución.
\item \textbf{Acción por menú de la \Gls{gui}:} se refiere a las 
    acciones que el usuario puede realizar seleccionando una opción en los menús 
    principales que presenta la interfaz.
\item \textbf{Acción por herramienta:} se refiere a las actividades que el usuario 
    puede realizar cuando tiene seleccionada una herramienta y que no involucre el 
    uso del menú contextual.
\end{itemize}


Las variables medidas durante las simulaciones y realización de la tarea son las
siguientes:

\begin{itemize}

\item \textbf{Tiempo de realización de la primera acción por menú contextual:} 
    cuanto tiempo le toma al usuario realizar una acción por menú contextual la 
    primera vez.

\item \textbf{Tiempo de realización de la primera acción por \Gls{gui}:} cuanto 
    tiempo le toma al usuario realizar una acción por menú de 
    interfaz gráfica de usuario la primera vez.
    
\item \textbf{Tiempo de realización de la primera acción por herramienta:} cuanto 
    tiempo le toma al usuario realizar una acción por herramienta la primera vez.
    
\item \textbf{Tiempo de realización de las siguientes acciones por menú contextual:} 
    cuanto tiempo le toma al usuario realizar una acción por menú 
    contextual las siguientes veces.

\item \textbf{Tiempo de realización de las siguientes acciones por \Gls{gui}:} 
    cuanto tiempo le toma al usuario realizar una acción 
    por interfaz gráfica de usuario las siguientes veces.

\item \textbf{Tiempo de realización de las siguientes acciones por herramienta:} 
    cuanto tiempo le toma al usuario realizar una acción por herramienta 
    las siguientes veces.

\item \textbf{Tiempo total:} se refiere al tiempo empleado por el usuario para 
    completar la tarea asignada.

\item \textbf{Número de pasos realizados:} cantidad de pasos requeridos en la tarea 
    que son realizados por el usuario en la simulación. 

\item \textbf{Cantidad de movimientos espaciales por tipo:} número de veces en que se 
    modifica el estado de la cámara para realizar las acciones deseadas agrupados por 
    tipo (desplazamiento, acercamiento/desplazamiento).

\end{itemize}

En cuanto a la encuesta, las siguientes son las variables que fueron consideradas 
y medidas:

\begin{itemize}

\item \textbf{Calidad gráfica:} realismo y calidad de los modelos utilizados.

\item \textbf{Interacción:} desenvolvimiento en el entorno y utilización del 
    hardware.

\item \textbf{Interacción con objetos:} utilización errónea de objetos.

\item \textbf{Características del entorno:} realismo del escenario y de los 
    objetos utilizados.

\item \textbf{Usabilidad de la interfaz:} facilidad de uso de las opciones 
    proveídas por la interfaz.

\item \textbf{Integración con el hardware:} facilidad de uso de la solución con 
    un dispositivo móvil. 

\end{itemize}

\subsection{Métricas}

Para el análisis de la encuesta realizada a los usuarios, se utiliza la escala de Likert
con la \emph{Doble estandarización} explicada en~\ref{sec:likert}, y en el análisis de la 
interacción del usuario con la solución se utilizan las grabaciones registradas durante la 
prueba.

Haciendo uso de las variables descriptas anteriormente, las métricas
utilizadas son las siguientes:

\begin{itemize}
    
\item \textbf{Tiempo promedio de realización de las siguientes acciones por menú contextual:} 
    se obtiene dividiendo la cantidad total de tiempo empleado en realizar acciones por menú 
    contextual por el número de veces que se realizaron esas acciones, a excepción de la primera 
    vez. 
    
\item \textbf{Tiempo promedio de realización de las siguientes acciones por \Gls{gui}:} 
    se obtiene dividiendo la cantidad total de tiempo empleado en realizar acciones por \Gls{gui} 
    por el número de veces que se realizaron esas acciones, a excepción de la primera 
    vez. 
    
\item \textbf{Tiempo promedio de realización de las siguientes acciones por herramienta:} 
    se obtiene dividiendo la cantidad total de tiempo empleado en realizar acciones por 
    herramienta por el número de veces que se realizaron esas acciones, a excepción de la 
    primera vez. 
    
\item \textbf{Promedio de pasos correctos:} se obtiene dividiendo la cantidad de 
    pasos requeridos realizados por los usuarios sobre la cantidad de pasos requeridos. 
    
\item \textbf{Promedio de movimientos por tipo:} se obtiene dividiendo el número de 
    movimientos que fueron realizados agrupados por tipo (desplazamiento, acercamiento/
    desplazamiento) por la cantidad de usuarios.
    
\item \textbf{Promedio del tiempo total:} se obtiene dividiendo el tiempo total empleado 
    por los usuario para completar la tarea asignada por el número de usuarios.

\end{itemize}

