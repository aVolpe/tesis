%! TEX root = ../main.tex

\section{Métricas generales}

\observacion{No se entiende} 
Se definen métricas generales  que son utilizadas por más de una prueba o
encuesta que forma parte de la evaluación. Se describen aquellas que se
consideran que son requeridas detallar.

A continuación se describe la escala de Likert que es una métrica utilizada en
la \emph{Encuesta subjetiva} y en la encuesta correspondiente a la \emph{Prueba
de interfaz de usuario}. Por otra parte se describe también en que consiste la
correlación de Pearson, métrica que es utilizada para medir el grado de relación
entre variables de las encuestas realizadas, registros de actividades, entre
otros.

Cabe destacar que en~\cite{norman2010likert} se demuestra que, aunque el tamaño
de la muestra sea pequeña, los datos no puedan ser distribuidos normalmente o
los datos sean de escalas de tipo Likert, los métodos paramétricos como el
análisis de varianza, la regresión y la correlación pueden ser utilizados.


\subsection{Escala de Likert}
\label{sec:likert}

Para la valoración de las variables medidas en la \emph{Prueba de interfaz de
    usuario} y la {Encuesta subjetiva} se utiliza la escala de
Likert\cite{Allen:2007} de 7 valores posibles. La escala de Likert es utilizada
para permitir a las personas indicar cuánto están de acuerdo o en desacuerdo con
respecto a ciertos puntos. Los valores utilizados, son:

\begin{enumerate}
    \item Totalmente en desacuerdo.
    \item En desacuerdo.
    \item Parcialmente en desacuerdo.
    \item Neutral.
    \item Parcialmente de acuerdo.
    \item De acuerdo.
    \item Totalmente de acuerdo.
\end{enumerate}

Una vez valoradas y registradas todas las respuestas y con el objetivo de
eliminar las tendencias en la forma en la que son completadas las
encuestas\cite{Fischer2010} se utiliza el método de \emph{Doble Estandarización}
recomendado en~\cite{Pagolu2011}. Este método, consiste en dos
estandarizaciones, la primera por fila, que en este caso representa a los
individuos y la segunda por columna donde cada columna representa una de las
diferentes preguntas de la encuesta.

Siendo:
\begin{itemize}
	\item $\min_i$ la respuesta de menor valor del usuario $i$.
	\item $\max_i$ la respuesta de mayor valor del usuario $i$.
\end{itemize}

Para cada respuesta $s$ del usuario $i$, el valor ajustado, por la primera 
normalización, $s_1$ se define como:

\begin{equation}
s_1{_i}=\frac{s-\min_i}{\max_i-\min_i}
\end{equation}

\observacion{Considerar resumir}
Y luego siendo:
\begin{itemize}
	\item $groupmin_i$ la respuesta ajustada de menor valor en el grupo $i$.
	\item $groupmax_i$ la respuesta ajustada de mayor valor en el grupo $i$
\end{itemize}

Para cada respuesta ajustada $s_1{_i}$ del usuario $i$, el valor ajustado $sa_i$ se
define como:	

\begin{equation}
sa_i=\frac{s_{1_i}-groupmin_i}{groupmax_i-groupmin_i}
\end{equation}

Obteniendo así un valor normalizado, tanto por individuo, como por pregunta, en
el rango $0$ y $1$.

Para la valoración absoluta de cada  item se utiliza la media de cada columna o
respuesta a una pregunta de la encuesta.

Siendo:
\begin{itemize} 
\item $r_{k_i}$ la respuesta del usuario $i$ a la pregunta $k$.
\item $t_k$ la cantidad total de usuarios que respondieron la pregunta $k$.
\end{itemize}

El puntaje promedio de cada pregunta o item evaluado  $p_k$ en la encuesta se
define como:

\begin{equation}
p_k = \frac{\sum_{i=1}^n{r_{k_i}}}{t_k}
\end{equation}

\subsubsection{Manejo de información faltante}
\label{sec:informacion_faltante}
\observacion{No repetir tanto existe}

En toda encuesta pueden existir preguntas que no son respondidas por los
encuestados. Para tratar estos escenarios, se debe categorizar el patrón de
ocurrencia de la falta de
respuestas\cite{leite2010performance,tsikriktsis2005review}:

\begin{description}
    \item[Información faltante completamente aleatoria:] cuando la información
        faltante es independiente de la variable medida y de otras variables.
    \item[Información faltante aleatoria:] cuando la información faltante
        depende de otras variables, pero no de la variable en sí. 
    \item[Información faltante no aleatoria:] cuando hay una relación entre la
        información faltante y el valor de la variable.
\end{description}

Una vez categorizado el patrón de ocurrencia, existen a su vez tres
mecanismos~\cite{tsikriktsis2005review} principales para lidiar con información
faltante como son la eliminación, el reemplazo y los  procedimientos basados en
modelo.~\cite{tsikriktsis2005review} recomienda utilizar un mecanismo de
reemplazo para escalas del tipo Likert.

Las técnicas de reemplazo se clasifican en tres grandes
grupos\cite{tsikriktsis2005review}:
\begin{enumerate*}[label=\itshape\alph*\upshape)]
\item basadas en el promedio,
\item basadas en regresión, e,
\item imputación \emph{hot deck}.
\end{enumerate*}

\fixme{De estas técnicas se seleccionó la sustitución}{Resaltar}. Basada por
promedio ya que las relaciones entre las variables son bajas y los datos
faltantes son menos del $10\%$. La sustitución basada por promedio se divide
nuevamente en tres grupos\cite{tsikriktsis2005review}; promedio
\begin{enumerate*}[label=\itshape\alph*\upshape.]
\item total,
\item del subgrupo, y,
\item por caso.
\end{enumerate*}

La sustitución por promedio total es elegida debido a que la relación entre la
variable que falta y las demás variables en los datos es relativamente baja, es
fácil de usar y retiene la muestra. La sustitución por promedio total se realiza
obteniendo el promedio de todas las respuestas de la pregunta cuya respuesta
falte, la sustitución de subgrupo es similar, solo que se limita a aquellos
sujetos del mismo subgrupo del sujeto que no respondió, y finalmente, la
sustitución por caso, es el promedio de las respuestas válidas del sujeto.

\subsection{Correlación de variables aleatorias}
\label{sec:correlacion}

Las correlaciones se utilizan durante una etapa exploratoria o de observación de
la investigación para determinar las variables que tienen al menos una relación
estadística con cada uno de los diseños experimentales. Las correlaciones
también se utilizan para determinar el grado de asociación entre variables
dependientes e independientes. Por otro lado, el coeficiente de correlación se
utiliza comúnmente para cuantificar el grado de asociación entre dos variables
\cite{BoslaughStatistics2008}.

La correlación de Pearson\cite{BoslaughStatistics2008} mide la relación que
existe entre dos variables, $X$ e $Y$, el mismo esta comprendido entre $-1$ y
$1$, en su punto más bajo ($-1$) indica que una de las dos variables crece mientras
la otra decrece, y en su punto más alto ($1$), indica que ambas crecen o
decrecen conjuntamente, el valor $0$, indica que no existe una relación entre
ambas variables.

El coeficiente para las variables $X$ e $Y$ está dado por:

\begin{equation}
r = \frac{\sum_{i=1}^n{(\frac{x_i-\bar{x}}{s_x})({\frac{y_i-\bar{y}}{s_y}})}}%
{n - 1}
\end{equation}

donde:

\begin{itemize}
    \item ($x_i$, $y_i$) es el conjunto de coordenadas de las variables $X$ e $Y$.
    \item $\bar{x}$ es la media de la variable $X$.
    \item $\bar{y}$ es la media de la variable $Y$.
    \item $s_x$ es la desviación estándar de la variable $X$.
    \item $s_y$ es la desviación estándar de la variable $Y$.
    \item $n - 1$ son los grados de libertad.
\end{itemize}
