\section{Metodología}

\label{sec:metodologia}

Para la obtención de la información deseada acerca de la utilización de la
solución y de su impacto en el aprendizaje, se utilizan tres mecanismos que en
conjunto nos ayudan a responder a los objetivos de la evaluación.

La evaluación se divide en tres partes, una encuesta subjetiva que pretende
obtener información acerca de la satisfacción de los usuarios, de ahora en
adelante \textbf{encuesta subjetiva}, un examen que mide los factores
relacionados al aprendizaje, de ahora en adelante \textbf{encuesta objetiva}  y
finalmente una evaluación del registro de actividad de cada jugador que es
registrado automáticamente por la solución.


\subsection{Registro de actividad}

La aplicación almacena información relacionada a la actividad del usuario,
incluyendo cuando y como utiliza las opciones, los pasos que realiza, el orden y
las condiciones de la escena cuando realiza cada acción.

El registro como un todo es enviado cada vez que el usuario desee, este envío
requiere una conexión a internet por ello no es automático. Adicionalmente el
último día de la prueba, todos los registros fueron enviados para que sean
analizados.


\subsection{Encuesta de aprobación}

Al final del periodo de prueba, cada alumno de la muestra completa una encuesta
con 31 preguntas que se utilizan para validar las hipótesis. Las preguntas están
agrupadas en dos grupos, el primero cuenta con 27 preguntas cuyas respuestas son
en forma la escala de  Likert, como se explica en \ref{sec:metrica}, de 7
opciones cada una, la segunda parte cuenta con 4 preguntas abiertas. Estas
preguntas están formuladas teniendo en cuenta las variables descriptas en
\ref{sec:variables}.

Cada encuesta fue entregada a los alumnos que acordaron participar en la
simulación, mientras completaban la encuesta, un guía estuvo presente para
responder cualquier duda que surgiera.

\subsection{Encuesta de ubicación tecnológica}

\todox{Mover esto a un lugar mas adecuado}
\label{subsec:ubicacion}

A fin de obtener información acerca del nivel de acceso a la tecnología de los
alumnos, se llevo a cabo una encuesta sobre la totalidad del universo definido
en \ref{subsec:universo}, la misma cuenta con diez preguntas, las cuales buscan
obtener información acerca del modelo de dispositivo móvil que tiene cada
estudiante, el acceso del mismo a Internet, y su predisposición a ayudar en el
experimento.

