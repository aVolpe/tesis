\section{Objetivos}
\label{sec:objetivos}

Los \emph{Registros de actividades} y \emph{Encuesta objetiva} buscan obtener 
información sobre el aprendizaje y la utilización de la solución, mientras que 
la \emph{Encuesta subjetiva} busca obtener información acerca de las fortalezas y 
debilidades de una simulación para el entrenamiento de enfermeros y de la solución 
propuesta.

Se definen los objetivos principales de la evaluación como siguen:

\begin{enumerate}

\item Validar las hipótesis asumidas durante el desarrollo de la
    solución.
    % Revisar

\item Evaluar los puntos fuertes y débiles de la solución.
%en cuanto a los criterios definidos.
    %   Usabilidad
    %   Ubicuidad
    %   Realismo
    %   Mirar la encuesta pare ver a que se refiere esto

\item Determinar los factores que afectan al uso de herramientas similares para
    apoyo a profesionales de enfermería
    %   Estos son los puntos que me parecen que pueden responder a este objetivo
    %%   SC7.Es importante mostrar los detalles de los pasos incorrectos ya que no sería suficiente sólo decir qué pasos se hicieron correctamente.
    %%   SC14.La opción de hablar para que aparezca el menú de ordenes verbales hace que el juego sea mas interactivo, siendo más similar a la realidad
    %%   SC15.La escenografía en los juegos permite que entremos en ambiente para realizar los procedimientos.
    %%   SC16.Los gráficos en tres dimensiones nos ayudan a entender mejor el entorno y las posibles acciones
    %%   SC17.La falta de pistas durante el desarrollo de una partida permite plasmar y medir el conocimiento acerca del tema
    %%   SC18.Es importante dar una puntuación total para ver el rendimiento
    %%   SC19.Motiva compartir y ver el progreso con otras personas a través del Facebook
    %%   SC20.La cantidad de tiempo de cada procedimiento jugado, motiva a seguir jugando para mejorarlo
    %%   SC21.El puntaje de cada procedimiento jugado, motiva a seguir jugando para mejorarlo
    %%   SC22.Interactuar con un paciente que reacciona a mis acciones, es mejor que utilizar un maniquí inmóvil
    %%   SC23.La utilización de la simulación en todo momento provee más facilidades para poner en práctica los conocimientos con respecto a las demás alternativas.
    %%   SC24.La utilización de herramientas alternativas, como la simulación, es útil para complementar el estudio en clase y laboratorio
    %%   SC27.Juegos cortos permiten jugar varias veces de seguido
    
    % Los factores serían
    %%  Nivel de detalle de retroalimentación       SC 7
    %%  Interfaz a través de voz                    SC14
    %%  Entorno realista                            SC15 - SC16 - SC22 - SC23
    %%  Interacción social                          SC19 
    %%  Puntuación y tiempo                         SC18 - SC21
    %%  Duración                                    SC20 - SC27
    %%  Utilidad                                    SC23 - SC24
    
\item Determinar el nivel de aceptación de la solución.

\item Evaluar la utilización de la solución, y el progreso de los usuarios.

\item Identificar la influencia de la utilización de la solución en el ámbito
    pedagógico.

\item Determinar correlaciones entre variables estudiadas, a fin de determinar
    la influencia de la utilización de la solución.
    
\end{enumerate}

