\section{Métrica}
\label{sec:metrica}

Los datos de las variables descriptas en la sección~\ref{sec:variables} fueron
recolectados utilizando como metodología la encuesta de aprobación como se
indico en la sección~\ref{sec:metodologia}, para la valoración de las variables
medidas se utilizo la escala de Likert~\cite{Allen:2007} la cual consto de 7
valores posibles.

El valor mas bajo utilizado en la escala de Likert fue 1/``Totalmente en
desacuerdo'' y el mas alto fue 7/``Totalmente de acuerdo''.

Una vez valoradas y registradas todas las respuestas y con el objetivo de
eliminar las tendencias en la forma en la son completadas las encuestas
\cite{Fischer2010} se utiliza el método de Doble Estandarización recomendado por
\cite{Pagolu2011}. Este método, consiste en dos estandarizaciones, la primera
por fila que en este caso representa a los individuos y la segunda por columna
donde cada columna representa una de las diferentes preguntas de la encuesta.

Siendo:
\begin{itemize}
	\item $\min_i$ la respuesta de menor valor del usuario $i$.
	\item $\max_i$ la respuesta de mayor valor del usuario $i$.
\end{itemize}

Para cada respuesta $s$ del usuario $i$, el valor ajustado, por la primera 
normalización, $s_1$ se define como:

\begin{equation*}
s_1=\frac{s-\min_i}{\max_i-\min_i}
\end{equation*}

Y luego siendo:
\begin{itemize}
	\item $groupmin_i$ la respuesta ajustada de menor valor en el grupo $i$.
	\item $groupmax_i$ la respuesta ajustada de mayor valor en el grupo $i$
\end{itemize}

Para cada respuesta ajustada $s_1$ del usuario $i$, el valor ajustado $s_a$ se
define como:	

\begin{equation*}
s_a=\frac{s_1-groupmin_i}{groupmax_i-groupmin_i}
\end{equation*}

Obteniendo así un valor normalizado, tanto por individuo, como por pregunta, en
el rango $0$ y $1$.

Con el resultado final de la estandarización podemos diferenciar cuales son los
puntos fuertes y cuales los puntos débiles de la aplicación con respecto a las
respuestas dadas por los usuarios. Estos valores son relativos a las respuestas
originales dadas a la encuesta por lo usuarios de la aplicación.

Para la valoración absoluta de cada  item se utiliza la media de cada columna o
respuesta a una pregunta de la encuesta.

Siendo:
\begin{itemize} 
\item $r_{k_i}$ la respuesta del usuario $i$ a la pregunta $k$.
\item $t_k$ la cantidad total de usuarios que respondieron la pregunta $k$.
\end{itemize}

El puntaje promedio de cada pregunta o item evaluado  $p_k$ en la encuesta se
define como:

\begin{equation*}
p_k = \frac{\sum_{i=1}^n{r_k_i}}{t_k}
\end{equation*}


