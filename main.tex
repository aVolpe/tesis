\documentclass[final,fmstyle]{./util/ucathesis}

\usepackage[T1]{fontenc}
\usepackage[spanish]{babel}
\usepackage[utf8]{inputenc}
\usepackage{csquotes}
\usepackage{graphicx}
\usepackage[showframe=false]{geometry}
\usepackage{pdflscape}
\usepackage[inline]{enumitem}
\usepackage{pgfgantt}
\usepackage[bookmarks]{hyperref}
\usepackage{changepage}
\usepackage{booktabs}
\usepackage{listings}
\usepackage{xcolor}
\usepackage{xargs}

\usepackage{tikz}
\usepackage[style=numeric,sorting=none,backend=biber]{biblatex}

\usepackage{glossaries}
\usepackage{etoolbox}
\usepackage[xindy]{imakeidx}

\usepackage[colorinlistoftodos,prependcaption,textsize=tiny]{todonotes}

%\MakePerPage{footnote}
\addbibresource{bibliography.bib} 

%Datos de la tesís
\title{Construccionismo como complemento a la enseñanza tradicional:  una
	aplicación a la formación de profesionales del área de enfermería}
\author{Mirta González y Arturo Volpe}
\degree{Informática}

\advisor{Ing.}{Martín Abente Lahaye,M.Sc.}

\logosource{./graphics/logo.jpg}
\institution{Universidad Nacional de Asunción}
\faculty{Facultad Politécnica}
\address{San Lorenzo - Paraguay}

%% templates
\lstdefinestyle{sharpc}{language=[Sharp]C, frame=lr, rulecolor=\color{blue!80!black}}

\newtoks\customtok


\renewcommand*{\newacronymhook}{%
 \edef\dosetkeys{\noexpand\setkeys{glossentry}{user1={},\the\glskeylisttok}}%
 \dosetkeys
 \ifcsempty{@glo@useri}%
 {%
   \expandafter\customtok\expandafter{\the\glsshorttok}%
 }%
 {%
   \edef\custom{\the\glsshorttok, \csexpandonce{@glo@useri}}%
   \expandafter\customtok\expandafter{\custom}%
 }%
}

\newcommand*{\custompostdesc}[1]{%
  \ifcsempty{glo@#1@useri}{}{(\glsentryuseri{#1})}%
}

\renewcommand*{\CustomAcronymFields}{%
  user1={},%
  name={\the\glsshorttok},%
  description={\the\glslongtok\noexpand\custompostdesc{\the\glslabeltok}},%
  first={\the\glslongtok\space(\the\customtok)},%
  firstplural={\the\glslongtok\noexpand\acrpluralsuffix\space(\the\customtok)}%
  text={\the\glsshorttok},%
  plural={\the\glsshorttok\noexpand\acrpluralsuffix}%
}

\newcommandx{\todox}[2][1=]{\todo[linecolor=gray,backgroundcolor=gray!25,bordercolor=gray,#1]{#2}}


\SetCustomStyle
\makeglossaries
\makeindex

\begin{document}

\newacronym[user1=Conseil Européen pour la Recherche Nucléaire]{cern}{CERN}{Organización Europea para la Investigación Nuclear}
\newacronym[user1=One Laptop Per Child]{olpc}{OLPC}{Una computadora por niño}
\newacronym[user1=Massachusetts Institute of Technology]{mit}{MIT}{Instituto Tecnológico de Massachusetts}
\newacronym{tic}{TIC's}{Tecnologías de la información y la comunicación}
\newacronym[user1=Event-Condition-Action]{eca}{ECA}{acciones condicionadas por eventos}
\newacronym{iab}{IAB}{Instituto Andrés Barbero}
\newacronym[user1=Graphics Processing Unit]{gpu}{GPU}{Unidad de procesamiento de gráficos}
\newacronym[user1=Application Programming Interface]{api}{API}{Interfaz de programación de aplicaciones }



\maketitle

% Tabla de contenidos
\tableofcontents
% Lista de figuras
\listoffigures
% Lista de tablas
%\listoftables
% Lista de algoritmos
\listofalgorithms

\listoftodos[Notes]


\printglossary[type=\acronymtype,title=Lista de Siglas]

\addcontentsline{toc}{chapter}{Lista de Siglas}


\mainmatter

\chapter{Introducción}

Sección de la introducción al uso de las \Gls{tic}.


\chapter{TIC en la educación}

Las \Gls{tic} son un conjunto de herramientas tecnológicas y recursos utilizados
para comunicar, crear, diseminar, almacenar y manejar la
información\cite{unesco:ict}. Estas tecnologías abarcan computadoras personales,
internet, radio, televisión y telefonía\cite{tinio:ict}.

Las \Gls{tic} fueron utilizadas como complemento a la educación desde los
inicias de la misma con la radio y la televisión. Fueron vistas como un
complemento a las herramientas utilizadas en clase, como complemento del libro,
o como una herramienta que elimina la distancia física entre el profesor y el
alumno\cite{unesco:ict}. 

La utilización de las \Gls{tic} no mostró una utilidad clara hasta la
utilización masiva de las computadoras, es en esta área donde se encontraron los
resultados más prometedores\cite{unesco:ict}. 

Las principales ventajas de la utilización de las \Gls{tic} en la educación es
su aplicabilidad en áreas que no pueden ser cubiertas por otras alternativas,
como son:

\begin{description}

    \item[Nuevos modelos pedagógicos] teorías como el constructivismo moderno
	    enfatizan el proceso de como adquirir conocimiento y no solamente el
	    conocimiento en sí.

    \item[Recursos remotos] Las herramientas tradicionales como las bibliotecas,
	    escuelas y universidades están limitadas a un espacio físico, con
	    las \Gls{tic} este requerimiento físico desaparece, prueba de ello
	    es el Internet, que es la colección más grande de información y esta
	    disponible prácticamente a cualquier estudiante\cite{tinio:ict}.

    \item[Colaboración distribuida] como consecuencia del punto anterior, los
	    alumnos pueden colaborar de manera más sencilla pues no tienen
	    limitaciones físicas. Además, los alumnos pueden consultar con
	    expertos que están en linea, e incluso tener mentores en linea,
	    estas tutorías pueden ser uno a uno, por ejemplo mediante
	    comunicaciones por correo electrónico. Además permite la
	    colaboración masiva entre estudiantes de intereses comunes, mediante
	    foros y redes sociales\cite{unesco:ict}.

    \item[Motivación para aprender] la utilización de contenido multimedia puede
	    ayudar a que los alumnos deseen involucrarse más en el proceso
	    educativo.

    \item[Adquisición de habilidades básicas] las habilidades necesarias para
	    utilizar de manera efectiva las \Gls{tic} se están convirtiendo en
	    una necesidad básica, un aprendizaje guiado por las mismas puede
	    ayudar a una rapida asimilación de los conceptos relacionados.

\end{description}

Uno de los desafíos más importantes que enfrentan las \Gls{tic} para convertirse
en una alternativa viable es la inversión en infraestructura
necesaria\cite{unesco:ict}. 

\section{Historia de las TIC en la educación}

La historia de las \Gls{tic} en educación comienza con la Universidad Abierta
del Reino Unido\footnote{Open University of United Kingdom} que en 1969 se
establece como la primera institución educativa dedicada a la enseñanza a
distancia utilizando las, para aquel entonces, nuevas
tecnologías\cite{tinio:ict}.

En 1973 Vint Cerf creo el protocolo TCP/IP y es considerado el nacimiento de
Internet\cite{white:ict}, lo que permitió que la información pueda ser
transmitida de manera más sencilla, tiempo después con la aparición de las
computadoras personales en 1977\cite{white:ict}. 

Otro hito tecnológico se dio en la \Gls{cern} en el año 1989 cuando se concibió
lo que hoy se conoce como \emph{World Wide Web}, permitiendo que los usuarios de
la \emph{Web} puedan compartir archivos mediante un protocolo
estándar\cite{white:ict}. 

Con las principales eventos que marcaron la evolución tecnológica de las
\Gls{tic} en la educación, se divide su historia en cinco partes. Las mismas se
pueden dividir en dos secciones, las primeras tres corresponden a los comienzos
y donde los alumnos eran receptores de información, época denominada, y la
segunda denominada\emph{push}\cite{white:ict} que es aquella donde los alumnos
participan de su educación y son creadores activos de conocimiento.

\subsection{Programación, ejercicios y prácticas}

Este periodo que abarca desde la aparición de las primeras computadoras
personales hasta el final de la década de 1980, este periodo se caracterizo por
computadoras muy limitadas, nula interacción multimedia y escasez de programas
especializados. Se enseñaba programación básica\cite{leinonen:ict}, no por la
necesidad de educar programadores, sino por la creencia de que así se
desarrollarían habilidades matemáticas y lógicas en los alumnos. Los programas
eran muy simples y se basaban en matemáticas y nociones básicas del idioma. 

Esta clase de ejercicios no ayudaron a los alumnos a obtener un aprendizaje
profundo, pues era fácil resolverlos a través de la prueba y el error, la mayor
parte del tiempo servían para distraer a los alumnos no interesados en la
programación mientras el profesor enseñaba programación a aquellos que parecían
interesados\cite{leinonen:ict}.


\subsection{Entrenamiento basado en computadoras}

Cuando aparecieron en el mercado computadoras con multimedia, se argumento que
los ejercicios de la era anterior fallaron en su objetivo de una educación
profunda por que no contenían multimedia\cite{leinonen:ict}, las aplicaciones
eran distribuidas por CD-ROM, y así se actualizaban de manera más frecuente, y
podían contener gran cantidad de contenido multimedia.

En este periodo se desarrollaron una gran cantidad de aplicaciones educativas
que más tarde serían conocidas como \emph{Edutainment}\footnote{Education +
	Enteirtainment, se traduce como educación entretenida}, estas pretendían
agregarle entretenimiento a la educación, se veía al como un receptor pasivo de
información que debía asimilarla, y para aumentar el compromiso, el
entretenimiento era agregado\cite{resnick:2004}.

Las bases pedagógicas de esta se basada en la capacidad de ciertos estudiantes
de aprender mejor cuando interactúan con contenido multimedia, la \emph{prueba y
	error} aún estaban presentes, pero no eran presentados inmediatamente,
sino más bien una vez que el alumno ya debería haber asimilado los conceptos y
funcionaban como pruebas de adquisición de conocimiento. Este tipo de contenido
tampoco logro la enseñanza profunda, solamente fueron efectivos en el
aprendizaje de idiomas, fallando en todos los demás campos\cite{leinonen:ict},
además los contenidos muchas veces estaban desactualizados y obtener nuevas
versiones no era una tarea sencilla.

Varios gobiernos apoyaron de manera agresiva la introducción de las \Gls{tic} en
educación\cite{mcdougall2006theory} y se realizo un importante avance teórico
con los trabajos sobre el aprendizaje construccionista de Papert y Harel (1991),
y la influencia de las computadoras sobre el aprendizaje y la mente de Marvin
Minksy (1987)~\cite{mcdougall2006theory}.

A comienzos de la decada de 1990, con la popularización de Internet, se le vio
como solución al problema de las poco frecuentes actualizaciones de aplicaciones
educativas, su utilización no tenia bases pedagógicas, más bien se basaban en la
facilidad de distribuir contenido por la \emph{Web}, el principal inconveniente
era la velocidad del Internet, no era suficiente para proveer entornos ricos en
multimedia como lo hacían los CD-ROM\cite{leinonen:ict}.

\subsection{e-Learning}

La bases pedagógicas de esta son similares a la era del entrenamiento basado en
computadoras, se distribuye contenido masivamente a los alumnos, y luego, de
manera muy discreta se permite a los mismos colaborar, dejando siempre en claro
que primero se debe asimilar toda la información posible y luego relacionarse
con los demás\cite{leinonen:ict}.

El \emph{e-Learning} apareció a finales de la década de 1990 y tubo su apogeo en
mediados de la década del 2000, apoyada por la gran penetración de las \Gls{tic}
en la población\cite{punie:ict}.

Todos los paradigmas anteriores viven dentro del \emph{e-Learning}, permitiendo
compartir contenido multimedia y realizar pruebas del tipo \emph{prueba-error}. 

Si bien en las anteriores épocas, el uso de las \Gls{tic} estaba más orientado
hacia la educación básica y secundaría, el \emph{e-Learning} actualmente es más
utilizado en la educación terciaría\cite{punie:ict}.

La utilización del \emph{e-Learning} tiene varios grados de aplicación en
entornos reales\cite{punie:ict}, que van desde ser simples elementos
complementarios a la clase, como por ejemplo un repositorio para las
diapositivas y otros materiales de clase, hasta cursos completamente en linea,
donde la clase ha sido completamente sustituída.


%\input{tics/arte}
\section{Problemas actuales}
\label{tics:problemas}

Durante la historia de las \Gls{tic} en la educación, se han encontrado
diferentes dificultades a la hora de aplicar los nuevos conceptos en la
educación, desde los primeros enfoques que carecían de bases pedagógicas válidas
hasta la actualidad.

El principal problema es falta de motivación de los profesionales de la
educación para emplear las \Gls{tic}\cite{punie:ict}\cite{ict:romeo}.

El contenido proveído actualmente puede ser considerado como un conjunto de
buenas prácticas\cite{punie:ict} y así, omiten completa o parcialmente el
contexto donde esa buena práctica fue generado.

A la hora de educar a los educadores en la utilización de las \Gls{tic} no se
debe medir medidas cuantitavias, como las notas o el número de cursos, sino más
bien el impacto cualitativo de la educación

Las \Gls{tic} han tenido un impacto positivo en la educación\cite{punie:ict},
pero no han obtenido el impacto esperado.

Iniciativas como el \emph{Edutainment}, prometían ser la solución a los
problemas educacionales, sin embargo, su implementación no cumplio con las
expectativas, obtuvieron una reputación negativa, y hoy en día son considerados
como el peor tipo de educación, pues son un ejercicio de \emph{prueba-error}
ocultos bajo un juego poco entretenido\cite{resnick:2004}. La principal critica
contra los \emph{Edutainment} es su incapacidad de enseñar como aplicar
conceptos aprendidos a un entorno real\cite{resnick:2004}.

Mientras que la utilización de las \Gls{tic} puede eliminar problemas actuales
como el aislamiento y la falta de pensamiento de alto nivel\cite{punie:ict}, la
brecha social existente implica otro riesgo para la utilización de las \Gls{tic}
en la educación, aquellos que no posean los recursos económicos necesarios para
acceder a la misma no se verán beneficiados por las \Gls{tic}\cite{punie:ict}.


 

\section{Construccionismo y las TIC's}
\label{sec:tics_CONSTRUCCIONISMO}

El construccionismo es una corriente pedagógica que parte de una concepción del
aprendizaje según la cual la persona aprende por medio de su interacción
dinámica con el mundo físico, social y cultural en el que está
inmerso\cite{valdivia:sg}.

Posee un enfoque diferente en cuanto al uso de las \Gls{tic} en la educación.
Esta pedagogía se diferencia de la educación tradicional en que el estudiante ya
no es un receptor pasivo de información, en cambio, el mismo participa
activamente del proceso de aprendizaje construyendo su propio conocimiento. Se
diferencia del instruccionismo en que el construccionismo utiliza la tecnología
como medio cognitivo y no para la entrega de contenido.

Se considera al construccionismo como una alternativa prometedora a la educación
tradicional. Desde el punto de vista tecnológico, es ideal pues el mismo
requiere un alto dinamismo en el traspaso del conocimiento
\cite{sasha:construtivism}. 

El construccionismo y las \Gls{tic} siempre han estado relacionados, ya que el
mismo se originó con un lenguaje de programación (LOGO)\cite{ict:ttc}. Un
característica importante de esta relación es que tienen la capacidad de
eliminar los problemas de distancia\cite{mariluz:seiousgames}.


\subsection{Historia}

A mediados de la década de 1960 Seymour Papert, un matemático sudafricano, llegó
a los Estados Unidos, donde fue co-fundador del Laboratorio de Inteligencia
Artificial del \Gls{mit} con Marvin Minsky\cite{logo:sg}. 


En la década de $1980$, Seymour Papert acuñó el término construccionismo en la
Fundación Nacional de Ciencia de los Estados Unidos titulada
\enquote{Constructionism: A New Opportunity for Elementary Science Education}
para presentar un método pedagógico que se basaba en muchas de las ideas de la
educación progresiva estudiadas por el estadounidense John Dewey en el inicio
del siglo 20 en su escuela experimental en la Universidad de Chicago. Dewey
quería poner gran parte de la responsabilidad de aprender en los estudiantes que
han nacido con el don de aprendizaje y la creación de conocimiento en sus
propios términos\cite{historia:2014}.

Papert también fue influenciado por Maria Montessori quien luego de grados en
pediatría, medicina, psicología y filosofía comenzó su propia escuela
experimental para niños pequeños. En lugar de que ella misma estableciera
formalmente las tareas, vio como sus estudiantes actuaron por su cuenta. Ella
siguió los intereses de los estudiantes y observó como respondieron a sus
entornos especialmente preparados. Ella preparó el escenario pero no ofreció una
guía explícita\cite{historia:2014}.

Papert trabajó directamente con el psicólogo suizo Jean Piaget a quien Ernst von
Glasersfeld llamó \enquote{El pionero de la teoría constructivista del
    aprendizaje}. 
% esto de glaserfeld pio importa?
Piaget, al igual que Dewey, Montessori y otros, desarrolló su teoría de la
educación y la construcción del conocimiento observando e interactuando con los
niños. De estas observaciones nació el movimiento constructivista. El
contructivismo se basa en que el conocimiento debe ser construído por el
estudiante y los nuevos significados deben ser obtenidos relacionándolos a
significados anteriores en el propio sistema de relaciones del
estudiante\cite{historia:2014}. 

Papert admitió que jugó más con la palabra construcción. El contruccionismo de
Papert se diferencia del constructivismo de Piaget en que los estudiantes
construyen las ideas o partes del mundo utilizando herramientas. Para Papert los
estudiantes necesitan contruir modelos de partes de su mundo con el fin de
comprender más plenamente el significado, el 
% la definición de modelos podemos poner un footnote?
contenido y la dinámica de las partes. La elaboración de representaciones
mentales mediante la construcción y el intercambio es la metáfora del marco
construccionista\cite{historia:2014}.

%Durante $1980$, Seymor Papert, Wally Feurzeig, Marvin Minsky y John McCarthy y los
%miembros del Departamento de Inteligencia Artificial del \Gls{mit} y una
%compañía de tecnología en Cambridge, Massachusetts, desarrollaron un nuevo
%lenguaje de programación llamado LOGO que tenía por objeto que los estudiantes
%construyeran sus modelos mentales en notación LOGO. Este juego
%introduciría de forma natural las ideas de los procedimientos, funciones,
%variables, recursividad, la modularidad, simulación, verificación, entre %otros\cite{historia:2014}.

Papert trabajó con el equipo de Bolt, Beranek y Newman, liderado por Wallace
Feurzeig, que creó la primera versión del lenguaje de programación LOGO en 1967.
LOGO es un dialecto de Lisp, fue diseñado como una herramienta para el
aprendizaje. Sus características, como la modularidad, extensibilidad,
interactividad y flexibilidad, derivan de este objetivo \cite{logo:sg}.

Los desarrolladores de LOGO no solo alentaron la promoción de formas
construccionistas de enseñanza y aprendizaje sino también alentaron otra forma
de aprendizaje nueva y no tradicional con las diferentes herramientas
tecnológicas\cite{historia:2014}. 

Por lo tanto, se puede decir que la creación de LOGO permitió la creación del
construccionismo\cite{historia:2014}.

    
    
% ESTE
% Cuando se produjo LOGO y se acuñó el construccionismo, la comunidad del 
% construccionismo era en su mayoría ingenieros informáticos y matemáticos\cite{historia:2014}.
% Esto es relevante?

\observacion{Toda la sección hay que reordernar}

\subsection{Bases Pedagógicas}

Para el construccionismo, el conocimiento es construido por el estudiante en
lugar de ser trasmitido por el profesor \cite{moses:2003} y esto sucede particularmente cuando
el mismo se compromete en la elaboración de un producto o artefacto que tenga un
significado y pueda ser compartido\cite{valdivia:sg}. De esta manera, se permite
a los estudiantes elaborar sus propias interpretaciones razonadas del mundo
mediante la interacción con el mismo. El profesor actúa como guía para el estudiante 
en la construcción de su conocimiento, aportando conocimiento y experiencia.

Según Papert, los alumnos estarán mucho más involucrados en su aprendizaje si
construyen artefactos que los demás pueden ver, criticar y tal vez utilizar. Y
además, el alumno se enfrenta a problemas complejos con estas construcciones,
harán el esfuerzo por resolver problemas y aprender ya que la construcción les
motivará\cite{const:vs}.

El enfoque construccionista establece que los seres humanos conocen y aprenden
de formas diferentes por lo tanto, no se puede elaborar una jerarquía de estilos
de aprendizajes\cite{valdivia:sg}.

\subsection{Actualidad}

%El construccionismo pone énfasis en el \emph{Aprender haciendo}, esta idea
%\fixme{mejora}{ref} la práctica educativa tradicional o instruccionismo. %El instruccionismo
%se basa en el concepto de que existe un profesor y un estudiante, el profesor
%transfiere el conocimiento que ha adquirido a un alumno que es receptor pasivo
%de información de esta manera, se enfoca más en la capacidad del profesor. 

Existen varios proyectos o emprendimientos que incluyen al contruccionismo como
base pedagógica, para la mayoría de ellos las computadoras son esenciales
mientras que para otros el mayor esfuerzo está en la incorporación de la
tecnología en su práctica educativa\cite{papertian:const}.

Algunos de estos emprendimientos son:

\begin{itemize}

\item \textbf{Lenguaje de programación LOGO}: El lenguaje LOGO es la cuna del construccionismo. %, se basa en el
%	principio de que se aprende mejor haciendo, pero se aprende todavía
%	mejor si se combina la acción con la verbalización  y la reflexión
%	acerca de lo que se ha hecho. 
	Fundamentalmente consiste en presentar a
	los niños retos intelectuales que puedan ser resueltos mediante el
	desarrollo de programas en LOGO. El proceso de revisión manual de los
	errores contribuye a que el niño desarrolle habilidades metacognitivas
	al poner en práctica procesos de auto-corrección\cite{logo:sg}.
	%http://es.wikipedia.org/wiki/Logo_(lenguaje_de_programaci%C3%B3n)


    %A mediados de la década de 1960 Seymour
	%Papert, que había estado trabajando con Piaget en Ginebra, llegó a
	%Estados Unidos donde co-fundó el Laboratorio de Inteligencia Artifical
	%del MIT con Marvin Minsky. Papert trabajó con el equipo de Bolt, Beranek
	%y Newman, liderado por Wallace Feurzeig, que creó la primera versión del
	%logotipo en 1967. A lo largo de la década de 1970 Logo fuen incubado en
	%el MIT y algunos otros sitios de investigación. El lenguaje de
	%programación Logo, un dialecto de Lisp, fue diseñado como una
	%herramienta para el aprendizaje. Sus características como la
	%modularidad, extensibilidad, interactividad y flexibilidad se derivan de
	%este objetivo. 
	%http://el.media.mit.edu/logo-foundation/logo/index.html

    Las actividades de programación LOGO se realizan en las áreas de matemática, lenguaje, 
    música, robótica, telecomunicaciones y ciencias. LOGO es accesible para novatos, 
    incluyendo niños pequeños, y también es compatible con exploraciones complejas y 
    proyectos sofisticados realizados por usuarios experimentados\cite{logo:sg}.
    
    Uno de los ambientes más populares de LOGO incluye a la tortuga, originalmente 
    era un robot que era puesto en el suelo y se podía dirigir su movimiento 
    escribiendo comandos en el ordenador. Pronto la tortuga emigró a las 
    pantallas gráficas de los ordenadores en donde se las utiliza para dibujar 
    formas y diseños\cite{logo:sg}.

    

%\item[Simulación] La simulación en el ámbito de la educación fue evolucionando
%desde simples motores de reglas hasta complejos entornos, la simulación
%demostró ser una herramienta muy útil el ámbito laboral
%\cite{mariluz:seiousgames}, pues enseña al alumno a encarar situaciones muy
%difíciles de representar en entornos completamente controlados y provee
%mecanismos para comprobar la efectividad de la herramienta. 

%Actualmente la simulación se utiliza más en el ámbito empresarial pues las
%empresas son las más necesitas de innovar en el ámbito de la enseñanza. Un
%ejemplo de esta necesidad se da, por ejemplo, en el entrenamiento de nuevos
%vendedores, es muy difícil enseñar a un vendedor como debe vender los productos
%con un pizzarón y/o una presentación, en cambio la simulación permite que el
%mismo pueda probar cosas nuevas y experiencias de sus compañeros (o
%instructor), convirtiendo así el aprendizaje en
%colectivo\cite{mariluz:seiousgames}. En el ámbito académico la simulación mas
%utilizada en campos físicos (como simulación de fluidos), meteorología
%(simulación de tormentas y fenómenos climáticos), etc. 

%\item[Serious Games] Diseñado con el propósito de aprender. Generalmente hace
%uso de la simulación para permitir un aprendizaje más realista.

%\item[Lego Serious Play] Es una iniciativa de Lego que busca fomentar el
%pensamiento creativo por medio de la construcción por parte de los estudiantes
%de su identidad y experiencias utilizando legos. 

\item \textbf{\Gls{olpc}}: es una asociación sin ánimos de lucro cuyo esfuerzo
    se centra en dotar a los niños de una computadora duradera, accesible y
    potente en los países en desarrollo, se dice que es un descendiente directo
    del construccionismo\cite{papertian:const}.
	
    Surgió dentro del MIT Media Lab, \Gls{olpc} propone un cambio de paradigma
    basado en un modelo de aprendizaje en el que cada alumno disponga de su
    propia computadora portátil y que se pueda conectar a internet, de forma
    totalmente gratuita, desde su escuela. A partir de esta política se pretende
    disminuir la brecha tecnológica y de acceso de información en países más
    desfavorecidos en comparación con los países del primer
    mundo\cite{videojuegos:gonzaleztardon}.
	
	%Con esto se busca que
	%la computadora personal sea utilizada como un laboratorio intelectual y
	%un vehículo para la auto-expresión. OLPC no tiene que ver con la
	%escolarización o la escuela, más bien las utiliza como medio de
	%distribución de las computadoras a los niños, los cuales pueden
	%utilizarlas para aprender en cualquier lugar y momento. Se busca
	%fomentar el aprendizaje natural, es decir, aquel aprendizaje sin
	%enseñanza.

    %\fixme{Los problemas atribuidos al experimento}{experimento?} OLPC son
    %predominantemente las críticas a la política, el liderazgo o de la
    %intransigencia de la escuela en vez del construccionismo o computadora
    %personal para los niños pobres. El experimento audaz de Nicholas Negroponte
    %(co-fundador de \Gls{olpc}) y Sugata Mitra para dejar las computadoras desde
    %un helicóptero sobre una aldea de África se basa en la creencia en el
    %construccionismo\cite{papertian:const}.

\item \textbf{Fabricación personal}: Neil Gershenfeld, colega de Papert en el
    Media Lab del \Gls{mit} dictó un curso titulado \emph{Cómo hacer casi
        cualquier cosa}. La idea se centraba en la creación de  la tecnología
    que se necesita para resolver los problemas que se poseen. Esta
    auto-confianza, la autonomía personal y la agencia sobre la tecnología han
    estado en el centro de trabajo de Papert durante años. Papert no sólo
    defendió la idea de que los niños posean computadoras personales, sino
    también que a la larga ellos debían mantenerlas, repararlas e incluso
    construirlas.

	Junto con la capacidad para utilizar la tecnología para inventar
	soluciones a los problemas de significado personal, los estudiantes no
	sólo tienen acceso a la información, sino que tienen una mayor capacidad
	para darle forma a su mundo. La fabricación personal promueve la visión
	de Papert \emph{Si se puede utilizar la tecnología para hacer las cosas,
	usted puede hacer las cosas mucho más interesantes y usted
	puede aprender mucho más haciéndolo}\cite{papertian:const}.

\end{itemize}

%http://constructingmodernknowledge.com/cmk08/wp-content/uploads/2012/10/StagerConstructionism2012.pdf



%! TEX root = ../main.tex
\chapter{Propuesta de solución}
\label{chap:solucion}

%

\section{Acciones condicionadas por eventos}

Las \gls{eca} son aquellas que son lanzadas una vez que se cumple un determinado
evento\cite{bailey2004event}. En las bases de datos relacionales, son conocidos



Las mismas pueden ser utilizadas para notificar que un determinado conjunto de
eventos ha ocurrido\cite{bailey2004event}, así como servir para almacenar
información acerca de la utilización de un determinado recurso.

\subsection{Motivación}

Las reglas del tipo \gls{eca} permiten reaccionar a determinados eventos, en
forma de una única regla, la cual facilita la declaración de las
mismas\cite{bailey2004event}.

Son principalmente útiles para analizar el comportamiento en tiempo real de un
sistema\cite{bailey2004event}.
%TODO agregar más motivaciones.



\subsection{Declaración}

Una \gls{eca}, se define como:

\begin{center}
	 Cuando ocurren una serie de \emph{eventos}, y se cumple una
	 \emph{condición}, entonces realizar una acción. \emph{Acción}
\end{center}

Los eventos determinan cuando una regla debe ser activada, los mismos se dividen
en dos categorías, primitivos y compuestos, los primeros son detectables, por
ejemplo, cuando se inserta una jeringa, y los compuestos, son la combinación de
uno o más primitivos\cite{bailey2004event}. Los eventos compuestos, se unen
mediante:
\begin{enumerate*}[label=\itshape\alph*\upshape)]
\item conjunción (\emph{y}),
\item disjunción (\emph{o}), y
\item secuencia (\emph{entonces}).
\end{enumerate*}
Sin embargo, no siempre son necesarios todas las posibles combinaciones, y las
combinaciones sencillas son más fáciles de optimizar y
probar\cite{bailey2004event}.

Una regla  puede tener argumentos, los cuales son el entorno en el cual se lanzo
el evento que lo lanza.

Las condiciones determinan si el entorno es el necesario para que la regla sea
activada.

La acción a ejecutar describe la lógica que debe ser ejecutada cuando se han
lanzado los eventos y la condición de la regla se ha cumplido.

\subsubsection{Dependencia entre reglas}

Las reglas pueden depender de otras reglas, lo cual se puede ver como que la
finalización de una regla es un evento que otra regla espera para poder ser
activada.

Las reglas pueden agregar información a un contexto compartido por todas las
reglas, de esta manera, se puede pasar parámetros entre distintas reglas, por
ejemplo, la regla \emph{Retirar Torniquete}, depende de la regla \emph{Insertar Torniquete}, pero debe responder solamente al torniquete
que ha activado la regla de inserción, es decir, el usuario puede extraer varios
torniquetes, y la regla no debe activarse, hasta que se extraiga el torniquete
que activo la primer regla.

Así, la regla \emph{Retirar Torniquete} depende de la regla \emph{Insertar
Torniquete}, y esta relación entre reglas, se da en dos formas:

\begin{itemize}
\item  \emph{Dependencia fuerte:} la regla \emph{Retirar Torniquete} solamente podrá
	ser elegida para ser lanzada cuando la regla \emph{Insertar Torniquete}
	haya sido cumplida.
\item  \emph{Dependencia de contexto}: la regla \emph{Retirar Torniquete} no se
	activará cuando los eventos a los que escucha se terminen, sino cuando
	los eventos a los que escucha sean lanzados con los parámetros adecuados
	(se extraiga el torniquete que lanzo la regla de inserción).
\end{itemize}




\subsection{Modelo de ejecución}

Para ejecutar un motor de reglas del tipo \gls{eca}, se debe tener en cuenta
principalmente dos factores, 
\begin{enumerate*}[label=\itshape\alph*\upshape)]
\item  Como se verifica el cumplimiento de cada regla, y, 
\item  Que ocurre cuando varias reglas son lanzadas al mismo tiempo.
\end{enumerate*}.

Para ambos casos se puede tomar un enfoque \emph{inmediato}, es decir que
inmediatamente cuando se lanza un evento, o se cumple una condición, se ejecuta
la regla. Además existen otros dos modos de ejecución, \emph{deferida}, y
\emph{desacoplada}, en la primera, se espera hasta que el lanzador del evento
culmine su trabajo, y luego se ejecute la regla, pero en la misma unidad de
trabajo, mientras que en la ejecución desacoplada, se encolan los trabajos y
otro hilo es el encargado de ejecutar las reglas.

La propuesta implementada, utiliza una ejecución inmediata, principalmente por
la sencillez de las reglas, es decir, las reglas no realizar un trabajo pesado,
solamente controlan el estado del entorno y lo validan.

Además, la ejecución inmediata es importante por que el entorno no sufre
modificaciones entre el evento lanzado y la ejecución de la regla, según
\cite{bailey2004event}, este es el factor más importante para determinar el tipo
de ejecución deseado.



\subsubsection{Estados de una regla}

Una regla puede estar en uno de los siguientes estados:

\begin{description}
\item[BEGIN] Es una regla que recién fue creada, no realiza ninguna
	acción.
\item[WAITING\_FOR\_RULE] Es un estado en el que esta esperando que otras reglas
	sean lanzadas.
\item[WAITING\_FOR\_EVENT] Es un estado en el que esta escuchando a que sean
	lanzados los eventos a los que escucha, este es el estado principal.
\item[WAITING\_FOR\_CONDITION] La regla ya no espera por ningún evento y las
	reglas de las que depende ya han sido lanzadas, se verifica cada cierto
	tiempo si el entorno cumple con una condición definida.
\item[FINISH] La regla ha sido lanzada, con un resultado no determinado, se pudo
	haber cumplido, como no, es el estado final de una regla. Cuando una
	regla llega a este estado, se lanza su evento de finalización.
\end{description}

Una regla puede estar en solo un estado, y solamente se permite que el estado
avance, desde \emph{BEGIN} hasta \emph{FINISH}.

\subsubsection{Ciclo de vida}

Cuando una regla es definida, y insertada al motor de reglas, inmediatamente
pasa al estado \emph{BEGIN}, luego se verifica si la misma depende de otras
reglas, sí este es el caso, pasa al estado \emph{WAITING\_FOR\_RULE} y escucha a
los eventos de finalización de las reglas anteriores.

Una vez que las reglas anteriores han sido finalizadas, la regla pasa al estado
\emph{WAITING\_FOR\_EVENT} sí deben escuchar por algún evento, en caso contrario
pasan al estado \emph{WAITING\_FOR\_CONDITION}.

Una vez que la regla está en estado \emph{WAITING\_FOR\_CONDITION}, pasa a un
motor que ejecuta su condición cada cierto tiempo, si la condición se cumple, la
regla se ejecuta, y la misma pasa a estado \emph{FINISH}, momento en el cual
notifica a las reglas que dependen de ella que ha sido lanzada.

Una vez que la regla esta en estado \emph{FINISH}, la misma sale del esquema de
ejecución, y solo esta disponible para obtener resultados.




\section{Descripción general de la solución}

Una vez descriptos los fundamentos teóricos del uso de videojuegos en el ambiente educativo sobre todo en aquellos que requieren de mucha practica para plasmar los conocimientos. Siendo el área de enfermería una de estas áreas definimos las problemáticas actuales de esta y seleccionamos algunos procedimientos que serán utilizados para el contenido de la solución. A continuación se busca converger todos los aspectos descriptos en
capítulos anteriores.

La solución propuesta en este trabajo consiste en el desarrollo de una aplicación para dispositivos móviles que se define como un juego serio llamado YAVE el cual incluye ciertos aspectos de gamification. El juego consiste en ofrecer a los usuarios, en este caso alumnos de enfermería, un medio en el cual puedan realizar procedimientos de enfermería y el cual les puede servir como herramienta de apoyo en su aprendizaje.

YAVE permitirá al usuario poder seleccionar el procedimiento que quiera realizar, en cada procedimiento se le dará la posibilidad de interactuar con un paciente y con una conjunto de objetos que forman parte de las herramientas requeridas para realizar el procedimiento en cuestión. Además, le permitirá realizar acciones de bioseguridad.

La aplicación no solo le permitirá al usuario realizar los procedimientos para poner en practica sus conocimientos sobre el mismo sino también evaluará al usuario, dándole al final de la partida una puntuación final y diciéndole cuales son los pasos que realizó de manera correcta y cuales de manera incorrecta, proporcionándole una pequeña información sobre su error.
%! TEX root = ../main.tex
\section{Tecnologías disponibles}

Para el desarrollo de videojuegos se utilizan programas o herramientas especializadas en ello llamadas motores de videojuegos. A continuación se da una breve introducción de lo que es un motor de videojuego o motor gráfico.

El término “motor gráfico” o “motor de videojuegos” hace referencia a una serie de rutinas de programación que permiten el diseño, la creación, el desarrollo y la representación gráfica de un videojuego\cite{videojuego:telechea}.

Además, la gran mayoría de estos motores ofrecen a su vez características y funciones que facilitan la construcción del videojuego, como el motor físico (software capaz de realizar simulaciones de ciertos sistemas físicos como la dinámica de un cuerpo rígido, el movimiento de un fluido o la elasticidad) o detector de colisiones, sonidos, scripting, animaciones, inteligencia artificial, redes, streaming, administración de memoria, etc\cite{videojuego:telechea}.

El motor de juego utilizado depende de las características que posea el videojuego que se quiere desarrollar. Por lo mismo, a continuación se da una breve descripción de los motores mas utilizados actualmente para que posteriormente se puede seleccionar el mas adecuado.

\subsection{UDK (Unreal Development Kit)}

Unreal Engine\cite{unrealengine} es el motor de juegos desarrollado por Epic Games, se ofrece bajo un plan de suscripción de 19 dolares por mes. El servicio de suscripción permite a los desarrolladores unirse a una comunidad dedicada a la construcción de grandes juegos y evolución de Unreal Engine. La suscripción se puede cancelar en cualquier momento.

UnReal Engine 4 permite desarrollar juegos para plataformas como Windows PC, Mac, iOS y Android. También es compatible con Xbox One y PlayStation 4. Existen además trabajos recientes sobre otras plataformas como HTLM5 y Linux.

Sin embargo existe una versión gratuita de UnrealEngine, el Unreal Development Kit o UDK. El UDK es la edición gratuita de Unreal Engine 3 que proporciona acceso al galardonado motor de juegos 3D y herramienta profesional que se utiliza en el desarrollo de videojuegos blockbuster, visualización arquitectónica, el desarrollo de juegos para móviles, modelos 3D, películas digitales y más. Utilizando UDK se pueden implementar juegos y aplicaciones en Windows PC, iOS y Mac.


\subsection{Blender Game Engine}

Blender Game Engine\cite{blender} es el motor de juego de Blender Foundation que permite crear aplicaciones 3D interactivas o simulaciones. La principal diferencia entre el Blender Game Engine y el Blender convencional está en el proceso de renderizado. En el motor Blender normal las imágenes y animaciones se construyen fuera de linea es decir, una vez generadas no pueden ser modificadas. Por el contrario Blender Game Engine genera las escenas de forma continua en tiempo real e incorpora facilidades para la interacción del usuario durante el proceso de renderización.

Procesa la lógica de sonido, de la física y la representación de simulaciones en orden secuencial. El motor esta
escrito en C++. Posee un editor que proporciona una profunda interacción con la simulación, su funcionalidad se
puede ampliar con scripts Python y esta diseñado para abstraer las características complejas del motor en una interfaz de usuario simple, que no requiere programación.

El motor de juego puede simular contenido dentro de Blender, sin embargo, también incluye la posibilidad de exportar en plataformas como Windows, Linux y Mac OS. También hay soporte básico para plataformas móviles con el proyecto Android Blender Player GSOC 2012.

\subsection{CryEngine 3}

CryEngine 3\cite{cryengine} es el motor de juegos desarrollado por Crytek. CryEngine es un motor avanzado para el desarrollo de juegos, películas, simulaciones de alta calidad y aplicaciones interactivas. Es uno de los procesadores de alta gama mas rápido en el mundo, con características diseñadas específicamente para Windows PC, PlayStation 3 y Xbox 360.

Existe una versión gratuita de CryEngine con todas las funcionalidades. CryEngine es  WYSIWYG (What You See Is What You Get). 

CryEngine posee un conjunto de herramientas utilizadas para el análisis de rendimiento. La ultima versión
de CryEngine, CryEngine 3 es el único motor con la última tecnología en iluminación, física e inteligencia
artificial si se está desarrollando en Windows PC, PlayStation 3 y Xbox 360.


\subsection{ShiVa3D}

ShiVa3D\cite{shiva} es un paquete para el desarrollo de juegos y aplicaciones 3D, viene en un formato fácil de usar, pero con un editor WYSIWYG (What You See Is What You Get) muy potente.

ShiVa puede exportar juegos y aplicaciones para más de 20 plataformas de destino, incluyendo móviles como iOS, Android, BlackBerry y Windows Phone, de escritorio como Windows, Mac OS X y Linux, los navegadores web con soporte Flash y HTML5, así como Consolas como la Xbox 360, PlayStation3 y Nintendo Wii. La herramienta de creación se ejecuta en Windows y Mac OS X. Software y hardware Apple son requeridos con el fin de construir juegos para OS X y iOS.

ShiVa también ofrece el ShiVa Server, el cual es la solución perfecta para aplicaciones multi-jugador y multiusuario. Dependiendo de su ancho de banda, puede manejar miles de jugadores al mismo tiempo, sincronizar sus clientes, e incluso dejarlos hablar el uno al otro a través de VoIP. ShiVa Server está disponible como una licencia independiente. ShiVa Server está disponible en Windows y Linux.


\subsection{Unity3D}

Unity\cite{unity3d} es un poderoso motor para el desarrollo de juegos, un poderoso motor derenderizado totalmente integrado con un conjunto completo de herramientas intuitivas y flujos de trabajo rápidos para crear contenido 3D interactivo, desarrollo multi-plataforma sencillo, miles de activos de calidad listos para usar en la tienda
de activos y una gran comunidad donde se intercambian conocimientos.

En Unity se pueden mezclar de forma muy sencilla elementos 2D con 3D. Posee un editor intuitivo y flexible, 
el nivel visual y de audio son de gran calidad, un sistema de animación poderoso y flexible. Mantiene el 
rendimiento y la calidad de la escena manteniéndola fluida cualquiera sea el tamaño de pantalla. 

Permite desarrollar juegos para múltiples plataformas. Para plataformas moviles iOS, Andriod, Windows Phone 8, BlackBerry 10; plataformas de escritorio como Windows, aplicaciones de la Windows Store, Mac y Linux; plataformas web como Internet Explorer, Mozzilla Firefox, Google Chrome; y plataformas de consolas como Xbox 360, Xbox One, Wii, Wii U, Nintendo 3DS.

La versión paga de Unity, Unity Pro permite además plataformas como PlayStation 4, PlayStation 3 y PlayStation VITA.

Debido a la alta popularidad de Unity, un paquete fue desarrollado por Facebook para colocar la API de
Facebook en un SDK escrito en C\#, el cual es atractivo y fácil de usar en la tienda de activos de Unity.

\section{Selección de plataforma}
\section{Hipótesis de la simulación}
\label{sec:hipotesis}


\observacion{Ver en la tesis de Tardon: \emph{no es necesario simular todos los
        pasos}.}
\observacion{Ver si hay que cambiar la palabra trivial}

Las escenas seleccionadas y definidas en~\ref{sec:seleccion_escenas} representan
las acciones que deben realizar los profesionales de enfermería a la hora de
realizar los procedimientos seleccionados, por limitaciones técnicas,
tecnológicas y de tiempo, no es posible realizar una simulación de todos los
pasos requeridos.

Los factores que influyen en que partes se simularán, que partes estarán
presentes solamente a través de opciones y que partes se omitirán son:

\begin{itemize}

    \item \textbf{Limitaciones técnicas}: acciones como la simulación del agua
        (necesarios para el lavado de manos), requieren de requisitos de
        hardware avanzados y un tiempo considerable de desarrollo. Las acciones
        que escapan al alcance del hardware y tiempo de los desarrolladores no
        son simuladas.

    \item \textbf{Importancia}: no todos los pasos definidos en el procedimiento
        oficial son necesarios, por ejemplo, la colocación de los elementos
        cerca del lugar de trabajo, es un paso necesario, pero es considerado un
        paso poco importante y trivial.

        La importancia es evaluada por profesionales del \Gls{iab}, los cuales
        dieron su opinión acerca de cada aspecto simulado, el mismo es tenido en
        cuenta para determinar la importancia de cada acción.

    \item \textbf{Facilidad de realización en la vida real o en el laboratorio}:
        ciertos pasos son triviales en la vida real pero requieren un esfuerzo
        significativo para ser simuladas, como por ejemplo el lavado de manos es
        un procedimiento al que los alumnos están acostumbrados.

        La facilidad que tienen los alumnos con las acciones fue determinada por
        profesores del \Gls{iab}, determinaron que acciones son triviales para
        los alumnos y cuales presentan mayores dificultades en su vida
        profesional.

        Otro aspecto que influye en la facilidad de realización de los
        procedimientos es la familiarización, si los alumnos están
        familiarizados con los procedimientos, estos no son simulados.

\end{itemize}

Estas hipótesis sirven para acotar el alcance de la simulación, definen qué se
simulará y cual es del detalle necesario para alcanzar las competencias básicas.

Existen hipótesis que son globales para toda la simulación, las mismas son:

\begin{itemize}

    \item \textbf{Comandos de voz con interfaz}: para enviar una petición al
        paciente (por ejemplo, preguntarle su nombre), no es necesario
        identificar las palabras del usuario, sino más bien detectar que ha
        hablado y listar las posibles acciones que se pueden realizar.

    \item \textbf{Utilización de la interfaz}: para realizar una acción con los
        elementos, es suficiente con presionar el mismo y seleccionar una acción
        de una lista de opciones, no hace falta emular todas las posibles.

    \item \textbf{Acciones de bioseguridad}:\todox{Definir bioseguridad} Las
        acciones de bioseguridad, se realizan a través de una opción en la
        interfaz gráfica.

\end{itemize}

Otras hipótesis, son tomadas por escena, las dos escenas simuladas son
diferentes en el modo de interacción del usuario con su entorno, por ejemplo, en
la escena de extracción de sangre, el usuario interactúa con el paciente a
través de objetos, en la evaluación de Glasgow, la interacción con el paciente
es directa.

\subsection{Extracción de sangre}
\label{sec:hemocultivo_hipotesis}

Se presentan los pasos mostrados en la sección~\ref{sec:hemocultivo_protocolo},
y adicionalmente se establecen las hipótesis punto por punto y las
consideraciones que deben ser tomadas.

\begin{itemize}

\item \textbf{Preparar el equipo}: la preparación del equipo es un aspecto muy
    importante del procedimiento, pero no es un punto único de la extracción de
    sangre, además las prácticas de los alumnos cubren completamente este paso
    según comentarios de los profesores. \emph{Este paso no se simula}.

\item \textbf{Explicación al paciente del procedimiento a realizar}: es un
    aspecto importante del procedimiento, pero la simulación de una conversación
    alumno-paciente es compleja, según comentarios de los profesores, es
    suficiente con que los alumnos sepan que lo deben realizar y en que moento,
    no es necesario simular la conversación en sí. \emph{Este paso se simula a
        través de un comando de voz con la interfaz}.

\item \textbf{Asepsia de las manos}: este paso forma parte de un área más amplia
    conocida como bioseguridad, la cual es un aspecto transversal a todos los
    procedimientos realizados por los enfermeros. 
    La implementación de una simulación del lavado de mano es compleja, y es un
    aspecto que, al igual que la preparación del equipo, está cubierta por los
    laboratorios, aún así, es necesario que los alumnos sepan en que momento
    deben realizar la asepsia de sus manos. \emph{Se simula este paso a través de una
        opción en la interfaz}, no se simulan los pormenores del lavado de manos.

\item \textbf{Llevar el equipo a la unidad en donde se encuentra el paciente}:
    este es un paso trivial que deben realizar los profesionales, la simulación
    de este proceso no es importante según comentarios de los profesores. Este
    proceso no tiene importancia según los profesionales del \Gls{iab}, \emph{este
        paso no se simula}.

\item \textbf{Vestirse con bata estéril, tapaboca y gorro}: al igual que la
    asepsia de las manos, es importante que los alumnos sepan que lo deben
    hacer, pero no es importante que se simule como lo hacen. 

    Los estudiantes están familiarizados con estas acciones, \emph{se simula el
        momento y el orden en el que el jugador lo hace} a través de una opción
    en la interfaz, no se simula el proceso en sí.

\item \textbf{Calzarse los guantes}: es un paso relacionado a la bioseguridad,
    es importante que se sepa en que momento debe realizarse, pero no es
    necesario simular el proceso. 
    \emph{Se simula el momento y el orden en el que se realiza}, no se simulan
    los pormenores de la acción.

\item \textbf{Ubicar al paciente en posición adecuada}: la ubicación del
    paciente durante la extracción de sangre es un factor determinante para que
    la extracción pueda ser realizada correctamente.

    Los alumnos están familiarizados con este proceso según opinión de los
    profesionales, \emph{este paso no se simula}, el paciente está en la
    posición adecuada al inicio de la simulación.

\item \textbf{Elegir la zona a puncionar}: existen varias partes del antebrazo
    donde se puede proceder a realizar una inyección, el conocimiento de las
    mismas, y el procedimiento para detectarlas, es un factor importante del
    proceso.
    
    Las venas del cuerpo humano se detectan palpando los antebrazos, y sintiendo
    el pulso del paciente, existen dos áreas donde el pulso es suficientemente
    fuerte como para sel sentido, estos puntos y el pulso del paciente deben ser
    detectables por el jugador.

    \emph{Los puntos donde se debe punzar deben ser identificables en la
        simulación}. 

\item \textbf{Colocación del torniquete}: La ubicación y el momento de la
    colocación del torniquete es de vital importancia para el procedimiento, el
    mecanismo utilizado para colocarlo no es relevante, pues el mismo es
    trivial.

    \emph{El hecho de colocar el torniquete es simulado}, el mecanismo para
    hacerlo no es importante.

\item \textbf{Solicitar al paciente que cierre el puño}: El momento en el cual
    se solicita al paciente que cierre la mano es vital para que el
    procedimiento de extracción sea satisfactorio.

    \emph{Este paso es simulado} a través de un comando de voz.

\item \textbf{Esterilizar la zona de punción}: la esterilización de la zona de
    punción es un factor de suma importancia para el procedimiento, así como el
    momento en el que se realiza, \emph{el jugador debería poder esterilizar} la
    zona antes de insertar la jeringa \todox{ver si se debe poner mas detalle para explicar 
    que no se simula enteramente este paso}.
    
\item \textbf{Extraer el protector de la aguja}: La extracción del protector de
    la aguja es un paso necesario, pero trivial, el hecho de retirar el
    protector de la jeringa \emph{no es un paso necesario para el logro de las
        competencias básicas necesarias}, por ello, no se simula.

\item \textbf{Puncionar la piel con la aguja}: este es un paso central en el
    procedimiento, en el se deben tener en cuenta aspectos como la posición
    donde se realiza la punción, y el angulo con el que ingresa la aguja.

    La posición donde se realiza la punción es importante por que depende de la
    ubicación donde se colocó el torniquete, y debe ser en uno de los puntos del
    brazo donde existen venas capaces de soportar el procedimiento, \emph{en
        cada brazo existen dos puntos donde se puede inyectar}.

    En cuanto al ángulo de punción, es un conocimiento importante que deben
    tener los alumnos, el conocimiento es teórico y según comentarios de los
    profesores, es un tema en el cual los alumnos tienen suficiente práctica en
    el laboratorio, \emph{No se simula el ángulo en el cual se inserta la
        jeringa}, es decir, la jeringa siempre se inserta en el mismo angulo.

\item \textbf{Tensar la zona de punción}: el proceso de tensar la zona de
    punción se realiza momentos durante la inserción de la jeringa, el mismo es
    trivial, y para simularlo se requiere que el usuario utilice tres dedos al
    mismo tiempo (dos para tensar y otro para realizar la punción), lo cual
    dificulta la utilización de la solución.

    \emph{Este paso no se simula} por la dificultad técnica que implica utilizar
    tres dedos para realizar una tarea, conjuntamente con la facilidad con que
    se realiza acción. 

\item \textbf{Remover el torniquete}: Al igual que en la colocación del
    torniquete, \emph{se simula el momento} de la extracción por que es
    importante, pero no los detalles de la extracción.

\item \textbf{Solicitar la apertura del puño}: el momento exacto donde se debe
    solicitar al paciente que abra la mano es fundamental para la realización
    correcta de la simulación. \emph{Este paso se simula} a través de un comando
    de voz.

\item \textbf{Extraer la muestra se sangre necesaria}: este es el paso central
    de la práctica, tanto el momento, como la forma es importante simular.

    La extracción de la sangre \emph{se simula} la acción de la extracción, pero
    no detalles como la sangre extraída, la velocidad de extracción y la fuerza
    necesaria por limitaciones de la tecnología.

\item \textbf{Presionar el brazo y extraer la aguja}: la presión del brazo para
    extraer la jeringa es un paso trivial, en cambio el momento en el que se
    extrae la aguja es conocimiento necesario para el procedimiento.

    \emph{No se simula esta acción}, pues es un paso al que los alumnos están
    acostumbrados y de simularse, agrega una complejidad adicional a la
    extracción, que es un paso que se realiza al mismo tiempo.

\item \textbf{Colocar algodón con alcohol en el punto de punción}: este paso es
    importante, tanto el momento en el cual se debe realizar, como la forma de
    realizarlo.

    \emph{Se simula la colocación del algodón}, así como el tiempo que se debe
    presionar el mismo.

\item \textbf{Sellar la muestra y enviarlo a su destinatario}: es necesario que
    los alumnos sepan que este paso debe ser realizado, pero los detalles del
    mismo, no son necesarios para el logro de las competencias básicas,
    \emph{este paso no se simula}.


\item \textbf{Retirar la bata, tapaboca, gorro y guantes}: es necesario que los
    alumnos sepan que deben desechar todos los elementos que fueron utilizados
    durante el proceso, la forma de hacerlo no es necesaria.

    \emph{Se simula el momento en el que se extraen los elementos} a través de
    una opción en la interfaz.

\item \textbf{Asepsia de las manos}: la asepsia final de las manos es un paso
    necesario para el procedimiento, así como la asepsia inicial, es importante
    que los alumnos sepan el momento en cual deben realizarlo, \emph{este paso
        se simula} a través de una opción en la interfaz.

\end{itemize}

Estas hipótesis afectan directamente el desarrollo de la aplicación, dictando
que partes del procedimiento se simulan y como, pueden ser vistas como
requisitos funcionales de la solución.

\subsection{Evaluación de glasgow}
\label{sec:glasgow_hipotesis}

En la sección~\ref{sec:glasgow_protocolo} se definieron los pasos necesarios
para poder llevar a cabo el procedimiento, este procedimiento es un paso
rutinario que deben realizar los profesionales para poder determinar rápidamente
el estado de conciencia de un individuo. 

El paso central de la práctica es la valoración del paciente y la generación del
diagnostico, los demás pasos, no colaboran en el desarrollo de la competencia
básica y según opiniones de los profesores no son importantes.

Así, es suficiente con simular al paciente y las reacciones que tiene ante las
acciones del jugador, las siguientes hipótesis se basan en la interacción
paciente/jugador.

El último paso del procedimiento es el registro final de la puntuación, el mismo
es utilizado como mecanismo de evaluación y el registro en sí no se simula, es
decir, se solicita al jugador que realice el diagnostico (mediante un menú),
pero no en la condiciones que se utilizan en la vida real (anotando en el
registro médico del paciente).

Para simular la medición del estado del paciente, se toman las siguientes
hipótesis:

\begin{itemize}
    \item La provocación de un estimulo doloroso al paciente es una accion
        necesaria, pero no así los detalles de la misma, \emph{se simula el
            estimulo con una opción} al personar al paciente en alguna
        extremidad.
    \item El dialogo jugador/paciente se realiza a través de comandos de voz, el
        nombre del paciente es una información que se conoce de ante mano y
        \emph{Se simulan 7 posibles preguntas}, que incluyen solicitudes de
        apertura ocular, movimiento de extremidades, y preguntas generales como
        el nombre del paciente, el día y el lugar.
\end{itemize}



%! TEX root = ../main.tex

\section{Solución}
\label{sec:solucion}

\observacion{Ver donde pone la interacción con la cámara}

Se describe la arquitectura propuesta para la realización de una juego serio, se
utiliza la guía básica definida por~\cite{pereira2009design} y descrita
en~\ref{sec:desarrollo}.

Esta sección se enfoca en los aspectos técnicos de la creación del juego serio,
las competencias básicas relacionadas con la educación (segundo paso de la guía
descrita en~\ref{sec:desarrollo}) se define en las
secciones~\ref{sec:glasgow} y~\ref{sec:hemocultivo}.

\subsection{Partes de la simulación}

La simulación se compone de tres elementos principales, entidades (que son
objetos de la vida real), acciones (que son provocadas por las entidades) y
eventos (que son el resultado de una acción). 

Existen otros elementos dentro de la simulación, como la sala y la iluminación,
los mismos son importantes para crear un entorno similar a la realidad y son
estáticos, es decir no interactúan con el usuario más que para limitar la
exploración en el escenario y/o resaltar aspectos relevantes.

\subsubsection{Entidades}

Cualquier objeto o componente en el sistema que requiera la representación
explícita en el modelo\cite{banks2000dm}. Las entidades tienen atributos. Los
atributos son las características de una determinada entidad que son exclusivos
de esa entidad.

Una entidad tiene en todo momento, un estado y una lista de acciones que
puede realizar, esta lista de acciones esta definida por el estado del mismo,
las condiciones en la que se encuentra el entorno y la práctica actual.

La entidad \enquote{Enfermero} es la que es controlada por el usuario, a través
de la interacción con la interfaz gráfica.

\subsubsection{Acciones}

Las entidades se comunican a través de acciones, las cuales pueden tener
diversos orígenes, siempre una entidad inicia una acción. Las acciones provocan
cambios en el ambiente y provocan eventos. Las acciones no solo las
realiza el usuario, sino cualquier entidad.

Como ejemplo, una acción es esterilizar las manos, esta acción provoca un
cambio en el ambiente (las manos ahora son estériles) y fue realizada por la
interacción entre el usuario y la interfaz gráfica.

\subsubsection{Eventos}

Los eventos son ocurrencias instantáneas que cambia el estado de un
sistema\cite{banks2000dm}, cada acción que se realiza provoca una acción, y los
eventos son la mecanismo que tiene una entidad para ser notificada de las
acciones de otras entidades.

\subsubsection{Interacción con el entorno}

El usuario se desenvuelve en un entorno de tres dimensiones, en el cual realiza las
actividades relacionadas a la práctica, se distinguen dos tipos de movimientos
principales que el usuario puede realizar:

\begin{itemize}
    \item \textbf{Alejamiento o acercamiento}: es el acto de acercar o alejar la
        cámara, y por consiguiente al usuario del paciente. Se realiza
        utilizando dos dedos, para realizar un acercamiento, mientras se
        mantiene presionada la pantalla con ambos dedos, se procede a alejar un
        dedo del otro, para realizar un alejamiento, se debe acercar ambos
        dedos.
    \item \textbf{Rotación}: se refiere al movimiento de rotación al rededor de
        un foco, que en ambas escenas es el paciente, para realizara, se utiliza
        un dedo, y se mueve en dedo en cualquier dirección, la cámara, se moverá
        en la dirección contraria.
\end{itemize}

\subsection{Grafo de estados}

La solución tiene varias escenarios, y dentro de cada escenario, existen varias
pantallas que muestran información relevante de acuerdo a la situación de la
simulación, en~\ref{fig:grafo_estados} se observa la interacción entre las
diferentes pantallas y escenarios.

\begin{figure}[H] 
\centering 
\includegraphics[scale=0.5]{propuesta/grafo_escenas.png}
\caption{Navegación entre escenarios y pantallas. Los escenarios son los
    rectángulos con un borde dos rayas, y las pantallas tienen un borde con una
    sola raya.}
\label{fig:grafo_estados}
\end{figure}

La solución inicia con un escenario denominado \emph{Inicio}, en el cual se
permite al usuario observar los detalles del entorno simulado a la vez que
muestra las opciones que permiten iniciar las diferentes prácticas, compartir
su actividad, enviar los datos de utilización y finalmente salir de la
simulación.

Si el usuario selecciona en el \emph{inicio} la opción \emph{Extracción de
    sangre}, se inicia el escenario denominado \emph{Extracción de sangre}, en
el cual el usuario puede realizar el procedimiento de extracción de sangre, si
el usuario selecciona la opción \emph{Fin}, la simulación termina y se dirige a
el escenario \emph{Pantalla de resultados}.

Al seleccionar la opción \emph{Evaluación Glasgow}, se inicia el escenario
denominado \emph{Glasgow}, donde el usuario debe evaluar a un paciente en el
centro del escenario, si el usuario presiona la opción \emph{Fin} se inicia la
pantalla denominada \emph{Evaluar al paciente}, donde el usuario diagnostica el
estado del paciente, y finalmente al presionar el botón \emph{Fin}, la
simulación finaliza y se inicia el escenario \emph{Pantalla de resultados}.

La opción \emph{Exploración Glasgow} es similar, la diferencia es que antes de
iniciar el escenario \emph{Glasgow}, aparece la pantalla \emph{Elegir estado de
    paciente}, en el cual el usuario selecciona un estado para que el paciente
actué de acuerdo al mismo, luego se inicia la escena \emph{Glasgow} y si el
usuario presiona el botón \emph{Fin}, se inicia el escenario \emph{Pantalla de
    resultados}.

La pantalla de resultados muestra la información acerca de las acciones que
realizo el usuario, proveyendo información a modo de retroalimentación, en esta
pantalla el usuario puede compartir sus resultados por las redes sociales,
reiniciar el escenario y finalmente, poder volver a la \emph{Pantalla de
    inicio}.


\subsection{Inicio}

\subsubsection{Descripción del entorno}

La escena mostrada como pantalla de inicio de la aplicación muestra como fondo la sala de 
un hospital con los elementos típicos de estos lugares, esta es la que se utiliza como 
escenografía principal en los escenas de los procedimientos, haciendo que el usuario entre 
en ambiente. Además de este fondo, se muestras varias opciones en forma de botones que serán 
descriptas a continuación y un mensaje en donde se recomienda al usuario el uso de auriculares.

\subsubsection{Opciones}

Las opciones disponibles en la pantalla de inicio son presentadas en forma de
botones los cuales tienen una breve descripción que identifica la función que
cumplen. 

\todox{Agregar descripción del escenario y si es necesario pantalla donde se
    pone el número}

\begin{itemize}
\item Botón \enquote{Enviar Progreso}: esta función envía toda la información
    acerca de la actividad que el usuario realizo en la aplicación a un servidor
    backend que se encarga de almacenar estos datos.
\item Botón \enquote{Salir de la simulación}: esta función permite salir de la
    aplicación.
\item Botón \enquote{Facebook}: esta función permite al usuario ingresar a su
    cuenta de Facebook.
\item Botón \enquote{Extracción de sangre}: esta función permite ingresar a la
    escena correspondiente al procedimiento de extracción de muestras de sangre
    permitiendo al usuario jugar una nueva partida.
\item Botón \enquote{Explorar Glasgow}: esta función permite ingresar a la
    escena correspondiente al procedimiento para explorar las reacción de un
    paciente con un diagnostico especifico de la escala de Glasgow permitiendo
    al usuario jugar una nueva partida.
\item Botón \enquote{Evaluar Glasgow}: esta función permite ingresar a la escena
    correspondiente al procedimiento para la valoración y diagnostico de la
    escala de Glasgow para un paciente con estado aleatorio permitiendo al
    usuario jugar una nueva partida.
\end{itemize}


\subsection{Extracción de muestras de sangre}

A continuación se detallan cada una de las opciones y formas disponibles de
interactuar con la escena del procedimiento de extracción de muestras de sangre.

\subsubsection{Descripción del entorno}

Al seleccionar el procedimiento de extracción de sangre en la pantalla de inicio 
la aplicación inmediatamente muestra la escena del procedimiento, se muestra una 
sala de hospital igual a la de la pantalla de inicio pero con un paciente en una 
de las camas, a este paciente es a quien se le realizara el procedimiento.

La posición inicial de la cámara se ubica en un ángulo en donde se puedan ver 
bien los brazos del paciente para facilitar al usuario la realización del 
procedimiento.

\todox{Agregar descripción}

\subsubsection{Descripción de la interfaz}

La interfaz principal de este escenario posee dos menús, uno a cada lado de la
pantalla, las opciones son representadas como botones que poseen una imagen
intuitiva\todox{Ver si no hay que agregar esto como hipótesis} que representa la
función que realizan. 

\subsubsection{Entidades}

En la extracción de sangre existen dos entidades principales, el paciente y el
usuario, cada entidad mantiene un estado independiente de la otra entidad.

El paciente es una entidad con estado complejo, el cual es constantemente
modificado por las acciones del usuario, en resumen, la información que contiene
el estado del paciente es:

\begin{itemize}
    \item \textbf{Jeringas}: un paciente puede tener cero o más jeringas en
        cualquier momento, no se limita la cantidad de jeringas que puede
        insertar el usuario.
    \item \textbf{Manos}: almacena el estado de las manos, el paciente reacciona
        ante peticiones del usuario, puede abrir o cerrar cualquier mano en
        cualquier momento.
    \item \textbf{Torniquetes}: es el conjunto de torniquetes que tiene
        actualmente el paciente, notar que los torniquetes pueden ser colocados
        en cualquier parte del brazo, pero existen lugares \enquote{correctos} y
        lugares \enquote{incorrectos}, la diferencia consiste en la distancia a
        los puntos de extracción, estos lugares están predefinidos.
    \item \textbf{Zonas esterilizadas}: son aquellas áreas del cuerpo que el
        usuario esterilizó, no existe un límite para las zonas esterilizadas.
        Una vez que una jeringa es extraída, una zona esterilizada pasa a estar
        contaminada y a la espera de que el usuario la presione.
    \item \textbf{Zonas presionadas}: son aquellas zonas que, una vez
        contaminadas por la extracción de una jeringa, han sido presionadas por
        el usuario.
    \item \textbf{Contaminado}: define si alguna acción realizada por el usuario
        provoco que el paciente se contamine, existen varias cadenas de eventos
        que pueden provocar que esto ocurra:
        \begin{itemize}
            \item Inyección de una jeringa cuando existe otra inyectada.
            \item Inyección en un lugar en lugares inadecuados.
            \item Inyección en un lugar no esterilizado.
            \item Inyección en un brazo cuya mano este abierta.
            \item Inyección fuera del alcance de los torniquetes actuales.
            \item Interacción con el paciente sin que el mismo tenga la mano
                estéril.
        \end{itemize}
        Es importante notar que este estado no es afectado directamente por una
        acción del usuario, sino por la consecuencia de una acción.
\end{itemize}

El \emph{usuario o enfermero} mantiene un estado en todo momento del cual
dependen sus acciones, por ejemplo, si la mano del paciente no esta
esterilizada, cualquier interacción con el paciente provocara que el paciente se
contamine.

\begin{itemize}
    \item \textbf{Manos}: almacena la información acerca de la esterilidad de
        las manos.
    \item \textbf{Guantes, gorro, bata y tapaboca}: almacenan la información
        acerca de los equipamientos que tiene el usuario en un momento dado.
    \item \textbf{Elemento actual}: es el elemento que esta activo en
        cualquier momento, un elemento es una herramienta de la vida real,
        como por ejemplo un torniquete, una gaza.
\end{itemize}

\subsubsection{Acciones}


\paragraph{Comando de voz}

Para representar la interacción del usuario con el paciente usando la voz se
implemento un menú que es activado y mostrado en pantalla cuando el usuario
habla, este menú muestra una seria de ordenes que el usuario le haría al
paciente normalmente hablándole. Las opciones de menú se detalla a continuación:

\begin{itemize}
\item Explicar procedimiento: esta función sirve para detectar si el usuario
    realizo la acción de explicar al paciente acerca del procedimiento. 
\item Abrir la mano izquierda: esta función le indica al paciente que abra su
    mano izquierda, como resultado el paciente realiza esta acción.
\item Cerrar la mano izquierda: esta función le indica al paciente que cierre su
    mano izquierda, como resultado el paciente realiza esta acción.
\item Abrir la mano derecha: esta función le indica al paciente que abra su mano
    derecha, como resultado el paciente realiza esta acción.
\item Cerrar la mano derecha: esta función le indica al paciente que cierre su
    mano derecha, como resultado el paciente realiza esta acción.
\end{itemize}

\paragraph{Opciones}

En este menú se despliegan los botones que representan las opciones de
bioseguridad. Es decir, acciones como lavarse las manos, calzarse guantes,
ponerse gorro, ponerse bata y ponerse tapaboca.

Los elementos de bioseguridad que actualmente tiene puesto el usuario se
representan como se describió anteriormente y se muestran en la parte baja de la
pantalla. Desaparece quitarse que en ese caso se representa al volver a
seleccionar la misma opción.

\paragraph{Elementos}

En este menú se despliegan los botones que representan a lo elementos que se
utilizan para realizar el procedimiento, una vez presionado ese elemento queda
seleccionado. Solo un elemento puede ser seleccionado a la vez. Si el mismo
botón se vuelve a presionar inmediatamente después de haber sido presionado, el
elemento queda de-seleccionado.

%Estas opciones van cambiando el estado del jugador y pueden ser seleccionados
%mas de una opción a la vez además de permitir de-seleccionar una opción
%volviendo a tocar el botón correspondiente. También posee la opción de
%finalizar la partida la cual manda al usuario a la pantalla de resultados.
%% REVISAR ESTO , el comienzo es sobre opciones y el final sobre elementos %%

La herramienta seleccionada actualmente para realizar el procedimiento se se
muestra en la forma descripta anteriormente arriba de la pantalla principal del
procedimiento. Esta imagen representa lo que actualmente tiene en las manos el
jugador. Desaparece al de-seleccionar o terminar de usar la herramienta.

\todox{Agregar colocación}
\todox{Agregar utilización}

\subsubsection{Eventos}
\subsubsection{Motor de reglas}
\subsubsection{Registro de actividad}


\subsection{Valoración de la escala de Glasgow}

\subsubsection{Descripción del escenario}

La interfaz principal de este escenario posee un botón de finalización de
partida al costado con una imagen intuitiva que representa la función que
realiza. Este botón manda al usuario a la pantalla de resultados.

\subsubsection{Entidades}
\subsubsection{Acciones} 
\subsubsection{Eventos} 
\subsubsection{Pantalla de diagnostico}
\subsubsection{Registro de actividad}
\paragraph{Elementos y opciones}


\subsection{Pantalla de resultados}
\subsubsection{Descripción del escenario}
\subsubsection{Retroalimentacion}
\subsubsection{Gamificacion}
\subsubsection{Reinicio}
\subsubsection{Puntuación}
\subsubsection{Tiempo utilizado}
\subsubsection{Facebook 2}

\subsection{Partes de la simulación}
    \subsubsection{Entidades}
    \subsubsection{Eventos}
    \subsubsection{Acciones}
    \subsubsection{Interacción con la cámara}

\subsection{Grafo del desarrollo}
% podemos poner acá un gráfico mas o menos así (ver graphviz)
%           /---> Hemocultivo --\
%          /                     \              /-> Reiniciar
%  Inicio ------> Glasgow 1 -------> Resultados --> Inicio
%         \ \                    /              \-> Facebook 2
%          \ \--> Glasgow 2 ----/
%           \---> Salir 
%            \--> Facebook 1
%             \-> Enviar resultados

\subsection{Pantalla de inicio}
    \subsubsection{Descripción del escenario}
    \subsubsection{Enviar datos}
    \subsubsection{Glasgow}
    \subsubsection{Extracción de sangre}
    \subsubsection{Facebook 1}

\subsection{Extracción de sangre}
    \subsubsection{Descripción del escenario}
    \subsubsection{Descripción de la interfaz}
    \subsubsection{Entidades} % definimos cuales son las entidades
        %\subsubsubsection{Estado del enfermero}
        %\subsubsubsection{Objeto seleccionado}
    \subsubsection{Acciones} % definimos cuales son las acciones de esas entidades
        %\subsubsubsection{Comandos de voz}
        %\subsubsubsection{Opciones} %bata,mano,guante,y eso
        %\subsubsubsection{Elementos}
            %\subsubsubsubsection{Colocación}
            %\subsubsubsubsection{Utilización}
    \subsubsection{Eventos} % definimos cuales son los eventos que se lanzan en este proceso
    \subsubsection{Motor de reglas} % se define como funciona el motor de reglas acá
    \subsubsection{Registro de actividad} % se define como se registra las acciones del usuario (cuales)


\subsection{Glasgow 1 y 2}
    \subsubsection{Descripción del escenario}
    \subsubsection{Entidades} % definimos cuales son las entidades
        %\subsubsubsection{Reacciones del paciente}
    \subsubsection{Acciones} % definimos cuales son las acciones de esas entidades
        %\subsubsubsection{Acciones sobre el paciente}
        %\subsubsubsection{Comandos de voz}
    \subsubsection{Eventos} % definimos cuales son los eventos que se lanzan en este proceso
    \subsubsection{Pantalla de diagnostico}
    \subsubsection{Registro de actividad} % se define como se registra las acciones del usuario (cuales)
    
\subsection{Pantalla de resultados}
    \subsubsection{Descripción del escenario}
    \subsubsection{Retroalimentacion}
    \subsubsection{Gamificacion}
    \subsubsection{Reinicio}
    \subsubsection{Puntuación}
    \subsubsection{Tiempo utilizado}
    \subsubsection{Facebook 2}

%! TEX root = ../main.tex
\section{Evaluación en tiempo de ejecución}

Las acciones realizadas por los usuarios dentro de la aplicación son evaluadas
para determinar si realizo o no el procedimiento de manera correcta y así
brindarle información al usuario sobre su rendimiento.

En esta sección se explica como son evaluados las acciones de los usuarios para
los diferentes procedimientos simulados.

\subsection{Extracción de muestras de sangre}

Para la evaluación de las acciones del usuario en este procedimiento se utilizo
un motor de reglas denominado \enquote{Acciones condicionadas por eventos}. A
continuación se explica en detalle cada aspecto relacionado tanto al motor como
a la forma de evaluación del rendimiento del usuario.

\subsubsection{Acciones condicionadas por eventos}

Un evento es la ocurrencia de un hecho en particular, y son identificados por un
nombre y un conjunto de parámetros, por ejemplo, cuando un evento es cuando el
enfermero inserta una Jeringa, el nombre de este evento es
\enquote{jeringa}.inserted, y sus parámetros podrían ser el lugar y el tiempo
de la inserción, así, la influencia del estudiante en la simulación es una
sucesión de eventos.

Por cada acción que realiza el usuario dentro de la simulación, existe un evento
relacionado, por consiguiente, es razonable estudiar algunos eventos para
determinar si los pasos realizados corresponden con los deseados. 

Para determinar si una sucesión de eventos es la correcta, se definen reglas,
una regla es una asociación de una condición y una acción, la condición define
si el entorno es el adecuado para realizar una acción, la cual es un
procedimiento que realiza la lógica deseada.

Las \gls{eca} son aquellas que son activadas una vez que se cumplen determinados
eventos\cite{bailey2004event}. En las bases de datos relacionales, son conocidos
como triggers, es decir, una base de datos relacional (u orientada a objetos) es
un motor de reglas \gls{eca}\cite{bailey2004event}\cite{behrends2006combining}.

Las mismas pueden ser utilizadas para notificar que un determinado conjunto de
eventos ha ocurrido\cite{bailey2004event}, así como servir para almacenar
información acerca de la utilización de un determinado recurso.


\paragraph{Motivación}

Las reglas del tipo \gls{eca} permiten reaccionar a determinados eventos, en
forma de una única regla, la cual facilita la declaración de las
mismas\cite{bailey2004event}.

Son principalmente útiles para analizar el comportamiento en tiempo real de un
sistema en una forma
reactiva\cite{bailey2004event}\cite{de2001eca}\cite{bailey2002analysis}, esta
característica esta impulsada principalmente por que son ejecutadas después de
la ocurrencia de un evento, y el entorno no es modificado, pudiendo así acceder
al mismo entorno que el qué lanzo el evento.

Definir si las acciones de un usuario son correctas utilizando un motor
\gls{eca} es sencillo desde el punto de vista que sólo se deben definir un
conjunto de acciones que se deben realizar, y agregar una acción que verifica si
los pasos realizados fueron los correctos.

\paragraph{Declaración}

Una \gls{eca}, se define como\cite{bailey2004event}\cite{behrends2006combining}:

\begin{center}
	 Cuando ocurren una serie de \emph{eventos}, y se cumple una
	 \emph{condición}, entonces realizar una \emph{Acción}.
\end{center}

Los \emph{eventos} determinan cuando una regla debe ser activada, los mismos se
dividen en dos categorías\cite{behrends2006combining}, primitivos y compuestos,
los primeros son detectables, por ejemplo, cuando se inserta una jeringa, y los
compuestos, son la combinación de uno o más
primitivos\cite{bailey2004event}\cite{behrends2006combining}. Los eventos
compuestos, se unen mediante:
\begin{enumerate*}[label=\itshape\alph*\upshape)]
\item conjunción (\emph{y}),
\item disyunción (\emph{o}), y
\item secuencia (\emph{entonces}).
\end{enumerate*}
Sin embargo, no siempre son necesarios todas las posibles combinaciones, y las
combinaciones sencillas son más fáciles de optimizar y
probar\cite{bailey2004event}.

La \emph{condición} de una regla determina si el entorno es el necesario para que la
regla sea activada, en esta condición el entorno que lanzó el evento esta
disponible.

La \emph{acción} a ejecutar describe la lógica que debe ser ejecutada cuando se han
lanzado los eventos y la condición de la regla se ha cumplido.

\paragraph{Dependencia entre reglas}

Las reglas pueden depender de otras reglas, lo cual se puede ver como que la
finalización de una regla es un evento que otra regla espera para poder ser
activada.

Las reglas pueden agregar información a un contexto compartido por todas las
reglas, de esta manera, se puede pasar parámetros entre distintas reglas, por
ejemplo, la regla \emph{Retirar Torniquete}, depende de la regla \emph{Insertar
Torniquete}, pero debe responder solamente al torniquete que ha activado
la regla de inserción, es decir, el usuario puede extraer varios torniquetes, y
la regla no debe activarse, hasta que se extraiga el torniquete que activo la
primer regla.

Así, la regla \emph{Retirar Torniquete} depende de la regla \emph{Insertar
Torniquete}, y esta relación entre reglas, se da en dos
formas\cite{bailey2004event}:

\begin{itemize}
\item  \emph{Dependencia fuerte:} la regla \emph{Retirar Torniquete} solamente podrá
	ser elegida para ser lanzada cuando la regla \emph{Insertar Torniquete}
	haya sido cumplida.
\item  \emph{Dependencia de contexto}: la regla \emph{Retirar Torniquete} no se
	activará cuando los eventos a los que escucha se terminen, sino cuando
	los eventos a los que escucha sean lanzados con los parámetros adecuados
	(se extraiga el torniquete que lanzo la regla de inserción).
\end{itemize}

\paragraph{Representación}

La definición de las reglas se realiza de la siguiente forma;
\begin{algorithm}[H]
\caption{Creación de regla de verificación de calzado de guantes}
\label{alg:rule:guante}
\lstset{style=sharpc}
\begin{lstlisting}
Rule.New("Regla de verificacion de calzado de guantes").
     When("enfermero.guantes.calzar").
     Then(e => e.Patient.ManosLimpias()).
\end{lstlisting}
\end{algorithm}
%TODO agregar indice de algoritmos

La regla anterior controla que el estudiante ha realizado la acción ``Calzarse
los guantes'', y en ese momento tenga las manos limpias, la variable \emph{e},
es el entorno, y a través de la propiedad \emph{Patient} obtiene el estado del
paciente en ese momento.

\paragraph{Modelo de ejecución}

Para ejecutar un motor de reglas del tipo \gls{eca}, se debe tener en cuenta
principalmente dos factores, 
\begin{enumerate*}[label=\itshape\alph*\upshape)]
\item  Como se verifica el cumplimiento de cada regla, y, 
\item  Que ocurre cuando varias reglas son lanzadas al mismo tiempo
\end{enumerate*}.

Para ambos casos se puede tomar un enfoque \emph{inmediato}, es decir que
inmediatamente cuando se lanza un evento, o se cumple una condición, se ejecuta
la regla. Además existen otros dos modos de ejecución, \emph{deferida}, y
\emph{desacoplada}, en la primera, se espera hasta que el lanzador del evento
culmine su trabajo, y luego se ejecuta la regla, pero en la misma unidad de
trabajo, mientras que en la ejecución desacoplada, se encolan los trabajos y
otro hilo es el encargado de ejecutar las reglas. Estos modos están inspirados
en las bases de datos relacionales, el deferido se ejecuta en la misma
transacción, y el desacoplado, inmediatamente después de que la transacción
termine\cite{bailey2004event}.

La propuesta implementada, utiliza una ejecución inmediata, principalmente por
la sencillez de las reglas, es decir, las reglas no realizar un proceso complejo,
solamente controlan el estado del entorno y lo validan.

Además, la ejecución inmediata es importante por que el entorno no sufre
modificaciones entre el evento lanzado y la ejecución de la regla, según
\cite{bailey2004event}, este es el factor más importante para determinar el tipo
de ejecución deseado.



\paragraph{Estados de una regla}

Una regla puede estar en uno de los siguientes estados:

\begin{description}
\item[BEGIN] Es una regla que recién fue creada, no realiza ninguna
	acción.
\item[WAITING\_FOR\_RULE] Es un estado en el que esta esperando que otras reglas
	sean lanzadas. En este estado, es un suscriptor de las reglas por la que
	espera, y no forma parte del ciclo de ejecución del motor de reglas.
\item[WAITING\_FOR\_EVENT] Es un estado en el que esta escuchando a que sean
	lanzados los eventos a los que escucha, este es el estado principal. En
	este estado, es un suscriptor de los eventos por los que espera, y no
	forma parte del ciclo de ejecución del motor de reglas. Se diferencia
	del estado anterior, en que los eventos escuchados pueden ser lanzados
	por cualquier objeto del entorno, no necesariamente una regla.
\item[WAITING\_FOR\_CONDITION] La regla ya no espera por ningún evento y las
	reglas de las que depende ya han sido lanzadas, se verifica cada cierto
	tiempo si el entorno cumple con una condición definida. 
\item[FINISH] La regla ha sido lanzada, con un resultado no determinado, se pudo
	haber cumplido, como no, es el estado final de una regla. Cuando una
	regla llega a este estado, se lanza su evento de finalización.
\end{description}

Una regla puede estar en solo un estado, y solamente se permite que el estado
avance, desde \emph{BEGIN} hasta \emph{FINISH}.


\paragraph{Ciclo de vida}

Cuando una regla es definida, y insertada al motor de reglas, inmediatamente
pasa al estado \emph{BEGIN}, luego se verifica si la misma depende de otras
reglas, sí este es el caso, pasa al estado \emph{WAITING\_FOR\_RULE} y escucha a
los eventos de finalización de las reglas anteriores.

Una vez que las reglas anteriores han sido finalizadas, la regla pasa al estado
\emph{WAITING\_FOR\_EVENT} sí deben escuchar por algún evento, en caso contrario
pasan al estado \emph{WAITING\_FOR\_CONDITION}.

Una vez que la regla está en estado \emph{WAITING\_FOR\_CONDITION}, pasa a un
motor que ejecuta su condición cada cierto tiempo, si la condición se cumple, la
regla se ejecuta, y la misma pasa a estado \emph{FINISH}, momento en el cual
notifica a las reglas que dependen de ella que ha sido lanzada.

Una vez que la regla esta en estado \emph{FINISH}, la misma sale del esquema de
ejecución, y solo esta disponible para obtener resultados.

Según el ejemplo de la regla definida en el código\ref{alg:rule:guante}, la
regla al terminar de ser construida pasa a estado \emph{BEGIN}, al no depender
de otras reglas, pasa inmediatamente al estado \emph{WAITING\_FOR\_EVENT},
cuando es lanzado el evento, la regla ejecuta la acción y pasa al estado
\emph{FINISH}.

\paragraph{Motor de ejecución}

Un motor de reglas \gls{eca}, requiere de un proceso que evalúe constantemente
las reglas para verificar si las mismas deben ser lanzadas o
no\cite{bailey2004event}\cite{galton2002two}, este motor puede utilizar el
algoritmo de RETE\cite{de2001eca} para realizar esta verificación, en la
propuesta presentada, la cantidad de reglas definidas, y la no dependencia
circular entre ellas, hace innecesario la implementación de tal
algoritmo\cite{de2001eca}. 

El motor de reglas actúa sobre aquellas reglas en estado
\emph{WAITING\_FOR\_CONDITION} e invoca al procedimiento que se encarga de
validar si la regla puede ser activada (el procedimiento es único por cada
regla), si el mismo determina que la regla puede ser lanzada, el motor ejecuta
la acción de la regla y modifica el estado de la regla a \emph{FINISH}.


\subsubsection{Definición de reglas}

La reglas del procedimiento de extracción de sangre fueron definidas de acuerdo
a los pasos requeridos según el protocolo del procedimiento y al orden en el que
son requeridos. Es decir, cada paso del protocolo tiene asociado una regla
dentro del motor que lo representa y las condiciones asociadas a cada regla
están determinadas por el orden en que deben realizarse dentro del protocolo.

Cada regla tiene una o mas condiciones que deben ser cumplidas para que un paso
del protocolo realizado se considere correcto.

\subsubsection{Retroalimentación y puntuación final}
\label{sec:puntuacion_hemocultivo}

Cada regla tiene asociado un peso, de acuerdo a la dificultad de realizar el
paso, este peso es utilizado al final de la partida para darle una puntuación al
usuario acerca de su rendimiento en la partida.

Además, un regla puede quedar en uno de diferentes estados al final de la
partida como se mostró anteriormente, cada uno de esos estados posee un
significado en el contexto del procedimiento y por lo tanto tiene información
asociada para que al final de la partida se muestre una retroalimentación
correcta al usuario por paso.

\subsection{Valoración de la escala de Glasgow}
\label{sec:puntuacion_glasgow}

Para la evaluación del rendimiento del usuario en el momento de llevar a cabo el
procedimiento de valoración de la escala de Glasgow se tuvo un enfoque
completamente diferente al del procedimiento de extracción de muestras de sangre
debido a la naturaleza propia del procedimiento. 

Como se explico anteriormente, el paciente puede estar en ciertos estados
específicos dentro de la escala, y además dentro de cada estado reacciona de un
forma en particular por lo tanto, al inicio de la partida un componente interno
de la aplicación selecciona de forma aleatoria un estado para el paciente, de
forma tal que cada vez que una partida sea jugada no se repitan los estados de
forma seguida.

El estado aleatorio del paciente es guardado en una variable que no es
modificada hasta que se reinicie la partida. Al final de la partida, la
aplicación pide al usuario que valore el estado del paciente que le fue
presentado, una vez que el usuario confirme su respuesta la aplicación la
compara con el estado guardado y de esta forma puede informar al usuario acerca
de su rendimiento en el diagnostico.

Además, cada posible respuesta dada por el usuario contiene información
relacionada al contexto del procedimiento y a la situación actual presentada la
cual es utilizada como retroalimentación al final de la partida. La puntuación
final dada depende de la cantidad de valoraciones correctas dadas por el usuario
para la respuesta verbal, motora, ocular y nivel de gravedad del paciente.










\section{Inconvenientes de diseño}

Los mayores inconvenientes de diseño de la aplicación se dieron en el momento de
validar tanto el contenido de la aplicación como la interfaz de usuario, para
sobrellevar estos inconvenientes fueron requeridos la intervención de terceros.

A continuación se explica en detalle cada uno.

\subsection{Interfaz de usuario}

Como parte del diseño y desarrollo de la aplicación propuesta como solución se
realizó una prueba de interfaz de usuario con alumnos de la carrera de
ingeniería en informática de la Facultad Politécnica de la Universidad Nacional
de Asunción, estas pruebas fueron realizadas con personas que están
acostumbradas al uso de interfaces similares y que de hecho pueden ser mas
criticas a la hora de evaluarlas. Esta prueba se explica en detalle en el
capítulo~\ref{chap:evaluacion} y los resultados en el capítulo~\ref{chap:analisis}.

Principalmente son dos las cualidades de una interfaz gráfica que se pueden
someter a prueba: su funcionalidad y su usabilidad. Con la primera se pretende
responder preguntas como ¿Se puede usar cierta función?, ¿Funciona como se
espera?, o ¿Es correcta?; y con la segunda, a ¿Puede el usuario utilizar
fácilmente la función?, o ¿Su uso es intuitivo y fácil de
aprender?\cite{fragaverificacion}.

Las pruebas de interfaces de usuario ayudan a que los usuarios puedan
concentrarse mas en el problema en vez de poner los esfuerzos en recordar todas
las opciones que ofrece la aplicación que se utiliza para resolver el
problema\cite{horowitz1993graphical}.

Luego de las pruebas con usuarios con experiencia en interfaces móviles, se
hicieron correcciones a los problemas encontrados en la interfaz, los mayores
inconvenientes fueron con respecto a la usabilidad y la interacción tanto con el
entorno como con los objetos dentro de la simulación Estas correcciones, como
paso posterior, fueron probadas por profesores de la carrera de enfermería del
Instituto Andrés Barbero los cuales dieron su visto bueno.

Otra de las razones por las cual la prueba fue realizada con alumnos que no
formaban parte de la población a la que iba dirigida la aplicación, es la poco
disponibilidad de tiempo con la que cuentan los alumnos de enfermería y mas aun
los profesionales que están encargados de su aprendizaje.

\subsection{Validaciones de contenido}

Llamamos validación de la simulación o la aplicación desarrollada al hecho de
que el contenido de la misma sea correcto y además que la forma de realizar o
representar dicho procedimiento este acorde al mismo. Este tipo de validaciones
fueron realizadas reiteradamente en reuniones con distintos profesores de la
carrera de enfermería del Instituto Andrés Barbero.

Cada corrección solicitada fue evaluada y aprobada posteriormente por los
mismos. Como validación final la aplicación fue presentada en totalidad frente a
un plenario de cuatro profesores del instituto.

El mayor inconveniente en cuanto a las validaciones fueron la forma de
representación tanto de la información como de la simulación de objetos.


%! TEX root = ../main.tex

\chapter{Evaluación y resultados}
\label{chap:evaluacion}


En este capítulo se detallan las metodologías utilizadas para la evaluación de la 
solución. Estas metodologías están orientadas a valorar los 
aspectos pedagógicos, de diseño, de implementación y de evaluación de la solución 
para determinar su aplicabilidad como herramienta de apoyo al proceso de 
aprendizaje.

%En este capítulo se definen las herramientas diseñadas y utilizadas para evaluar
%la utilización de los juegos serios en el aprendizaje, estas herramientas están
%orientadas a la validación de las hipótesis planteadas en la
%sección~\ref{sec:hipotesis}, así como la evaluación de aspectos pedagógicos, de
%utilidad y de la participación activa del usuario. Estas herramientas se
%utilizan para evaluar a \gls{nombre}.
%
%La evaluación se divide en cinco partes principales:

Las metodologías utilizadas son las siguientes:

\begin{itemize}

    \item \textbf{Prueba preliminar de usabilidad:} es una prueba inicial para
        medir la calidad de la interfaz y la interacción con la misma, esta
        evaluación es realizada con personas no relacionadas al área de
        enfermería, específicamente con alumnos de la carrera de
        Ingeniería en Informática de la \gls{fpuna}.

        La prueba es llevada a cabo durante el desarrollo de la solución a
        diferencia de las demás, las cuales son realizadas una vez terminada la
        solución.

    \item \textbf{Encuesta para determinar la muestra:} es una encuesta acerca del nivel de
        acceso a la tecnología que poseen los alumnos del 4to año  de la carrera 
        de licenciatura en enfermería del \Gls{iab},
        de ahora en más la población objetivo, esta encuesta sirve para definir
        la muestra.

        Se realizar la distribución de \gls{nombre} a los
        alumnos que cumplen con los requisitos mínimos y desean participar de
        las pruebas.%, luego se les realiza la siguientes pruebas para medir su
%        nivel de aceptación y aprendizaje, así como el tiempo y frecuencia de
%        utilización de la solución.

    \item \textbf{Encuesta para evaluar la solución:} es una encuesta realizada
        a cada sujeto de la muestra, donde se busca la opinión del mismo acerca
        de la solución y factores relacionados a la misma. 

    \item \textbf{Encuesta para evaluar conocimiento:} es un cuestionario que es
        completado por la población objetivo, donde se mide el conocimiento de
        los mismos. Con los resultados se busca contrastar el rendimiento de la muestra 
        con respecto a los demás alumnos.
        
        %, se utiliza a los \revisar{También participan} alumnos que
%        no forman parte de la \fixme{muestra}{}, como grupo de control.

%    \item Encuesta para evaluar el conocimiento: es un cuestionario cuyo
%        objetivo es medir el nivel de conocimiento sobre los temas simulados en
%        \gls{nombre}. 
%        
%        La muestra en esta encuesta es la población entera, los alumnos que no
%        utilizaron la aplicación forman un grupo de control.
        
    \item \textbf{Registro de actividades:} es información almacenada por la
        solución automáticamente, y contiene datos acerca del uso y el desempeño
        del alumno.
        
\end{itemize}

Además de describir las metodologías en detalle, también son mostrados y analizados 
los resultados obtenidos en cada una de las evaluaciones realizadas. Los resultados 
son expuestos en forma de tablas y gráficos para mejorar su interpretación.

%El capitulo define los objetivos de la evaluación, describe brevemente conceptos
%transversales a las técnicas utilizadas. Luego, por cada prueba realizada, se
%definen las metodologías, métricas y variables utilizadas en cada parte de la
%evaluación, y se muestran los resultados de las pruebas, al final del capítulo
%se muestran correlaciones entre las variables estudiadas.

\section{Objetivos}


\begin{itemize}
    \item Determinar el nivel de aceptación de la propuesta.
    \item 

\end{itemize}

%! TEX root = ../main.tex

\section{Métricas generales utilizadas en la evaluación}

En esta sección se describen aquellas métricas que son utilizadas por 
más de una metodología para la evaluación de la solución, las 
cuales se consideran que son importantes detallar.
%En esta sección se definen las métricas que son utilizadas por las pruebas y
%encuestas que forman parte de esta evaluación.

Una de estas métricas es la escala de \textit{Likert}, la cual es una métrica utilizada 
en la \emph{Encuesta para evaluar la solución}
y en la encuesta correspondiente a la \emph{Prueba preliminar de usabilidad}. Otra métrica 
utilizada es la correlación de \textit{Pearson}, esta
métrica es utilizada para medir el grado de relación entre variables de las
encuestas realizadas, los registros de actividades, entre otros.

Cabe destacar que en~\cite{norman2010likert} se demuestra que, aunque el tamaño
de la muestra sea pequeña y los datos no puedan ser distribuidos normalmente o
los datos sean de escalas de tipo \textit{Likert}, los métodos paramétricos como
el análisis de varianza, la regresión y la correlación pueden ser utilizados.


\subsection{Escala de Likert}
\label{sec:likert}

Para la valoración de las variables medidas en la \emph{Prueba preliminar de usabilidad} y 
la {Encuesta para evaluar la solución} se utiliza la escala de
\textit{Likert}\cite{Allen:2007} de 7 valores posibles. La escala de
\textit{Likert} es utilizada para permitir a las personas indicar cuánto están
de acuerdo o en desacuerdo con respecto a ciertos puntos. Los valores
utilizados, son:

\begin{enumerate}
    \item Totalmente en desacuerdo.
    \item En desacuerdo.
    \item Parcialmente en desacuerdo.
    \item Neutral.
    \item Parcialmente de acuerdo.
    \item De acuerdo.
    \item Totalmente de acuerdo.
\end{enumerate}

Una vez valoradas y registradas todas las respuestas y con el objetivo de
eliminar las tendencias en la forma en la que son completadas las
encuestas\cite{Fischer2010} se utiliza el método de \emph{Doble Estandarización}
recomendado en~\cite{Pagolu2011}. Este método, consiste en dos
estandarizaciones, la primera por fila, que en este caso representa a los
individuos y la segunda por columna donde cada columna representa una de las
diferentes preguntas de la encuesta.

Siendo:
\begin{itemize}
	\item $\min_i$ la respuesta de menor valor del usuario $i$.
	\item $\max_i$ la respuesta de mayor valor del usuario $i$.
\end{itemize}

Para cada respuesta $s$ del usuario $i$, el valor ajustado, por la primera 
normalización, $s_1$ se define como:

\begin{equation}
s_1{_i}=\frac{s-\min_i}{\max_i-\min_i}
\end{equation}

%\observacion{Considerar resumir}
Y luego siendo:
\begin{itemize}
	\item $groupmin_i$ la respuesta ajustada de menor valor en el grupo $i$.
	\item $groupmax_i$ la respuesta ajustada de mayor valor en el grupo $i$
\end{itemize}

Para cada respuesta ajustada $s_1{_i}$ del usuario $i$, el valor ajustado $sa_i$ se
define como:	

\begin{equation}
sa_i=\frac{s_{1_i}-groupmin_i}{groupmax_i-groupmin_i}
\end{equation}

Obteniendo así un valor normalizado, tanto por individuo, como por pregunta, en
el rango $0$ y $1$.

Para la valoración absoluta de cada  item se utiliza la media de cada columna o
respuesta a una pregunta de la encuesta.

Siendo:
\begin{itemize} 
\item $r_{k_i}$ la respuesta del usuario $i$ a la pregunta $k$.
\item $t_k$ la cantidad total de usuarios que respondieron la pregunta $k$.
\end{itemize}

El puntaje promedio de cada pregunta o item evaluado  $p_k$ en la encuesta se
define como:

\begin{equation}
p_k = \frac{\sum_{i=1}^n{r_{k_i}}}{t_k}
\end{equation}

%\subsubsection{Manejo de información faltante}
%\label{sec:informacion_faltante}
%\observacion{No repetir tanto existe}
%
%En toda encuesta pueden haber preguntas que no son respondidas por los encuestados, 
%en este tipo de situaciones existen tres posibles formas de categorizar el 
%patrón de ocurrencia de la falta de 
%respuestas\cite{leite2010performance,tsikriktsis2005review}:
%
%\begin{description}
%    \item[Información faltante completamente aleatoria:] cuando la información
%        faltante es independiente de la variable medida y de otras variables.
%    \item[Información faltante aleatoria:] cuando la información faltante depende
%        de otras variables, pero no de la variable en sí. 
%    \item[Información faltante no aleatoria:] cuando hay una relación entre la
%        información faltante y el valor de la variable.
%\end{description}
%
%Una vez categorizado el patrón de ocurrencia, existen a su vez tres
%mecanismos~\cite{tsikriktsis2005review} principales para lidiar con información
%faltante como son la eliminación, el reemplazo y los  procedimientos basados en
%modelo.~\cite{tsikriktsis2005review} recomienda utilizar un mecanismo de
%reemplazo para escalas del tipo \textit{Likert}.
%
%Las técnicas de reemplazo se clasifican en tres grandes
%grupos\cite{tsikriktsis2005review}:
%\begin{enumerate*}[label=\itshape\alph*\upshape)]
%\item basadas en el promedio,
%\item basadas en regresión, e,
%\item imputación \emph{hot deck}.
%\end{enumerate*}
%
%\fixme{De estas técnicas se seleccionó la sustitución}{Resaltar}. Basada por
%promedio ya que las relaciones entre las variables son bajas y los datos
%faltantes son menos del $10\%$. La sustitución basada por promedio se divide
%nuevamente en tres grupos\cite{tsikriktsis2005review}; promedio
%\begin{enumerate*}[label=\itshape\alph*\upshape.]
%\item total,
%\item del subgrupo, y,
%\item por caso.
%\end{enumerate*}
%
%La sustitución por promedio total es elegida debido a que la relación entre la
%variable que falta y las demás variables en los datos es relativamente baja, es
%fácil de usar y retiene la muestra. La sustitución por promedio total se realiza
%obteniendo el promedio de todas las respuestas de la pregunta cuya respuesta
%falte, la sustitución de subgrupo es similar, solo que se limita a aquellos
%sujetos del mismo subgrupo del sujeto que no respondió, y finalmente, la
%sustitución por caso, es el promedio de las respuestas válidas del sujeto.

\subsection{Correlación de variables aleatorias}
\label{sec:correlacion}

Las correlaciones se utilizan durante una etapa exploratoria o de observación de
la investigación para determinar las variables que tienen al menos una relación
estadística con cada uno de los diseños experimentales. Las correlaciones
también se utilizan para determinar el grado de asociación entre variables
dependientes e independientes. Por otro lado, el coeficiente de correlación se
utiliza comúnmente para cuantificar el grado de asociación entre dos variables
\cite{BoslaughStatistics2008}.

La correlación de Pearson\cite{BoslaughStatistics2008} mide la relación que
existe entre dos variables, $X$ e $Y$, el mismo esta comprendido entre $-1$ y
$1$, en su punto más bajo ($-1$) indica que una de las dos variables crece mientras
la otra decrece, y en su punto más alto ($1$), indica que ambas crecen o
decrecen conjuntamente, el valor $0$, indica que no existe una relación entre
ambas variables.

El coeficiente para las variables $X$ e $Y$ está dado por:

\begin{equation}
r = \frac{\sum_{i=1}^n{(\frac{x_i-\bar{x}}{s_x})({\frac{y_i-\bar{y}}{s_y}})}}%
{n - 1}
\end{equation}

donde:

\begin{itemize}
    \item ($x_i$, $y_i$) es el conjunto de coordenadas de las variables $X$ e $Y$.
    \item $\bar{x}$ es la media de la variable $X$.
    \item $\bar{y}$ es la media de la variable $Y$.
    \item $s_x$ es la desviación estándar de la variable $X$.
    \item $s_y$ es la desviación estándar de la variable $Y$.
    \item $n - 1$ son los grados de libertad.
\end{itemize}

\section{Prueba preliminar de usabilidad}
\label{sec:interfaz}

Durante el desarrollo de la solución se realizó una prueba para evaluar la 
interfaz de usuario, específicamente buscando la retroalimentación de usuarios 
acostumbrados a tecnología similar a la utilizada en la solución.

Esta prueba ayuda en el proceso de diseño e implementación de la solución con 
las características mencionadas en los objetivos del trabajo y acorde a los 
requerimientos. De esta manera se pueden identificar los aspectos que deben 
ser mejorados.

La prueba consta de dos partes importantes involucradas en la recolección
de datos para su posterior análisis. Estas partes son las siguientes:

\begin{description}

\item[Simulación:] luego de una explicación acerca de las funciones y manejos
    generales de la solución por parte de los encargados de la prueba, cada usuario
    completa una tarea que consiste en realizar el procedimiento de venopunción con la 
    solución, como ayuda, recibe una hoja con una lista de todos los pasos 
    necesarios para llevar a cabo el procedimiento.
    	
    Las simulaciones son grabadas con programas de captura de pantalla, así
    como por detectores de eventos táctiles.
    	
\item[Encuesta:] posteriormente se le provee una encuesta a cada
    usuario la cual es utilizada para obtener una idea general acerca de la
    calidad de la simulación según la percepción de los usuarios. Esta encuesta 
    contiene preguntas que son medidas mediante la escala de tipo Likert. 

\end{description} 

\subsection{Muestra}

La prueba de usabilidad de la interfaz de usuario se realiza con alumnos de la
carrera de Ingeniería en Informática de la \Gls{fpuna}, sin experiencia previa
tanto con la solución como con los procedimientos simulados, pero sí
familiarizados con la utilización de dispositivos móviles. La muestra no
requiere de sujetos que sean parte del \emph{población objetivo} ya que sólo
está orientada a mejorar aspectos de interfaz de usuario y no el contenido de la
solución, además se considera que la muestra puede brindar una evaluación más
crítica debido a su familiarización con interfaces similares a la de la
solución.

El número de muestras tomadas fue 8, ya que según~\cite{nielsen2000} son
necesarios al menos $5$ participantes para poder obtener resultados
significativos en una prueba de usabilidad. Además,~\cite{ritch2009} asegura que
la teoría de~\cite{nielsen2000} es verdadera especialmente para pruebas simples. 

Se fundamenta el número de participantes, y que es una prueba sencilla, ya que:

\begin{itemize}

\item La prueba no debería tomar más de $10$ minutos en ser realizada.

\item Se busca solamente obtener información acerca de la interfaz, y no el
    funcionamiento en sí de la simulación, pues los usuarios no son expertos en
    el área y no tienen conocimiento acerca las tareas.

\item No se busca evaluar el aspecto pedagógico de la solución sino sólo su interfaz gráfica.
%\item No se busca medir el aprendizaje del usuario en temas no relacionados a la
%    interfaz, es decir, no se mide el aprendizaje del usuario en el tema
%    simulado\revisar{No se entiende, no se mide el aspecto pedagógico solo la
%    interfaz gráfica de la simulación}.

\item El procedimiento de enfermería a realizarse con la solución está bien definido 
y los pasos necesarios están a disposición del usuario en todo momento.

\end{itemize}

\subsection{Variables}
\label{sec:evaluacion_interfaz_variables}

Antes de definir las variables, se deben primero definir los conceptos 
relacionados a los tipos de acciones que pueden realizarse sobre el paciente 
virtual en la solución, los mismos son:

%\observacion{Cual se encarga del diseño de la simulación?}
\begin{itemize}
\item \textbf{Acción por menú contextual:} se refiere a las acciones que el usuario 
    puede realizar utilizando el menú contextual que aparece sobre cada uno de los elementos 
    disponibles en la solución.
\item \textbf{Acción por menú de la \Gls{gui}:} se refiere a las 
    acciones que el usuario puede realizar seleccionando una opción en los menús 
    principales que presenta la interfaz de la solución.
\item \textbf{Acción con elemento:} se refiere a las actividades que el usuario 
    puede realizar cuando tiene seleccionado un elemento y que no involucre el 
    uso del menú contextual.
\end{itemize}


Las variables medidas durante la realización de la tarea con la solución son las
siguientes:

%\observacion{No repetir tanto la descripción en el título}

\begin{itemize}

\item \textbf{Tiempo de realización de la primera acción por tipo:} cuanto tiempo 
	le toma al usuario realizar la primera vez una acción agrupado por tipo (por menú 
	contextual, por menú de la \Gls{gui}, con elementos).

%\item \textbf{Tiempo de realización de la primera acción por menú contextual:} 
%    cuanto tiempo le toma al usuario realizar una acción por menú contextual la 
%    primera vez.
%
%\item \textbf{Tiempo de realización de la primera acción por \Gls{gui}:} cuanto 
%    tiempo le toma al usuario realizar una acción por menú de 
%    interfaz gráfica de usuario la primera vez.
%    
%\item \textbf{Tiempo de realización de la primera acción por herramienta:} cuanto 
%    tiempo le toma al usuario realizar una acción por herramienta la primera vez.

\item \textbf{Tiempo de realización de las siguientes acciones por tipo:} cuanto tiempo 
	le toma al usuario realizar las siguientes veces una acción agrupado por 
	tipo (por menú contextual, por menú de la \Gls{gui}, con elementos).
 
%\item \textbf{Tiempo de realización de las siguientes acciones por menú contextual:} 
%    cuanto tiempo le toma al usuario realizar una acción por menú 
%    contextual las siguientes veces.
%
%\item \textbf{Tiempo de realización de las siguientes acciones por \Gls{gui}:} 
%    cuanto tiempo le toma al usuario realizar una acción 
%    por interfaz gráfica de usuario las siguientes veces.
%
%\item \textbf{Tiempo de realización de las siguientes acciones por herramienta:} 
%    cuanto tiempo le toma al usuario realizar una acción por herramienta 
%    las siguientes veces.

\item \textbf{Tiempo total:} se refiere al tiempo empleado por el usuario para 
    completar la tarea asignada.

\item \textbf{Número de pasos realizados:} cantidad de pasos requeridos en la tarea 
    que son realizados por el usuario en la simulación. 

\item \textbf{Cantidad de movimientos espaciales por tipo:} número de veces en que se 
    modifica el estado de la cámara para realizar las acciones deseadas agrupados por 
    tipo (desplazamiento, acercamiento/alejamiento).

%    \observacion{Esto donde entra?}

\end{itemize}

En cuanto a la encuesta, las siguientes son las variables que fueron consideradas 
y medidas:

\begin{itemize}

\item \textbf{Calidad gráfica:} realismo y calidad de los modelos utilizados.

\item \textbf{Interacción:} desenvolvimiento en el entorno y utilización del 
    hardware.

\item \textbf{Interacción con objetos:} utilización errónea de objetos.

\item \textbf{Características del entorno:} realismo del escenario y de los 
    objetos utilizados.

\item \textbf{Usabilidad de la interfaz:} facilidad de uso de las opciones 
    proveídas por la interfaz.

\item \textbf{Integración con el hardware:} facilidad de uso de la solución con 
    un dispositivo móvil. 

\end{itemize}

\subsection{Métricas}

Para la medición de las variables relacionadas a la encuesta,  se utiliza la escala
de Likert con la \emph{Doble estandarización} explicada en la
sección~\ref{sec:likert}. 

En cambio, para la medición de las variables relacionadas a la interacción del usuario con 
la solución se utilizan las grabaciones registradas durante las pruebas y las
las siguientes métricas:

%Para el análisis de la encuesta realizada a los usuarios, se utiliza la escala
%de Likert con la \emph{Doble estandarización} explicada en la
%sección~\ref{sec:likert}, y en el análisis de la interacción del usuario con la
%solución se utilizan las grabaciones registradas durante la prueba.
%
%Haciendo uso de las variables descriptas anteriormente, las métricas
%utilizadas son las siguientes:

\begin{itemize}
    
\item \textbf{Tiempo promedio de realización de las siguientes acciones por menú contextual:} 
    se obtiene dividiendo la cantidad total de tiempo empleado en realizar acciones por menú 
    contextual por el número de veces que se realizaron esas acciones, sin considerar la primera 
    vez. 
    
\item \textbf{Tiempo promedio de realización de las siguientes acciones por \Gls{gui}:} 
    se obtiene dividiendo la cantidad total de tiempo empleado en realizar acciones por \Gls{gui} 
    por el número de veces que se realizaron esas acciones, sin considerar la primera 
    vez. 
    
\item \textbf{Tiempo promedio de realización de las siguientes acciones por herramienta:} 
    se obtiene dividiendo la cantidad total de tiempo empleado en realizar acciones por 
    herramienta por el número de veces que se realizaron esas acciones, sin considerar la 
    primera vez. 
    
\item \textbf{Promedio de pasos correctos:} se obtiene dividiendo la cantidad de 
    pasos requeridos realizados por los usuarios sobre la cantidad de pasos requeridos. 
    
\item \textbf{Promedio de movimientos por tipo:} se obtiene dividiendo el número de 
    movimientos que fueron realizados agrupados por tipo (desplazamiento, acercamiento/
    desplazamiento) por la cantidad de usuarios.
    
\item \textbf{Promedio del tiempo total:} se obtiene dividiendo el tiempo total empleado 
    por los usuarios para completar la tarea asignada por el número de usuarios.

\end{itemize}

\subsection{Resultados obtenidos}
\label{sec:res_interfaz}

A continuación de muestran y analizan los resultados obtenidos en la prueba. Los resultados 
se dividen en \emph{simulación} y \emph{encuesta} para una mejor comprensión.

\subsubsection{Simulación}

Las grabaciones realizadas a las sesiones de los usuarios se utilizan para medir
el grado de facilidad de aprendizaje de la interfaz de usuario.

Dados los tres tipos de acciones descritos en~\ref{sec:evaluacion_interfaz_variables}, la
tabla~\ref{tab:interfaz_tiempo_acciones} muestra el tiempo, en segundos,
que le tomo a cada usuario realizar una acción la primera vez y 
el tiempo que les tomo en promedio las demás veces, para cada uno de los tipos 
de acciones.

%\observacion{Hacer énfasis en la comparación entre el primer y los siguientes}

\begin{table}[!hbt]
\centering
\begin{tabular}{|c|c|c|c|c|c|c|}
\hline
& \multicolumn{2}{c|}{\textbf{Menú Contextual}} &
\multicolumn{2}{c|}{\textbf{Menú de la Interfaz}} & \multicolumn{2}{c|}{\textbf{Herramienta}}\\
\hline
\textbf{Usuario}  & \textbf{Primera} & \textbf{Siguientes} & \textbf{Primera} & \textbf{Siguientes} & \textbf{Primera} & \textbf{Siguientes} \\
\hline 1          & 8                & 2.25                & 3                & 9.14                & 11               & 3.0 \\
\hline 2          & 30               & 7.00                & 4                & 3.57                & 7                & 4.5 \\
\hline 3          & 5                & 2.25                & 5                & 1.86                & 1                & 1.0 \\
\hline 4          & 2                & 13.00               & 4                & 2.00                & 1                & 0.5 \\
\hline 5          & 18               & 2.75                & 6                & 4.43                & 6                & 3.0 \\
\hline 6          & 4                & 14.25               & 11               & 7.86                & 13               & 4.0 \\
\hline 7          & 5                & 8.00                & 4                & 4.71                & 20               & 2.5 \\
\hline 8          & 3                & 2.33                & 10               & 3.57                & 3                & 6.5 \\
\hline
\textbf{Promedio} & \textbf{9.38}    & \textbf{6.37}       & \textbf{5.88}    & \textbf{4.64}       & \textbf{7.75}    & \textbf{3.125} \\
\hline
\end{tabular}
\caption{Tiempo por acciones la primera vez y las siguientes veces que se realizo}
\label{tab:interfaz_tiempo_acciones}
\end{table}

En la tabla~\ref{tab:interfaz_tiempo_acciones} se observa consistentemente una 
mejora en el tiempo de realización de un tipo de acción con respecto a la primera vez 
que es realizada. 

\begin{filecontents}{interfazuso.dat}
n   p       s
1	9.38	6.48
2   5.88	4.64
3   7.75	3.13
\end{filecontents}
\pgfplotstableread{interfazuso.dat}{\InterfazUso}

\begin{figure}[H]
    
        \centering
        \begin{tikzpicture}[scale=.8]
           \begin{axis}[ybar,%
              legend pos=outer north east,
              xmin=1,
              xmax=3,
              x=2.5cm,
              enlarge x limits={abs=1cm},
              xtick=data,
              symbolic x coords={0,1,2,3,4},
              ymin=0,ymax=10,
              %ytick={0,2,4,6,8,10},
              xticklabels={Contextual,Interfaz,Herramienta},
              ylabel= Tiempo (s),
              xlabel= Tipo de acción,
              bar width=10pt,
              %enlarge x limits={abs=2},
                ]   
        \addplot[color=blue!90,ybar,fill=blue!55,area legend] table [x = {n}, y = {p}] {\InterfazUso};
        \addlegendentry{Primer}
        \addplot[color=red!90,ybar,fill=red!55,area legend] table [x = {n}, y = {s}]
        {\InterfazUso};
        \addlegendentry[align=left]{Promedio \\ siguientes}
        \end{axis}
        \end{tikzpicture}
        \caption{Tiempo por tipo de acción}
        \label{fig:interfaz_tiempo_acciones}
\end{figure}

En la figura~\ref{fig:interfaz_tiempo_acciones} se observa como en promedio el
usuario aprende, y en las siguientes acciones similares demora menos tiempo,
este es un factor importante y es el objetivo de esta prueba pues muestra que la
interfaz es fácil de usar, y con tres tipos de acciones, el usuario puede
utilizarla sin mayores inconvenientes. Se observa una mejoría del $30\%$ en las
\emph{Acciones por menú contextual}, $21\%$ en las \emph{Acciones por menú de la
    \Gls{gui}} y finalmente, una mejoría del $60\%$ en las \emph{Acciones con elementos}.


\begin{table}[hbt]
\centering
\small
\begin{tabular}{lrrr}
\toprule
\textbf{Jugador}  & \textbf{Desplazamiento} & \textbf{Acercamiento/alejamiento} & \textbf{Total} \\
\midrule
1        & 18         & 2    & 20 \\
2        & 7          & 8    & 15 \\
3        & 14         & 12   & 26 \\
4        & 9          & 14   & 23 \\
5        & 5          & 8    & 13 \\
6        & 14         & 4    & 18 \\
7        & 16         & 3    & 19 \\
8        & 4          & 3    &  7 \\
\midrule
\textbf{Promedio} & \textbf{10,88}      & \textbf{6,75} & \textbf{17,63} \\
\bottomrule
\end{tabular}
\caption{Cantidad de movimientos espaciales}
\label{tab:interfaz_cantidad_espaciales}
\end{table}

En la tabla~\ref{tab:interfaz_cantidad_espaciales} se observa la cantidad de
movimientos espaciales realizados por los usuarios, se observa que en promedio
se desplazaron $10,88$ veces por el escenario, y $6,75$ veces acercaron o
alejaron la cámara del paciente.

No existe una cantidad mínima o máxima de movimientos que el usuario debe realizar para acercar, 
alejar o desplazar la cámara. Los datos mostrados en la tabla~\ref{tab:interfaz_cantidad_espaciales} 
muestran que no son necesarias demasiados movimientos. Teniendo en cuenta esta información y la 
proveída en la tabla~\ref{tab:interfaz_tiempo_total}, se concluye 
que en promedio los usuarios realizan $1,7$ movimientos por minuto.

\begin{table}[!hbt]
\centering
\small
\begin{tabular}{lrrr}
\toprule
\textbf{Alumno} & \textbf{Tiempo (min)} \\
\midrule
1        & 8:32 \\
2        & 6:03 \\
3        & 8:33 \\
4        & 5:17 \\
5        & 6:55 \\
6        & 8:40 \\
7        & 7:03 \\
8        & 10:27 \\
\midrule
\textbf{Promedio} & \textbf{7:41} \\
\bottomrule
\end{tabular}
\caption{Tiempo de prueba por usuario}
\label{tab:interfaz_tiempo_total}
\end{table}

El tiempo total que se observa en la tabla~\ref{tab:interfaz_tiempo_total},
muestra que en promedio a cada alumno le tomo $7:41$ minutos realizar todos los
pasos especificados, es importante notar que este tiempo incluye el tiempo de
adaptación. 

La tabla~\ref{tab:interfaz_acciones} nos muestra la cantidad de pasos
realizados por los alumnos de un total de 19. Se observa que en promedio 
realizaron $16.75$ pasos.

\begin{table}[htb]
\centering
\small
\begin{tabular}{lrrr}
\toprule
\textbf{Alumno} & \textbf{Pasos realizados (19)} \\
\midrule
1 & 19 \\
2 & 15 \\
3 & 18 \\
4 & 15 \\
5 & 18 \\
6 & 16 \\
7 & 19 \\
8 & 14 \\
\midrule
\textbf{Promedio} & \textbf{16,75} \\
\bottomrule
\end{tabular}
\caption{Pasos realizados por alumno}
\label{tab:interfaz_acciones}
\end{table}



\subsubsection{Encuesta}


La encuesta es utilizada para obtener el grado de disconformidad de los usuarios
con respecto a la solución. Se utiliza la disconformidad para resaltar los
puntos débiles, así, aquellas variables que tengan el mayor porcentaje serán las
que deban ser mejoradas.


En la tabla~\ref{tab:interfaz_disconformidad_metrica} se observan que las
mayores disconformidades son la usabilidad de la interfaz de usuario que llega
al $51\%$, la interacción de los usuarios con el entorno que llega al $50\%$ y
la interacción con los objetos que llega al $49\%$. Otras disconformidades con
menor porcentaje son las características del entorno con un  $33\%$, la
integración con el hardware con un $27\%$ y por último la calidad gráfica con un
$17\%$.


\begin{table}[htb]
\centering
\begin{tabular}{lr}
\toprule
\textbf{Variable} & \textbf{Disconformidad (0-1)}\\
\midrule
Calidad Gráfica         & 0.17 \\
Interacción Entorno     & 0.50\\
Interacción Objetos     & 0.49\\
Características Entorno & 0.33\\
Usabililidad Interfaz   & 0.51\\
Integración Hardware    & 0.27\\
\bottomrule
\end{tabular}
\caption{Disconformidad por variable}
\label{tab:interfaz_disconformidad_metrica}
\end{table}

%La conclusión de esta prueba de interfaz, es que si bien, pudo ser utilizada sin
%mayores inconvenientes, existe un alto grado de disconformidad con la interfaz,
%además cabe resaltar, los sujetos de prueba son personas acostumbradas al uso de
%tecnologías similares. Otros puntos débiles encontrados en esta prueba son la
%interacción con el entorno y  con los objetos.

Como consecuencia de los resultados obtenidos, la usabilidad de interfaz y la interacción con objetos y 
con el entorno son mejoradas para obtener la versión final de la solución que es utilizada por 
los estudiantes de enfermería. Las demás pruebas mencionadas en este capítulo son realizadas con 
la versión final de la solución.
%elementos sufren modificaciones a fin de su utilización con usuarios no
%técnicos.

%Las demás pruebas mencionadas en este capítulo son realizadas con la versión
%final de la solución, la cual es obtenida luego de las mejoras realizadas a los
%puntos débiles detectados por esta prueba.

%! TEX root = ../main.tex

\section{Encuesta de ubicación}
\label{sec:ubicacion}

Para recabar información acerca del nivel de acceso  de los alumnos a la
tecnología, se realiza una encuesta que cuenta con diez preguntas, las cuales
buscan conocer sobre el modelo de dispositivo móvil, el acceso a
Internet, y la predisposición de cada alumno a ayudar en la prueba.

Con los resultados de la encuesta de ubicación tecnológica, se seleccionan
aquellos alumnos que posean dispositivos móviles que superan o igualan las
especificaciones descriptas más adelante. De esta encuesta se obtendrán los 
usuarios que formarán parte de la población que evaluará la versión final de 
la solución.

\subsection{Muestra}

En el año $2014$, el \Gls{iab} cuenta con $124$ alumnos en el cuarto año distribuidos en
tres secciones, el cual es considerado el \emph{Universo}. De los 124, 93 de
ellos estuvieron interesados en participar de la prueba y completaron la encuesta.

\subsection{Variables}

Se definen $2$ factores necesarios para que un alumno pueda ser considerado como
sujeto de prueba, el primero es la predisposición del mismo a participar de la
prueba y el segundo es que posea un dispositivo móvil que supere los requisitos
mínimos. Además de estos dos factores, la conexión a internet es requerida, pues 
los registros de actividad de cada dispositivo deben ser enviados y almacenados 
para su posterior interpretación y análisis. A continuación se describen las variables 
consideradas.


\begin{itemize}

\item \textbf{Requisitos mínimos:} son aquellos requerimientos técnicos con los que 
    debe cumplir completamente el dispositivo móvil del usuario para que la 
    solución tenga un desempeño que garantice una experiencia fluida a la hora de 
    utilizarla. Estos requisitos son:
    \begin{itemize}
        %%\item Sistema Operativo Android $4.0$ o superior
        \item Memoria ram de $512$MB o superior.
        \item Velocidad de procesador de $800$ GHz o superior.
        \item \Gls{gpu} Mali 400 o superior.
        %\item Conexión frecuente a internet.
    \end{itemize}
    Los requisitos de \textit{hardware} mencionados, son requeridos por las
    características de la simulación, una \Gls{gpu} es requerida por los gráficos en 
    tres dimensiones.

\item \textbf{Tipo de acceso a internet:} el tipo de acceso a internet que posee el 
    usuario en su dispositivo móvil. Puede ser una de las siguientes opciones:
    plan post-pago, paquetes pre-pago, acceso ocasional y sin acceso.
    
\item \textbf{Sistema Operativo:} se refiere al tipo de sistema operativo que posee 
    el dispositivo móvil del usuario.

    \observacion{Donde se menciona?}
    
\end{itemize}

\subsection{Métricas}

Las métricas utilizadas para estudiar los datos recogidos son sencillas ya que
sólo buscan determinar la población que evaluará la solución, estas métricas son
las siguientes:

\begin{itemize}
\item Porcentaje de encuestados con dispositivos móviles que cumplen y que no cumplen con 
los requisitos mínimos.
\item Porcentaje del tipo de acceso a internet de los usuarios.
\item Porcentaje del tipo de sistema operativo que poseen los dispositivos móviles de los 
encuestados.
\end{itemize}



%! TEX root = ../main.tex

\section{Registro de actividades}
\label{sec:registro}

Las metodologías anteriormente descritas incluyen encuestas que miden
el conocimiento del alumno y su opinión con respecto a la solución propuesta,
para poder formar una opinión válida primero deben experimentar con la misma, para
ello se instala la solución en los dispositivos móviles de los alumnos que forman 
parte de la población objetivo.

La instalación de la solución se lleva a cabo en el \Gls{iab}, y se procede 
a mostrar un vídeo de la simulación, explicar la interfaz y realizar una muestra 
de como desenvolverse en el entorno.

El período de prueba se extiende por 20 días, el mismo no es
\fixme{controlado}{Asistido}, es decir que existen factores que no pueden ser
controlados, como:

\begin{itemize}
    \item Tiempo dedicado a la simulación.
    \item Que todas las acciones provengan del alumno.
    \item Solamente el conocimiento del alumno es puesto a prueba, es decir, no
        se puede controlar que no reciba ayuda externa.
\end{itemize}

Por estos motivos, el uso de la solución propuesta no puede ser considerado
el único factor relacionado con los resultados de la encuesta objetiva
descrita en~\ref{sec:objetiva}.

La solución propuesta almacena información relacionada a la actividad del
usuario, incluyendo cuando y como utiliza las acciones, los pasos que realiza,
el orden y las condiciones de la escena cuando realiza cada acción.

El registro como un todo es enviado cada vez que el usuario desee, este envío
requiere una conexión a internet por ello no es automático. Adicionalmente el
último día de la prueba, todos los registros fueron enviados para que sean
analizados.

El registro de actividades ayuda a identificar las  fortalezas y debilidades 
de la solución en cuanto al diseño y utilidad. Sobre todo, ayuda a medir 
el impacto pedagógico al permitir contrarrestar el uso y desempeño del usuario 
con el puntaje obtenido por el mismo en la \emph{Encuesta Objetiva}.

\subsection{Muestra}

La muestra esta conformada por los $11$ alumnos que aceptaron formar parte de 
la prueba y poseían dispositivos móviles que cumplen con los requisitos
mínimos.

\subsection{Variables}

La utilización de la simulación, y el registro de las actividades genera una
gran cantidad de información, los factores que se desean medir están
relacionados a aquellos que pueden ser contrastados con los resultados de la
encuesta objetiva.

La utilización de la simulación nos permite obtener información relevante acerca
de como se utilizo la misma, se definen los criterios a medir:

\begin{description}

\item[Cantidad de partidas:] se define como la cantidad de veces que un alumno
    inicia una escena. 

\item[Tiempo total:] es la suma del tiempo empleado en todas las partidas.

\item[Tiempo total de partidas por usuario y por tipo:] es el tiempo total 
    empleado para jugar las partidas discriminadas por tipo y por usuario.

\item[Cantidad de acciones:] es la cantidad total de acciones realizadas por 
    los usuarios.
 
\item[Cantidad de partidas realizadas por usuario y por tipo:] es el número de 
    partidas jugadas por usuario discriminado por el procedimiento al que 
    corresponde.

\item[Cantidad de usuarios:] es el número de usuarios que utilizaron la solución.
    
\item[Puntuación de las partidas:] Dado el registro de reglas cumplidas en una partida 
    del procedimiento de extracción de sangre o el diagnóstico dado por el usuario 
    en una partida del procedimiento de valoración de la escala de Glasgow, se 
    puede obtener el desempeño del usuario en las partida. Esto puede ser contrastado 
    con la puntuación obtenida por el usuario en la \emph{Encuesta Objetiva}.

%\item[Puntuación por regla cumplida] Las variables definidas
%    en~\ref{sec:objetiva}, pueden ser contrastadas con la puntuación obtenida
%    por los alumnos en la simulación.

\end{description}

\subsection{Métricas}

\begin{description}
\item[Promedio de tiempo por partida:] se obtiene dividiendo el tiempo total empleado 
    en las partidas por el número de partidas.
\item[Promedio de acciones por partida:] se obtiene dividiendo el cantidad total de 
    acciones realizas por el usuario por el número de partidas.
\item[Promedio de partidas por usuario:] se obtiene dividiendo el número total de partidas
    por el número de usuarios que utilizaron la solución.
\item[Total de sesiones jugadas por tipo:] es la suma de la cantidad de partidas jugadas por 
    usuario y tipo.
\item[Total de tiempo jugado por tipo:] es la suma de la cantidad de tiempo empleado en una 
    partida por usuario y tipo.
\item[Promedio de siguientes puntajes por tipo y por usuario:] se obtiene dividiendo la suma 
    de los puntajes obtenidos en cada tipo de escenario por la cantidad de veces que jugó el 
    usuario, a excepción de la primera vez.
\end{description}

Además de las métricas descriptas también se utiliza la correlación de Pearson como se 
explica en \ref{sec:correlacion} para identificar las relaciones entre los datos obtenidos en 
la \emph{Encuesta Objetiva} y los obtenidos en el \emph{Registro de actividades}.

\subsection{Resultados}

Las actividad de los usuarios es registrada y almacenada para su análisis, a
continuación se presentan los resultados de ese análisis, el mismo fue descrito
en~\ref{sec:registro}, en la tabla~\ref{tab:log_total} se observa un resumen del
experimento, en cuanto a tiempo, partidas y acciones.


\begin{table}[H]
\centering
\begin{tabular}{lrrrrrrrr}
\toprule
\textbf{Variable}                         & \textbf{Valor} \\
\midrule
Tiempo total                     & 11134\tabletodo{Seguro?} \\
Partidas                         & 99 \\
Acciones                         & 2944 \\
Promedio de tiempo por partida   & 112 \\
Promedio de acciones por partida & 30 \\
Promedio de partidas por usuario & 12,37 \\
Usuarios                         & 8 \\
\bottomrule
\end{tabular}
\caption{Resumen de la información extraída del registro de actividades.}
\label{tab:log_total}
\end{table}

\observacion{Falta unidad de tiempo}
\begin{table}[H]
\centering
\begin{tabular}{lrrrrrrrr}
\toprule
& \multicolumn{2}{c}{Extracción de sangre} \\
\cmidrule(lr){2-3} 
Alumno   & Sesiones jugadas & Tiempo jugado \\
\midrule
 1       & 5                & 1202 \\
 2       & 19               & 2507 \\
 4       & 5                & 398  \\
 5       & 6                & 768  \\
 6       & 17               & 2371 \\
 7       & 7                & 707  \\
 9       & 1                & 126  \\
10       & 8                & 960  \\
\midrule
Total   & 68               & 9039 \\
\bottomrule
\end{tabular}
\caption{Número de partidas y tiempo total por alumno en segundos, en la escena
    de extracción de sangre.}
\label{tab:log_hemocultivo_partida}
\end{table}

La cantidad de partidas jugadas por usuario, se ven en la
tabla~\ref{tab:log_hemocultivo_partida}, se observa que existen $3$ alumnos que no
participaron de la prueba o no se registro su actividad.

Los registros pueden no ser registrados sí
\begin{enumerate*}[label=\itshape\alph*\upshape)]
    \item el usuario utilizo la solución, no envió los datos y, luego
        desinstalo la solución o borro los datos de la misma, o,
    \item el usuario no utilizo la solución.
\end{enumerate*}

En la tabla~\ref{tab:log_glasgow_random_partida}, se observa la cantidad de
sesiones y tiempo total por alumno, en la escena de \textit{Glasgow}, en modo de
evaluación. Se observa que $5$ alumnos participaron en $22$ sesiones, en total
jugaron $1768$ segundos.

\begin{table}[H]
\centering
\begin{tabular}{lrrrrrrrr}
\toprule
& \multicolumn{2}{c}{Glasgow (Evaluación)} \\
                   \cmidrule(lr){2-3} 
Número de alumno   & Sesiones jugadas                            & Tiempo jugado \\
\midrule
1     & 4  & 211 \\
2     & 8  & 738 \\
4     & 3  & 132 \\
6     & 1  & 97  \\
7     & 6  & 590 \\
\midrule
Total & 22 & 1768 \\
\bottomrule
\end{tabular}
\caption{Número de partidas y tiempo total por alumno en segundos, en la escena
    \textit{Glasgow}, en modo evaluación}
\label{tab:log_glasgow_random_partida}
\end{table}


\begin{table}[H]
\centering
\begin{tabular}{lrrrrrrrr}
\toprule
& \multicolumn{2}{c}{Glasgow (Exploración)} \\
                   \cmidrule(lr){2-3} 
Número de alumno   & Sesiones jugadas                            & Tiempo jugado \\
\midrule
1        & 2 & 79 \\
2        & 3 & 80 \\
4        & 3 & 89 \\
6        & 1 & 79 \\
\midrule
Total   & 9 & 327 \\
\bottomrule
\end{tabular}
\caption{Número de partidas y tiempo total por alumno en segundos, en la escena
    \textit{Glasgow}, en modo exploración}
\label{tab:log_glasgow_custom_partida}
\end{table}


En las tablas~\ref{tab:log_hemocultivo_puntaje}
y~\ref{tab:log_glasgow_random_puntaje} se muestran los primeros puntajes y un
promedio de los puntajes siguientes obtenidos por cada usuario en los
procedimientos de extracción de sangre y de la evaluación de la escala de
Glasgow. Se debe tener en cuenta el tiempo y las cantidades de veces que cada
alumno jugó cada uno de los procedimientos para valorar los resultados
mostrados. 

\begin{table}[H]
\centering
\begin{tabular}{lrrrrrrrr}
\toprule
& \multicolumn{2}{c}{Extracción de sangre} \\
\cmidrule(lr){2-3} 
Número de alumno  & Primer Puntaje & Siguientes Puntajes \\
\midrule
 1                & 11             & 14.3 \\
 2                & 9              & 10.6 \\
 4                & 3              & 3.3  \\
 5                & 3              & 6.8  \\
 6                & 3              & 5.8  \\
 7                & 4              & 4    \\
 9                & 16             & \\
10                & 3              & 7.2  \\
\midrule
\textbf{Promedio} & 6.5            & 7.42 \\
\bottomrule
\end{tabular}
\caption{Puntaje obtenido la primera vez y el promedio de las siguientes veces
    por alumno, en la escena de extracción de sangre.}
\label{tab:log_hemocultivo_puntaje}
\end{table}


\begin{table}[H]
\centering
\begin{tabular}{lrrrrrrrr}
\toprule
& \multicolumn{2}{c}{Glasgow (Evaluación)} \\
                   \cmidrule(lr){2-3} 
Número de alumno   & Primer Puntaje & Siguientes Puntajes \\
\midrule
1     & 1 & 1.5 \\
2     & 2 & 2.3 \\
4     & 1 & 1.5 \\
6     & 2 & 2 \\
7     & 0 & 1 \\
\midrule
\textbf{Promedio} & 1.2 & 1.66 \\
\bottomrule
\end{tabular}
\caption{Puntaje obtenido la primera vez y el promedio de las siguientes veces
    por alumno, en la escena \textit{Glasgow}, en modo evaluación}
\label{tab:log_glasgow_random_puntaje}
\end{table}

Se observa en las tablas~\ref{tab:log_hemocultivo_puntaje}
y~\ref{tab:log_glasgow_random_puntaje} los alumnos que participaron de la prueba
mejoran su desempeño a medida que aumenta el número de partidas. 

Es importante notar que la cantidad de partidas no es uniforme entre los
alumnos, es decir hay alumnos con más de $10$ partidas y usuarios con menos de
$5$, por ello, no es posible demostrar que existe un progreso a medida que
aumenta el número de partidas.


%! TEX root = ../main.tex

\section{Encuesta para evaluar la solución}
\label{sec:subjetiva}

Al final del período de prueba de la solución cada alumno que forma parte de la muestra
completa una encuesta con $31$ preguntas que se utilizan para validar las
consideraciones de diseño, las cuales fueron explicadas en el capítulo~\ref{chap:requerimientos}
y para medir sus apreciaciones sobre otros aspectos de la solución que serán
detallados más adelante en esta sección. 

Las preguntas están agrupadas en dos, el primer grupo cuenta con $27$ preguntas
cerradas, es decir de una sola respuesta en una lista de opciones, el segundo
grupo cuenta con $4$ preguntas abiertas, es decir los encuestados pueden dar
respuestas libres a las preguntas. 

De esta forma, se busca identificar las fortalezas y debilidades de la solución,
además de evaluar la solución en cuanto a factores de exploración,
representación, motivación, inmersión, retroalimentación y pedagogía, de acuerdo
a las apreciaciones de los miembros de la muestra.

\subsection{Muestra}

La encuesta es entregada a los $11$ alumnos de la población objetivo que acordaron
participar en la prueba y que fueron seleccionados como resultado de la 
\emph{Encuesta para determinar la muestra}, mientras completan la encuesta, un guía está presente
para responder cualquier duda.


\subsection{Variables}
\label{sec:variables}
%\observacion{Parece haber tanta referencia como para separar}

%A continuación se describen las variables que tienen por objetivo demostrar la
%validez de las consideraciones de diseño planteadas en este trabajo descritas en el
%capítulo~\ref{chap:requerimientos} y la medición de otros aspectos de la
%solución relacionados con los objetivos de este trabajo descritos en la
%sección~\ref{sec:objetivos_generales}. 

Las variables a medir son agrupadas en
factores, los cuales representan aquellos aspectos de la solución propuesta que
buscan ser evaluados.

Cabe resaltar que la medición de estas variables se realizan
exclusivamente de acuerdo a las valoraciones dadas por la muestra en cada una de
las preguntas que forman parte de la encuesta.

\subsubsection{Exploración}
\label{sec:sub_exploracion}
%\observacion{Creo que el nombre es confuso por que parece referirse al tema de
%interacción con el entorno}
%
%\observacion{Se podría simplificar el siguiente párrafo} 

Este factor se refiere a los aspectos de la solución que permiten al usuario 
explorar el entorno durante la partida. Para facilitar esta exploración se 
busca proveer  facilidad de uso, intuitividad y realismo en cuanto a las acciones y
situaciones que se presentan en la solución para que de esta manera, los
elementos que la componen no representen para el jugador un obstáculo que impida
su uso.

%Este factor esta \fixme{relacionado}{} con \fixme{la característica}{} que posee
%la solución en cuanto a la \fixme{oportunidad}{} que \fixme{brinda}{} al usuario
%para \fixme{explorar}{} cada uno de los elementos del entorno simulado
%(paciente, herramientas propias del procedimiento). En este sentido, se busca
%proveer facilidad de uso, intuitividad y realismo en cuanto a las acciones y
%situaciones que se presentan en la solución para que de esta manera, los
%elementos que la componen no representen para el jugador un obstáculo que impida
%su uso.
%\observacion{El factor que mide la eficiencia es acaso a explorar el entorno?}

Las variables que miden este aspecto son las siguientes:

\begin{description}



\item[Funciones realizadas por los elementos del juego:] se refiere a si las 
	simplificaciones realizadas en la solución en cuanto a las funciones de cada uno 
	de los elementos del juego facilitan su uso.

%se refiere a la
%    correctitud con la que una herramienta o elemento del juego representa las
%    funciones que el mismo puede realizar en la vida real, en este sentido, se
%    evalúa el realismo con el que es representado tal elemento.

\item[Aleatoriedad para afianzar conocimientos:] se refiere al beneficio que
    puede traer el hecho de que el estado del paciente en el juego sea aleatorio
    en cuanto a la posibilidad que esto brinda al jugador para poner a prueba
    sus conocimientos teóricos.

\item[Aleatoriedad para representar realismo:] se refiere al uso de estados
    aleatorios en el paciente para que de esta forma el procedimiento se asemeje
    más a una situación real e invite al usuario a explorar el entorno.

\item[Intuitividad:] se refiere a lo intuitivo que puede ser la
    utilización de los elementos del juego.

\end{description}

\subsubsection{Representación}
\label{sec:sub_representacion}

Este factor está relacionado con la calidad y suficiencia con la que se
representan los diferentes objetos que son simulados en la solución. La
representación abarca tanto funcionalidad como aspecto del objeto.

De esta manera, se busca permitir al usuario realizar con los objetos las
acciones que requiera para llevar a cabo el procedimiento que se le presente en
la solución, y además, representar estos elementos de la mejor manera posible.

Las variables que miden estos aspectos son las siguientes:

\begin{description}

\item [Respuestas del paciente:] se refiere a la suficiencia de las respuestas 
    motrices, oculares y verbales que realiza el paciente en la escena 
    correspondiente a la valoración de la escala de Glasgow.

\item[Distinción entre los estados del paciente:] se refiere a si los diferentes
    estados del paciente son distinguidos correctamente en el procedimiento de
    valoración de la escala de Glasgow ya que esto es importante para que el
    usuario pueda diagnosticar correctamente al paciente.

\item[Acciones con los elementos:] se refiere a si las diferentes acciones que
    pueden realizarse con los elementos o herramientas del juego en un
    determinado procedimiento de enfermería son suficientes para ese
    procedimiento, ya que, debido a las limitaciones de la tecnología estas
    acciones son limitadas.

\end{description}

\subsubsection{Motivación}
\label{sec:sub_motivacion}

Este factor está relacionado con la importancia de incluir en la solución
aquellas características lúdicas que son propias de un videojuego convencional. Se
busca conocer el valor de estas características en cuanto a la motivación que
puedan producir en los usuarios tanto para volver a utilizar la solución como
para superarse en cada partida.

Las variables que miden estos aspectos son las siguientes:

\begin{description}

\item[Motivación del puntaje:] se refiere a que tanto motiva al jugador que la
    solución le proporcione un puntaje total al final de cada partida para poder
    mejorar constantemente siendo este puntaje como una evaluación final de todo
    lo que realizó durante la partida.

\item[Importancia del puntaje:] se refiere a que tan importante es para un
    jugador que se le proporcione un puntaje total al final de cada partida para
    poder visualizar su rendimiento.

\item[Socialización de los puntajes:] se refiere a si el hecho de que las
    personas del mismo entorno compartan sus puntajes, experiencias y logros en
    las partidas a través de redes sociales pueda ser motivador.

\item[Medición del tiempo:] se refiere a que tanto motiva al
    jugador que la solución le proporcione el tiempo que duro su partida
    sirviendo este tiempo como una evaluación de su precisión a la hora de
    realizar el procedimiento que se le presente.

\end{description} 


\subsubsection{Inmersión}
\label{sec:sub_inmersion}

Este factor está relacionado con la percepción de formar
parte de la escena. Es decir, se trata de evaluar que tanto el usuario puede
sentir que realmente se encuentra dentro del juego para que de este modo el
pueda entrar en ambiente para realizar los procedimientos que se le presenten en
sus partidas de juego.

Las variables que miden este aspecto son las siguientes:

\begin{description}

\item[Escenografía para entrar en ambiente:] se refiere a la importancia de la
    escenografía de la partida para que el jugador entre en ambiente para
    realizar el procedimiento que se le presente.

\item[Juegos cortos:] se refiere a si el hecho de
    que los procedimientos presentados en las partidas sean cortos contribuye a
    repetir las partidas varias veces de seguido entrando en un estado de
    inmersión.

\item[Gráficos en tres dimensiones para entender el entorno:] se refiere a la
    importancia que tiene el uso de gráficos en tres dimensiones para que el
    usuario pueda entender mejor el entorno y las posibles acciones que puede
    realizar.

\item[Realismo a través de ordenes verbales:] se refiere a si el hecho de que la
    solución brinde la posibilidad de que aparezca un menú de ordenes verbales
    en el momento en que el jugador habla hace que la acción de dar ordenes
    verbales se asemeje más a la realidad.

\item[Sentido de pertenencia:] se refiere a si la simulación ayuda al
    jugador a sentirse parte del laboratorio, dando cierto realismo a la escena
    que se le presenta.

\end{description}

\subsubsection{Utilidad}
\label{sec:sub_utilidad}

Este factor está relacionado con el potencial de la solución como herramienta 
de apoyo al proceso de aprendizaje de los estudiantes de enfermería.

Las variables que miden este aspecto son las siguientes: 

\begin{description}
\item[Simulación para complementar el estudio en clase y laboratorio:] se
    refiere a que tanto potencial tienen las herramientas alternativas como la
    simulación para complementar a los métodos de aprendizaje tradicionales
    que son el estudio en clase y en el laboratorio.

\item[Simulación como proveedor de facilidades para el estudio:] se refiere a si las
    herramientas alternativas como la solución proveen más facilidades para
    poner en practica los conocimientos con respecto a los demás métodos de
    aprendizaje que son los libros, laboratorios y el campo de prácticas.

\item[Interacción con el paciente:] se refiere a si el hecho de que el jugador
    pueda interactuar con un paciente que responde a las acciones del jugador 
    implica una mejora con respecto a otros materiales utilizados en los 
    laboratorios de práctica.

\end{description}

%\observacion{Algunos parecen estar fuera de lugar (los que tienen ?)}

\subsubsection{Retroalimentación}
\label{sec:sub_retroalimentacion}

Este factor está relacionado con la importancia de ofrecer al jugador
información acerca de sus logros y errores de manera tal que el pueda estar
consciente de sus puntos fuertes y sus puntos débiles en los diversos
procedimientos que realice en la solución.

Las variables que miden este aspecto son las siguientes:

\begin{description}[style=unboxed]

\item[Detalles de los pasos realizados incorrectamente:] se refiere a 
    la importancia que tiene para el usuario que la solución no sólo le 
    diga los pasos que realizó de manera incorrecta sino también el por qué 
    de ello.

\item[Retroalimentación suficiente respecto a los pasos realizados:] se refiere 
    a si son suficientes las justificaciones breves acerca de las causas por las 
    cuales se realizó incorrectamente un paso.

\item[Representación iconográfica de conceptos y acciones en la \Gls{gui}:] 
    se refiere a la suficiencia de mostrar iconos en la interfaz de 
    la solución para representar el estado actual del jugador.

\end{description}

\subsubsection{Pedagogía}
\label{sec:sub_pedagogia}

Este factor está relacionado al beneficio que puede traer la
solución para apoyar el aprendizaje del jugador. De esta manera, se busca
obtener la validez real de este tipo de herramientas como aporte al aprendizaje,
proveyendo mas interacción al jugador.

Las variables que miden este aspecto son las siguientes:

\begin{description}

\item[Potencial para memorizar y comprender el procedimiento:] se refiere a
    que tanto ayuda la solución al usuario para entender los procedimientos que se
    le presentan y para no olvidar los pasos de cada uno de ellos.

\item[Retroalimentación limitada:] se refiere a que tan efectivo
    resulta no dar pistas al jugador en el momento de realizar un procedimiento
    para que pueda plasmar y medir sus conocimientos.

\item[Acciones a través de botones:] se refiere a que tan
    suficiente es representar determinadas acciones  con un botón debido a
    limitaciones en la tecnología.

\end{description}


\subsection{Métricas}

La métrica utilizada en las preguntas cerradas es la escala de Likert haciendo
uso de la \emph{Doble estandarización}, como se describe en la
sección~\ref{sec:likert}. Esto ayuda a determinar los puntos fuertes y débiles
de los aspectos evaluados.

Además se utilizan promedios hallados teniendo en cuenta las respuestas de los
usuarios en cada una de las preguntas cerradas para determinar el nivel de
aceptación promedio en cuanto a los temas abordados en las preguntas.

\subsubsection{Manejo de información faltante}
\label{sec:informacion_faltante}
%\observacion{No repetir tanto existe}

Debido a que hubieron preguntas no respondidas en una de las encuestas, se utilizaron 
métodos para tratar esa información faltante. En este tipo de situaciones existen tres 
posibles formas de categorizar el patrón de ocurrencia de la falta de 
respuestas\cite{leite2010performance, tsikriktsis2005review}:

\begin{description}
    \item[Información faltante completamente aleatoria:] cuando la información
        faltante es independiente de la variable medida y de otras variables.
    \item[Información faltante aleatoria:] cuando la información faltante depende
        de otras variables, pero no de la variable en sí. 
    \item[Información faltante no aleatoria:] cuando hay una relación entre la
        información faltante y el valor de la variable.
\end{description}

En esta caso el patrón corresponde al tipo \emph{Información faltante completamente aleatoria}. 
Existen a tres mecanismos~\cite{tsikriktsis2005review} principales para lidiar con información
faltante para este patrón: la eliminación, el sustitución y los  procedimientos basados en
modelo.~\cite{tsikriktsis2005review} recomienda utilizar un mecanismo de
reemplazo para escalas del tipo \textit{Likert}.

Las técnicas de sustitución se clasifican en tres grandes
grupos\cite{tsikriktsis2005review}:
\begin{enumerate*}[label=\itshape\alph*\upshape)]
\item basadas en el promedio,
\item basadas en regresión, e,
\item imputación \emph{hot deck}.
\end{enumerate*}

De estas técnicas se seleccionó \emph{la sustitución} basado en el 
promedio ya que las relaciones entre las variables son bajas y los datos
faltantes son menos del $10\%$. La sustitución basada por promedio se divide
nuevamente en tres grupos\cite{tsikriktsis2005review}; promedio
\begin{enumerate*}[label=\itshape\alph*\upshape.]
\item total,
\item del subgrupo, y,
\item por caso.
\end{enumerate*}

La sustitución por promedio total es elegida debido a que la relación entre la
variable que falta y las demás variables en los datos es relativamente baja, es
fácil de usar y retiene la muestra. 

%La sustitución por promedio total se realiza
%obteniendo el promedio de todas las respuestas de la pregunta cuya respuesta
%falte, la sustitución de subgrupo es similar, solo que se limita a aquellos
%sujetos del mismo subgrupo del sujeto que no respondió, y finalmente, la
%sustitución por caso, es el promedio de las respuestas válidas del sujeto.


\subsection{Resultados obtenidos}
\label{sec:res_subjetiva}

Los resultados obtenidos con la encuesta son separados en \emph{Preguntas cerradas} y 
\emph{Preguntas abiertas} para una mejor comprensión.

\subsubsection{Preguntas cerradas}
La tabla~\ref{tab:subjetiva_conformidad_exploracion} muestra 
las respuestas de los alumnos a las preguntas relacionadas al factor
exploración, son cuatro preguntas, las cuales fueron descritas
en~\ref{sec:sub_exploracion}. Según los datos, la simplificación de las funciones 
de los elementos es el punto con menor valoración, no obstante tiene una valoración 
promedio de \emph{Parcialmente de acuerdo}.
%\observacion{Revisar bien los tiempos}
%\observacion{En este punto uno ya se olvida de la escala}
%\observacion{Algo que resaltar de todas estas tablas?}

\begin{table}[H]
\centering
\begin{tabular}{@{} *{5}{r} @{}}
\toprule
& \multicolumn{4}{c}{Exploración} \\
\cmidrule(lr){2-5}
Alumno &
\parbox{2.5cm}{Facilidad de uso}  &
\parbox{3cm}{Funciones realizadas por los elementos del juego} &
\parbox{3cm}{Aleatoriedad para afianzar conocimientos} &
\parbox{2.5cm}{Aleatoriedad para representar realismo} \\
\midrule
1         & 2   & 6   & 5   & 6  \\
2         & 6   & 6   & 4   & 6  \\
3         & 3   & 3   & 5   & 5  \\
4         & 6   & 6   & 6   & 6  \\
5         & 6   & 6   & 2   & 5  \\
6         & 6   & 6   & 6   & 6  \\
7         & 7   & 7   & 7   & 7  \\
8         & 6   & 6   & 7   & 7  \\
9         & 5   & 7   & 7   & 7  \\
10        & 6   & 7   & 6   & 6  \\
11        & 7   & 6   & 7   & 6  \\
\midrule
\textbf{Promedio}  & \textbf{5}   & \textbf{6}   & \textbf{6}   & \textbf{6} \\
\bottomrule
\end{tabular}
\caption{Resultados de la \emph{Encuesta para evaluar la solución} relacionados al factor exploración}
\label{tab:subjetiva_conformidad_exploracion}
\end{table}

La tabla~\ref{tab:subjetiva_conformidad_representacion} agrupa las respuestas de
los alumnos según la calidad de representación, son cinco preguntas, las cuales
fueron descritas en~\ref{sec:sub_representacion}. Según los datos, los puntos débiles 
son las diferentes respuesta verbales que brinda el paciente virtual y la distinción 
entre los estados del paciente, ambos recibieron en promedio una valoración de 
\emph{Neutral}. El punto más fuerte tiene que ver con los movimientos motrices del 
paciente virtual, con una valoración promedio de \emph{De acuerdo}.

\begin{table}[H]
\centering
\begin{tabular}{@{} *{6}{r} @{}}
\toprule
& \multicolumn{5}{c}{Representación} \\
\cmidrule(lr){2-6}
& & \multicolumn{3}{c}{Respuestas del paciente} & \\
\cmidrule(lr){3-5}
Alumno &
\parbox{2.5cm}{Acciones con los elementos} &
\parbox{2.5cm}{Movimientos oculares del paciente} &
\parbox{2.5cm}{Reacción verbal del paciente} &
\parbox{2.5cm}{Movimientos motrices del paciente} &
\parbox{2.5cm}{Distinción entre los estados del paciente} \\
\midrule
1  & 6 & 6 & 2 & 5 & 2  \\
2  & 4 & 5 & 5 & 6 & 4  \\
3  & 5 & 3 & 3 & 3 & 3  \\
4  & 6 & 5 & 2 & 4 & 2  \\
5  & 2 & 2 & 6 & 6 & 6  \\
6  & 6 & 4 & 6 & 6 & 6  \\
7  & 7 & 6 & 5 & 7 & 5  \\
8  & 6 & 7 & 7 & 7 & 5  \\
9  & 5 & 6 & 2 & 7 & 6  \\
10 & 6 & 4 & 4 & 4 & 5  \\
11 & 6 & 4 & 6 & 6 & 5  \\
\midrule
\textbf{Promedio}  & \textbf{5} & \textbf{5} & \textbf{4} & \textbf{6} & \textbf{4} \\
\bottomrule
\end{tabular}
\caption{Resultados de la \emph{Encuesta para evaluar la solución} relacionados al factor
    representación}
\label{tab:subjetiva_conformidad_representacion}
\end{table}

La tabla~\ref{tab:subjetiva_conformidad_motivacion} muestra las respuestas de
los alumnos a las preguntas relacionadas al factor \textit{Motivación}, son
cinco preguntas, las cuales fueron descritas en~\ref{sec:sub_motivacion}. Según los datos, 
el punto débil es la socialización del rendimiento con una valoración promedio de 
\emph{Parcialmente de acuerdo}. 

\begin{table}[H]
\centering
\begin{tabular}{@{} *{5}{r} @{}}
\toprule
& \multicolumn{4}{c}{Motivación} \\
\cmidrule(lr){2-5}
Alumno &
\parbox{2.5cm}{Importancia del puntaje} &
\parbox{3cm}{Socialización de los puntajes} &
\parbox{3cm}{Medición del tiempo} &
\parbox{2.5cm}{Motivación del puntaje} \\
\midrule
1  & 6 & 4 & 4 & 7  \\
2  & 7 & 4 & 6 & 6  \\
3  & 6 & 6 & 5 & 6  \\
4  & 1 & 4 & 6 & 1  \\
5  & 2 & 2 & 7 & 7  \\
6  & 6 & 5 & 4 & 6  \\
7  & 7 & 7 & 6 & 7  \\
8  & 7 & 7 & 7 & 7  \\
9  & 7 & 7 & 7 & 7  \\
10 & 7 & 4 & 5 & 7  \\
11 & 5 & 4 & 5 & 6  \\
\midrule
\textbf{Promedio}  & \textbf{6}   & \textbf{5}   & \textbf{6}   & \textbf{6} \\
\bottomrule
\end{tabular}
\caption{Resultados de la \emph{Encuesta para evaluar la solución} relacionados al factor motivación}
\label{tab:subjetiva_conformidad_motivacion}
\end{table}

La tabla~\ref{tab:subjetiva_conformidad_inmersion} muestra las respuestas de
los alumnos a las preguntas relacionadas al factor \textit{Inmersión}, son
cinco preguntas, las cuales fueron descritas en~\ref{sec:sub_inmersion}. Según los datos, 
todos los puntos fueron valorados en promedio de la misma manera, \emph{De acuerdo}.

\begin{table}[H]
\centering
\begin{tabular}{@{} *{6}{r} @{}}
\toprule
& \multicolumn{5}{c}{Inmersión} \\
\cmidrule(lr){2-6}
Alumno &
\parbox{2.5cm}{Realismo a través de ordenes verbales}                 &
\parbox{2.5cm}{Escenografía para entrar en ambiente}                  &
\parbox{2.5cm}{Gráficos en tres dimensiones para entender el entorno} &
\parbox{2.5cm}{Sentido de pertenencia}                                &
\parbox{2.5cm}{Juegos cortos}           \\
\midrule
1  & 4 & 6 & 4 & 5 & 3  \\
2  & 6 & 6 & 6 & 6 & 6  \\
3  & 6 & 6 & 6 & 5 & 6  \\
4  & 4 & 6 & 7 & 5 & 6  \\
5  & 6 & 6 & 5 & 6 & 6  \\
6  & 6 & 6 & 6 & 4 & 4  \\
7  & 7 & 7 & 7 & 7 & 7  \\
8  & 6 & 7 & 7 & 7 & 7  \\
9  & 6 & 7 & 7 & 7 & 7  \\
10 & 6 & 3 & 4 & 6 & 6  \\
11 & 5 & 3 & 5 & 5 & 4  \\
\midrule
\textbf{Promedio}  & \textbf{6} & \textbf{6} & \textbf{6} & \textbf{6} & \textbf{6} \\
\bottomrule
\end{tabular}
\caption{Resultados de la \emph{Encuesta para evaluar la solución} relacionados al factor inmersión}
\label{tab:subjetiva_conformidad_inmersion}
\end{table}

La tabla~\ref{tab:subjetiva_conformidad_utilidad} agrupa las respuestas de los
alumnos según la utilidad de la solución, son tres preguntas, las cuales fueron
descritas en~\ref{sec:sub_utilidad}. Según los datos, el punto débil se da en cuanto 
a la utilidad de utilizar al paciente virtual en comparación con un maniquí, con una 
valoración promedio de \emph{Parcialmente de acuerdo}.


\begin{table}[H]
\centering
\begin{tabular}{@{} *{6}{r} @{}}
\toprule
& \multicolumn{3}{c}{Utilidad} \\
\cmidrule(lr){2-4}
Alumno &
\parbox{4cm}{Interacción con el paciente} &
\parbox{4cm}{Provee facilidades para el estudio} &
\parbox{4cm}{Complementa el estudio en clase y laboratorio} \\
\midrule
1  & 7 & 5 & 7  \\
2  & 6 & 6 & 6  \\
3  & 6 & 6 & 6  \\
4  & 2 & 6 & 6  \\
5  & 2 & 6 & 6  \\
6  & 6 & 6 & 6  \\
7  & 7 & 6 & 7  \\
8  & 5 & 6 & 7  \\
9  & 7 & 7 & 7  \\
10 & 1 & 7 & 7  \\
11 & 6 & 4 & 5  \\
\midrule
\textbf{Promedio}  & \textbf{5} & \textbf{6} & \textbf{6} \\
\bottomrule
\end{tabular}
\caption{Resultados de la \emph{Encuesta para evaluar la solución} relacionados al factor utilidad}
\label{tab:subjetiva_conformidad_utilidad}
\end{table}

La tabla~\ref{tab:subjetiva_conformidad_retroalimentacion} agrupa las respuestas
de los alumnos según la calidad de retroalimentación, son tres preguntas, las
cuales fueron descritas en~\ref{sec:sub_retroalimentacion}. Según los datos, los puntos 
débiles son la representación del estado de los objetos de la simulación a través de imágenes y 
el nivel de detalle brindado al usuario acerca de las razones por las que no realizó 
correctamente un paso, ambos poseen una valoración promedio de \emph{Parcialmente de acuerdo}.

\begin{table}[H]
\centering
\begin{tabular}{@{} *{4}{r} @{}}
\toprule
& \multicolumn{3}{c}{Retroalimentación} \\
\cmidrule(lr){2-4}
Alumno &
\parbox{4cm}{Representación iconográfica de conceptos y acciones en la \Gls{gui}}  &
\parbox{4cm}{Retroalimentación suficiente respecto a los pasos realizados} &
\parbox{4cm}{Detalles de los pasos realizados incorrectamente} \\
\midrule
1  & 3 & 2 & 7  \\
2  & 5 & 4 & 6  \\
3  & 3 & 6 & 6  \\
4  & 6 & 6 & 6  \\
5  & 6 & 1 & 6  \\
6  & 2 & 6 & 6  \\
7  & 6 & 7 & 7  \\
8  & 6 & 6 & 7  \\
9  & 6 & 6 & 7  \\
10 & 5 & 4 & 6  \\
11 & 4 & 5 & 6  \\
\midrule
\textbf{Promedio}  & \textbf{5} & \textbf{5} & \textbf{6} \\
\bottomrule
\end{tabular}
\caption{Resultados de la \emph{Encuesta para evaluar la solución} relacionados al factor
    retroalimentación}
\label{tab:subjetiva_conformidad_retroalimentacion}
\end{table}

La tabla~\ref{tab:subjetiva_conformidad_pedagogia} agrupa las respuestas de los
alumnos según el factor pedagógico, son tres preguntas, las cuales fueron
descritas en~\ref{sec:sub_pedagogia}.  Según los datos, 
todos los puntos fueron valorados en promedio de la misma manera, \emph{Pacialmente de acuerdo}.

%\observacion{Habrá que replantear algunos nombres (falta de pistas como)}
\begin{table}[H]
\centering
\begin{tabular}{@{} *{4}{r} @{}}
\toprule
& \multicolumn{3}{c}{Pedagogía} \\
\cmidrule(lr){2-4}
Alumno &
\parbox{4cm}{Acciones a través de botones} &
\parbox{4cm}{Retroalimentación limitada} &
\parbox{4cm}{Potencial para comprender el procedimiento} \\
\midrule
1  & 6 & 6 & 6  \\
2  & 6 & 6 & 7  \\
3  & 4 & 6 & 6  \\
4  & 6 & 7 & 6  \\
5  & 7 & 5 & 6  \\
6  & 4 & 4 & 6  \\
7  & 7 & 6 & 7  \\
8  & 6 & 7 & 7  \\
9  & 7 & 7 & 7  \\
10 & 6 & 7 & 7  \\
11 & 5 & 6 & 5  \\
\midrule
\textbf{Promedio}  & \textbf{6} & \textbf{6} & \textbf{6} \\
\bottomrule
\end{tabular}
\caption{Resultados de la \emph{Encuesta para evaluar la solución} relacionados al factor pedagogía}
\label{tab:subjetiva_conformidad_pedagogia}
\end{table}

%\subsubsection{Agrupamiento de datos}

Los resultados se resumen en la tabla~\ref{tab:subjetiva_conformidad_resumen},
donde se muestra el número de alumno para identificar a un alumno y el promedio de sus
respuestas en la encuesta por factor estudiado. Se puede observar que el promedio total por 
cada factor indica que los de menor valoración son la representación y la retroalimentación 
con una valoración promedio de \emph{Parcialmente de acuerdo}.


Como se explica en la sección~\ref{sec:likert}, estos resultados están sujetos a
tendencias, para ello se aplica el método de doble
estandarización\cite{Pagolu2011}.

%, se muestra el promedio de las mismas.

%\observacion{Retroalimentación esta marcado con un circulo}

\begin{table}[H]
\centering
\begin{tabular}{llllllllr}
\toprule
\textbf{\shortstack{Número de \\alumno}}         &
\begin{sideways}\textbf{Motivación}                    \end{sideways}        &
\begin{sideways}\textbf{Exploración}                     \end{sideways}        &
\begin{sideways}\textbf{Inmersión}                       \end{sideways}        &
\begin{sideways}\textbf{Pedagogía}                       \end{sideways}        &
\begin{sideways}\textbf{Representación}                  \end{sideways}        &
\begin{sideways}\textbf{Retroalimentación}               \end{sideways}        &
\begin{sideways}\textbf{Utilidad}                        \end{sideways}        &
\textbf{\shortstack{Promedio\\de respuestas}}\\
\midrule
1              & 5 & 5 & 4 & 6 & 4 & 4 & 6 & 5 \\
2              & 6 & 6 & 6 & 6 & 5 & 5 & 6 & 6 \\
3              & 4 & 6 & 6 & 5 & 3 & 5 & 6 & 5 \\
4              & 6 & 3 & 6 & 6 & 4 & 6 & 5 & 5 \\
5              & 5 & 5 & 6 & 6 & 4 & 4 & 5 & 5 \\
6              & 6 & 5 & 5 & 5 & 6 & 5 & 6 & 5 \\
7              & 7 & 7 & 7 & 7 & 6 & 7 & 7 & 7 \\
8              & 7 & 7 & 7 & 7 & 6 & 6 & 6 & 7 \\
9              & 7 & 7 & 7 & 7 & 5 & 6 & 7 & 6 \\
10             & 6 & 6 & 5 & 7 & 5 & 5 & 5 & 5 \\
11             & 7 & 5 & 4 & 5 & 5 & 5 & 5 & 5 \\
\midrule
Promedio Total & 6 & 6 & 6 & 6 & 5 & 5 & 6 & 6 \\
\bottomrule
\end{tabular}
\caption{Resultados de la \emph{Encuesta para evaluar la solución}}
\label{tab:subjetiva_conformidad_resumen}
\end{table}

%Se observa que el puntaje más bajo en el promedio final es 5 que significa
%\textit{Parcialmente de acuerdo}, y el más alto es 7, que significa
%\textit{Totalmente de acuerdo}, se observa además el puntaje 6, que significa
%\textit{De acuerdo}. 

Con el resultado final de la estandarización diferenciamos cuáles son los puntos
fuertes y cuáles los puntos débiles de la solución propuesta con respecto a las
respuestas dadas por los usuarios. Estos valores son relativos a las respuestas
originales dadas en la encuesta, los resultados se muestran en la
tabla~\ref{tab:subjetiva_conformidad_corregida}. Los datos demuestran que, el punto 
más fuerte es la utilidad seguida de la exploración, el punto más débil es la representación 
seguida de la retroalimentación. De esta forma se puede valorar mejor cada uno de los factores 
a diferencia de los datos mostrados en~\ref{tab:subjetiva_conformidad_resumen}.

\begin{table}[H]
\centering
\begin{tabular}{lrrrrrrrr}
\toprule
\textbf{\shortstack{Número de \\alumno}}                                &
\begin{sideways}\textbf{Motivación}                    \end{sideways} &
\begin{sideways}\textbf{Exploración}                     \end{sideways} &
\begin{sideways}\textbf{Inmersión}                       \end{sideways} &
\begin{sideways}\textbf{Pedagogía}                       \end{sideways} &
\begin{sideways}\textbf{Representación}                  \end{sideways} &
\begin{sideways}\textbf{Retroalimentación}               \end{sideways} &
\begin{sideways}\textbf{Utilidad}                        \end{sideways} &
\textbf{\shortstack{Promedio\\de respuestas}}\\
\midrule
1              & 0.45 & 0.55 & 0.20 & 0.63 & 0.44 & 0.41 & 0.82 & 0.47 \\
2              & 0.33 & 0.53 & 0.49 & 0.61 & 0.27 & 0.13 & 0.52 & 0.41 \\
3              & 0.17 & 0.86 & 0.87 & 0.67 & 0.13 & 0.67 & 1.00 & 0.60 \\
4              & 0.75 & 0.31 & 0.63 & 0.81 & 0.47 & 0.78 & 0.54 & 0.59 \\
5              & 0.46 & 0.58 & 0.69 & 0.67 & 0.57 & 0.50 & 0.54 & 0.58 \\
6              & 1.00 & 0.73 & 0.68 & 0.42 & 0.90 & 0.67 & 1.00 & 0.78 \\
7              & 1.00 & 0.79 & 1.00 & 0.67 & 0.50 & 0.87 & 0.78 & 0.80 \\
8              & 0.75 & 1.00 & 0.83 & 0.75 & 0.70 & 0.70 & 0.44 & 0.75 \\
9              & 0.90 & 1.00 & 0.93 & 1.00 & 0.64 & 0.92 & 1.00 & 0.90 \\
10             & 0.79 & 0.74 & 0.54 & 0.92 & 0.60 & 0.60 & 0.67 & 0.68 \\
11             & 0.75 & 0.42 & 0.08 & 0.25 & 0.60 & 0.35 & 0.25 & 0.39 \\
\midrule
\textbf{Promedio Total} & 0.67 & 0.68 & 0.63 & 0.67 & 0.53 & 0.60 & 0.69 & 0.63 \\
\bottomrule
\end{tabular}
\caption{Resultados de la \emph{Encuesta para evaluar la solución} con doble estandarización}
\label{tab:subjetiva_conformidad_corregida}
\end{table}

%Es importante notar que los datos de la
%tabla~\ref{tab:subjetiva_conformidad_corregida} son relativas a los datos de la
%tabla~\ref{tab:subjetiva_conformidad_resumen}, es decir, que la representación
%es el punto más débil, aún así, se ve que
%en~\ref{tab:subjetiva_conformidad_resumen} que el valor es $5$ de $7$, lo que
%indica que es un punto aceptable, y entre los factores analizados es el que
%menos aprobación obtuvo.

En la tabla~\ref{tab:resultado_resumen_aspectos_aceptacion} se puede observar el 
resumen de las tablas~\ref{tab:subjetiva_conformidad_resumen} y~\ref{tab:subjetiva_conformidad_corregida}.

%
%Adicionalmente, se puede \emph{Evaluar los puntos fuertes y débiles de la
%    solución}, utilizando los datos con doble estandarización de la
%tabla~\ref{tab:subjetiva_conformidad_corregida}, se crea la
%tabla~\ref{tab:resultado_resumen_aspectos_aceptacion}, donde se observa la
%apreciación de los usuarios por cada aspecto estudiado.

\begin{table}[H]
\centering
\begin{tabular}{lcr}
\toprule
Factores        & Promedio Subjetiva      & Promedio estandarizado \\
\midrule
Motivación        & De acuerdo              & $0.67$  \\
Exploración       & De acuerdo              & $0.68$  \\
Inmersión         & De acuerdo              & $0.63$  \\
Pedagogía         & De acuerdo              & $0.67$  \\
Representación    & Parcialmente de acuerdo & $0.53$  \\
Retroalimentación & Parcialmente de acuerdo & $0.60$  \\
Utilidad          & De acuerdo              & $0.69$  \\
\bottomrule
\end{tabular}
\caption{Aceptación por aspecto de la solución}
\label{tab:resultado_resumen_aspectos_aceptacion}
\end{table}

Para  obtener una mejor visión de las fortalezas y debilidades de la solución
propuesta, se presenta el gráfico de \emph{kiviat}~\ref{fig:subjetiva_kiviat}.

%\observacion{Pulir la manera en la que hacen referencia a este tópico, aclarando
%que son percepciones desde el punto de vista del usuario}
\begin{figure}[H]
\centering
\begin{tikzpicture}[label distance=.15cm]
\tkzKiviatDiagram[scale=1,%
                    lattice=9,
                    %step=10,
                    ]
                {Motivación,
                 Facilidad de Exploración,
                 Sensación de Inmersión,
                 Pedagogía,
                 Representación,
                 Retroalimentación,
                 Utilidad}
\tkzKiviatLine[thick,
                color=blue!25!white,
                mark=ball,
                ball color=blue,
                mark size=5pt,
                opacity=.2, 
                fill=blue!20](6.7,6.8,6.3,6.7,5.3,6.0,6.9)
\tkzKiviatGrad[prefix={$0,$},unity=1](1) 
\end{tikzpicture}
\label{fig:subjetiva_kiviat}
\caption{Gráfico de Kiviat de los factores evaluados}
\end{figure}

%Se observa que las principales debilidades de la solución son la representación
%y la retroalimentación, y las fortalezas la utilidad, pedagogía, exploración, y
%la motivación.

Por último, con las informaciones obtenidas de las respuestas de los usuarios a la encuesta, es posible 
\emph{Validar las consideraciones de diseño asumidas
    durante el desarrollo de la solución}, lo cual es uno de los objetivos de
este capítulo. En la tabla~\ref{tab:resultado_resumen_hipotesis} se observa la
opinión de los alumnos con respecto a las consideraciones de diseño asumidas
en~\ref{sec:hipotesis}, todas las consideraciones fueron aceptadas.

%\begin{table}[!hbt]
%\centering
%\begin{tabular}{lcr}
%\toprule
%Hipótesis                        & Promedio encuesta      & Promedio estandarizado \\
%\midrule
%Comandos de voz con interfaz     & De acuerdo              & $0,55$ \\
%Extracción uniforme de elementos & Parcialmente de acuerdo & $0,65$ \\
%Acciones de bioseguridad         & De acuerdo              & $0,58$ \\
%Representación iconográfica      & Parcialmente de acuerdo & $0,53$ \\
%Factores motivadores             & De acuerdo              & $0,65$ \\
%Falta de pistas                  & De acuerdo              & $0,61$ \\
%\bottomrule
%\end{tabular}
%\caption{Hipótesis con su aceptación}\label{tab:resultado_resumen_hipotesis}
%\end{table}

\begin{table}[H]
\centering
\begin{tabular}{lcr}
\toprule
& \multicolumn{2}{c}{Promedio} \\
%\cmidrule(c){2-3} 
\cmidrule(lr){2-3}
Consideraciones de diseño          & Encuesta                & Estandarizado \\
\midrule
C1. Interacción a través de la voz & De acuerdo              & $0,55$ \\
C2. Extracción de elementos        & Parcialmente de acuerdo & $0,65$ \\
C3. Bioseguridad                   & De acuerdo              & $0,58$ \\
C4. Representación Iconográfica    & Parcialmente de acuerdo & $0,53$ \\
C5. Motivación                     & De acuerdo              & $0,65$ \\
C6. Retroalimentación limitada     & De acuerdo              & $0,61$ \\
C7. Movilidad                      & De acuerdo              & $0,66$ \\
\bottomrule
\end{tabular}
\caption{Aceptación por consideración de diseño}
\label{tab:resultado_resumen_hipotesis}
\end{table}

\subsection{Preguntas abiertas}
\label{sec:res_subjetiva_abiertas}

La parte final de la encuesta que respondieron los alumnos cuenta con preguntas abiertas, 
donde los alumnos expresaron sus
opiniones sobre los aspectos que rodean al uso de este tipo de soluciones al
aprendizaje de enfermería.


\begin{itemize}
    \item El $100\%$ de los alumnos mencionó que este tipo de soluciones son
        beneficiosas para el aprendizaje de procedimientos de enfermería.
    \item El $64\%$ de los alumnos mencionó que la principal dificultad para
        utilizar la solución es el factor tiempo.
    \item El $45\%$ de los alumnos mencionó que la solución es completa,
        mientras que el $18\%$ sugirió más elementos e interacción con el
        paciente.
\end{itemize}


Con esta información se puede \emph{determinar el nivel de aceptación de la
    solución}, se observa que el $100\%$ de los alumnos cree que es beneficioso
contar con este tipo de soluciones.

%! TEX root = ../main.tex

\section{Encuesta para evaluar el conocimiento}
\label{sec:objetiva}

A fin de obtener información acerca del conocimiento de los alumnos  que forman parte de la 
población objetivo, es decir, aquellos que utilizaron 
la solución propuesta y los que no la utilizaron, los cuales constituyen el grupo 
de control, se realiza una encuesta que consta de diez preguntas.

La encuesta mide el nivel de conocimiento del alumno sobre los dos temas
simulados, contiene preguntas de nivel básico, medio y avanzado. Las mismas son
formuladas utilizando la lista de competencias básicas que debe tener un alumno
para aprobar la materia \textbf{Enfermería en Urgencias II}. Las preguntas son
verificadas  por los profesores de la cátedra. Cada pregunta tiene el mismo
peso, así la puntuación más baja obtenible es $0$, y la más alta es $10$.

De esta manera se busca evaluar la influencia pedagógica de la 
solución como herramienta de apoyo al aprendizaje.


\subsection{Muestra}
%\observacion{Se repite mucho lo de las muestras hay 2 universos nomas?}

La población objetivo cuenta con $124$ alumnos, de los cuales $11$ son la muestra seleccionada
para la prueba de la solución, y los $113$ alumnos restantes son utilizados
como grupo de control.

\subsection{Variables}

Se busca medir el puntaje total de los alumnos en la \emph{Encuesta para evaluar el conocimiento}. Esto 
se obtiene de la siguiente manera.

Siendo:

\begin{itemize}
    \item $po_i{_k}$ la respuesta del usuario $i$ a la pregunta $k$
    \item $n$ total de preguntas, igual a 10
    \item $tc$ total de alumnos en el grupo de control, igual a 113.
    \item $t$ total de alumnos, igual a 124
    \item $ts$ total de sujetos de estudio, igual a 11.
\end{itemize}

Se define el puntaje total $pto_i$ del alumno $i$ como, 

\begin{equation*}
    pto_i = \sum_{j=1}^n{po_i{_j}}
\end{equation*}


\subsection{Métricas}

Como se mencionó, la \emph{Encuesta para evaluar el conocimiento} busca medir el rendimiento de los 
alumnos, para ello se utiliza como métrica principal el promedio de acierto, 
tanto del conjunto total de alumnos de la población objetivo, como de los que participaron de la
prueba, y del grupo de control por separado.

Se define el promedio total de los alumnos, $promtotal$ como:

\begin{equation*}
    promtotal = \frac{\sum_{i=1}^t{pto_i}}{t}
\end{equation*}

Se obtienen los promedios del grupo de control ($promcontrol$) y del grupo de alumno que
participo en la prueba para evaluar la solución ($promsujetos$) de la misma manera.

\subsection{Resultados obtenidos}
\label{sec:res_objetiva}

Como se detalló en la sección~\ref{sec:objetiva}, la encuesta realizada a cada
usuario, parte de la prueba, es utilizada para obtener una comparación en cuanto
al rendimiento de los usuarios que forman parte de la muestra y los que forman
parte del grupo de control.


%\observacion{A esta altura ya no se entiende que es promcontrol}
%\observacion{No estaría mal poner algún tipo de información que diga a que
%aspectos se relacionad cada pregunta}

La tabla~\ref{tab:objetiva_rendimiento_por_pregunta} muestra el nivel de acierto
en promedio por pregunta de los usuarios que forman parte de la muestra y de los
que forman parte del grupo de contro. Según estos datos, en el $60\%$ de los casos 
hay una leve mejoría en cuanto al nivel de acierto para los usuarios que forman 
parte de la muestra.

\begin{table}[H]
\centering
\begin{tabular}{lrrr}
\toprule
& \multicolumn{3}{c}{Promedio} \\
\cmidrule(lr){2-4}
\textbf{Pregunta} & 
\textbf{Muestra} & 
\textbf{Grupo Control} & 
\textbf{Total} \\ 
\midrule
ES1. Torniquete           & 0.36 & 0.18 & 0.20 \\
ES2. Guantes              & 0.64 & 0.60 & 0.60 \\
ES3. Manos                & 0.09 & 0.14 & 0.13 \\
ES4. Bioseguridad         & 0.27 & 0.25 & 0.26 \\
ES5. Explicación          & 0.82 & 0.56 & 0.59 \\
\midrule
EG1. Diagnóstico Global 1 & 0.00 & 0.18 & 0.16 \\
EG2. Diagnóstico Global 2 & 0.64 & 0.51 & 0.53 \\
EG3. Respuesta ocular     & 0.45 & 0.28 & 0.29 \\
EG4. Respuesta motora     & 0.18 & 0.32 & 0.31 \\
EG5. Respuesta verbal     & 0.36 & 0.45 & 0.45 \\
\midrule
\textbf{Sumatoria}: & 3.82 & 3.47 & 3.49  \\
\bottomrule
\end{tabular}
\caption{Rendimiento promedio de usuarios por pregunta}
\label{tab:objetiva_rendimiento_por_pregunta}
\end{table}

Los datos sólo sugieren levemente una tendencia a la mejoría de los puntajes
para los usuarios que forman parte de la muestra, sin embargo, estos datos no
pueden ser tomados para realizar conclusiones ya que la cantidad de sesiones de
juego por usuario no se considera suficiente para que el uso de la solución
propuesta afecte realmente en el aprendizaje del usuario. Cabe destacar, que 
tanto la muestra como el grupo de control respondieron a la encuesta luego del examen 
final de la materia \emph{Enfermería en Urgencias II}, la cual requería el dominio de ambos 
temas simulados.

\section{Correlación entre variables}
\label{sec:correlacion}

En esta sección se busca analizar las relaciones que puedan haber entre 
el uso de la solución y el rendimiento en la encuesta para evaluar el conocimiento, 
considerando sólo a los alumnos que participaron de la prueba de la solución.

En la tabla~\ref{tab:all_correlation} se observa la correlación entre seis
variables estudiadas, a fin de observar si existe alguna relación entre los
valores, se utiliza la correlación de \emph{Pearson}, descrita
en~\ref{sec:def_correlacion}. Las variables corresponden al \emph{Registro de
    actividades} y a los resultados de la \emph{Encuesta para medir el
    conocimiento}.


Las correlaciones  más significativas mostradas en la
tabla~\ref{tab:all_correlation}, son:

\begin{itemize}

\item Puntaje máximo obtenido en el procedimiento de venopunción en la solución
    y tiempo  jugado en el procedimiento de venopunción, $0.30$, relación
    positiva moderada. Así como una correlación positiva fuerte ($0,61$) entre
    puntaje máximo obtenido en el procedimiento de \textit{Glasgow} (evaluación)
    y tiempo jugado en el procedimiento de \textit{Glasgow} (evaluación).
    
    Esto podría sugerir que mientras más se utiliza la solución, mejor
    rendimiento se obtiene. Es un punto positivo pues muestra que los usuarios
    aprenden a utilizarla y mejoran con el tiempo.

\item Puntaje máximo obtenido en el procedimiento de venopunción en la solución
    y puntaje obtenido en el examen en lo referente a venopunción, $0.74$,
    relación positiva muy fuerte. Así como una correlación positiva fuerte
    ($0,54$) entre el puntaje máximo obtenido en el procedimiento de
    \textit{Glasgow} (evaluación) y puntaje obtenido en el examen en lo
    referente a \textit{Glasgow}.
    
    Esto podría sugerir que los alumnos con mejor rendimiento en la solución,
    obtuvieron el mejor rendimiento en la evaluación.

\item Tiempo jugado en el procedimiento de \textit{Glasgow} (evaluación) y
    puntaje obtenido en el examen en lo referente a \textit{Glasgow}, $0.86$,
    relación positiva muy fuerte. Lo que puede sugerir que los usuarios que más
    tiempo invirtieron en el procedimiento \textit{Glasgow}, también obtuvieron
    mejor puntaje en el examen.

\item Existe una correlación positiva moderada ($0,29$) entre el tiempo de
    utilización del procedimiento Venopunción, y la utilización del
    procedimiento \textit{Glasgow}, lo que sugiere que los usuarios dedicaron un
    tiempo similar en ambos procedimientos.

\item Existe una correlación positiva moderada ($0.35$) entre el puntaje mayor
    en el procedimiento Venopunción, y el tiempo de juego en el procedimiento
    \textit{Glasgow}, lo que parece indicar que los usuarios que completaron la
    mayor parte del procedimiento Venopunción, dedicaron más tiempo al
    procedimiento \textit{Glasgow}. 

\item Puntaje obtenido en el examen en lo referente a venopunción y puntaje
    obtenido en el examen en lo referente a \textit{Glasgow}, $0.78$, relación
    positiva muy fuerte. Esto podría sugerir que el nivel de conocimiento de los
    alumnos sobre ambos procedimientos está relacionado.

\end{itemize}

\begin{table}[H]
\centering

\begin{tabular}{lrrrrrr}
\toprule
        &
\begin{sideways}\textbf{Puntaje Máx Venopunción (juego)}\end{sideways}  &
\begin{sideways}\textbf{Puntaje Máx Glasgow (juego)}\end{sideways}        &
\begin{sideways}\textbf{Tiempo Jugado Venopunción}\end{sideways}         &
\begin{sideways}\textbf{Tiempo Jugado Glasgow}\end{sideways} &
\begin{sideways}\textbf{Puntaje Venopunción (examen)}\end{sideways}  &
\begin{sideways}\textbf{Puntaje Glasgow (examen)}\end{sideways}    \\
\midrule
Puntaje Máx Venopunción (juego)   & 1             & 0.12          & \textbf{0.30} & \textbf{0.35} & \textbf{0.74} & 0.55 \\
Puntaje Máx Glasgow (juego)       & 0.12          & 1             & 0.32          & \textbf{0.61} & 0             & \textbf{0.54}\\
Tiempo Jugado Venopunción         & \textbf{0.30} & 0.32          & 1             & 0.29          & 0.04          & 0.05\\
Tiempo Jugado Glasgow             & \textbf{0.35} & \textbf{0.61} & 0.29          & 1             & 0.69          & \textbf{0.86}\\
Puntaje Prom Venopunción (examen) & \textbf{0.74} & 0             & 0.04          & 0.69          & 1             & \textbf{0.78} \\
Puntaje Prom Glasgow (examen)     & 0.55          & \textbf{0.54} & 0.05          & \textbf{0.86} & \textbf{0.78} & 1 \\
\bottomrule               
\end{tabular}
\caption{Correlación entre factores estudiados} 
\label{tab:all_correlation}
\end{table}


%! TEX root = ../main.tex

\chapter{Evaluación}
\label{chap:evaluacion}


Este capitulo define los mecanismos utilizados para evaluar la solución
propuesta, los mismos están orientados a la validación de las hipótesis
planteadas durante el desarrollo de la solución, lo que incluye aspectos
pedagógicos, de utilidad y de la participación activa del usuario entre otros
descriptos más adelante. Como parte de la evaluación se miden ciertas variables
relacionadas a los aspectos mencionados.

La evaluación se divide en cuatro partes principales:

\begin{description}
    \item[Encuesta de ubicación] Es una encuesta acerca del nivel de acceso a la
        tecnología que poseen los alumnos del 4to año del \Gls{iab}, de ahora en
        más \textit{el Universo}, esta encuesta sirve para definir la muestra.

    \item[Encuesta Subjetiva] Es una encuesta realizada a cada sujeto que
        participa de la solución, donde se busca la opinión del mismo acerca de
        la solución y factores relacionados a la misma. 

    \item[Encuesta Objetiva] Es un cuestionario que es completado por el
        universo de alumnos, donde se mide el conocimiento de los mismos, se
        utilizan a los alumnos que no son la muestra, como grupo de control.

    \item[Registro] Es información almacenada por la solución automáticamente,
        que contiene datos acerca de su utilización y el desempeño del alumno.
\end{description}


Adicionalmente se realiza una evaluación inicial para medir la calidad de la
interfaz y la interacción con la misma, esta evaluación es realizada con
personas no relacionadas al área de enfermería.

El capitulo define los objetivos de la evaluación, describe brevemente conceptos
transversales a las técnicas utilizadas y luego define las metodologías,
métricas y variables utilizadas en cada experimento.

\section{Objetivos}
\label{sec:objetivos}

La obtención de los registros de actividad y examenes buscan obtener información sobre el
aprendizaje y la utilización de la solución, mientras que la encuesta de
satisfacción busca obtener información acerca de las fortalezas y debilidades de
una simulación para el entrenamiento de enfermeros y de la solución propuesta.

Se definen los objetivos principales de la evaluación como sigue:

\begin{itemize}
\item Validar las hipótesis asumidas durante el desarrollo de la solución.
\item 
\end{itemize}







%Ideas tipo tiro al aire de La princesa de cocho por si sirvan
%Para la evaluacion acerca de la validez de las hipotesis planteadas en este trabajo se hacen uso de metodologias como registros de actividades de los usuarios cuando utilizan la aplicacion y encuestas para valorar la opinion de los mismos sobre las caracteristicas de la aplicacion.

%Los objetivos principales de la evaluacion son los siguientes:

%* Verificar la validez de las hipotesis planteadas en el desarrollo de la solucion.
%* Proponer criterios que puedan utilizarse para la evaluacion de aplicaciones de esta naturaleza.
%* Obtener conclusiones acerca de factores externos que afectan el uso de la aplicacion.
%* Identificar los puntos importantes en los que se debe poner enfasis en el desarrollo de las aplicaciones con esta naturaleza.
%* Obtener sugerencias de las correlaciones entre el registro de actividades y el examenen de conocimientos realizado a los usuarios de la aplicacion.
%* Obtener conclusiones generales acerca del uso de la aplicacion como herramienta de apoyo al aprendizaje de estudiantes de enfermeria.










%! TEX root = ../main.tex
%! TEX root = ../main.tex

\section{Métricas generales}

Existen métricas que son usadas por más de un experimento\revisar{Ver el termino
    correcto}, a continuación se describen estas métricas:

\subsection{Escala de Likert}
\label{sec:likert}

Para la valoración de las variables medidas se utiliza la escala de
Likert\cite{Allen:2007} de 7 valores posibles. La escala de Likert es utilizada
para permitir a las personas indicar cuánto están de acuerdo o en desacuerdo con
respecto a ciertos puntos. Los valores utilizados, son:

\begin{enumerate}
    \item Totalmente en desacuerdo
    \item En desacuerdo
    \item Parcialmente en desacuerdo
    \item Neutral
    \item Parcialmente de acuerdo
    \item De acuerdo
    \item Totalmente de acuerdo
\end{enumerate}

Una vez valoradas y registradas todas las respuestas y con el objetivo de
eliminar las tendencias en la forma en la que son completadas las
encuestas\cite{Fischer2010} se utiliza el método de Doble Estandarización
recomendado en~\cite{Pagolu2011}. Este método, consiste en dos
estandarizaciones, la primera por fila, que en este caso representa a los
individuos y la segunda por columna donde cada columna representa una de las
diferentes preguntas de la encuesta.

Siendo:
\begin{itemize}
	\item $\min_i$ la respuesta de menor valor del usuario $i$.
	\item $\max_i$ la respuesta de mayor valor del usuario $i$.
\end{itemize}

Para cada respuesta $s$ del usuario $i$, el valor ajustado, por la primera 
normalización, $s_1$ se define como:

\begin{equation*}
s_1{_i}=\frac{s-\min_i}{\max_i-\min_i}
\end{equation*}

Y luego siendo:
\begin{itemize}
	\item $groupmin_i$ la respuesta ajustada de menor valor en el grupo $i$.
	\item $groupmax_i$ la respuesta ajustada de mayor valor en el grupo $i$
\end{itemize}

Para cada respuesta ajustada $s_1{_i}$ del usuario $i$, el valor ajustado $sa_i$ se
define como:	

\begin{equation*}
sa_i=\frac{s_{1_i}-groupmin_i}{groupmax_i-groupmin_i}
\end{equation*}

Obteniendo así un valor normalizado, tanto por individuo, como por pregunta, en
el rango $0$ y $1$.

Para la valoración absoluta de cada  item se utiliza la media de cada columna o
respuesta a una pregunta de la encuesta.

Siendo:
\begin{itemize} 
\item $r_{k_i}$ la respuesta del usuario $i$ a la pregunta $k$.
\item $t_k$ la cantidad total de usuarios que respondieron la pregunta $k$.
\end{itemize}

El puntaje promedio de cada pregunta o item evaluado  $p_k$ en la encuesta se
define como:

\begin{equation*}
p_k = \frac{\sum_{i=1}^n{r_{k_i}}}{t_k}
\end{equation*}

\subsubsection{Manejo de información faltante}
\label{sec:informacion_faltante}

\observacion{Falta mejorar}
En toda encuesta pueden existir preguntas que no sean respondidas, y existen
tres posibles formas de categorizar el patrón de ocurrencia de la falta de
respuestas\cite{leite2010performance,tsikriktsis2005review}:

\begin{description}
    \item[Información faltante completamente aleatoria] Cuando la información
        faltante es independiente de la variable medida y de otras variables.
    \item[Información faltante aleatoria] Cuando la información faltante depende
        de otras variables, pero no de la variable en sí. 
    \item[Información faltante no aleatoria] Cuando hay una relación entre la
        información faltante y el valor de la variable.
\end{description}

Existen tres mecanismos\revisar{No repitan}\cite{tsikriktsis2005review}
principales para lidiar con información faltante, eliminación, reemplazo, y
procedimientos basados en modelo.~\cite{tsikriktsis2005review} recomienda
utilizar un mecanismo de reemplazo para escalas del tipo Likert.

Las técnicas de reemplazo se clasifican en tres grandes
grupos\cite{tsikriktsis2005review}:
\begin{enumerate*}[label=\itshape\alph*\upshape.]
\item basadas en el promedio,
\item basadas en regresión, y,
\item imputación \emph{hot deck}.
\end{enumerate*}

La sustitución basada por promedio, se divide nuevamente en tres
grupos\cite{tsikriktsis2005review}; promedio
\begin{enumerate*}[label=\itshape\alph*\upshape.]
\item total,
\item del subgrupo, y,
\item por caso.
\end{enumerate*}

La sustitución del promedio total se realiza obteniendo el promedio de todas las
respuestas de esta pregunta, la sustitución de subgrupo es similar, solo que se
limita a aquellos sujetos del mismo subgrupo del sujeto que no respondió, y
finalmente, la sustitución por caso, es el promedio de las respuestas válidas
del sujeto.

\subsection{Correlación de variables aleatorias}
\label{sec:correlacion}

\observacion{Falta un mini parrafo que explique en forma general que es la
    correlación y después mencionar a pearson}

La correlación de Pearson\cite{BoslaughStatistics2008} mide la relación que
existe entre dos variables, $X$ e $Y$, el mismo esta comprendido entre $-1$ y
$1$, en su punto más bajo ($-1$) indica una de las dos variables crece mientras
la otra decrece, y en su punto más alto ($1$), indica que ambas crecen o
decrecen conjuntamente, el valor $0$, indica que no existe una relación entre
ambas variables.

\replantear{\cite{norman2010likert} menciona que la misma puede ser utilizada
    para variables medidas con la escala de Likert, aún cuando la misma es
    utilizada normalmente para variables cuantitativas.}


%! TEX root = ../main.tex

\section{Encuesta de ubicación}
\label{sec:ubicacion}

A fin de obtener información acerca del nivel de acceso  de los alumnos a la
tecnología, se realiza una encuesta que cuenta con diez preguntas, las cuales
buscan obtener información acerca del modelo de dispositivo móvil, el acceso a
Internet, y la predisposición de cada alumno a ayudar en el experimento.

El \Gls{iab} contó\martin{En que tiempo debe ir? El resto esta en presente.} en
el 2014 con 124 alumnos en el cuarto año distribuidos en tres secciones, el cual es considerado
el Universo. De los 124 alumnos, 93 de ellos estuvieron dispuestos a participar de la prueba
y completaron la encuesta.

Se agrupan a los alumnos en diferentes grupos para determinar si sus
dispositivos celulares son capaces de ejecutar la solución de manera fluida, los
requisitos mínimos para garantizar esta experiencia son:

\begin{itemize}
    \item Sistema Operativo Android 4.0 o superior\todox{Explicar por que
            android}
    \item Memoria ram de 512MB o superior.
    \item Velocidad de procesador de 1 GHz o superior.
    \item GPU \todox{No se como poner la GPU, hay demasiada variedad}
\end{itemize}

Con los resultados de la encuesta de ubicación tecnológica, se seleccionan
aquellos alumnos que posean dispositivos móviles que superan o igualan las
especificaciones, se seleccionan un total de 19 estudiantes.

\martin{Hace falta más detalles? Una sección de métricas y variables, o es
    suficiente con mencionar los criterios mínimos de selección?}

\section{Encuesta objetiva}
\label{sec:objetiva}

A fin de obtener información comparativa acerca del conocimiento de los alumnos
que utilizaron la solución propuesta y los que no la utilizaban, utilizados
como un grupo de control, se realizo un examen, que consta de diez preguntas.

El examen busca medir el nivel de conocimiento del alumno sobre los dos temas
simulados, contiene preguntas de nivel básico, medio y avanzado.

Las preguntas fueron formuladas utilizando la lista de competencias básicas que
debe tener un alumno para aprobar la materia \textbf{Enfermería en Urgencias
    II}, y posteriormente fueron aprobadas por los profesores de la cátedra.

Cada pregunta tiene el mismo peso, así la puntuación más baja obtenible es 0, y
la más alta es 10.

\section{Muestra}

El universo cuenta con 124 alumnos, de los cuales 11 son la muestra seleccionada
para el experimento, entonces se utilizan a los 113 alumnos restantes como un 
grupo de control.

\section{Encuesta subjetiva}
\label{sec:subjetiva}

Al final del periodo de prueba, cada alumno de la muestra completa una encuesta
con 31 preguntas que se utilizan para validar las hipótesis. Las preguntas están
agrupadas en dos, el primer grupo cuenta con 27 preguntas cerradas, es decir de
una sola respuesta en una lista de opciones, el segundo grupo cuenta con 4
preguntas abiertas. 

Cada encuesta es entregada a los alumnos que acordaron participar en el
experimento, mientras completan la encuesta, un guía está presente para
responder cualquier duda.

La métrica utilizada en las preguntas cerradas es la escala de Likert, descrita
en la sección~\ref{sec:likert}.

\subsection{Variables}
\label{sec:variables}

De acuerdo a los objetivos planteados en la sección~\ref{sec:objetivos}, se
busca describir los factores analizados en las pruebas y las variables
relacionadas a los mismos, las cuales, tienen por objetivo demostrar la validez
de las hipótesis planteadas en este trabajo.

Las variables se presentaran agrupadas en factores, los mismos representan
aquellos aspectos de la solución propuesta que buscan ser evaluados.

\subsubsection{Exploración}
\label{sec:sub_exploracion}

Este factor esta relacionado con la característica que posee la solución en
cuanto a la oportunidad que brinda al usuario para explorar cada uno de los
elementos del entorno simulado (paciente, herramientas propias del
procedimiento). En este sentido, se busca proveer facilidad de uso, intuitividad
y realismo en cuanto a las acciones y situaciones que se presentan en la
solución para que de esta manera, los elementos que la componen no representen
para el jugador un obstáculo que impida su uso.

Las variables que miden este aspecto son las siguientes:

\begin{description}

\item[Funciones realizadas por los elementos del juego] se refiere a la
    correctitud con la que una herramienta o elemento del juego representa las
    funciones que el mismo puede realizar en la vida real, en este sentido, se
    evalúa el realismo con el que es representado tal elemento.

\item[Aleatoriedad para afianzar conocimientos] se refiere al beneficio que
    puede traer el hecho de que el estado del paciente en el juego sea aleatorio
    en cuanto a la posibilidad que esto brinda al jugador para poner a prueba
    sus conocimientos teóricos.

\item[Aleatoriedad para representar realismo] se refiere al uso de estados
    aleatorios en el paciente para que de esta forma el procedimiento se asemeje
    mas a una situación real.

\item[Facilidad de uso] se refiere a lo fácil e intuitivo  que puede ser la
    utilización de los elementos del juego.

\end{description}

\subsubsection{Representación}
\label{sec:sub_representacion}

Este factor esta relacionado con la calidad y suficiencia con la que se
representan los diferentes objetos que son simulados en la solución. La
representación abarca tanto funcionalidad como aspecto del objeto.

De esta manera, se busca permitir al jugador realizar con los objetos las
acciones que requiera para llevar a cabo el procedimiento que se le presente en
la solución, y además, representar estos elementos de la mejor manera posible,
de forma realista.

Las variables que miden estos aspectos son las siguientes:

\begin{description}

\item [Movimientos motrices del paciente] se refiere a la suficiencia de los
    movimientos motrices que realiza el paciente en el escena correspondiente a
    la valoración de la escala de Glasgow.

\item [Movimientos oculares del paciente] se refiere a la suficiencia de los
    movimientos oculares que realiza el paciente en la escena correspondiente a
    la valoración del escala de Glasgow.

\item [Reacción verbal del paciente] se refiere a la suficiencia de las
    reacciones o respuestas verbales que realiza el paciente en la escena
    correspondiente a la valoración de la escala de Glasgow.

\item[Distinción entre los estados del paciente] se refiere a si los diferentes
    estados del paciente son distinguidos correctamente en el procedimiento de
    valoración de la Escala de Glasgow ya que esto es importante para que el
    jugador pueda diagnosticar correctamente al paciente.

\item[Acciones las herramientas] se refiere a si las diferentes acciones que
    pueden realizarse con los elementos o herramientas del juego en un
    determinado procedimiento de enfermería son suficientes para ese
    procedimiento, ya que, debido a las limitaciones de la tecnología estas
    acciones son limitadas.

\end{description}

\subsubsection{Gamificación}
\label{sec:sub_gamificacion}

Este factor esta relacionado con la importancia de incluir en la solución
aquellas características que son propias de un juego de vídeo convencional. Se
busca conocer el valor de estas características en cuanto a la motivación que
puedan producir en los jugadores tanto para volver a utilizar la solución como
para superarse en cada juego.

Las variables que miden estos aspectos son las siguientes:

\begin{description}

\item[Motivación del puntaje] se refiere a que tanto motiva al jugador que la
    solución le proporcione un puntaje total al final de cada partida para poder
    mejorar constantemente siendo este puntaje como una evaluación final de todo
    lo que realizo dentro de la partida.

\item[Importancia del puntaje] se refiere a que tan importante es para un
    jugador que se le proporcione un puntaje total al final de cada partida para
    poder visualizar su rendimiento.

\item[Socialización de los puntajes] se refiere a si el hecho de que las
    personas del mismo entorno compartan sus puntajes, experiencias y logros en
    las partidas a través de redes sociales pueda ser motivador.

\item[Medición del tiempo como motivación] se refiere a que tanto motiva al
    jugador que la solución le proporcione el tiempo que duro su partida
    sirviendo este tiempo como una evaluación de su precisión a la hora de
    realizar el procedimiento que se le presente.

\end{description} 


\subsubsection{Inmersión}
\label{sec:sub_inmersion}

Este factor esta relacionado con el sentimiento de formar parte de la escena. Es
decir, se trata de evaluar que tanto un jugador puede sentir que realmente se
encuentra dentro del juego para que de este modo el pueda entrar en ambiente
para realizar los procedimientos que se le presenten en sus partidas de juego.

Las variables que miden este aspecto son las siguientes:

\begin{description}

\item[Escenografía para entrar en ambiente] se refiere a la importancia de la
    escenografía de la partida para que el jugador entre en ambiente para
    realizar el procedimiento que se le presente.

\item[Juegos cortos como ayuda para la repetición] se refiere a si el hecho de
    que los procedimientos presentados en las partidas sean cortos contribuye a
    repetir las partidas varias veces de seguido entrando en un estado de
    inmersión.

\item[Gráficos en tres dimensiones para entender el entorno] se refiere a la
    importancia que tiene el uso de gráficos en tres dimensiones para que el
    jugador pueda entender mejor el entorno y las posibles acciones que puede
    realizar.

\item[Realismo a través de ordenes verbales] se refiere a si el hecho de que la
    solución brinde la posibilidad de que aparezca un menú de ordenes verbales
    en el momento en que el jugador habla hace que la acción de dar ordenes
    verbales se asemeje mas a la realidad.

\item[Simulación como herramienta] se refiere a si la simulación ayuda al
    jugador a sentirse parte del laboratorio, dando cierto realismo a la escena
    que se le presenta.

\end{description}

\subsubsection{Utilidad}
\label{sec:sub_utilidad}

Este factor esta relacionado con lo útil que puede ser la solución como
herramienta de apoyo al proceso de aprendizaje de los estudiantes de enfermería.

Las variables que miden este aspecto son las siguientes:

\begin{description}

\item[Simulación para complementar el estudio en clase y laboratorio] se
    refiere a que tanto las herramientas alternativas como la simulación pueden
    complementar a los métodos de aprendizaje tradicionales que son el estudio
    en clase y en el laboratorio.

\item[Simulación provee más facilidades para el estudio] se refiere a si las
    herramientas alternativas como la solución proveen más facilidades para
    poner en practica los conocimientos con respecto a los demás métodos de
    aprendizaje que son los libros, laboratorios y el campo de practicas.

\item[Interacción con el paciente] se refiere a si el hecho de que el jugador
    pueda interactuar con un paciente que responde a las acciones del jugador es
    mejor que utilizar un maniquí inmóvil como el de los laboratorios de
    practica.

\end{description}

\subsubsection{Retroalimentación}
\label{sec:sub_retroalimentacion}

Este factor esta relacionado con la importancia de ofrecer al jugador
información acerca de sus logros y errores de manera tal que el pueda estar
consciente de sus puntos fuertes y sus puntos débiles en los diversos
procedimientos que realice en la solución.

Las variables que miden este aspecto son las siguientes:

\begin{description}

\item[Detalles de los pasos realizados incorrectamente] se refiere a que tan
    importante es para el jugador que la solución no solo le diga los pasos que
    hizo de manera incorrecta sino también las causas por las cuales no los
    realizo correctamente.

\item[Suficiencia de los detalles de los pasos realizados incorrectamente] se
    refiere a sí son suficientes las justificaciones breves acerca de las causas
    por las cuales que realizo incorrectamente un paso.

\item[Iconos para representar el estado del jugador] se refiere a la
    suficiencia de mostrar iconos en la interfaz de la solución para
    representar el estado actual del jugador.

\end{description}

\subsubsection{Pedagogía}
\label{sec:sub_pedagogia}

Este factor esta relacionado a la utilidad y al beneficio que puede traer la
solución para apoyar el aprendizaje del jugador. De esta manera, se busca
obtener la validez real de este tipo de herramientas como aporte al aprendizaje,
proveyendo mas interacción al jugador.

Las variables que miden este aspecto son las siguientes:

\begin{description}

\item[La solución para memorizar y comprender el procedimiento] se refiere a
    que tanto ayuda la solución al jugador para entender los procedimientos que se
    le presenten y para memorizar los pasos de cada uno de ellos.

\item[Falta de pistas como ayuda al aprendizaje] se refiere a que tan efectivo
    resulta no dar pistas al jugador en el momento de realizar un procedimiento
    para que pueda plasmar y medir sus conocimientos.

\item[Suficiencia de los botones que indican acciones] se refiere a que tan
    suficiente es representar determinadas acciones  con un botón debido a
    limitaciones en la tecnología.

\end{description}

\section{Registro de actividades}

La solución propuesta almacena información relacionada a la actividad del
usuario, incluyendo cuando y como utiliza las opciones, los pasos que realiza,
el orden y las condiciones de la escena cuando realiza cada acción.

El registro como un todo es enviado cada vez que el usuario desee, este envío
requiere una conexión a internet por ello no es automático. Adicionalmente el
último día de la prueba, todos los registros fueron enviados para que sean
analizados.


\todox{Agregar metricas y variables}

\section{Interfaz de usuario}

La primera fue realizada con alumnos de la carrera de Ingeniería en Informática
de la Facultad Politécnica que pertenece a la Universidad Nacional de Asunción,
sin experiencia previa tanto con la solución como con los procedimientos
simulados, pero si familizarizados con la utilización de dispositivos móviles.

\cite{nielsen2000} recomienda una muestra de 5 personas para pruebas de
usabilidad.~\cite{ritch2009} menciona que con 5 individuos, se encuentran 85\%
de los errores en promedio, y que un grupo de 5 a 10 personas es adecuado para
pruebas de usabilidad sencillas.

Esta prueba no es de gran complejidad, el procedimiento es sencillo y esta
bien definido, se busca determinar que problemas presenta la interfaz, que
impedimentos encuentran usuarios acostumbrados a la tecnología pero no al
procedimiento, por ello se elige una muestra de 8 alumnos.

\subsection{Muestra}

La primera fue realizada con alumnos de la carrera de Ingeniería en Informática
de la Facultad Politécnica que pertenece a la Universidad Nacional de Asunción,
sin experiencia previa tanto con la solución como con los procedimientos
simulados, pero si familizarizados con la utilización de dispositivos móviles.

\cite{nielsen2000} recomienda una muestra de 5 personas para pruebas de
usabilidad.~\cite{ritch2009} menciona que con 5 individuos, se encuentran 85\%
de los errores en promedio, y que un grupo de 5 a 10 personas es adecuado para
pruebas de usabilidad sencillas.

Esta prueba no es de gran complejidad, el procedimiento es sencillo y esta
bien definido, se busca determinar que problemas presenta la interfaz, que
impedimentos encuentran usuarios acostumbrados a la tecnología pero no al
procedimiento, por ello se elige una muestra de 8 alumnos.



\printbibliography

\end{document}


