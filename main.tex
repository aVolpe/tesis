\documentclass[final,fmstyle]{./util/ucathesis}

\usepackage[T1]{fontenc}
\usepackage[spanish]{babel}
\usepackage[utf8]{inputenc}
\usepackage{csquotes}
\usepackage{graphicx}
\usepackage[showframe=false]{geometry}
\usepackage{pdflscape}
\usepackage[inline]{enumitem}
\usepackage{pgfgantt}
\usepackage[bookmarks]{hyperref}
\usepackage{changepage}
\usepackage{booktabs}
\usepackage{listings}
\usepackage{xcolor}
\usepackage{colortbl}
\usepackage{xargs}
\usepackage{rotating}
\usepackage{tkz-kiviat}
\usepackage{multirow}
\usepackage[normalem]{ulem}
\newcommand{\removed}[1]{\uline{#1}}


\usepackage{tikz}
\usepackage[style=numeric,sorting=none,backend=biber]{biblatex}

\usepackage{glossaries}
\usepackage{etoolbox}
\usepackage[xindy]{imakeidx}

\usepackage[colorinlistoftodos,prependcaption,spanish]{todonotes}

\DeclareUnicodeCharacter{00E3}{á}


\usetikzlibrary{shapes,arrows}

\tikzstyle{block} = [rectangle, draw, fill=gray!20, 
    text width=10em, text centered, rounded corners, minimum height=4em]
\tikzstyle{line} = [draw, -latex']

%\MakePerPage{footnote}
\addbibresource{bibliography.bib} 

%Datos de la tesís
\title{Construccionismo como complemento a la enseñanza tradicional:  una
	aplicación a la formación de profesionales del área de enfermería}
\author{Mirta González y Arturo Volpe}
\degree{Informática}

\advisor{Ing.}{Martín Abente Lahaye,M.Sc.}

\logosource{./graphics/logo.jpg}
\institution{Universidad Nacional de Asunción}
\faculty{Facultad Politécnica}
\address{San Lorenzo - Paraguay}

%% templates
\lstdefinestyle{sharpc}{language=[Sharp]C, frame=lr, rulecolor=\color{blue!80!black}}

\newtoks\customtok


\renewcommand*{\newacronymhook}{%
 \edef\dosetkeys{\noexpand\setkeys{glossentry}{user1={},\the\glskeylisttok}}%
 \dosetkeys
 \ifcsempty{@glo@useri}%
 {%
   \expandafter\customtok\expandafter{\the\glsshorttok}%
 }%
 {%
   \edef\custom{\the\glsshorttok, \csexpandonce{@glo@useri}}%
   \expandafter\customtok\expandafter{\custom}%
 }%
}

\newcommand*{\custompostdesc}[1]{%
  \ifcsempty{glo@#1@useri}{}{(\glsentryuseri{#1})}%
}

\renewcommand*{\CustomAcronymFields}{%
  user1={},%
  name={\the\glsshorttok},%
  description={\the\glslongtok\noexpand\custompostdesc{\the\glslabeltok}},%
  first={\the\glslongtok\space(\the\customtok)},%
  firstplural={\the\glslongtok\noexpand\acrpluralsuffix\space(\the\customtok)}%
  text={\the\glsshorttok},%
  plural={\the\glsshorttok\noexpand\acrpluralsuffix}%
}

\newcommandx{\todox}[2][1=]{\todo[linecolor=gray,backgroundcolor=white,#1]{#2}}
\newcommandx{\martin}[2][1=]{\todo[linecolor=blue,backgroundcolor=white,#1]{#2}}
\newcommandx{\replantear}[2][1=]{\todo[inline,caption={Replantear},#1]{#2}}
\newcommandx{\observacion}[2][1=]{\todo[inline,caption={Observacion},#1]{#2}}
\newcommandx{\pregunta}[2][1=]{\todo[inline,caption={Pregunta},#1]{¿#2?}}
\newcommandx{\revisar}[2][1=]{\todo[size=\small,caption={Revision},linecolor=red,backgroundcolor=white,#1]{#2}}
\newcommand{\tabletodo}[2][]{\begin{minipage}{3cm}\todo[inline,#1]{#2}\end{minipage}}
\newcommand{\fixme}[2]{\uline{#1}\todo[linecolor=blue,backgroundcolor=green,size=\small]{#2}}

%\newcommand{\fixme}[2]{\uline{#1}\todo[linecolor=black,backgroundcolor=green,size=\Small]{\begin{spacing}{0.5}#2\end{spacing}}}

% COLOR para las tablas
\definecolor{gris}{gray}{0.85}
\definecolor{agua}{rgb}{0.88,1,1}

\SetCustomStyle
\makeglossaries
\makeindex

\begin{document}

\newacronym[user1=Conseil Européen pour la Recherche Nucléaire]{cern}{CERN}{Organización Europea para la Investigación Nuclear}
\newacronym[user1=One Laptop Per Child]{olpc}{OLPC}{Una computadora por niño}
\newacronym[user1=Massachusetts Institute of Technology]{mit}{MIT}{Instituto Tecnológico de Massachusetts}
\newacronym{tic}{TIC's}{Tecnologías de la información y la comunicación}
\newacronym[user1=Event-Condition-Action]{eca}{ECA}{acciones condicionadas por eventos}
\newacronym{iab}{IAB}{Instituto Andrés Barbero}
\newacronym[user1=Graphics Processing Unit]{gpu}{GPU}{Unidad de procesamiento de gráficos}
\newacronym[user1=Application Programming Interface]{api}{API}{Interfaz de programación de aplicaciones }



\maketitle

% Tabla de contenidos
\tableofcontents
% Lista de figuras
\listoffigures
% Lista de tablas
\listoftables
% Lista de algoritmos
\listofalgorithms

\listoftodos
\todototoc

\printglossary[type=\acronymtype,title=Lista de Siglas]

\addcontentsline{toc}{chapter}{Lista de Siglas}

\mainmatter

%! TEX root = ../main.tex
\chapter{Introducción}


En la actualidad, la educación tradicional o instruccionismo se basa en el
concepto de que el profesor transfiere el conocimiento que ha adquirido de
diferentes métodos (educación, experiencia, etc.) a un alumno que es un receptor
pasivo de información\cite{laptop:instructionism}. 

En la educación tradicional el uso de las \Gls{tic} cumplía con un rol en el 
cual sólo eran un mecanismo más para transmitir el conocimiento del maestro al alumno,
reemplazando libros y presentaciones. Actualmente, el uso de las tecnologías tiene un
papel más activo dentro del proceso de aprendizaje.

Además del instruccionismo existen otras corrientes pedagógicas relacionadas a
las \Gls{tic}, entre las que se pueden citar el conductismo, el constructivismo
y el construccionismo. Este trabajo se centra en el construccionismo.

El construccionismo tiene un enfoque diferente en cuanto a las \Gls{tic} ya que
utiliza la tecnología como un mecanismo de adquisición de conocimiento y no como
una herramienta proveedora del mismo\cite{sasha:construtivism}, es decir se
centra en como se generan los conocimientos y no en como proveer información
previamente establecida. 

Algunos de los proyectos que incluyen al construccionismo son: 
el lenguaje de programación LOGO, las simulaciones educativas de entornos virtuales, 
los juegos serios y Lego Serious Play.

El aprendizaje apoyado por las \Gls{tic} y el construccionismo utilizando
específicamente los juegos serios tienen la capacidad de eliminar los problemas
de distancia, en el ámbito empresarial se utilizan los juegos serios para
enseñar a grupos de personas a trabajar en equipo, incluso cuando estos se
encuentran a distancias que le impiden reunirse en una clase
tradicional\cite{mariluz:seiousgames}, es así que estudios anteriores han
demostrado que el construccionismo es una alternativa muy prometedora al
instruccionismo\cite{sasha:construtivism}.
   
De los proyectos citados anteriormente, la simulación es una opción tecnológica
propicia para el construccionismo ya que permite utilizar la exploración y el
ensayo para desarrollar habilidades y pericias en entornos
controlados\cite{humphreys2013developing}. La simulación puede ser complementada
con la utilización de juegos serios ya que los mismos proveen una oportunidad
muy importante para enseñar y desarrollar profesionales\cite{sg:aoverview}.

Basándonos en lo explicado anteriormente, se considera interesante el campo de
investigación que involucra el desarrollo de aplicaciones tecnológicas que
ayuden a estudiantes en el proceso de aprendizaje. 

Estas herramientas tienen especial importancia en ambientes donde las
limitaciones, de espacio, y tiempo dificultan la aplicación de técnicas
tradicionales. Uno de estos casos es el entrenamiento de profesionales de
Enfermería los cuales requieren de varias horas de práctica durante su
preparación, y la misma se realiza en instituciones como hospitales escuela,
donde los alumnos son supervisados por profesionales mientras realizan las
prácticas.

Además, según \cite{humphreys2013developing} los alumnos de enfermería, son
estudiantes divergentes, es decir aprenden a través de experimentación activa, e
interiorizan el conocimiento reflexionando sobre la experiencia. La simulación
es una herramienta ideal para este tipo de
estudiantes\cite{humphreys2013developing}.

Teniendo en cuenta que los mismos forman parte de un grupo de profesionales que
requieren de un alto grado de prácticas y que en la actualidad se presentan
varios factores que a un estudiante de esta carrera le impide poner a prueba
todo el tiempo sus conocimientos y por todo lo expuesto anteriormente, en este
trabajo se propone el desarrollo de un juego serio, que involucra la simulación
de laboratorios virtuales, como una herramienta para el proceso de aprendizaje
de los alumnos de la carrera de enfermería.

%! TEX root = ../main.tex
\section{Objetivo General}
\label{sec:objetivos_generales}

Realizar una investigación sobre el rol actual de las tecnologías de la
información y la comunicación en la educación, poniendo énfasis en las 
corrientes pedagógicas actuales, y evaluar las distintas tecnologías 
disponibles para introducirlo en un área especifica.

% Enfoque 3.89
%El objetivo general de la presente tesis es investigar, estudiar y evaluar 
%las diferentes tecnologías disponibles para apoyar el proceso de aprendizaje 
%utilizando como base pedagógica el construccionismo para  diseñar un esquema 
%de desarrollo que se pueda implementar en el área de enfermería. 
%\observacion{Creo que es la segunda vez que pregunto. Pero lo de enfermería es un
%    caso de prueba, no es el objetivo general (ver correcciones generales)}


% Enqoue NEGATIVO
% \fixme{diseñar e implementar un
    %aplicación}{No puede ser un objetivo general!} 
%estudiar y evaluar tecnologías que permitan  apoyar el
%proceso de aprendizaje de los alumnos de la carrera de enfermería, utilizando
%como base pedagógica el construccionismo.


% Enfoque 1
%% Diseñar un esquema de desarrollo de tecnologías que permitan complementar el
%% proceso de aprendizaje de los alumnos de la carrera de enfermería, utilizando
%% como base pedagógica el construccionismo.

% Enfoque 2
%% Investigar, estudiar y evaluar las aplicaciones de juegos serios 
%% construccionistas como herramientas complementarias a la educación tradicional.
\section{Objetivos Específicos}

Con el fin de aproximarnos a nuestro propósito, se formulan los siguientes
objetivos específicos:

\begin{enumerate}
    \item Proveer un resumen de los fundamentos y estado actual de la corriente
        \emph{Construccionismo}, como herramienta pedagógica y su relación con
        las \gls{tic}.
        % Proveer un resumen de los fundamentos y estado del arte de las
        % corrientes actuales relacionadas con las \gls{tic}.

    \item Proveer un resumen actualizado acerca de los juegos serios y
        corrientes afines como herramientas para la aplicación del
        construccionismo.
    
    \item Proveer un resumen actualizado de las áreas de aplicabilidad de los
        juegos serios, poniendo énfasis en las áreas que permiten un enfoque
        construccionista.
        
    \item Identificar las características del área de enfermería que hacen que
        la misma sea un contexto factible para la aplicación de los juegos
        serios basados en el construccionismo.
    
    \item Analizar, evaluar y seleccionar las herramientas que nos permitan la
        implementación de un juego serio que simule un laboratorio de
        enfermería.
        
    \item Diseñar e implementar un juego serio construccionista que permita
        exponer las ventajas y desventajas como modelo de apoyo a la enseñanza
        tradicional. 
        %\observacion{Recuerden tener/exponer las ventajas y desventajas en la
        %    conclusión}

    \item Evaluar la solución propuesta para la obtención de datos que permitan
        medir las fortalezas y debilidades desde el punto de vista del usuario,
        en cuanto a factores de:
        \begin{enumerate*}[label=\itshape\alph*\upshape)]
            \item exploración,
            \item representación,
            \item motivación,
            \item inmersión,
            \item utilidad,
            \item retroalimentación, y,
            \item pedagogía.
        \end{enumerate*}

    \item Identificar las fortalezas y debilidades de los métodos utilizados
        para definir su aplicabilidad como herramienta de apoyo. 

    \item Identificar desde el ámbito del diseño, desarrollo y evaluación los puntos que deben
        tenerse en cuenta a la hora de implementar este tipo de herramientas de
        apoyo.
\end{enumerate}

\section{Estructura del libro}
    

%\fixme{A continuación se explica el proceso que fue realizado para alcanzar los objetivos citados 
%    anteriormente, las etapas de este proceso se presentan en diferentes capítulos los cuales 
%    tratan de diversos aspectos.}{Borrar}

% TIC's en la educación
%En el capitulo~\ref{chap:tics} se expande el estado actual de las \Gls{tic} en
%la educación, desde sus inicios, las expectativas creadas, los progresos
%realizados, y las experiencias, la evolución de la teoría hasta hoy en día.

En el capítulo~\ref{chap:tics} se presenta un resumen sobre el uso de las
\Gls{tic} en la educación, desde sus inicios apoyando a la educación
tradicional hasta su rol actual en las nuevas corrientes pedagógicas. Se
hace énfasis en la corriente pedagógica llamada construccionismo, explicando
su historia, base pedagógica, el rol y la forma de uso de las \Gls{tic} en
la misma.

% Juegos serios
%En el capitulo~\ref{chap:juegos_serios} se describen las características y áreas de aplicación de un 
%juego serio, incluyendo corrientes relacionadas y algunos casos de éxito. Se muestra 
%además un esbozo de como desarrollar un juego serio.

En el capítulo~\ref{chap:juegos_serios} se describen a los juegos serios, siendo 
estos una de las herramientas tecnológicas usadas dentro del proceso de aprendizaje, 
se detallan sus características, áreas de aplicación, las corrientes tecnológicas relacionadas, 
y el proceso de desarrollo además de describir algunos casos de éxito. También se describen 
tecnologías relacionadas a los juegos serios.


% Definición del problema

En el capítulo~\ref{chap:problema} se definen las características del problema
que se quiere abordar para darle una solución tecnológica. Enfocándonos en el
área de enfermería se describe la situación actual y una propuesta de solución tecnológica 
que puede brindar apoyo en el proceso de aprendizaje.



%nseñanza utilizado en el \Gls{iab}, enfocado específicamente a la carrera de
%Licenciatura en Enfermería, la distribución de los cursos, carga horaria,
%prácticas de laboratorio y prácticas en hospitales, así como los mecanismos de
%evaluación utilizados.

%Además se citan los problemas existentes, como las dificultades que tienen los
%alumnos en los distintos aspectos que influyen en la vida de un estudiante, y
%problemas inherentes a la enseñanza de profesiones técnicas con métodos de
%enseñanza tradicionales. Y por último se describen los procedimientos de enfermería 
%en los que se enfocará este trabajo.


% Hipotesis y requerimientos de la solucion

En el capítulo \ref{chap:requerimientos} se definen los criterios que deben tenerse en 
cuenta para seleccionar el contenido a abordar en la solución, se seleccionan y describen 
los procedimientos de enfermería que formarán parte del contenido, el alcance  
y los requisitos que debe cumplir la solución.

%En el capítulo~\ref{chap:requerimientos} se describen los requisitos que deben 
%tenerse en cuenta para el desarrollo de la solución propuesta, incluyendo las hipótesis asumidas
%con respecto a diferentes aspectos de los procedimientos de enfermería.

%Tecnologías utilizadas

En el capítulo~\ref{chap:tecnologias} se describen las diversas tecnologías utilizadas 
en el desarrollo del proyecto, haciendo especial énfasis en los motores de juego ya que 
los mismos proveen el entorno principal para el desarrollo de la solución, se realiza 
una comparación entre los motores de juegos actuales y se justifica por que se seleccionó 
uno en específico. 


% Propuesta de solución

%El capitulo~\ref{chap:solucion} describe una solución  para la aplicabilidad de juegos serios 
%en un contexto con las características mencionadas en el capitulo anterior utilizando los conceptos
%estudiados en el capítulo~\ref{chap:tics}, se proponen, además, consideraciones
%a tener en cuenta referentes a los componentes e interacción entre los mismos.

%Así mismo se definen mecanismos para tratar con los limites de una simulación, y
%como utilizar las herramientas descritas en el capítulo~\ref{chap:tics} en
%armonía con los objetivos pedagógicos.

En el capítulo~\ref{chap:solucion} se describe en detalle la solución propuesta basada en 
la aplicabilidad de los juegos serios en un contexto con las características mencionadas en 
el capítulo~\ref{chap:problema} y teniendo en cuenta las requisitos detallados en el 
capítulo~\ref{chap:requerimientos} haciendo uso de las tecnologías descritas en el capítulo 
anterior. 

% Definición de evaluación

Dada la solución propuesta, en el capítulo~\ref{chap:evaluacion} se propone una
forma de evaluarla, teniendo en cuenta las hipótesis planteadas durante su
desarrollo  y los objetivos del presente trabajo. Esta evaluación consiste en
una serie de pruebas, se define el universo, la muestra, y los criterios de
selección de la muestra.

% Análisis de resultados

Se presentan además los resultados de las diferentes
pruebas llevadas a cabo, obtenidos en forma tabular y con gráficos para
facilitar la comprensión. %Además se estudian factores de correlación entre las
%distintas variables medidas.

% Conclusiones

En el capítulo~\ref{chap:conclusion} se presentan las
conclusiones obtenidas a partir de los resultados obtenidos y la experiencia
adquirida durante el desarrollo de la solución.

% Trabajo furuto

Finalmente, en el capítulo~\ref{chap:futuro} se describen algunos trabajos posibles que pueden
ser desarrollados en el área teniendo en cuenta el presente trabajo. 

%\observacion{Volver a revisar más adelante}


%! TEX root = ../main.tex
\chapter{TIC's en la Educación}
\label{chap:tics}

\observacion{Enfocar desde un marco común (tendencias)}

Las \Gls{tic} son un conjunto de herramientas tecnológicas y recursos utilizados
para comunicar, crear, diseminar, almacenar y manejar la
información\cite{unesco:ict}. Estas tecnologías abarcan computadoras personales,
internet, radio, televisión y telefonía\cite{tinio:ict}.

Las \Gls{tic} fueron utilizadas como complemento a la educación desde los
inicios de la misma con la radio y la televisión. Fueron vistas como un
complemento a las herramientas utilizadas en clase, como complemento del libro,
o como una herramienta que elimina la distancia física entre el profesor y el
alumno\cite{unesco:ict}. 

A continuación se describe en detalle la relación de las  \Gls{tic} con la educación, 
explicando las corrientes pedagógicas conocidas como instruccionismo y construccionismo 
para detallar la relación de las mismas con la tecnología como herramienta de apoyo.

\section{Educación Tradicional}

\fixme{La educación tradicional o instruccionismo se basa en el concepto de que
    existe un profesor y un alumno. El profesor transfiere el conocimiento que
    ha adquirido de diferentes métodos (educación, experiencia, etc) a un alumno
    que es un receptor pasivo de información\cite{johnson2005instructionism}.
}{No copy/paste en la intro}

Se enfoca más en el profesor, y en la enseñanza, y en el producto final como
resultado de un proceso no interactivo y bien
documentado\cite{igi:instructionism}. Los mecanismos tradicionales para poder
probar la efectividad de este tipo de enseñanza son los exámenes.

\fixme{La principal}{Ablandar} critica a este modelo es que se enfoca la
enseñanza y no el aprendizaje, mientras más se enseña, más se aprende.
\textbf{Esto contradice al sentido común en el sentido de que, cosas básicas
    como caminar o hablar, aprendemos sin la necesidad de un
    profesor\cite{ackoff:education}\cite{johnson2005instructionism}.}{Más
    énfasis en el aprender haciendo y usar referencias, no ``el sentido común''}

Tradicionalmente el rol de las TIC's en la educación se vio relegada a la de
sustituto del libro y/o de presentaciones en clase, es decir, es un mecanismo
más para trasmitir el conocimiento del maestro al alumno.


\section{TIC's en la educación}

\begin{frame}{Instruccionismo}
\end{frame}
\begin{frame}{Conductismo}
\end{frame}
\begin{frame}{Constructivismo}
\end{frame}
\begin{frame}{Construccionismo}
\end{frame}
\begin{frame}{Ventajas}

    \begin{itemize}[<+->]
        \item Nuevos modelos pedagógicos
        \item Eliminación de distancias
        \item Colaboración distribuida
        \item Motivación para aprender
        \item Adquisición de habilidades básicas
    \end{itemize}
\end{frame}
\begin{frame}{Desafíos}
\end{frame}

\section{Problemas actuales}

Durante la historia de las \Gls{tic} en la educación, se han encontrado
diferentes dificultades a la hora de aplicar los nuevos conceptos en la
educación, desde los primeros enfoques que carecían de bases pedagógicas válidas
hasta la actualidad.

El principal problema es falta de motivación de los profesionales de la
educación para emplear las \Gls{tic}\cite{punie:ict}\cite{ict:romeo}.

El contenido proveído actualmente puede ser considerado como un conjunto de
buenas prácticas\cite{punie:ict} y así, omiten completa o parcialmente el
contexto donde esa buena práctica fue generado.

Las \Gls{tic} han tenido un impacto positivo en la educación\cite{punie:ict},
pero no han obtenido el impacto esperado.

Iniciativas como el \emph{edutainment} que prometían ser la solución a los
problemas educacionales no cumplieron las expectativas. Sucesivos fracasos en
los resultados obtenidos dotaron a los \emph{edutainment} de una reputación
negativa, y hoy en día son considerados como el peor tipo de educación, pues son
un ejercicio de \emph{prueba-error} ocultos bajo un juego poco
entretenido\cite{resnick:2004}. La principal critica contra los
\emph{edutainment} es su incapacidad de enseñar como aplicar conceptos
aprendidos a un entorno real\cite{resnick:2004}.

Mientras que la utilización de las \Gls{tic} puede eliminar problemas actuales
como el aislamiento y la falta de pensamiento de alto nivel\cite{punie:ict}, la
brecha social existente implica otro riesgo para la utilización de las \Gls{tic}
en la educación, aquellos que no posean los recursos económicos necesarios para
acceder a la misma no se verán beneficiados por las \Gls{tic}\cite{punie:ict}.

Muchas empresas que están en el área de las \Gls{tic} en educación siguen en la
época donde los juegos son prueba y error, esto no significa que los mismos no
funcionen, sino que pueden ser mejorados
considerablemente\cite{egenfeldt2007third}.

Otro de los problemas actuales es la dificultad comercial impuesta por la
historia de los mismos, es muy difícil para los juegos actuales presentar
promesas realistas, principalmente por el antecedente sentado por los
edutainment\cite{egenfeldt2007third}
\section{Construccionismo y las TIC's}
\label{sec:tics_CONSTRUCCIONISMO}

\fixme{El construccionismo es una corriente}{decir primero que es, antes de
    comprara} pedagógica con un enfoque diferente en cuanto al uso de las
\Gls{tic} en la educación. Esta pedagogía se diferencia de la educación
tradicional en que el estudiante ya no es un receptor pasivo de información, en
cambio, el mismo participa activamente del proceso de aprendizaje construyendo
su propio conocimiento. 

\fixme{El construccionismo}{no repetir} utiliza la tecnología como medio
cognitivo  a \fixme{diferencia}{} de la educación tradicional que la utiliza para la
entrega de contenido. 

\fixme{El construccionismo}{no repetir} es un alternativa prometedora a la
educación tradicional. Desde el punto de vista tecnológico, el construccionismo
es ideal pues el mismo requiere un alto dinamismo en el traspaso del
conocimiento \cite{sasha:construtivism}. 

\fixme{El construccionismo}{no repetir} y las \Gls{tic} siempre han estado
relacionados, ya que el mismo se originó con un lenguaje de programación
(LOGO)\cite{ict:ttc}. Un característica importante de esta relación es que
tienen la capacidad de eliminar los problemas de
distancia\cite{mariluz:seiousgames}.


\subsection{Historia}

En la decada de $1980$, \emph{Seymour Papert} adoptó el término construccionismo
para representar una método pedagógico practicado por \fixme{John Dewey}{?} a
principios del siglo 20. Este método buscaba que la responsabilidad de aprender
recaiga en el estudiante. 

Papert trabajó directamente con el psicólogo evolutivo y filósofo suizo Jean
Piaget. \fixme{Este}{qué?} último, había elaborado con anterioridad sus teorías
de la educación y construcción del conocimiento al ver e interactuar con los
niños y a partir de esta observación dio origen al constructivismo, según el
cual, el conocimiento debe ser construido por el estudiante y los nuevos
significados deben ser obtenidos relacionándolos con significados anteriores por
los mismos estudiantes haciendo uso así de sus propios sistemas de relaciones.

El construccionismo se \fixme{diferencia}{} de lo anterior en que los estudiantes
construyen las ideas o partes del mundo utilizando herramientas. La elaboración
de representaciones mentales mediante la construcción y el intercambio es la
metáfora del marco construccionista. 

Durante $1980$, Seymor Papert, Wally Feurzeig, Marvin Minsky y John McCarthy y los
miembros del Departamento de Inteligencia Artificial del \Gls{mit} y una
compañía de tecnología en Cambridge, Massachusetts, desarrollaron un nuevo
lenguaje de programación llamado LOGO que tenía por objeto que los estudiantes
construyeran sus \fixme{modelos}{que modelos} en notación LOGO@. Este juego
introduciría de forma natural las ideas de los procedimientos, funciones,
variables, recursividad, la modularidad, simulación, verificación, entre otros.

La creación del lenguaje de programación LOGO dio inicio al construccionismo.

Los desarrolladores de LOGO no solo alentaron la promoción de formas
construccionistas de enseñanza y aprendizaje sino también alentaron otra forma
de aprendizaje nueva y no tradicional con las diferentes herramientas tecnológicas. 

De vuelta en la década de 1980, cuando se produjo LOGO y se acuñó el
construccionismo, la comunidad del construccionismo era en su mayoría  ingenieros
informáticos y matemáticos\cite{historia:2014}.

\observacion{Toda la sección hay que reordernar}

\subsection{Bases Pedagógicas}

Para el construccionismo, el conocimiento es construido por el estudiante en
lugar de ser trasmitido por el \fixme{profesor}{explicar el rol del profesor en
    el construccionismo}\cite{moses:2003} y esto sucede particularmente cuando
el mismo se compromete en la elaboración de un producto o artefacto que tenga un
significado y pueda ser compartido\cite{valdivia:sg}. De esta manera, se permite
a los estudiantes elaborar sus propias interpretaciones razonadas del mundo
mediante la interacción con el mismo.

Según Papert, los alumnos estarán mucho más involucrados en su aprendizaje si
construyen artefactos que los demás pueden ver, criticar y tal vez utilizar. Y
además, el alumno se enfrenta a problemas complejos con estas construcciones,
harán el esfuerzo por resolver problemas y aprender ya que la construcción les
motivará\cite{const:vs}.

El enfoque construccionista establece que los seres humanos conocen y aprenden
de formas diferentes por lo tanto, no se puede elaborar una jerarquía de estilos
de aprendizajes\cite{valdivia:sg}.

\subsection{Estado del Arte}
\observacion{Construccionismo en la presente}

El construccionismo pone énfasis en el \emph{Aprender haciendo}, esta idea
\fixme{mejora}{ref} la práctica educativa tradicional o instruccionismo. El instruccionismo
se basa en el concepto de que existe un profesor y un estudiante, el profesor
transfiere el conocimiento que ha adquirido a un alumno que es receptor pasivo
de información de esta manera, se enfoca más en la capacidad del profesor. 

Existen varios emprendimientos o \emph{\fixme{amigos del contruccionismo}{?}},
para la mayoría de ellos las computadoras son esenciales mientras que para otros
el mayor esfuerzo está en la incorporación de la tecnología en su práctica
educativa\cite{papertian:const}.

Algunos de estos emprendimientos son:

\observacion{Cambiar el ``description'' por ``itemize''}
\begin{description}

\item[Lenguaje de programación LOGO]  A mediados de la década de 1960 Seymour
	Papert, que había estado trabajando con Piaget en Ginebra, llegó a
	Estados Unidos donde co-fundó el Laboratorio de Inteligencia Artifical
	del MIT con Marvin Minsky. Papert trabajó con el equipo de Bolt, Beranek
	y Newman, liderado por Wallace Feurzeig, que creó la primera versión del
	logotipo en 1967. A lo largo de la década de 1970 Logo fuen incubado en
	el MIT y algunos otros sitios de investigación. El lenguaje de
	programación Logo, un dialecto de Lisp, fue diseñado como una
	herramienta para el aprendizaje. Sus características como la
	modularidad, extensibilidad, interactividad y flexibilidad se derivan de
	este objetivo. 
	%http://el.media.mit.edu/logo-foundation/logo/index.html

	El lenguaje Logo es la cuna del construccionismo, se basa en el
	principio de que se aprende mejor haciendo, pero se aprende todavía
	mejor si se combina la acción con la verbalización  y la reflexión
	acerca de lo que se ha hecho. Fundamentalmente consiste en presentar a
	los niños retos intelectuales que puedan ser resueltos mediante el
	desarrollo de programas en Logo. El proceso de revisión manual de los
	errores contribuye a que el niño desarrolle habilidades metacognitivas
	al poner en práctica procesos de auto-corrección\cite{logo:sg}.
	%http://es.wikipedia.org/wiki/Logo_(lenguaje_de_programaci%C3%B3n)

    \observacion{Ver que decir  qu eno sea repetitivo con todo lo que ya se
        dijo.}


%\item[Simulación] La simulación en el ámbito de la educación fue evolucionando
%desde simples motores de reglas hasta complejos entornos, la simulación
%demostró ser una herramienta muy útil el ámbito laboral
%\cite{mariluz:seiousgames}, pues enseña al alumno a encarar situaciones muy
%difíciles de representar en entornos completamente controlados y provee
%mecanismos para comprobar la efectividad de la herramienta. 

%Actualmente la simulación se utiliza más en el ámbito empresarial pues las
%empresas son las más necesitas de innovar en el ámbito de la enseñanza. Un
%ejemplo de esta necesidad se da, por ejemplo, en el entrenamiento de nuevos
%vendedores, es muy difícil enseñar a un vendedor como debe vender los productos
%con un pizzarón y/o una presentación, en cambio la simulación permite que el
%mismo pueda probar cosas nuevas y experiencias de sus compañeros (o
%instructor), convirtiendo así el aprendizaje en
%colectivo\cite{mariluz:seiousgames}. En el ámbito académico la simulación mas
%utilizada en campos físicos (como simulación de fluidos), meteorología
%(simulación de tormentas y fenómenos climáticos), etc. 

%\item[Serious Games] Diseñado con el propósito de aprender. Generalmente hace
%uso de la simulación para permitir un aprendizaje más realista.

%\item[Lego Serious Play] Es una iniciativa de Lego que busca fomentar el
%pensamiento creativo por medio de la construcción por parte de los estudiantes
%de su identidad y experiencias utilizando legos. 

\item[\Gls{olpc}]. El esfuerzo se centra en dotar a los niños de una computadora
	duradera, accesible y potente en los países en desarrollo, se dice que
	es un descendiente directo del construccionismo. Con esto se busca que
	la computadora personal sea utilizada como un laboratorio intelectual y
	un vehículo para la auto-expresión. OLPC no tiene que ver con la
	escolarización o la escuela, más bien las utiliza como medio de
	distribución de las computadoras a los niños, los cuales pueden
	utilizarlas para aprender en cualquier lugar y momento. Se busca
	fomentar el aprendizaje natural, es decir, aquel aprendizaje sin
	enseñanza.

    \fixme{Los problemas atribuidos al experimento}{experimento?} OLPC son
    predominantemente las críticas a la política, el liderazgo o de la
    intransigencia de la escuela en vez del construccionismo o computadora
    personal para los niños pobres. El experimento audaz de Nicholas Negroponte
    (co-fundador de \Gls{olpc}) y Sugata Mitra para dejar las computadoras desde
    un helicóptero sobre una aldea de África se basa en la creencia en el
    construccionismo\cite{papertian:const}.

    \observacion{Poner referencias}
    \observacion{Terrible descripción del proyecto OLPG}
    \observacion{No se de que portal sacaron esta descripción}

\item[Fabricación personal] Neil Gershenfeld, colega de Papert en el Media Lab
	del \Gls{mit} dictó un curso titulado \emph{Cómo hacer casi cualquier
		cosa}. La idea se centraba en la creación de  la tecnología que
	se necesita para resolver los problemas que se poseen. Esta
	auto-confianza, la autonomía personal y la agencia sobre la tecnología
	han estado en el centro de trabajo de Papert durante años. Papert no
	sólo defendió la idea de que los niños posean computadoras personales,
	sino también que a la larga ellos debían mantenerlas, repararlas e
	incluso construirlas.

	Junto con la capacidad para utilizar la tecnología para inventar
	soluciones a los problemas de significado personal, los estudiantes no
	sólo tienen acceso a la información, sino que tienen una mayor capacidad
	para darle forma a su mundo. La fabricación personal promueve la visión
	de Papert \emph{Si se puede utilizar la tecnología para hacer las cosas,
		usted puede hacer las cosas muchos más interesantes y usted
		puede aprender mucho más haciéndolo}\cite{papertian:const}.

\end{description}

%http://constructingmodernknowledge.com/cmk08/wp-content/uploads/2012/10/StagerConstructionism2012.pdf


%\section{Simulación}
\label{sec:tics_SIMULACION}

\obervacion{No se entiende como se llego a esto}

\observacion{Ver como organizar el contenido para que no sea una bolsa de
    conceptos}.

La simulación se define como el proceso de diseñar un modelo de un sistema real
y, llevar a cabo experimentos con este modelo, con el fin o bien de entender el
comportamiento del sistema o de la evaluación de distintas estrategias para la
operación del sistema\cite{ingalls2008introduction}. 
%[ingalls2008introduction]

Un juego y una simulación podrían llegar a ser muy parecidos, a veces los juegos
tienen motores de simulación\footnote{Un motor de simulación es un conjunto de
objetos y métodos que se utilizan para la construcción de modelos de
simulación que están dentro de las aplicaciones}, una de las diferencias
es que la simulación es muy dependiente del contexto. 

La simulación en el ámbito de la educación fue evolucionando desde simples
motores de reglas hasta complejos entornos, la simulación demostró ser una
herramienta muy útil en el ámbito laboral\cite{mariluz:seiousgames}, pues enseña
al alumno a encarar situaciones muy difíciles de representar en entornos
completamente controlados y provee mecanismos para comprobar la efectividad de
la herramienta. 

Actualmente la simulación se utiliza más en el ámbito empresarial pues las
empresas son las más necesitadas de innovar en el ámbito de la enseñanza. Un
ejemplo de esta necesidad se da, por ejemplo, en el entrenamiento de nuevos
vendedores, es muy difícil enseñar a un vendedor como debe vender los productos
con un pizarrón y/o una presentación, en cambio la simulación permite que el
mismo pueda probar cosas nuevas y experiencias de sus compañeros (o instructor),
convirtiendo así el aprendizaje en colectivo\cite{mariluz:seiousgames}. En el
ámbito académico la simulación es más utilizada en campos físicos (como simulación
de fluidos), meteorología (simulación de tormentas y fenómenos climáticos), etc. 

Existen dos tipos de simulaciones, en primer lugar están las experimentales que
ponen al estudiante en el lugar de un profesional y requieren que el mismo tome
decisiones para alcanzar los objetivos y en segundo lugar están las simbólicas
que buscan que el estudiante deduzca eventos, principios y mejores prácticas
\cite{charsky:2010}. 

\fixme{Una simulación}{sección?} esta conformada por:

\begin{description}

\item[Entidades] Cualquier objeto o componente en el sistema que requiera
	representación explícita en el modelo se define como
	entidad\cite{banks2000dm}. Las entidades poseen atributos. Los atributos
	son las características de una determinada entidad que son exclusivos de
	esa entidad. Por último, son aquellas que cambian el estado de una
	simulación. Ejemplo de entidades son: un médico o una jeringa en una
	simulación médica.

\item[Acciones] Las entidades interactúan entre sí a través de acciones. Estas
	acciones puede causar cambios en el estado de la simulación además de
	eventos. Ejemplo de una acción en una simulación médica es la
	esterilización de un instrumento.

\item[Eventos] Los eventos son hechos que ocurren de manera controlada pero no
	siempre predecible en el entorno simulado, los mismos afectan a las
	entidades y deben obligar a realizar alguna de las acciones disponibles
	para tal evento. Ejemplo de un evento en un simulación médica es un paro
	cardíaco del paciente.

\end{description}

La confianza en el modelo o la simulación según\cite{DoDSysEng2001} se establece
mediante:

\begin{description}

\item[La verificación] Es el proceso de determinar si la implementación
	representa con precisión las especificaciones del diseño. 

\item[La validación] Es el proceso de determinar el grado en el que el modelo
	representa de forma exacta la realidad de acuerdo al uso que se tiene
	previsto darle y el nivel de confianza que debe tenerse en la
	evaluación.

\item[La acreditación] Es el proceso de certificación de un modelo para su uso
	con un propósito específico.
%[DoDSysEng2001]

\end{description}



%En la actualidad, la utilización de la simulación como herramienta para el
%entrenamiento es cada vez mayor por partes de los profesores, quienes están
%cada vez más familiarizados con la tecnología. 

%Según\cite{humphreys2013developing} los tipos de estudiantes definidos por Kolb
%son:

%\begin{description}

%\item[Accommodating learners] Aprenden de la experiencia e interiorizan el
%aprendizaje a través de experimentación activa. 

%\item[Diverging learners] Aprenden a través de experimentación activa, e
%interiorizan el conocimiento reflexionando sobre la experiencia. 

%\item[Coverning learners] Aprenden a través del pensamiento abstracto e
%interiorizan el conocimiento a través de la experimentación activa.

%\item[Assimilating learners] Aprenden a través del pensamiento abstracto y las
%interiorizan reflexionando sobre las mismas. 
	
%\end{description}

%Teniendo en cuenta el caso de la enfermería, la misma es una ciencia que atrae
%a alumnos del tipo \emph{Diverging learners}, y la simulación es una
%herramienta ideal para este tipo de estudiantes.

La mayoría de la literatura encontrada acerca de la simulación y los cuidados de
la salud no proporcionan muchos detalles acerca de la implementación de modelos
de simulación en áreas amplias, se cree que esto se debe a la complejidad de
representar las actividades relacionas al cuidado de la salud dentro de un
modelo de simulación que debe, de hecho, ser una simplificación de las mismas.
Esta simplificación puede ser un proceso sumamente complejo, por lo cual la
mayoría se centra en una parte de las actividades hospitalarias pero no así en
todas. Cuanto mayor sea el detalle, la simulación \fixme{conducirá}{} a una representación
más realista lo cual aumenta la confianza en los grupos de interés, sin embargo,
más detalle requiere más datos validados y esto puede ser costoso de
obtener\cite{guna:simulation}.

Algunas aplicaciones específicas en el cuidado de la salud son:
\observacion{Incluir un párrafo porque el énfasis en simulación y salud}

\begin{description}

\item[Departamento de emergencia y accidentes] La mayoría de los trabajos
	realizados en esta área se refieren a la optimización de tiempo de
	espera de los pacientes y la organización del personal, de las
	habitaciones,de las ambulancias, para dar mejor atención a los
    pacientes. Un ejemplo de esto es \emph{Edsim} que se utiliza para aumentar
    el rendimiento en un departamento de emergencias en los Estados Unidos como
    parte de un sistema que permite el desvío de ambulancias en los períodos
    pico de demanda, el cual incluye la introducción de salones de descarga y la
    disminución del tiempo de estancia, sin pasar por el
    triaje\cite{guna:simulation}. 
	
\item[Instalaciones para pacientes hospitalizados] Los trabajos se centran en la
    mejora en la atención con respecto al flujo de pacientes así como la
    ocupación de camas. Muchos trabajos tratan de demostrar como se podrían
    utilizar modelos matemáticos para esto. \emph{Harper y Shahani} presentaron
    un modelo de simulación flexible relacionado a estás cuestiones de pacientes
    hospitalizados, el mismo utiliza \emph{TOCHSIM}, \fixme{flexible en el sentido de
        que aborda también}{pulir} problemas como la creación de una nueva
    unidad en el hospital\cite{guna:simulation}.

\item[Clínicas para pacientes ambulatorios] \fixme{En este sentido}{?} la
    simulación se utiliza para minimizar el tiempo de espera de los pacientes en
    clínicas externas, \fixme{es decir}{pulir}, aquellas en las que se sacan citas. El tiempo
    de espera no sólo implica la espera dentro de la clínica sino también el
    tiempo que pasa entre el momento en el que se solicita un cita y el día de
    la cita. Un ejemplo de esto es \emph{CLINSIM} que se utilizó en el Reino
    Unido para observar como la política de operación puede influir en los
    tiempos de espera de los pacientes\cite{guna:simulation}. 

\item[Formación médica y quirúrgica] Se centran en tareas específicas y en la
	formación de un conjunto limitado de habilidades referentes a estas
	tareas. Los ejemplos más recientes son entrenamiento para un intubación
	esofágica, capacitación y evaluación de capacidades laparoscópicas,
	entrenamiento para la palpación de tumores de mama\cite{mantovani:vr}. 

\item[Sistemas de formación de emergencias] Se refieren a aquellas simulaciones
	diseñadas para la rápida respuesta médica. Incluye desde pacientes
	virtuales dinámicos cuya acción por parte del estudiante produce un
	cambio clínico en el mismo y una respuesta al estudiante.  Otro ejemplo
	es el utilizado en la marina de EE.UU que intenta formar a los
	profesionales para su rápida acción frente a desastres civiles y donde
	la estabilización de pacientes se tenga que dar con recursos
	limitados~\cite{mantovani:vr}. 

\item[Entrenamiento para profesionales de salud mental] Janssen LP creó una
	simulación para educar a los psiquiatras y profesionales de la salud en
	lo que es tener esquizofrenia llamada \emph{el viaje en autobús} que
	trata de mostrar lo que pasa dentro de de la mente de una persona con
	esquizofrenia cuando viaja en autobús en base a experiencias relatadas
	por pacientes y médicos\cite{mantovani:vr}. 

\end{description}

\observacion{Buscar manera de poner todas estas secciones en un mismo plano,
    como para que tenga sentido el ir explicándolas (falta un conector), por
    ejemplo, donde se sitúan todos estos trabajos en la linea de tiempo que
    presentaron inicialmente? Esta demasiado desconectado e aislado.
    
    Se necesita complementar mas esa primera sección de manera tal a que se
    entienda de donde salio todo esto.}

%\section{Gamification}
\label{sec:tics_GAMIFICATION}
\todox{Agregar mas contenido}

Es el uso de mecánicas tradicionalmente usadas en los videojuegos en contextos distintos
a los juegos. Según Shell \cite{hj:gamification} un juego es una actividad cuyo fin es resolver un problema de manera entretenida. 

La \emph{Gamification}, mejora la actividad del usuario, el \emph{engagement} (enganchamiento o compromiso con el juego), el aprendizaje, la puntualidad (capacidad de completar una tarea o asignación antes del tiempo designado), el retorno a la inversión, la calidad y la colaboración.

\subsection{Principios}

Los principios de la gamification moderna según \cite{hj:gamification} son los 
siguientes:

\begin{itemize}
\item Objetivos bien definidos.
\item Mejor registro de resultados y tablas de puntuación.
\item Retroalimentación frecuente.
\item Libertad de elección de método para realizar la tarea.
\item Enseñanza y retroalimentación constante.
\end{itemize}

Cuando se mide el desempeño, el rendimiento mejora, cuando el rendimiento se mide y además se informa sobre esto, la tasa de mejora acelera. Cuando la retroalimentación se
presenta en forma de tablas y gráficos el impacto es aun mayor.

\subsection{Propiedades gamification}

Esencialmente, gamification intenta aplicar la mecánica de los juegos en otros entornos, como el ambiente educativo. Este concepto no está directamente relacionado con el diseño del juego, sino que trata de involucrar al usuario a través de pequeñas dosis de desafíos y recompensas con el fin de conseguir que el usuario realice ciertas acciones en diferentes ambientes\cite{breaking:gamification}.

Gamification trabaja para satisfacer algunos de los deseos humanos más fundamentales: el reconocimiento y la recompensa, de estado, de logros, competencia y colaboración, la auto-expresión, y el altruismo.\cite{breaking:gamification}.

La mecánica del juego pueden ser de diferentes tipos\cite{breaking:gamification}, tales como:

\begin{itemize}
	\item Comportamiento (centrado en el comportamiento humano y la psiquis humana),
	\item Retroalimentación (en relación con el ciclo de retroalimentación en la mecánica de juego, y
	\item La progresión (utilizada para estructurar y extender la acumulación de habilidades significativas).
\end{itemize}


Existen otros mecanismos de juego que se pueden utilizar para los materiales gamification y actividades educativas\cite{breaking:gamification}, tales como:

\begin{itemize}
	\item El tiempo (los jugadores tienen un tiempo limitado para realizar una tarea).
	\item La exploración (los jugadores tienen que explorar y descubrir cosas que les sorprenderán).
	\item Los desafíos entre los usuarios (los jugadores pueden darse desafíos unos a otros y competir para el logro de los objetivos, los objetos, medallas, etc.).
\end{itemize}

Para que sea eficaz a largo plazo, gamification debe ser algo más que la adición de este tipo de elementos para un contexto no-juego, también debe actuar sobre la motivación intrínseca de los jugadores\cite{framework:gamification}. 

Con el fin de tener una motivación intrínseca para realizar una tarea, la persona debe mantenerse en un estado entre la ansiedad (si el desafío supera las capacidades de la persona) y el aburrimiento (si la persona siente que la tarea es demasiado fácil ). Este es un estado conocido como flujo. Objetivos claros, un sentido de control, retroalimentación inmediata y, sobre todo, un equilibrio entre habilidad y reto son algunos de los factores que contribuyen a fluir\cite{framework:gamification}.

La relación, el deseo de interactuar y conectarse con otras personas, es una de las necesidades humanas innatas que conducen a la motivación intrínseca\cite{framework:gamification}.

Por lo tanto, los sistemas con gamification no sólo deben abordar la motivación extrínseca de los jugadores, sino también considerar la forma de conducir a los jugadores la motivación intrínseca. Debería centrarse en cómo crear experiencias significativas, proporcionar un sentido de relación entre los jugadores, mejorar su reconocimiento social, y dar la autonomía y el propósito de sus acciones. También debe mantener a los jugadores en un estado de flujo y proporcionar una experiencia divertida conjunto\cite{framework:gamification}. 


\subsection{Elementos del juego}

Los elementos del juego son el conjunto de componentes y características de los juegos de vídeo que se pueden utilizar en contextos no-juego\cite{framework:gamification}.

El flujo y diversión deben ser considerados en un diseño como sección transversal del sistema, transversal a los otros componentes\cite{framework:gamification}.

A continuación mapeamos como se implementarían estos conceptos con los elementos del juego, según\cite{framework:gamification}:

\begin{itemize}
	\item Retroalimentación y recompensas: puntos, barras de progreso, insignias, trofeos, tabla de calificación.
	\item Amigos: compartir, invitar a amigos, dar/comercializar/vender bienes virtuales, tablas de clasificación (gráfico social).
	\item Jugabilidad: niveles, objetivos intermedios, objetivos claros, fracaso divertido, reglas, economía virtual, calendarios de recompensas.
\end{itemize}

Los componentes transversales de flujo y la diversión se logran a través de la forma en que las actividades se establecen en el sistema. El dominio y el progreso son los que hacen que las experiencias sean divertidas. La sensación de dominio y el progreso se puede implementar a través de los elementos de la jugabilidad, los amigos y conceptos de retroalimentación y recompensas. Lo mismo ocurre con el flujo. El jugador puede mantenerse en un canal de flujo cuando él o ella está óptimamente desafiado proporcionando tareas que no son ni demasiado fácil ni demasiado difícil. Esto podría lograrse proporcionando retroalimentación inmediata, objetivos intermedios y diferentes niveles de progresión. De esta manera, el reto es equilibrado con las habilidades del jugador\cite{framework:gamification}.

%\section{Serious Game}
\label{sec:tics_JUEGO_SERIO}

Un \emph{Serious Game} es un vídeo juego elaborado con el propósito primario que
no es el de entretener\cite{sg:aoverview}, sino tienen una finalidad educativa
explícita y cuidadosamente pensada, utiliza la tecnología y los conceptos de la
industria de los vídeo juegos para encontrar solución a problemas reales. Es
decir, se utilizan para definir los juegos que poseen una pedagogía incluida,
algún tipo de evaluación ya sea interna o externa y lo que hay que aprender
(contenido) integrado\cite{damien:sg}.

Los \emph{Serious Game} proveen una oportunidad muy importante para ayudar en la
enseñanza y desarrollo de profesionales, por que ayudan a crear el tipo de
educación que los adultos prefieren, proveen mecanismos para que los estudiantes
cometan errores y experimenten con sus ideas, con su conocimiento y con la
teoría en un ambiente protegido sin riesgos para la vida o la identidad. 

Los beneficios que brindan los \emph{Serious Game} se acentúan en la medida en
la que los mismos proveen entornos más completos en donde realmente se puedan
poner en práctica la teoría, esto ayuda a una comprensión más profunda del área
de interés.

La principal diferencia entre los \emph{Serious Game} y otras aplicaciones de
\emph{E-Learing} es su enfoque en la creación de una experiencia de aprendizaje
significativo, relevante y atractivo. En un \emph{Serious Game} existen metas
claras de aprendizaje pero las mismas se encuentran en un contexto significativo
en donde se deben aplicar los conocimientos y hacer uso de herramientas que
están a disposición para obtener éxito en la resolución de los problemas
presentados. Estos problemas se equilibran a través de la retroalimentación y
otras estrategias para mantener el interés del estudiante
\cite{papertian:const}.
%. Todo esto hace que en los \emph{Serious Game} el principal objetivo sea ganar
%el juego no aprender, sin embargo sólo se puede hacer esto dominando el
%aprendizaje

\fixme{El campo de los \emph{Serious Game} rechaza la idea de que los profesionales de
    la educación pueden ser reemplazados fácilmente}{Obs: que es cada sección?,
    un enfoque? Una técnica? Un buzzword?}, para ellos la labor de estos
profesionales es imprescindible para la reflexión y orientación del aprendizaje.
Es cierto que se puede llegar a aprender sin el apoyo de un profesional de la
educación pero se corre el riesgo de perder el enfoque y la eficacia
\cite{elearning:seiousgames}. 

El \emph{serious Game} no se trata de una modelo de aprendizaje pasajero. Varios
autores como \emph{Johan Huizinga}, \emph{Jean Piaget}, \emph{Wittgenstin} y
\emph{Seymour Papert} han reconocido su importancia  como objeto de aprendizaje.
Los juegos deben ser elaborados teniendo en cuenta el nivel cognitivo del
estudiante, es decir, su etapa de aprendizaje y en que el aprendizaje difiere de
acuerdo a la etapa de vida en la que se encuentre un estudiante. Mediante la
práctica repetida de actividades relacionadas al área de interés se desarrollan
habilidades y destrezas\cite{education:games}. 

\observacion{Se podría hacer una comparación? (entre todos)}

Los siguientes son ejemplos de algunas áreas que utilizan Serious Game:

\begin{description}

\item[Militar] Los primeros juegos a menudo se basaban en lucha o combate.
	Durante más de 30 años los juegos han sido reconocidos como herramientas
	factibles en el entrenamiento de militares. En 1996 fue lanzado un juego
	llamado \emph{Marine Doom} en donde la tarea de los jugadores era el
	aprendizaje de formas de ataque, conservación de municiones, comunicarse
	con eficacia, dar órdenes al equipo de trabajo entre otros. De esta
	manera tuvo lugar una forma de entrenamiento más atractivo, sin el
	costo, dificultad, riesgos e inconvenientes que implicaría el mismo
	entrenamiento en un entorno real. Además se podían crear situaciones que
	en el mundo real serían muy difíciles de replicar y donde los errores
	pueden ser catastróficos además, permite la repetición hasta alcanzar la
	maestría\cite{education:games}.


\item[Salud] Este tipo de juegos son cada vez mayores, los juegos de salud se
	utilizan para la formación de profesionales basada en la simulación. En
	2008 el Centro de Simulación Hollier en Birmingham, Reino Unido, realizó
	una prueba que permitió a médicos jóvenes experimentar y entrenar para
	diversos escenarios médicos a través de maniquíes virtuales como
	pacientes, de este modo el aprendizaje se da por la experiencia. En su
	disertación, Roger D. Smith, realizó una comparación entre la enseñanza
	tradicional y la formación mediante realidad virtual y el uso de
	herramientas basadas en la tecnología de juegos en cuanto a la cirugía
	laparoscópica. Como conclusión afirmó que lo último era más barato,
	requería menos tiempo y que permitió menos errores médicos cuando los
	médicos se presentaban en una cirugía real debido a, entre otras cosas,
	la posibilidad de repetición de la experiencia sin riesgo
	alguno\cite{education:games}.



\item[Juegos corporativos] Este tipo de juegos se han utilizado para la
	selección de personal, la mejora de comunicación entre los directivos y
	su personal de confianza, y la formación de nuevos empleados. Un ejemplo
	de estos juegos es el INNOV8 de IBM que ayuda en el entrenamiento de los
	estudiantes acerca de la gestión de procesos de negocios. Los Serious
	Game pueden ser utilizados incluso para elaborar planes de
	negocios\cite{education:games}. 

\end{description}


%\section{Desarrollo de Serious Game}


Pereira\cite{pereira2009design} en el diseño del juego \emph{Living Forest} utiliza los pasos definidos a continuación como modelo de creación de un juego serio a partir de la definición previa de las competencias básicas que se desean enseñar.

Primero se definen las competencias básicas y luego se diseña y desarrolla el juego. A
continuación la figura \ref{fig:tics_flujo_diseño_prop} muestra el proceso de desarrollo y luego se explica cada ítem.

\begin{figure}[ht!]
\centering
\begin{tikzpicture}[auto]
    % Place nodes
    \node [block] (1) {Objetivos de diseño};
    \node [block, right of=1, node distance=5cm] (2) {Competencias básicas relacionadas con la educación};
    \node [block, right of=2, node distance=5cm] (3) {Investigación del dominio};
    \node [block, below of=3, node distance=3cm] (4) {Diseño del juego};
    \node [block, left of=4, node distance=5cm] (5) {Tiempo en el juego};
    \node [block, left of=5, node distance=5cm] (6) {Acciones de jugabilidad};
    \node [block, below of=6, node distance=3cm] (7) {Indicadores};
    \node [block, right of=7, node distance=5cm] (8) {Representación e interacción};
    \node [block, right of=8, node distance=5cm] (9) {Implementación};
    \node [block, below of=9, node distance=3cm] (10) {Evaluación};
    % Draw edges
    \path [line] (1) -- (2);
    \path [line] (2) -- (3);
    \path [line] (3) -- (4);
    \path [line] (4) -- (5);
    \path [line] (5) -- (6);
    \path [line] (6) -- (7);
    \path [line] (7) -- (8);
    \path [line] (8) -- (9);
    \path [line] (9) -- (10);
\end{tikzpicture}

\caption{Flujo de diseño propuesto de un Serious Game}
\label{fig:tics_flujo_diseño_prop}
\end{figure}


\subsection{Partes del flujo de diseño}
\subsubsection{Objetivos de diseño}
Definen cuál es el propósito del juego

\subsubsection{Competencias básicas relacionadas con la educación} 

Se identifican aquellas que influyen en el diseño del juego

\subsubsection{Investigación del dominio}
\begin{itemize}
	\item Recabar información importante para el diseño.
	\item Participación de un experto en el dominio.
	\item Estadísticas sobre características.
	\item Una pregunta que surge es: que grado de detalle debe ser modelado?.
	\item Que incluir y que considerar.
	\item Identificar conjuntos de acciones hipotéticas a modelar (jugadores).
	\item Comprensión del dominio (funciones).
	\item Para cada función, analizar los elementos y servicios relacionados que se podrían modelar.
\end{itemize}

\subsubsection{Diseño del juego}
\begin{itemize}
	\item A partir de la idea original y basado en la información recogida. 
	\item Determinar el papel desempeñado por el jugador (de acuerdo a la semántica y pragmática de las acciones y decisiones que está llamado a hacer). 
	\item Aproximación a la realidad y exploración con tiempo limitado. 
\end{itemize}

\subsubsection{Tiempo de juego}
\begin{itemize}
	\item Permitir al jugador experimentar las consecuencias de la toma de decisiones a corto plazo.
	\item Tiempo para la primera experiencia (Adaptación).
\end{itemize}

\subsubsection{Indicadores}
Indicadores del progreso de un jugador, como por ejemplo: puntaje, cumplimiento de objetivos, etc.

\subsubsection{Representación e interacción}
\begin{itemize}
	\item Implementación de las representaciones de la escena del juego.
	\item Elementos que forman partes de la escena.
	\item Representar visualmente el concepto que se modela en la lógica del juego.
	\item Animaciones.
	\item Sonido.
	\item Punto de vista del jugador en la interfaz del juego.
	\item Movimientos de cámaras y zoom.
	\item Mantener coherencia en las representaciones.
	\item Diseño de componente de la interfaz. Por ejemplo panel de indicadores, barra de herramientas.
\end{itemize}

\subsubsection{Implementación} 
\begin{itemize}
	\item Distintas tareas del proceso de desarrollo. 
	\item Plataforma. La producción del juego involucra el estado del arte, el desarrollo de componentes, el estado del arte, la programación y el testeo. 
	\item Iteraciones (implementación y evaluación). 
	\item Realizar optimizaciones. 
	\item Poner énfasis en la estética, retro-alimentación y el estado. 
\end{itemize}

\subsubsection{Evaluación} 
\begin{itemize}
	\item Realizar varias sesiones de evaluación durante el desarrollo. Por ejemplo, con los responsables o expertos, miembros de la audiencia objetivo. 
	\item La evaluación con los grupos de interés se centra en la adaptación del juego (usabilidad). 
	\item La evaluación con los expertos se centra en la validación del modelo de la simulación (refinamiento). 
	\item Evaluación con la audiencia objetivo para probar el juego en un escenario(parecido al final) y evaluar los aspectos relacionados con el proceso de aprendizaje.
\end{itemize}


%\section{Actualidad}
\label{sec:tics_ACTUALIDAD}

\todox{Agregar mas contenido}

La relación de los videojuegos con la formación surge en los años 90 y ha llegado hasta
la actualidad en plena efervescencia, siendo aplicados en casi todos los ámbitos de la 
educación tanto formal como no formal. Los juegos serios para el entrenamiento de habilidades se pueden considerar una evolución de las técnicas de entrenamiento basadas
en la realidad virtual que se desarrollaron en los años 90 y que en la actualidad se han
transformado, por su potencial motivacional, de simulaciones puras a videojuegos\cite{videojuegos:gonzaleztardon}.

\subsection{Casos de éxito}


\subsubsection{Caso 1. Triage Trainer}
	

\emph{Tipo: } Simulación de entrenamiento.

\emph{Destinatarios: } Médicos, enfermeros, paramédicos y otros rescatistas.

\emph{Contenido} Entrenamiento para evaluar a los pacientes en un lugar de
emergencia.

\emph{Desarrollado por: } TruSim.

\begin{figure}[h!] 
	\centering 
	\includegraphics[scale=0.5]{tics/images/triage.png}
	\caption{Ambientación de Triage}
	\label{fig:triage}
\end{figure}

\emph{Visión General: } Triage Trainer se desarrolla en una escena de explosión
en una calle (ver~\ref{fig:triage}) la cual es un incidente mayor, y está
diseñado para formar profesionales que puedan participar en una escena de un
incidente de este tipo (médicos, enfermeros, paramédicos, rescatistas). Los
jugadores deben realizar un triaje, es decir, evaluar el grado de las lesiones
de víctima generadas aleatoriamente utilizando los protocolos y controles
médicos adecuados, además de priorizar a las víctimas para el tratamiento. La
apariencia física de cada víctima es imitada con precisión como los signos
vitales, los síntomas y sobre todo los patrones de tiempo para el deterioro de
las lesiones, es decir, la condición de una víctima cambia de forma realista con
el tiempo (ver~\ref{fig:triage_patient1}.

\begin{figure}[h!]
	\centering 
	\includegraphics[scale=0.5]{tics/images/patient_side.jpg}
	\caption{Evolución de un paciente en Triage}
	\label{fig:triage_patient1}
\end{figure}

Los jugadores reciben retroalimentación acerca de su rendimiento, incluyendo la
precisión de sus chequeos, si los pacientes fueron priorizados en el orden
correcto y el tiempo que les llevó completar el triaje, en comparación con la de
un experto.

\emph{Evaluación: } La retroalimentación de los participantes que utilizaron
Triage Trainer sugiere que el mismo cumplió exitosamente sus fines. Los
jugadores asociaron su experiencia de juego con su experiencia en el mundo real
y muchos de ellos sentían que realmente estaban allí. Se espera que los
jugadores puedan tomar decisiones bajo presión, lo que ayudará a su desarrollo
cognitivo. También se observó que los jugadores tienden a discutir sus
experiencias con sus compañeros de curso, lo que también podría tener un impacto
en su aprendizaje.

Un elemento que no fue evaluado por TruSim debido a que no era logísticamente
posible fue el impacto de las pruebas en la retención del conocimiento y el
cambio de comportamiento de los jugadores~\cite{education:games}. 


\subsubsection{Caso 2 SimVenture}

\emph{Tipo: } Juego de simulación de negocios.

\emph{Destinatarios: } Personas de 14 a 30 años.

\emph{Contenido: } Las realidades de la creación y funcionamiento de un negocio.

\emph{Desarrollado por: } Venture Simulations.

\emph{Visión General: } En el inicio del
juego(ver~\ref{fig:simventure_tutorial}), a los jugadores se les brinda
informaciones y antecedentes para que que se ubiquen en escena. Ellos deben
empezar a dirigir su propio negocio en su casa de fabricación y venta de
computadoras, mientras deben mantener un trabajo de tiempo completo
independiente. El juego lleva a los jugadores a la ejecución de un negocio en su
propia casa a la extensión del mismo a más locales, lo que requiere contración
de personal. Los jugadores son capaces de avanzar en el juego a través del
aprendizaje de los elementos importantes del empresariado organizadas en cuatro
categorías: organización, ventas/marketing, finanzas y operaciones. Los
jugadores toman decisiones acerca de las actividades dentro de estas áreas y
observan los resultados de sus acciones. 

\begin{figure}[h!]
	\centering 
	\includegraphics[scale=0.5]{tics/images/simventure-tutorial.jpg}
	\caption{Tutorial de SimVenture}
	\label{fig:simventure_tutorial}
\end{figure}

Los jugadores obtienen retroalimentación sobre un número de diferentes
parámetros. En un nivel básico, se puede simplemente revisar la cantidad de
ingresos que están generando. Además de esto, el éxito puede ser medido por la
cantidad de pedidos que han recibido para sus productos. También se proporciona
retroalimentación visual para representar la eficiencia de la organización y su
felicidad como individuo.

\emph{Evaluación: } Phil Warren, director de estudios de negocios en Snaith
School, ha utilizado SimVenture como complemento al plan de estudios. Según el
mismo, el plan de estudios por lo general sólo requiere que los estudiantes
aprender sobre los diversos elementos diferentes del negocio de forma aislada y
sin embargo, cualquier decisión que se tome en una de las partes de un negocio
tiene efecto en las demás. SimVenture se vió como una oportunidad de aplicar los
conocimientos aprendidos en clase en una actividad práctica, además se observó
que permitir que los estudiantes jueguen en pares da un espacio para la
discusión en torno a las decisiones y aprenden de sus errores
juntos~\cite{education:games}.


%! TEX root = ../main.tex
\chapter{Definición del Problema}
\label{chap:problema}

%\observacion{\textbf{Mirta}: VNC (que grabe y haga streaming) para la presentación}
% Checkear estos:
%       https://play.google.com/store/apps/details?id=com.vlcforandroid.vlcdirectprofree
%       https://play.google.com/store/apps/details?id=com.mobzapp.screenstream&hl=es_419

Definidos el potencial y las áreas de aplicación de los juegos serios en la
educación, en este capítulo se describe el problema abordado, así como una
descripción de los principales criterios y factores que debe tener un juego
serio.

Luego se selecciona un contexto de aplicación, se describen sus principales
características, y se propone una solución a los problemas encontrados.


\section{Descripción general}

Según~\cite{education:games} un juego serio debe cumplir los siguientes tres
criterios:

\begin{itemize}

\item \textbf{Implementación técnica}: se refiere a la actividad de programación
    y ejecución de un patrón de diseño. Incluye la perfecta integración de los
    elementos de diseño en el videojuego. 

\item \textbf{Adecuación para la educación:} la capacidad del videojuego para hacer
    frente a las metas curriculares o educativas y la habilidad o el
    conocimiento del jugador relativo a los contenidos educativos que se aborde.

\item \textbf{Integración total con los objetivos pedagógicos:} la integración
    del patrón de diseño y el videojuego en general con los objetivos
    educativos.

\end{itemize}

Adicionalmente, el diseño del mismo se tiene que centrar en cuatro factores o
dimensiones, las cuales son\cite{education:games}:

\begin{itemize}
\item \textbf{Contexto:} es decir, donde ocurre el aprendizaje, lo que va desde
    aspectos macro, como  factores políticos, económicos e históricos, hasta
    aspectos micro como la experiencia y  antecedentes de los profesores, costos
    de licencia, entre otros.
\item \textbf{Tipo de aprendizaje:} para el individuo o grupo, requiere que se
    considere su  estilo de aprendizaje y sus conocimientos previos, y qué
    métodos se ajustan mejor a sus  necesidades.
\item \textbf{Modo de representación:} lo que incluye el nivel de interactividad
    requerido, la fidelidad y  el nivel de inmersión producido. Además cubre la
    narración de los hechos, la separación de los  aspectos de inmersión con la
    reflexión de haber utilizado el videojuego. Y de manera importante  enfatiza
    el potencial de retroalimentación que refuerza el aprendizaje.
\item \textbf{Principios pedagógicos:} es necesario reflexionar sobre los
    modelos de aprendizaje lo  que permite producir apropiados planes de
    lecciones.
\end{itemize}

Estas dimensiones no pueden ser consideradas individualmente, todas están
relacionadas  como se muestra en el figura~\ref{fig:desarrollo_dimensiones}.

\begin{figure}[H]
\centering
\includegraphics[scale=0.5]{juegos_serios/desarrollo_dimensiones.png}
\caption{Relación entre las cuatro dimensiones a considerarse en un videojuego
    basado en aprendizaje}
\label{fig:desarrollo_dimensiones}
\end{figure}

De modo a contrastar la práctica con el conocimiento teórico adquirido en la
investigación del estado del arte de las \gls{tic} en la educación y de los
juegos serios, se incluye el diseño e implementación de un juego serio como
parte de este trabajo.

Esta aplicación debe cumplir con los criterios y factores definidos
anteriormente para obtener conclusiones en cuanto a los factores pedagógicos, de
diseño, de implementación y de evaluación.

\section{Contexto de aplicación} 
% Debe tener en cuenta que acá se define el problema

Uno de los principales campos de aplicación de los juegos serios en la educación
es el área de la salud, un ejemplo de esto es la formación de profesionales de
enfermería. Según~\cite{humphreys2013developing} los alumnos de enfermería son
estudiantes divergentes\footnote{Los alumnos divergentes son aquellos que
    aprenden a través de experimentación activa, e interiorizan el conocimiento
    reflexionando sobre la experiencia\cite{humphreys2013developing}}, los
juegos serios son una herramienta ideal para este tipo de
estudiantes\cite{humphreys2013developing}. 

Así, se selecciona como contexto de aplicación a la enseñanza de la carrera
Licenciatura en Enfermería en el \Gls{iab}. Por ello, a continuación se describe el estado
actual de la enseñanza, incluyendo como se estructura la carrera, su plan de estudios, y
las competencias que debe tener el profesional de enfermería recién egresado.

%! TEX root = ../main.tex

\subsection{Estado actual}

La enfermería es una profesión técnica, en esta sección se hace una breve reseña
de los métodos de enseñanza fuera del aula que se utilizan actualmente. Se
describe como se realiza la evaluación de los estudiantes, tanto en el área
teórica como en el área práctica. Luego se detallan los principales problemas a
abordar, describiendo los inconvenientes que tiene la metodología actual.

Una observación importante es que la información no bibliográfica presentada en
esta sección, es fruto de reuniones con profesores, encargados y directores de
la carrera de enfermería del \Gls{iab}.

De este modo, se describe el contexto en el que se quiere aplicar una solución.


\subsubsection{Plan de estudio}
\label{sec:plan_estudio}

La carrera de licenciatura en enfermería en el \Gls{iab} tiene una duración de $4$
años, es completamente presencial y tiene una carga total de $3745$
horas. Cada alumno debe aprobar $57$ materias.

Para completar las horas necesarias, las clases se desarrollan en 
dos turnos de manera continua, con excepción de los días donde existen
prácticas de campo. La mayoría de las materias son teóricas, desde segundo curso
acceden a los laboratorios especializados del instituto, y desde el tercer curso
realizan prácticas de campo en hospitales escuela y hospitales con los cuales el
\Gls{iab} tiene convenios.

La carrera cuenta con $150$ alumnos nuevos por año, los mismos se dividen en tres secciones. 
El perfil del egresado de la carrera de licenciatura en enfermería 
es\cite{iab:enfermeria}:

\begin{displayquote}

El profesional egresado de la Licenciatura en Enfermería será capaz de
desempeñar eficientemente el saber teórico y práctico en el campo de su
profesión, valorar las necesidades y problemas bio-psico-sociales y espirituales
del individuo, familia y comunidad, brindando apoyo y proponiendo alternativas
de solución, practicar los valores de honradez, solidaridad y respeto al ser
humano en la prestación de servicios de la salud.

\end{displayquote}

Existen tres formas principales de enseñanza dentro del \Gls{iab}, 
\begin{enumerate*}[label=\itshape\alph*\upshape.]
\item las clases teóricas, 
\item las prácticas de laboratorio y, 
\item las prácticas de campo.
\end{enumerate*}


El plan de estudios se centra en las \emph{competencias básicas} que debe tener
cada alumno al finalizar la materia, estas competencias son facilitadas al
inicio de cada asignatura a los alumnos.

Las competencias básicas son los conocimientos teóricos y prácticos que debe
tener todo profesional de enfermería recién egresado, estas competencias son el
eje central de la carrera y en la obtención de las mismas se centran todas las
actividades curriculares y no curriculares (congresos, encuentros, etc)
realizadas por el \Gls{iab}.



\subsubsection{Prácticas en laboratorios}
\label{sec:practica_lab}

El \Gls{iab} cuenta con un laboratorio especializado para la práctica de los
estudiantes de enfermería. El laboratorio es utilizado por los alumnos 
desde su segundo año de formación, y en el mismo se desarrollan todas las materias 
prácticas, de manera a realizar una formación previa a las prácticas de campo 
explicadas más adelante.

El número de alumnos dificulta la enseñanza individual, por ello las prácticas se
dividen en dos partes, en la primera, similar a una aula tradicional, los
alumnos se sientan y observan al profesor realizar una simulación de
procedimientos sobre un voluntario, en este punto, el profesor realiza las
observaciones que crea son necesarias para llevar a cabo la práctica
profesional, da consejos y responde a las dudas de los alumnos.
Se utilizan modelos del cuerpo humano para simular
algunos procedimientos, en la figura~\ref{fig:iab_veno} se observa un modelo del
brazo humano utilizado para procedimientos de venopunción.

\begin{figure}[h!t] 
\centering 
\includegraphics[scale=0.2,natwidth=100,natheight=100]{problema/iab_sala_2.jpg}
\caption{Elementos utilizados para mostrar procedimientos de venopunción}
\label{fig:iab_veno}
\end{figure}

En la segunda parte, los alumnos pasan a un laboratorio que contiene las
herramientas necesarias para la práctica (maniquíes, camas de hospital, y otros
elementos, como se ven en la figura~\ref{fig:iab_lab}), donde pueden explorar y
practicar siempre bajo tutela del profesor. Esto se diferencia
principalmente de la primera parte, en que hay más material para las pruebas y los
alumnos pueden realizar por sí mismos una simulación de los procedimientos.

\begin{figure}[h!t] 
\centering 
\includegraphics[scale=0.3]{problema/iab_sala_1.jpg}
\caption{Laboratorio de enfermería del \Gls{iab}}
\label{fig:iab_lab}
\end{figure}


El maniquí que se observa en la figura~\ref{fig:iab_mani}, tiene ciertas
características que facilitan la práctica, por ejemplo, tiene un esquema de los
vasos sanguíneos en ambos brazos. Este maniquí se utiliza además para mostrar
las partes del cuerpo donde se puede realizar la venopunción, para mostrar la
zona específica donde se debe realizar la reanimación, y otras zonas importantes
para la práctica de enfermería.


\begin{figure}[b!t] 
\centering 
\includegraphics[scale=0.15]{problema/iab_sala_3.jpg}
\caption{Una instructora de laboratorio muestra las partes del maniquí utilizado
    para en el laboratorio de enfermería.}
\label{fig:iab_mani}
\end{figure}

Además existen varias camas de hospitales (como se observa en la
figura~\ref{fig:iab_lab} a la izquierda), donde se practica la higienización del
paciente, como utilizar los mecanismos de ajuste de la cama, las diferentes
telas utilizadas para las sábanas, y otros aspectos relacionados al cuidado de
un paciente en cama.


\subsubsection{Prácticas de campo}
\label{sec:practica_hos}

Las prácticas de campo son aquellas prácticas profesionales que son
realizadas por los alumnos con pacientes humanos y en hospitales, bajo
supervisión de un profesional y bajo una continua evaluación de sus
acciones, las mismas son llevadas a cabo una vez que los alumnos finalizan
las prácticas de laboratorio.


Los alumnos del \Gls{iab} participan en prácticas de campo 
en diferentes hospitales dependiendo de las necesidades de cada
materia, por ejemplo, los alumnos de \textit{Enfermería en Urgencias} realizan
sus prácticas en el \textit{Centro de Emergencias Médicas}, otros hospitales
utilizados, son el \textit{Hospital de Clínicas}, y diversos hospitales del
\textit{Instituto de Previsión Social}.


Para controlar y medir la evolución de los estudiantes existe un grupo de
profesores cuya función es guiar a los alumnos durante las prácticas de campo,
este grupo de profesores son denominados \textbf{instructores}.

Las prácticas se realizan en grupos que varían de $4$ a $10$ alumnos, dependiendo de
la disponibilidad de instructores y de si el área es crítica o no\footnote{Se
dice que un paciente esta en estado crítico si su vida depende de un
procedimiento externo, como una transfusión de sangre. Un área se considera
crítica si los pacientes en su mayoría son críticos}, un instructor puede
manejar más de un grupo en diferentes horarios. 

Cada instructor posee un planilla por alumno donde se realiza el seguimiento de
sus actividades. La creación de esta planilla de actividades es responsabilidad
del instructor, el instructor debe basarse en las competencias básicas de la
asignatura y la misma es validada por la dirección de la carrera, se considera
que un alumno ha adquirido la pericia\footnote{Sabiduría, práctica, experiencia y 
habilidad en una ciencia o arte.} necesaria para una asignatura solo si pudo
completar la planilla del instructor. Son registradas todas
las actividades del alumno, pero sólo son tomadas en cuenta para el progreso final 
aquellas que son realizadas con la pericia requerida.

\subsubsection{Evaluación a los estudiantes}
\label{sec:problema_evaluacion}

Como dicta su perfil, un egresado de Licenciatura en Enfermería 
debe ser capaz de desempeñar eficientemente su
profesión, el mecanismo que se utiliza para garantizar esto son las
evaluaciones.

Una evaluación es un proceso que permite verificar el grado del progreso del
estudiante en el logro de los objetivos propuestos en cada
asignatura\cite{iab:est_enfemeria}, existen tres tipos de evaluaciones, exámenes
parciales, exámenes finales y evaluación de la práctica de campo.

% VER PARA BORRAR
La cantidad de evaluaciones parciales\footnote{Se
    define examen parcial aquel que mide el rendimiento del período
    correspondiente\cite{iab:est_enfemeria}} está determinada por la materia y el
consenso de los profesores titulares\cite{iab:est_enfemeria}. En cuanto a las 
evaluaciones finales, existen tres períodos en los cuales un alumno puede rendir 
el examen final.


Cada alumno necesita de un $75\%$ de asistencia presencial para tener derecho a
las evaluaciones, así mismo, cada alumno requiere como mínimo $80\%$ de la carga
horaria en prácticas profesionales, el $20\%$ restante lo debe cumplir en un
periodo establecido por el \Gls{iab}.
% esto del 20% restante se puede borrar

En cuanto a la evaluación de la práctica de campo, el enfoque es
subjetivo, es decir depende exclusivamente del instructor de la práctica
determinar si un alumno cuenta o no con la pericia necesaria.
    
Cabe destacar que, si el alumno no aprueba sus prácticas de campo no tiene 
derecho al examen teórico de la materia en cuestión y por lo tanto debe volver 
a cursar la materia. De esta forma, aprobar la práctica de campo se convierte 
en un requisito importante para el progreso del alumno en su vida académica.

El objetivo final de estas evaluaciones es el de evaluar si un alumno comprende
y tiene la pericia necesaria en todas las competencias básicas de una
asignatura.



\subsection{Problemas actuales}
\label{sec:problemas_actuales}
%\observacion{Quizás convenga restar más, tipo sección 4.2}

Si bien el nivel actual de los egresados del \Gls{iab} en la carrera
Licenciatura en Enfermería es satisfactorio, existen ciertos inconvenientes, los
mismos son recabados de distintas fuentes, como tesis de
alumnos\cite{iab:tesis_atencion,iab:tesis_alumnos} y apreciaciones de los
profesores y de alumnos egresados.

En algunos casos los profesores de campo prefieren tener las primeras clases en
el laboratorio antes de ir a los hospitales debido a que se suelen presentar los
siguientes inconvenientes:

\begin{itemize}

    \item \textbf{Falta de preparación de los alumnos:} ciertos detalles
        necesarios para la práctica de campo no son completamente cubiertos en
        el laboratorio.

    \item \textbf{Nerviosismo ante primera práctica:} ciertos alumnos reaccionan
        de manera inesperada la primera vez que deben realizar una práctica,
        esto se debe principalmente a que ciertos procedimientos son impactantes
        y ni el laboratorio ni el aula pueden preparar para este tipo de
        experiencias.
                
    \item \textbf{Definición de un protocolo de comunicación:} en la práctica de
        campo, los profesores necesitan comunicarse con sus alumnos de una
        manera rápida y eficiente, debido a esto los profesores enseñan a sus
        alumnos ciertos códigos que son utilizados para corregir, notificar y
        enseñar durante la práctica.
          
\end{itemize}


En cuanto al punto de vista de los alumnos, los principales inconvenientes que
tienen los alumnos de enfermería del \Gls{iab} son:

\begin{itemize}

    \item \textbf{Carga horaria de trabajos prácticos:} se refiere al tiempo
        necesario por los estudiantes para llevar a cabo un trabajo
        práctico\cite{iab:tesis_alumnos}.
        
    \item \textbf{Carga horaria de materias teóricas:} se refiere al tiempo que
        consumen las materias teóricas, cuyo tiempo de estudio es reducido por
        la necesidad de acudir a prácticas de campo en horarios
        variados\cite{iab:tesis_alumnos}.
        

    \item \textbf{Falta de materiales para los profesores:} la enfermería es un
        área en constante evolución, los materiales se vuelven obsoletos
        rápidamente, y los profesores no cuentan con una fuente actualizada de
        información\cite{iab:tesis_alumnos}.
        
    \item \textbf{Problemas de transporte:} la ubicación del \Gls{iab} facilita
        el acceso al mismo desde rutas internacionales, pero no se puede decir
        lo mismo de los hospitales donde se realizan prácticas
        profesionales\cite{iab:tesis_alumnos}. 
        
        Este problema es acentuado por la gran cantidad de tiempo que deben
        pasar los alumnos en los medios de transporte para moverse desde sus
        respectivos hogares hasta el \Gls{iab} o a los campos de
        práctica\cite{iab:tesis_alumnos}.
        
        Adicionalmente, la población del \Gls{iab} esta compuesta en su gran
        mayoría por personas de niveles económicos medio-bajos y un gran
        porcentaje de los alumnos son del interior del
        país\cite{iab:tesis_alumnos}.
    
    \item \textbf{Preparación para las prácticas:} los alumnos rara vez están
        completamente preparados la primera vez que realizan una práctica de
        campo, muchos sufren ataques de pánico y no pueden reaccionar de manera
        correcta.
        
    \item \textbf{Cantidad de alumnos:} debido a la cantidad de alumnos, las
        prácticas de campo rara vez se realizan en un sólo hospital, para
        asignaturas críticas, se forman aproximadamente $35$ grupos de
        estudiantes.

\end{itemize}

%! TEX root = ../main.tex

\section{Propuesta de solución}

% NO VEO COMO EL JUEGO MEJORE ESO por que no se refiere a la falta de comunicacion a la
% hora de darle una retroalimentacion al alumno si no al lenguaje que van a utilizar
%
%\observacion{Estos problemas puede ser atacados por su solución. Se refiere
%a los problemas de comunicación entre profesor y alumno}

Una vez mencionado el estado actual de la formación de profesionales de
enfermería en el \Gls{iab} incluyendo sus problemáticas actuales, \fixme{no
    resulta}{Refinar} extraño pensar en una herramienta tecnológica que les
sirva de apoyo en su proceso de aprendizaje en forma de un juego serio.

Los principales problemas que puede abordar una solución con estas
características son los siguientes:

\observacion{Cambiar el formato, que sea algo así: Tema, solución y luego motivo
    (Párrafos separados)}

\begin{itemize}

\item \textbf{Evaluación}
    
    Las prácticas de campo y en laboratorio son un requisito para aprobar las
    materias que requieren prácticas, no aprobarlas significa volver a cursarlas
    por lo que es importante para un alumno en cuanto a su vida académica y para
    los profesores en cuanto a asegurar que los alumnos tengan los conocimientos
    requeridos.
    
    Una solución tecnológica permite un enfoque objetivo, esto es una diferencia
    sustancial con el mecanismo actual, en el cual la nota del alumno depende de
    la opinión del profesor, esto permite, entre otras cosas, que todos los
    alumnos sean evaluados de manera más similar.
    
\item \textbf{Progreso}

    Si bien existe una planilla de progreso del alumno, esta planilla almacena
    sólo los éxitos del alumno, es decir, cada vez que el instructor
    \emph{considere} que el alumno realizó una tarea de manera correcta, marca
    una casilla en su planilla de progreso.

    Una ventaja de la utilización de la tecnología para la misma tarea, es la
    capacidad que tiene para almacenar información, se podría almacenar no sólo
    cuantas veces cometió un error, sino también los detalles que llevaron al
    error entre múltiples datos interesantes para analizar el avance del alumno,
    como por ejemplo, en que parte del procedimiento encuentra más dificultades,
    cuanto tiempo tarda en realizar el procedimiento, etc.
    
\item \textbf{Tiempo de práctica}
     
    El tiempo que los estudiantes pasan en clases y prácticas es muy extenso por
    lo que no les queda casi tiempo para actividades extras.
    
    En cuanto al aspecto tecnológico, existen herramientas que permiten crear
    soluciones que puedan ser utilizadas de múltiples formas, en el celular, la
    computadora, etc. Es interesante contrastar esta posibilidad con uno de los
    problemas comunes, como es la falta de tiempo, ya que estas herramientas les
    puede permitir estudiar en el tiempo que están fuera de clases.
    
    Dado el problema de tiempo, una solución tecnológica a este debería poder
    utilizarse en cualquier momento y la experiencia no debe ser extensa.

    
\item \textbf{Factor psicológico}

    % A QUE CITAS TE REFERIS??? VOY A COMENTAR
    En el aspecto psicológico, actualmente, existen casos donde los alumnos no
    pueden manejar la primera experiencia con un paciente, la utilización de
    esta solución podría ayudar %(acá citas,
  %  hay varias que sirven, ver ventajas juegos serios) 
    al alumno a entender, interpretar y actuar en una situación realista.   
    
    Esta herramienta también puede enseñarle u orientarle cuando el estudiante
    no haga correctamente los procedimientos, dándole una retroalimentación, y
    permitiendo al mismo experimentar las situaciones sin poner en riesgo su
    vida y la del paciente, adicionalmente no hay riesgos económicos, como el
    desperdicio de material u herramientas.
    
\item \textbf{Ubicuidad}

    Actualmente las prácticas de laboratorio están centralizadas en el
    \Gls{iab}, y las prácticas de campo se realizan en diferentes hospitales.
    Los alumnos invierten gran parte de su tiempo en el transporte hasta el
    lugar de la práctica.
    
    Existen alternativas tecnológicas que permiten al usuario experimentar en
    entornos virtuales desde sus teléfonos móviles, lo que permitiría a los
    mismos utilizarlo en cualquier momento, siendo el único requisito tener el
    dispositivo móvil.
    
    
\item \textbf{Realismo}
    
    % COMO QUE?? VOY A COMENTAR
    %Las prácticas en los laboratorios tienen problemas de realismo, pues
    %utilizan un maniquí estático, el cual tiene ciertos características, como .
    
    Uno de los desafíos impuestos por perseguir la ubicuidad, es el nivel de
    realismo posible, al permitir que la solución corra en un dispositivo móvil,
    la cantidad de espacio para mostrar detalles es reducido y la forma de
    uso de la solución debe ser sencilla. Así, la solución no podrá representar
    un maniquí con la misma facilidad de manipulación que el maniquí del
    \Gls{iab}, si bien, algunos detalles pueden ser más realistas, la
    utilización del mismo no lo será.
    
    Es importante notar que no es un objetivo de la solución realizar una
    simulación detallada de los procedimientos, sino proveer una experiencia que
    permita al usuario comprender el procedimiento y sumergirse en el entorno.
    Según~\cite{videojuegos:gonzaleztardon} es necesario definir minuciosamente
    qué factores son simulados y cuáles no, pues si se simulan demasiados
    factores, el usuario podría perderse en los detalles. 
    
\item \textbf{Enfoque individual}
    
    La cantidad de alumnos dificulta la orientación individual por parte de los
    profesores, por ejemplo, en las materias no críticas, existen $7$ alumnos
    por instructor, en las practicas de laboratorio, existen $50$ alumnos por
    profesor.

    Un entorno virtual permite tratar al alumno individualmente, permitiendo al
    mismo experimentar en un entorno sin poner en riesgo a ningún ser humano.

\end{itemize}




% Una encuesta sobre el acceso a tecnología móvil que tienen los estudiantes de la
% carrera de licenciatura en enfermería detallado en el capitulo xxx y cuyos resultados 
% se muestran en el cap xxx nos indican que en su mayoria tienen acceso a un telefono 
% inteligente por lo que utilizar esta caracteristica como una propuesta para el 
% problema de tiempo resulta ser interesante.


%%%%%%%%%%% ANTES DE ESTO YA TIENE QUE DECIRSE LO DE LOS DISPOSITIVOS MOVILES POR QUE
%%%%%%%%% ACA YA DICE QUE VAMOS A HACER UNA SOLUCION MOVIL

%%%%%%%%%%%%%%%%%
%%%%%%%%%%%%%%%%%
% MIRTA VE ESTO %
%%%%%%%%%%%%%%%%%
%%%%%%%%%%%%%%%%%
Observando estos problemas se propone el desarrollo de una aplicación para
dispositivos móviles que se define como un juego serio llamado \Gls{yave}, el
cual toma prestados conceptos del construccionismo y de las simulaciones
educativas, con el objetivo de proveer un entorno virtual que permita a los
alumnos de enfermería realizar procedimientos en un entorno seguro.

El juego consiste en ofrecer a los usuarios, en este caso alumnos de enfermería,
un medio en el cual puedan realizar procedimientos de enfermería y cuyo objetivo
es servir como herramienta de apoyo en el aprendizaje.

Esta solución ayudaría a los estudiantes a tener más oportunidades de poner en
práctica sus conocimientos con un paciente virtual que incluso puede reaccionar
a sus acciones a diferencia de un maniquí de laboratorio, además les permite
poder hacerlo en cualquier lugar y momento.

\observacion{Donde quedo al final el tema de diseño y las limitaciones?}
\observacion{Las justificaciones están escondidas en cada item}

%\pregunta{Martín: tenemos que hablar más sobre las limitaciones tecnológicas}
%\observacion{En alguna parte de por acá tenemos que hablar de las limitaciones
%    tecnológicas, pues más adelante en cuanto se definen los criterios se
%    necesita.}
%\observacion{No se puede hablar de selección de procedimiento sin hablar de las
%    limitaciones tecnológicas. Hay que describir brevemente cual es el prototipo
%    de dispositivo e interfaz que se quiere utilizar. Y justificar (este es más
%    fácil)}


%! TEX root = ../main.tex
\chapter{Propuesta de solución}
\label{chap:solucion}

%

\section{Acciones condicionadas por eventos}

Las \gls{eca} son aquellas que son lanzadas una vez que se cumple un determinado
evento\cite{bailey2004event}. En las bases de datos relacionales, son conocidos



Las mismas pueden ser utilizadas para notificar que un determinado conjunto de
eventos ha ocurrido\cite{bailey2004event}, así como servir para almacenar
información acerca de la utilización de un determinado recurso.

\subsection{Motivación}

Las reglas del tipo \gls{eca} permiten reaccionar a determinados eventos, en
forma de una única regla, la cual facilita la declaración de las
mismas\cite{bailey2004event}.

Son principalmente útiles para analizar el comportamiento en tiempo real de un
sistema\cite{bailey2004event}.
%TODO agregar más motivaciones.



\subsection{Declaración}

Una \gls{eca}, se define como:

\begin{center}
	 Cuando ocurren una serie de \emph{eventos}, y se cumple una
	 \emph{condición}, entonces realizar una acción. \emph{Acción}
\end{center}

Los eventos determinan cuando una regla debe ser activada, los mismos se dividen
en dos categorías, primitivos y compuestos, los primeros son detectables, por
ejemplo, cuando se inserta una jeringa, y los compuestos, son la combinación de
uno o más primitivos\cite{bailey2004event}. Los eventos compuestos, se unen
mediante:
\begin{enumerate*}[label=\itshape\alph*\upshape)]
\item conjunción (\emph{y}),
\item disjunción (\emph{o}), y
\item secuencia (\emph{entonces}).
\end{enumerate*}
Sin embargo, no siempre son necesarios todas las posibles combinaciones, y las
combinaciones sencillas son más fáciles de optimizar y
probar\cite{bailey2004event}.

Una regla  puede tener argumentos, los cuales son el entorno en el cual se lanzo
el evento que lo lanza.

Las condiciones determinan si el entorno es el necesario para que la regla sea
activada.

La acción a ejecutar describe la lógica que debe ser ejecutada cuando se han
lanzado los eventos y la condición de la regla se ha cumplido.

\subsubsection{Dependencia entre reglas}

Las reglas pueden depender de otras reglas, lo cual se puede ver como que la
finalización de una regla es un evento que otra regla espera para poder ser
activada.

Las reglas pueden agregar información a un contexto compartido por todas las
reglas, de esta manera, se puede pasar parámetros entre distintas reglas, por
ejemplo, la regla \emph{Retirar Torniquete}, depende de la regla \emph{Insertar Torniquete}, pero debe responder solamente al torniquete
que ha activado la regla de inserción, es decir, el usuario puede extraer varios
torniquetes, y la regla no debe activarse, hasta que se extraiga el torniquete
que activo la primer regla.

Así, la regla \emph{Retirar Torniquete} depende de la regla \emph{Insertar
Torniquete}, y esta relación entre reglas, se da en dos formas:

\begin{itemize}
\item  \emph{Dependencia fuerte:} la regla \emph{Retirar Torniquete} solamente podrá
	ser elegida para ser lanzada cuando la regla \emph{Insertar Torniquete}
	haya sido cumplida.
\item  \emph{Dependencia de contexto}: la regla \emph{Retirar Torniquete} no se
	activará cuando los eventos a los que escucha se terminen, sino cuando
	los eventos a los que escucha sean lanzados con los parámetros adecuados
	(se extraiga el torniquete que lanzo la regla de inserción).
\end{itemize}




\subsection{Modelo de ejecución}

Para ejecutar un motor de reglas del tipo \gls{eca}, se debe tener en cuenta
principalmente dos factores, 
\begin{enumerate*}[label=\itshape\alph*\upshape)]
\item  Como se verifica el cumplimiento de cada regla, y, 
\item  Que ocurre cuando varias reglas son lanzadas al mismo tiempo.
\end{enumerate*}.

Para ambos casos se puede tomar un enfoque \emph{inmediato}, es decir que
inmediatamente cuando se lanza un evento, o se cumple una condición, se ejecuta
la regla. Además existen otros dos modos de ejecución, \emph{deferida}, y
\emph{desacoplada}, en la primera, se espera hasta que el lanzador del evento
culmine su trabajo, y luego se ejecute la regla, pero en la misma unidad de
trabajo, mientras que en la ejecución desacoplada, se encolan los trabajos y
otro hilo es el encargado de ejecutar las reglas.

La propuesta implementada, utiliza una ejecución inmediata, principalmente por
la sencillez de las reglas, es decir, las reglas no realizar un trabajo pesado,
solamente controlan el estado del entorno y lo validan.

Además, la ejecución inmediata es importante por que el entorno no sufre
modificaciones entre el evento lanzado y la ejecución de la regla, según
\cite{bailey2004event}, este es el factor más importante para determinar el tipo
de ejecución deseado.



\subsubsection{Estados de una regla}

Una regla puede estar en uno de los siguientes estados:

\begin{description}
\item[BEGIN] Es una regla que recién fue creada, no realiza ninguna
	acción.
\item[WAITING\_FOR\_RULE] Es un estado en el que esta esperando que otras reglas
	sean lanzadas.
\item[WAITING\_FOR\_EVENT] Es un estado en el que esta escuchando a que sean
	lanzados los eventos a los que escucha, este es el estado principal.
\item[WAITING\_FOR\_CONDITION] La regla ya no espera por ningún evento y las
	reglas de las que depende ya han sido lanzadas, se verifica cada cierto
	tiempo si el entorno cumple con una condición definida.
\item[FINISH] La regla ha sido lanzada, con un resultado no determinado, se pudo
	haber cumplido, como no, es el estado final de una regla. Cuando una
	regla llega a este estado, se lanza su evento de finalización.
\end{description}

Una regla puede estar en solo un estado, y solamente se permite que el estado
avance, desde \emph{BEGIN} hasta \emph{FINISH}.

\subsubsection{Ciclo de vida}

Cuando una regla es definida, y insertada al motor de reglas, inmediatamente
pasa al estado \emph{BEGIN}, luego se verifica si la misma depende de otras
reglas, sí este es el caso, pasa al estado \emph{WAITING\_FOR\_RULE} y escucha a
los eventos de finalización de las reglas anteriores.

Una vez que las reglas anteriores han sido finalizadas, la regla pasa al estado
\emph{WAITING\_FOR\_EVENT} sí deben escuchar por algún evento, en caso contrario
pasan al estado \emph{WAITING\_FOR\_CONDITION}.

Una vez que la regla está en estado \emph{WAITING\_FOR\_CONDITION}, pasa a un
motor que ejecuta su condición cada cierto tiempo, si la condición se cumple, la
regla se ejecuta, y la misma pasa a estado \emph{FINISH}, momento en el cual
notifica a las reglas que dependen de ella que ha sido lanzada.

Una vez que la regla esta en estado \emph{FINISH}, la misma sale del esquema de
ejecución, y solo esta disponible para obtener resultados.




\section{Descripción general de la solución}

Una vez descriptos los fundamentos teóricos del uso de videojuegos en el ambiente educativo sobre todo en aquellos que requieren de mucha practica para plasmar los conocimientos. Siendo el área de enfermería una de estas áreas definimos las problemáticas actuales de esta y seleccionamos algunos procedimientos que serán utilizados para el contenido de la solución. A continuación se busca converger todos los aspectos descriptos en
capítulos anteriores.

La solución propuesta en este trabajo consiste en el desarrollo de una aplicación para dispositivos móviles que se define como un juego serio llamado YAVE el cual incluye ciertos aspectos de gamification. El juego consiste en ofrecer a los usuarios, en este caso alumnos de enfermería, un medio en el cual puedan realizar procedimientos de enfermería y el cual les puede servir como herramienta de apoyo en su aprendizaje.

YAVE permitirá al usuario poder seleccionar el procedimiento que quiera realizar, en cada procedimiento se le dará la posibilidad de interactuar con un paciente y con una conjunto de objetos que forman parte de las herramientas requeridas para realizar el procedimiento en cuestión. Además, le permitirá realizar acciones de bioseguridad.

La aplicación no solo le permitirá al usuario realizar los procedimientos para poner en practica sus conocimientos sobre el mismo sino también evaluará al usuario, dándole al final de la partida una puntuación final y diciéndole cuales son los pasos que realizó de manera correcta y cuales de manera incorrecta, proporcionándole una pequeña información sobre su error.
%! TEX root = ../main.tex
\section{Tecnologías disponibles}

Para el desarrollo de videojuegos se utilizan programas o herramientas especializadas en ello llamadas motores de videojuegos. A continuación se da una breve introducción de lo que es un motor de videojuego o motor gráfico.

El término “motor gráfico” o “motor de videojuegos” hace referencia a una serie de rutinas de programación que permiten el diseño, la creación, el desarrollo y la representación gráfica de un videojuego\cite{videojuego:telechea}.

Además, la gran mayoría de estos motores ofrecen a su vez características y funciones que facilitan la construcción del videojuego, como el motor físico (software capaz de realizar simulaciones de ciertos sistemas físicos como la dinámica de un cuerpo rígido, el movimiento de un fluido o la elasticidad) o detector de colisiones, sonidos, scripting, animaciones, inteligencia artificial, redes, streaming, administración de memoria, etc\cite{videojuego:telechea}.

El motor de juego utilizado depende de las características que posea el videojuego que se quiere desarrollar. Por lo mismo, a continuación se da una breve descripción de los motores mas utilizados actualmente para que posteriormente se puede seleccionar el mas adecuado.

\subsection{UDK (Unreal Development Kit)}

Unreal Engine\cite{unrealengine} es el motor de juegos desarrollado por Epic Games, se ofrece bajo un plan de suscripción de 19 dolares por mes. El servicio de suscripción permite a los desarrolladores unirse a una comunidad dedicada a la construcción de grandes juegos y evolución de Unreal Engine. La suscripción se puede cancelar en cualquier momento.

UnReal Engine 4 permite desarrollar juegos para plataformas como Windows PC, Mac, iOS y Android. También es compatible con Xbox One y PlayStation 4. Existen además trabajos recientes sobre otras plataformas como HTLM5 y Linux.

Sin embargo existe una versión gratuita de UnrealEngine, el Unreal Development Kit o UDK. El UDK es la edición gratuita de Unreal Engine 3 que proporciona acceso al galardonado motor de juegos 3D y herramienta profesional que se utiliza en el desarrollo de videojuegos blockbuster, visualización arquitectónica, el desarrollo de juegos para móviles, modelos 3D, películas digitales y más. Utilizando UDK se pueden implementar juegos y aplicaciones en Windows PC, iOS y Mac.


\subsection{Blender Game Engine}

Blender Game Engine\cite{blender} es el motor de juego de Blender Foundation que permite crear aplicaciones 3D interactivas o simulaciones. La principal diferencia entre el Blender Game Engine y el Blender convencional está en el proceso de renderizado. En el motor Blender normal las imágenes y animaciones se construyen fuera de linea es decir, una vez generadas no pueden ser modificadas. Por el contrario Blender Game Engine genera las escenas de forma continua en tiempo real e incorpora facilidades para la interacción del usuario durante el proceso de renderización.

Procesa la lógica de sonido, de la física y la representación de simulaciones en orden secuencial. El motor esta
escrito en C++. Posee un editor que proporciona una profunda interacción con la simulación, su funcionalidad se
puede ampliar con scripts Python y esta diseñado para abstraer las características complejas del motor en una interfaz de usuario simple, que no requiere programación.

El motor de juego puede simular contenido dentro de Blender, sin embargo, también incluye la posibilidad de exportar en plataformas como Windows, Linux y Mac OS. También hay soporte básico para plataformas móviles con el proyecto Android Blender Player GSOC 2012.

\subsection{CryEngine 3}

CryEngine 3\cite{cryengine} es el motor de juegos desarrollado por Crytek. CryEngine es un motor avanzado para el desarrollo de juegos, películas, simulaciones de alta calidad y aplicaciones interactivas. Es uno de los procesadores de alta gama mas rápido en el mundo, con características diseñadas específicamente para Windows PC, PlayStation 3 y Xbox 360.

Existe una versión gratuita de CryEngine con todas las funcionalidades. CryEngine es  WYSIWYG (What You See Is What You Get). 

CryEngine posee un conjunto de herramientas utilizadas para el análisis de rendimiento. La ultima versión
de CryEngine, CryEngine 3 es el único motor con la última tecnología en iluminación, física e inteligencia
artificial si se está desarrollando en Windows PC, PlayStation 3 y Xbox 360.


\subsection{ShiVa3D}

ShiVa3D\cite{shiva} es un paquete para el desarrollo de juegos y aplicaciones 3D, viene en un formato fácil de usar, pero con un editor WYSIWYG (What You See Is What You Get) muy potente.

ShiVa puede exportar juegos y aplicaciones para más de 20 plataformas de destino, incluyendo móviles como iOS, Android, BlackBerry y Windows Phone, de escritorio como Windows, Mac OS X y Linux, los navegadores web con soporte Flash y HTML5, así como Consolas como la Xbox 360, PlayStation3 y Nintendo Wii. La herramienta de creación se ejecuta en Windows y Mac OS X. Software y hardware Apple son requeridos con el fin de construir juegos para OS X y iOS.

ShiVa también ofrece el ShiVa Server, el cual es la solución perfecta para aplicaciones multi-jugador y multiusuario. Dependiendo de su ancho de banda, puede manejar miles de jugadores al mismo tiempo, sincronizar sus clientes, e incluso dejarlos hablar el uno al otro a través de VoIP. ShiVa Server está disponible como una licencia independiente. ShiVa Server está disponible en Windows y Linux.


\subsection{Unity3D}

Unity\cite{unity3d} es un poderoso motor para el desarrollo de juegos, un poderoso motor derenderizado totalmente integrado con un conjunto completo de herramientas intuitivas y flujos de trabajo rápidos para crear contenido 3D interactivo, desarrollo multi-plataforma sencillo, miles de activos de calidad listos para usar en la tienda
de activos y una gran comunidad donde se intercambian conocimientos.

En Unity se pueden mezclar de forma muy sencilla elementos 2D con 3D. Posee un editor intuitivo y flexible, 
el nivel visual y de audio son de gran calidad, un sistema de animación poderoso y flexible. Mantiene el 
rendimiento y la calidad de la escena manteniéndola fluida cualquiera sea el tamaño de pantalla. 

Permite desarrollar juegos para múltiples plataformas. Para plataformas moviles iOS, Andriod, Windows Phone 8, BlackBerry 10; plataformas de escritorio como Windows, aplicaciones de la Windows Store, Mac y Linux; plataformas web como Internet Explorer, Mozzilla Firefox, Google Chrome; y plataformas de consolas como Xbox 360, Xbox One, Wii, Wii U, Nintendo 3DS.

La versión paga de Unity, Unity Pro permite además plataformas como PlayStation 4, PlayStation 3 y PlayStation VITA.

Debido a la alta popularidad de Unity, un paquete fue desarrollado por Facebook para colocar la API de
Facebook en un SDK escrito en C\#, el cual es atractivo y fácil de usar en la tienda de activos de Unity.

\section{Selección de plataforma}
\section{Hipótesis de la simulación}
\label{sec:hipotesis}


\observacion{Ver en la tesis de Tardon: \emph{no es necesario simular todos los
        pasos}.}
\observacion{Ver si hay que cambiar la palabra trivial}

Las escenas seleccionadas y definidas en~\ref{sec:seleccion_escenas} representan
las acciones que deben realizar los profesionales de enfermería a la hora de
realizar los procedimientos seleccionados, por limitaciones técnicas,
tecnológicas y de tiempo, no es posible realizar una simulación de todos los
pasos requeridos.

Los factores que influyen en que partes se simularán, que partes estarán
presentes solamente a través de opciones y que partes se omitirán son:

\begin{itemize}

    \item \textbf{Limitaciones técnicas}: acciones como la simulación del agua
        (necesarios para el lavado de manos), requieren de requisitos de
        hardware avanzados y un tiempo considerable de desarrollo. Las acciones
        que escapan al alcance del hardware y tiempo de los desarrolladores no
        son simuladas.

    \item \textbf{Importancia}: no todos los pasos definidos en el procedimiento
        oficial son necesarios, por ejemplo, la colocación de los elementos
        cerca del lugar de trabajo, es un paso necesario, pero es considerado un
        paso poco importante y trivial.

        La importancia es evaluada por profesionales del \Gls{iab}, los cuales
        dieron su opinión acerca de cada aspecto simulado, el mismo es tenido en
        cuenta para determinar la importancia de cada acción.

    \item \textbf{Facilidad de realización en la vida real o en el laboratorio}:
        ciertos pasos son triviales en la vida real pero requieren un esfuerzo
        significativo para ser simuladas, como por ejemplo el lavado de manos es
        un procedimiento al que los alumnos están acostumbrados.

        La facilidad que tienen los alumnos con las acciones fue determinada por
        profesores del \Gls{iab}, determinaron que acciones son triviales para
        los alumnos y cuales presentan mayores dificultades en su vida
        profesional.

        Otro aspecto que influye en la facilidad de realización de los
        procedimientos es la familiarización, si los alumnos están
        familiarizados con los procedimientos, estos no son simulados.

\end{itemize}

Estas hipótesis sirven para acotar el alcance de la simulación, definen qué se
simulará y cual es del detalle necesario para alcanzar las competencias básicas.

Existen hipótesis que son globales para toda la simulación, las mismas son:

\begin{itemize}

    \item \textbf{Comandos de voz con interfaz}: para enviar una petición al
        paciente (por ejemplo, preguntarle su nombre), no es necesario
        identificar las palabras del usuario, sino más bien detectar que ha
        hablado y listar las posibles acciones que se pueden realizar.

    \item \textbf{Utilización de la interfaz}: para realizar una acción con los
        elementos, es suficiente con presionar el mismo y seleccionar una acción
        de una lista de opciones, no hace falta emular todas las posibles.

    \item \textbf{Acciones de bioseguridad}:\todox{Definir bioseguridad} Las
        acciones de bioseguridad, se realizan a través de una opción en la
        interfaz gráfica.

\end{itemize}

Otras hipótesis, son tomadas por escena, las dos escenas simuladas son
diferentes en el modo de interacción del usuario con su entorno, por ejemplo, en
la escena de extracción de sangre, el usuario interactúa con el paciente a
través de objetos, en la evaluación de Glasgow, la interacción con el paciente
es directa.

\subsection{Extracción de sangre}
\label{sec:hemocultivo_hipotesis}

Se presentan los pasos mostrados en la sección~\ref{sec:hemocultivo_protocolo},
y adicionalmente se establecen las hipótesis punto por punto y las
consideraciones que deben ser tomadas.

\begin{itemize}

\item \textbf{Preparar el equipo}: la preparación del equipo es un aspecto muy
    importante del procedimiento, pero no es un punto único de la extracción de
    sangre, además las prácticas de los alumnos cubren completamente este paso
    según comentarios de los profesores. \emph{Este paso no se simula}.

\item \textbf{Explicación al paciente del procedimiento a realizar}: es un
    aspecto importante del procedimiento, pero la simulación de una conversación
    alumno-paciente es compleja, según comentarios de los profesores, es
    suficiente con que los alumnos sepan que lo deben realizar y en que moento,
    no es necesario simular la conversación en sí. \emph{Este paso se simula a
        través de un comando de voz con la interfaz}.

\item \textbf{Asepsia de las manos}: este paso forma parte de un área más amplia
    conocida como bioseguridad, la cual es un aspecto transversal a todos los
    procedimientos realizados por los enfermeros. 
    La implementación de una simulación del lavado de mano es compleja, y es un
    aspecto que, al igual que la preparación del equipo, está cubierta por los
    laboratorios, aún así, es necesario que los alumnos sepan en que momento
    deben realizar la asepsia de sus manos. \emph{Se simula este paso a través de una
        opción en la interfaz}, no se simulan los pormenores del lavado de manos.

\item \textbf{Llevar el equipo a la unidad en donde se encuentra el paciente}:
    este es un paso trivial que deben realizar los profesionales, la simulación
    de este proceso no es importante según comentarios de los profesores. Este
    proceso no tiene importancia según los profesionales del \Gls{iab}, \emph{este
        paso no se simula}.

\item \textbf{Vestirse con bata estéril, tapaboca y gorro}: al igual que la
    asepsia de las manos, es importante que los alumnos sepan que lo deben
    hacer, pero no es importante que se simule como lo hacen. 

    Los estudiantes están familiarizados con estas acciones, \emph{se simula el
        momento y el orden en el que el jugador lo hace} a través de una opción
    en la interfaz, no se simula el proceso en sí.

\item \textbf{Calzarse los guantes}: es un paso relacionado a la bioseguridad,
    es importante que se sepa en que momento debe realizarse, pero no es
    necesario simular el proceso. 
    \emph{Se simula el momento y el orden en el que se realiza}, no se simulan
    los pormenores de la acción.

\item \textbf{Ubicar al paciente en posición adecuada}: la ubicación del
    paciente durante la extracción de sangre es un factor determinante para que
    la extracción pueda ser realizada correctamente.

    Los alumnos están familiarizados con este proceso según opinión de los
    profesionales, \emph{este paso no se simula}, el paciente está en la
    posición adecuada al inicio de la simulación.

\item \textbf{Elegir la zona a puncionar}: existen varias partes del antebrazo
    donde se puede proceder a realizar una inyección, el conocimiento de las
    mismas, y el procedimiento para detectarlas, es un factor importante del
    proceso.
    
    Las venas del cuerpo humano se detectan palpando los antebrazos, y sintiendo
    el pulso del paciente, existen dos áreas donde el pulso es suficientemente
    fuerte como para sel sentido, estos puntos y el pulso del paciente deben ser
    detectables por el jugador.

    \emph{Los puntos donde se debe punzar deben ser identificables en la
        simulación}. 

\item \textbf{Colocación del torniquete}: La ubicación y el momento de la
    colocación del torniquete es de vital importancia para el procedimiento, el
    mecanismo utilizado para colocarlo no es relevante, pues el mismo es
    trivial.

    \emph{El hecho de colocar el torniquete es simulado}, el mecanismo para
    hacerlo no es importante.

\item \textbf{Solicitar al paciente que cierre el puño}: El momento en el cual
    se solicita al paciente que cierre la mano es vital para que el
    procedimiento de extracción sea satisfactorio.

    \emph{Este paso es simulado} a través de un comando de voz.

\item \textbf{Esterilizar la zona de punción}: la esterilización de la zona de
    punción es un factor de suma importancia para el procedimiento, así como el
    momento en el que se realiza, \emph{el jugador debería poder esterilizar} la
    zona antes de insertar la jeringa \todox{ver si se debe poner mas detalle para explicar 
    que no se simula enteramente este paso}.
    
\item \textbf{Extraer el protector de la aguja}: La extracción del protector de
    la aguja es un paso necesario, pero trivial, el hecho de retirar el
    protector de la jeringa \emph{no es un paso necesario para el logro de las
        competencias básicas necesarias}, por ello, no se simula.

\item \textbf{Puncionar la piel con la aguja}: este es un paso central en el
    procedimiento, en el se deben tener en cuenta aspectos como la posición
    donde se realiza la punción, y el angulo con el que ingresa la aguja.

    La posición donde se realiza la punción es importante por que depende de la
    ubicación donde se colocó el torniquete, y debe ser en uno de los puntos del
    brazo donde existen venas capaces de soportar el procedimiento, \emph{en
        cada brazo existen dos puntos donde se puede inyectar}.

    En cuanto al ángulo de punción, es un conocimiento importante que deben
    tener los alumnos, el conocimiento es teórico y según comentarios de los
    profesores, es un tema en el cual los alumnos tienen suficiente práctica en
    el laboratorio, \emph{No se simula el ángulo en el cual se inserta la
        jeringa}, es decir, la jeringa siempre se inserta en el mismo angulo.

\item \textbf{Tensar la zona de punción}: el proceso de tensar la zona de
    punción se realiza momentos durante la inserción de la jeringa, el mismo es
    trivial, y para simularlo se requiere que el usuario utilice tres dedos al
    mismo tiempo (dos para tensar y otro para realizar la punción), lo cual
    dificulta la utilización de la solución.

    \emph{Este paso no se simula} por la dificultad técnica que implica utilizar
    tres dedos para realizar una tarea, conjuntamente con la facilidad con que
    se realiza acción. 

\item \textbf{Remover el torniquete}: Al igual que en la colocación del
    torniquete, \emph{se simula el momento} de la extracción por que es
    importante, pero no los detalles de la extracción.

\item \textbf{Solicitar la apertura del puño}: el momento exacto donde se debe
    solicitar al paciente que abra la mano es fundamental para la realización
    correcta de la simulación. \emph{Este paso se simula} a través de un comando
    de voz.

\item \textbf{Extraer la muestra se sangre necesaria}: este es el paso central
    de la práctica, tanto el momento, como la forma es importante simular.

    La extracción de la sangre \emph{se simula} la acción de la extracción, pero
    no detalles como la sangre extraída, la velocidad de extracción y la fuerza
    necesaria por limitaciones de la tecnología.

\item \textbf{Presionar el brazo y extraer la aguja}: la presión del brazo para
    extraer la jeringa es un paso trivial, en cambio el momento en el que se
    extrae la aguja es conocimiento necesario para el procedimiento.

    \emph{No se simula esta acción}, pues es un paso al que los alumnos están
    acostumbrados y de simularse, agrega una complejidad adicional a la
    extracción, que es un paso que se realiza al mismo tiempo.

\item \textbf{Colocar algodón con alcohol en el punto de punción}: este paso es
    importante, tanto el momento en el cual se debe realizar, como la forma de
    realizarlo.

    \emph{Se simula la colocación del algodón}, así como el tiempo que se debe
    presionar el mismo.

\item \textbf{Sellar la muestra y enviarlo a su destinatario}: es necesario que
    los alumnos sepan que este paso debe ser realizado, pero los detalles del
    mismo, no son necesarios para el logro de las competencias básicas,
    \emph{este paso no se simula}.


\item \textbf{Retirar la bata, tapaboca, gorro y guantes}: es necesario que los
    alumnos sepan que deben desechar todos los elementos que fueron utilizados
    durante el proceso, la forma de hacerlo no es necesaria.

    \emph{Se simula el momento en el que se extraen los elementos} a través de
    una opción en la interfaz.

\item \textbf{Asepsia de las manos}: la asepsia final de las manos es un paso
    necesario para el procedimiento, así como la asepsia inicial, es importante
    que los alumnos sepan el momento en cual deben realizarlo, \emph{este paso
        se simula} a través de una opción en la interfaz.

\end{itemize}

Estas hipótesis afectan directamente el desarrollo de la aplicación, dictando
que partes del procedimiento se simulan y como, pueden ser vistas como
requisitos funcionales de la solución.

\subsection{Evaluación de glasgow}
\label{sec:glasgow_hipotesis}

En la sección~\ref{sec:glasgow_protocolo} se definieron los pasos necesarios
para poder llevar a cabo el procedimiento, este procedimiento es un paso
rutinario que deben realizar los profesionales para poder determinar rápidamente
el estado de conciencia de un individuo. 

El paso central de la práctica es la valoración del paciente y la generación del
diagnostico, los demás pasos, no colaboran en el desarrollo de la competencia
básica y según opiniones de los profesores no son importantes.

Así, es suficiente con simular al paciente y las reacciones que tiene ante las
acciones del jugador, las siguientes hipótesis se basan en la interacción
paciente/jugador.

El último paso del procedimiento es el registro final de la puntuación, el mismo
es utilizado como mecanismo de evaluación y el registro en sí no se simula, es
decir, se solicita al jugador que realice el diagnostico (mediante un menú),
pero no en la condiciones que se utilizan en la vida real (anotando en el
registro médico del paciente).

Para simular la medición del estado del paciente, se toman las siguientes
hipótesis:

\begin{itemize}
    \item La provocación de un estimulo doloroso al paciente es una accion
        necesaria, pero no así los detalles de la misma, \emph{se simula el
            estimulo con una opción} al personar al paciente en alguna
        extremidad.
    \item El dialogo jugador/paciente se realiza a través de comandos de voz, el
        nombre del paciente es una información que se conoce de ante mano y
        \emph{Se simulan 7 posibles preguntas}, que incluyen solicitudes de
        apertura ocular, movimiento de extremidades, y preguntas generales como
        el nombre del paciente, el día y el lugar.
\end{itemize}



%! TEX root = ../main.tex

\section{Solución}
\label{sec:solucion}

\observacion{Ver donde pone la interacción con la cámara}

Se describe la arquitectura propuesta para la realización de una juego serio, se
utiliza la guía básica definida por~\cite{pereira2009design} y descrita
en~\ref{sec:desarrollo}.

Esta sección se enfoca en los aspectos técnicos de la creación del juego serio,
las competencias básicas relacionadas con la educación (segundo paso de la guía
descrita en~\ref{sec:desarrollo}) se define en las
secciones~\ref{sec:glasgow} y~\ref{sec:hemocultivo}.

\subsection{Partes de la simulación}

La simulación se compone de tres elementos principales, entidades (que son
objetos de la vida real), acciones (que son provocadas por las entidades) y
eventos (que son el resultado de una acción). 

Existen otros elementos dentro de la simulación, como la sala y la iluminación,
los mismos son importantes para crear un entorno similar a la realidad y son
estáticos, es decir no interactúan con el usuario más que para limitar la
exploración en el escenario y/o resaltar aspectos relevantes.

\subsubsection{Entidades}

Cualquier objeto o componente en el sistema que requiera la representación
explícita en el modelo\cite{banks2000dm}. Las entidades tienen atributos. Los
atributos son las características de una determinada entidad que son exclusivos
de esa entidad.

Una entidad tiene en todo momento, un estado y una lista de acciones que
puede realizar, esta lista de acciones esta definida por el estado del mismo,
las condiciones en la que se encuentra el entorno y la práctica actual.

La entidad \enquote{Enfermero} es la que es controlada por el usuario, a través
de la interacción con la interfaz gráfica.

\subsubsection{Acciones}

Las entidades se comunican a través de acciones, las cuales pueden tener
diversos orígenes, siempre una entidad inicia una acción. Las acciones provocan
cambios en el ambiente y provocan eventos. Las acciones no solo las
realiza el usuario, sino cualquier entidad.

Como ejemplo, una acción es esterilizar las manos, esta acción provoca un
cambio en el ambiente (las manos ahora son estériles) y fue realizada por la
interacción entre el usuario y la interfaz gráfica.

\subsubsection{Eventos}

Los eventos son ocurrencias instantáneas que cambia el estado de un
sistema\cite{banks2000dm}, cada acción que se realiza provoca una acción, y los
eventos son la mecanismo que tiene una entidad para ser notificada de las
acciones de otras entidades.

\subsubsection{Interacción con el entorno}

El usuario se desenvuelve en un entorno de tres dimensiones, en el cual realiza las
actividades relacionadas a la práctica, se distinguen dos tipos de movimientos
principales que el usuario puede realizar:

\begin{itemize}
    \item \textbf{Alejamiento o acercamiento}: es el acto de acercar o alejar la
        cámara, y por consiguiente al usuario del paciente. Se realiza
        utilizando dos dedos, para realizar un acercamiento, mientras se
        mantiene presionada la pantalla con ambos dedos, se procede a alejar un
        dedo del otro, para realizar un alejamiento, se debe acercar ambos
        dedos.
    \item \textbf{Rotación}: se refiere al movimiento de rotación al rededor de
        un foco, que en ambas escenas es el paciente, para realizara, se utiliza
        un dedo, y se mueve en dedo en cualquier dirección, la cámara, se moverá
        en la dirección contraria.
\end{itemize}

\subsection{Grafo de estados}

La solución tiene varias escenarios, y dentro de cada escenario, existen varias
pantallas que muestran información relevante de acuerdo a la situación de la
simulación, en~\ref{fig:grafo_estados} se observa la interacción entre las
diferentes pantallas y escenarios.

\begin{figure}[H] 
\centering 
\includegraphics[scale=0.5]{propuesta/grafo_escenas.png}
\caption{Navegación entre escenarios y pantallas. Los escenarios son los
    rectángulos con un borde dos rayas, y las pantallas tienen un borde con una
    sola raya.}
\label{fig:grafo_estados}
\end{figure}

La solución inicia con un escenario denominado \emph{Inicio}, en el cual se
permite al usuario observar los detalles del entorno simulado a la vez que
muestra las opciones que permiten iniciar las diferentes prácticas, compartir
su actividad, enviar los datos de utilización y finalmente salir de la
simulación.

Si el usuario selecciona en el \emph{inicio} la opción \emph{Extracción de
    sangre}, se inicia el escenario denominado \emph{Extracción de sangre}, en
el cual el usuario puede realizar el procedimiento de extracción de sangre, si
el usuario selecciona la opción \emph{Fin}, la simulación termina y se dirige a
el escenario \emph{Pantalla de resultados}.

Al seleccionar la opción \emph{Evaluación Glasgow}, se inicia el escenario
denominado \emph{Glasgow}, donde el usuario debe evaluar a un paciente en el
centro del escenario, si el usuario presiona la opción \emph{Fin} se inicia la
pantalla denominada \emph{Evaluar al paciente}, donde el usuario diagnostica el
estado del paciente, y finalmente al presionar el botón \emph{Fin}, la
simulación finaliza y se inicia el escenario \emph{Pantalla de resultados}.

La opción \emph{Exploración Glasgow} es similar, la diferencia es que antes de
iniciar el escenario \emph{Glasgow}, aparece la pantalla \emph{Elegir estado de
    paciente}, en el cual el usuario selecciona un estado para que el paciente
actué de acuerdo al mismo, luego se inicia la escena \emph{Glasgow} y si el
usuario presiona el botón \emph{Fin}, se inicia el escenario \emph{Pantalla de
    resultados}.

La pantalla de resultados muestra la información acerca de las acciones que
realizo el usuario, proveyendo información a modo de retroalimentación, en esta
pantalla el usuario puede compartir sus resultados por las redes sociales,
reiniciar el escenario y finalmente, poder volver a la \emph{Pantalla de
    inicio}.


\subsection{Inicio}

\subsubsection{Descripción del entorno}

La escena mostrada como pantalla de inicio de la aplicación muestra como fondo la sala de 
un hospital con los elementos típicos de estos lugares, esta es la que se utiliza como 
escenografía principal en los escenas de los procedimientos, haciendo que el usuario entre 
en ambiente. Además de este fondo, se muestras varias opciones en forma de botones que serán 
descriptas a continuación y un mensaje en donde se recomienda al usuario el uso de auriculares.

\subsubsection{Opciones}

Las opciones disponibles en la pantalla de inicio son presentadas en forma de
botones los cuales tienen una breve descripción que identifica la función que
cumplen. 

\todox{Agregar descripción del escenario y si es necesario pantalla donde se
    pone el número}

\begin{itemize}
\item Botón \enquote{Enviar Progreso}: esta función envía toda la información
    acerca de la actividad que el usuario realizo en la aplicación a un servidor
    backend que se encarga de almacenar estos datos.
\item Botón \enquote{Salir de la simulación}: esta función permite salir de la
    aplicación.
\item Botón \enquote{Facebook}: esta función permite al usuario ingresar a su
    cuenta de Facebook.
\item Botón \enquote{Extracción de sangre}: esta función permite ingresar a la
    escena correspondiente al procedimiento de extracción de muestras de sangre
    permitiendo al usuario jugar una nueva partida.
\item Botón \enquote{Explorar Glasgow}: esta función permite ingresar a la
    escena correspondiente al procedimiento para explorar las reacción de un
    paciente con un diagnostico especifico de la escala de Glasgow permitiendo
    al usuario jugar una nueva partida.
\item Botón \enquote{Evaluar Glasgow}: esta función permite ingresar a la escena
    correspondiente al procedimiento para la valoración y diagnostico de la
    escala de Glasgow para un paciente con estado aleatorio permitiendo al
    usuario jugar una nueva partida.
\end{itemize}


\subsection{Extracción de muestras de sangre}

A continuación se detallan cada una de las opciones y formas disponibles de
interactuar con la escena del procedimiento de extracción de muestras de sangre.

\subsubsection{Descripción del entorno}

Al seleccionar el procedimiento de extracción de sangre en la pantalla de inicio 
la aplicación inmediatamente muestra la escena del procedimiento, se muestra una 
sala de hospital igual a la de la pantalla de inicio pero con un paciente en una 
de las camas, a este paciente es a quien se le realizara el procedimiento.

La posición inicial de la cámara se ubica en un ángulo en donde se puedan ver 
bien los brazos del paciente para facilitar al usuario la realización del 
procedimiento.

\todox{Agregar descripción}

\subsubsection{Descripción de la interfaz}

La interfaz principal de este escenario posee dos menús, uno a cada lado de la
pantalla, las opciones son representadas como botones que poseen una imagen
intuitiva\todox{Ver si no hay que agregar esto como hipótesis} que representa la
función que realizan. 

\subsubsection{Entidades}

En la extracción de sangre existen dos entidades principales, el paciente y el
usuario, cada entidad mantiene un estado independiente de la otra entidad.

El paciente es una entidad con estado complejo, el cual es constantemente
modificado por las acciones del usuario, en resumen, la información que contiene
el estado del paciente es:

\begin{itemize}
    \item \textbf{Jeringas}: un paciente puede tener cero o más jeringas en
        cualquier momento, no se limita la cantidad de jeringas que puede
        insertar el usuario.
    \item \textbf{Manos}: almacena el estado de las manos, el paciente reacciona
        ante peticiones del usuario, puede abrir o cerrar cualquier mano en
        cualquier momento.
    \item \textbf{Torniquetes}: es el conjunto de torniquetes que tiene
        actualmente el paciente, notar que los torniquetes pueden ser colocados
        en cualquier parte del brazo, pero existen lugares \enquote{correctos} y
        lugares \enquote{incorrectos}, la diferencia consiste en la distancia a
        los puntos de extracción, estos lugares están predefinidos.
    \item \textbf{Zonas esterilizadas}: son aquellas áreas del cuerpo que el
        usuario esterilizó, no existe un límite para las zonas esterilizadas.
        Una vez que una jeringa es extraída, una zona esterilizada pasa a estar
        contaminada y a la espera de que el usuario la presione.
    \item \textbf{Zonas presionadas}: son aquellas zonas que, una vez
        contaminadas por la extracción de una jeringa, han sido presionadas por
        el usuario.
    \item \textbf{Contaminado}: define si alguna acción realizada por el usuario
        provoco que el paciente se contamine, existen varias cadenas de eventos
        que pueden provocar que esto ocurra:
        \begin{itemize}
            \item Inyección de una jeringa cuando existe otra inyectada.
            \item Inyección en un lugar en lugares inadecuados.
            \item Inyección en un lugar no esterilizado.
            \item Inyección en un brazo cuya mano este abierta.
            \item Inyección fuera del alcance de los torniquetes actuales.
            \item Interacción con el paciente sin que el mismo tenga la mano
                estéril.
        \end{itemize}
        Es importante notar que este estado no es afectado directamente por una
        acción del usuario, sino por la consecuencia de una acción.
\end{itemize}

El \emph{usuario o enfermero} mantiene un estado en todo momento del cual
dependen sus acciones, por ejemplo, si la mano del paciente no esta
esterilizada, cualquier interacción con el paciente provocara que el paciente se
contamine.

\begin{itemize}
    \item \textbf{Manos}: almacena la información acerca de la esterilidad de
        las manos.
    \item \textbf{Guantes, gorro, bata y tapaboca}: almacenan la información
        acerca de los equipamientos que tiene el usuario en un momento dado.
    \item \textbf{Elemento actual}: es el elemento que esta activo en
        cualquier momento, un elemento es una herramienta de la vida real,
        como por ejemplo un torniquete, una gaza.
\end{itemize}

\subsubsection{Acciones}


\paragraph{Comando de voz}

Para representar la interacción del usuario con el paciente usando la voz se
implemento un menú que es activado y mostrado en pantalla cuando el usuario
habla, este menú muestra una seria de ordenes que el usuario le haría al
paciente normalmente hablándole. Las opciones de menú se detalla a continuación:

\begin{itemize}
\item Explicar procedimiento: esta función sirve para detectar si el usuario
    realizo la acción de explicar al paciente acerca del procedimiento. 
\item Abrir la mano izquierda: esta función le indica al paciente que abra su
    mano izquierda, como resultado el paciente realiza esta acción.
\item Cerrar la mano izquierda: esta función le indica al paciente que cierre su
    mano izquierda, como resultado el paciente realiza esta acción.
\item Abrir la mano derecha: esta función le indica al paciente que abra su mano
    derecha, como resultado el paciente realiza esta acción.
\item Cerrar la mano derecha: esta función le indica al paciente que cierre su
    mano derecha, como resultado el paciente realiza esta acción.
\end{itemize}

\paragraph{Opciones}

En este menú se despliegan los botones que representan las opciones de
bioseguridad. Es decir, acciones como lavarse las manos, calzarse guantes,
ponerse gorro, ponerse bata y ponerse tapaboca.

Los elementos de bioseguridad que actualmente tiene puesto el usuario se
representan como se describió anteriormente y se muestran en la parte baja de la
pantalla. Desaparece quitarse que en ese caso se representa al volver a
seleccionar la misma opción.

\paragraph{Elementos}

En este menú se despliegan los botones que representan a lo elementos que se
utilizan para realizar el procedimiento, una vez presionado ese elemento queda
seleccionado. Solo un elemento puede ser seleccionado a la vez. Si el mismo
botón se vuelve a presionar inmediatamente después de haber sido presionado, el
elemento queda de-seleccionado.

%Estas opciones van cambiando el estado del jugador y pueden ser seleccionados
%mas de una opción a la vez además de permitir de-seleccionar una opción
%volviendo a tocar el botón correspondiente. También posee la opción de
%finalizar la partida la cual manda al usuario a la pantalla de resultados.
%% REVISAR ESTO , el comienzo es sobre opciones y el final sobre elementos %%

La herramienta seleccionada actualmente para realizar el procedimiento se se
muestra en la forma descripta anteriormente arriba de la pantalla principal del
procedimiento. Esta imagen representa lo que actualmente tiene en las manos el
jugador. Desaparece al de-seleccionar o terminar de usar la herramienta.

\todox{Agregar colocación}
\todox{Agregar utilización}

\subsubsection{Eventos}
\subsubsection{Motor de reglas}
\subsubsection{Registro de actividad}


\subsection{Valoración de la escala de Glasgow}

\subsubsection{Descripción del escenario}

La interfaz principal de este escenario posee un botón de finalización de
partida al costado con una imagen intuitiva que representa la función que
realiza. Este botón manda al usuario a la pantalla de resultados.

\subsubsection{Entidades}
\subsubsection{Acciones} 
\subsubsection{Eventos} 
\subsubsection{Pantalla de diagnostico}
\subsubsection{Registro de actividad}
\paragraph{Elementos y opciones}


\subsection{Pantalla de resultados}
\subsubsection{Descripción del escenario}
\subsubsection{Retroalimentacion}
\subsubsection{Gamificacion}
\subsubsection{Reinicio}
\subsubsection{Puntuación}
\subsubsection{Tiempo utilizado}
\subsubsection{Facebook 2}

\subsection{Partes de la simulación}
    \subsubsection{Entidades}
    \subsubsection{Eventos}
    \subsubsection{Acciones}
    \subsubsection{Interacción con la cámara}

\subsection{Grafo del desarrollo}
% podemos poner acá un gráfico mas o menos así (ver graphviz)
%           /---> Hemocultivo --\
%          /                     \              /-> Reiniciar
%  Inicio ------> Glasgow 1 -------> Resultados --> Inicio
%         \ \                    /              \-> Facebook 2
%          \ \--> Glasgow 2 ----/
%           \---> Salir 
%            \--> Facebook 1
%             \-> Enviar resultados

\subsection{Pantalla de inicio}
    \subsubsection{Descripción del escenario}
    \subsubsection{Enviar datos}
    \subsubsection{Glasgow}
    \subsubsection{Extracción de sangre}
    \subsubsection{Facebook 1}

\subsection{Extracción de sangre}
    \subsubsection{Descripción del escenario}
    \subsubsection{Descripción de la interfaz}
    \subsubsection{Entidades} % definimos cuales son las entidades
        %\subsubsubsection{Estado del enfermero}
        %\subsubsubsection{Objeto seleccionado}
    \subsubsection{Acciones} % definimos cuales son las acciones de esas entidades
        %\subsubsubsection{Comandos de voz}
        %\subsubsubsection{Opciones} %bata,mano,guante,y eso
        %\subsubsubsection{Elementos}
            %\subsubsubsubsection{Colocación}
            %\subsubsubsubsection{Utilización}
    \subsubsection{Eventos} % definimos cuales son los eventos que se lanzan en este proceso
    \subsubsection{Motor de reglas} % se define como funciona el motor de reglas acá
    \subsubsection{Registro de actividad} % se define como se registra las acciones del usuario (cuales)


\subsection{Glasgow 1 y 2}
    \subsubsection{Descripción del escenario}
    \subsubsection{Entidades} % definimos cuales son las entidades
        %\subsubsubsection{Reacciones del paciente}
    \subsubsection{Acciones} % definimos cuales son las acciones de esas entidades
        %\subsubsubsection{Acciones sobre el paciente}
        %\subsubsubsection{Comandos de voz}
    \subsubsection{Eventos} % definimos cuales son los eventos que se lanzan en este proceso
    \subsubsection{Pantalla de diagnostico}
    \subsubsection{Registro de actividad} % se define como se registra las acciones del usuario (cuales)
    
\subsection{Pantalla de resultados}
    \subsubsection{Descripción del escenario}
    \subsubsection{Retroalimentacion}
    \subsubsection{Gamificacion}
    \subsubsection{Reinicio}
    \subsubsection{Puntuación}
    \subsubsection{Tiempo utilizado}
    \subsubsection{Facebook 2}

%! TEX root = ../main.tex
\section{Evaluación en tiempo de ejecución}

Las acciones realizadas por los usuarios dentro de la aplicación son evaluadas
para determinar si realizo o no el procedimiento de manera correcta y así
brindarle información al usuario sobre su rendimiento.

En esta sección se explica como son evaluados las acciones de los usuarios para
los diferentes procedimientos simulados.

\subsection{Extracción de muestras de sangre}

Para la evaluación de las acciones del usuario en este procedimiento se utilizo
un motor de reglas denominado \enquote{Acciones condicionadas por eventos}. A
continuación se explica en detalle cada aspecto relacionado tanto al motor como
a la forma de evaluación del rendimiento del usuario.

\subsubsection{Acciones condicionadas por eventos}

Un evento es la ocurrencia de un hecho en particular, y son identificados por un
nombre y un conjunto de parámetros, por ejemplo, cuando un evento es cuando el
enfermero inserta una Jeringa, el nombre de este evento es
\enquote{jeringa}.inserted, y sus parámetros podrían ser el lugar y el tiempo
de la inserción, así, la influencia del estudiante en la simulación es una
sucesión de eventos.

Por cada acción que realiza el usuario dentro de la simulación, existe un evento
relacionado, por consiguiente, es razonable estudiar algunos eventos para
determinar si los pasos realizados corresponden con los deseados. 

Para determinar si una sucesión de eventos es la correcta, se definen reglas,
una regla es una asociación de una condición y una acción, la condición define
si el entorno es el adecuado para realizar una acción, la cual es un
procedimiento que realiza la lógica deseada.

Las \gls{eca} son aquellas que son activadas una vez que se cumplen determinados
eventos\cite{bailey2004event}. En las bases de datos relacionales, son conocidos
como triggers, es decir, una base de datos relacional (u orientada a objetos) es
un motor de reglas \gls{eca}\cite{bailey2004event}\cite{behrends2006combining}.

Las mismas pueden ser utilizadas para notificar que un determinado conjunto de
eventos ha ocurrido\cite{bailey2004event}, así como servir para almacenar
información acerca de la utilización de un determinado recurso.


\paragraph{Motivación}

Las reglas del tipo \gls{eca} permiten reaccionar a determinados eventos, en
forma de una única regla, la cual facilita la declaración de las
mismas\cite{bailey2004event}.

Son principalmente útiles para analizar el comportamiento en tiempo real de un
sistema en una forma
reactiva\cite{bailey2004event}\cite{de2001eca}\cite{bailey2002analysis}, esta
característica esta impulsada principalmente por que son ejecutadas después de
la ocurrencia de un evento, y el entorno no es modificado, pudiendo así acceder
al mismo entorno que el qué lanzo el evento.

Definir si las acciones de un usuario son correctas utilizando un motor
\gls{eca} es sencillo desde el punto de vista que sólo se deben definir un
conjunto de acciones que se deben realizar, y agregar una acción que verifica si
los pasos realizados fueron los correctos.

\paragraph{Declaración}

Una \gls{eca}, se define como\cite{bailey2004event}\cite{behrends2006combining}:

\begin{center}
	 Cuando ocurren una serie de \emph{eventos}, y se cumple una
	 \emph{condición}, entonces realizar una \emph{Acción}.
\end{center}

Los \emph{eventos} determinan cuando una regla debe ser activada, los mismos se
dividen en dos categorías\cite{behrends2006combining}, primitivos y compuestos,
los primeros son detectables, por ejemplo, cuando se inserta una jeringa, y los
compuestos, son la combinación de uno o más
primitivos\cite{bailey2004event}\cite{behrends2006combining}. Los eventos
compuestos, se unen mediante:
\begin{enumerate*}[label=\itshape\alph*\upshape)]
\item conjunción (\emph{y}),
\item disyunción (\emph{o}), y
\item secuencia (\emph{entonces}).
\end{enumerate*}
Sin embargo, no siempre son necesarios todas las posibles combinaciones, y las
combinaciones sencillas son más fáciles de optimizar y
probar\cite{bailey2004event}.

La \emph{condición} de una regla determina si el entorno es el necesario para que la
regla sea activada, en esta condición el entorno que lanzó el evento esta
disponible.

La \emph{acción} a ejecutar describe la lógica que debe ser ejecutada cuando se han
lanzado los eventos y la condición de la regla se ha cumplido.

\paragraph{Dependencia entre reglas}

Las reglas pueden depender de otras reglas, lo cual se puede ver como que la
finalización de una regla es un evento que otra regla espera para poder ser
activada.

Las reglas pueden agregar información a un contexto compartido por todas las
reglas, de esta manera, se puede pasar parámetros entre distintas reglas, por
ejemplo, la regla \emph{Retirar Torniquete}, depende de la regla \emph{Insertar
Torniquete}, pero debe responder solamente al torniquete que ha activado
la regla de inserción, es decir, el usuario puede extraer varios torniquetes, y
la regla no debe activarse, hasta que se extraiga el torniquete que activo la
primer regla.

Así, la regla \emph{Retirar Torniquete} depende de la regla \emph{Insertar
Torniquete}, y esta relación entre reglas, se da en dos
formas\cite{bailey2004event}:

\begin{itemize}
\item  \emph{Dependencia fuerte:} la regla \emph{Retirar Torniquete} solamente podrá
	ser elegida para ser lanzada cuando la regla \emph{Insertar Torniquete}
	haya sido cumplida.
\item  \emph{Dependencia de contexto}: la regla \emph{Retirar Torniquete} no se
	activará cuando los eventos a los que escucha se terminen, sino cuando
	los eventos a los que escucha sean lanzados con los parámetros adecuados
	(se extraiga el torniquete que lanzo la regla de inserción).
\end{itemize}

\paragraph{Representación}

La definición de las reglas se realiza de la siguiente forma;
\begin{algorithm}[H]
\caption{Creación de regla de verificación de calzado de guantes}
\label{alg:rule:guante}
\lstset{style=sharpc}
\begin{lstlisting}
Rule.New("Regla de verificacion de calzado de guantes").
     When("enfermero.guantes.calzar").
     Then(e => e.Patient.ManosLimpias()).
\end{lstlisting}
\end{algorithm}
%TODO agregar indice de algoritmos

La regla anterior controla que el estudiante ha realizado la acción ``Calzarse
los guantes'', y en ese momento tenga las manos limpias, la variable \emph{e},
es el entorno, y a través de la propiedad \emph{Patient} obtiene el estado del
paciente en ese momento.

\paragraph{Modelo de ejecución}

Para ejecutar un motor de reglas del tipo \gls{eca}, se debe tener en cuenta
principalmente dos factores, 
\begin{enumerate*}[label=\itshape\alph*\upshape)]
\item  Como se verifica el cumplimiento de cada regla, y, 
\item  Que ocurre cuando varias reglas son lanzadas al mismo tiempo
\end{enumerate*}.

Para ambos casos se puede tomar un enfoque \emph{inmediato}, es decir que
inmediatamente cuando se lanza un evento, o se cumple una condición, se ejecuta
la regla. Además existen otros dos modos de ejecución, \emph{deferida}, y
\emph{desacoplada}, en la primera, se espera hasta que el lanzador del evento
culmine su trabajo, y luego se ejecuta la regla, pero en la misma unidad de
trabajo, mientras que en la ejecución desacoplada, se encolan los trabajos y
otro hilo es el encargado de ejecutar las reglas. Estos modos están inspirados
en las bases de datos relacionales, el deferido se ejecuta en la misma
transacción, y el desacoplado, inmediatamente después de que la transacción
termine\cite{bailey2004event}.

La propuesta implementada, utiliza una ejecución inmediata, principalmente por
la sencillez de las reglas, es decir, las reglas no realizar un proceso complejo,
solamente controlan el estado del entorno y lo validan.

Además, la ejecución inmediata es importante por que el entorno no sufre
modificaciones entre el evento lanzado y la ejecución de la regla, según
\cite{bailey2004event}, este es el factor más importante para determinar el tipo
de ejecución deseado.



\paragraph{Estados de una regla}

Una regla puede estar en uno de los siguientes estados:

\begin{description}
\item[BEGIN] Es una regla que recién fue creada, no realiza ninguna
	acción.
\item[WAITING\_FOR\_RULE] Es un estado en el que esta esperando que otras reglas
	sean lanzadas. En este estado, es un suscriptor de las reglas por la que
	espera, y no forma parte del ciclo de ejecución del motor de reglas.
\item[WAITING\_FOR\_EVENT] Es un estado en el que esta escuchando a que sean
	lanzados los eventos a los que escucha, este es el estado principal. En
	este estado, es un suscriptor de los eventos por los que espera, y no
	forma parte del ciclo de ejecución del motor de reglas. Se diferencia
	del estado anterior, en que los eventos escuchados pueden ser lanzados
	por cualquier objeto del entorno, no necesariamente una regla.
\item[WAITING\_FOR\_CONDITION] La regla ya no espera por ningún evento y las
	reglas de las que depende ya han sido lanzadas, se verifica cada cierto
	tiempo si el entorno cumple con una condición definida. 
\item[FINISH] La regla ha sido lanzada, con un resultado no determinado, se pudo
	haber cumplido, como no, es el estado final de una regla. Cuando una
	regla llega a este estado, se lanza su evento de finalización.
\end{description}

Una regla puede estar en solo un estado, y solamente se permite que el estado
avance, desde \emph{BEGIN} hasta \emph{FINISH}.


\paragraph{Ciclo de vida}

Cuando una regla es definida, y insertada al motor de reglas, inmediatamente
pasa al estado \emph{BEGIN}, luego se verifica si la misma depende de otras
reglas, sí este es el caso, pasa al estado \emph{WAITING\_FOR\_RULE} y escucha a
los eventos de finalización de las reglas anteriores.

Una vez que las reglas anteriores han sido finalizadas, la regla pasa al estado
\emph{WAITING\_FOR\_EVENT} sí deben escuchar por algún evento, en caso contrario
pasan al estado \emph{WAITING\_FOR\_CONDITION}.

Una vez que la regla está en estado \emph{WAITING\_FOR\_CONDITION}, pasa a un
motor que ejecuta su condición cada cierto tiempo, si la condición se cumple, la
regla se ejecuta, y la misma pasa a estado \emph{FINISH}, momento en el cual
notifica a las reglas que dependen de ella que ha sido lanzada.

Una vez que la regla esta en estado \emph{FINISH}, la misma sale del esquema de
ejecución, y solo esta disponible para obtener resultados.

Según el ejemplo de la regla definida en el código\ref{alg:rule:guante}, la
regla al terminar de ser construida pasa a estado \emph{BEGIN}, al no depender
de otras reglas, pasa inmediatamente al estado \emph{WAITING\_FOR\_EVENT},
cuando es lanzado el evento, la regla ejecuta la acción y pasa al estado
\emph{FINISH}.

\paragraph{Motor de ejecución}

Un motor de reglas \gls{eca}, requiere de un proceso que evalúe constantemente
las reglas para verificar si las mismas deben ser lanzadas o
no\cite{bailey2004event}\cite{galton2002two}, este motor puede utilizar el
algoritmo de RETE\cite{de2001eca} para realizar esta verificación, en la
propuesta presentada, la cantidad de reglas definidas, y la no dependencia
circular entre ellas, hace innecesario la implementación de tal
algoritmo\cite{de2001eca}. 

El motor de reglas actúa sobre aquellas reglas en estado
\emph{WAITING\_FOR\_CONDITION} e invoca al procedimiento que se encarga de
validar si la regla puede ser activada (el procedimiento es único por cada
regla), si el mismo determina que la regla puede ser lanzada, el motor ejecuta
la acción de la regla y modifica el estado de la regla a \emph{FINISH}.


\subsubsection{Definición de reglas}

La reglas del procedimiento de extracción de sangre fueron definidas de acuerdo
a los pasos requeridos según el protocolo del procedimiento y al orden en el que
son requeridos. Es decir, cada paso del protocolo tiene asociado una regla
dentro del motor que lo representa y las condiciones asociadas a cada regla
están determinadas por el orden en que deben realizarse dentro del protocolo.

Cada regla tiene una o mas condiciones que deben ser cumplidas para que un paso
del protocolo realizado se considere correcto.

\subsubsection{Retroalimentación y puntuación final}
\label{sec:puntuacion_hemocultivo}

Cada regla tiene asociado un peso, de acuerdo a la dificultad de realizar el
paso, este peso es utilizado al final de la partida para darle una puntuación al
usuario acerca de su rendimiento en la partida.

Además, un regla puede quedar en uno de diferentes estados al final de la
partida como se mostró anteriormente, cada uno de esos estados posee un
significado en el contexto del procedimiento y por lo tanto tiene información
asociada para que al final de la partida se muestre una retroalimentación
correcta al usuario por paso.

\subsection{Valoración de la escala de Glasgow}
\label{sec:puntuacion_glasgow}

Para la evaluación del rendimiento del usuario en el momento de llevar a cabo el
procedimiento de valoración de la escala de Glasgow se tuvo un enfoque
completamente diferente al del procedimiento de extracción de muestras de sangre
debido a la naturaleza propia del procedimiento. 

Como se explico anteriormente, el paciente puede estar en ciertos estados
específicos dentro de la escala, y además dentro de cada estado reacciona de un
forma en particular por lo tanto, al inicio de la partida un componente interno
de la aplicación selecciona de forma aleatoria un estado para el paciente, de
forma tal que cada vez que una partida sea jugada no se repitan los estados de
forma seguida.

El estado aleatorio del paciente es guardado en una variable que no es
modificada hasta que se reinicie la partida. Al final de la partida, la
aplicación pide al usuario que valore el estado del paciente que le fue
presentado, una vez que el usuario confirme su respuesta la aplicación la
compara con el estado guardado y de esta forma puede informar al usuario acerca
de su rendimiento en el diagnostico.

Además, cada posible respuesta dada por el usuario contiene información
relacionada al contexto del procedimiento y a la situación actual presentada la
cual es utilizada como retroalimentación al final de la partida. La puntuación
final dada depende de la cantidad de valoraciones correctas dadas por el usuario
para la respuesta verbal, motora, ocular y nivel de gravedad del paciente.










\section{Inconvenientes de diseño}

Los mayores inconvenientes de diseño de la aplicación se dieron en el momento de
validar tanto el contenido de la aplicación como la interfaz de usuario, para
sobrellevar estos inconvenientes fueron requeridos la intervención de terceros.

A continuación se explica en detalle cada uno.

\subsection{Interfaz de usuario}

Como parte del diseño y desarrollo de la aplicación propuesta como solución se
realizó una prueba de interfaz de usuario con alumnos de la carrera de
ingeniería en informática de la Facultad Politécnica de la Universidad Nacional
de Asunción, estas pruebas fueron realizadas con personas que están
acostumbradas al uso de interfaces similares y que de hecho pueden ser mas
criticas a la hora de evaluarlas. Esta prueba se explica en detalle en el
capítulo~\ref{chap:evaluacion} y los resultados en el capítulo~\ref{chap:analisis}.

Principalmente son dos las cualidades de una interfaz gráfica que se pueden
someter a prueba: su funcionalidad y su usabilidad. Con la primera se pretende
responder preguntas como ¿Se puede usar cierta función?, ¿Funciona como se
espera?, o ¿Es correcta?; y con la segunda, a ¿Puede el usuario utilizar
fácilmente la función?, o ¿Su uso es intuitivo y fácil de
aprender?\cite{fragaverificacion}.

Las pruebas de interfaces de usuario ayudan a que los usuarios puedan
concentrarse mas en el problema en vez de poner los esfuerzos en recordar todas
las opciones que ofrece la aplicación que se utiliza para resolver el
problema\cite{horowitz1993graphical}.

Luego de las pruebas con usuarios con experiencia en interfaces móviles, se
hicieron correcciones a los problemas encontrados en la interfaz, los mayores
inconvenientes fueron con respecto a la usabilidad y la interacción tanto con el
entorno como con los objetos dentro de la simulación Estas correcciones, como
paso posterior, fueron probadas por profesores de la carrera de enfermería del
Instituto Andrés Barbero los cuales dieron su visto bueno.

Otra de las razones por las cual la prueba fue realizada con alumnos que no
formaban parte de la población a la que iba dirigida la aplicación, es la poco
disponibilidad de tiempo con la que cuentan los alumnos de enfermería y mas aun
los profesionales que están encargados de su aprendizaje.

\subsection{Validaciones de contenido}

Llamamos validación de la simulación o la aplicación desarrollada al hecho de
que el contenido de la misma sea correcto y además que la forma de realizar o
representar dicho procedimiento este acorde al mismo. Este tipo de validaciones
fueron realizadas reiteradamente en reuniones con distintos profesores de la
carrera de enfermería del Instituto Andrés Barbero.

Cada corrección solicitada fue evaluada y aprobada posteriormente por los
mismos. Como validación final la aplicación fue presentada en totalidad frente a
un plenario de cuatro profesores del instituto.

El mayor inconveniente en cuanto a las validaciones fueron la forma de
representación tanto de la información como de la simulación de objetos.


%! TEX root = ../main.tex

\chapter{Evaluación}
\label{chap:evaluacion}


Este capitulo define los mecanismos utilizados para evaluar la solución
propuesta, los mismos están orientados a la validación de las hipótesis
planteadas durante el desarrollo de la solución, lo que incluye aspectos
pedagógicos, de utilidad y de la participación activa del usuario entre otros
descriptos más adelante. Como parte de la evaluación se miden ciertas variables
relacionadas a los aspectos mencionados.

La evaluación se divide en cuatro partes principales:

\begin{description}
    \item[Encuesta de ubicación] Es una encuesta acerca del nivel de acceso a la
        tecnología que poseen los alumnos del 4to año del \Gls{iab}, de ahora en
        más \textit{el Universo}, esta encuesta sirve para definir la muestra.

    \item[Encuesta Subjetiva] Es una encuesta realizada a cada sujeto que
        participa de la solución, donde se busca la opinión del mismo acerca de
        la solución y factores relacionados a la misma. 

    \item[Encuesta Objetiva] Es un cuestionario que es completado por el
        universo de alumnos, donde se mide el conocimiento de los mismos, se
        utilizan a los alumnos que no son la muestra, como grupo de control.

    \item[Registro] Es información almacenada por la solución automáticamente,
        que contiene datos acerca de su utilización y el desempeño del alumno.
\end{description}


Adicionalmente se realiza una evaluación inicial para medir la calidad de la
interfaz y la interacción con la misma, esta evaluación es realizada con
personas no relacionadas al área de enfermería.

El capitulo define los objetivos de la evaluación, describe brevemente conceptos
transversales a las técnicas utilizadas y luego define las metodologías,
métricas y variables utilizadas en cada experimento.

\section{Objetivos}
\label{sec:objetivos}

La obtención de los registros de actividad y examenes buscan obtener información sobre el
aprendizaje y la utilización de la solución, mientras que la encuesta de
satisfacción busca obtener información acerca de las fortalezas y debilidades de
una simulación para el entrenamiento de enfermeros y de la solución propuesta.

Se definen los objetivos principales de la evaluación como sigue:

\begin{itemize}
\item Validar las hipótesis asumidas durante el desarrollo de la solución.
\item 
\end{itemize}







%Ideas tipo tiro al aire de La princesa de cocho por si sirvan
%Para la evaluacion acerca de la validez de las hipotesis planteadas en este trabajo se hacen uso de metodologias como registros de actividades de los usuarios cuando utilizan la aplicacion y encuestas para valorar la opinion de los mismos sobre las caracteristicas de la aplicacion.

%Los objetivos principales de la evaluacion son los siguientes:

%* Verificar la validez de las hipotesis planteadas en el desarrollo de la solucion.
%* Proponer criterios que puedan utilizarse para la evaluacion de aplicaciones de esta naturaleza.
%* Obtener conclusiones acerca de factores externos que afectan el uso de la aplicacion.
%* Identificar los puntos importantes en los que se debe poner enfasis en el desarrollo de las aplicaciones con esta naturaleza.
%* Obtener sugerencias de las correlaciones entre el registro de actividades y el examenen de conocimientos realizado a los usuarios de la aplicacion.
%* Obtener conclusiones generales acerca del uso de la aplicacion como herramienta de apoyo al aprendizaje de estudiantes de enfermeria.










%! TEX root = ../main.tex
%! TEX root = ../main.tex

\section{Métricas generales}

Existen métricas que son usadas por más de un experimento\revisar{Ver el termino
    correcto}, a continuación se describen estas métricas:

\subsection{Escala de Likert}
\label{sec:likert}

Para la valoración de las variables medidas se utiliza la escala de
Likert\cite{Allen:2007} de 7 valores posibles. La escala de Likert es utilizada
para permitir a las personas indicar cuánto están de acuerdo o en desacuerdo con
respecto a ciertos puntos. Los valores utilizados, son:

\begin{enumerate}
    \item Totalmente en desacuerdo
    \item En desacuerdo
    \item Parcialmente en desacuerdo
    \item Neutral
    \item Parcialmente de acuerdo
    \item De acuerdo
    \item Totalmente de acuerdo
\end{enumerate}

Una vez valoradas y registradas todas las respuestas y con el objetivo de
eliminar las tendencias en la forma en la que son completadas las
encuestas\cite{Fischer2010} se utiliza el método de Doble Estandarización
recomendado en~\cite{Pagolu2011}. Este método, consiste en dos
estandarizaciones, la primera por fila, que en este caso representa a los
individuos y la segunda por columna donde cada columna representa una de las
diferentes preguntas de la encuesta.

Siendo:
\begin{itemize}
	\item $\min_i$ la respuesta de menor valor del usuario $i$.
	\item $\max_i$ la respuesta de mayor valor del usuario $i$.
\end{itemize}

Para cada respuesta $s$ del usuario $i$, el valor ajustado, por la primera 
normalización, $s_1$ se define como:

\begin{equation*}
s_1{_i}=\frac{s-\min_i}{\max_i-\min_i}
\end{equation*}

Y luego siendo:
\begin{itemize}
	\item $groupmin_i$ la respuesta ajustada de menor valor en el grupo $i$.
	\item $groupmax_i$ la respuesta ajustada de mayor valor en el grupo $i$
\end{itemize}

Para cada respuesta ajustada $s_1{_i}$ del usuario $i$, el valor ajustado $sa_i$ se
define como:	

\begin{equation*}
sa_i=\frac{s_{1_i}-groupmin_i}{groupmax_i-groupmin_i}
\end{equation*}

Obteniendo así un valor normalizado, tanto por individuo, como por pregunta, en
el rango $0$ y $1$.

Para la valoración absoluta de cada  item se utiliza la media de cada columna o
respuesta a una pregunta de la encuesta.

Siendo:
\begin{itemize} 
\item $r_{k_i}$ la respuesta del usuario $i$ a la pregunta $k$.
\item $t_k$ la cantidad total de usuarios que respondieron la pregunta $k$.
\end{itemize}

El puntaje promedio de cada pregunta o item evaluado  $p_k$ en la encuesta se
define como:

\begin{equation*}
p_k = \frac{\sum_{i=1}^n{r_{k_i}}}{t_k}
\end{equation*}

\subsubsection{Manejo de información faltante}
\label{sec:informacion_faltante}

\observacion{Falta mejorar}
En toda encuesta pueden existir preguntas que no sean respondidas, y existen
tres posibles formas de categorizar el patrón de ocurrencia de la falta de
respuestas\cite{leite2010performance,tsikriktsis2005review}:

\begin{description}
    \item[Información faltante completamente aleatoria] Cuando la información
        faltante es independiente de la variable medida y de otras variables.
    \item[Información faltante aleatoria] Cuando la información faltante depende
        de otras variables, pero no de la variable en sí. 
    \item[Información faltante no aleatoria] Cuando hay una relación entre la
        información faltante y el valor de la variable.
\end{description}

Existen tres mecanismos\revisar{No repitan}\cite{tsikriktsis2005review}
principales para lidiar con información faltante, eliminación, reemplazo, y
procedimientos basados en modelo.~\cite{tsikriktsis2005review} recomienda
utilizar un mecanismo de reemplazo para escalas del tipo Likert.

Las técnicas de reemplazo se clasifican en tres grandes
grupos\cite{tsikriktsis2005review}:
\begin{enumerate*}[label=\itshape\alph*\upshape.]
\item basadas en el promedio,
\item basadas en regresión, y,
\item imputación \emph{hot deck}.
\end{enumerate*}

La sustitución basada por promedio, se divide nuevamente en tres
grupos\cite{tsikriktsis2005review}; promedio
\begin{enumerate*}[label=\itshape\alph*\upshape.]
\item total,
\item del subgrupo, y,
\item por caso.
\end{enumerate*}

La sustitución del promedio total se realiza obteniendo el promedio de todas las
respuestas de esta pregunta, la sustitución de subgrupo es similar, solo que se
limita a aquellos sujetos del mismo subgrupo del sujeto que no respondió, y
finalmente, la sustitución por caso, es el promedio de las respuestas válidas
del sujeto.

\subsection{Correlación de variables aleatorias}
\label{sec:correlacion}

\observacion{Falta un mini parrafo que explique en forma general que es la
    correlación y después mencionar a pearson}

La correlación de Pearson\cite{BoslaughStatistics2008} mide la relación que
existe entre dos variables, $X$ e $Y$, el mismo esta comprendido entre $-1$ y
$1$, en su punto más bajo ($-1$) indica una de las dos variables crece mientras
la otra decrece, y en su punto más alto ($1$), indica que ambas crecen o
decrecen conjuntamente, el valor $0$, indica que no existe una relación entre
ambas variables.

\replantear{\cite{norman2010likert} menciona que la misma puede ser utilizada
    para variables medidas con la escala de Likert, aún cuando la misma es
    utilizada normalmente para variables cuantitativas.}


%! TEX root = ../main.tex

\section{Encuesta de ubicación}
\label{sec:ubicacion}

A fin de obtener información acerca del nivel de acceso  de los alumnos a la
tecnología, se realiza una encuesta que cuenta con diez preguntas, las cuales
buscan obtener información acerca del modelo de dispositivo móvil, el acceso a
Internet, y la predisposición de cada alumno a ayudar en el experimento.

El \Gls{iab} contó\martin{En que tiempo debe ir? El resto esta en presente.} en
el 2014 con 124 alumnos en el cuarto año distribuidos en tres secciones, el cual es considerado
el Universo. De los 124 alumnos, 93 de ellos estuvieron dispuestos a participar de la prueba
y completaron la encuesta.

Se agrupan a los alumnos en diferentes grupos para determinar si sus
dispositivos celulares son capaces de ejecutar la solución de manera fluida, los
requisitos mínimos para garantizar esta experiencia son:

\begin{itemize}
    \item Sistema Operativo Android 4.0 o superior\todox{Explicar por que
            android}
    \item Memoria ram de 512MB o superior.
    \item Velocidad de procesador de 1 GHz o superior.
    \item GPU \todox{No se como poner la GPU, hay demasiada variedad}
\end{itemize}

Con los resultados de la encuesta de ubicación tecnológica, se seleccionan
aquellos alumnos que posean dispositivos móviles que superan o igualan las
especificaciones, se seleccionan un total de 19 estudiantes.

\martin{Hace falta más detalles? Una sección de métricas y variables, o es
    suficiente con mencionar los criterios mínimos de selección?}

\section{Encuesta objetiva}
\label{sec:objetiva}

A fin de obtener información comparativa acerca del conocimiento de los alumnos
que utilizaron la solución propuesta y los que no la utilizaban, utilizados
como un grupo de control, se realizo un examen, que consta de diez preguntas.

El examen busca medir el nivel de conocimiento del alumno sobre los dos temas
simulados, contiene preguntas de nivel básico, medio y avanzado.

Las preguntas fueron formuladas utilizando la lista de competencias básicas que
debe tener un alumno para aprobar la materia \textbf{Enfermería en Urgencias
    II}, y posteriormente fueron aprobadas por los profesores de la cátedra.

Cada pregunta tiene el mismo peso, así la puntuación más baja obtenible es 0, y
la más alta es 10.

\section{Muestra}

El universo cuenta con 124 alumnos, de los cuales 11 son la muestra seleccionada
para el experimento, entonces se utilizan a los 113 alumnos restantes como un 
grupo de control.

\section{Encuesta subjetiva}
\label{sec:subjetiva}

Al final del periodo de prueba, cada alumno de la muestra completa una encuesta
con 31 preguntas que se utilizan para validar las hipótesis. Las preguntas están
agrupadas en dos, el primer grupo cuenta con 27 preguntas cerradas, es decir de
una sola respuesta en una lista de opciones, el segundo grupo cuenta con 4
preguntas abiertas. 

Cada encuesta es entregada a los alumnos que acordaron participar en el
experimento, mientras completan la encuesta, un guía está presente para
responder cualquier duda.

La métrica utilizada en las preguntas cerradas es la escala de Likert, descrita
en la sección~\ref{sec:likert}.

\subsection{Variables}
\label{sec:variables}

De acuerdo a los objetivos planteados en la sección~\ref{sec:objetivos}, se
busca describir los factores analizados en las pruebas y las variables
relacionadas a los mismos, las cuales, tienen por objetivo demostrar la validez
de las hipótesis planteadas en este trabajo.

Las variables se presentaran agrupadas en factores, los mismos representan
aquellos aspectos de la solución propuesta que buscan ser evaluados.

\subsubsection{Exploración}
\label{sec:sub_exploracion}

Este factor esta relacionado con la característica que posee la solución en
cuanto a la oportunidad que brinda al usuario para explorar cada uno de los
elementos del entorno simulado (paciente, herramientas propias del
procedimiento). En este sentido, se busca proveer facilidad de uso, intuitividad
y realismo en cuanto a las acciones y situaciones que se presentan en la
solución para que de esta manera, los elementos que la componen no representen
para el jugador un obstáculo que impida su uso.

Las variables que miden este aspecto son las siguientes:

\begin{description}

\item[Funciones realizadas por los elementos del juego] se refiere a la
    correctitud con la que una herramienta o elemento del juego representa las
    funciones que el mismo puede realizar en la vida real, en este sentido, se
    evalúa el realismo con el que es representado tal elemento.

\item[Aleatoriedad para afianzar conocimientos] se refiere al beneficio que
    puede traer el hecho de que el estado del paciente en el juego sea aleatorio
    en cuanto a la posibilidad que esto brinda al jugador para poner a prueba
    sus conocimientos teóricos.

\item[Aleatoriedad para representar realismo] se refiere al uso de estados
    aleatorios en el paciente para que de esta forma el procedimiento se asemeje
    mas a una situación real.

\item[Facilidad de uso] se refiere a lo fácil e intuitivo  que puede ser la
    utilización de los elementos del juego.

\end{description}

\subsubsection{Representación}
\label{sec:sub_representacion}

Este factor esta relacionado con la calidad y suficiencia con la que se
representan los diferentes objetos que son simulados en la solución. La
representación abarca tanto funcionalidad como aspecto del objeto.

De esta manera, se busca permitir al jugador realizar con los objetos las
acciones que requiera para llevar a cabo el procedimiento que se le presente en
la solución, y además, representar estos elementos de la mejor manera posible,
de forma realista.

Las variables que miden estos aspectos son las siguientes:

\begin{description}

\item [Movimientos motrices del paciente] se refiere a la suficiencia de los
    movimientos motrices que realiza el paciente en el escena correspondiente a
    la valoración de la escala de Glasgow.

\item [Movimientos oculares del paciente] se refiere a la suficiencia de los
    movimientos oculares que realiza el paciente en la escena correspondiente a
    la valoración del escala de Glasgow.

\item [Reacción verbal del paciente] se refiere a la suficiencia de las
    reacciones o respuestas verbales que realiza el paciente en la escena
    correspondiente a la valoración de la escala de Glasgow.

\item[Distinción entre los estados del paciente] se refiere a si los diferentes
    estados del paciente son distinguidos correctamente en el procedimiento de
    valoración de la Escala de Glasgow ya que esto es importante para que el
    jugador pueda diagnosticar correctamente al paciente.

\item[Acciones las herramientas] se refiere a si las diferentes acciones que
    pueden realizarse con los elementos o herramientas del juego en un
    determinado procedimiento de enfermería son suficientes para ese
    procedimiento, ya que, debido a las limitaciones de la tecnología estas
    acciones son limitadas.

\end{description}

\subsubsection{Gamificación}
\label{sec:sub_gamificacion}

Este factor esta relacionado con la importancia de incluir en la solución
aquellas características que son propias de un juego de vídeo convencional. Se
busca conocer el valor de estas características en cuanto a la motivación que
puedan producir en los jugadores tanto para volver a utilizar la solución como
para superarse en cada juego.

Las variables que miden estos aspectos son las siguientes:

\begin{description}

\item[Motivación del puntaje] se refiere a que tanto motiva al jugador que la
    solución le proporcione un puntaje total al final de cada partida para poder
    mejorar constantemente siendo este puntaje como una evaluación final de todo
    lo que realizo dentro de la partida.

\item[Importancia del puntaje] se refiere a que tan importante es para un
    jugador que se le proporcione un puntaje total al final de cada partida para
    poder visualizar su rendimiento.

\item[Socialización de los puntajes] se refiere a si el hecho de que las
    personas del mismo entorno compartan sus puntajes, experiencias y logros en
    las partidas a través de redes sociales pueda ser motivador.

\item[Medición del tiempo como motivación] se refiere a que tanto motiva al
    jugador que la solución le proporcione el tiempo que duro su partida
    sirviendo este tiempo como una evaluación de su precisión a la hora de
    realizar el procedimiento que se le presente.

\end{description} 


\subsubsection{Inmersión}
\label{sec:sub_inmersion}

Este factor esta relacionado con el sentimiento de formar parte de la escena. Es
decir, se trata de evaluar que tanto un jugador puede sentir que realmente se
encuentra dentro del juego para que de este modo el pueda entrar en ambiente
para realizar los procedimientos que se le presenten en sus partidas de juego.

Las variables que miden este aspecto son las siguientes:

\begin{description}

\item[Escenografía para entrar en ambiente] se refiere a la importancia de la
    escenografía de la partida para que el jugador entre en ambiente para
    realizar el procedimiento que se le presente.

\item[Juegos cortos como ayuda para la repetición] se refiere a si el hecho de
    que los procedimientos presentados en las partidas sean cortos contribuye a
    repetir las partidas varias veces de seguido entrando en un estado de
    inmersión.

\item[Gráficos en tres dimensiones para entender el entorno] se refiere a la
    importancia que tiene el uso de gráficos en tres dimensiones para que el
    jugador pueda entender mejor el entorno y las posibles acciones que puede
    realizar.

\item[Realismo a través de ordenes verbales] se refiere a si el hecho de que la
    solución brinde la posibilidad de que aparezca un menú de ordenes verbales
    en el momento en que el jugador habla hace que la acción de dar ordenes
    verbales se asemeje mas a la realidad.

\item[Simulación como herramienta] se refiere a si la simulación ayuda al
    jugador a sentirse parte del laboratorio, dando cierto realismo a la escena
    que se le presenta.

\end{description}

\subsubsection{Utilidad}
\label{sec:sub_utilidad}

Este factor esta relacionado con lo útil que puede ser la solución como
herramienta de apoyo al proceso de aprendizaje de los estudiantes de enfermería.

Las variables que miden este aspecto son las siguientes:

\begin{description}

\item[Simulación para complementar el estudio en clase y laboratorio] se
    refiere a que tanto las herramientas alternativas como la simulación pueden
    complementar a los métodos de aprendizaje tradicionales que son el estudio
    en clase y en el laboratorio.

\item[Simulación provee más facilidades para el estudio] se refiere a si las
    herramientas alternativas como la solución proveen más facilidades para
    poner en practica los conocimientos con respecto a los demás métodos de
    aprendizaje que son los libros, laboratorios y el campo de practicas.

\item[Interacción con el paciente] se refiere a si el hecho de que el jugador
    pueda interactuar con un paciente que responde a las acciones del jugador es
    mejor que utilizar un maniquí inmóvil como el de los laboratorios de
    practica.

\end{description}

\subsubsection{Retroalimentación}
\label{sec:sub_retroalimentacion}

Este factor esta relacionado con la importancia de ofrecer al jugador
información acerca de sus logros y errores de manera tal que el pueda estar
consciente de sus puntos fuertes y sus puntos débiles en los diversos
procedimientos que realice en la solución.

Las variables que miden este aspecto son las siguientes:

\begin{description}

\item[Detalles de los pasos realizados incorrectamente] se refiere a que tan
    importante es para el jugador que la solución no solo le diga los pasos que
    hizo de manera incorrecta sino también las causas por las cuales no los
    realizo correctamente.

\item[Suficiencia de los detalles de los pasos realizados incorrectamente] se
    refiere a sí son suficientes las justificaciones breves acerca de las causas
    por las cuales que realizo incorrectamente un paso.

\item[Iconos para representar el estado del jugador] se refiere a la
    suficiencia de mostrar iconos en la interfaz de la solución para
    representar el estado actual del jugador.

\end{description}

\subsubsection{Pedagogía}
\label{sec:sub_pedagogia}

Este factor esta relacionado a la utilidad y al beneficio que puede traer la
solución para apoyar el aprendizaje del jugador. De esta manera, se busca
obtener la validez real de este tipo de herramientas como aporte al aprendizaje,
proveyendo mas interacción al jugador.

Las variables que miden este aspecto son las siguientes:

\begin{description}

\item[La solución para memorizar y comprender el procedimiento] se refiere a
    que tanto ayuda la solución al jugador para entender los procedimientos que se
    le presenten y para memorizar los pasos de cada uno de ellos.

\item[Falta de pistas como ayuda al aprendizaje] se refiere a que tan efectivo
    resulta no dar pistas al jugador en el momento de realizar un procedimiento
    para que pueda plasmar y medir sus conocimientos.

\item[Suficiencia de los botones que indican acciones] se refiere a que tan
    suficiente es representar determinadas acciones  con un botón debido a
    limitaciones en la tecnología.

\end{description}

\section{Registro de actividades}

La solución propuesta almacena información relacionada a la actividad del
usuario, incluyendo cuando y como utiliza las opciones, los pasos que realiza,
el orden y las condiciones de la escena cuando realiza cada acción.

El registro como un todo es enviado cada vez que el usuario desee, este envío
requiere una conexión a internet por ello no es automático. Adicionalmente el
último día de la prueba, todos los registros fueron enviados para que sean
analizados.


\todox{Agregar metricas y variables}

\section{Interfaz de usuario}

La primera fue realizada con alumnos de la carrera de Ingeniería en Informática
de la Facultad Politécnica que pertenece a la Universidad Nacional de Asunción,
sin experiencia previa tanto con la solución como con los procedimientos
simulados, pero si familizarizados con la utilización de dispositivos móviles.

\cite{nielsen2000} recomienda una muestra de 5 personas para pruebas de
usabilidad.~\cite{ritch2009} menciona que con 5 individuos, se encuentran 85\%
de los errores en promedio, y que un grupo de 5 a 10 personas es adecuado para
pruebas de usabilidad sencillas.

Esta prueba no es de gran complejidad, el procedimiento es sencillo y esta
bien definido, se busca determinar que problemas presenta la interfaz, que
impedimentos encuentran usuarios acostumbrados a la tecnología pero no al
procedimiento, por ello se elige una muestra de 8 alumnos.

\subsection{Muestra}

La primera fue realizada con alumnos de la carrera de Ingeniería en Informática
de la Facultad Politécnica que pertenece a la Universidad Nacional de Asunción,
sin experiencia previa tanto con la solución como con los procedimientos
simulados, pero si familizarizados con la utilización de dispositivos móviles.

\cite{nielsen2000} recomienda una muestra de 5 personas para pruebas de
usabilidad.~\cite{ritch2009} menciona que con 5 individuos, se encuentran 85\%
de los errores en promedio, y que un grupo de 5 a 10 personas es adecuado para
pruebas de usabilidad sencillas.

Esta prueba no es de gran complejidad, el procedimiento es sencillo y esta
bien definido, se busca determinar que problemas presenta la interfaz, que
impedimentos encuentran usuarios acostumbrados a la tecnología pero no al
procedimiento, por ello se elige una muestra de 8 alumnos.


\chapter{Análisis de Resultados}
\label{chap:analisis}

\observacion{Poner descripción al nombre de las pruebas, por ejemplo no objetiva
    o subjetiva}
\observacion{Juntar cap análisis y resultados}
\observacion{Borrar los nos}
\observacion{Ver lo de aleatoriedad (variable) en subjetiva}


\observacion{En general creo que falta más comentarios sobre los valores que se
exponen como para dar pie a conclusiones más adelante}

En este capítulo se muestran los resultados obtenidos en cada una de las pruebas
y encuestas descriptas en el capítulo anterior. Los \fixme{valores}{Resultados?}
obtenidos son expuestos en \fixme{distintos formatos}{Cuales?} para mejorar su
interpretación.

\fixme{En la primera}{Cual?} parte se describen los resultados obtenidos de la
\emph{Prueba de interfaz de usuario} los cuales revelan los aspectos que
debieron ser mejorados durante el desarrollo de la solución. 

\fixme{En la segunda parte}{Sección?}, se muestran los resultados obtenidos de
la \emph{Encuesta de ubicación} los cuales muestran el porcentaje de alumnos que
cumplen con los requisitos para formar parte del grupo de evaluación de la
solución.

En la tercera parte, se muestran los resultados obtenidos de la \emph{Encuesta
subjetiva} en donde se muestran las fortalezas y debilidades de la solución así
como datos que miden la aceptación de la solución por parte del usuario. 

En la cuarta parte, se muestran los resultados obtenidos de la \emph{Encuesta
objetiva} en donde se muestra el rendimiento de los alumnos en cuanto a
preguntas relacionadas a los procedimientos simulados, tanto de los que forman
parte de la muestra como de los que forman parte del grupo de control. Sin
embargo, la diferencia de rendimientos entre estos dos grupos no pueden ser
considerada como absoluta ya que no se considera que la cantidad de partidas que
fueron jugadas por los usuarios sean suficientes para influenciar notablemente
en su rendimiento.

En la quinta parte, se muestran los resultados obtenidos del \emph{Registro de
actividades} en donde se muestra el desempeño de los usuarios en cada una de las
partidas jugadas, así como el grado de uso de la solución, entre otros aspectos
importantes.

Finalmente, en la sexta parte, se muestran las relaciones entre los resultados
obtenidos en la \emph{Encuesta objetiva} y el \emph{Registro de actividades}.


%! TEX root = ../main.tex
\section{Interfaz de Usuario}
\label{sec:res_INTERFAZ}
\observacion{Pruebas preliminares de usabilidad}

En la sección~\ref{sec:interfaz} se describe la prueba de interfaz realizada
durante el desarrollo de \fixme{la solución}{de la UI}, a fin de validar las
hipótesis asumidas y evaluar la usabilidad de la solución.

Esta prueba se divide en dos partes, en la primera, denominada
\emph{Simulación}, los sujetos de prueba utilizan la aplicación, y en la
segunda parte, denominada \emph{Encuesta}, los mismos completan una encuesta
sobre su apreciación de la solución.

\subsection{Simulación}

Las grabaciones realizadas a las sesiones de los usuarios se utilizan para medir
el grado de facilidad de aprendizaje de la interfaz de usuario.

Dados los tres grupo descriptos en~\ref{sec:evaluacion_interfaz_variables}, la
tabla~\ref{tab:interfaz_tiempo_acciones} \fixme{muestra}{tiempo} el tiempo, en segundos,
que le tomo a cada usuario realizar cada una de las acciones la primera vez y
el tiempo que les tomo en promedio las demás veces, para cada una de los grupos
de acciones.

\observacion{Hacer énfasis en la comparación entre el primer y los siguientes}

\begin{table}[!hbt]
\centering
\begin{tabular}{|c|c|c|c|c|c|c|}
\hline
\rowcolor{gris} \textbf{} & \multicolumn{2}{|c|}{\textbf{Menú Contextual}} &
\multicolumn{2}{|c|}{\textbf{Menú de la Interfaz}} &
\multicolumn{2}{|c|}{\textbf{Herramienta}}\\
\hline
\rowcolor{gris} Usuario & Primera & Siguientes & Primera & Siguientes & Primera & Siguientes \\
\hline 1 &  8 &  2.25 &  3 & 9.14 & 11 & 3.0 \\
\hline 2 & 30 &  7.00 &  4 & 3.57 &  7 & 4.5 \\
\hline 3 &  5 &  2.25 &  5 & 1.86 &  1 & 1.0 \\
\hline 4 &  2 & 13.00 &  4 & 2.00 &  1 & 0.5 \\
\hline 5 & 18 &  2.75 &  6 & 4.43 &  6 & 3.0 \\
\hline 6 &  4 & 14.25 & 11 & 7.86 & 13 & 4.0 \\
\hline 7 &  5 &  8.00 &  4 & 4.71 & 20 & 2.5 \\
\hline 8 &  3 &  2.33 & 10 & 3.57 &  3 & 6.5 \\
\hline
\textbf{Promedio} & 9.38 & 6.37 & 5.88 & 4.64 & 7.75 & 3.125 \\
\end{tabular}
\caption{Tiempo por acciones la primera vez y las siguientes veces que se realizo}
\label{tab:interfaz_tiempo_acciones}
\end{table}

En la tabla~\ref{tab:interfaz_tiempo_acciones} se observa consistentemente una
\fixme{mejora}{En relación a qué?} en el tiempo de realización de una acción con
respecto a la primera vez que es realizada. 

\begin{figure}[hbt!]
\centering
\includegraphics[width=14cm]{resultados/imagenes/interfaz_tiempo_actividades.png}
\caption{Tiempo por tipo de actividad}
\label{fig:interfaz_tiempo_acciones}
\end{figure}

En la figura~\ref{fig:interfaz_tiempo_acciones} se observa como en promedio el
usuario aprende, y en las siguientes acciones similares demora menos tiempo,
este es un factor importante y es el objetivo de esta prueba pues muestra que la
interfaz es fácil de usar, y con tres movimientos básicos, el usuario puede
utilizarla sin mayores inconvenientes. Se observa una mejoría del $30\%$ en las
acciones \emph{Menú Contextual}, $21\%$ en las acciones de tipo \emph{Menú de la
    Interfaz} y finalmente, una mejoría del $60\%$ en las acciones de tipo
\emph{Herramienta}.


La tabla~\ref{tab:interfaz_cantidad_espaciales} nos muestra la cantidad de
movimientos espaciales realizados por los usuarios, se observa que en promedio
se desplazaron $10,88$ veces por el escenario, y $6,75$ veces acercaron o
alejaron la cámara del paciente.

\observacion{Hay que dejar bien en claro de donde sale esto, por que es
importante entender el significado de esta diferencia}

\begin{table}[H]
\centering
\begin{tabular}{lrrr}
\toprule
\textbf{Jugador}  & \textbf{Movimiento} & \textbf{Zoom} & \textbf{Total} \\
\midrule
1        & 18         & 2    & 20 \\
2        & 7          & 8    & 15 \\
3        & 14         & 12   & 26 \\
4        & 9          & 14   & 23 \\
5        & 5          & 8    & 13 \\
6        & 14         & 4    & 18 \\
7        & 16         & 3    & 19 \\
8        & 4          & 3    &  7 \\
\midrule
\textbf{Promedio} & \textbf{10,88}      & \textbf{6,75} & \textbf{17,63} \\
\bottomrule
\end{tabular}
\caption{Cantidad de movimientos espaciales}
\label{tab:interfaz_cantidad_espaciales}
\end{table}

No existe una cantidad mínima o máxima que el usuario debería acercar o mover la
cámara, en cambio, los datos mostrados en~\ref{tab:interfaz_cantidad_espaciales},
muestran que no son necesarias demasiadas acciones, juntando esta información,
con la información proveída en la tabla~\ref{tab:interfaz_tiempo_total}, se ve
que en promedio los usuarios realizaron $1,7$ movimientos por minuto.

\begin{table}[!hbt]
\centering
\begin{tabular}{lrrr}
\toprule
\textbf{Alumno} & \textbf{Tiempo} \\
\midrule
1        & 8:32 \\
2        & 6:03 \\
3        & 8:33 \\
4        & 5:17 \\
5        & 6:55 \\
6        & 8:40 \\
7        & 7:03 \\
8        & 10:27 \\
\midrule
\textbf{Promedio} & \textbf{7:41} \\
\bottomrule
\end{tabular}
\caption{Tiempo de prueba por usuario}
\label{tab:interfaz_tiempo_total}
\end{table}

El tiempo total que se observa en la tabla~\ref{tab:interfaz_tiempo_total},
muestra que en promedio a cada alumno le tomo $7:41$ minutos realizar todos los
pasos especificados, es importante notar que este tiempo incluye el tiempo de
adaptación. 



\begin{table}[!hbt]
\centering
\begin{tabular}{lrrr}
\toprule
\textbf{Alumno} & \textbf{Pasos requeridos} \\
\midrule
1 & 19 \\
2 & 15 \\
3 & 18 \\
4 & 15 \\
5 & 18 \\
6 & 16 \\
7 & 19 \\
8 & 14 \\
\midrule
\textbf{Promedio} & \textbf{16,75} \\
\bottomrule
\end{tabular}
\caption{Acciones realizadas por usuario}
\label{tab:interfaz_acciones}
\end{table}

La tabla~\ref{tab:interfaz_acciones} nos muestra la cantidad de acciones que
realizaron los alumnos, junto con las acciones correctas para llevar a cabo el
procedimiento. Se observa que en promedio realizaron $16.75$ acciones correctas,
esto permite identificar en que parte de procedimiento los usuarios tienen
inconvenientes en cuanto al uso de la interfaz.

\subsection{Encuesta}

\fixme{Como fue descripta en la sección~\ref{sec:interfaz} la encuesta es
realizada a cada usuario}{Mejorar}, y es utilizada para obtener el grado de
disconformidad de los mismos. Se utiliza la disconformidad para resaltar los
puntos débiles, y así, aquellas variables que tengan el mayor porcentaje serán
las que deben ser arregladas.

Las preguntas que forman parte de la  encuesta son agrupadas en cuanto a
aspectos de calidad gráfica, interacción con el entorno, interacción con los
objetos, características del entorno, usabilidad de la interfaz e integración
con el hardware.

Luego de estas agrupaciones obtenemos el resultado que se muestra en la
tabla~\ref{tab:interfaz_disconformidad_metrica}. En esta tabla se puede observar
que las mayores disconformidades son con respecto a la usabilidad de la interfaz
que llega al $51\%$, la interacción de los usuarios con el entorno que llega al
$50\%$, la interacción con los objetos que llega al $49\%$. Luego, se pueden
observar también otras disconformidades con menor porcentaje, las
características del entorno con un  $33\%$, la integración con el hardware con
un  $27\%$ y por ultimo, la calidad gráfica con un  $17\%$.

\observacion{Hay que explicar que estas pruebas se hicieron de forma previa a
las demás y que se arreglan algunos casos}

\begin{table}[H]
\centering
\begin{tabular}{lr}
\toprule
Métrica & Disconformidad \\
\midrule
Calidad Gráfica         & 0.17 \\
Interacción Entorno     & 0.50\\
Interacción Objetos     & 0.49\\
Características Entorno & 0.33\\
Usabililidad Interfaz   & 0.51\\
Integración Hardware    & 0.27\\
\bottomrule
\end{tabular}
\caption{Disconformidad por métrica}
\label{tab:interfaz_disconformidad_metrica}
\end{table}

La conclusión de esta prueba de interfaz, es que si bien, pudo ser utilizada sin
mayores inconvenientes, existe un alto grado de disconformidad con la interfaz,
además cabe resaltar, los sujetos de prueba son personas acostumbradas al uso de
tecnologías similares. Otros puntos débiles encontrados en esta prueba son la
interacción con el entorno y  con los objetos.

Como resultado de esta prueba, la interfaz y la interacción con objetos y
elementos sufren modificaciones a fin de su utilización con usuarios no
técnicos.

Las demás pruebas mencionadas en este capítulo son realizadas con la versión
final de la solución, la cual es obtenida luego de las mejoras realizadas a los
puntos débiles detectados por esta prueba.

%! TEX root = ../main.tex


\section{Encuesta de Ubicación}
\label{sec:res_UBICACION}

Como se indicó en la sección~\ref{sec:ubicacion}, se agrupa a los alumnos
encuestados de acuerdo a las características de sus dispositivos móviles y del
acceso a internet.

En la figura~\ref{fig:ubicacion_acceso_internet} se puede observar que de 93
encuestados, el $94,6\%$ tiene acceso a internet al menos en algún momento y que
solo el $5.4\%$ no tiene acceso a internet en sus dispositivos móviles.

\begin{figure}[ht!]
\centering
\includegraphics[scale=0.8]{resultados/imagenes/ubicacion_acceso_internet.png}
\caption{Acceso a internet desde dispositivos móviles}
\label{fig:ubicacion_acceso_internet}
\observacion{no habia otras caracteristicas}
\end{figure}

Por otro lado, en la figura\ref{fig:ubicacion_sistemas_operativos} se muestra
los sistemas operativos móviles utilizados por los usuarios encuestados. Se
puede observar que Android lidera con un $61.3\%$, le sigue Windows Phone con un
$12.9\%$.

\begin{figure}[ht!]
\centering
\includegraphics[scale=0.8]{resultados/imagenes/ubicacion_sistemas_operativos.png}
\caption{Sistemas operativos móviles utilizados}
\label{fig:ubicacion_sistemas_operativos}
\observacion{Meter a la par de otros}
\end{figure}

Por último, se discrimina a los encuestados para determinar cuantos de ellos
tiene dispositivos móviles que cumplen los requisitos mínimos para utilizar la
solución propuesta según lo descrito en la sección~\ref{sec:ubicacion}. En la
figura~\ref{fig:ubicacion_requisitos_minimos} se puede observar que el $18,3\%$
de los encuestados cumplen con los requisitos.

\begin{figure}[ht!]
\centering
\includegraphics[scale=0.8]{resultados/imagenes/ubicacion_requisitos_minimos.png}
\caption{Dispositivos que cumplen con los requisitos mínimos para la prueba}
\label{fig:ubicacion_requisitos_minimos}
\observacion{Que conclusiones podría quitar de esto?}
\end{figure}

\observacion{Acercar más los gráficos}

%! TEX root = ../main.tex
\section{Encuesta Subjetiva}
\label{sec:res_subjetiva}
\observacion{Describir mejor la \enquote{Prueba de opinion}}

La información recogida por la encuesta muestra que hay datos faltantes, como se
explico en~\ref{sec:informacion_faltante}, esta información faltante es
completamente aleatoria en relación a la variable medida y a las demás
variables, de hecho, una sola encuesta tiene información faltante, así, se
establece que el tipo de información faltante es \emph{Información faltante
    completamente aleatoria}.

En su resumen de las diferentes técnicas y cuando se deben utilizar,
\cite{tsikriktsis2005review}, recomienda la utilización de la sustitución basada
en promedio por caso. Así, se completan los valores faltantes con el promedio de
respuestas completadas por el usuario.

\subsection{Resultados}
\label{sec:res_subjetiva}

Se presentan a continuación los resultados de las encuestas, agrupados por los
factores definidos en~\ref{sec:variables}.

La tabla~\ref{tab:subjetiva_conformidad_exploracion} \fixme{nos muestra}{tiempo}
las respuestas de los alumnos a las preguntas relacionadas al factor
exploración, son cuatro preguntas, las cuales fueron descritas
en~\ref{sec:sub_exploracion}. 
\observacion{Revisar bien los tiempos}
\observacion{En este punto uno ya se olvida de la escala}
\observacion{Algo que resaltar de todas estas tablas?}

\begin{table}[H]
\centering
\begin{tabular}{@{} *{5}{r} @{}}
\toprule
& \multicolumn{4}{c}{Exploración} \\
\cmidrule(lr){2-5}
Alumno &
\parbox{2.5cm}{Facilidad de uso}  &
\parbox{3cm}{Funciones realizadas por los elementos del juego} &
\parbox{3cm}{Aleatoriedad para afianzar conocimientos} &
\parbox{2.5cm}{Aleatoriedad para representar realismo} \\
\midrule
1         & 2   & 6   & 5   & 6  \\
2         & 6   & 6   & 4   & 6  \\
3         & 3   & 3   & 5   & 5  \\
4         & 6   & 6   & 6   & 6  \\
5         & 6   & 6   & 2   & 5  \\
6         & 6   & 6   & 6   & 6  \\
7         & 7   & 7   & 7   & 7  \\
8         & 6   & 6   & 7   & 7  \\
9         & 5   & 7   & 7   & 7  \\
10        & 6   & 7   & 6   & 6  \\
11        & 7   & 6   & 7   & 6  \\
\midrule
\textbf{Promedio}  & \textbf{5}   & \textbf{6}   & \textbf{6}   & \textbf{6} \\
\bottomrule
\end{tabular}
\caption{Resultados de la encuesta subjetiva relacionados al factor Exploración}
\label{tab:subjetiva_conformidad_exploracion}
\end{table}

La tabla~\ref{tab:subjetiva_conformidad_representacion} agrupa las respuestas de
los alumnos según la calidad de presentación, son cinco preguntas, las cuales
fueron descritas en~\ref{sec:sub_representacion}. 

\begin{table}[H]
\centering
\begin{tabular}{@{} *{6}{r} @{}}
\toprule
& \multicolumn{5}{c}{Representación} \\
\cmidrule(lr){2-6}
& & \multicolumn{3}{c}{Respuestas del paciente} & \\
\cmidrule(lr){3-5}
Alumno &
\parbox{2.5cm}{Acciones con las herramientas} &
\parbox{2.5cm}{Movimientos oculares del paciente} &
\parbox{2.5cm}{Reacción verbal del paciente} &
\parbox{2.5cm}{Movimientos motrices del paciente} &
\parbox{2.5cm}{Distinción entre los estados del paciente} \\
\midrule
1  & 6 & 6 & 2 & 5 & 2  \\
2  & 4 & 5 & 5 & 6 & 4  \\
3  & 5 & 3 & 3 & 3 & 3  \\
4  & 6 & 5 & 2 & 4 & 2  \\
5  & 2 & 2 & 6 & 6 & 6  \\
6  & 6 & 4 & 6 & 6 & 6  \\
7  & 7 & 6 & 5 & 7 & 5  \\
8  & 6 & 7 & 7 & 7 & 5  \\
9  & 5 & 6 & 2 & 7 & 6  \\
10 & 6 & 4 & 4 & 4 & 5  \\
11 & 6 & 4 & 6 & 6 & 5  \\
\midrule
\textbf{Promedio}  & \textbf{5} & \textbf{5} & \textbf{4} & \textbf{6} & \textbf{4} \\
\bottomrule
\end{tabular}
\caption{Resultados de la encuesta subjetiva relacionados al factor
    Representación}
\label{tab:subjetiva_conformidad_representacion}
\end{table}

La tabla~\ref{tab:subjetiva_conformidad_motivacion} muestra las respuestas de
los alumnos a las preguntas relacionadas al factor \textit{Motivación}, son
cinco preguntas, las cuales fueron descritas en~\ref{sec:sub_motivacion}. 

\begin{table}[H]
\centering
\begin{tabular}{@{} *{5}{r} @{}}
\toprule
& \multicolumn{4}{c}{Motivación} \\
\cmidrule(lr){2-5}
Alumno &
\parbox{2.5cm}{Importancia del puntaje} &
\parbox{3cm}{Socialización de los puntajes} &
\parbox{3cm}{Medición del tiempo} &
\parbox{2.5cm}{Motivación del puntaje} \\
\midrule
1  & 6 & 4 & 4 & 7  \\
2  & 7 & 4 & 6 & 6  \\
3  & 6 & 6 & 5 & 6  \\
4  & 1 & 4 & 6 & 1  \\
5  & 2 & 2 & 7 & 7  \\
6  & 6 & 5 & 4 & 6  \\
7  & 7 & 7 & 6 & 7  \\
8  & 7 & 7 & 7 & 7  \\
9  & 7 & 7 & 7 & 7  \\
10 & 7 & 4 & 5 & 7  \\
11 & 5 & 4 & 5 & 6  \\
\midrule
\textbf{Promedio}  & \textbf{6}   & \textbf{5}   & \textbf{6}   & \textbf{6} \\
\bottomrule
\end{tabular}
\caption{Resultados de la encuesta subjetiva relacionados al factor Motivación}
\label{tab:subjetiva_conformidad_motivacion}
\end{table}

La tabla~\ref{tab:subjetiva_conformidad_inmersion} muestra las respuestas de
los alumnos a las preguntas relacionadas al factor \textit{Inmersión}, son
cinco preguntas, las cuales fueron descritas en~\ref{sec:sub_inmersion}. 

\begin{table}[H]
\centering
\begin{tabular}{@{} *{6}{r} @{}}
\toprule
& \multicolumn{5}{c}{Inmersión} \\
\cmidrule(lr){2-6}
Alumno &
\parbox{2.5cm}{Realismo a través de ordenes verbales} &
\parbox{2.5cm}{Escenografía para entrar en ambiente} &
\parbox{2.5cm}{Gráficos en tres dimensiones para entender el entorno} &
\parbox{2.5cm}{Simulación como herramienta} &
\parbox{2.5cm}{Juegos cortos como ayuda para la repetición} \\
\midrule
1  & 4 & 6 & 4 & 5 & 3  \\
2  & 6 & 6 & 6 & 6 & 6  \\
3  & 6 & 6 & 6 & 5 & 6  \\
4  & 4 & 6 & 7 & 5 & 6  \\
5  & 6 & 6 & 5 & 6 & 6  \\
6  & 6 & 6 & 6 & 4 & 4  \\
7  & 7 & 7 & 7 & 7 & 7  \\
8  & 6 & 7 & 7 & 7 & 7  \\
9  & 6 & 7 & 7 & 7 & 7  \\
10 & 6 & 3 & 4 & 6 & 6  \\
11 & 5 & 3 & 5 & 5 & 4  \\
\midrule
\textbf{Promedio}  & \textbf{6} & \textbf{6} & \textbf{6} & \textbf{6} & \textbf{6} \\
\bottomrule
\end{tabular}
\caption{Resultados de la encuesta subjetiva relacionados al factor Inmersión}
\label{tab:subjetiva_conformidad_inmersion}
\end{table}

La tabla~\ref{tab:subjetiva_conformidad_utilidad} agrupa las respuestas de los
alumnos según la utilidad de la solución, son tres preguntas, las cuales fueron
descritas en~\ref{sec:sub_utilidad}. 


\begin{table}[H]
\centering
\begin{tabular}{@{} *{6}{r} @{}}
\toprule
& \multicolumn{3}{c}{Utilidad} \\
\cmidrule(lr){2-4}
Alumno &
\parbox{4cm}{Interacción con el paciente} &
\parbox{4cm}{Complementar el estudio en clase y laboratorio} &
\parbox{4cm}{Proveedor de facilidades para el estudio} \\
\midrule
1  & 7 & 5 & 7  \\
2  & 6 & 6 & 6  \\
3  & 6 & 6 & 6  \\
4  & 2 & 6 & 6  \\
5  & 2 & 6 & 6  \\
6  & 6 & 6 & 6  \\
7  & 7 & 6 & 7  \\
8  & 5 & 6 & 7  \\
9  & 7 & 7 & 7  \\
10 & 1 & 7 & 7  \\
11 & 6 & 4 & 5  \\
\midrule
\textbf{Promedio}  & \textbf{5} & \textbf{6} & \textbf{6} \\
\bottomrule
\end{tabular}
\caption{Resultados de la encuesta subjetiva relacionados al factor Utilidad}
\label{tab:subjetiva_conformidad_utilidad}
\end{table}

La tabla~\ref{tab:subjetiva_conformidad_retroalimentacion} agrupa las respuestas
de los alumnos según la calidad de retroalimentación, son tres preguntas, las
cuales fueron descritas en~\ref{sec:sub_retroalimentacion}. 

\begin{table}[H]
\centering
\begin{tabular}{@{} *{4}{r} @{}}
\toprule
& \multicolumn{3}{c}{Retroalimentación} \\
\cmidrule(lr){2-4}
Alumno &
\parbox{4cm}{Representación iconográfica de conceptos y acciones en la \Gls{gui}}  &
\parbox{4cm}{Retroalimentación suficiente respecto a los pasos realizados} &
\parbox{4cm}{Detalles de los pasos realizados incorrectamente} \\
\midrule
1  & 3 & 2 & 7  \\
2  & 5 & 4 & 6  \\
3  & 3 & 6 & 6  \\
4  & 6 & 6 & 6  \\
5  & 6 & 1 & 6  \\
6  & 2 & 6 & 6  \\
7  & 6 & 7 & 7  \\
8  & 6 & 6 & 7  \\
9  & 6 & 6 & 7  \\
10 & 5 & 4 & 6  \\
11 & 4 & 5 & 6  \\
\midrule
\textbf{Promedio}  & \textbf{5} & \textbf{5} & \textbf{6} \\
\bottomrule
\end{tabular}
\caption{Resultados de la encuesta subjetiva relacionados al factor
    Retroalimentación}
\label{tab:subjetiva_conformidad_retroalimentacion}
\end{table}

La tabla~\ref{tab:subjetiva_conformidad_pedagogia} agrupa las respuestas de los
alumnos según el factor pedagógico, son tres preguntas, las cuales fueron
descritas en~\ref{sec:sub_pedagogia}. 

\observacion{Habrá que replantear algunos nombres (falta de pistas como)}
\begin{table}[H]
\centering
\begin{tabular}{@{} *{4}{r} @{}}
\toprule
& \multicolumn{3}{c}{Pedagogía} \\
\cmidrule(lr){2-4}
Alumno &
\parbox{4cm}{Suficiencia de los botones que indican acciones} &
\parbox{4cm}{Falta de pistas como ayuda al aprendizaje} &
\parbox{4cm}{Potencial para comprender el procedimiento} \\
\midrule
1  & 6 & 6 & 6  \\
2  & 6 & 6 & 7  \\
3  & 4 & 6 & 6  \\
4  & 6 & 7 & 6  \\
5  & 7 & 5 & 6  \\
6  & 4 & 4 & 6  \\
7  & 7 & 6 & 7  \\
8  & 6 & 7 & 7  \\
9  & 7 & 7 & 7  \\
10 & 6 & 7 & 7  \\
11 & 5 & 6 & 5  \\
\midrule
\textbf{Promedio}  & \textbf{6} & \textbf{6} & \textbf{6} \\
\bottomrule
\end{tabular}
\caption{Resultados de la encuesta subjetiva relacionados al factor Pedagogía}
\label{tab:subjetiva_conformidad_pedagogia}
\end{table}

\subsection{Agrupamiento de datos}

Los resultados se resumen en la tabla~\ref{tab:subjetiva_conformidad_resumen},
se muestra el número de alumno para identificar a un alumno y el promedio de sus
respuestas en la encuesta, se muestra el promedio de las mismas.

\observacion{Retroalimentación esta marcado con un circulo}

\begin{table}[H]
\begin{tabular}{llllllllr}
\toprule
\textbf{\shortstack{Número de \\alumno}}         &
\begin{sideways}\textbf{Motivación}                    \end{sideways}        &
\begin{sideways}\textbf{Exploración}                     \end{sideways}        &
\begin{sideways}\textbf{Inmersión}                       \end{sideways}        &
\begin{sideways}\textbf{Pedagogía}                       \end{sideways}        &
\begin{sideways}\textbf{Representación}                  \end{sideways}        &
\begin{sideways}\textbf{Retroalimentación}               \end{sideways}        &
\begin{sideways}\textbf{Utilidad}                        \end{sideways}        &
\textbf{\shortstack{Promedio\\de respuestas}}\\
\midrule
1              & 5 & 5 & 4 & 6 & 4 & 4 & 6 & 5 \\
2              & 6 & 6 & 6 & 6 & 5 & 5 & 6 & 6 \\
3              & 4 & 6 & 6 & 5 & 3 & 5 & 6 & 5 \\
4              & 6 & 3 & 6 & 6 & 4 & 6 & 5 & 5 \\
5              & 5 & 5 & 6 & 6 & 4 & 4 & 5 & 5 \\
6              & 6 & 5 & 5 & 5 & 6 & 5 & 6 & 5 \\
7              & 7 & 7 & 7 & 7 & 6 & 7 & 7 & 7 \\
8              & 7 & 7 & 7 & 7 & 6 & 6 & 6 & 7 \\
9              & 7 & 7 & 7 & 7 & 5 & 6 & 7 & 6 \\
10             & 6 & 6 & 5 & 7 & 5 & 5 & 5 & 5 \\
11             & 7 & 5 & 4 & 5 & 5 & 5 & 5 & 5 \\
\midrule
Promedio Total & 6 & 6 & 6 & 6 & 5 & 5 & 6 & 6 \\
\bottomrule
\end{tabular}
\caption{Resultados de la encuesta subjetiva}
\label{tab:subjetiva_conformidad_resumen}
\end{table}

Se observa que el el puntaje más bajo en el promedio final es 5 que significa
\textit{Parcialmente de acuerdo}, y el más alto es 7, que significa
\textit{Totalmente de acuerdo}, se observa además el puntaje 6, que significa
\textit{De acuerdo}. 


Como se explica en la sección~\ref{sec:likert}, estos resultados están sujetos a
tendencias, para ello se aplica el método de doble
estandarización\cite{Pagolu2011}.

Con el resultado final de la estandarización diferenciamos cuales son los puntos
fuertes y cuales los puntos débiles de la solución propuesta con respecto a las
respuestas dadas por los usuarios. Estos valores son relativos a las respuestas
originales dadas en la encuesta, los resultados se muestran en la
tabla~\ref{tab:subjetiva_conformidad_corregida}.

\begin{table}[H]
\centering
\begin{tabular}{lrrrrrrrr}
\toprule
\textbf{\shortstack{Número de \\alumno}}                                &
\begin{sideways}\textbf{Motivación}                    \end{sideways} &
\begin{sideways}\textbf{Exploración}                     \end{sideways} &
\begin{sideways}\textbf{Inmersión}                       \end{sideways} &
\begin{sideways}\textbf{Pedagogía}                       \end{sideways} &
\begin{sideways}\textbf{Representación}                  \end{sideways} &
\begin{sideways}\textbf{Retroalimentación}               \end{sideways} &
\begin{sideways}\textbf{Utilidad}                        \end{sideways} &
\textbf{\shortstack{Promedio\\de respuestas}}\\
\midrule
1              & 0.45 & 0.55 & 0.20 & 0.63 & 0.44 & 0.41 & 0.82 & 0.47 \\
2              & 0.33 & 0.53 & 0.49 & 0.61 & 0.27 & 0.13 & 0.52 & 0.41 \\
3              & 0.17 & 0.86 & 0.87 & 0.67 & 0.13 & 0.67 & 1.00 & 0.60 \\
4              & 0.75 & 0.31 & 0.63 & 0.81 & 0.47 & 0.78 & 0.54 & 0.59 \\
5              & 0.46 & 0.58 & 0.69 & 0.67 & 0.57 & 0.50 & 0.54 & 0.58 \\
6              & 1.00 & 0.73 & 0.68 & 0.42 & 0.90 & 0.67 & 1.00 & 0.78 \\
7              & 1.00 & 0.79 & 1.00 & 0.67 & 0.50 & 0.87 & 0.78 & 0.80 \\
8              & 0.75 & 1.00 & 0.83 & 0.75 & 0.70 & 0.70 & 0.44 & 0.75 \\
9              & 0.90 & 1.00 & 0.93 & 1.00 & 0.64 & 0.92 & 1.00 & 0.90 \\
10             & 0.79 & 0.74 & 0.54 & 0.92 & 0.60 & 0.60 & 0.67 & 0.68 \\
11             & 0.75 & 0.42 & 0.08 & 0.25 & 0.60 & 0.35 & 0.25 & 0.39 \\
\midrule
\textbf{Promedio Total} & 0.67 & 0.68 & 0.63 & 0.67 & 0.53 & 0.60 & 0.69 & 0.63 \\
\bottomrule
\end{tabular}
\caption{Resultados de la encuesta subjetiva con doble estandarización}
\label{tab:subjetiva_conformidad_corregida}
\end{table}

Es importante notar que los datos la
tabla~\ref{tab:subjetiva_conformidad_corregida} son relativas a los datos de la
tabla~\ref{tab:subjetiva_conformidad_resumen}, es decir, que la representación
es el punto más débil, aún así, se ve que
en~\ref{tab:subjetiva_conformidad_resumen} que el valor es $5$ de $7$, lo que
indica que es un punto aceptable, y entre los factores analizados es el que
menos aprobación obtuvo.


Con la información obtenida, es posible \emph{Validar las hipótesis asumidas
    durante el desarrollo de la solución}, el cual es uno de los objetivos de
este capítulo. En la tabla~\ref{tab:resultado_resumen_hipotesis} se observa la
opinion de los alumnos con respecto a las hipótesis asumidas
en~\ref{sec:hipotesis}. Se observa una aceptación a las hipótesis asumidas.

\begin{table}[!hbt]
\centering
\begin{tabular}{lcr}
\toprule
Hipótesis                        & Promedio Subjetiva      & Promedio estandarizado \\
\midrule
Comandos de voz con interfaz     & De acuerdo              & $0,55$ \\
Extracción uniforme de elementos & Parcialmente de acuerdo & $0,65$ \\
Acciones de bioseguridad         & De acuerdo              & $0,58$ \\
Representación iconográfica      & Parcialmente de acuerdo & $0,53$ \\
Factores motivadores             & De acuerdo              & $0,65$ \\
Falta de pistas                  & De acuerdo              & $0,61$ \\
\bottomrule
\end{tabular}
\caption{Hipótesis con su aceptación}\label{tab:resultado_resumen_hipotesis}
\end{table}

Adicionalmente, se puede \emph{Evaluar los puntos fuertes y débiles de la
    solución}, utilizando los datos con doble estandarización de la
tabla~\ref{tab:subjetiva_conformidad_corregida}, se crea la
tabla~\ref{tab:resultado_resumen_aspectos_aceptacion}, donde se observa la
apreciación de los usuarios por cada aspecto estudiado.

\begin{table}[!hbt]
\centering
\begin{tabular}{lcr}
\toprule
Factores        & Promedio Subjetiva      & Promedio estandarizado \\
\midrule
Motivación        & De acuerdo              & $0.67$  \\
Exploración       & De acuerdo              & $0.68$  \\
Inmersión         & De acuerdo              & $0.63$  \\
Pedagogía         & De acuerdo              & $0.67$  \\
Representación    & Parcialmente de acuerdo & $0.53$  \\
Retroalimentación & Parcialmente de acuerdo & $0.60$  \\
Utilidad          & De acuerdo              & $0.69$  \\
\bottomrule
\end{tabular}
\caption{Aceptación por aspecto de la solución}
\label{tab:resultado_resumen_aspectos_aceptacion}
\end{table}

Para  obtener una mejor visión de las fortalezas y debilidades de la solución
propuesta, se presenta el gráfico de \emph{kiviat}~\ref{fig:subjetiva_kiviat},
en la misma se puede observar cuales son los puntos débiles de la solución.

\observacion{Pulir la manera en la que hacen referencia a ete tópico, aclarando
que son percepciones desde el punto de vista del usuario}
\begin{figure}[!ht]
\begin{tikzpicture}[label distance=.15cm]
\tkzKiviatDiagram[radial=2,
                    lattice=2, step=2,
                    scale=2.3]%
                {Motivación,
                 Exploración,
                 Inmersión,
                 Pedagogía,
                 Representación,
                 Retroalimentación,
                 Utilidad}
\tkzKiviatLine[thick,
                color=blue,
                mark=ball,
                ball color=red,
                mark size=1pt,opacity=.2, 
                fill=red!20](0.67,0.68,0.63,0.67,0.53,0.60,0.69)
\end{tikzpicture}
\label{fig:subjetiva_kiviat}
\caption{Gráfico de Kiviat de los factores evaluados}
\end{figure}

Se observa que las principales debilidades de la solución son la representación
y la retroalimentación, y las fortalezas la utilidad, pedagogía, exploración, y
la motivación.

\subsection{Preguntas abiertas}
\label{sec:res_subjetiva_abiertas}

En la parte final de la encuesta que completaron los alumnos, que formaron parte
de la prueba, cuenta con preguntas abiertas, donde los alumnos expresaron sus
opiniones sobre los aspectos que rodean al uso de este tipo de soluciones al
aprendizaje de enfermería.


\begin{itemize}
    \item El $100\%$ de los alumnos menciono que este tipo de soluciones son
        beneficiosas para el aprendizaje de procedimientos de enfermería.
    \item El $64\%$ de los alumnos menciono que la principal dificultad para
        utilizar la solución es el factor tiempo.
    \item El $45\%$ de los alumnos menciono que la solución esta completa,
        mientras que el $18\%$ sugirió más elementos e interacción con el
        paciente.
\end{itemize}


Con esta información se puede \emph{determinar el nivel de aceptación de la
    solución}, se observa que el $100\%$ de los alumnos cree que es beneficioso
contar con este tipo de soluciones.

\section{Encuesta Objetiva}
\label{sec:res_OBJETIVA}

Como se detalló en la sección~\ref{sec:objetiva}, la encuesta realizada a cada
usuario, parte del experimento, es utilizada para obtener una comparación en
cuanto al rendimiento de los usuarios que forman parte de la muestra y los que
forman parte del grupo de control.

La tabla \ref{tab:objetiva_rendimiento_por_pregunta} muestra el nivel de acierto
en promedio por pregunta de los usuarios que forman parte de la muestra y de los que
forman parte del grupo de control, con sus respectivas desviaciones estándar. Según
estos datos, en el $60\%$ del examen hay una leve mejoría en cuanto al nivel de acierto
para los usuarios que forman parte de la muestra.

\begin{table}[!hbt]
\centering
\begin{tabular}{|l|r|r|r|r|}
\hline
\rowcolor{gris} 
\textbf{Pregunta} & 
\textbf{Prom. Jugó} & 
\textbf{Des. Jugó} & 
\textbf{Prom. No Jugó} & 
\textbf{Des. No Jugó} \\
\hline
1 & 0.36 & 0.50 & 0.18 & 0.39 \\
\hline
2 & 0.64 & 0.50 & 0.60 & 0.49 \\
\hline
3 & 0.09 & 0.30 & 0.14 & 0.34 \\
\hline
4 & 0.27 & 0.47 & 0.25 & 0.44 \\
\hline
5 & 0.82 & 0.40 & 0.56 & 0.50 \\
\hline
6 & 0 & 0 & 0.18 & 0.39 \\
\hline
7 & 0.64 & 0.50 & 0.51 & 0.50 \\
\hline
8 & 0.45 & 0.52 & 0.27 & 0.45 \\
\hline
9 & 0.18 & 0.40 & 0.32 & 0.47 \\
\hline
10 & 0.36 & 0.50 & 0.45 & 0.50 \\
\hline
\end{tabular}
\caption{Rendimiento promedio de usuarios por pregunta}
\label{tab:objetiva_rendimiento_por_pregunta}
\end{table}

Los datos mostrados sólo sugieren levemente una tendencia a la mejoría de los puntajes 
para los usuarios que forman parte de la muestra, sin embargo, estos datos no pueden ser 
tomados para realizar conclusiones ya que la cantidad de sesiones de juego por usuario no 
se consideran suficientes para que el uso de la solución propuesta afecte realmente en el 
aprendizaje del mismo.

\section{Registro de actividad}

Las actividad de los usuarios es registrada y almacenada para su análisis, a
continuación se presentan los resultados de ese análisis, el mismo fue descrito
en~\ref{sec:registro}.


\begin{table}[!hbt]
\centering
\begin{tabular}{lrrrrrrrr}
\toprule
& \multicolumn{2}{c}{Extracción de sangre} \\
\cmidrule(lr){2-3} 
Número de alumno\tabletodo{\#alumno o ver mejor forma de
        decir que es uno  por fila} & 
Sesiones jugadas & 
Tiempo jugado \tabletodo{Tiempo en segundos}\\
\midrule
 1       & 5  & 1202 \\
 2       & 19 & 2507 \\
 4       & 5  & 398  \\
 5       & 6  & 768  \\
 6       & 17 & 2371 \\
 7       & 7  & 707  \\
 9       & 1  & 126  \\
10       & 8  & 960  \\
\midrule
Total:   & 68 & 9039 \\
\bottomrule
\end{tabular}
\caption{Número de partidas y tiempo total por alumno, en la escena de
    extracción de sangre.}
\label{tab:log_hemocultivo_partida}
\end{table}

La cantidad de partidas jugadas por usuario, se ven en la
tabla~\ref{tab:log_hemocultivo_partida}, se observa que existen 3 alumnos que no
participaron del experimento o no se registro su actividad.

Los registros pueden no ser registrados sí
\begin{enumerate*}[label=\itshape\alph*\upshape)]
    \item El usuario utilizo la solución, pero no envió los datos y, luego
        desinstalo la solución o borro los datos de la misma, o,
    \item El usuario no utilizo la solución.
\end{enumerate*}

En la tabla~\ref{tab:log_glasgow_random_partida}, se observa la cantidad de
sesiones y tiempo total por alumno, en la escena de \textit{Glasgow}, en modo de
evaluación.

Se observa que $5$ alumnos participaron en $22$ sesiones, en total jugaron
$1768$ segundos.

\begin{table}[!hbt]
\centering
\begin{tabular}{lrrrrrrrr}
\toprule
& \multicolumn{2}{c}{Glasgow (Evaluación)} \\
                   \cmidrule(lr){2-3} 
Número de alumno   & Sesiones jugadas                            & Tiempo jugado \\
\midrule
1     & 4  & 211 \\
2     & 8  & 738 \\
4     & 3  & 132 \\
6     & 1  & 97  \\
7     & 6  & 590 \\
\midrule
Total & 22 & 1768 \\
\bottomrule
\end{tabular}
\caption{Número de partidas y tiempo total por alumno, en la escena
    \textit{Glasgow}, en modo evaluación}
\label{tab:log_glasgow_random_partida}
\end{table}


\begin{table}[!hbt]
\centering
\begin{tabular}{lrrrrrrrr}
\toprule
& \multicolumn{2}{c}{Glasgow (Exploración)} \\
                   \cmidrule(lr){2-3} 
Número de alumno   & Sesiones jugadas                            & Tiempo jugado \\
\midrule
1        & 2 & 79 \\
2        & 3 & 80 \\
4        & 3 & 89 \\
6        & 1 & 79 \\
\midrule
Total:   & 9 & 327 \\
\bottomrule
\end{tabular}
\caption{Número de partidas y tiempo total por alumno, en la escena
    \textit{Glasgow}, en modo exploración}
\label{tab:log_glasgow_custom_partida}
\end{table}


En las tablas~\ref{tab:log_hemocultivo_puntaje}
y~\ref{tab:log_glasgow_random_puntaje} se muestran los primeros puntajes y un
promedio de los puntajes siguientes obtenidos por cada usuario en los
procedimientos de extracción de sangre y de la evaluación de la escala de
Glasgow. Se debe tener en cuenta el tiempo y las cantidades de veces que cada
alumno jugó cada uno de los procedimientos para valorar los resultados
mostrados. En general, los puntajes no muestran una mejora importante en los
puntajes\revisar{Esto contradice lo anterior}\revisar{Donde esta la de
    extracción(??)}.

\begin{table}[!hbt]
\centering
\begin{tabular}{lrrrrrrrr}
\toprule
                   & \multicolumn{2}{c}{Extracción de sangre} \\
                   \cmidrule(lr){2-3} 
Número de alumno   & Primer Puntaje                            & Siguientes Puntajes \\
\midrule
 1       & 18  & 13.5 \\
 2       & 9 & 10.6 \\
 4       & 3 & 3.3 \\
 5       & 3 & 6.8  \\
 6       & 3 & 5.8 \\
 7       & 4 & 4 \\
 9       & 16  &  \\
10       & 3 & 7.2 \\
\bottomrule
\end{tabular}
\caption{Puntaje obtenido la primera vez y el promedio de las siguientes veces
    por alumno, en la escena de extracción de sangre.}
\label{tab:log_hemocultivo_puntaje}
\end{table}


\begin{table}[!hbt]
\centering
\begin{tabular}{lrrrrrrrr}
\toprule
& \multicolumn{2}{c}{Glasgow (Evaluación)} \\
                   \cmidrule(lr){2-3} 
Número de alumno   & Primer Puntaje                            & Siguientes Puntajes \\
\midrule
1     & 1 & 1.5 \\
2     & 2 & 2.3 \\
4     & 1 & 1.5 \\
6     & 2 & 2 \\
7     & 0 & 1 \\
\bottomrule
\end{tabular}
\caption{Puntaje obtenido la primera vez y el promedio de las siguientes veces
    por alumno, en la escena \textit{Glasgow}, en modo evaluación}
\label{tab:log_glasgow_random_puntaje}
\end{table}

\observacion{Promedios  puntaje final no hay una forma de traficar el progreso?,
    algo más generar, por que osino no dice nada}

\section{Correlación entre variables}

% Datos globales
En la tabla~\ref{tab:all_correlation} se observa la correlación entre cinco
variables estudiadas, a fin de observar si existe alguna correlación entre los
valores, se utiliza la correlación de \emph{Pearson}, descrita
en~\ref{sec:correlacion}.

\begin{table}[H]
\centering
\begin{tabular}{lrrrrrr}
\toprule
        &
\begin{sideways}\textbf{Tiempo de Uso}\end{sideways}             &
\begin{sideways}\textbf{Encuesta subjetiva}\end{sideways}        &
\begin{sideways}\textbf{Encuesta objetiva}\end{sideways}         &
\begin{sideways}\textbf{Puntaje Máximo Extracción}\end{sideways} &
\begin{sideways}\textbf{Puntaje Máximo Glasgow}\end{sideways}    \\
\midrule
Tiempo de Uso             & 1    & -0.2  & 0.15  & 0.62 & 0.41 & 0.78 \\
Encuesta subjetiva        & -0.2 & 1     & -0.07 & 0.04 & 0.11 & -0.28\\
Encuesta objetiva         & 0.15 & -0.07 & 1     & 0.44 & 0.44 & 0.02 \\
Puntaje máximo Extracción & 0.62 & 0.04  & 0.44  & 1    & 0.96 & 0.44 \\
Puntaje máximo Glasgow    & 0.41 & 0.11  & 0.44  & 0.96 & 1    & 0.27 \\
\bottomrule               & 0.78 & -0.28 & 0.02  & 0.44 & 0.27 & 1    \\
\end{tabular}
\caption{Correlación entre factores estudiados} 
\label{tab:all_correlation}
\end{table}

Las correlaciones fuertes, que se observan en la
tabla~\ref{tab:all_correlation}, son:

\begin{itemize}
    \item Tiempo de uso y puntaje máximo extracción, $0,62$, correlación
        positiva fuerte.
    \item Tiempo de uso y puntaje máximo Glasgow, $0,78$, correlación positiva
        muy fuerte.
    \item Puntaje máximo extracción y encuesta objetiva, $0,44$, correlación
        positiva fuerte.
\end{itemize}


La tabla~\ref{tab:all_correlation} indica que existe una correlación positiva
fuerte ($0,62$ y $0,78$) entre el tiempo de uso y el puntaje más alto obtenido,
lo que sugiere que mientras más se utiliza la solución, se obtienen mejores
resultados. 

Una correlación positiva fuerte entre el puntaje máximo obtenido en la
Extracción y la encuesta objetiva ($0,44$), sugiere que existe una relación entre
el nivel de conocimientos de los alumnos y su desempeño en la práctica.


%\section{Objetivos}
\label{sec:resultados_objetivos}

Esta sección presenta un resumen de los resultados de los objetivos propuestos
para la evaluación en la sección~\label{sec:evaluacion_objetivos}. 

\subsection{Validar las hipótesis asumidas durante el desarrollo de la solución}

En la tabla~\ref{tab:resultado_resumen_hipotesis} se observa la opinion de los
alumnos con respecto a las hipótesis asumidas en~\ref{sec:hipotesis}. Se observa
una aceptación a las hipótesis asumidas.

\begin{table}[!hbt]
\centering
\begin{tabular}{lcr}
\toprule
Hipótesis                        & Promedio Subjetiva      & Promedio estandarizado \\
\midrule
Comandos de voz con interfaz     & De acuerdo              & $0,55$ \\
Extracción uniforme de elementos & Parcialmente de acuerdo & $0,65$ \\
Acciones de bioseguridad         & De acuerdo              & $0,58$ \\
Representación iconográfica      & Parcialmente de acuerdo & $0,53$ \\
Factores motivadores             & De acuerdo              & $0,65$ \\
Falta de pistas                  & De acuerdo              & $0,61$ \\
\bottomrule
\end{tabular}
\caption{Hipótesis con su aceptación}\label{tab:resultado_resumen_hipotesis}
\end{table}


\subsection{Evaluar los puntos fuertes y débiles de la solución}

Utilizando los datos con doble estandarización de la
tabla~\ref{tab:subjetiva_conformidad_corregida}, se crea la
tabla~\ref{tab:resultado_resumen_aspectos_aceptacion}, donde se observa la
apreciación de los usuarios por cada aspecto estudiado.

\begin{table}[!hbt]
\centering
\begin{tabular}{lcr}
\toprule
Hipótesis         & Promedio Subjetiva      & Promedio estandarizado \\
\midrule
Motivación        & De acuerdo              & $0.67$  \\
Exploración       & De acuerdo              & $0.68$  \\
Inmersión         & De acuerdo              & $0.63$  \\
Pedagogía         & De acuerdo              & $0.67$  \\
Representación    & Parcialmente de acuerdo & $0.53$  \\
Retroalimentación & Parcialmente de acuerdo & $0.60$  \\
Utilidad          & De acuerdo              & $0.69$  \\
\bottomrule
\end{tabular}
\caption{Aceptación por aspecto de la solución}
\label{tab:resultado_resumen_aspectos_aceptacion}
\end{table}

Para  obtener una mejor visión de las fortalezas y debilidades de la solución
propuesta, se presenta el gráfico de \emph{kiviat}~\ref{fig:subjetiva_kiviat},
en la misma se puede observar cuales son los puntos débiles de la solución.

\begin{figure}[!ht]
\begin{tikzpicture}[label distance=.15cm]
\tkzKiviatDiagram[radial=2,
                    lattice=2, step=2,
                    scale=2.3]%
                {Motivación,
                 Exploración,
                 Inmersión,
                 Pedagogía,
                 Representación,
                 Retroalimentación,
                 Utilidad}
\tkzKiviatLine[thick,
                color=blue,
                mark=ball,
                ball color=red,
                mark size=1pt,opacity=.2, 
                fill=red!20](0.67,0.68,0.63,0.67,0.53,0.60,0.69)
\end{tikzpicture}
\label{fig:subjetiva_kiviat}
\caption{Gráfico de Kiviat de los factores evaluados}
\end{figure}

Se observa que las principales debilidades de la solución son la representación
y la retroalimentación, y las fortalezas la utilidad, pedagogía, exploración, y
la motivación.

\subsection{Determinar el nivel de aceptación de la solución}

En la sección~\ref{sec:res_subjetiva_abiertas} se muestra un resumen de la
apreciación de los alumnos hacia la solución, en este punto se observa que el
$100\%$ de los alumnos cree que es beneficioso contar con este tipo de
soluciones.


\subsection{Evaluar la utilización de la solución, y el progreso de los
    usuarios}

Se observa en las tablas~\ref{tab:log_hemocultivo_puntaje}
y~\ref{tab:log_glasgow_random_puntaje} los alumnos que participaron de la prueba
mejoran su desempeño a medida que aumenta el número de partidas. 

Es importante notar que la cantidad de partidas no es uniforme entre los
alumnos, es decir hay alumnos con más de $10$ partidas y usuarios con menos de
$5$, por ello, es difícil demostrar que existe un progreso a medida que aumenta
el número de partidas.

\subsection{Identificar la influencia de la utilización de la solución en el ámbito
    pedagógico}

Los datos mostrados en la sección~\ref{sec:res_objetiva} sólo sugieren levemente
una tendencia a la mejoría de los puntajes para los usuarios que forman parte de
la muestra, sin embargo, estos datos no pueden ser tomados para realizar
conclusiones ya que la cantidad de sesiones de juego por usuario no se considera
suficiente para que el uso de la solución propuesta afecte realmente en el
aprendizaje del mismo.

\subsection{Determinar correlaciones entre variables estudiadas}

% Datos globales
En la tabla~\ref{tab:all_correlation} se observa la correlación entre cinco
variables estudiadas, a fin de observar si existe alguna correlación entre los
valores, se utiliza la correlación de \emph{Pearson}, descrita
en~\ref{sec:correlacion}.

\begin{table}[H]
\centering
\begin{tabular}{lrrrrrr}
\toprule
        &
\begin{sideways}\textbf{Tiempo de Uso}\end{sideways}             &
\begin{sideways}\textbf{Encuesta subjetiva}\end{sideways}        &
\begin{sideways}\textbf{Encuesta objetiva}\end{sideways}         &
\begin{sideways}\textbf{Puntaje Máximo Extracción}\end{sideways} &
\begin{sideways}\textbf{Puntaje Máximo Glasgow}\end{sideways}    \\
\midrule
Tiempo de Uso             & 1    & -0.2  & 0.15  & 0.62 & 0.41 & 0.78 \\
Encuesta subjetiva        & -0.2 & 1     & -0.07 & 0.04 & 0.11 & -0.28\\
Encuesta objetiva         & 0.15 & -0.07 & 1     & 0.44 & 0.44 & 0.02 \\
Puntaje máximo Extracción & 0.62 & 0.04  & 0.44  & 1    & 0.96 & 0.44 \\
Puntaje máximo Glasgow    & 0.41 & 0.11  & 0.44  & 0.96 & 1    & 0.27 \\
\bottomrule               & 0.78 & -0.28 & 0.02  & 0.44 & 0.27 & 1    \\
\end{tabular}
\caption{Correlación entre factores estudiados} 
\label{tab:all_correlation}
\end{table}

Las correlaciones fuertes, que se observan en la
tabla~\ref{tab:all_correlation}, son:

\begin{itemize}
    \item Tiempo de uso y puntaje máximo extracción, $0,62$, correlación
        positiva fuerte.
    \item Tiempo de uso y puntaje máximo Glasgow, $0,78$, correlación positiva
        muy fuerte.
    \item Puntaje máximo extracción y encuesta objetiva, $0,44$, correlación
        positiva fuerte.
\end{itemize}


La tabla~\ref{tab:all_correlation} indica que existe una correlación positiva
fuerte ($0,62$ y $0,78$) entre el tiempo de uso y el puntaje más alto obtenido,
lo que sugiere que mientras más se utiliza la solución, se obtienen mejores
resultados. 

Una correlación positiva fuerte entre el puntaje máximo obtenido en la
Extracción y la encuesta objetiva ($0,44$), sugiere que existe una relación entre
el nivel de conocimientos de los alumnos y su desempeño en la práctica.



% TODO:

%   Ver lo de si se mantiene lo de relación objetiva-logs
%   Cambiar la sección de preguntas abiertas para poner una lista con
%   conclusiones por con porcentajes.


\printbibliography

\end{document}


